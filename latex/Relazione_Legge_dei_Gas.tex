\documentclass[10pt,a4paper]{article} % Prepara un documento con un font grande

\input{./preamboli_e_stili/pacchetti.tex}

\DeclareGraphicsExtensions{.pdf, .png, .jpg} % Se due immagini hanno lo stesso nome sceglile secondo l'ordine di filetype qui
\graphicspath{ {./img/} }					 % Path delle immagini 

\input{./preamboli_e_stili/titolo_Calorimetro.tex}
\input{./preamboli_e_stili/stili_float.tex}


%////////////////////////////////////////////////////////////////////////////////////////////////////////////////////////////
%////////////////////////////////////////////////////////////////////////////////////////////////////////////////////////////
% Fine dei dati iniziali per il latex: il documento finale inizierà da qui
\begin{document}

\maketitle % Produce il titolo a partire dai comandi \title, \author e \date

\vspace{15mm}
\begin{center}
\includegraphics[width=0.6\textwidth]{calorimeter}
\end{center}

\vfill
% Le varie sezioni
%\section{Obiettivi}
\begin{abstract}
	\noindent
	

\end{abstract}

\newpage

\tableofcontents % Prepara l'indice generale

%\begin{multicols}{2}

\section{Apparato strumentale}
	


\section{Metodologia di misura}
	


\newpage
\section{Presentazione dei dati}			
	\subsection{Tabelle}
	\begin{multicols}{2}
	

	\end{multicols}
	\clearpage
	\subsection{Grafici}
	\begin{grafico}
  \centering
\begin{tikzpicture}[gnuplot]
%% generated with GNUPLOT 4.6p3 (Lua 5.1; terminal rev. 99, script rev. 100)
%% mar 27 mag 2014 21:26:41 CEST
\path (0.000,0.000) rectangle (12.500,8.750);
\gpcolor{color=gp lt color border}
\gpsetlinetype{gp lt border}
\gpsetlinewidth{1.00}
\draw[gp path] (1.504,1.657)--(1.684,1.657);
\node[gp node right] at (1.320,1.657) { 0.7};
\draw[gp path] (1.504,2.330)--(1.684,2.330);
\node[gp node right] at (1.320,2.330) { 0.8};
\draw[gp path] (1.504,3.002)--(1.684,3.002);
\node[gp node right] at (1.320,3.002) { 0.9};
\draw[gp path] (1.504,3.674)--(1.684,3.674);
\node[gp node right] at (1.320,3.674) { 1};
\draw[gp path] (1.504,4.347)--(1.684,4.347);
\node[gp node right] at (1.320,4.347) { 1.1};
\draw[gp path] (1.504,5.019)--(1.684,5.019);
\node[gp node right] at (1.320,5.019) { 1.2};
\draw[gp path] (1.504,5.692)--(1.684,5.692);
\node[gp node right] at (1.320,5.692) { 1.3};
\draw[gp path] (1.504,6.364)--(1.684,6.364);
\node[gp node right] at (1.320,6.364) { 1.4};
\draw[gp path] (1.504,7.036)--(1.684,7.036);
\node[gp node right] at (1.320,7.036) { 1.5};
\draw[gp path] (1.504,7.709)--(1.684,7.709);
\node[gp node right] at (1.320,7.709) { 1.6};
\draw[gp path] (3.245,0.985)--(3.245,1.165);
\node[gp node center] at (3.245,0.677) { 0.06};
\draw[gp path] (4.985,0.985)--(4.985,1.165);
\node[gp node center] at (4.985,0.677) { 0.08};
\draw[gp path] (6.726,0.985)--(6.726,1.165);
\node[gp node center] at (6.726,0.677) { 0.1};
\draw[gp path] (8.466,0.985)--(8.466,1.165);
\node[gp node center] at (8.466,0.677) { 0.12};
\draw[gp path] (10.206,0.985)--(10.206,1.165);
\node[gp node center] at (10.206,0.677) { 0.14};
\draw[gp path] (1.504,8.267)--(1.504,1.657);
\draw[gp path] (2.212,0.985)--(11.583,0.985);
\node[gp node center,rotate=-270] at (0.246,4.683) {Pressione $[\frac{Kgf}{cm^2} C]$};
\node[gp node center] at (6.725,0.215) {Volume $[cm^3]$};
\node[gp node left] at (10.111,2.243) {$1^a$};
\gpcolor{color=gp lt color 0}
\gpsetpointsize{4.00}
\gppoint{gp mark 1}{(8.346,6.633)}
\gppoint{gp mark 1}{(4.507,6.633)}
\gppoint{gp mark 1}{(4.502,6.633)}
\gppoint{gp mark 1}{(7.010,6.633)}
\gppoint{gp mark 1}{(5.768,6.633)}
\gppoint{gp mark 1}{(6.130,6.633)}
\gppoint{gp mark 1}{(6.890,6.633)}
\gppoint{gp mark 1}{(6.899,6.633)}
\gppoint{gp mark 1}{(6.889,6.633)}
\gppoint{gp mark 1}{(5.533,6.633)}
\gppoint{gp mark 1}{(6.854,6.633)}
\gppoint{gp mark 1}{(7.190,6.633)}
\gppoint{gp mark 1}{(7.215,6.633)}
\gppoint{gp mark 1}{(7.186,6.633)}
\gppoint{gp mark 1}{(7.199,6.633)}
\gppoint{gp mark 1}{(7.265,6.633)}
\gppoint{gp mark 1}{(7.257,6.633)}
\gppoint{gp mark 1}{(6.060,6.633)}
\gppoint{gp mark 1}{(6.064,6.633)}
\gppoint{gp mark 1}{(8.793,6.633)}
\gppoint{gp mark 1}{(8.792,6.633)}
\gppoint{gp mark 1}{(8.791,6.633)}
\gppoint{gp mark 1}{(8.792,6.633)}
\gppoint{gp mark 1}{(8.792,6.633)}
\gppoint{gp mark 1}{(8.792,6.633)}
\gppoint{gp mark 1}{(6.603,6.633)}
\gppoint{gp mark 1}{(6.610,6.633)}
\gppoint{gp mark 1}{(7.911,6.633)}
\gppoint{gp mark 1}{(8.357,6.640)}
\gppoint{gp mark 1}{(4.514,6.640)}
\gppoint{gp mark 1}{(7.036,6.640)}
\gppoint{gp mark 1}{(7.030,6.640)}
\gppoint{gp mark 1}{(7.016,6.640)}
\gppoint{gp mark 1}{(7.046,6.640)}
\gppoint{gp mark 1}{(7.042,6.640)}
\gppoint{gp mark 1}{(5.772,6.640)}
\gppoint{gp mark 1}{(5.775,6.640)}
\gppoint{gp mark 1}{(6.137,6.640)}
\gppoint{gp mark 1}{(6.137,6.640)}
\gppoint{gp mark 1}{(5.534,6.640)}
\gppoint{gp mark 1}{(5.532,6.640)}
\gppoint{gp mark 1}{(6.864,6.640)}
\gppoint{gp mark 1}{(6.874,6.640)}
\gppoint{gp mark 1}{(7.236,6.640)}
\gppoint{gp mark 1}{(7.259,6.640)}
\gppoint{gp mark 1}{(7.274,6.640)}
\gppoint{gp mark 1}{(6.064,6.640)}
\gppoint{gp mark 1}{(8.800,6.640)}
\gppoint{gp mark 1}{(8.824,6.640)}
\gppoint{gp mark 1}{(8.808,6.640)}
\gppoint{gp mark 1}{(6.612,6.640)}
\gppoint{gp mark 1}{(7.942,6.640)}
\gppoint{gp mark 1}{(7.931,6.640)}
\gppoint{gp mark 1}{(7.973,6.640)}
\gppoint{gp mark 1}{(7.953,6.640)}
\gppoint{gp mark 1}{(4.521,6.646)}
\gppoint{gp mark 1}{(7.052,6.646)}
\gppoint{gp mark 1}{(5.780,6.646)}
\gppoint{gp mark 1}{(6.146,6.646)}
\gppoint{gp mark 1}{(6.906,6.646)}
\gppoint{gp mark 1}{(5.541,6.646)}
\gppoint{gp mark 1}{(7.269,6.646)}
\gppoint{gp mark 1}{(7.276,6.646)}
\gppoint{gp mark 1}{(7.292,6.646)}
\gppoint{gp mark 1}{(6.075,6.646)}
\gppoint{gp mark 1}{(8.824,6.646)}
\gppoint{gp mark 1}{(6.617,6.646)}
\gppoint{gp mark 1}{(7.972,6.646)}
\gppoint{gp mark 1}{(4.528,6.653)}
\gppoint{gp mark 1}{(5.790,6.653)}
\gppoint{gp mark 1}{(5.783,6.653)}
\gppoint{gp mark 1}{(6.152,6.653)}
\gppoint{gp mark 1}{(6.911,6.653)}
\gppoint{gp mark 1}{(5.550,6.653)}
\gppoint{gp mark 1}{(5.553,6.653)}
\gppoint{gp mark 1}{(7.291,6.653)}
\gppoint{gp mark 1}{(7.313,6.653)}
\gppoint{gp mark 1}{(6.103,6.653)}
\gppoint{gp mark 1}{(6.088,6.653)}
\gppoint{gp mark 1}{(8.842,6.653)}
\gppoint{gp mark 1}{(8.836,6.653)}
\gppoint{gp mark 1}{(7.972,6.653)}
\gppoint{gp mark 1}{(11.121,2.243)}
\gpcolor{color=gp lt color border}
\node[gp node left] at (10.111,1.935) {$2^a$};
\gpcolor{color=gp lt color 1}
\gppoint{gp mark 2}{(4.131,4.683)}
\gppoint{gp mark 2}{(4.072,4.683)}
\gppoint{gp mark 2}{(4.066,4.683)}
\gppoint{gp mark 2}{(3.406,4.683)}
\gppoint{gp mark 2}{(3.399,4.683)}
\gppoint{gp mark 2}{(3.405,4.683)}
\gppoint{gp mark 2}{(4.694,4.683)}
\gppoint{gp mark 2}{(4.693,4.683)}
\gppoint{gp mark 2}{(4.693,4.683)}
\gppoint{gp mark 2}{(3.556,4.683)}
\gppoint{gp mark 2}{(3.558,4.683)}
\gppoint{gp mark 2}{(4.910,4.683)}
\gppoint{gp mark 2}{(3.936,4.683)}
\gppoint{gp mark 2}{(2.959,4.683)}
\gppoint{gp mark 2}{(2.957,4.683)}
\gppoint{gp mark 2}{(2.954,4.683)}
\gppoint{gp mark 2}{(6.581,4.683)}
\gppoint{gp mark 2}{(6.575,4.683)}
\gppoint{gp mark 2}{(6.572,4.683)}
\gppoint{gp mark 2}{(4.418,4.683)}
\gppoint{gp mark 2}{(4.904,4.683)}
\gppoint{gp mark 2}{(4.917,4.683)}
\gppoint{gp mark 2}{(4.889,4.683)}
\gppoint{gp mark 2}{(4.895,4.683)}
\gppoint{gp mark 2}{(2.474,4.690)}
\gppoint{gp mark 2}{(4.136,4.690)}
\gppoint{gp mark 2}{(3.411,4.690)}
\gppoint{gp mark 2}{(3.412,4.690)}
\gppoint{gp mark 2}{(4.695,4.690)}
\gppoint{gp mark 2}{(3.563,4.690)}
\gppoint{gp mark 2}{(4.268,4.690)}
\gppoint{gp mark 2}{(3.947,4.690)}
\gppoint{gp mark 2}{(2.961,4.690)}
\gppoint{gp mark 2}{(2.959,4.690)}
\gppoint{gp mark 2}{(6.577,4.690)}
\gppoint{gp mark 2}{(6.577,4.690)}
\gppoint{gp mark 2}{(4.427,4.690)}
\gppoint{gp mark 2}{(6.011,4.696)}
\gppoint{gp mark 2}{(2.477,4.696)}
\gppoint{gp mark 2}{(4.142,4.696)}
\gppoint{gp mark 2}{(4.082,4.696)}
\gppoint{gp mark 2}{(4.081,4.696)}
\gppoint{gp mark 2}{(4.079,4.696)}
\gppoint{gp mark 2}{(3.419,4.696)}
\gppoint{gp mark 2}{(3.414,4.696)}
\gppoint{gp mark 2}{(4.700,4.696)}
\gppoint{gp mark 2}{(4.699,4.696)}
\gppoint{gp mark 2}{(4.698,4.696)}
\gppoint{gp mark 2}{(4.702,4.696)}
\gppoint{gp mark 2}{(4.699,4.696)}
\gppoint{gp mark 2}{(3.584,4.696)}
\gppoint{gp mark 2}{(3.575,4.696)}
\gppoint{gp mark 2}{(3.571,4.696)}
\gppoint{gp mark 2}{(3.580,4.696)}
\gppoint{gp mark 2}{(4.918,4.696)}
\gppoint{gp mark 2}{(4.916,4.696)}
\gppoint{gp mark 2}{(3.949,4.696)}
\gppoint{gp mark 2}{(2.963,4.696)}
\gppoint{gp mark 2}{(6.578,4.696)}
\gppoint{gp mark 2}{(6.578,4.696)}
\gppoint{gp mark 2}{(4.426,4.696)}
\gppoint{gp mark 2}{(4.929,4.696)}
\gppoint{gp mark 2}{(4.918,4.696)}
\gppoint{gp mark 2}{(6.013,4.703)}
\gppoint{gp mark 2}{(2.512,4.703)}
\gppoint{gp mark 2}{(2.502,4.703)}
\gppoint{gp mark 2}{(2.497,4.703)}
\gppoint{gp mark 2}{(2.484,4.703)}
\gppoint{gp mark 2}{(2.494,4.703)}
\gppoint{gp mark 2}{(2.488,4.703)}
\gppoint{gp mark 2}{(2.504,4.703)}
\gppoint{gp mark 2}{(2.499,4.703)}
\gppoint{gp mark 2}{(2.490,4.703)}
\gppoint{gp mark 2}{(2.481,4.703)}
\gppoint{gp mark 2}{(2.501,4.703)}
\gppoint{gp mark 2}{(4.148,4.703)}
\gppoint{gp mark 2}{(4.184,4.703)}
\gppoint{gp mark 2}{(4.171,4.703)}
\gppoint{gp mark 2}{(4.167,4.703)}
\gppoint{gp mark 2}{(4.188,4.703)}
\gppoint{gp mark 2}{(4.164,4.703)}
\gppoint{gp mark 2}{(4.159,4.703)}
\gppoint{gp mark 2}{(4.155,4.703)}
\gppoint{gp mark 2}{(4.152,4.703)}
\gppoint{gp mark 2}{(4.084,4.703)}
\gppoint{gp mark 2}{(4.083,4.703)}
\gppoint{gp mark 2}{(4.082,4.703)}
\gppoint{gp mark 2}{(3.427,4.703)}
\gppoint{gp mark 2}{(3.424,4.703)}
\gppoint{gp mark 2}{(4.704,4.703)}
\gppoint{gp mark 2}{(3.589,4.703)}
\gppoint{gp mark 2}{(3.594,4.703)}
\gppoint{gp mark 2}{(4.922,4.703)}
\gppoint{gp mark 2}{(4.271,4.703)}
\gppoint{gp mark 2}{(3.953,4.703)}
\gppoint{gp mark 2}{(2.966,4.703)}
\gppoint{gp mark 2}{(2.966,4.703)}
\gppoint{gp mark 2}{(2.967,4.703)}
\gppoint{gp mark 2}{(6.578,4.703)}
\gppoint{gp mark 2}{(6.579,4.703)}
\gppoint{gp mark 2}{(4.938,4.703)}
\gppoint{gp mark 2}{(11.121,1.935)}
\gpcolor{color=gp lt color border}
\node[gp node left] at (10.111,1.627) {$3^a$};
\gpcolor{color=gp lt color 2}
\gppoint{gp mark 3}{(3.063,1.657)}
\gppoint{gp mark 3}{(3.999,1.657)}
\gppoint{gp mark 3}{(4.004,1.657)}
\gppoint{gp mark 3}{(2.401,1.657)}
\gppoint{gp mark 3}{(2.402,1.657)}
\gppoint{gp mark 3}{(2.401,1.657)}
\gppoint{gp mark 3}{(2.438,1.657)}
\gppoint{gp mark 3}{(2.457,1.657)}
\gppoint{gp mark 3}{(3.146,1.657)}
\gppoint{gp mark 3}{(3.037,1.657)}
\gppoint{gp mark 3}{(2.217,1.657)}
\gppoint{gp mark 3}{(2.218,1.657)}
\gppoint{gp mark 3}{(2.791,1.657)}
\gppoint{gp mark 3}{(2.789,1.657)}
\gppoint{gp mark 3}{(2.787,1.657)}
\gppoint{gp mark 3}{(2.480,1.657)}
\gppoint{gp mark 3}{(4.469,1.657)}
\gppoint{gp mark 3}{(3.110,1.657)}
\gppoint{gp mark 3}{(3.111,1.657)}
\gppoint{gp mark 3}{(3.106,1.657)}
\gppoint{gp mark 3}{(3.112,1.657)}
\gppoint{gp mark 3}{(3.112,1.657)}
\gppoint{gp mark 3}{(2.435,1.657)}
\gppoint{gp mark 3}{(2.627,1.657)}
\gppoint{gp mark 3}{(3.062,1.664)}
\gppoint{gp mark 3}{(4.007,1.664)}
\gppoint{gp mark 3}{(2.405,1.664)}
\gppoint{gp mark 3}{(2.404,1.664)}
\gppoint{gp mark 3}{(2.439,1.664)}
\gppoint{gp mark 3}{(2.440,1.664)}
\gppoint{gp mark 3}{(2.455,1.664)}
\gppoint{gp mark 3}{(2.454,1.664)}
\gppoint{gp mark 3}{(2.453,1.664)}
\gppoint{gp mark 3}{(3.151,1.664)}
\gppoint{gp mark 3}{(3.042,1.664)}
\gppoint{gp mark 3}{(2.216,1.664)}
\gppoint{gp mark 3}{(2.797,1.664)}
\gppoint{gp mark 3}{(2.803,1.664)}
\gppoint{gp mark 3}{(2.801,1.664)}
\gppoint{gp mark 3}{(2.795,1.664)}
\gppoint{gp mark 3}{(2.798,1.664)}
\gppoint{gp mark 3}{(2.794,1.664)}
\gppoint{gp mark 3}{(2.482,1.664)}
\gppoint{gp mark 3}{(4.476,1.664)}
\gppoint{gp mark 3}{(4.486,1.664)}
\gppoint{gp mark 3}{(3.103,1.664)}
\gppoint{gp mark 3}{(3.100,1.664)}
\gppoint{gp mark 3}{(3.098,1.664)}
\gppoint{gp mark 3}{(3.101,1.664)}
\gppoint{gp mark 3}{(3.061,1.671)}
\gppoint{gp mark 3}{(4.008,1.671)}
\gppoint{gp mark 3}{(2.407,1.671)}
\gppoint{gp mark 3}{(2.448,1.671)}
\gppoint{gp mark 3}{(2.443,1.671)}
\gppoint{gp mark 3}{(2.441,1.671)}
\gppoint{gp mark 3}{(2.444,1.671)}
\gppoint{gp mark 3}{(2.451,1.671)}
\gppoint{gp mark 3}{(2.446,1.671)}
\gppoint{gp mark 3}{(2.449,1.671)}
\gppoint{gp mark 3}{(3.152,1.671)}
\gppoint{gp mark 3}{(2.448,1.671)}
\gppoint{gp mark 3}{(3.152,1.671)}
\gppoint{gp mark 3}{(2.214,1.671)}
\gppoint{gp mark 3}{(2.213,1.671)}
\gppoint{gp mark 3}{(2.805,1.671)}
\gppoint{gp mark 3}{(2.484,1.671)}
\gppoint{gp mark 3}{(2.497,1.671)}
\gppoint{gp mark 3}{(4.483,1.671)}
\gppoint{gp mark 3}{(3.095,1.671)}
\gppoint{gp mark 3}{(3.090,1.671)}
\gppoint{gp mark 3}{(3.094,1.671)}
\gppoint{gp mark 3}{(3.091,1.671)}
\gppoint{gp mark 3}{(3.097,1.671)}
\gppoint{gp mark 3}{(3.089,1.671)}
\gppoint{gp mark 3}{(2.438,1.671)}
\gppoint{gp mark 3}{(2.629,1.671)}
\gppoint{gp mark 3}{(2.626,1.671)}
\gppoint{gp mark 3}{(3.060,1.678)}
\gppoint{gp mark 3}{(3.057,1.678)}
\gppoint{gp mark 3}{(3.055,1.678)}
\gppoint{gp mark 3}{(4.009,1.678)}
\gppoint{gp mark 3}{(4.014,1.678)}
\gppoint{gp mark 3}{(2.408,1.678)}
\gppoint{gp mark 3}{(2.450,1.678)}
\gppoint{gp mark 3}{(2.450,1.678)}
\gppoint{gp mark 3}{(2.447,1.678)}
\gppoint{gp mark 3}{(3.153,1.678)}
\gppoint{gp mark 3}{(3.052,1.678)}
\gppoint{gp mark 3}{(2.212,1.678)}
\gppoint{gp mark 3}{(2.807,1.678)}
\gppoint{gp mark 3}{(2.505,1.678)}
\gppoint{gp mark 3}{(2.495,1.678)}
\gppoint{gp mark 3}{(2.490,1.678)}
\gppoint{gp mark 3}{(2.501,1.678)}
\gppoint{gp mark 3}{(2.488,1.678)}
\gppoint{gp mark 3}{(2.503,1.678)}
\gppoint{gp mark 3}{(2.492,1.678)}
\gppoint{gp mark 3}{(4.486,1.678)}
\gppoint{gp mark 3}{(3.088,1.678)}
\gppoint{gp mark 3}{(3.087,1.678)}
\gppoint{gp mark 3}{(2.442,1.678)}
\gppoint{gp mark 3}{(2.633,1.678)}
\gppoint{gp mark 3}{(11.121,1.627)}
\gpcolor{color=gp lt color border}
\node[gp node left] at (10.111,1.319) {$4^a$};
\gpcolor{color=gp lt color 3}
\gppoint{gp mark 4}{(11.129,8.247)}
\gppoint{gp mark 4}{(8.874,8.247)}
\gppoint{gp mark 4}{(8.881,8.247)}
\gppoint{gp mark 4}{(10.509,8.247)}
\gppoint{gp mark 4}{(10.530,8.247)}
\gppoint{gp mark 4}{(8.970,8.247)}
\gppoint{gp mark 4}{(8.990,8.247)}
\gppoint{gp mark 4}{(8.957,8.247)}
\gppoint{gp mark 4}{(11.233,8.247)}
\gppoint{gp mark 4}{(11.261,8.247)}
\gppoint{gp mark 4}{(11.405,8.247)}
\gppoint{gp mark 4}{(11.372,8.247)}
\gppoint{gp mark 4}{(8.537,8.247)}
\gppoint{gp mark 4}{(11.133,8.253)}
\gppoint{gp mark 4}{(8.890,8.253)}
\gppoint{gp mark 4}{(8.901,8.253)}
\gppoint{gp mark 4}{(10.554,8.253)}
\gppoint{gp mark 4}{(11.409,8.253)}
\gppoint{gp mark 4}{(11.370,8.253)}
\gppoint{gp mark 4}{(11.305,8.253)}
\gppoint{gp mark 4}{(11.442,8.253)}
\gppoint{gp mark 4}{(11.424,8.253)}
\gppoint{gp mark 4}{(11.346,8.253)}
\gppoint{gp mark 4}{(8.536,8.253)}
\gppoint{gp mark 4}{(9.732,8.253)}
\gppoint{gp mark 4}{(11.173,8.260)}
\gppoint{gp mark 4}{(11.177,8.260)}
\gppoint{gp mark 4}{(11.153,8.260)}
\gppoint{gp mark 4}{(11.145,8.260)}
\gppoint{gp mark 4}{(11.201,8.260)}
\gppoint{gp mark 4}{(8.924,8.260)}
\gppoint{gp mark 4}{(8.915,8.260)}
\gppoint{gp mark 4}{(8.934,8.260)}
\gppoint{gp mark 4}{(10.575,8.260)}
\gppoint{gp mark 4}{(10.592,8.260)}
\gppoint{gp mark 4}{(10.573,8.260)}
\gppoint{gp mark 4}{(9.005,8.260)}
\gppoint{gp mark 4}{(9.008,8.260)}
\gppoint{gp mark 4}{(9.000,8.260)}
\gppoint{gp mark 4}{(11.478,8.260)}
\gppoint{gp mark 4}{(11.436,8.260)}
\gppoint{gp mark 4}{(11.449,8.260)}
\gppoint{gp mark 4}{(11.465,8.260)}
\gppoint{gp mark 4}{(11.442,8.260)}
\gppoint{gp mark 4}{(11.486,8.260)}
\gppoint{gp mark 4}{(8.554,8.260)}
\gppoint{gp mark 4}{(8.546,8.260)}
\gppoint{gp mark 4}{(8.561,8.260)}
\gppoint{gp mark 4}{(9.748,8.260)}
\gppoint{gp mark 4}{(11.235,8.267)}
\gppoint{gp mark 4}{(11.225,8.267)}
\gppoint{gp mark 4}{(11.217,8.267)}
\gppoint{gp mark 4}{(8.956,8.267)}
\gppoint{gp mark 4}{(8.937,8.267)}
\gppoint{gp mark 4}{(8.941,8.267)}
\gppoint{gp mark 4}{(10.615,8.267)}
\gppoint{gp mark 4}{(10.601,8.267)}
\gppoint{gp mark 4}{(10.597,8.267)}
\gppoint{gp mark 4}{(11.538,8.267)}
\gppoint{gp mark 4}{(11.583,8.267)}
\gppoint{gp mark 4}{(11.551,8.267)}
\gppoint{gp mark 4}{(11.534,8.267)}
\gppoint{gp mark 4}{(8.604,8.267)}
\gppoint{gp mark 4}{(8.579,8.267)}
\gppoint{gp mark 4}{(9.761,8.267)}
\gppoint{gp mark 4}{(11.121,1.319)}
\gpcolor{color=gp lt color border}
\draw[gp path] (1.504,8.267)--(1.504,1.657);
\draw[gp path] (2.212,0.985)--(11.583,0.985);
%% coordinates of the plot area
\gpdefrectangularnode{gp plot 1}{\pgfpoint{1.504cm}{0.985cm}}{\pgfpoint{11.947cm}{8.381cm}}
\end{tikzpicture}
%% gnuplot variables

\caption{Alcune isobare estratte dai grafici}
\label{img:isob}
\end{grafico}

\begin{grafico}
  \centering
\begin{tikzpicture}[gnuplot]
%% generated with GNUPLOT 4.6p3 (Lua 5.1; terminal rev. 99, script rev. 100)
%% mar 27 mag 2014 22:34:39 CEST
\gpmonochromelines
\path (0.000,0.000) rectangle (12.500,8.750);
\gpcolor{color=gp lt color border}
\gpsetlinetype{gp lt border}
\gpsetlinewidth{1.00}
\draw[gp path] (2.240,1.962)--(2.420,1.962);
\node[gp node right] at (2.056,1.962) { 2e-05};
\draw[gp path] (2.240,2.939)--(2.420,2.939);
\node[gp node right] at (2.056,2.939) { 4e-05};
\draw[gp path] (2.240,3.916)--(2.420,3.916);
\node[gp node right] at (2.056,3.916) { 6e-05};
\draw[gp path] (2.240,4.894)--(2.420,4.894);
\node[gp node right] at (2.056,4.894) { 8e-05};
\draw[gp path] (2.240,5.871)--(2.420,5.871);
\node[gp node right] at (2.056,5.871) { 0.0001};
\draw[gp path] (2.240,6.848)--(2.420,6.848);
\node[gp node right] at (2.056,6.848) { 0.00012};
\draw[gp path] (2.240,0.985)--(2.240,1.165);
\node[gp node center] at (2.240,0.677) { 0};
\draw[gp path] (4.181,0.985)--(4.181,1.165);
\node[gp node center] at (4.181,0.677) { 5000};
\draw[gp path] (6.123,0.985)--(6.123,1.165);
\node[gp node center] at (6.123,0.677) { 10000};
\draw[gp path] (8.064,0.985)--(8.064,1.165);
\node[gp node center] at (8.064,0.677) { 15000};
\draw[gp path] (10.006,0.985)--(10.006,1.165);
\node[gp node center] at (10.006,0.677) { 20000};
\draw[gp path] (2.240,7.508)--(2.240,1.449);
\draw[gp path] (2.240,0.985)--(11.635,0.985);
\node[gp node center,rotate=-270] at (0.246,4.405) {Moli $[mol]$};
\node[gp node center] at (7.093,0.215) {Numero misura};
\node[gp node center] at (7.093,8.287) {Moli calcolate};
\node[gp node left] at (9.743,1.319) {Moli};
\gpcolor{color=gp lt color 0}
\gpsetlinetype{gp lt plot 0}
\draw[gp path] (10.663,1.319)--(11.579,1.319);
\draw[gp path] (2.240,7.508)--(2.240,7.503)--(2.241,7.497)--(2.241,7.494)--(2.242,7.490)%
  --(2.242,7.484)--(2.242,7.478)--(2.243,7.470)--(2.243,7.456)--(2.243,7.444)--(2.244,7.439)%
  --(2.244,7.432)--(2.245,7.432)--(2.245,7.428)--(2.246,7.415)--(2.246,7.406)--(2.247,7.393)%
  --(2.247,7.387)--(2.247,7.380)--(2.248,7.369)--(2.248,7.352)--(2.249,7.347)--(2.249,7.335)%
  --(2.249,7.323)--(2.250,7.307)--(2.250,7.294)--(2.250,7.292)--(2.251,7.286)--(2.251,7.275)%
  --(2.252,7.277)--(2.252,7.264)--(2.252,7.250)--(2.253,7.237)--(2.253,7.230)--(2.254,7.229)%
  --(2.254,7.224)--(2.254,7.222)--(2.255,7.218)--(2.255,7.213)--(2.256,7.208)--(2.256,7.210)%
  --(2.256,7.201)--(2.257,7.186)--(2.257,7.180)--(2.257,7.164)--(2.258,7.158)--(2.258,7.146)%
  --(2.259,7.138)--(2.259,7.136)--(2.259,7.129)--(2.260,7.126)--(2.260,7.116)--(2.261,7.111)%
  --(2.261,7.103)--(2.261,7.094)--(2.262,7.087)--(2.262,7.074)--(2.263,7.056)--(2.263,7.051)%
  --(2.263,7.045)--(2.264,7.040)--(2.264,7.027)--(2.264,7.013)--(2.265,7.000)--(2.265,6.989)%
  --(2.266,6.979)--(2.266,6.959)--(2.266,6.948)--(2.267,6.943)--(2.267,6.932)--(2.268,6.923)%
  --(2.268,6.909)--(2.268,6.898)--(2.269,6.888)--(2.269,6.875)--(2.270,6.868)--(2.270,6.856)%
  --(2.270,6.845)--(2.271,6.829)--(2.271,6.820)--(2.271,6.807)--(2.272,6.795)--(2.272,6.774)%
  --(2.273,6.757)--(2.273,6.750)--(2.273,6.745)--(2.274,6.740)--(2.274,6.730)--(2.275,6.728)%
  --(2.275,6.722)--(2.275,6.714)--(2.276,6.709)--(2.276,6.698)--(2.276,6.692)--(2.277,6.680)%
  --(2.277,6.668)--(2.278,6.657)--(2.278,6.648)--(2.278,6.644)--(2.279,6.638)--(2.279,6.630)%
  --(2.280,6.623)--(2.280,6.616)--(2.280,6.599)--(2.281,6.586)--(2.281,6.577)--(2.282,6.573)%
  --(2.282,6.568)--(2.282,6.563)--(2.283,6.556)--(2.283,6.543)--(2.283,6.523)--(2.284,6.506)%
  --(2.284,6.492)--(2.285,6.488)--(2.285,6.474)--(2.285,6.468)--(2.286,6.462)--(2.286,6.457)%
  --(2.287,6.441)--(2.287,6.433)--(2.287,6.421)--(2.288,6.412)--(2.288,6.403)--(2.289,6.397)%
  --(2.289,6.399)--(2.289,6.391)--(2.290,6.380)--(2.290,6.369)--(2.290,6.354)--(2.291,6.342)%
  --(2.291,6.326)--(2.292,6.319)--(2.292,6.305)--(2.292,6.288)--(2.293,6.278)--(2.293,6.268)%
  --(2.294,6.260)--(2.294,6.252)--(2.294,6.251)--(2.295,6.246)--(2.295,6.235)--(2.296,6.232)%
  --(2.296,6.224)--(2.296,6.220)--(2.297,6.211)--(2.297,6.209)--(2.297,6.202)--(2.298,6.195)%
  --(2.298,6.190)--(2.299,6.172)--(2.299,6.160)--(2.299,6.154)--(2.300,6.138)--(2.300,6.127)%
  --(2.301,6.120)--(2.301,6.107)--(2.301,6.099)--(2.302,6.091)--(2.302,6.089)--(2.303,6.081)%
  --(2.303,6.072)--(2.303,6.065)--(2.304,6.057)--(2.304,6.046)--(2.304,6.033)--(2.305,6.020)%
  --(2.305,6.012)--(2.306,5.999)--(2.306,5.986)--(2.306,5.978)--(2.307,5.977)--(2.307,5.969)%
  --(2.308,5.964)--(2.308,5.954)--(2.308,5.940)--(2.309,5.934)--(2.309,5.922)--(2.310,5.909)%
  --(2.310,5.896)--(2.310,5.886)--(2.311,5.877)--(2.311,5.866)--(2.311,5.854)--(2.312,5.847)%
  --(2.312,5.839)--(2.313,5.828)--(2.313,5.825)--(2.313,5.815)--(2.314,5.806)--(2.314,5.807)%
  --(2.315,5.798)--(2.315,5.793)--(2.315,5.785)--(2.316,5.779)--(2.316,5.773)--(2.316,5.768)%
  --(2.317,5.759)--(2.317,5.754)--(2.318,5.750)--(2.318,5.740)--(2.318,5.728)--(2.319,5.721)%
  --(2.319,5.712)--(2.320,5.703)--(2.320,5.697)--(2.320,5.690)--(2.321,5.681)--(2.321,5.671)%
  --(2.322,5.665)--(2.322,5.660)--(2.322,5.654)--(2.323,5.649)--(2.323,5.645)--(2.323,5.642)%
  --(2.324,5.638)--(2.324,5.628)--(2.325,5.620)--(2.325,5.609)--(2.325,5.605)--(2.326,5.593)%
  --(2.326,5.586)--(2.327,5.581)--(2.327,5.572)--(2.327,5.566)--(2.328,5.561)--(2.328,5.554)%
  --(2.329,5.540)--(2.329,5.535)--(2.329,5.529)--(2.330,5.519)--(2.330,5.511)--(2.330,5.503)%
  --(2.331,5.501)--(2.331,5.493)--(2.332,5.488)--(2.332,5.480)--(2.332,5.471)--(2.333,5.466)%
  --(2.333,5.457)--(2.334,5.452)--(2.334,5.449)--(2.334,5.443)--(2.335,5.435)--(2.335,5.430)%
  --(2.336,5.421)--(2.336,5.417)--(2.336,5.412)--(2.337,5.410)--(2.337,5.403)--(2.337,5.395)%
  --(2.338,5.381)--(2.338,5.371)--(2.339,5.363)--(2.339,5.353)--(2.339,5.347)--(2.340,5.339)%
  --(2.340,5.335)--(2.341,5.333)--(2.341,5.331)--(2.341,5.329)--(2.342,5.324)--(2.342,5.318)%
  --(2.343,5.313)--(2.343,5.310)--(2.343,5.302)--(2.344,5.288)--(2.344,5.283)--(2.344,5.281)%
  --(2.345,5.274)--(2.345,5.268)--(2.346,5.263)--(2.346,5.261)--(2.346,5.257)--(2.347,5.249)%
  --(2.347,5.241)--(2.348,5.238)--(2.348,5.233)--(2.348,5.226)--(2.349,5.221)--(2.349,5.216)%
  --(2.349,5.213)--(2.350,5.203)--(2.350,5.194)--(2.351,5.187)--(2.351,5.184)--(2.351,5.180)%
  --(2.352,5.174)--(2.352,5.171)--(2.353,5.160)--(2.353,5.157)--(2.353,5.143)--(2.354,5.132)%
  --(2.354,5.116)--(2.355,5.106)--(2.355,5.093)--(2.355,5.081)--(2.356,5.067)--(2.356,5.055)%
  --(2.356,5.041)--(2.357,5.030)--(2.357,5.021)--(2.358,5.015)--(2.358,5.004)--(2.358,4.995)%
  --(2.359,4.988)--(2.359,4.978)--(2.360,4.971)--(2.360,4.966)--(2.360,4.960)--(2.361,4.955)%
  --(2.361,4.950)--(2.362,4.944)--(2.362,4.938)--(2.362,4.930)--(2.363,4.921)--(2.363,4.914)%
  --(2.363,4.910)--(2.364,4.904)--(2.364,4.901)--(2.365,4.897)--(2.365,4.889)--(2.365,4.883)%
  --(2.366,4.879)--(2.366,4.874)--(2.367,4.867)--(2.367,4.858)--(2.367,4.849)--(2.368,4.839)%
  --(2.368,4.824)--(2.369,4.816)--(2.369,4.815)--(2.369,4.807)--(2.370,4.801)--(2.370,4.788)%
  --(2.370,4.783)--(2.371,4.776)--(2.371,4.768)--(2.372,4.758)--(2.372,4.750)--(2.372,4.744)%
  --(2.373,4.743)--(2.373,4.738)--(2.374,4.732)--(2.374,4.725)--(2.374,4.721)--(2.375,4.715)%
  --(2.375,4.714)--(2.376,4.709)--(2.376,4.707)--(2.376,4.705)--(2.377,4.700)--(2.377,4.694)%
  --(2.377,4.690)--(2.378,4.680)--(2.378,4.677)--(2.379,4.671)--(2.379,4.667)--(2.379,4.664)%
  --(2.380,4.660)--(2.380,4.654)--(2.381,4.649)--(2.381,4.642)--(2.381,4.637)--(2.382,4.628)%
  --(2.382,4.620)--(2.382,4.619)--(2.383,4.614)--(2.383,4.610)--(2.384,4.609)--(2.384,4.602)%
  --(2.384,4.597)--(2.385,4.589)--(2.385,4.583)--(2.386,4.579)--(2.386,4.576)--(2.386,4.570)%
  --(2.387,4.565)--(2.387,4.560)--(2.388,4.555)--(2.388,4.547)--(2.388,4.539)--(2.389,4.530)%
  --(2.389,4.520)--(2.389,4.518)--(2.390,4.512)--(2.390,4.503)--(2.391,4.499)--(2.391,4.496)%
  --(2.391,4.488)--(2.392,4.482)--(2.392,4.471)--(2.393,4.466)--(2.393,4.462)--(2.393,4.459)%
  --(2.394,4.457)--(2.394,4.452)--(2.395,4.450)--(2.395,4.447)--(2.396,4.448)--(2.396,4.445)%
  --(2.396,4.440)--(2.397,4.425)--(2.397,4.413)--(2.398,4.406)--(2.398,4.402)--(2.398,4.400)%
  --(2.399,4.399)--(2.399,4.397)--(2.400,4.395)--(2.400,4.394)--(2.400,4.392)--(2.401,4.381)%
  --(2.401,4.363)--(2.402,4.352)--(2.402,4.351)--(2.402,4.348)--(2.403,4.345)--(2.403,4.344)%
  --(2.403,4.341)--(2.404,4.341)--(2.404,4.339)--(2.405,4.341)--(2.405,4.338)--(2.405,4.336)%
  --(2.406,4.336)--(2.406,4.329)--(2.407,4.323)--(2.407,4.321)--(2.407,4.317)--(2.408,4.313)%
  --(2.408,4.306)--(2.409,4.302)--(2.409,4.300)--(2.409,4.298)--(2.410,4.292)--(2.410,4.287)%
  --(2.410,4.286)--(2.411,4.278)--(2.411,4.274)--(2.412,4.270)--(2.412,4.265)--(2.412,4.260)%
  --(2.413,4.254)--(2.413,4.246)--(2.414,4.243)--(2.414,4.238)--(2.414,4.236)--(2.415,4.229)%
  --(2.415,4.227)--(2.416,4.221)--(2.416,4.217)--(2.416,4.210)--(2.417,4.201)--(2.417,4.191)%
  --(2.417,4.187)--(2.418,4.185)--(2.418,4.179)--(2.419,4.174)--(2.419,4.167)--(2.419,4.157)%
  --(2.420,4.151)--(2.420,4.145)--(2.421,4.143)--(2.421,4.140)--(2.421,4.136)--(2.422,4.135)%
  --(2.422,4.128)--(2.422,4.125)--(2.423,4.117)--(2.423,4.114)--(2.424,4.112)--(2.424,4.107)%
  --(2.424,4.102)--(2.425,4.097)--(2.425,4.094)--(2.426,4.091)--(2.426,4.087)--(2.426,4.084)%
  --(2.427,4.081)--(2.427,4.077)--(2.428,4.071)--(2.428,4.068)--(2.429,4.066)--(2.429,4.064)%
  --(2.429,4.060)--(2.430,4.058)--(2.430,4.047)--(2.431,4.044)--(2.431,4.038)--(2.431,4.030)%
  --(2.432,4.023)--(2.432,4.019)--(2.433,4.015)--(2.433,4.011)--(2.433,4.004)--(2.434,3.997)%
  --(2.434,3.989)--(2.435,3.979)--(2.435,3.975)--(2.435,3.969)--(2.436,3.962)--(2.436,3.956)%
  --(2.436,3.951)--(2.437,3.947)--(2.437,3.944)--(2.438,3.938)--(2.438,3.929)--(2.438,3.922)%
  --(2.439,3.919)--(2.439,3.913)--(2.440,3.910)--(2.440,3.909)--(2.440,3.906)--(2.441,3.902)%
  --(2.441,3.899)--(2.442,3.898)--(2.442,3.894)--(2.442,3.889)--(2.443,3.886)--(2.443,3.884)%
  --(2.443,3.881)--(2.444,3.877)--(2.444,3.873)--(2.445,3.872)--(2.445,3.868)--(2.445,3.864)%
  --(2.446,3.860)--(2.446,3.853)--(2.447,3.851)--(2.447,3.850)--(2.447,3.848)--(2.448,3.844)%
  --(2.448,3.838)--(2.449,3.835)--(2.449,3.831)--(2.449,3.828)--(2.450,3.826)--(2.450,3.821)%
  --(2.450,3.817)--(2.451,3.812)--(2.451,3.810)--(2.452,3.803)--(2.452,3.799)--(2.452,3.796)%
  --(2.453,3.792)--(2.453,3.788)--(2.454,3.784)--(2.454,3.778)--(2.454,3.772)--(2.455,3.769)%
  --(2.455,3.763)--(2.455,3.759)--(2.456,3.752)--(2.456,3.747)--(2.457,3.744)--(2.457,3.741)%
  --(2.457,3.740)--(2.458,3.739)--(2.458,3.734)--(2.459,3.732)--(2.459,3.730)--(2.459,3.723)%
  --(2.460,3.719)--(2.460,3.715)--(2.461,3.714)--(2.461,3.712)--(2.462,3.708)--(2.462,3.703)%
  --(2.462,3.698)--(2.463,3.693)--(2.463,3.688)--(2.464,3.680)--(2.464,3.674)--(2.464,3.671)%
  --(2.465,3.668)--(2.465,3.664)--(2.466,3.663)--(2.466,3.659)--(2.466,3.654)--(2.467,3.652)%
  --(2.467,3.649)--(2.468,3.645)--(2.468,3.642)--(2.468,3.639)--(2.469,3.636)--(2.469,3.633)%
  --(2.470,3.627)--(2.470,3.622)--(2.471,3.620)--(2.471,3.616)--(2.471,3.614)--(2.472,3.610)%
  --(2.472,3.607)--(2.473,3.604)--(2.473,3.601)--(2.473,3.597)--(2.474,3.593)--(2.474,3.591)%
  --(2.475,3.587)--(2.475,3.586)--(2.475,3.583)--(2.476,3.580)--(2.476,3.577)--(2.476,3.573)%
  --(2.477,3.569)--(2.477,3.565)--(2.478,3.562)--(2.478,3.556)--(2.478,3.550)--(2.479,3.545)%
  --(2.479,3.539)--(2.480,3.536)--(2.480,3.535)--(2.480,3.528)--(2.481,3.524)--(2.481,3.517)%
  --(2.482,3.511)--(2.482,3.505)--(2.482,3.502)--(2.483,3.500)--(2.483,3.493)--(2.483,3.488)%
  --(2.484,3.484)--(2.484,3.477)--(2.485,3.474)--(2.485,3.469)--(2.485,3.466)--(2.486,3.462)%
  --(2.486,3.457)--(2.487,3.452)--(2.487,3.449)--(2.487,3.446)--(2.488,3.445)--(2.488,3.441)%
  --(2.488,3.438)--(2.489,3.437)--(2.489,3.433)--(2.490,3.431)--(2.490,3.425)--(2.490,3.420)%
  --(2.491,3.416)--(2.491,3.415)--(2.492,3.414)--(2.492,3.408)--(2.492,3.403)--(2.493,3.401)%
  --(2.493,3.398)--(2.494,3.396)--(2.494,3.393)--(2.494,3.385)--(2.495,3.383)--(2.495,3.379)%
  --(2.495,3.376)--(2.496,3.370)--(2.496,3.366)--(2.497,3.363)--(2.497,3.359)--(2.497,3.355)%
  --(2.498,3.348)--(2.498,3.345)--(2.499,3.341)--(2.499,3.340)--(2.499,3.337)--(2.500,3.334)%
  --(2.500,3.329)--(2.501,3.329)--(2.501,3.326)--(2.501,3.321)--(2.502,3.319)--(2.502,3.313)%
  --(2.502,3.312)--(2.503,3.307)--(2.503,3.303)--(2.504,3.301)--(2.504,3.297)--(2.504,3.296)%
  --(2.505,3.291)--(2.505,3.285)--(2.506,3.283)--(2.506,3.281)--(2.506,3.278)--(2.507,3.273)%
  --(2.507,3.271)--(2.508,3.268)--(2.508,3.266)--(2.508,3.263)--(2.509,3.260)--(2.509,3.258)%
  --(2.509,3.256)--(2.510,3.255)--(2.510,3.251)--(2.511,3.251)--(2.511,3.245)--(2.511,3.243)%
  --(2.512,3.239)--(2.512,3.237)--(2.513,3.235)--(2.513,3.229)--(2.513,3.222)--(2.514,3.218)%
  --(2.514,3.216)--(2.515,3.210)--(2.515,3.208)--(2.515,3.199)--(2.516,3.195)--(2.516,3.190)%
  --(2.516,3.185)--(2.517,3.181)--(2.517,3.180)--(2.518,3.175)--(2.518,3.169)--(2.518,3.163)%
  --(2.519,3.156)--(2.519,3.151)--(2.520,3.146)--(2.520,3.140)--(2.520,3.135)--(2.521,3.130)%
  --(2.521,3.128)--(2.522,3.124)--(2.522,3.123)--(2.522,3.121)--(2.523,3.119)--(2.523,3.117)%
  --(2.523,3.114)--(2.524,3.111)--(2.525,3.106)--(2.525,3.105)--(2.525,3.103)--(2.526,3.102)%
  --(2.526,3.099)--(2.527,3.096)--(2.527,3.095)--(2.527,3.090)--(2.528,3.087)--(2.528,3.082)%
  --(2.528,3.079)--(2.529,3.074)--(2.529,3.073)--(2.530,3.072)--(2.530,3.070)--(2.531,3.066)%
  --(2.531,3.062)--(2.532,3.059)--(2.532,3.055)--(2.532,3.052)--(2.533,3.049)--(2.533,3.045)%
  --(2.534,3.044)--(2.534,3.040)--(2.534,3.038)--(2.535,3.036)--(2.535,3.035)--(2.535,3.033)%
  --(2.536,3.029)--(2.536,3.026)--(2.537,3.023)--(2.537,3.017)--(2.537,3.013)--(2.538,3.008)%
  --(2.538,3.006)--(2.539,3.004)--(2.539,3.001)--(2.539,2.999)--(2.540,2.996)--(2.540,2.995)%
  --(2.541,2.993)--(2.541,2.991)--(2.541,2.989)--(2.542,2.988)--(2.542,2.985)--(2.543,2.982)%
  --(2.543,2.978)--(2.544,2.975)--(2.544,2.974)--(2.544,2.970)--(2.545,2.969)--(2.545,2.967)%
  --(2.546,2.964)--(2.546,2.961)--(2.546,2.958)--(2.547,2.954)--(2.547,2.949)--(2.548,2.945)%
  --(2.548,2.944)--(2.548,2.939)--(2.549,2.937)--(2.549,2.936)--(2.550,2.935)--(2.550,2.933)%
  --(2.551,2.927)--(2.551,2.924)--(2.551,2.919)--(2.552,2.916)--(2.552,2.914)--(2.553,2.911)%
  --(2.553,2.909)--(2.553,2.907)--(2.554,2.902)--(2.554,2.900)--(2.555,2.894)--(2.555,2.892)%
  --(2.555,2.891)--(2.556,2.886)--(2.556,2.884)--(2.557,2.881)--(2.557,2.875)--(2.558,2.871)%
  --(2.558,2.867)--(2.558,2.866)--(2.559,2.864)--(2.559,2.863)--(2.560,2.861)--(2.560,2.859)%
  --(2.560,2.857)--(2.561,2.857)--(2.561,2.855)--(2.562,2.853)--(2.562,2.852)--(2.563,2.849)%
  --(2.563,2.845)--(2.563,2.841)--(2.564,2.838)--(2.564,2.835)--(2.565,2.832)--(2.565,2.830)%
  --(2.565,2.827)--(2.566,2.823)--(2.566,2.821)--(2.567,2.818)--(2.567,2.815)--(2.567,2.814)%
  --(2.568,2.811)--(2.568,2.809)--(2.568,2.806)--(2.569,2.804)--(2.569,2.801)--(2.570,2.799)%
  --(2.570,2.797)--(2.570,2.793)--(2.571,2.793)--(2.571,2.791)--(2.572,2.788)--(2.572,2.787)%
  --(2.572,2.786)--(2.573,2.783)--(2.573,2.780)--(2.574,2.778)--(2.574,2.776)--(2.574,2.774)%
  --(2.575,2.772)--(2.575,2.770)--(2.575,2.767)--(2.576,2.765)--(2.576,2.764)--(2.577,2.760)%
  --(2.577,2.757)--(2.578,2.754)--(2.578,2.752)--(2.579,2.751)--(2.579,2.749)--(2.579,2.746)%
  --(2.580,2.745)--(2.580,2.740)--(2.581,2.737)--(2.581,2.736)--(2.581,2.731)--(2.582,2.728)%
  --(2.582,2.724)--(2.582,2.722)--(2.583,2.721)--(2.583,2.720)--(2.584,2.718)--(2.584,2.715)%
  --(2.584,2.712)--(2.585,2.708)--(2.585,2.706)--(2.586,2.704)--(2.586,2.702)--(2.586,2.701)%
  --(2.587,2.696)--(2.587,2.693)--(2.588,2.692)--(2.588,2.690)--(2.588,2.688)--(2.589,2.685)%
  --(2.589,2.681)--(2.589,2.678)--(2.590,2.677)--(2.590,2.673)--(2.591,2.671)--(2.591,2.670)%
  --(2.591,2.668)--(2.592,2.665)--(2.592,2.664)--(2.593,2.660)--(2.593,2.658)--(2.593,2.654)%
  --(2.594,2.654)--(2.594,2.653)--(2.594,2.651)--(2.595,2.648)--(2.595,2.647)--(2.596,2.643)%
  --(2.596,2.642)--(2.596,2.640)--(2.597,2.639)--(2.597,2.636)--(2.598,2.635)--(2.598,2.634)%
  --(2.598,2.632)--(2.599,2.629)--(2.599,2.628)--(2.600,2.626)--(2.600,2.624)--(2.600,2.623)%
  --(2.601,2.620)--(2.601,2.619)--(2.601,2.615)--(2.602,2.609)--(2.602,2.607)--(2.603,2.604)%
  --(2.603,2.602)--(2.603,2.598)--(2.604,2.593)--(2.604,2.589)--(2.605,2.586)--(2.605,2.583)%
  --(2.605,2.580)--(2.606,2.577)--(2.606,2.574)--(2.607,2.571)--(2.607,2.570)--(2.607,2.567)%
  --(2.608,2.565)--(2.608,2.563)--(2.608,2.561)--(2.609,2.557)--(2.609,2.553)--(2.610,2.549)%
  --(2.610,2.548)--(2.610,2.544)--(2.611,2.542)--(2.611,2.540)--(2.612,2.540)--(2.612,2.538)%
  --(2.612,2.536)--(2.613,2.534)--(2.613,2.533)--(2.614,2.531)--(2.614,2.529)--(2.614,2.528)%
  --(2.615,2.526)--(2.615,2.524)--(2.615,2.522)--(2.616,2.521)--(2.616,2.519)--(2.617,2.520)%
  --(2.617,2.515)--(2.618,2.512)--(2.618,2.510)--(2.619,2.507)--(2.619,2.505)--(2.619,2.503)%
  --(2.620,2.500)--(2.620,2.495)--(2.621,2.494)--(2.621,2.491)--(2.621,2.488)--(2.622,2.488)%
  --(2.622,2.485)--(2.622,2.484)--(2.623,2.483)--(2.623,2.479)--(2.624,2.478)--(2.624,2.475)%
  --(2.624,2.473)--(2.625,2.471)--(2.625,2.468)--(2.626,2.465)--(2.626,2.464)--(2.626,2.461)%
  --(2.627,2.457)--(2.627,2.455)--(2.628,2.449)--(2.628,2.448)--(2.628,2.447)--(2.629,2.446)%
  --(2.629,2.443)--(2.629,2.442)--(2.630,2.440)--(2.630,2.438)--(2.631,2.435)--(2.631,2.433)%
  --(2.631,2.429)--(2.632,2.428)--(2.632,2.425)--(2.633,2.425)--(2.633,2.422)--(2.634,2.420)%
  --(2.634,2.418)--(2.635,2.417)--(2.635,2.414)--(2.636,2.412)--(2.636,2.410)--(2.636,2.406)%
  --(2.637,2.403)--(2.637,2.401)--(2.638,2.400)--(2.638,2.398)--(2.638,2.395)--(2.639,2.395)%
  --(2.639,2.391)--(2.640,2.389)--(2.640,2.388)--(2.640,2.386)--(2.641,2.384)--(2.641,2.382)%
  --(2.641,2.379)--(2.642,2.377)--(2.642,2.376)--(2.643,2.374)--(2.643,2.373)--(2.643,2.371)%
  --(2.644,2.371)--(2.644,2.368)--(2.645,2.366)--(2.645,2.364)--(2.645,2.363)--(2.646,2.359)%
  --(2.646,2.356)--(2.647,2.355)--(2.647,2.352)--(2.647,2.349)--(2.648,2.346)--(2.648,2.342)%
  --(2.648,2.340)--(2.649,2.340)--(2.649,2.337)--(2.650,2.337)--(2.650,2.335)--(2.650,2.334)%
  --(2.651,2.332)--(2.651,2.329)--(2.652,2.327)--(2.652,2.325)--(2.652,2.324)--(2.653,2.323)%
  --(2.653,2.321)--(2.654,2.320)--(2.654,2.316)--(2.654,2.314)--(2.655,2.312)--(2.655,2.309)%
  --(2.656,2.308)--(2.656,2.304)--(2.657,2.303)--(2.657,2.302)--(2.657,2.299)--(2.658,2.298)%
  --(2.658,2.297)--(2.659,2.295)--(2.659,2.293)--(2.659,2.292)--(2.660,2.291)--(2.660,2.289)%
  --(2.661,2.287)--(2.661,2.286)--(2.661,2.283)--(2.662,2.281)--(2.662,2.280)--(2.662,2.277)%
  --(2.663,2.276)--(2.663,2.275)--(2.664,2.273)--(2.664,2.271)--(2.664,2.270)--(2.665,2.267)%
  --(2.665,2.265)--(2.666,2.263)--(2.666,2.260)--(2.666,2.258)--(2.667,2.258)--(2.667,2.256)%
  --(2.667,2.254)--(2.668,2.252)--(2.668,2.251)--(2.669,2.248)--(2.669,2.245)--(2.669,2.243)%
  --(2.670,2.242)--(2.670,2.241)--(2.671,2.238)--(2.671,2.237)--(2.671,2.235)--(2.672,2.234)%
  --(2.672,2.233)--(2.673,2.231)--(2.673,2.228)--(2.674,2.225)--(2.674,2.223)--(2.674,2.221)%
  --(2.675,2.221)--(2.675,2.218)--(2.676,2.216)--(2.676,2.215)--(2.677,2.214)--(2.677,2.212)%
  --(2.678,2.209)--(2.678,2.206)--(2.678,2.204)--(2.679,2.203)--(2.679,2.201)--(2.680,2.195)%
  --(2.680,2.192)--(2.680,2.190)--(2.681,2.188)--(2.681,2.186)--(2.682,2.183)--(2.683,2.182)%
  --(2.683,2.179)--(2.683,2.178)--(2.684,2.176)--(2.684,2.174)--(2.685,2.172)--(2.685,2.171)%
  --(2.685,2.170)--(2.686,2.168)--(2.686,2.166)--(2.687,2.164)--(2.687,2.161)--(2.687,2.159)%
  --(2.688,2.156)--(2.688,2.153)--(2.689,2.153)--(2.689,2.151)--(2.690,2.149)--(2.690,2.147)%
  --(2.691,2.144)--(2.691,2.143)--(2.692,2.141)--(2.692,2.139)--(2.692,2.137)--(2.693,2.135)%
  --(2.693,2.133)--(2.694,2.132)--(2.694,2.128)--(2.695,2.127)--(2.695,2.125)--(2.695,2.124)%
  --(2.696,2.122)--(2.697,2.121)--(2.697,2.119)--(2.697,2.117)--(2.698,2.115)--(2.698,2.113)%
  --(2.699,2.111)--(2.699,2.110)--(2.699,2.108)--(2.700,2.106)--(2.700,2.104)--(2.701,2.103)%
  --(2.701,2.102)--(2.701,2.100)--(2.702,2.096)--(2.702,2.094)--(2.703,2.092)--(2.703,2.090)%
  --(2.704,2.088)--(2.704,2.087)--(2.704,2.085)--(2.705,2.084)--(2.705,2.081)--(2.706,2.081)%
  --(2.706,2.079)--(2.706,2.078)--(2.707,2.076)--(2.707,2.074)--(2.707,2.073)--(2.708,2.071)%
  --(2.708,2.069)--(2.709,2.067)--(2.709,2.065)--(2.709,2.064)--(2.710,2.062)--(2.710,2.061)%
  --(2.711,2.059)--(2.711,2.058)--(2.712,2.056)--(2.712,2.054)--(2.713,2.052)--(2.713,2.050)%
  --(2.713,2.048)--(2.714,2.045)--(2.714,2.044)--(2.714,2.042)--(2.715,2.040)--(2.715,2.038)%
  --(2.716,2.036)--(2.716,2.034)--(2.716,2.031)--(2.717,2.029)--(2.717,2.027)--(2.718,2.025)%
  --(2.718,2.024)--(2.719,2.023)--(2.719,2.022)--(2.720,2.020)--(2.720,2.018)--(2.720,2.017)%
  --(2.721,2.015)--(2.721,2.012)--(2.721,2.011)--(2.722,2.011)--(2.722,2.010)--(2.723,2.008)%
  --(2.723,2.007)--(2.723,2.005)--(2.724,2.002)--(2.724,2.000)--(2.725,1.997)--(2.725,1.996)%
  --(2.725,1.993)--(2.726,1.991)--(2.727,1.989)--(2.727,1.987)--(2.728,1.987)--(2.728,1.984)%
  --(2.728,1.983)--(2.729,1.981)--(2.729,1.979)--(2.730,1.978)--(2.730,1.976)--(2.730,1.974)%
  --(2.731,1.974)--(2.731,1.973)--(2.732,1.969)--(2.732,1.967)--(2.733,1.966)--(2.733,1.965)%
  --(2.734,1.963)--(2.734,1.962)--(2.734,1.960)--(2.735,1.958)--(2.735,1.957)--(2.736,1.955)%
  --(2.736,1.953)--(2.737,1.951)--(2.737,1.949)--(2.737,1.948)--(2.738,1.947)--(2.738,1.945)%
  --(2.739,1.943)--(2.739,1.942)--(2.739,1.939)--(2.740,1.937)--(2.740,1.935)--(2.741,1.934)%
  --(2.741,1.932)--(2.742,1.932)--(2.742,1.930)--(2.742,1.927)--(2.743,1.925)--(2.743,1.922)%
  --(2.744,1.921)--(2.744,1.920)--(2.744,1.919)--(2.745,1.917)--(2.745,1.916)--(2.746,1.914)%
  --(2.746,1.911)--(2.746,1.910)--(2.747,1.908)--(2.747,1.906)--(2.747,1.904)--(2.748,1.902)%
  --(2.748,1.901)--(2.749,1.899)--(2.749,1.898)--(2.749,1.896)--(2.750,1.895)--(2.750,1.893)%
  --(2.751,1.890)--(2.751,1.889)--(2.751,1.888)--(2.752,1.887)--(2.752,1.885)--(2.753,1.885)%
  --(2.753,1.883)--(2.753,1.881)--(2.754,1.880)--(2.754,1.879)--(2.754,1.877)--(2.755,1.875)%
  --(2.755,1.873)--(2.756,1.872)--(2.756,1.870)--(2.756,1.868)--(2.757,1.865)--(2.757,1.863)%
  --(2.758,1.861)--(2.758,1.859)--(2.758,1.858)--(2.759,1.857)--(2.759,1.853)--(2.760,1.852)%
  --(2.760,1.851)--(2.760,1.850)--(2.761,1.848)--(2.761,1.847)--(2.761,1.845)--(2.762,1.845)%
  --(2.762,1.843)--(2.763,1.843)--(2.763,1.841)--(2.763,1.840)--(2.764,1.838)--(2.765,1.838)%
  --(2.765,1.836)--(2.766,1.834)--(2.766,1.830)--(2.767,1.828)--(2.767,1.825)--(2.767,1.824)%
  --(2.768,1.821)--(2.768,1.820)--(2.768,1.817)--(2.769,1.817)--(2.769,1.816)--(2.770,1.816)%
  --(2.770,1.813)--(2.770,1.812)--(2.771,1.810)--(2.771,1.808)--(2.772,1.806)--(2.772,1.805)%
  --(2.772,1.803)--(2.773,1.802)--(2.773,1.801)--(2.773,1.799)--(2.774,1.799)--(2.774,1.797)%
  --(2.775,1.795)--(2.775,1.793)--(2.775,1.791)--(2.776,1.791)--(2.777,1.789)--(2.777,1.786)%
  --(2.778,1.785)--(2.778,1.783)--(2.779,1.781)--(2.779,1.780)--(2.779,1.778)--(2.780,1.777)%
  --(2.780,1.775)--(2.780,1.774)--(2.781,1.773)--(2.781,1.772)--(2.782,1.770)--(2.782,1.769)%
  --(2.782,1.765)--(2.783,1.763)--(2.784,1.760)--(2.784,1.757)--(2.784,1.756)--(2.785,1.756)%
  --(2.786,1.754)--(2.786,1.752)--(2.786,1.750)--(2.787,1.747)--(2.787,1.746)--(2.787,1.744)%
  --(2.788,1.743)--(2.788,1.741)--(2.789,1.740)--(2.789,1.739)--(2.789,1.737)--(2.790,1.736)%
  --(2.790,1.734)--(2.791,1.732)--(2.791,1.729)--(2.791,1.728)--(2.792,1.727)--(2.793,1.726)%
  --(2.793,1.725)--(2.793,1.723)--(2.794,1.723)--(2.794,1.721)--(2.794,1.719)--(2.795,1.718)%
  --(2.795,1.717)--(2.796,1.717)--(2.796,1.715)--(2.796,1.714)--(2.797,1.713)--(2.797,1.711)%
  --(2.798,1.710)--(2.798,1.708)--(2.798,1.706)--(2.799,1.705)--(2.799,1.703)--(2.800,1.700)%
  --(2.800,1.698)--(2.800,1.696)--(2.801,1.695)--(2.801,1.694)--(2.801,1.693)--(2.802,1.692)%
  --(2.802,1.690)--(2.803,1.690)--(2.803,1.689)--(2.803,1.687)--(2.804,1.686)--(2.805,1.684)%
  --(2.806,1.682)--(2.807,1.681)--(2.807,1.679)--(2.807,1.678)--(2.808,1.676)--(2.808,1.675)%
  --(2.808,1.674)--(2.809,1.673)--(2.809,1.671)--(2.810,1.669)--(2.811,1.669)--(2.812,1.667)%
  --(2.812,1.664)--(2.812,1.663)--(2.813,1.661)--(2.813,1.659)--(2.813,1.657)--(2.814,1.656)%
  --(2.814,1.655)--(2.815,1.652)--(2.815,1.651)--(2.815,1.649)--(2.816,1.647)--(2.817,1.646)%
  --(2.817,1.645)--(2.818,1.644)--(2.818,1.643)--(2.819,1.643)--(2.819,1.641)--(2.820,1.639)%
  --(2.820,1.637)--(2.820,1.634)--(2.821,1.632)--(2.821,1.630)--(2.822,1.628)--(2.822,1.626)%
  --(2.823,1.625)--(2.824,1.624)--(2.824,1.623)--(2.824,1.622)--(2.825,1.621)--(2.826,1.619)%
  --(2.826,1.618)--(2.826,1.617)--(2.827,1.615)--(2.827,1.613)--(2.828,1.612)--(2.828,1.611)%
  --(2.829,1.609)--(2.829,1.608)--(2.829,1.607)--(2.830,1.607)--(2.830,1.605)--(2.831,1.603)%
  --(2.831,1.601)--(2.832,1.599)--(2.833,1.598)--(2.833,1.597)--(2.833,1.596)--(2.834,1.595)%
  --(2.834,1.593)--(2.834,1.601)--(2.835,1.601)--(2.836,1.602)--(2.836,1.603)--(2.837,1.603)%
  --(2.837,1.604)--(2.838,1.605)--(2.839,1.604)--(2.840,1.604)--(2.840,1.605)--(2.841,1.606)%
  --(2.841,1.607)--(2.841,1.606)--(2.842,1.607)--(2.842,1.606)--(2.843,1.607)--(2.843,1.608)%
  --(2.844,1.608)--(2.844,1.609)--(2.845,1.608)--(2.845,1.607)--(2.846,1.607)--(2.846,1.608)%
  --(2.847,1.609)--(2.848,1.610)--(2.848,1.611)--(2.848,1.610)--(2.849,1.610)--(2.850,1.610)%
  --(2.850,1.611)--(2.851,1.612)--(2.852,1.613)--(2.852,1.614)--(2.853,1.614)--(2.854,1.615)%
  --(2.855,1.616)--(2.855,1.615)--(2.856,1.616)--(2.856,1.615)--(2.857,1.616)--(2.857,1.617)%
  --(2.858,1.618)--(2.858,1.619)--(2.859,1.619)--(2.859,1.620)--(2.859,1.621)--(2.860,1.621)%
  --(2.860,1.622)--(2.861,1.623)--(2.862,1.624)--(2.862,1.623)--(2.862,1.624)--(2.863,1.626)%
  --(2.864,1.626)--(2.864,1.627)--(2.864,1.628)--(2.865,1.629)--(2.865,1.630)--(2.866,1.630)%
  --(2.866,1.631)--(2.867,1.632)--(2.867,1.633)--(2.868,1.633)--(2.868,1.634)--(2.869,1.633)%
  --(2.869,1.634)--(2.870,1.634)--(2.870,1.635)--(2.871,1.636)--(2.871,1.635)--(2.871,1.636)%
  --(2.872,1.637)--(2.873,1.638)--(2.873,1.637)--(2.874,1.638)--(2.875,1.639)--(2.875,1.640)%
  --(2.876,1.641)--(2.876,1.640)--(2.877,1.641)--(2.877,1.640)--(2.878,1.641)--(2.878,1.643)%
  --(2.878,1.645)--(2.879,1.646)--(2.879,1.645)--(2.880,1.646)--(2.880,1.647)--(2.881,1.646)%
  --(2.881,1.647)--(2.881,1.648)--(2.882,1.648)--(2.883,1.649)--(2.883,1.648)--(2.883,1.649)%
  --(2.884,1.649)--(2.884,1.650)--(2.885,1.650)--(2.885,1.651)--(2.885,1.652)--(2.886,1.652)%
  --(2.886,1.653)--(2.887,1.654)--(2.887,1.655)--(2.888,1.655)--(2.888,1.654)--(2.889,1.654)%
  --(2.890,1.654)--(2.890,1.655)--(2.891,1.655)--(2.892,1.656)--(2.892,1.655)--(2.892,1.656)%
  --(2.893,1.656)--(2.893,1.657)--(2.894,1.658)--(2.894,1.659)--(2.895,1.660)--(2.895,1.661)%
  --(2.896,1.661)--(2.896,1.662)--(2.897,1.661)--(2.897,1.662)--(2.898,1.663)--(2.899,1.663)%
  --(2.899,1.664)--(2.900,1.665)--(2.900,1.666)--(2.900,1.667)--(2.901,1.667)--(2.902,1.668)%
  --(2.902,1.669)--(2.903,1.669)--(2.903,1.671)--(2.904,1.672)--(2.904,1.673)--(2.905,1.674)%
  --(2.905,1.675)--(2.906,1.676)--(2.906,1.677)--(2.907,1.677)--(2.907,1.678)--(2.907,1.679)%
  --(2.908,1.678)--(2.908,1.679)--(2.909,1.678)--(2.909,1.679)--(2.909,1.680)--(2.910,1.679)%
  --(2.910,1.680)--(2.911,1.681)--(2.911,1.680)--(2.911,1.681)--(2.912,1.681)--(2.913,1.681)%
  --(2.913,1.682)--(2.914,1.682)--(2.914,1.683)--(2.915,1.683)--(2.915,1.684)--(2.916,1.683)%
  --(2.916,1.684)--(2.917,1.686)--(2.917,1.685)--(2.918,1.685)--(2.918,1.686)--(2.919,1.687)%
  --(2.919,1.688)--(2.920,1.688)--(2.920,1.689)--(2.921,1.689)--(2.921,1.690)--(2.922,1.691)%
  --(2.923,1.691)--(2.923,1.692)--(2.924,1.693)--(2.924,1.694)--(2.925,1.695)--(2.926,1.696)%
  --(2.927,1.697)--(2.927,1.698)--(2.928,1.698)--(2.929,1.699)--(2.930,1.699)--(2.931,1.700)%
  --(2.932,1.701)--(2.932,1.700)--(2.932,1.701)--(2.933,1.702)--(2.934,1.703)--(2.935,1.703)%
  --(2.935,1.702)--(2.936,1.701)--(2.936,1.702)--(2.937,1.702)--(2.937,1.701)--(2.937,1.702)%
  --(2.938,1.702)--(2.938,1.703)--(2.939,1.704)--(2.939,1.703)--(2.940,1.705)--(2.941,1.706)%
  --(2.942,1.707)--(2.942,1.708)--(2.942,1.709)--(2.943,1.709)--(2.943,1.707)--(2.944,1.708)%
  --(2.944,1.709)--(2.944,1.710)--(2.945,1.711)--(2.945,1.713)--(2.946,1.712)--(2.947,1.714)%
  --(2.947,1.715)--(2.947,1.716)--(2.948,1.718)--(2.948,1.719)--(2.949,1.722)--(2.949,1.723)%
  --(2.949,1.726)--(2.950,1.729)--(2.950,1.730)--(2.951,1.732)--(2.951,1.734)--(2.952,1.735)%
  --(2.952,1.737)--(2.952,1.738)--(2.953,1.739)--(2.953,1.740)--(2.954,1.741)--(2.954,1.742)%
  --(2.954,1.743)--(2.955,1.743)--(2.956,1.745)--(2.956,1.746)--(2.957,1.747)--(2.957,1.748)%
  --(2.958,1.749)--(2.958,1.751)--(2.959,1.752)--(2.959,1.754)--(2.960,1.754)--(2.960,1.756)%
  --(2.961,1.757)--(2.961,1.758)--(2.961,1.760)--(2.962,1.762)--(2.962,1.763)--(2.963,1.765)%
  --(2.963,1.767)--(2.963,1.768)--(2.964,1.770)--(2.965,1.771)--(2.965,1.772)--(2.965,1.774)%
  --(2.966,1.776)--(2.966,1.778)--(2.966,1.779)--(2.967,1.780)--(2.967,1.779)--(2.968,1.780)%
  --(2.968,1.782)--(2.968,1.783)--(2.969,1.782)--(2.969,1.784)--(2.970,1.784)--(2.970,1.785)%
  --(2.971,1.786)--(2.972,1.787)--(2.972,1.788)--(2.973,1.788)--(2.973,1.790)--(2.974,1.791)%
  --(2.974,1.792)--(2.975,1.792)--(2.975,1.793)--(2.975,1.794)--(2.976,1.796)--(2.976,1.797)%
  --(2.977,1.798)--(2.978,1.800)--(2.978,1.801)--(2.979,1.802)--(2.979,1.803)--(2.979,1.804)%
  --(2.980,1.804)--(2.980,1.805)--(2.980,1.807)--(2.981,1.808)--(2.981,1.809)--(2.982,1.809)%
  --(2.982,1.811)--(2.982,1.812)--(2.983,1.815)--(2.983,1.817)--(2.984,1.818)--(2.984,1.819)%
  --(2.984,1.820)--(2.985,1.822)--(2.985,1.823)--(2.985,1.825)--(2.986,1.827)--(2.986,1.828)%
  --(2.987,1.830)--(2.987,1.831)--(2.988,1.831)--(2.989,1.832)--(2.989,1.833)--(2.990,1.835)%
  --(2.991,1.836)--(2.991,1.837)--(2.991,1.840)--(2.992,1.844)--(2.992,1.845)--(2.993,1.845)%
  --(2.993,1.846)--(2.994,1.846)--(2.994,1.848)--(2.995,1.848)--(2.995,1.851)--(2.996,1.854)%
  --(2.996,1.857)--(2.996,1.860)--(2.997,1.861)--(2.997,1.862)--(2.998,1.861)--(2.999,1.862)%
  --(2.999,1.863)--(3.000,1.863)--(3.000,1.865)--(3.001,1.865)--(3.001,1.867)--(3.001,1.868)%
  --(3.002,1.869)--(3.002,1.870)--(3.003,1.870)--(3.003,1.872)--(3.003,1.873)--(3.004,1.874)%
  --(3.005,1.876)--(3.005,1.879)--(3.006,1.881)--(3.006,1.883)--(3.007,1.884)--(3.007,1.886)%
  --(3.008,1.888)--(3.008,1.890)--(3.008,1.891)--(3.009,1.893)--(3.009,1.894)--(3.010,1.895)%
  --(3.010,1.896)--(3.010,1.897)--(3.011,1.899)--(3.011,1.900)--(3.012,1.901)--(3.012,1.902)%
  --(3.012,1.903)--(3.013,1.905)--(3.013,1.906)--(3.013,1.909)--(3.014,1.911)--(3.014,1.913)%
  --(3.015,1.914)--(3.015,1.916)--(3.015,1.918)--(3.016,1.919)--(3.016,1.921)--(3.017,1.921)%
  --(3.018,1.923)--(3.018,1.924)--(3.019,1.926)--(3.019,1.927)--(3.019,1.929)--(3.020,1.931)%
  --(3.021,1.932)--(3.021,1.933)--(3.022,1.934)--(3.022,1.935)--(3.022,1.937)--(3.023,1.937)%
  --(3.023,1.939)--(3.024,1.941)--(3.024,1.942)--(3.024,1.943)--(3.025,1.944)--(3.025,1.946)%
  --(3.025,1.948)--(3.026,1.949)--(3.026,1.951)--(3.027,1.952)--(3.027,1.955)--(3.028,1.956)%
  --(3.028,1.957)--(3.029,1.960)--(3.029,1.962)--(3.029,1.964)--(3.030,1.964)--(3.030,1.967)%
  --(3.031,1.967)--(3.031,1.969)--(3.031,1.971)--(3.032,1.973)--(3.032,1.974)--(3.032,1.978)%
  --(3.033,1.981)--(3.033,1.983)--(3.034,1.985)--(3.034,1.986)--(3.034,1.987)--(3.035,1.988)%
  --(3.035,1.989)--(3.036,1.989)--(3.036,1.990)--(3.036,1.992)--(3.037,1.992)--(3.037,1.994)%
  --(3.038,1.995)--(3.038,1.997)--(3.038,1.999)--(3.039,1.998)--(3.039,1.999)--(3.040,2.000)%
  --(3.040,2.001)--(3.041,2.003)--(3.041,2.005)--(3.041,2.006)--(3.042,2.008)--(3.042,2.010)%
  --(3.043,2.010)--(3.043,2.011)--(3.043,2.012)--(3.044,2.012)--(3.044,2.013)--(3.045,2.015)%
  --(3.045,2.016)--(3.045,2.017)--(3.046,2.018)--(3.046,2.020)--(3.047,2.021)--(3.047,2.022)%
  --(3.048,2.025)--(3.048,2.026)--(3.048,2.028)--(3.049,2.030)--(3.049,2.032)--(3.050,2.032)%
  --(3.050,2.034)--(3.050,2.035)--(3.051,2.036)--(3.051,2.038)--(3.052,2.041)--(3.052,2.040)%
  --(3.052,2.042)--(3.053,2.044)--(3.053,2.045)--(3.054,2.046)--(3.054,2.048)--(3.055,2.049)%
  --(3.055,2.051)--(3.056,2.052)--(3.056,2.054)--(3.057,2.055)--(3.057,2.057)--(3.057,2.059)%
  --(3.058,2.060)--(3.058,2.062)--(3.059,2.063)--(3.059,2.065)--(3.060,2.066)--(3.060,2.069)%
  --(3.061,2.071)--(3.061,2.072)--(3.062,2.073)--(3.062,2.074)--(3.062,2.075)--(3.063,2.077)%
  --(3.063,2.078)--(3.064,2.078)--(3.064,2.080)--(3.065,2.082)--(3.065,2.083)--(3.065,2.085)%
  --(3.066,2.086)--(3.066,2.087)--(3.067,2.088)--(3.067,2.090)--(3.068,2.091)--(3.068,2.092)%
  --(3.069,2.093)--(3.069,2.095)--(3.070,2.098)--(3.070,2.100)--(3.071,2.100)--(3.071,2.102)%
  --(3.071,2.104)--(3.072,2.106)--(3.072,2.109)--(3.072,2.112)--(3.073,2.114)--(3.073,2.115)%
  --(3.074,2.118)--(3.074,2.120)--(3.074,2.122)--(3.075,2.124)--(3.075,2.125)--(3.076,2.128)%
  --(3.076,2.131)--(3.077,2.132)--(3.077,2.133)--(3.078,2.135)--(3.078,2.137)--(3.078,2.139)%
  --(3.079,2.141)--(3.079,2.143)--(3.080,2.145)--(3.080,2.146)--(3.081,2.149)--(3.081,2.151)%
  --(3.081,2.152)--(3.082,2.153)--(3.082,2.154)--(3.083,2.154)--(3.083,2.155)--(3.083,2.157)%
  --(3.084,2.158)--(3.084,2.160)--(3.085,2.162)--(3.085,2.163)--(3.085,2.165)--(3.086,2.167)%
  --(3.086,2.170)--(3.086,2.171)--(3.087,2.172)--(3.087,2.175)--(3.088,2.177)--(3.088,2.178)%
  --(3.088,2.181)--(3.089,2.183)--(3.089,2.185)--(3.090,2.187)--(3.090,2.189)--(3.090,2.192)%
  --(3.091,2.193)--(3.091,2.195)--(3.091,2.197)--(3.092,2.197)--(3.092,2.199)--(3.093,2.201)%
  --(3.093,2.203)--(3.094,2.206)--(3.095,2.207)--(3.095,2.209)--(3.095,2.210)--(3.096,2.211)%
  --(3.096,2.213)--(3.097,2.217)--(3.097,2.218)--(3.097,2.221)--(3.098,2.220)--(3.098,2.222)%
  --(3.099,2.223)--(3.099,2.225)--(3.100,2.225)--(3.100,2.226)--(3.100,2.229)--(3.101,2.229)%
  --(3.101,2.230)--(3.102,2.232)--(3.102,2.234)--(3.102,2.235)--(3.103,2.237)--(3.103,2.238)%
  --(3.104,2.240)--(3.104,2.241)--(3.104,2.243)--(3.105,2.244)--(3.105,2.246)--(3.106,2.248)%
  --(3.107,2.249)--(3.107,2.252)--(3.107,2.253)--(3.108,2.255)--(3.108,2.257)--(3.109,2.257)%
  --(3.109,2.259)--(3.110,2.263)--(3.110,2.265)--(3.111,2.267)--(3.111,2.271)--(3.111,2.275)%
  --(3.112,2.278)--(3.112,2.281)--(3.112,2.282)--(3.113,2.287)--(3.113,2.290)--(3.114,2.293)%
  --(3.114,2.295)--(3.114,2.298)--(3.115,2.301)--(3.115,2.304)--(3.116,2.305)--(3.116,2.307)%
  --(3.117,2.307)--(3.117,2.309)--(3.118,2.309)--(3.118,2.312)--(3.119,2.313)--(3.119,2.314)%
  --(3.119,2.315)--(3.120,2.317)--(3.121,2.318)--(3.121,2.320)--(3.121,2.322)--(3.122,2.322)%
  --(3.122,2.324)--(3.123,2.325)--(3.123,2.327)--(3.123,2.328)--(3.124,2.329)--(3.125,2.330)%
  --(3.125,2.332)--(3.125,2.334)--(3.126,2.335)--(3.126,2.337)--(3.126,2.340)--(3.127,2.340)%
  --(3.127,2.342)--(3.128,2.344)--(3.128,2.345)--(3.128,2.348)--(3.129,2.349)--(3.129,2.351)%
  --(3.130,2.352)--(3.130,2.354)--(3.130,2.355)--(3.131,2.355)--(3.131,2.357)--(3.131,2.360)%
  --(3.132,2.360)--(3.132,2.362)--(3.133,2.364)--(3.133,2.366)--(3.133,2.367)--(3.134,2.369)%
  --(3.134,2.370)--(3.135,2.372)--(3.135,2.374)--(3.135,2.376)--(3.136,2.377)--(3.136,2.380)%
  --(3.137,2.382)--(3.137,2.384)--(3.137,2.385)--(3.138,2.387)--(3.138,2.389)--(3.138,2.390)%
  --(3.139,2.391)--(3.139,2.393)--(3.140,2.395)--(3.140,2.397)--(3.140,2.398)--(3.141,2.399)%
  --(3.141,2.401)--(3.142,2.403)--(3.142,2.405)--(3.143,2.406)--(3.143,2.407)--(3.144,2.411)%
  --(3.144,2.412)--(3.144,2.414)--(3.145,2.414)--(3.145,2.416)--(3.146,2.419)--(3.147,2.420)%
  --(3.147,2.422)--(3.147,2.424)--(3.148,2.426)--(3.148,2.428)--(3.149,2.429)--(3.149,2.432)%
  --(3.149,2.433)--(3.150,2.435)--(3.150,2.437)--(3.151,2.438)--(3.151,2.439)--(3.151,2.440)%
  --(3.152,2.445)--(3.152,2.448)--(3.152,2.453)--(3.153,2.454)--(3.153,2.456)--(3.154,2.458)%
  --(3.154,2.461)--(3.154,2.463)--(3.155,2.465)--(3.155,2.468)--(3.156,2.469)--(3.156,2.471)%
  --(3.156,2.472)--(3.157,2.473)--(3.157,2.475)--(3.158,2.476)--(3.158,2.478)--(3.158,2.481)%
  --(3.159,2.484)--(3.159,2.486)--(3.159,2.488)--(3.160,2.491)--(3.160,2.490)--(3.161,2.492)%
  --(3.161,2.494)--(3.161,2.496)--(3.162,2.497)--(3.162,2.500)--(3.163,2.503)--(3.163,2.504)%
  --(3.163,2.506)--(3.164,2.506)--(3.164,2.507)--(3.165,2.509)--(3.166,2.511)--(3.166,2.514)%
  --(3.167,2.517)--(3.167,2.521)--(3.168,2.522)--(3.168,2.523)--(3.168,2.525)--(3.169,2.527)%
  --(3.169,2.529)--(3.170,2.531)--(3.170,2.532)--(3.170,2.534)--(3.171,2.537)--(3.171,2.538)%
  --(3.171,2.540)--(3.172,2.543)--(3.172,2.545)--(3.173,2.546)--(3.173,2.548)--(3.173,2.549)%
  --(3.174,2.551)--(3.174,2.552)--(3.175,2.554)--(3.175,2.558)--(3.175,2.560)--(3.176,2.563)%
  --(3.176,2.565)--(3.177,2.567)--(3.177,2.569)--(3.177,2.570)--(3.178,2.571)--(3.178,2.575)%
  --(3.178,2.576)--(3.179,2.578)--(3.179,2.580)--(3.180,2.583)--(3.180,2.587)--(3.181,2.588)%
  --(3.182,2.592)--(3.182,2.593)--(3.182,2.594)--(3.183,2.596)--(3.183,2.597)--(3.184,2.598)%
  --(3.184,2.601)--(3.184,2.602)--(3.185,2.605)--(3.185,2.610)--(3.186,2.613)--(3.186,2.616)%
  --(3.187,2.618)--(3.187,2.621)--(3.187,2.624)--(3.188,2.627)--(3.188,2.628)--(3.189,2.629)%
  --(3.189,2.634)--(3.189,2.635)--(3.190,2.637)--(3.190,2.640)--(3.191,2.640)--(3.191,2.642)%
  --(3.191,2.643)--(3.192,2.646)--(3.192,2.648)--(3.193,2.650)--(3.193,2.651)--(3.194,2.654)%
  --(3.194,2.657)--(3.194,2.660)--(3.195,2.662)--(3.195,2.664)--(3.196,2.666)--(3.196,2.670)%
  --(3.196,2.674)--(3.197,2.677)--(3.197,2.679)--(3.197,2.683)--(3.198,2.686)--(3.198,2.688)%
  --(3.199,2.693)--(3.199,2.696)--(3.199,2.700)--(3.200,2.701)--(3.200,2.706)--(3.201,2.709)%
  --(3.201,2.713)--(3.201,2.715)--(3.202,2.718)--(3.202,2.720)--(3.203,2.723)--(3.203,2.724)%
  --(3.203,2.727)--(3.204,2.728)--(3.204,2.730)--(3.204,2.731)--(3.205,2.734)--(3.205,2.737)%
  --(3.206,2.738)--(3.206,2.741)--(3.206,2.743)--(3.207,2.745)--(3.207,2.747)--(3.208,2.750)%
  --(3.208,2.752)--(3.209,2.753)--(3.209,2.755)--(3.210,2.756)--(3.210,2.761)--(3.210,2.762)%
  --(3.211,2.765)--(3.211,2.766)--(3.211,2.768)--(3.212,2.771)--(3.212,2.774)--(3.213,2.775)%
  --(3.213,2.777)--(3.213,2.779)--(3.214,2.781)--(3.214,2.785)--(3.215,2.789)--(3.215,2.792)%
  --(3.215,2.794)--(3.216,2.798)--(3.216,2.801)--(3.217,2.805)--(3.217,2.808)--(3.217,2.811)%
  --(3.218,2.818)--(3.218,2.821)--(3.218,2.825)--(3.219,2.828)--(3.219,2.831)--(3.220,2.834)%
  --(3.220,2.837)--(3.220,2.840)--(3.221,2.845)--(3.221,2.846)--(3.222,2.848)--(3.222,2.852)%
  --(3.222,2.854)--(3.223,2.856)--(3.223,2.857)--(3.224,2.860)--(3.224,2.863)--(3.224,2.865)%
  --(3.225,2.868)--(3.225,2.871)--(3.225,2.872)--(3.226,2.874)--(3.226,2.876)--(3.227,2.877)%
  --(3.227,2.879)--(3.227,2.881)--(3.228,2.885)--(3.228,2.887)--(3.229,2.890)--(3.229,2.891)%
  --(3.229,2.893)--(3.230,2.894)--(3.230,2.895)--(3.231,2.897)--(3.231,2.901)--(3.231,2.904)%
  --(3.232,2.909)--(3.232,2.911)--(3.232,2.912)--(3.233,2.915)--(3.233,2.917)--(3.234,2.919)%
  --(3.234,2.923)--(3.234,2.926)--(3.235,2.930)--(3.235,2.931)--(3.236,2.935)--(3.236,2.938)%
  --(3.236,2.939)--(3.237,2.941)--(3.237,2.945)--(3.237,2.946)--(3.238,2.950)--(3.238,2.952)%
  --(3.239,2.955)--(3.239,2.957)--(3.239,2.960)--(3.240,2.963)--(3.240,2.969)--(3.241,2.972)%
  --(3.241,2.975)--(3.241,2.978)--(3.242,2.981)--(3.242,2.986)--(3.243,2.990)--(3.243,2.993)%
  --(3.243,2.995)--(3.244,2.997)--(3.244,3.000)--(3.245,3.000)--(3.245,3.002)--(3.246,3.005)%
  --(3.246,3.006)--(3.246,3.009)--(3.247,3.010)--(3.247,3.013)--(3.248,3.014)--(3.248,3.016)%
  --(3.249,3.019)--(3.249,3.020)--(3.250,3.023)--(3.250,3.026)--(3.250,3.031)--(3.251,3.035)%
  --(3.251,3.039)--(3.251,3.042)--(3.252,3.045)--(3.252,3.046)--(3.253,3.046)--(3.253,3.050)%
  --(3.253,3.051)--(3.254,3.057)--(3.254,3.061)--(3.255,3.062)--(3.255,3.064)--(3.255,3.070)%
  --(3.256,3.073)--(3.256,3.075)--(3.257,3.078)--(3.257,3.083)--(3.257,3.087)--(3.258,3.091)%
  --(3.258,3.095)--(3.258,3.099)--(3.259,3.102)--(3.259,3.106)--(3.260,3.108)--(3.260,3.111)%
  --(3.260,3.115)--(3.261,3.118)--(3.261,3.120)--(3.262,3.122)--(3.262,3.123)--(3.262,3.125)%
  --(3.263,3.128)--(3.263,3.129)--(3.264,3.130)--(3.264,3.132)--(3.264,3.133)--(3.265,3.137)%
  --(3.265,3.140)--(3.266,3.142)--(3.266,3.146)--(3.267,3.148)--(3.267,3.151)--(3.267,3.153)%
  --(3.268,3.154)--(3.268,3.158)--(3.269,3.161)--(3.269,3.166)--(3.269,3.167)--(3.270,3.171)%
  --(3.270,3.172)--(3.270,3.175)--(3.271,3.178)--(3.271,3.180)--(3.272,3.185)--(3.272,3.187)%
  --(3.272,3.192)--(3.273,3.196)--(3.273,3.200)--(3.274,3.203)--(3.274,3.206)--(3.274,3.210)%
  --(3.275,3.213)--(3.275,3.217)--(3.276,3.219)--(3.276,3.223)--(3.276,3.228)--(3.277,3.233)%
  --(3.277,3.236)--(3.277,3.239)--(3.278,3.242)--(3.278,3.244)--(3.279,3.247)--(3.279,3.249)%
  --(3.279,3.251)--(3.280,3.253)--(3.280,3.256)--(3.281,3.259)--(3.281,3.264)--(3.281,3.267)%
  --(3.282,3.269)--(3.282,3.272)--(3.283,3.274)--(3.283,3.278)--(3.283,3.281)--(3.284,3.284)%
  --(3.284,3.286)--(3.284,3.288)--(3.285,3.293)--(3.285,3.298)--(3.286,3.300)--(3.286,3.302)%
  --(3.286,3.304)--(3.287,3.306)--(3.287,3.310)--(3.288,3.314)--(3.288,3.317)--(3.288,3.319)%
  --(3.289,3.323)--(3.289,3.326)--(3.290,3.329)--(3.290,3.332)--(3.290,3.335)--(3.291,3.339)%
  --(3.291,3.343)--(3.291,3.347)--(3.292,3.350)--(3.292,3.355)--(3.293,3.358)--(3.293,3.363)%
  --(3.293,3.364)--(3.294,3.367)--(3.294,3.370)--(3.295,3.371)--(3.295,3.372)--(3.295,3.375)%
  --(3.296,3.380)--(3.296,3.383)--(3.297,3.386)--(3.297,3.389)--(3.297,3.395)--(3.298,3.396)%
  --(3.298,3.399)--(3.298,3.403)--(3.299,3.407)--(3.299,3.409)--(3.300,3.413)--(3.300,3.415)%
  --(3.301,3.419)--(3.301,3.425)--(3.302,3.427)--(3.302,3.432)--(3.303,3.434)--(3.303,3.436)%
  --(3.303,3.439)--(3.304,3.442)--(3.304,3.444)--(3.305,3.446)--(3.305,3.451)--(3.305,3.454)%
  --(3.306,3.460)--(3.306,3.464)--(3.307,3.466)--(3.307,3.472)--(3.307,3.473)--(3.308,3.477)%
  --(3.308,3.481)--(3.309,3.485)--(3.309,3.486)--(3.309,3.490)--(3.310,3.493)--(3.310,3.498)%
  --(3.310,3.502)--(3.311,3.503)--(3.311,3.508)--(3.312,3.512)--(3.312,3.519)--(3.312,3.522)%
  --(3.313,3.525)--(3.313,3.527)--(3.314,3.531)--(3.314,3.535)--(3.314,3.537)--(3.315,3.540)%
  --(3.315,3.544)--(3.316,3.546)--(3.316,3.549)--(3.316,3.554)--(3.317,3.556)--(3.317,3.559)%
  --(3.317,3.563)--(3.318,3.568)--(3.318,3.569)--(3.319,3.573)--(3.319,3.575)--(3.319,3.578)%
  --(3.320,3.580)--(3.320,3.584)--(3.321,3.586)--(3.321,3.591)--(3.322,3.596)--(3.322,3.597)%
  --(3.323,3.599)--(3.323,3.603)--(3.323,3.612)--(3.324,3.614)--(3.324,3.615)--(3.324,3.620)%
  --(3.325,3.622)--(3.325,3.624)--(3.326,3.628)--(3.326,3.630)--(3.326,3.634)--(3.327,3.638)%
  --(3.327,3.641)--(3.328,3.647)--(3.328,3.650)--(3.329,3.651)--(3.329,3.655)--(3.330,3.661)%
  --(3.330,3.666)--(3.330,3.672)--(3.331,3.676)--(3.331,3.679)--(3.331,3.683)--(3.332,3.685)%
  --(3.332,3.689)--(3.333,3.691)--(3.333,3.694)--(3.333,3.696)--(3.334,3.702)--(3.334,3.705)%
  --(3.335,3.710)--(3.335,3.715)--(3.335,3.717)--(3.336,3.721)--(3.336,3.726)--(3.337,3.727)%
  --(3.337,3.729)--(3.337,3.734)--(3.338,3.739)--(3.338,3.740)--(3.338,3.745)--(3.339,3.750)%
  --(3.339,3.752)--(3.340,3.758)--(3.340,3.762)--(3.340,3.766)--(3.341,3.769)--(3.341,3.772)%
  --(3.342,3.777)--(3.342,3.783)--(3.342,3.786)--(3.343,3.790)--(3.343,3.793)--(3.343,3.801)%
  --(3.344,3.805)--(3.344,3.808)--(3.345,3.809)--(3.345,3.813)--(3.345,3.816)--(3.346,3.820)%
  --(3.346,3.822)--(3.347,3.823)--(3.347,3.830)--(3.347,3.833)--(3.348,3.838)--(3.348,3.845)%
  --(3.349,3.854)--(3.349,3.860)--(3.349,3.867)--(3.350,3.874)--(3.350,3.881)--(3.350,3.887)%
  --(3.351,3.890)--(3.351,3.894)--(3.352,3.901)--(3.352,3.905)--(3.352,3.909)--(3.353,3.913)%
  --(3.353,3.914)--(3.354,3.916)--(3.354,3.920)--(3.354,3.923)--(3.355,3.927)--(3.355,3.930)%
  --(3.356,3.932)--(3.356,3.937)--(3.356,3.942)--(3.357,3.947)--(3.357,3.950)--(3.357,3.955)%
  --(3.358,3.959)--(3.358,3.962)--(3.359,3.966)--(3.359,3.967)--(3.359,3.969)--(3.360,3.974)%
  --(3.360,3.982)--(3.361,3.987)--(3.361,3.994)--(3.361,3.999)--(3.362,4.002)--(3.362,4.004)%
  --(3.363,4.005)--(3.363,4.009)--(3.363,4.014)--(3.364,4.016)--(3.364,4.018)--(3.364,4.024)%
  --(3.365,4.029)--(3.365,4.032)--(3.366,4.035)--(3.366,4.041)--(3.366,4.043)--(3.367,4.048)%
  --(3.367,4.052)--(3.368,4.057)--(3.368,4.062)--(3.368,4.066)--(3.369,4.073)--(3.369,4.078)%
  --(3.370,4.082)--(3.370,4.089)--(3.370,4.093)--(3.371,4.097)--(3.371,4.101)--(3.371,4.105)%
  --(3.372,4.110)--(3.372,4.111)--(3.373,4.116)--(3.373,4.119)--(3.373,4.122)--(3.374,4.127)%
  --(3.374,4.131)--(3.375,4.135)--(3.375,4.136)--(3.375,4.140)--(3.376,4.146)--(3.376,4.148)%
  --(3.376,4.151)--(3.377,4.156)--(3.377,4.160)--(3.378,4.164)--(3.378,4.169)--(3.378,4.172)%
  --(3.379,4.177)--(3.379,4.179)--(3.380,4.180)--(3.380,4.182)--(3.380,4.185)--(3.381,4.191)%
  --(3.381,4.197)--(3.382,4.205)--(3.382,4.211)--(3.382,4.214)--(3.383,4.220)--(3.383,4.225)%
  --(3.383,4.231)--(3.384,4.235)--(3.384,4.238)--(3.385,4.243)--(3.385,4.247)--(3.385,4.252)%
  --(3.386,4.257)--(3.386,4.260)--(3.387,4.265)--(3.387,4.269)--(3.387,4.279)--(3.388,4.283)%
  --(3.388,4.291)--(3.389,4.295)--(3.389,4.301)--(3.389,4.311)--(3.390,4.318)--(3.390,4.324)%
  --(3.390,4.329)--(3.391,4.334)--(3.391,4.340)--(3.392,4.343)--(3.392,4.347)--(3.392,4.349)%
  --(3.393,4.354)--(3.393,4.359)--(3.394,4.366)--(3.394,4.374)--(3.394,4.384)--(3.395,4.387)%
  --(3.395,4.394)--(3.396,4.398)--(3.396,4.402)--(3.396,4.405)--(3.397,4.409)--(3.397,4.411)%
  --(3.397,4.416)--(3.398,4.424)--(3.398,4.435)--(3.399,4.438)--(3.399,4.440)--(3.400,4.443)%
  --(3.400,4.448)--(3.401,4.452)--(3.401,4.457)--(3.401,4.465)--(3.402,4.474)--(3.402,4.478)%
  --(3.403,4.483)--(3.403,4.493)--(3.403,4.500)--(3.404,4.503)--(3.404,4.508)--(3.404,4.514)%
  --(3.405,4.520)--(3.405,4.528)--(3.406,4.535)--(3.406,4.536)--(3.406,4.542)--(3.407,4.548)%
  --(3.407,4.555)--(3.408,4.561)--(3.408,4.569)--(3.408,4.574)--(3.409,4.581)--(3.409,4.586)%
  --(3.409,4.589)--(3.410,4.596)--(3.410,4.600)--(3.411,4.605)--(3.411,4.607)--(3.411,4.611)%
  --(3.412,4.617)--(3.412,4.627)--(3.413,4.633)--(3.413,4.636)--(3.413,4.644)--(3.414,4.645)%
  --(3.414,4.652)--(3.415,4.661)--(3.415,4.663)--(3.415,4.668)--(3.416,4.674)--(3.416,4.679)%
  --(3.416,4.685)--(3.417,4.688)--(3.417,4.697)--(3.418,4.703)--(3.418,4.710)--(3.418,4.721)%
  --(3.419,4.728)--(3.419,4.736)--(3.420,4.741)--(3.420,4.747)--(3.420,4.755)--(3.421,4.759)%
  --(3.421,4.763)--(3.422,4.770)--(3.422,4.778)--(3.422,4.784)--(3.423,4.790)--(3.423,4.798)%
  --(3.423,4.804)--(3.424,4.810)--(3.424,4.822)--(3.425,4.828)--(3.425,4.831)--(3.425,4.838)%
  --(3.426,4.843)--(3.426,4.850)--(3.427,4.854)--(3.427,4.864)--(3.427,4.874)--(3.428,4.885)%
  --(3.428,4.896)--(3.429,4.903)--(3.429,4.916)--(3.429,4.920)--(3.430,4.919)--(3.430,4.928)%
  --(3.430,4.936)--(3.431,4.941)--(3.431,4.944)--(3.432,4.954)--(3.432,4.962)--(3.432,4.972)%
  --(3.433,4.982)--(3.433,4.990)--(3.434,4.995)--(3.434,5.000)--(3.434,5.004)--(3.435,5.008)%
  --(3.435,5.015)--(3.436,5.023)--(3.436,5.031)--(3.436,5.038)--(3.437,5.043)--(3.437,5.047)%
  --(3.437,5.054)--(3.438,5.066)--(3.438,5.073)--(3.439,5.079)--(3.439,5.088)--(3.439,5.096)%
  --(3.440,5.103)--(3.440,5.110)--(3.441,5.111)--(3.441,5.118)--(3.441,5.126)--(3.442,5.137)%
  --(3.442,5.142)--(3.443,5.151)--(3.443,5.159)--(3.443,5.164)--(3.444,5.167)--(3.444,5.171)%
  --(3.444,5.175)--(3.445,5.182)--(3.445,5.184)--(3.446,5.192)--(3.446,5.206)--(3.446,5.219)%
  --(3.447,5.229)--(3.447,5.231)--(3.448,5.238)--(3.448,5.241)--(3.448,5.244)--(3.449,5.254)%
  --(3.449,5.264)--(3.449,5.276)--(3.450,5.287)--(3.450,5.298)--(3.451,5.304)--(3.451,5.311)%
  --(3.451,5.319)--(3.452,5.324)--(3.452,5.334)--(3.453,5.349)--(3.453,5.354)--(3.453,5.365)%
  --(3.454,5.379)--(3.454,5.390)--(3.455,5.397)--(3.455,5.404)--(3.455,5.415)--(3.456,5.419)%
  --(3.456,5.426)--(3.456,5.435)--(3.457,5.438)--(3.457,5.440)--(3.458,5.447)--(3.458,5.449)%
  --(3.458,5.457)--(3.459,5.463)--(3.459,5.471)--(3.460,5.478)--(3.460,5.490)--(3.460,5.494)%
  --(3.461,5.504)--(3.461,5.515)--(3.462,5.527)--(3.462,5.538)--(3.462,5.547)--(3.463,5.555)%
  --(3.463,5.560)--(3.463,5.569)--(3.464,5.574)--(3.464,5.579)--(3.465,5.589)--(3.465,5.593)%
  --(3.465,5.602)--(3.466,5.607)--(3.466,5.617)--(3.467,5.620)--(3.467,5.629)--(3.467,5.634)%
  --(3.468,5.636)--(3.468,5.638)--(3.469,5.645)--(3.469,5.654)--(3.469,5.667)--(3.470,5.679)%
  --(3.470,5.698)--(3.470,5.710)--(3.471,5.726)--(3.471,5.739)--(3.472,5.748)--(3.472,5.756)%
  --(3.472,5.764)--(3.473,5.769)--(3.473,5.779)--(3.474,5.790)--(3.474,5.801)--(3.474,5.813)%
  --(3.475,5.823)--(3.475,5.829)--(3.476,5.838)--(3.476,5.851)--(3.476,5.858)--(3.477,5.865)%
  --(3.477,5.868)--(3.477,5.876)--(3.478,5.887)--(3.478,5.900)--(3.479,5.907)--(3.479,5.914)%
  --(3.479,5.917)--(3.480,5.925)--(3.481,5.933)--(3.481,5.935)--(3.481,5.944)--(3.482,5.954)%
  --(3.482,5.961)--(3.482,5.970)--(3.483,5.984)--(3.483,5.995)--(3.484,6.006)--(3.484,6.009)%
  --(3.484,6.017)--(3.485,6.025)--(3.485,6.036)--(3.486,6.054)--(3.486,6.066)--(3.486,6.077)%
  --(3.487,6.092)--(3.487,6.107)--(3.488,6.121)--(3.488,6.128)--(3.488,6.135)--(3.489,6.146)%
  --(3.489,6.164)--(3.489,6.178)--(3.490,6.192)--(3.490,6.202)--(3.491,6.211)--(3.491,6.219)%
  --(3.491,6.229)--(3.492,6.250)--(3.492,6.262)--(3.493,6.272)--(3.493,6.287)--(3.493,6.298)%
  --(3.494,6.311)--(3.494,6.327)--(3.495,6.332)--(3.495,6.343)--(3.495,6.351)--(3.496,6.361)%
  --(3.496,6.366)--(3.496,6.373)--(3.497,6.385)--(3.497,6.397)--(3.498,6.403)--(3.498,6.406)%
  --(3.498,6.413)--(3.499,6.423)--(3.499,6.435)--(3.500,6.442)--(3.500,6.453)--(3.500,6.456)%
  --(3.501,6.471)--(3.501,6.484)--(3.502,6.503)--(3.502,6.512)--(3.502,6.526)--(3.503,6.543)%
  --(3.503,6.560)--(3.503,6.571)--(3.504,6.582)--(3.504,6.593)--(3.505,6.608)--(3.505,6.626)%
  --(3.505,6.631)--(3.506,6.631)--(3.506,6.639)--(3.507,6.650)--(3.507,6.655)--(3.507,6.659)%
  --(3.508,6.669)--(3.508,6.679)--(3.509,6.687)--(3.509,6.697)--(3.509,6.708)--(3.510,6.713)%
  --(3.510,6.724)--(3.510,6.735)--(3.511,6.746)--(3.511,6.751)--(3.512,6.766)--(3.512,6.767)%
  --(3.512,6.785)--(3.513,6.801)--(3.513,6.809)--(3.514,6.817)--(3.514,6.818)--(3.515,6.823)%
  --(3.515,6.841)--(3.515,6.856)--(3.516,6.877)--(3.516,6.887)--(3.517,6.898)--(3.517,6.911)%
  --(3.517,6.927)--(3.518,6.947)--(3.518,6.966)--(3.519,6.982)--(3.519,7.002)--(3.519,7.014)%
  --(3.520,7.017)--(3.520,7.023)--(3.521,7.026)--(3.521,7.038)--(3.521,7.041)--(3.522,7.044)%
  --(3.522,7.050)--(3.522,7.056)--(3.523,7.067)--(3.523,7.094)--(3.524,7.115)--(3.524,7.135)%
  --(3.524,7.158)--(3.525,7.176)--(3.525,7.189)--(3.526,7.201)--(3.526,7.208)--(3.526,7.212)%
  --(3.527,7.220)--(3.527,7.222)--(3.528,7.229)--(3.528,7.245)--(3.528,7.262)--(3.529,7.279)%
  --(3.529,7.290)--(3.529,7.307)--(3.530,7.321)--(3.530,7.336)--(3.531,7.354)--(3.531,7.368)%
  --(3.531,7.373)--(3.532,7.385)--(3.532,7.390)--(3.533,7.401)--(3.533,7.402)--(3.533,7.410)%
  --(3.534,7.421)--(3.534,7.431)--(3.535,7.436)--(3.535,7.445)--(3.535,7.451)--(3.536,7.459)%
  --(3.536,7.463)--(3.536,7.471)--(3.537,7.474)--(3.537,7.479)--(3.538,6.658)--(3.538,6.649)%
  --(3.538,6.635)--(3.539,6.619)--(3.539,6.596)--(3.540,6.583)--(3.540,6.570)--(3.540,6.554)%
  --(3.541,6.539)--(3.541,6.532)--(3.542,6.527)--(3.542,6.521)--(3.542,6.513)--(3.543,6.510)%
  --(3.543,6.501)--(3.543,6.499)--(3.544,6.487)--(3.544,6.476)--(3.545,6.470)--(3.545,6.466)%
  --(3.545,6.455)--(3.546,6.442)--(3.546,6.432)--(3.547,6.430)--(3.547,6.422)--(3.547,6.416)%
  --(3.548,6.406)--(3.548,6.390)--(3.549,6.380)--(3.549,6.371)--(3.549,6.363)--(3.550,6.353)%
  --(3.550,6.340)--(3.550,6.326)--(3.551,6.314)--(3.551,6.296)--(3.552,6.284)--(3.552,6.280)%
  --(3.552,6.271)--(3.553,6.271)--(3.553,6.265)--(3.554,6.257)--(3.554,6.252)--(3.554,6.244)%
  --(3.555,6.235)--(3.555,6.233)--(3.555,6.232)--(3.556,6.226)--(3.556,6.216)--(3.557,6.203)%
  --(3.557,6.194)--(3.557,6.186)--(3.558,6.174)--(3.558,6.165)--(3.559,6.159)--(3.559,6.147)%
  --(3.559,6.141)--(3.560,6.126)--(3.560,6.119)--(3.561,6.108)--(3.561,6.100)--(3.561,6.094)%
  --(3.562,6.088)--(3.562,6.080)--(3.562,6.073)--(3.563,6.061)--(3.563,6.051)--(3.564,6.048)%
  --(3.564,6.040)--(3.564,6.028)--(3.565,6.019)--(3.565,6.014)--(3.566,6.011)--(3.566,6.004)%
  --(3.566,5.996)--(3.567,5.991)--(3.567,5.976)--(3.568,5.971)--(3.568,5.966)--(3.568,5.961)%
  --(3.569,5.954)--(3.569,5.941)--(3.569,5.928)--(3.570,5.921)--(3.570,5.910)--(3.571,5.901)%
  --(3.571,5.899)--(3.571,5.887)--(3.572,5.879)--(3.572,5.872)--(3.573,5.868)--(3.573,5.865)%
  --(3.573,5.863)--(3.574,5.855)--(3.574,5.843)--(3.575,5.840)--(3.575,5.835)--(3.575,5.827)%
  --(3.576,5.823)--(3.576,5.817)--(3.576,5.810)--(3.577,5.805)--(3.577,5.799)--(3.578,5.791)%
  --(3.578,5.779)--(3.578,5.771)--(3.579,5.760)--(3.579,5.751)--(3.580,5.743)--(3.580,5.737)%
  --(3.580,5.730)--(3.581,5.729)--(3.581,5.717)--(3.582,5.716)--(3.582,5.713)--(3.582,5.707)%
  --(3.583,5.701)--(3.583,5.692)--(3.583,5.686)--(3.584,5.676)--(3.584,5.673)--(3.585,5.666)%
  --(3.585,5.664)--(3.585,5.658)--(3.586,5.646)--(3.586,5.638)--(3.587,5.632)--(3.587,5.627)%
  --(3.588,5.621)--(3.588,5.610)--(3.588,5.605)--(3.589,5.597)--(3.589,5.594)--(3.590,5.585)%
  --(3.590,5.582)--(3.590,5.571)--(3.591,5.564)--(3.591,5.558)--(3.592,5.553)--(3.592,5.543)%
  --(3.592,5.531)--(3.593,5.526)--(3.593,5.523)--(3.594,5.522)--(3.594,5.513)--(3.594,5.501)%
  --(3.595,5.496)--(3.595,5.484)--(3.595,5.477)--(3.596,5.470)--(3.596,5.464)--(3.597,5.460)%
  --(3.597,5.455)--(3.597,5.448)--(3.598,5.442)--(3.598,5.436)--(3.599,5.426)--(3.599,5.417)%
  --(3.599,5.415)--(3.600,5.408)--(3.600,5.397)--(3.601,5.393)--(3.601,5.386)--(3.601,5.381)%
  --(3.602,5.372)--(3.602,5.367)--(3.602,5.362)--(3.603,5.352)--(3.603,5.343)--(3.604,5.338)%
  --(3.604,5.328)--(3.604,5.319)--(3.605,5.311)--(3.605,5.300)--(3.606,5.294)--(3.606,5.290)%
  --(3.606,5.288)--(3.607,5.288)--(3.607,5.287)--(3.608,5.284)--(3.608,5.282)--(3.608,5.276)%
  --(3.609,5.273)--(3.609,5.269)--(3.609,5.266)--(3.610,5.264)--(3.610,5.260)--(3.611,5.258)%
  --(3.611,5.255)--(3.611,5.251)--(3.612,5.247)--(3.612,5.239)--(3.613,5.235)--(3.613,5.229)%
  --(3.613,5.226)--(3.614,5.217)--(3.614,5.211)--(3.615,5.206)--(3.615,5.204)--(3.615,5.202)%
  --(3.616,5.194)--(3.616,5.192)--(3.616,5.184)--(3.617,5.183)--(3.617,5.175)--(3.618,5.168)%
  --(3.618,5.162)--(3.618,5.155)--(3.619,5.151)--(3.619,5.149)--(3.620,5.143)--(3.620,5.138)%
  --(3.621,5.132)--(3.621,5.127)--(3.622,5.124)--(3.622,5.119)--(3.622,5.112)--(3.623,5.107)%
  --(3.623,5.102)--(3.623,5.096)--(3.624,5.089)--(3.624,5.081)--(3.625,5.079)--(3.625,5.073)%
  --(3.626,5.072)--(3.626,5.063)--(3.627,5.056)--(3.627,5.052)--(3.627,5.049)--(3.628,5.044)%
  --(3.628,5.037)--(3.629,5.031)--(3.629,5.027)--(3.630,5.019)--(3.630,5.014)--(3.630,5.013)%
  --(3.631,5.007)--(3.631,5.002)--(3.632,4.999)--(3.632,4.996)--(3.632,4.991)--(3.633,4.990)%
  --(3.633,4.989)--(3.634,4.985)--(3.634,4.982)--(3.634,4.979)--(3.635,4.974)--(3.635,4.965)%
  --(3.635,4.959)--(3.636,4.953)--(3.637,4.943)--(3.637,4.941)--(3.637,4.934)--(3.638,4.933)%
  --(3.638,4.927)--(3.639,4.922)--(3.639,4.917)--(3.639,4.910)--(3.640,4.907)--(3.640,4.905)%
  --(3.641,4.904)--(3.641,4.901)--(3.641,4.896)--(3.642,4.892)--(3.642,4.887)--(3.642,4.877)%
  --(3.643,4.874)--(3.643,4.867)--(3.644,4.867)--(3.644,4.861)--(3.644,4.858)--(3.645,4.852)%
  --(3.645,4.850)--(3.646,4.845)--(3.646,4.842)--(3.646,4.837)--(3.647,4.832)--(3.647,4.828)%
  --(3.648,4.822)--(3.648,4.819)--(3.648,4.816)--(3.649,4.813)--(3.649,4.810)--(3.649,4.807)%
  --(3.650,4.801)--(3.650,4.797)--(3.651,4.791)--(3.651,4.785)--(3.651,4.783)--(3.652,4.784)%
  --(3.652,4.775)--(3.653,4.772)--(3.653,4.769)--(3.653,4.761)--(3.654,4.756)--(3.654,4.753)%
  --(3.655,4.748)--(3.655,4.745)--(3.655,4.737)--(3.656,4.732)--(3.656,4.731)--(3.656,4.728)%
  --(3.657,4.726)--(3.657,4.719)--(3.658,4.719)--(3.658,4.717)--(3.658,4.708)--(3.659,4.707)%
  --(3.659,4.702)--(3.660,4.700)--(3.660,4.697)--(3.660,4.694)--(3.661,4.692)--(3.661,4.691)%
  --(3.662,4.689)--(3.662,4.690)--(3.663,4.686)--(3.663,4.684)--(3.664,4.688)--(3.664,4.686)%
  --(3.665,4.681)--(3.665,4.682)--(3.665,4.680)--(3.666,4.680)--(3.666,4.649)--(3.667,4.632)%
  --(3.667,4.628)--(3.667,4.622)--(3.668,4.613)--(3.668,4.602)--(3.668,4.589)--(3.669,4.584)%
  --(3.669,4.582)--(3.670,4.577)--(3.670,4.571)--(3.670,4.561)--(3.671,4.549)--(3.671,4.544)%
  --(3.672,4.536)--(3.672,4.527)--(3.672,4.518)--(3.673,4.514)--(3.673,4.515)--(3.674,4.511)%
  --(3.674,4.507)--(3.675,4.509)--(3.675,4.507)--(3.675,4.501)--(3.676,4.497)--(3.676,4.490)%
  --(3.677,4.487)--(3.677,4.484)--(3.677,4.474)--(3.678,4.470)--(3.678,4.464)--(3.679,4.457)%
  --(3.679,4.451)--(3.679,4.448)--(3.680,4.443)--(3.680,4.439)--(3.681,4.434)--(3.681,4.432)%
  --(3.682,4.433)--(3.682,4.432)--(3.682,4.429)--(3.683,4.428)--(3.684,4.430)--(3.684,4.418)%
  --(3.684,4.401)--(3.685,4.385)--(3.686,4.386)--(3.686,4.383)--(3.686,4.380)--(3.687,4.375)%
  --(3.687,4.371)--(3.688,4.362)--(3.688,4.355)--(3.688,4.349)--(3.689,4.341)--(3.689,4.338)%
  --(3.689,4.332)--(3.690,4.327)--(3.691,4.325)--(3.691,4.321)--(3.691,4.317)--(3.692,4.316)%
  --(3.692,4.315)--(3.693,4.314)--(3.693,4.308)--(3.693,4.303)--(3.694,4.301)--(3.694,4.296)%
  --(3.694,4.291)--(3.695,4.289)--(3.695,4.285)--(3.696,4.279)--(3.696,4.277)--(3.696,4.275)%
  --(3.697,4.270)--(3.697,4.266)--(3.698,4.259)--(3.698,4.249)--(3.698,4.242)--(3.699,4.239)%
  --(3.699,4.233)--(3.700,4.226)--(3.700,4.222)--(3.700,4.215)--(3.701,4.210)--(3.701,4.204)%
  --(3.701,4.199)--(3.702,4.195)--(3.702,4.192)--(3.703,4.191)--(3.703,4.188)--(3.703,4.186)%
  --(3.704,4.176)--(3.704,4.172)--(3.705,4.168)--(3.705,4.164)--(3.705,4.162)--(3.706,4.161)%
  --(3.706,4.155)--(3.707,4.154)--(3.707,4.149)--(3.708,4.145)--(3.708,4.142)--(3.708,4.139)%
  --(3.709,4.140)--(3.709,4.134)--(3.710,4.131)--(3.710,4.126)--(3.710,4.118)--(3.711,4.115)%
  --(3.711,4.109)--(3.712,4.104)--(3.712,4.099)--(3.712,4.097)--(3.713,4.091)--(3.713,4.086)%
  --(3.714,4.078)--(3.714,4.073)--(3.714,4.067)--(3.715,4.058)--(3.715,4.052)--(3.715,4.049)%
  --(3.716,4.046)--(3.716,4.039)--(3.717,4.029)--(3.717,4.022)--(3.717,4.015)--(3.718,4.010)%
  --(3.718,4.006)--(3.719,4.003)--(3.719,4.001)--(3.719,3.999)--(3.720,3.997)--(3.720,3.991)%
  --(3.721,3.990)--(3.721,3.984)--(3.721,3.983)--(3.722,3.980)--(3.722,3.978)--(3.723,3.975)%
  --(3.723,3.973)--(3.724,3.969)--(3.724,3.966)--(3.724,3.964)--(3.725,3.960)--(3.725,3.958)%
  --(3.726,3.958)--(3.726,3.955)--(3.726,3.951)--(3.727,3.946)--(3.727,3.943)--(3.728,3.937)%
  --(3.728,3.934)--(3.728,3.930)--(3.729,3.923)--(3.729,3.918)--(3.729,3.915)--(3.730,3.911)%
  --(3.730,3.906)--(3.731,3.904)--(3.731,3.897)--(3.731,3.892)--(3.732,3.888)--(3.732,3.883)%
  --(3.733,3.880)--(3.733,3.876)--(3.733,3.875)--(3.734,3.864)--(3.734,3.860)--(3.734,3.854)%
  --(3.735,3.850)--(3.735,3.844)--(3.736,3.840)--(3.736,3.833)--(3.736,3.830)--(3.737,3.826)%
  --(3.737,3.821)--(3.738,3.818)--(3.738,3.815)--(3.738,3.813)--(3.739,3.809)--(3.739,3.805)%
  --(3.740,3.800)--(3.740,3.796)--(3.740,3.794)--(3.741,3.789)--(3.741,3.783)--(3.741,3.777)%
  --(3.742,3.774)--(3.742,3.765)--(3.743,3.762)--(3.743,3.757)--(3.743,3.756)--(3.744,3.753)%
  --(3.744,3.751)--(3.745,3.746)--(3.745,3.742)--(3.745,3.736)--(3.746,3.734)--(3.746,3.732)%
  --(3.747,3.730)--(3.747,3.723)--(3.747,3.718)--(3.748,3.711)--(3.748,3.708)--(3.748,3.702)%
  --(3.749,3.695)--(3.750,3.689)--(3.750,3.685)--(3.750,3.681)--(3.751,3.680)--(3.751,3.678)%
  --(3.752,3.677)--(3.752,3.675)--(3.752,3.669)--(3.753,3.663)--(3.753,3.659)--(3.754,3.657)%
  --(3.754,3.655)--(3.754,3.653)--(3.755,3.649)--(3.755,3.645)--(3.756,3.642)--(3.756,3.640)%
  --(3.757,3.636)--(3.757,3.632)--(3.757,3.626)--(3.758,3.625)--(3.758,3.622)--(3.759,3.620)%
  --(3.759,3.616)--(3.759,3.614)--(3.760,3.611)--(3.760,3.607)--(3.761,3.603)--(3.761,3.598)%
  --(3.761,3.594)--(3.762,3.589)--(3.762,3.585)--(3.762,3.580)--(3.763,3.575)--(3.763,3.569)%
  --(3.764,3.568)--(3.764,3.562)--(3.764,3.559)--(3.765,3.551)--(3.765,3.548)--(3.766,3.546)%
  --(3.766,3.544)--(3.766,3.540)--(3.767,3.538)--(3.767,3.535)--(3.767,3.534)--(3.768,3.531)%
  --(3.768,3.529)--(3.769,3.524)--(3.769,3.519)--(3.769,3.517)--(3.770,3.513)--(3.770,3.515)%
  --(3.771,3.511)--(3.771,3.509)--(3.771,3.506)--(3.772,3.499)--(3.772,3.491)--(3.773,3.488)%
  --(3.773,3.487)--(3.773,3.481)--(3.774,3.476)--(3.774,3.473)--(3.774,3.472)--(3.775,3.468)%
  --(3.776,3.463)--(3.776,3.460)--(3.776,3.453)--(3.777,3.446)--(3.777,3.440)--(3.778,3.434)%
  --(3.778,3.429)--(3.778,3.421)--(3.779,3.418)--(3.779,3.414)--(3.780,3.412)--(3.780,3.407)%
  --(3.780,3.402)--(3.781,3.395)--(3.781,3.391)--(3.781,3.393)--(3.782,3.391)--(3.782,3.388)%
  --(3.783,3.386)--(3.783,3.382)--(3.783,3.378)--(3.784,3.373)--(3.784,3.369)--(3.785,3.365)%
  --(3.785,3.360)--(3.785,3.357)--(3.786,3.354)--(3.786,3.350)--(3.787,3.348)--(3.787,3.342)%
  --(3.787,3.338)--(3.788,3.331)--(3.788,3.329)--(3.788,3.327)--(3.789,3.324)--(3.789,3.319)%
  --(3.790,3.315)--(3.790,3.308)--(3.790,3.301)--(3.791,3.295)--(3.791,3.291)--(3.792,3.285)%
  --(3.792,3.280)--(3.792,3.277)--(3.793,3.274)--(3.793,3.273)--(3.794,3.270)--(3.794,3.267)%
  --(3.795,3.265)--(3.795,3.263)--(3.795,3.259)--(3.796,3.256)--(3.797,3.254)--(3.797,3.251)%
  --(3.797,3.250)--(3.798,3.244)--(3.798,3.241)--(3.799,3.238)--(3.799,3.236)--(3.799,3.234)%
  --(3.800,3.231)--(3.800,3.230)--(3.800,3.228)--(3.801,3.226)--(3.801,3.221)--(3.802,3.217)%
  --(3.802,3.216)--(3.802,3.213)--(3.803,3.210)--(3.803,3.208)--(3.804,3.206)--(3.804,3.201)%
  --(3.804,3.202)--(3.805,3.196)--(3.805,3.194)--(3.806,3.191)--(3.806,3.188)--(3.806,3.185)%
  --(3.807,3.183)--(3.807,3.181)--(3.807,3.179)--(3.808,3.174)--(3.808,3.172)--(3.809,3.169)%
  --(3.809,3.165)--(3.809,3.161)--(3.810,3.157)--(3.810,3.155)--(3.811,3.152)--(3.811,3.150)%
  --(3.812,3.149)--(3.812,3.146)--(3.813,3.143)--(3.813,3.140)--(3.813,3.139)--(3.814,3.138)%
  --(3.814,3.135)--(3.814,3.134)--(3.815,3.133)--(3.815,3.130)--(3.816,3.125)--(3.816,3.121)%
  --(3.816,3.115)--(3.817,3.113)--(3.817,3.112)--(3.818,3.110)--(3.818,3.109)--(3.818,3.106)%
  --(3.819,3.104)--(3.819,3.101)--(3.820,3.096)--(3.820,3.093)--(3.820,3.091)--(3.821,3.090)%
  --(3.821,3.087)--(3.821,3.084)--(3.822,3.081)--(3.822,3.079)--(3.823,3.076)--(3.823,3.074)%
  --(3.823,3.070)--(3.824,3.065)--(3.825,3.063)--(3.825,3.060)--(3.825,3.059)--(3.826,3.059)%
  --(3.826,3.057)--(3.827,3.055)--(3.827,3.051)--(3.827,3.046)--(3.828,3.044)--(3.828,3.041)%
  --(3.828,3.038)--(3.829,3.035)--(3.829,3.034)--(3.830,3.031)--(3.830,3.028)--(3.830,3.025)%
  --(3.831,3.025)--(3.831,3.024)--(3.832,3.022)--(3.832,3.021)--(3.833,3.020)--(3.834,3.017)%
  --(3.834,3.014)--(3.834,3.011)--(3.835,3.006)--(3.835,3.003)--(3.835,3.000)--(3.836,2.996)%
  --(3.836,2.989)--(3.837,2.986)--(3.837,2.982)--(3.837,2.978)--(3.838,2.975)--(3.838,2.972)%
  --(3.839,2.969)--(3.839,2.966)--(3.839,2.963)--(3.840,2.960)--(3.840,2.957)--(3.840,2.955)%
  --(3.841,2.953)--(3.842,2.950)--(3.842,2.947)--(3.842,2.944)--(3.843,2.940)--(3.843,2.938)%
  --(3.844,2.936)--(3.844,2.934)--(3.844,2.932)--(3.845,2.927)--(3.845,2.923)--(3.846,2.919)%
  --(3.846,2.910)--(3.846,2.904)--(3.847,2.900)--(3.847,2.899)--(3.847,2.897)--(3.848,2.897)%
  --(3.848,2.895)--(3.849,2.891)--(3.849,2.889)--(3.849,2.886)--(3.850,2.883)--(3.850,2.881)%
  --(3.851,2.878)--(3.851,2.874)--(3.851,2.871)--(3.852,2.871)--(3.852,2.865)--(3.853,2.863)%
  --(3.853,2.860)--(3.853,2.859)--(3.854,2.856)--(3.854,2.853)--(3.854,2.849)--(3.855,2.848)%
  --(3.856,2.846)--(3.856,2.845)--(3.856,2.843)--(3.857,2.840)--(3.857,2.839)--(3.858,2.837)%
  --(3.858,2.835)--(3.858,2.834)--(3.859,2.832)--(3.859,2.828)--(3.860,2.825)--(3.860,2.824)%
  --(3.860,2.820)--(3.861,2.816)--(3.861,2.814)--(3.861,2.811)--(3.862,2.808)--(3.862,2.807)%
  --(3.863,2.804)--(3.863,2.801)--(3.863,2.799)--(3.864,2.797)--(3.864,2.795)--(3.865,2.794)%
  --(3.865,2.792)--(3.866,2.790)--(3.867,2.788)--(3.867,2.786)--(3.867,2.785)--(3.868,2.783)%
  --(3.868,2.782)--(3.868,2.778)--(3.869,2.774)--(3.869,2.773)--(3.870,2.772)--(3.870,2.769)%
  --(3.870,2.768)--(3.871,2.765)--(3.871,2.763)--(3.872,2.760)--(3.872,2.759)--(3.872,2.756)%
  --(3.873,2.755)--(3.873,2.753)--(3.873,2.751)--(3.874,2.750)--(3.874,2.748)--(3.875,2.746)%
  --(3.875,2.745)--(3.875,2.744)--(3.876,2.742)--(3.876,2.739)--(3.877,2.737)--(3.877,2.735)%
  --(3.877,2.732)--(3.878,2.730)--(3.878,2.727)--(3.879,2.725)--(3.879,2.724)--(3.879,2.721)%
  --(3.880,2.720)--(3.880,2.716)--(3.880,2.714)--(3.881,2.712)--(3.882,2.709)--(3.882,2.707)%
  --(3.882,2.705)--(3.883,2.703)--(3.883,2.702)--(3.884,2.699)--(3.884,2.697)--(3.884,2.696)%
  --(3.885,2.695)--(3.885,2.693)--(3.886,2.691)--(3.886,2.689)--(3.887,2.684)--(3.887,2.683)%
  --(3.887,2.681)--(3.888,2.681)--(3.888,2.680)--(3.889,2.680)--(3.889,2.677)--(3.889,2.676)%
  --(3.890,2.673)--(3.891,2.672)--(3.891,2.670)--(3.891,2.668)--(3.892,2.665)--(3.892,2.664)%
  --(3.893,2.665)--(3.893,2.662)--(3.893,2.659)--(3.894,2.656)--(3.894,2.654)--(3.894,2.653)%
  --(3.895,2.652)--(3.895,2.650)--(3.896,2.648)--(3.896,2.646)--(3.896,2.642)--(3.897,2.640)%
  --(3.898,2.638)--(3.898,2.637)--(3.898,2.636)--(3.899,2.632)--(3.899,2.631)--(3.900,2.629)%
  --(3.900,2.627)--(3.901,2.626)--(3.901,2.624)--(3.901,2.621)--(3.902,2.621)--(3.902,2.619)%
  --(3.903,2.618)--(3.903,2.616)--(3.904,2.615)--(3.905,2.615)--(3.905,2.613)--(3.905,2.612)%
  --(3.906,2.610)--(3.906,2.609)--(3.906,2.607)--(3.907,2.605)--(3.908,2.602)--(3.908,2.601)%
  --(3.908,2.599)--(3.909,2.599)--(3.909,2.597)--(3.910,2.597)--(3.910,2.596)--(3.910,2.595)%
  --(3.911,2.595)--(3.911,2.594)--(3.912,2.592)--(3.912,2.591)--(3.913,2.588)--(3.913,2.586)%
  --(3.913,2.584)--(3.914,2.583)--(3.914,2.582)--(3.915,2.581)--(3.915,2.579)--(3.915,2.578)%
  --(3.916,2.577)--(3.917,2.575)--(3.917,2.573)--(3.917,2.572)--(3.918,2.569)--(3.918,2.567)%
  --(3.919,2.565)--(3.919,2.564)--(3.919,2.562)--(3.920,2.560)--(3.920,2.556)--(3.920,2.553)%
  --(3.921,2.552)--(3.921,2.551)--(3.922,2.548)--(3.922,2.546)--(3.923,2.544)--(3.923,2.542)%
  --(3.924,2.542)--(3.924,2.539)--(3.924,2.537)--(3.925,2.535)--(3.925,2.533)--(3.926,2.530)%
  --(3.926,2.528)--(3.927,2.527)--(3.927,2.525)--(3.927,2.523)--(3.928,2.522)--(3.928,2.520)%
  --(3.929,2.518)--(3.929,2.515)--(3.929,2.514)--(3.930,2.514)--(3.930,2.512)--(3.931,2.509)%
  --(3.931,2.508)--(3.931,2.505)--(3.932,2.502)--(3.932,2.500)--(3.933,2.498)--(3.933,2.497)%
  --(3.933,2.494)--(3.934,2.491)--(3.934,2.488)--(3.934,2.486)--(3.935,2.484)--(3.935,2.483)%
  --(3.936,2.482)--(3.936,2.480)--(3.936,2.479)--(3.937,2.478)--(3.937,2.477)--(3.938,2.475)%
  --(3.938,2.473)--(3.938,2.472)--(3.939,2.467)--(3.939,2.465)--(3.940,2.462)--(3.940,2.461)%
  --(3.941,2.461)--(3.941,2.460)--(3.941,2.458)--(3.942,2.456)--(3.942,2.454)--(3.943,2.452)%
  --(3.943,2.451)--(3.943,2.449)--(3.944,2.448)--(3.945,2.446)--(3.945,2.443)--(3.945,2.442)%
  --(3.946,2.440)--(3.946,2.435)--(3.946,2.434)--(3.947,2.431)--(3.947,2.430)--(3.948,2.429)%
  --(3.948,2.428)--(3.948,2.426)--(3.949,2.425)--(3.949,2.422)--(3.950,2.421)--(3.950,2.418)%
  --(3.950,2.417)--(3.951,2.416)--(3.951,2.414)--(3.952,2.412)--(3.952,2.409)--(3.953,2.408)%
  --(3.953,2.406)--(3.953,2.403)--(3.954,2.401)--(3.954,2.399)--(3.955,2.398)--(3.955,2.396)%
  --(3.956,2.395)--(3.956,2.394)--(3.957,2.392)--(3.957,2.390)--(3.957,2.387)--(3.958,2.384)%
  --(3.958,2.382)--(3.959,2.381)--(3.959,2.378)--(3.959,2.377)--(3.960,2.376)--(3.960,2.374)%
  --(3.960,2.373)--(3.961,2.371)--(3.961,2.368)--(3.962,2.367)--(3.962,2.365)--(3.962,2.364)%
  --(3.963,2.362)--(3.964,2.361)--(3.964,2.359)--(3.964,2.358)--(3.965,2.357)--(3.965,2.356)%
  --(3.966,2.354)--(3.966,2.352)--(3.966,2.350)--(3.967,2.348)--(3.967,2.346)--(3.967,2.344)%
  --(3.968,2.343)--(3.968,2.342)--(3.969,2.340)--(3.969,2.337)--(3.969,2.335)--(3.970,2.335)%
  --(3.970,2.334)--(3.971,2.333)--(3.971,2.330)--(3.971,2.327)--(3.972,2.324)--(3.972,2.322)%
  --(3.973,2.319)--(3.973,2.316)--(3.973,2.312)--(3.974,2.310)--(3.974,2.308)--(3.974,2.306)%
  --(3.975,2.304)--(3.976,2.302)--(3.976,2.301)--(3.977,2.299)--(3.978,2.299)--(3.978,2.298)%
  --(3.978,2.297)--(3.979,2.297)--(3.979,2.296)--(3.979,2.294)--(3.980,2.292)--(3.980,2.291)%
  --(3.981,2.289)--(3.981,2.286)--(3.981,2.285)--(3.982,2.284)--(3.982,2.283)--(3.983,2.281)%
  --(3.983,2.280)--(3.984,2.279)--(3.984,2.277)--(3.985,2.277)--(3.985,2.275)--(3.986,2.273)%
  --(3.986,2.271)--(3.987,2.269)--(3.987,2.267)--(3.988,2.267)--(3.988,2.266)--(3.988,2.263)%
  --(3.989,2.262)--(3.989,2.259)--(3.990,2.257)--(3.990,2.256)--(3.990,2.253)--(3.991,2.252)%
  --(3.991,2.251)--(3.992,2.249)--(3.992,2.246)--(3.993,2.246)--(3.993,2.244)--(3.993,2.245)%
  --(3.994,2.243)--(3.994,2.240)--(3.995,2.238)--(3.995,2.237)--(3.995,2.235)--(3.996,2.233)%
  --(3.996,2.231)--(3.997,2.232)--(3.997,2.230)--(3.997,2.228)--(3.998,2.226)--(3.998,2.224)%
  --(3.999,2.223)--(3.999,2.222)--(3.999,2.221)--(4.000,2.219)--(4.000,2.217)--(4.001,2.217)%
  --(4.001,2.214)--(4.002,2.213)--(4.002,2.212)--(4.002,2.210)--(4.003,2.209)--(4.003,2.207)%
  --(4.004,2.207)--(4.004,2.206)--(4.005,2.203)--(4.005,2.201)--(4.006,2.199)--(4.006,2.197)%
  --(4.006,2.196)--(4.007,2.194)--(4.007,2.193)--(4.008,2.191)--(4.008,2.189)--(4.009,2.189)%
  --(4.009,2.188)--(4.009,2.187)--(4.010,2.186)--(4.011,2.185)--(4.011,2.184)--(4.012,2.183)%
  --(4.012,2.182)--(4.012,2.181)--(4.013,2.180)--(4.013,2.178)--(4.014,2.176)--(4.014,2.174)%
  --(4.014,2.169)--(4.015,2.168)--(4.015,2.166)--(4.016,2.164)--(4.017,2.164)--(4.017,2.162)%
  --(4.018,2.160)--(4.018,2.158)--(4.018,2.156)--(4.019,2.156)--(4.019,2.154)--(4.019,2.152)%
  --(4.020,2.152)--(4.020,2.150)--(4.021,2.147)--(4.021,2.146)--(4.021,2.145)--(4.022,2.143)%
  --(4.022,2.145)--(4.023,2.142)--(4.023,2.141)--(4.023,2.140)--(4.024,2.139)--(4.024,2.138)%
  --(4.025,2.137)--(4.025,2.135)--(4.025,2.133)--(4.026,2.131)--(4.026,2.129)--(4.026,2.127)%
  --(4.027,2.125)--(4.027,2.123)--(4.028,2.124)--(4.028,2.123)--(4.029,2.123)--(4.029,2.122)%
  --(4.030,2.123)--(4.030,2.121)--(4.030,2.120)--(4.031,2.119)--(4.031,2.118)--(4.032,2.118)%
  --(4.032,2.116)--(4.033,2.116)--(4.033,2.114)--(4.033,2.113)--(4.034,2.113)--(4.034,2.111)%
  --(4.035,2.110)--(4.035,2.107)--(4.035,2.105)--(4.036,2.105)--(4.036,2.104)--(4.037,2.104)%
  --(4.037,2.101)--(4.038,2.099)--(4.038,2.097)--(4.039,2.096)--(4.039,2.094)--(4.040,2.092)%
  --(4.040,2.090)--(4.041,2.089)--(4.041,2.087)--(4.042,2.086)--(4.042,2.085)--(4.042,2.084)%
  --(4.043,2.084)--(4.043,2.082)--(4.044,2.080)--(4.044,2.079)--(4.044,2.080)--(4.045,2.078)%
  --(4.045,2.077)--(4.046,2.076)--(4.046,2.074)--(4.046,2.072)--(4.047,2.072)--(4.047,2.071)%
  --(4.047,2.070)--(4.048,2.069)--(4.048,2.068)--(4.049,2.068)--(4.049,2.066)--(4.050,2.063)%
  --(4.050,2.062)--(4.051,2.062)--(4.051,2.061)--(4.051,2.059)--(4.052,2.057)--(4.052,2.055)%
  --(4.053,2.052)--(4.053,2.051)--(4.054,2.050)--(4.054,2.047)--(4.054,2.045)--(4.055,2.044)%
  --(4.055,2.043)--(4.056,2.042)--(4.056,2.040)--(4.056,2.039)--(4.057,2.037)--(4.057,2.035)%
  --(4.058,2.035)--(4.058,2.028)--(4.059,2.028)--(4.059,2.029)--(4.059,2.028)--(4.060,2.028)%
  --(4.060,2.026)--(4.061,2.026)--(4.061,2.027)--(4.061,2.026)--(4.062,2.025)--(4.063,2.025)%
  --(4.063,2.026)--(4.063,2.025)--(4.064,2.024)--(4.064,2.023)--(4.065,2.023)--(4.065,2.022)%
  --(4.066,2.022)--(4.067,2.023)--(4.067,2.022)--(4.068,2.020)--(4.069,2.020)--(4.070,2.020)%
  --(4.070,2.021)--(4.070,2.020)--(4.071,2.019)--(4.071,2.020)--(4.072,2.019)--(4.072,2.018)%
  --(4.072,2.019)--(4.073,2.019)--(4.073,2.018)--(4.073,2.019)--(4.074,2.019)--(4.075,2.018)%
  --(4.075,2.017)--(4.076,2.015)--(4.076,2.016)--(4.077,2.015)--(4.077,2.014)--(4.078,2.013)%
  --(4.078,2.014)--(4.079,2.013)--(4.079,2.014)--(4.079,2.012)--(4.080,2.012)--(4.080,2.011)%
  --(4.080,2.012)--(4.081,2.012)--(4.082,2.012)--(4.082,2.013)--(4.082,2.012)--(4.083,2.012)%
  --(4.083,2.013)--(4.084,2.013)--(4.084,2.015)--(4.085,2.015)--(4.085,2.016)--(4.086,2.014)%
  --(4.087,2.013)--(4.087,2.014)--(4.087,2.013)--(4.088,2.012)--(4.088,2.013)--(4.089,2.012)%
  --(4.089,2.011)--(4.090,2.011)--(4.091,2.011)--(4.091,2.010)--(4.091,2.009)--(4.092,2.010)%
  --(4.093,2.010)--(4.094,2.009)--(4.095,2.009)--(4.095,2.010)--(4.096,2.009)--(4.096,2.008)%
  --(4.097,2.008)--(4.097,2.007)--(4.098,2.008)--(4.098,2.009)--(4.099,2.009)--(4.099,2.010)%
  --(4.100,2.009)--(4.100,2.008)--(4.101,2.008)--(4.102,2.008)--(4.102,2.009)--(4.103,2.009)%
  --(4.103,2.010)--(4.104,2.010)--(4.105,2.009)--(4.105,2.008)--(4.105,2.009)--(4.106,2.009)%
  --(4.106,2.008)--(4.107,2.008)--(4.107,2.009)--(4.108,2.008)--(4.108,2.007)--(4.108,2.008)%
  --(4.109,2.007)--(4.109,2.006)--(4.110,2.006)--(4.110,2.007)--(4.111,2.006)--(4.111,2.007)%
  --(4.112,2.006)--(4.112,2.005)--(4.112,2.004)--(4.113,2.004)--(4.114,2.004)--(4.115,2.005)%
  --(4.115,2.004)--(4.116,2.004)--(4.116,2.003)--(4.117,2.002)--(4.117,2.003)--(4.118,2.004)%
  --(4.118,2.003)--(4.119,2.002)--(4.119,2.001)--(4.120,2.002)--(4.120,2.001)--(4.120,2.002)%
  --(4.121,2.002)--(4.121,2.001)--(4.122,2.000)--(4.122,2.001)--(4.122,1.999)--(4.123,1.999)%
  --(4.124,1.999)--(4.124,1.997)--(4.125,1.998)--(4.125,1.997)--(4.126,1.996)--(4.126,1.995)%
  --(4.127,1.995)--(4.128,1.996)--(4.128,1.995)--(4.129,1.996)--(4.129,1.995)--(4.130,1.995)%
  --(4.130,1.994)--(4.131,1.994)--(4.131,1.993)--(4.131,1.992)--(4.132,1.992)--(4.133,1.992)%
  --(4.133,1.991)--(4.134,1.989)--(4.135,1.988)--(4.136,1.988)--(4.136,1.987)--(4.137,1.986)%
  --(4.137,1.985)--(4.138,1.986)--(4.139,1.985)--(4.139,1.984)--(4.139,1.985)--(4.140,1.984)%
  --(4.141,1.983)--(4.141,1.982)--(4.141,1.983)--(4.142,1.983)--(4.143,1.982)--(4.143,1.983)%
  --(4.143,1.984)--(4.144,1.982)--(4.145,1.982)--(4.146,1.982)--(4.146,1.981)--(4.146,1.982)%
  --(4.147,1.982)--(4.148,1.981)--(4.148,1.980)--(4.149,1.980)--(4.149,1.979)--(4.150,1.980)%
  --(4.150,1.979)--(4.151,1.978)--(4.152,1.979)--(4.152,1.977)--(4.152,1.976)--(4.153,1.976)%
  --(4.153,1.975)--(4.154,1.974)--(4.154,1.973)--(4.155,1.972)--(4.155,1.971)--(4.156,1.970)%
  --(4.156,1.969)--(4.157,1.970)--(4.157,1.969)--(4.157,1.967)--(4.158,1.966)--(4.158,1.964)%
  --(4.159,1.963)--(4.160,1.962)--(4.160,1.961)--(4.161,1.961)--(4.162,1.961)--(4.162,1.962)%
  --(4.162,1.961)--(4.163,1.960)--(4.164,1.960)--(4.164,1.959)--(4.165,1.959)--(4.165,1.958)%
  --(4.166,1.957)--(4.166,1.956)--(4.167,1.957)--(4.167,1.956)--(4.168,1.956)--(4.168,1.955)%
  --(4.169,1.955)--(4.170,1.954)--(4.170,1.955)--(4.171,1.954)--(4.171,1.953)--(4.171,1.952)%
  --(4.172,1.953)--(4.172,1.952)--(4.173,1.952)--(4.173,1.953)--(4.174,1.952)--(4.174,1.951)%
  --(4.175,1.950)--(4.176,1.949)--(4.177,1.950)--(4.177,1.949)--(4.178,1.949)--(4.178,1.948)%
  --(4.179,1.948)--(4.179,1.947)--(4.179,1.948)--(4.180,1.947)--(4.181,1.946)--(4.181,1.945)%
  --(4.182,1.945)--(4.182,1.944)--(4.183,1.943)--(4.184,1.943)--(4.184,1.944)--(4.185,1.945)%
  --(4.185,1.944)--(4.186,1.943)--(4.186,1.942)--(4.187,1.942)--(4.187,1.941)--(4.188,1.942)%
  --(4.188,1.941)--(4.189,1.940)--(4.190,1.938)--(4.190,1.937)--(4.191,1.936)--(4.191,1.935)%
  --(4.191,1.936)--(4.192,1.935)--(4.192,1.936)--(4.193,1.935)--(4.193,1.933)--(4.193,1.932)%
  --(4.194,1.931)--(4.194,1.930)--(4.195,1.930)--(4.196,1.929)--(4.197,1.928)--(4.197,1.927)%
  --(4.197,1.926)--(4.198,1.926)--(4.199,1.926)--(4.199,1.925)--(4.200,1.924)--(4.200,1.925)%
  --(4.201,1.924)--(4.201,1.923)--(4.202,1.922)--(4.202,1.921)--(4.203,1.919)--(4.204,1.919)%
  --(4.204,1.917)--(4.205,1.918)--(4.205,1.917)--(4.205,1.916)--(4.206,1.916)--(4.207,1.915)%
  --(4.208,1.915)--(4.209,1.914)--(4.209,1.915)--(4.209,1.914)--(4.210,1.914)--(4.210,1.913)%
  --(4.211,1.913)--(4.211,1.914)--(4.211,1.913)--(4.212,1.913)--(4.212,1.912)--(4.212,1.911)%
  --(4.213,1.910)--(4.213,1.909)--(4.214,1.908)--(4.214,1.909)--(4.214,1.908)--(4.215,1.907)%
  --(4.216,1.906)--(4.216,1.905)--(4.216,1.906)--(4.217,1.905)--(4.218,1.905)--(4.218,1.904)%
  --(4.218,1.905)--(4.219,1.904)--(4.219,1.903)--(4.220,1.903)--(4.221,1.903)--(4.222,1.902)%
  --(4.223,1.901)--(4.223,1.900)--(4.224,1.899)--(4.224,1.898)--(4.224,1.900)--(4.225,1.899)%
  --(4.226,1.899)--(4.226,1.897)--(4.227,1.897)--(4.227,1.896)--(4.228,1.895)--(4.229,1.894)%
  --(4.229,1.893)--(4.230,1.892)--(4.230,1.891)--(4.230,1.890)--(4.231,1.889)--(4.231,1.888)%
  --(4.232,1.888)--(4.232,1.887)--(4.233,1.887)--(4.233,1.886)--(4.234,1.886)--(4.234,1.885)%
  --(4.235,1.884)--(4.235,1.883)--(4.235,1.882)--(4.236,1.883)--(4.236,1.884)--(4.237,1.882)%
  --(4.237,1.881)--(4.238,1.882)--(4.238,1.881)--(4.239,1.880)--(4.240,1.879)--(4.240,1.880)%
  --(4.241,1.879)--(4.241,1.878)--(4.242,1.878)--(4.242,1.876)--(4.243,1.876)--(4.243,1.875)%
  --(4.244,1.874)--(4.244,1.875)--(4.245,1.874)--(4.246,1.874)--(4.246,1.875)--(4.247,1.876)%
  --(4.247,1.875)--(4.247,1.876)--(4.248,1.875)--(4.248,1.874)--(4.249,1.872)--(4.249,1.871)%
  --(4.249,1.872)--(4.250,1.870)--(4.251,1.871)--(4.251,1.870)--(4.251,1.868)--(4.252,1.869)%
  --(4.252,1.868)--(4.253,1.868)--(4.253,1.867)--(4.254,1.867)--(4.254,1.865)--(4.255,1.865)%
  --(4.255,1.864)--(4.256,1.864)--(4.256,1.862)--(4.256,1.861)--(4.257,1.862)--(4.258,1.861)%
  --(4.258,1.862)--(4.258,1.861)--(4.259,1.860)--(4.259,1.861)--(4.259,1.860)--(4.260,1.858)%
  --(4.261,1.857)--(4.261,1.858)--(4.261,1.857)--(4.262,1.857)--(4.262,1.858)--(4.263,1.858)%
  --(4.264,1.856)--(4.264,1.857)--(4.265,1.856)--(4.265,1.855)--(4.266,1.855)--(4.266,1.854)%
  --(4.266,1.853)--(4.267,1.852)--(4.268,1.852)--(4.268,1.851)--(4.269,1.850)--(4.270,1.849)%
  --(4.270,1.848)--(4.270,1.847)--(4.271,1.847)--(4.271,1.846)--(4.272,1.846)--(4.272,1.847)%
  --(4.273,1.847)--(4.273,1.846)--(4.273,1.845)--(4.274,1.844)--(4.275,1.845)--(4.275,1.843)%
  --(4.276,1.844)--(4.277,1.843)--(4.277,1.842)--(4.278,1.841)--(4.278,1.839)--(4.279,1.839)%
  --(4.280,1.839)--(4.280,1.838)--(4.280,1.837)--(4.281,1.836)--(4.281,1.835)--(4.282,1.833)%
  --(4.282,1.831)--(4.283,1.830)--(4.283,1.831)--(4.284,1.831)--(4.285,1.831)--(4.285,1.830)%
  --(4.285,1.829)--(4.286,1.829)--(4.287,1.829)--(4.288,1.829)--(4.289,1.828)--(4.289,1.827)%
  --(4.289,1.828)--(4.290,1.827)--(4.290,1.826)--(4.291,1.825)--(4.291,1.824)--(4.291,1.825)%
  --(4.292,1.822)--(4.292,1.821)--(4.292,1.822)--(4.293,1.821)--(4.293,1.820)--(4.294,1.819)%
  --(4.294,1.818)--(4.295,1.817)--(4.295,1.816)--(4.296,1.816)--(4.296,1.815)--(4.297,1.814)%
  --(4.297,1.813)--(4.298,1.812)--(4.299,1.812)--(4.299,1.811)--(4.300,1.812)--(4.300,1.811)%
  --(4.301,1.811)--(4.301,1.812)--(4.302,1.811)--(4.302,1.812)--(4.303,1.811)--(4.303,1.810)%
  --(4.304,1.809)--(4.305,1.810)--(4.305,1.809)--(4.306,1.809)--(4.306,1.810)--(4.307,1.810)%
  --(4.308,1.811)--(4.308,1.810)--(4.309,1.811)--(4.309,1.810)--(4.310,1.810)--(4.310,1.811)%
  --(4.310,1.812)--(4.311,1.811)--(4.312,1.810)--(4.312,1.811)--(4.313,1.810)--(4.313,1.811)%
  --(4.314,1.810)--(4.314,1.809)--(4.315,1.808)--(4.315,1.807)--(4.315,1.806)--(4.316,1.806)%
  --(4.316,1.805)--(4.317,1.806)--(4.318,1.806)--(4.318,1.805)--(4.319,1.805)--(4.319,1.804)%
  --(4.320,1.805)--(4.320,1.804)--(4.321,1.804)--(4.322,1.804)--(4.323,1.804)--(4.323,1.803)%
  --(4.324,1.804)--(4.324,1.803)--(4.325,1.803)--(4.325,1.804)--(4.326,1.803)--(4.327,1.803)%
  --(4.327,1.802)--(4.327,1.801)--(4.328,1.800)--(4.328,1.799)--(4.329,1.798)--(4.329,1.797)%
  --(4.330,1.797)--(4.330,1.794)--(4.331,1.794)--(4.331,1.793)--(4.332,1.793)--(4.332,1.792)%
  --(4.332,1.791)--(4.333,1.789)--(4.333,1.788)--(4.334,1.787)--(4.335,1.788)--(4.336,1.787)%
  --(4.337,1.788)--(4.337,1.786)--(4.338,1.785)--(4.339,1.785)--(4.339,1.786)--(4.340,1.785)%
  --(4.340,1.784)--(4.341,1.784)--(4.342,1.783)--(4.342,1.782)--(4.343,1.781)--(4.344,1.780)%
  --(4.344,1.779)--(4.345,1.779)--(4.345,1.778)--(4.346,1.777)--(4.346,1.776)--(4.346,1.775)%
  --(4.347,1.776)--(4.348,1.775)--(4.348,1.776)--(4.348,1.775)--(4.349,1.773)--(4.349,1.772)%
  --(4.350,1.772)--(4.350,1.771)--(4.351,1.771)--(4.351,1.770)--(4.352,1.769)--(4.353,1.770)%
  --(4.353,1.769)--(4.353,1.768)--(4.354,1.768)--(4.354,1.769)--(4.355,1.769)--(4.355,1.768)%
  --(4.355,1.767)--(4.356,1.767)--(4.356,1.766)--(4.357,1.765)--(4.357,1.764)--(4.358,1.763)%
  --(4.359,1.762)--(4.360,1.761)--(4.361,1.761)--(4.361,1.760)--(4.362,1.759)--(4.362,1.758)%
  --(4.363,1.756)--(4.363,1.755)--(4.364,1.754)--(4.365,1.753)--(4.365,1.752)--(4.366,1.752)%
  --(4.367,1.752)--(4.367,1.753)--(4.368,1.753)--(4.369,1.752)--(4.369,1.751)--(4.370,1.751)%
  --(4.370,1.750)--(4.371,1.751)--(4.371,1.750)--(4.372,1.749)--(4.372,1.750)--(4.372,1.749)%
  --(4.373,1.748)--(4.373,1.747)--(4.374,1.747)--(4.374,1.745)--(4.375,1.745)--(4.375,1.744)%
  --(4.376,1.743)--(4.376,1.742)--(4.376,1.741)--(4.377,1.742)--(4.377,1.741)--(4.377,1.742)%
  --(4.378,1.742)--(4.378,1.741)--(4.379,1.741)--(4.380,1.741)--(4.380,1.740)--(4.381,1.740)%
  --(4.382,1.739)--(4.382,1.740)--(4.383,1.740)--(4.383,1.739)--(4.384,1.740)--(4.384,1.739)%
  --(4.384,1.738)--(4.385,1.738)--(4.385,1.737)--(4.386,1.736)--(4.386,1.734)--(4.386,1.735)%
  --(4.387,1.734)--(4.387,1.733)--(4.388,1.734)--(4.388,1.733)--(4.389,1.732)--(4.390,1.732)%
  --(4.391,1.731)--(4.391,1.730)--(4.392,1.730)--(4.393,1.730)--(4.393,1.729)--(4.393,1.727)%
  --(4.394,1.728)--(4.394,1.727)--(4.395,1.727)--(4.395,1.728)--(4.395,1.727)--(4.396,1.727)%
  --(4.396,1.726)--(4.397,1.725)--(4.397,1.724)--(4.398,1.723)--(4.398,1.722)--(4.398,1.721)%
  --(4.399,1.720)--(4.400,1.719)--(4.400,1.718)--(4.400,1.717)--(4.401,1.716)--(4.401,1.715)%
  --(4.402,1.712)--(4.402,1.711)--(4.403,1.712)--(4.403,1.711)--(4.404,1.711)--(4.404,1.710)%
  --(4.405,1.710)--(4.405,1.709)--(4.406,1.708)--(4.406,1.707)--(4.407,1.706)--(4.407,1.705)%
  --(4.407,1.706)--(4.408,1.705)--(4.408,1.704)--(4.409,1.704)--(4.409,1.703)--(4.410,1.704)%
  --(4.410,1.703)--(4.410,1.702)--(4.411,1.702)--(4.411,1.700)--(4.412,1.749)--(4.412,1.750)%
  --(4.412,1.752)--(4.413,1.754)--(4.413,1.756)--(4.414,1.758)--(4.414,1.759)--(4.414,1.763)%
  --(4.415,1.763)--(4.416,1.764)--(4.416,1.766)--(4.417,1.766)--(4.417,1.768)--(4.418,1.770)%
  --(4.418,1.772)--(4.419,1.772)--(4.419,1.773)--(4.419,1.775)--(4.420,1.777)--(4.421,1.778)%
  --(4.421,1.779)--(4.421,1.781)--(4.422,1.783)--(4.422,1.784)--(4.423,1.786)--(4.423,1.788)%
  --(4.423,1.789)--(4.424,1.791)--(4.424,1.793)--(4.424,1.795)--(4.425,1.798)--(4.425,1.799)%
  --(4.426,1.800)--(4.426,1.802)--(4.427,1.802)--(4.427,1.803)--(4.428,1.805)--(4.428,1.807)%
  --(4.429,1.807)--(4.429,1.808)--(4.430,1.808)--(4.430,1.809)--(4.430,1.811)--(4.431,1.813)%
  --(4.431,1.814)--(4.431,1.816)--(4.432,1.816)--(4.432,1.819)--(4.433,1.820)--(4.433,1.821)%
  --(4.433,1.823)--(4.434,1.824)--(4.434,1.826)--(4.435,1.826)--(4.435,1.828)--(4.435,1.829)%
  --(4.436,1.830)--(4.436,1.831)--(4.436,1.832)--(4.437,1.833)--(4.437,1.834)--(4.438,1.835)%
  --(4.438,1.837)--(4.439,1.839)--(4.440,1.841)--(4.440,1.843)--(4.441,1.846)--(4.441,1.848)%
  --(4.442,1.848)--(4.442,1.850)--(4.442,1.851)--(4.443,1.852)--(4.443,1.854)--(4.443,1.856)%
  --(4.444,1.858)--(4.444,1.860)--(4.445,1.861)--(4.445,1.863)--(4.445,1.865)--(4.446,1.867)%
  --(4.446,1.868)--(4.447,1.868)--(4.447,1.870)--(4.448,1.872)--(4.448,1.871)--(4.449,1.873)%
  --(4.449,1.875)--(4.450,1.878)--(4.450,1.879)--(4.451,1.881)--(4.451,1.882)--(4.452,1.883)%
  --(4.452,1.884)--(4.452,1.887)--(4.453,1.888)--(4.453,1.892)--(4.454,1.893)--(4.454,1.895)%
  --(4.454,1.896)--(4.455,1.897)--(4.455,1.898)--(4.456,1.899)--(4.456,1.901)--(4.457,1.901)%
  --(4.457,1.902)--(4.457,1.903)--(4.458,1.905)--(4.458,1.906)--(4.459,1.907)--(4.459,1.908)%
  --(4.459,1.910)--(4.460,1.911)--(4.460,1.913)--(4.461,1.913)--(4.462,1.917)--(4.462,1.918)%
  --(4.463,1.920)--(4.463,1.921)--(4.463,1.922)--(4.464,1.922)--(4.464,1.924)--(4.464,1.926)%
  --(4.465,1.927)--(4.465,1.929)--(4.466,1.930)--(4.466,1.932)--(4.466,1.933)--(4.467,1.935)%
  --(4.468,1.937)--(4.468,1.939)--(4.469,1.940)--(4.469,1.943)--(4.470,1.945)--(4.470,1.947)%
  --(4.471,1.949)--(4.471,1.948)--(4.471,1.950)--(4.472,1.952)--(4.472,1.954)--(4.473,1.956)%
  --(4.473,1.958)--(4.473,1.960)--(4.474,1.961)--(4.474,1.963)--(4.475,1.964)--(4.475,1.966)%
  --(4.475,1.967)--(4.476,1.969)--(4.476,1.971)--(4.476,1.973)--(4.477,1.975)--(4.477,1.976)%
  --(4.478,1.977)--(4.478,1.979)--(4.478,1.980)--(4.479,1.982)--(4.479,1.984)--(4.480,1.986)%
  --(4.480,1.988)--(4.480,1.987)--(4.481,1.988)--(4.481,1.989)--(4.482,1.990)--(4.482,1.992)%
  --(4.483,1.993)--(4.483,1.994)--(4.484,1.996)--(4.484,1.998)--(4.485,1.998)--(4.485,1.999)%
  --(4.486,2.000)--(4.487,2.002)--(4.487,2.004)--(4.487,2.006)--(4.488,2.008)--(4.488,2.010)%
  --(4.489,2.012)--(4.489,2.014)--(4.490,2.015)--(4.490,2.016)--(4.491,2.018)--(4.491,2.019)%
  --(4.492,2.021)--(4.492,2.023)--(4.492,2.026)--(4.493,2.028)--(4.494,2.030)--(4.494,2.031)%
  --(4.494,2.032)--(4.495,2.034)--(4.495,2.036)--(4.496,2.038)--(4.496,2.040)--(4.496,2.042)%
  --(4.497,2.043)--(4.497,2.045)--(4.498,2.047)--(4.498,2.048)--(4.499,2.050)--(4.499,2.051)%
  --(4.499,2.052)--(4.500,2.052)--(4.500,2.054)--(4.501,2.056)--(4.501,2.057)--(4.501,2.059)%
  --(4.502,2.060)--(4.502,2.062)--(4.503,2.063)--(4.503,2.065)--(4.503,2.068)--(4.504,2.069)%
  --(4.504,2.070)--(4.504,2.071)--(4.505,2.073)--(4.505,2.075)--(4.506,2.075)--(4.506,2.078)%
  --(4.506,2.079)--(4.507,2.081)--(4.508,2.081)--(4.508,2.083)--(4.509,2.085)--(4.509,2.088)%
  --(4.509,2.090)--(4.510,2.093)--(4.510,2.095)--(4.511,2.097)--(4.511,2.098)--(4.511,2.099)%
  --(4.512,2.101)--(4.512,2.104)--(4.513,2.104)--(4.513,2.105)--(4.513,2.108)--(4.514,2.110)%
  --(4.514,2.112)--(4.515,2.113)--(4.515,2.115)--(4.516,2.116)--(4.516,2.119)--(4.516,2.122)%
  --(4.517,2.124)--(4.517,2.125)--(4.518,2.126)--(4.518,2.127)--(4.519,2.129)--(4.519,2.130)%
  --(4.520,2.132)--(4.520,2.135)--(4.520,2.136)--(4.521,2.138)--(4.521,2.140)--(4.522,2.140)%
  --(4.522,2.141)--(4.522,2.143)--(4.523,2.144)--(4.523,2.146)--(4.523,2.149)--(4.524,2.149)%
  --(4.524,2.151)--(4.525,2.153)--(4.525,2.155)--(4.526,2.156)--(4.527,2.159)--(4.527,2.161)%
  --(4.527,2.163)--(4.528,2.164)--(4.528,2.167)--(4.529,2.169)--(4.529,2.170)--(4.529,2.171)%
  --(4.530,2.173)--(4.530,2.172)--(4.531,2.173)--(4.531,2.174)--(4.532,2.176)--(4.532,2.177)%
  --(4.532,2.178)--(4.533,2.180)--(4.534,2.181)--(4.534,2.182)--(4.535,2.184)--(4.535,2.187)%
  --(4.536,2.189)--(4.536,2.190)--(4.536,2.191)--(4.537,2.190)--(4.537,2.191)--(4.538,2.192)%
  --(4.538,2.193)--(4.539,2.194)--(4.539,2.197)--(4.539,2.198)--(4.540,2.198)--(4.540,2.213)%
  --(4.541,2.218)--(4.541,2.219)--(4.541,2.221)--(4.542,2.223)--(4.542,2.226)--(4.543,2.230)%
  --(4.543,2.232)--(4.544,2.234)--(4.544,2.238)--(4.544,2.240)--(4.545,2.243)--(4.545,2.247)%
  --(4.546,2.250)--(4.546,2.254)--(4.546,2.258)--(4.547,2.261)--(4.547,2.264)--(4.548,2.265)%
  --(4.548,2.267)--(4.549,2.267)--(4.549,2.268)--(4.549,2.270)--(4.550,2.270)--(4.550,2.272)%
  --(4.551,2.274)--(4.551,2.276)--(4.551,2.277)--(4.552,2.278)--(4.552,2.281)--(4.553,2.281)%
  --(4.553,2.285)--(4.553,2.286)--(4.554,2.288)--(4.554,2.291)--(4.555,2.291)--(4.555,2.292)%
  --(4.555,2.293)--(4.556,2.295)--(4.556,2.296)--(4.556,2.297)--(4.557,2.297)--(4.557,2.298)%
  --(4.558,2.298)--(4.558,2.302)--(4.558,2.311)--(4.559,2.317)--(4.559,2.318)--(4.560,2.318)%
  --(4.560,2.319)--(4.560,2.320)--(4.561,2.322)--(4.561,2.324)--(4.562,2.329)--(4.562,2.330)%
  --(4.562,2.332)--(4.563,2.335)--(4.563,2.339)--(4.564,2.341)--(4.564,2.342)--(4.565,2.342)%
  --(4.565,2.344)--(4.565,2.345)--(4.566,2.345)--(4.566,2.347)--(4.567,2.347)--(4.567,2.351)%
  --(4.567,2.353)--(4.568,2.353)--(4.568,2.355)--(4.569,2.356)--(4.569,2.359)--(4.570,2.361)%
  --(4.570,2.364)--(4.570,2.367)--(4.571,2.370)--(4.571,2.372)--(4.572,2.376)--(4.572,2.381)%
  --(4.572,2.386)--(4.573,2.390)--(4.573,2.391)--(4.574,2.394)--(4.574,2.396)--(4.574,2.400)%
  --(4.575,2.402)--(4.575,2.404)--(4.576,2.406)--(4.576,2.408)--(4.576,2.410)--(4.577,2.411)%
  --(4.577,2.413)--(4.578,2.415)--(4.578,2.419)--(4.579,2.420)--(4.579,2.422)--(4.579,2.423)%
  --(4.580,2.424)--(4.580,2.427)--(4.581,2.428)--(4.581,2.430)--(4.581,2.432)--(4.582,2.434)%
  --(4.582,2.435)--(4.582,2.436)--(4.583,2.438)--(4.583,2.440)--(4.584,2.441)--(4.584,2.445)%
  --(4.584,2.446)--(4.585,2.449)--(4.585,2.452)--(4.586,2.454)--(4.586,2.457)--(4.586,2.459)%
  --(4.587,2.460)--(4.587,2.462)--(4.588,2.469)--(4.588,2.472)--(4.588,2.475)--(4.589,2.479)%
  --(4.589,2.481)--(4.589,2.484)--(4.590,2.486)--(4.590,2.490)--(4.591,2.494)--(4.591,2.497)%
  --(4.591,2.501)--(4.592,2.504)--(4.593,2.507)--(4.593,2.510)--(4.593,2.511)--(4.594,2.514)%
  --(4.594,2.516)--(4.595,2.516)--(4.595,2.518)--(4.595,2.521)--(4.596,2.522)--(4.596,2.523)%
  --(4.596,2.525)--(4.597,2.527)--(4.597,2.529)--(4.598,2.532)--(4.598,2.536)--(4.598,2.539)%
  --(4.599,2.541)--(4.599,2.542)--(4.600,2.544)--(4.600,2.547)--(4.601,2.549)--(4.601,2.553)%
  --(4.602,2.555)--(4.602,2.557)--(4.602,2.560)--(4.603,2.562)--(4.603,2.564)--(4.603,2.567)%
  --(4.604,2.569)--(4.604,2.571)--(4.605,2.573)--(4.605,2.575)--(4.605,2.577)--(4.606,2.580)%
  --(4.606,2.583)--(4.607,2.584)--(4.607,2.587)--(4.607,2.591)--(4.608,2.593)--(4.608,2.597)%
  --(4.609,2.602)--(4.609,2.604)--(4.609,2.606)--(4.610,2.608)--(4.610,2.611)--(4.610,2.613)%
  --(4.611,2.616)--(4.611,2.617)--(4.612,2.619)--(4.612,2.623)--(4.612,2.625)--(4.613,2.626)%
  --(4.613,2.628)--(4.614,2.633)--(4.614,2.635)--(4.614,2.636)--(4.615,2.639)--(4.615,2.641)%
  --(4.615,2.643)--(4.616,2.647)--(4.616,2.649)--(4.617,2.653)--(4.617,2.656)--(4.617,2.658)%
  --(4.618,2.660)--(4.619,2.662)--(4.619,2.664)--(4.619,2.666)--(4.620,2.666)--(4.620,2.669)%
  --(4.621,2.671)--(4.621,2.675)--(4.621,2.679)--(4.622,2.681)--(4.622,2.685)--(4.622,2.688)%
  --(4.623,2.690)--(4.623,2.692)--(4.624,2.696)--(4.624,2.698)--(4.624,2.701)--(4.625,2.701)%
  --(4.625,2.703)--(4.626,2.704)--(4.626,2.707)--(4.626,2.709)--(4.627,2.713)--(4.627,2.716)%
  --(4.628,2.717)--(4.628,2.720)--(4.628,2.722)--(4.629,2.725)--(4.629,2.726)--(4.629,2.730)%
  --(4.630,2.732)--(4.631,2.734)--(4.631,2.737)--(4.631,2.740)--(4.632,2.743)--(4.632,2.746)%
  --(4.633,2.748)--(4.633,2.751)--(4.633,2.754)--(4.634,2.756)--(4.634,2.759)--(4.635,2.759)%
  --(4.635,2.761)--(4.635,2.764)--(4.636,2.767)--(4.636,2.770)--(4.636,2.773)--(4.637,2.776)%
  --(4.637,2.781)--(4.638,2.783)--(4.638,2.785)--(4.638,2.789)--(4.639,2.791)--(4.639,2.793)%
  --(4.640,2.797)--(4.640,2.799)--(4.640,2.802)--(4.641,2.803)--(4.641,2.804)--(4.642,2.804)%
  --(4.642,2.807)--(4.642,2.810)--(4.643,2.813)--(4.643,2.817)--(4.643,2.821)--(4.644,2.822)%
  --(4.644,2.825)--(4.645,2.829)--(4.645,2.832)--(4.645,2.835)--(4.646,2.839)--(4.646,2.847)%
  --(4.647,2.851)--(4.647,2.856)--(4.647,2.859)--(4.648,2.860)--(4.648,2.862)--(4.649,2.864)%
  --(4.649,2.866)--(4.649,2.868)--(4.650,2.874)--(4.650,2.878)--(4.650,2.884)--(4.651,2.886)%
  --(4.651,2.891)--(4.652,2.895)--(4.652,2.897)--(4.652,2.900)--(4.653,2.903)--(4.653,2.907)%
  --(4.654,2.911)--(4.654,2.915)--(4.654,2.920)--(4.655,2.923)--(4.655,2.926)--(4.655,2.927)%
  --(4.656,2.927)--(4.656,2.930)--(4.657,2.932)--(4.657,2.935)--(4.657,2.940)--(4.658,2.943)%
  --(4.658,2.948)--(4.659,2.950)--(4.659,2.953)--(4.659,2.957)--(4.660,2.959)--(4.660,2.963)%
  --(4.661,2.966)--(4.661,2.969)--(4.661,2.972)--(4.662,2.975)--(4.662,2.977)--(4.662,2.982)%
  --(4.663,2.984)--(4.663,2.986)--(4.664,2.989)--(4.664,2.992)--(4.664,2.998)--(4.665,3.003)%
  --(4.665,3.008)--(4.666,3.013)--(4.666,3.015)--(4.666,3.019)--(4.667,3.021)--(4.667,3.020)%
  --(4.668,3.021)--(4.668,3.024)--(4.669,3.027)--(4.669,3.028)--(4.669,3.031)--(4.670,3.033)%
  --(4.670,3.034)--(4.671,3.037)--(4.671,3.039)--(4.671,3.042)--(4.672,3.043)--(4.672,3.045)%
  --(4.673,3.048)--(4.673,3.052)--(4.673,3.053)--(4.674,3.056)--(4.674,3.061)--(4.675,3.064)%
  --(4.675,3.069)--(4.675,3.073)--(4.676,3.075)--(4.676,3.078)--(4.676,3.082)--(4.677,3.086)%
  --(4.677,3.089)--(4.678,3.092)--(4.678,3.097)--(4.678,3.101)--(4.679,3.105)--(4.679,3.110)%
  --(4.680,3.113)--(4.680,3.116)--(4.680,3.120)--(4.681,3.123)--(4.681,3.126)--(4.682,3.130)%
  --(4.682,3.132)--(4.682,3.135)--(4.683,3.138)--(4.683,3.141)--(4.683,3.145)--(4.684,3.150)%
  --(4.684,3.154)--(4.685,3.156)--(4.685,3.160)--(4.685,3.162)--(4.686,3.162)--(4.686,3.166)%
  --(4.687,3.167)--(4.687,3.169)--(4.687,3.170)--(4.688,3.170)--(4.688,3.177)--(4.688,3.180)%
  --(4.689,3.184)--(4.689,3.187)--(4.690,3.193)--(4.690,3.196)--(4.690,3.200)--(4.691,3.205)%
  --(4.691,3.207)--(4.692,3.208)--(4.692,3.214)--(4.692,3.217)--(4.693,3.220)--(4.693,3.223)%
  --(4.694,3.228)--(4.694,3.231)--(4.694,3.237)--(4.695,3.238)--(4.695,3.242)--(4.695,3.247)%
  --(4.696,3.250)--(4.696,3.253)--(4.697,3.257)--(4.697,3.259)--(4.697,3.265)--(4.698,3.269)%
  --(4.698,3.272)--(4.699,3.274)--(4.699,3.276)--(4.699,3.279)--(4.700,3.280)--(4.700,3.284)%
  --(4.701,3.286)--(4.701,3.289)--(4.701,3.292)--(4.702,3.297)--(4.702,3.302)--(4.702,3.305)%
  --(4.703,3.308)--(4.703,3.314)--(4.704,3.317)--(4.704,3.319)--(4.705,3.319)--(4.705,3.322)%
  --(4.706,3.322)--(4.706,3.324)--(4.706,3.326)--(4.707,3.327)--(4.707,3.328)--(4.708,3.335)%
  --(4.708,3.340)--(4.708,3.343)--(4.709,3.349)--(4.709,3.354)--(4.709,3.362)--(4.710,3.371)%
  --(4.710,3.379)--(4.711,3.385)--(4.711,3.387)--(4.711,3.390)--(4.712,3.398)--(4.712,3.402)%
  --(4.713,3.406)--(4.713,3.409)--(4.713,3.416)--(4.714,3.420)--(4.714,3.425)--(4.715,3.430)%
  --(4.715,3.431)--(4.715,3.435)--(4.716,3.437)--(4.716,3.443)--(4.716,3.450)--(4.717,3.455)%
  --(4.717,3.459)--(4.718,3.460)--(4.718,3.464)--(4.718,3.467)--(4.719,3.469)--(4.719,3.475)%
  --(4.720,3.480)--(4.720,3.488)--(4.720,3.493)--(4.721,3.499)--(4.721,3.502)--(4.721,3.504)%
  --(4.722,3.508)--(4.722,3.516)--(4.723,3.519)--(4.723,3.520)--(4.723,3.524)--(4.724,3.529)%
  --(4.724,3.536)--(4.725,3.539)--(4.725,3.544)--(4.725,3.551)--(4.726,3.553)--(4.726,3.559)%
  --(4.727,3.563)--(4.727,3.566)--(4.727,3.567)--(4.728,3.570)--(4.728,3.575)--(4.728,3.577)%
  --(4.729,3.579)--(4.730,3.582)--(4.730,3.587)--(4.730,3.592)--(4.731,3.596)--(4.731,3.597)%
  --(4.732,3.600)--(4.732,3.607)--(4.732,3.610)--(4.733,3.613)--(4.733,3.618)--(4.734,3.623)%
  --(4.734,3.631)--(4.734,3.636)--(4.735,3.641)--(4.735,3.649)--(4.735,3.653)--(4.736,3.655)%
  --(4.736,3.659)--(4.737,3.664)--(4.737,3.668)--(4.737,3.672)--(4.738,3.676)--(4.738,3.680)%
  --(4.739,3.682)--(4.739,3.685)--(4.739,3.691)--(4.740,3.693)--(4.740,3.694)--(4.741,3.695)%
  --(4.741,3.698)--(4.741,3.699)--(4.742,3.699)--(4.742,3.703)--(4.742,3.706)--(4.743,3.707)%
  --(4.743,3.715)--(4.744,3.719)--(4.744,3.726)--(4.744,3.732)--(4.745,3.735)--(4.745,3.737)%
  --(4.746,3.739)--(4.746,3.741)--(4.746,3.745)--(4.747,3.749)--(4.747,3.754)--(4.748,3.759)%
  --(4.748,3.762)--(4.748,3.766)--(4.749,3.770)--(4.749,3.773)--(4.749,3.779)--(4.750,3.787)%
  --(4.750,3.791)--(4.751,3.796)--(4.751,3.803)--(4.751,3.809)--(4.752,3.813)--(4.752,3.819)%
  --(4.753,3.824)--(4.753,3.828)--(4.753,3.837)--(4.754,3.846)--(4.754,3.853)--(4.755,3.858)%
  --(4.755,3.865)--(4.755,3.871)--(4.756,3.874)--(4.756,3.878)--(4.756,3.880)--(4.757,3.883)%
  --(4.757,3.888)--(4.758,3.892)--(4.758,3.894)--(4.758,3.898)--(4.759,3.902)--(4.759,3.906)%
  --(4.760,3.911)--(4.760,3.914)--(4.760,3.917)--(4.761,3.920)--(4.761,3.926)--(4.762,3.932)%
  --(4.762,3.935)--(4.763,3.940)--(4.763,3.944)--(4.764,3.949)--(4.764,3.954)--(4.765,3.957)%
  --(4.765,3.961)--(4.765,3.964)--(4.766,3.967)--(4.766,3.969)--(4.767,3.972)--(4.767,3.974)%
  --(4.767,3.976)--(4.768,3.980)--(4.768,3.984)--(4.768,3.991)--(4.769,3.997)--(4.769,4.003)%
  --(4.770,4.009)--(4.770,4.015)--(4.770,4.022)--(4.771,4.025)--(4.771,4.030)--(4.772,4.034)%
  --(4.772,4.037)--(4.772,4.040)--(4.773,4.047)--(4.773,4.048)--(4.774,4.050)--(4.774,4.055)%
  --(4.774,4.057)--(4.775,4.064)--(4.775,4.070)--(4.775,4.074)--(4.776,4.078)--(4.776,4.086)%
  --(4.777,4.090)--(4.777,4.094)--(4.777,4.099)--(4.778,4.102)--(4.778,4.103)--(4.779,4.106)%
  --(4.779,4.108)--(4.779,4.113)--(4.780,4.114)--(4.780,4.123)--(4.781,4.129)--(4.781,4.131)%
  --(4.781,4.136)--(4.782,4.139)--(4.782,4.144)--(4.782,4.145)--(4.783,4.149)--(4.783,4.151)%
  --(4.784,4.155)--(4.784,4.160)--(4.784,4.163)--(4.785,4.164)--(4.785,4.166)--(4.786,4.170)%
  --(4.786,4.173)--(4.786,4.177)--(4.787,4.187)--(4.787,4.193)--(4.788,4.202)--(4.788,4.204)%
  --(4.788,4.208)--(4.789,4.209)--(4.789,4.212)--(4.789,4.216)--(4.790,4.221)--(4.790,4.227)%
  --(4.791,4.229)--(4.791,4.233)--(4.791,4.238)--(4.792,4.247)--(4.792,4.253)--(4.793,4.261)%
  --(4.793,4.267)--(4.793,4.271)--(4.794,4.273)--(4.794,4.277)--(4.794,4.286)--(4.795,4.289)%
  --(4.795,4.292)--(4.796,4.296)--(4.796,4.305)--(4.796,4.309)--(4.797,4.309)--(4.797,4.312)%
  --(4.798,4.321)--(4.798,4.325)--(4.798,4.332)--(4.799,4.338)--(4.799,4.344)--(4.800,4.349)%
  --(4.800,4.353)--(4.800,4.360)--(4.801,4.367)--(4.801,4.376)--(4.801,4.384)--(4.802,4.396)%
  --(4.802,4.413)--(4.803,4.419)--(4.803,4.431)--(4.803,4.435)--(4.804,4.440)--(4.804,4.443)%
  --(4.805,4.449)--(4.805,4.455)--(4.805,4.466)--(4.806,4.470)--(4.806,4.477)--(4.807,4.488)%
  --(4.807,4.489)--(4.807,4.496)--(4.808,4.506)--(4.808,4.511)--(4.808,4.516)--(4.809,4.520)%
  --(4.809,4.524)--(4.810,4.531)--(4.810,4.534)--(4.810,4.539)--(4.811,4.546)--(4.811,4.550)%
  --(4.812,4.557)--(4.812,4.563)--(4.812,4.566)--(4.813,4.573)--(4.813,4.578)--(4.814,4.588)%
  --(4.814,4.593)--(4.814,4.598)--(4.815,4.602)--(4.815,4.611)--(4.815,4.619)--(4.816,4.622)%
  --(4.816,4.627)--(4.817,4.633)--(4.817,4.639)--(4.817,4.641)--(4.818,4.650)--(4.818,4.656)%
  --(4.819,4.669)--(4.819,4.675)--(4.819,4.684)--(4.820,4.693)--(4.820,4.701)--(4.821,4.708)%
  --(4.821,4.718)--(4.821,4.723)--(4.822,4.728)--(4.822,4.732)--(4.822,4.737)--(4.823,4.745)%
  --(4.823,4.753)--(4.824,4.762)--(4.824,4.767)--(4.824,4.776)--(4.825,4.778)--(4.825,4.786)%
  --(4.826,4.787)--(4.826,4.792)--(4.826,4.801)--(4.827,4.807)--(4.827,4.812)--(4.827,4.814)%
  --(4.828,4.825)--(4.828,4.827)--(4.829,4.830)--(4.829,4.835)--(4.829,4.840)--(4.830,4.844)%
  --(4.830,4.851)--(4.831,4.855)--(4.831,4.863)--(4.831,4.868)--(4.832,4.874)--(4.832,4.879)%
  --(4.833,4.887)--(4.833,4.894)--(4.833,4.899)--(4.834,4.905)--(4.834,4.906)--(4.834,4.916)%
  --(4.835,4.927)--(4.835,4.935)--(4.836,4.944)--(4.836,4.956)--(4.836,4.962)--(4.837,4.965)%
  --(4.837,4.968)--(4.838,4.977)--(4.838,4.981)--(4.838,4.982)--(4.839,4.990)--(4.839,4.996)%
  --(4.840,5.003)--(4.840,5.011)--(4.840,5.013)--(4.841,5.020)--(4.841,5.024)--(4.841,5.029)%
  --(4.842,5.035)--(4.842,5.039)--(4.843,5.047)--(4.843,5.057)--(4.843,5.065)--(4.844,5.070)%
  --(4.844,5.073)--(4.845,5.076)--(4.845,5.080)--(4.845,5.088)--(4.846,5.098)--(4.846,5.105)%
  --(4.847,5.117)--(4.847,5.124)--(4.847,5.132)--(4.848,5.141)--(4.848,5.150)--(4.848,5.157)%
  --(4.849,5.167)--(4.849,5.173)--(4.850,5.180)--(4.850,5.186)--(4.850,5.191)--(4.851,5.193)%
  --(4.851,5.202)--(4.852,5.211)--(4.852,5.219)--(4.852,5.221)--(4.853,5.227)--(4.853,5.239)%
  --(4.854,5.246)--(4.854,5.263)--(4.854,5.272)--(4.855,5.276)--(4.855,5.280)--(4.855,5.282)%
  --(4.856,5.288)--(4.856,5.295)--(4.857,5.307)--(4.857,5.310)--(4.857,5.315)--(4.858,5.318)%
  --(4.858,5.323)--(4.859,5.332)--(4.859,5.339)--(4.859,5.346)--(4.860,5.349)--(4.860,5.354)%
  --(4.861,5.366)--(4.861,5.368)--(4.861,5.371)--(4.862,5.373)--(4.862,5.382)--(4.862,5.391)%
  --(4.863,5.399)--(4.863,5.400)--(4.864,5.409)--(4.864,5.414)--(4.864,5.418)--(4.865,5.423)%
  --(4.865,5.431)--(4.866,5.436)--(4.866,5.440)--(4.866,5.450)--(4.867,5.458)--(4.867,5.472)%
  --(4.867,5.475)--(4.868,5.484)--(4.868,5.483)--(4.869,5.492)--(4.869,5.498)--(4.869,5.508)%
  --(4.870,5.516)--(4.870,5.524)--(4.871,5.533)--(4.871,5.540)--(4.871,5.544)--(4.872,5.548)%
  --(4.872,5.556)--(4.873,5.565)--(4.873,5.572)--(4.873,5.577)--(4.874,5.585)--(4.874,5.590)%
  --(4.874,5.596)--(4.875,5.596)--(4.875,5.599)--(4.876,5.609)--(4.876,5.619)--(4.876,5.631)%
  --(4.877,5.640)--(4.877,5.651)--(4.878,5.660)--(4.878,5.666)--(4.878,5.671)--(4.879,5.682)%
  --(4.879,5.694)--(4.880,5.700)--(4.880,5.709)--(4.880,5.712)--(4.881,5.717)--(4.881,5.723)%
  --(4.881,5.727)--(4.882,5.736)--(4.882,5.741)--(4.883,5.753)--(4.883,5.762)--(4.883,5.772)%
  --(4.884,5.786)--(4.884,5.801)--(4.885,5.802)--(4.885,5.816)--(4.885,5.822)--(4.886,5.833)%
  --(4.886,5.837)--(4.887,5.845)--(4.887,5.849)--(4.887,5.858)--(4.888,5.869)--(4.888,5.873)%
  --(4.888,5.880)--(4.889,5.877)--(4.889,5.882)--(4.890,5.894)--(4.890,5.896)--(4.890,5.902)%
  --(4.891,5.912)--(4.891,5.918)--(4.892,5.926)--(4.892,5.933)--(4.892,5.946)--(4.893,5.958)%
  --(4.893,5.966)--(4.894,5.976)--(4.894,5.982)--(4.894,5.991)--(4.895,6.007)--(4.895,6.029)%
  --(4.895,6.038)--(4.896,6.049)--(4.896,6.065)--(4.897,6.073)--(4.897,6.085)--(4.897,6.090)%
  --(4.898,6.094)--(4.898,6.102)--(4.899,6.109)--(4.899,6.116)--(4.899,6.120)--(4.900,6.128)%
  --(4.900,6.131)--(4.900,6.139)--(4.901,6.141)--(4.901,6.150)--(4.902,6.159)--(4.902,6.162)%
  --(4.902,6.166)--(4.903,6.172)--(4.903,6.179)--(4.904,6.184)--(4.904,6.191)--(4.904,6.200)%
  --(4.905,6.210)--(4.905,6.224)--(4.906,6.228)--(4.906,6.233)--(4.906,6.244)--(4.907,6.251)%
  --(4.907,6.266)--(4.907,6.275)--(4.908,6.284)--(4.908,6.291)--(4.909,6.295)--(4.909,6.304)%
  --(4.909,6.310)--(4.910,6.320)--(4.910,6.331)--(4.911,6.343)--(4.911,6.347)--(4.911,6.358)%
  --(4.912,6.363)--(4.912,6.368)--(4.913,6.381)--(4.913,6.393)--(4.913,6.407)--(4.914,6.417)%
  --(4.914,6.425)--(4.914,6.429)--(4.915,6.431)--(4.915,6.440)--(4.916,6.447)--(4.916,6.456)%
  --(4.916,6.461)--(4.917,6.473)--(4.917,6.483)--(4.918,6.491)--(4.918,6.502)--(4.918,6.520)%
  --(4.919,6.529)--(4.919,6.542)--(4.920,6.560)--(4.920,6.575)--(4.920,6.583)--(4.921,6.603)%
  --(4.921,6.610)--(4.921,6.620)--(4.922,6.627)--(4.922,6.647)--(4.923,6.665)--(4.923,6.676)%
  --(4.923,6.685)--(4.924,6.694)--(4.924,6.700)--(4.925,6.709)--(4.925,6.713)--(4.925,6.725)%
  --(4.926,6.735)--(4.926,6.749)--(4.927,6.758)--(4.927,6.770)--(4.927,6.784)--(4.928,6.796)%
  --(4.928,6.814)--(4.928,6.831)--(4.929,6.848)--(4.929,6.855)--(4.930,6.861)--(4.930,6.872)%
  --(4.930,6.884)--(4.931,6.900)--(4.931,6.917)--(4.932,6.932)--(4.932,6.941)--(4.932,6.384)%
  --(4.933,6.377)--(4.933,6.363)--(4.933,6.351)--(4.934,6.341)--(4.934,6.323)--(4.935,6.306)%
  --(4.935,6.292)--(4.935,6.285)--(4.936,6.275)--(4.936,6.262)--(4.937,6.247)--(4.937,6.236)%
  --(4.938,6.235)--(4.938,6.229)--(4.939,6.224)--(4.939,6.213)--(4.939,6.202)--(4.940,6.194)%
  --(4.940,6.184)--(4.940,6.170)--(4.941,6.155)--(4.941,6.146)--(4.942,6.133)--(4.942,6.126)%
  --(4.942,6.122)--(4.943,6.112)--(4.943,6.105)--(4.944,6.099)--(4.944,6.087)--(4.944,6.078)%
  --(4.945,6.067)--(4.945,6.058)--(4.946,6.044)--(4.946,6.028)--(4.946,6.027)--(4.947,6.023)%
  --(4.947,6.019)--(4.947,6.015)--(4.948,6.001)--(4.948,5.994)--(4.949,5.995)--(4.949,5.989)%
  --(4.950,5.983)--(4.950,5.968)--(4.951,5.958)--(4.951,5.945)--(4.951,5.942)--(4.952,5.936)%
  --(4.952,5.923)--(4.953,5.919)--(4.953,5.912)--(4.953,5.906)--(4.954,5.902)--(4.954,5.891)%
  --(4.954,5.890)--(4.955,5.877)--(4.955,5.874)--(4.956,5.874)--(4.956,5.872)--(4.956,5.861)%
  --(4.957,5.845)--(4.957,5.841)--(4.958,5.834)--(4.958,5.837)--(4.958,5.829)--(4.959,5.823)%
  --(4.959,5.811)--(4.960,5.799)--(4.960,5.790)--(4.960,5.782)--(4.961,5.780)--(4.961,5.773)%
  --(4.962,5.770)--(4.962,5.760)--(4.963,5.756)--(4.963,5.748)--(4.963,5.742)--(4.964,5.737)%
  --(4.964,5.734)--(4.965,5.723)--(4.965,5.713)--(4.965,5.702)--(4.966,5.692)--(4.966,5.680)%
  --(4.967,5.678)--(4.967,5.674)--(4.967,5.664)--(4.968,5.655)--(4.968,5.649)--(4.968,5.647)%
  --(4.969,5.645)--(4.969,5.647)--(4.970,5.639)--(4.970,5.628)--(4.971,5.621)--(4.971,5.613)%
  --(4.972,5.611)--(4.972,5.604)--(4.972,5.591)--(4.973,5.583)--(4.973,5.577)--(4.973,5.570)%
  --(4.974,5.568)--(4.974,5.562)--(4.975,5.559)--(4.975,5.555)--(4.975,5.549)--(4.976,5.542)%
  --(4.976,5.528)--(4.977,5.518)--(4.977,5.510)--(4.977,5.504)--(4.978,5.500)--(4.978,5.493)%
  --(4.979,5.490)--(4.979,5.483)--(4.980,5.478)--(4.980,5.467)--(4.980,5.458)--(4.981,5.451)%
  --(4.981,5.448)--(4.982,5.444)--(4.982,5.437)--(4.982,5.433)--(4.983,5.428)--(4.983,5.421)%
  --(4.984,5.416)--(4.984,5.412)--(4.984,5.405)--(4.985,5.399)--(4.985,5.390)--(4.986,5.381)%
  --(4.986,5.376)--(4.987,5.371)--(4.987,5.366)--(4.987,5.361)--(4.988,5.355)--(4.988,5.340)%
  --(4.989,5.332)--(4.989,5.326)--(4.989,5.327)--(4.990,5.319)--(4.990,5.316)--(4.991,5.311)%
  --(4.991,5.307)--(4.991,5.299)--(4.992,5.290)--(4.992,5.288)--(4.993,5.281)--(4.993,5.271)%
  --(4.993,5.262)--(4.994,5.259)--(4.994,5.254)--(4.994,5.249)--(4.995,5.249)--(4.995,5.241)%
  --(4.996,5.244)--(4.996,5.237)--(4.996,5.232)--(4.997,5.226)--(4.997,5.224)--(4.998,5.219)%
  --(4.998,5.217)--(4.998,5.215)--(4.999,5.207)--(4.999,5.199)--(5.000,5.194)--(5.000,5.184)%
  --(5.000,5.181)--(5.001,5.174)--(5.001,5.166)--(5.001,5.161)--(5.002,5.155)--(5.002,5.151)%
  --(5.003,5.148)--(5.003,5.139)--(5.003,5.133)--(5.004,5.130)--(5.004,5.127)--(5.005,5.125)%
  --(5.005,5.122)--(5.005,5.120)--(5.006,5.120)--(5.006,5.116)--(5.006,5.110)--(5.007,5.100)%
  --(5.007,5.093)--(5.008,5.084)--(5.008,5.078)--(5.008,5.071)--(5.009,5.064)--(5.009,5.057)%
  --(5.010,5.045)--(5.010,5.039)--(5.010,5.035)--(5.011,5.033)--(5.012,5.026)--(5.012,5.027)%
  --(5.012,5.025)--(5.013,5.024)--(5.013,5.021)--(5.013,5.022)--(5.014,5.022)--(5.014,5.019)%
  --(5.015,5.019)--(5.015,5.021)--(5.015,5.017)--(5.016,5.013)--(5.016,5.009)--(5.017,5.004)%
  --(5.017,4.998)--(5.018,4.988)--(5.018,4.982)--(5.019,4.968)--(5.019,4.960)--(5.019,4.952)%
  --(5.020,4.946)--(5.020,4.942)--(5.020,4.936)--(5.021,4.932)--(5.021,4.928)--(5.022,4.922)%
  --(5.022,4.915)--(5.022,4.912)--(5.023,4.905)--(5.023,4.898)--(5.024,4.895)--(5.024,4.883)%
  --(5.024,4.870)--(5.025,4.859)--(5.025,4.853)--(5.026,4.851)--(5.026,4.847)--(5.026,4.843)%
  --(5.027,4.837)--(5.027,4.824)--(5.027,4.817)--(5.028,4.811)--(5.028,4.802)--(5.029,4.790)%
  --(5.029,4.780)--(5.029,4.773)--(5.030,4.769)--(5.030,4.763)--(5.031,4.760)--(5.031,4.754)%
  --(5.031,4.747)--(5.032,4.735)--(5.032,4.728)--(5.033,4.717)--(5.033,4.705)--(5.033,4.694)%
  --(5.034,4.691)--(5.034,4.687)--(5.034,4.682)--(5.035,4.677)--(5.035,4.671)--(5.036,4.666)%
  --(5.036,4.658)--(5.036,4.652)--(5.037,4.644)--(5.037,4.633)--(5.038,4.630)--(5.038,4.631)%
  --(5.038,4.623)--(5.039,4.618)--(5.039,4.613)--(5.039,4.611)--(5.040,4.604)--(5.040,4.590)%
  --(5.041,4.582)--(5.041,4.579)--(5.041,4.570)--(5.042,4.570)--(5.042,4.565)--(5.043,4.558)%
  --(5.043,4.549)--(5.043,4.544)--(5.044,4.537)--(5.044,4.530)--(5.045,4.523)--(5.045,4.517)%
  --(5.045,4.510)--(5.046,4.505)--(5.046,4.502)--(5.047,4.497)--(5.047,4.491)--(5.048,4.488)%
  --(5.048,4.482)--(5.048,4.475)--(5.049,4.469)--(5.049,4.466)--(5.050,4.464)--(5.050,4.461)%
  --(5.051,4.454)--(5.051,4.450)--(5.052,4.444)--(5.052,4.441)--(5.052,4.437)--(5.053,4.429)%
  --(5.053,4.424)--(5.053,4.418)--(5.054,4.415)--(5.054,4.411)--(5.055,4.405)--(5.055,4.404)%
  --(5.055,4.402)--(5.056,4.401)--(5.056,4.398)--(5.057,4.395)--(5.057,4.393)--(5.057,4.390)%
  --(5.058,4.389)--(5.058,4.384)--(5.059,4.361)--(5.059,4.351)--(5.059,4.347)--(5.060,4.341)%
  --(5.060,4.337)--(5.060,4.335)--(5.061,4.332)--(5.061,4.325)--(5.062,4.324)--(5.062,4.321)%
  --(5.062,4.317)--(5.063,4.313)--(5.063,4.308)--(5.064,4.303)--(5.064,4.300)--(5.064,4.296)%
  --(5.065,4.289)--(5.065,4.281)--(5.066,4.278)--(5.066,4.275)--(5.066,4.268)--(5.067,4.262)%
  --(5.067,4.258)--(5.067,4.255)--(5.068,4.249)--(5.068,4.244)--(5.069,4.232)--(5.069,4.233)%
  --(5.069,4.224)--(5.070,4.226)--(5.070,4.221)--(5.071,4.221)--(5.071,4.217)--(5.072,4.214)%
  --(5.072,4.210)--(5.073,4.210)--(5.073,4.200)--(5.073,4.190)--(5.074,4.188)--(5.074,4.186)%
  --(5.074,4.185)--(5.075,4.184)--(5.075,4.180)--(5.076,4.179)--(5.076,4.174)--(5.077,4.170)%
  --(5.077,4.166)--(5.078,4.162)--(5.078,4.160)--(5.078,4.158)--(5.079,4.158)--(5.079,4.155)%
  --(5.080,4.152)--(5.080,4.146)--(5.081,4.146)--(5.081,4.144)--(5.081,4.134)--(5.082,4.121)%
  --(5.082,4.114)--(5.083,4.109)--(5.083,4.106)--(5.083,4.104)--(5.084,4.100)--(5.084,4.097)%
  --(5.085,4.091)--(5.085,4.087)--(5.085,4.080)--(5.086,4.076)--(5.086,4.068)--(5.086,4.066)%
  --(5.087,4.059)--(5.087,4.054)--(5.088,4.050)--(5.088,4.042)--(5.088,4.038)--(5.089,4.031)%
  --(5.089,4.020)--(5.090,4.017)--(5.090,4.012)--(5.090,4.010)--(5.091,4.008)--(5.092,4.005)%
  --(5.092,4.001)--(5.092,3.997)--(5.093,3.994)--(5.093,3.992)--(5.093,3.987)--(5.094,3.985)%
  --(5.094,3.984)--(5.095,3.979)--(5.095,3.978)--(5.095,3.972)--(5.096,3.970)--(5.096,3.967)%
  --(5.097,3.963)--(5.097,3.960)--(5.097,3.959)--(5.098,3.956)--(5.098,3.954)--(5.099,3.949)%
  --(5.099,3.946)--(5.099,3.941)--(5.100,3.938)--(5.100,3.932)--(5.100,3.927)--(5.101,3.925)%
  --(5.101,3.921)--(5.102,3.917)--(5.102,3.911)--(5.102,3.906)--(5.103,3.901)--(5.103,3.898)%
  --(5.104,3.897)--(5.104,3.891)--(5.104,3.889)--(5.105,3.885)--(5.105,3.881)--(5.106,3.876)%
  --(5.106,3.872)--(5.106,3.866)--(5.107,3.864)--(5.107,3.861)--(5.107,3.857)--(5.108,3.849)%
  --(5.108,3.841)--(5.109,3.834)--(5.109,3.831)--(5.109,3.829)--(5.110,3.828)--(5.110,3.825)%
  --(5.111,3.821)--(5.111,3.817)--(5.111,3.811)--(5.112,3.807)--(5.112,3.802)--(5.112,3.797)%
  --(5.113,3.792)--(5.113,3.788)--(5.114,3.784)--(5.114,3.780)--(5.114,3.779)--(5.115,3.774)%
  --(5.115,3.770)--(5.116,3.767)--(5.116,3.765)--(5.116,3.761)--(5.117,3.760)--(5.117,3.758)%
  --(5.118,3.755)--(5.118,3.754)--(5.118,3.750)--(5.119,3.746)--(5.119,3.743)--(5.119,3.740)%
  --(5.120,3.738)--(5.120,3.737)--(5.121,3.734)--(5.121,3.730)--(5.121,3.728)--(5.122,3.726)%
  --(5.122,3.724)--(5.123,3.723)--(5.123,3.720)--(5.123,3.719)--(5.124,3.717)--(5.125,3.719)%
  --(5.125,3.717)--(5.125,3.716)--(5.126,3.712)--(5.126,3.708)--(5.126,3.704)--(5.127,3.700)%
  --(5.127,3.697)--(5.128,3.692)--(5.128,3.691)--(5.128,3.684)--(5.129,3.681)--(5.129,3.676)%
  --(5.130,3.673)--(5.130,3.668)--(5.130,3.665)--(5.131,3.662)--(5.131,3.655)--(5.132,3.651)%
  --(5.132,3.647)--(5.132,3.643)--(5.133,3.641)--(5.133,3.637)--(5.133,3.635)--(5.134,3.629)%
  --(5.134,3.625)--(5.135,3.619)--(5.135,3.615)--(5.135,3.609)--(5.136,3.605)--(5.136,3.601)%
  --(5.137,3.599)--(5.137,3.597)--(5.137,3.592)--(5.138,3.589)--(5.138,3.584)--(5.139,3.581)%
  --(5.139,3.579)--(5.139,3.576)--(5.140,3.573)--(5.140,3.572)--(5.140,3.571)--(5.141,3.568)%
  --(5.141,3.564)--(5.142,3.562)--(5.142,3.560)--(5.143,3.559)--(5.143,3.557)--(5.144,3.553)%
  --(5.144,3.554)--(5.144,3.551)--(5.145,3.548)--(5.145,3.544)--(5.145,3.543)--(5.146,3.540)%
  --(5.146,3.539)--(5.147,3.539)--(5.147,3.536)--(5.147,3.533)--(5.148,3.530)--(5.148,3.527)%
  --(5.149,3.525)--(5.149,3.523)--(5.149,3.521)--(5.150,3.519)--(5.150,3.517)--(5.151,3.513)%
  --(5.151,3.510)--(5.151,3.508)--(5.152,3.507)--(5.152,3.501)--(5.152,3.499)--(5.153,3.498)%
  --(5.154,3.495)--(5.154,3.493)--(5.154,3.488)--(5.155,3.483)--(5.155,3.482)--(5.156,3.478)%
  --(5.156,3.477)--(5.156,3.476)--(5.157,3.474)--(5.157,3.473)--(5.158,3.469)--(5.158,3.468)%
  --(5.158,3.464)--(5.159,3.464)--(5.159,3.460)--(5.159,3.453)--(5.160,3.452)--(5.160,3.448)%
  --(5.161,3.446)--(5.161,3.445)--(5.161,3.442)--(5.162,3.439)--(5.162,3.436)--(5.163,3.436)%
  --(5.163,3.432)--(5.163,3.431)--(5.164,3.428)--(5.164,3.424)--(5.165,3.424)--(5.165,3.422)%
  --(5.165,3.424)--(5.166,3.420)--(5.166,3.418)--(5.167,3.417)--(5.167,3.413)--(5.168,3.410)%
  --(5.168,3.407)--(5.168,3.409)--(5.169,3.406)--(5.169,3.405)--(5.170,3.404)--(5.170,3.402)%
  --(5.171,3.399)--(5.171,3.396)--(5.172,3.393)--(5.172,3.392)--(5.172,3.390)--(5.173,3.390)%
  --(5.173,3.389)--(5.173,3.387)--(5.174,3.387)--(5.174,3.382)--(5.175,3.381)--(5.175,3.376)%
  --(5.175,3.373)--(5.176,3.369)--(5.176,3.366)--(5.177,3.366)--(5.177,3.362)--(5.177,3.359)%
  --(5.178,3.351)--(5.178,3.349)--(5.179,3.348)--(5.179,3.346)--(5.179,3.345)--(5.180,3.343)%
  --(5.180,3.340)--(5.181,3.339)--(5.181,3.336)--(5.182,3.335)--(5.182,3.334)--(5.182,3.331)%
  --(5.183,3.330)--(5.183,3.328)--(5.184,3.328)--(5.184,3.323)--(5.185,3.320)--(5.185,3.317)%
  --(5.185,3.314)--(5.186,3.311)--(5.186,3.312)--(5.187,3.310)--(5.187,3.307)--(5.187,3.303)%
  --(5.188,3.300)--(5.188,3.296)--(5.189,3.293)--(5.189,3.292)--(5.189,3.291)--(5.190,3.288)%
  --(5.191,3.284)--(5.191,3.282)--(5.191,3.279)--(5.192,3.278)--(5.192,3.275)--(5.192,3.273)%
  --(5.193,3.270)--(5.193,3.267)--(5.194,3.262)--(5.194,3.260)--(5.194,3.258)--(5.195,3.258)%
  --(5.195,3.255)--(5.196,3.254)--(5.196,3.251)--(5.196,3.248)--(5.197,3.248)--(5.197,3.245)%
  --(5.198,3.243)--(5.198,3.242)--(5.199,3.241)--(5.199,3.240)--(5.199,3.239)--(5.200,3.236)%
  --(5.200,3.233)--(5.201,3.231)--(5.201,3.228)--(5.201,3.225)--(5.202,3.222)--(5.203,3.220)%
  --(5.203,3.222)--(5.204,3.218)--(5.205,3.216)--(5.205,3.214)--(5.205,3.210)--(5.206,3.208)%
  --(5.206,3.205)--(5.206,3.204)--(5.207,3.202)--(5.208,3.202)--(5.208,3.201)--(5.208,3.200)%
  --(5.209,3.200)--(5.209,3.198)--(5.210,3.198)--(5.210,3.196)--(5.211,3.194)--(5.211,3.195)%
  --(5.212,3.195)--(5.212,3.194)--(5.213,3.192)--(5.213,3.191)--(5.213,3.190)--(5.214,3.189)%
  --(5.214,3.187)--(5.215,3.185)--(5.215,3.179)--(5.215,3.175)--(5.216,3.173)--(5.216,3.172)%
  --(5.217,3.169)--(5.218,3.167)--(5.218,3.165)--(5.218,3.161)--(5.219,3.157)--(5.219,3.156)%
  --(5.220,3.150)--(5.220,3.149)--(5.221,3.147)--(5.221,3.145)--(5.222,3.144)--(5.222,3.141)%
  --(5.223,3.139)--(5.223,3.138)--(5.224,3.137)--(5.224,3.135)--(5.224,3.136)--(5.225,3.134)%
  --(5.225,3.132)--(5.225,3.131)--(5.226,3.129)--(5.226,3.126)--(5.227,3.124)--(5.227,3.123)%
  --(5.227,3.121)--(5.228,3.120)--(5.228,3.119)--(5.229,3.116)--(5.229,3.115)--(5.229,3.116)%
  --(5.230,3.114)--(5.231,3.113)--(5.231,3.110)--(5.232,3.107)--(5.232,3.106)--(5.233,3.104)%
  --(5.233,3.102)--(5.234,3.099)--(5.234,3.097)--(5.235,3.094)--(5.236,3.093)--(5.236,3.092)%
  --(5.236,3.089)--(5.237,3.087)--(5.237,3.083)--(5.238,3.078)--(5.238,3.076)--(5.238,3.074)%
  --(5.239,3.071)--(5.239,3.070)--(5.239,3.069)--(5.240,3.069)--(5.240,3.067)--(5.241,3.066)%
  --(5.241,3.063)--(5.241,3.060)--(5.242,3.059)--(5.242,3.057)--(5.243,3.055)--(5.243,3.053)%
  --(5.243,3.052)--(5.244,3.048)--(5.244,3.045)--(5.245,3.042)--(5.245,3.039)--(5.245,3.036)%
  --(5.246,3.032)--(5.246,3.030)--(5.246,3.028)--(5.247,3.028)--(5.247,3.027)--(5.248,3.025)%
  --(5.248,3.023)--(5.248,3.021)--(5.249,3.017)--(5.249,3.011)--(5.250,3.007)--(5.250,3.005)%
  --(5.250,3.002)--(5.251,3.001)--(5.251,2.999)--(5.252,2.998)--(5.252,2.993)--(5.253,2.992)%
  --(5.253,2.988)--(5.253,2.984)--(5.254,2.982)--(5.254,2.980)--(5.255,2.977)--(5.255,2.974)%
  --(5.255,2.973)--(5.256,2.972)--(5.256,2.970)--(5.257,2.966)--(5.257,2.964)--(5.257,2.961)%
  --(5.258,2.961)--(5.258,2.958)--(5.258,2.957)--(5.259,2.954)--(5.259,2.951)--(5.260,2.948)%
  --(5.260,2.946)--(5.260,2.945)--(5.261,2.942)--(5.261,2.939)--(5.262,2.938)--(5.262,2.936)%
  --(5.262,2.934)--(5.263,2.932)--(5.264,2.929)--(5.264,2.928)--(5.265,2.925)--(5.265,2.924)%
  --(5.265,2.923)--(5.266,2.920)--(5.267,2.920)--(5.267,2.918)--(5.268,2.918)--(5.269,2.915)%
  --(5.269,2.914)--(5.269,2.912)--(5.270,2.912)--(5.270,2.907)--(5.271,2.904)--(5.271,2.900)%
  --(5.271,2.897)--(5.272,2.895)--(5.272,2.892)--(5.272,2.891)--(5.273,2.889)--(5.273,2.887)%
  --(5.274,2.884)--(5.274,2.880)--(5.274,2.876)--(5.275,2.874)--(5.275,2.870)--(5.276,2.868)%
  --(5.276,2.866)--(5.277,2.863)--(5.277,2.861)--(5.278,2.859)--(5.278,2.857)--(5.278,2.856)%
  --(5.279,2.855)--(5.279,2.853)--(5.279,2.851)--(5.280,2.852)--(5.280,2.848)--(5.281,2.847)%
  --(5.281,2.845)--(5.281,2.840)--(5.282,2.839)--(5.282,2.837)--(5.283,2.834)--(5.283,2.833)%
  --(5.283,2.829)--(5.284,2.827)--(5.284,2.826)--(5.285,2.821)--(5.285,2.817)--(5.285,2.815)%
  --(5.286,2.813)--(5.286,2.811)--(5.287,2.809)--(5.287,2.808)--(5.288,2.806)--(5.288,2.808)%
  --(5.289,2.806)--(5.289,2.805)--(5.290,2.802)--(5.290,2.799)--(5.290,2.798)--(5.291,2.793)%
  --(5.291,2.792)--(5.291,2.788)--(5.292,2.783)--(5.292,2.779)--(5.293,2.778)--(5.293,2.775)%
  --(5.293,2.771)--(5.294,2.768)--(5.294,2.764)--(5.295,2.761)--(5.295,2.758)--(5.295,2.754)%
  --(5.296,2.752)--(5.296,2.750)--(5.297,2.749)--(5.297,2.744)--(5.297,2.743)--(5.298,2.740)%
  --(5.298,2.738)--(5.298,2.739)--(5.299,2.736)--(5.299,2.735)--(5.300,2.733)--(5.300,2.731)%
  --(5.300,2.729)--(5.301,2.727)--(5.301,2.725)--(5.302,2.724)--(5.302,2.720)--(5.302,2.717)%
  --(5.303,2.717)--(5.303,2.714)--(5.304,2.710)--(5.304,2.709)--(5.304,2.708)--(5.305,2.708)%
  --(5.305,2.706)--(5.305,2.704)--(5.306,2.704)--(5.306,2.701)--(5.307,2.699)--(5.307,2.697)%
  --(5.307,2.694)--(5.308,2.690)--(5.308,2.689)--(5.309,2.687)--(5.309,2.685)--(5.309,2.684)%
  --(5.310,2.681)--(5.310,2.678)--(5.311,2.676)--(5.311,2.673)--(5.311,2.672)--(5.312,2.671)%
  --(5.312,2.668)--(5.312,2.665)--(5.313,2.665)--(5.313,2.661)--(5.314,2.661)--(5.314,2.658)%
  --(5.314,2.656)--(5.315,2.653)--(5.315,2.650)--(5.316,2.648)--(5.316,2.647)--(5.316,2.645)%
  --(5.317,2.642)--(5.317,2.640)--(5.318,2.638)--(5.318,2.635)--(5.318,2.633)--(5.319,2.630)%
  --(5.319,2.625)--(5.319,2.621)--(5.320,2.618)--(5.320,2.617)--(5.321,2.614)--(5.321,2.608)%
  --(5.321,2.604)--(5.322,2.600)--(5.322,2.597)--(5.323,2.592)--(5.323,2.590)--(5.323,2.588)%
  --(5.324,2.586)--(5.324,2.585)--(5.324,2.586)--(5.325,2.584)--(5.326,2.583)--(5.326,2.579)%
  --(5.326,2.577)--(5.327,2.575)--(5.327,2.574)--(5.328,2.573)--(5.328,2.569)--(5.329,2.569)%
  --(5.329,2.568)--(5.330,2.568)--(5.330,2.566)--(5.331,2.564)--(5.331,2.563)--(5.331,2.562)%
  --(5.332,2.563)--(5.333,2.562)--(5.333,2.559)--(5.333,2.554)--(5.334,2.550)--(5.334,2.549)%
  --(5.335,2.546)--(5.335,2.544)--(5.335,2.540)--(5.336,2.538)--(5.336,2.537)--(5.337,2.536)%
  --(5.337,2.537)--(5.337,2.535)--(5.338,2.535)--(5.338,2.532)--(5.338,2.531)--(5.339,2.528)%
  --(5.339,2.527)--(5.340,2.524)--(5.340,2.523)--(5.340,2.520)--(5.341,2.517)--(5.341,2.515)%
  --(5.342,2.513)--(5.342,2.510)--(5.342,2.506)--(5.343,2.503)--(5.343,2.500)--(5.344,2.498)%
  --(5.344,2.496)--(5.344,2.492)--(5.345,2.490)--(5.345,2.485)--(5.345,2.481)--(5.346,2.477)%
  --(5.346,2.475)--(5.347,2.473)--(5.347,2.471)--(5.347,2.470)--(5.348,2.469)--(5.348,2.468)%
  --(5.349,2.467)--(5.349,2.465)--(5.349,2.463)--(5.350,2.460)--(5.350,2.459)--(5.351,2.457)%
  --(5.352,2.455)--(5.352,2.452)--(5.352,2.450)--(5.353,2.447)--(5.353,2.445)--(5.354,2.443)%
  --(5.354,2.441)--(5.354,2.438)--(5.355,2.436)--(5.355,2.435)--(5.356,2.433)--(5.356,2.431)%
  --(5.356,2.429)--(5.357,2.427)--(5.357,2.425)--(5.358,2.424)--(5.358,2.423)--(5.358,2.421)%
  --(5.359,2.419)--(5.359,2.416)--(5.359,2.415)--(5.360,2.413)--(5.360,2.409)--(5.361,2.407)%
  --(5.361,2.404)--(5.361,2.400)--(5.362,2.398)--(5.362,2.396)--(5.363,2.394)--(5.363,2.391)%
  --(5.363,2.387)--(5.364,2.386)--(5.364,2.384)--(5.364,2.381)--(5.365,2.379)--(5.365,2.376)%
  --(5.366,2.374)--(5.366,2.372)--(5.366,2.368)--(5.367,2.365)--(5.368,2.364)--(5.368,2.362)%
  --(5.368,2.360)--(5.369,2.361)--(5.369,2.358)--(5.370,2.355)--(5.370,2.352)--(5.370,2.348)%
  --(5.371,2.346)--(5.371,2.343)--(5.371,2.339)--(5.372,2.337)--(5.372,2.335)--(5.373,2.332)%
  --(5.373,2.329)--(5.373,2.327)--(5.374,2.325)--(5.374,2.322)--(5.375,2.317)--(5.375,2.312)%
  --(5.375,2.309)--(5.376,2.307)--(5.376,2.306)--(5.377,2.305)--(5.377,2.303)--(5.377,2.302)%
  --(5.378,2.299)--(5.378,2.297)--(5.378,2.293)--(5.379,2.291)--(5.379,2.287)--(5.380,2.284)%
  --(5.380,2.283)--(5.380,2.282)--(5.381,2.281)--(5.381,2.280)--(5.382,2.275)--(5.382,2.272)%
  --(5.382,2.268)--(5.383,2.266)--(5.383,2.265)--(5.384,2.262)--(5.384,2.259)--(5.384,2.256)%
  --(5.385,2.255)--(5.385,2.250)--(5.386,2.248)--(5.387,2.247)--(5.387,2.245)--(5.387,2.242)%
  --(5.388,2.241)--(5.388,2.240)--(5.389,2.237)--(5.389,2.236)--(5.389,2.234)--(5.390,2.231)%
  --(5.391,2.229)--(5.391,2.228)--(5.391,2.226)--(5.392,2.224)--(5.392,2.221)--(5.392,2.220)%
  --(5.393,2.218)--(5.393,2.216)--(5.394,2.213)--(5.394,2.211)--(5.395,2.209)--(5.396,2.207)%
  --(5.396,2.206)--(5.396,2.204)--(5.397,2.202)--(5.397,2.199)--(5.397,2.196)--(5.398,2.194)%
  --(5.398,2.190)--(5.399,2.187)--(5.399,2.185)--(5.399,2.182)--(5.400,2.180)--(5.400,2.178)%
  --(5.401,2.176)--(5.401,2.175)--(5.401,2.174)--(5.402,2.174)--(5.402,2.171)--(5.403,2.169)%
  --(5.403,2.167)--(5.403,2.165)--(5.404,2.162)--(5.404,2.160)--(5.404,2.158)--(5.405,2.156)%
  --(5.405,2.154)--(5.406,2.154)--(5.406,2.152)--(5.407,2.150)--(5.408,2.149)--(5.408,2.146)%
  --(5.409,2.143)--(5.409,2.144)--(5.410,2.141)--(5.410,2.139)--(5.410,2.137)--(5.411,2.135)%
  --(5.411,2.133)--(5.411,2.131)--(5.412,2.129)--(5.412,2.127)--(5.413,2.124)--(5.413,2.123)%
  --(5.413,2.121)--(5.414,2.119)--(5.414,2.117)--(5.415,2.115)--(5.415,2.113)--(5.415,2.109)%
  --(5.416,2.107)--(5.416,2.105)--(5.417,2.103)--(5.417,2.101)--(5.417,2.100)--(5.418,2.099)%
  --(5.418,2.097)--(5.418,2.095)--(5.419,2.091)--(5.419,2.087)--(5.420,2.085)--(5.420,2.084)%
  --(5.421,2.082)--(5.421,2.081)--(5.422,2.080)--(5.422,2.078)--(5.422,2.076)--(5.423,2.074)%
  --(5.423,2.073)--(5.424,2.072)--(5.424,2.070)--(5.424,2.068)--(5.425,2.067)--(5.425,2.066)%
  --(5.425,2.065)--(5.426,2.063)--(5.426,2.061)--(5.427,2.058)--(5.427,2.056)--(5.427,2.057)%
  --(5.428,2.056)--(5.428,2.054)--(5.429,2.053)--(5.429,2.051)--(5.429,2.050)--(5.430,2.047)%
  --(5.430,2.046)--(5.431,2.043)--(5.431,2.042)--(5.432,2.039)--(5.432,2.037)--(5.432,2.035)%
  --(5.433,2.034)--(5.433,2.030)--(5.434,2.030)--(5.434,2.028)--(5.434,2.027)--(5.435,2.027)%
  --(5.436,2.024)--(5.436,2.023)--(5.437,2.022)--(5.437,2.020)--(5.437,2.019)--(5.438,2.017)%
  --(5.438,2.015)--(5.439,2.014)--(5.439,2.011)--(5.440,2.010)--(5.440,2.008)--(5.441,2.006)%
  --(5.441,2.003)--(5.441,2.002)--(5.442,2.000)--(5.442,1.999)--(5.443,1.998)--(5.443,1.996)%
  --(5.444,1.994)--(5.444,1.992)--(5.444,1.989)--(5.445,1.987)--(5.445,1.986)--(5.446,1.986)%
  --(5.446,1.985)--(5.446,1.984)--(5.447,1.982)--(5.447,1.980)--(5.448,1.979)--(5.448,1.976)%
  --(5.448,1.974)--(5.449,1.972)--(5.449,1.969)--(5.450,1.967)--(5.450,1.966)--(5.451,1.965)%
  --(5.451,1.963)--(5.451,1.961)--(5.452,1.959)--(5.452,1.958)--(5.453,1.957)--(5.453,1.955)%
  --(5.453,1.954)--(5.454,1.954)--(5.454,1.952)--(5.455,1.951)--(5.455,1.949)--(5.455,1.946)%
  --(5.456,1.944)--(5.456,1.942)--(5.457,1.938)--(5.458,1.936)--(5.458,1.933)--(5.458,1.932)%
  --(5.459,1.929)--(5.459,1.928)--(5.460,1.928)--(5.460,1.926)--(5.461,1.924)--(5.461,1.922)%
  --(5.462,1.922)--(5.462,1.921)--(5.462,1.920)--(5.463,1.918)--(5.464,1.916)--(5.464,1.914)%
  --(5.464,1.913)--(5.465,1.911)--(5.465,1.910)--(5.465,1.909)--(5.466,1.908)--(5.466,1.906)%
  --(5.467,1.905)--(5.467,1.903)--(5.467,1.901)--(5.468,1.900)--(5.469,1.898)--(5.469,1.896)%
  --(5.470,1.894)--(5.470,1.893)--(5.470,1.891)--(5.471,1.889)--(5.471,1.887)--(5.472,1.886)%
  --(5.472,1.885)--(5.473,1.885)--(5.473,1.883)--(5.474,1.880)--(5.474,1.879)--(5.474,1.877)%
  --(5.475,1.875)--(5.475,1.874)--(5.476,1.874)--(5.476,1.873)--(5.477,1.871)--(5.477,1.870)%
  --(5.477,1.869)--(5.478,1.867)--(5.478,1.866)--(5.479,1.866)--(5.479,1.865)--(5.479,1.864)%
  --(5.480,1.863)--(5.481,1.863)--(5.481,1.862)--(5.482,1.861)--(5.482,1.860)--(5.483,1.859)%
  --(5.483,1.858)--(5.484,1.857)--(5.484,1.855)--(5.484,1.854)--(5.485,1.853)--(5.485,1.851)%
  --(5.486,1.851)--(5.486,1.849)--(5.486,1.848)--(5.487,1.848)--(5.487,1.846)--(5.488,1.845)%
  --(5.488,1.843)--(5.488,1.841)--(5.489,1.839)--(5.489,1.837)--(5.490,1.836)--(5.491,1.835)%
  --(5.491,1.832)--(5.491,1.830)--(5.492,1.828)--(5.492,1.827)--(5.493,1.825)--(5.493,1.823)%
  --(5.494,1.823)--(5.495,1.822)--(5.495,1.820)--(5.496,1.818)--(5.496,1.817)--(5.497,1.815)%
  --(5.497,1.814)--(5.497,1.812)--(5.498,1.813)--(5.498,1.812)--(5.498,1.811)--(5.499,1.811)%
  --(5.499,1.809)--(5.500,1.808)--(5.500,1.807)--(5.501,1.805)--(5.502,1.804)--(5.502,1.802)%
  --(5.503,1.802)--(5.503,1.801)--(5.504,1.801)--(5.504,1.799)--(5.505,1.798)--(5.505,1.796)%
  --(5.506,1.796)--(5.506,1.793)--(5.507,1.792)--(5.507,1.790)--(5.508,1.789)--(5.508,1.787)%
  --(5.509,1.786)--(5.509,1.785)--(5.509,1.784)--(5.510,1.784)--(5.510,1.782)--(5.510,1.780)%
  --(5.511,1.780)--(5.511,1.778)--(5.512,1.778)--(5.512,1.777)--(5.512,1.776)--(5.513,1.774)%
  --(5.513,1.772)--(5.514,1.772)--(5.514,1.771)--(5.515,1.769)--(5.515,1.768)--(5.516,1.768)%
  --(5.516,1.767)--(5.517,1.767)--(5.517,1.765)--(5.517,1.764)--(5.518,1.764)--(5.519,1.764)%
  --(5.519,1.762)--(5.519,1.761)--(5.520,1.759)--(5.520,1.757)--(5.521,1.756)--(5.521,1.755)%
  --(5.521,1.753)--(5.522,1.752)--(5.522,1.751)--(5.523,1.749)--(5.523,1.748)--(5.523,1.747)%
  --(5.524,1.747)--(5.524,1.746)--(5.524,1.745)--(5.525,1.745)--(5.525,1.744)--(5.526,1.744)%
  --(5.526,1.742)--(5.527,1.741)--(5.527,1.739)--(5.528,1.737)--(5.528,1.734)--(5.529,1.734)%
  --(5.529,1.732)--(5.530,1.732)--(5.530,1.731)--(5.530,1.729)--(5.531,1.729)--(5.531,1.728)%
  --(5.531,1.726)--(5.532,1.724)--(5.532,1.722)--(5.533,1.721)--(5.533,1.720)--(5.534,1.719)%
  --(5.535,1.718)--(5.535,1.716)--(5.536,1.715)--(5.536,1.714)--(5.536,1.713)--(5.537,1.713)%
  --(5.537,1.712)--(5.538,1.711)--(5.538,1.709)--(5.539,1.708)--(5.540,1.707)--(5.540,1.706)%
  --(5.541,1.704)--(5.542,1.702)--(5.542,1.700)--(5.543,1.698)--(5.543,1.697)--(5.544,1.696)%
  --(5.545,1.696)--(5.545,1.694)--(5.546,1.693)--(5.546,1.692)--(5.547,1.691)--(5.547,1.689)%
  --(5.548,1.687)--(5.548,1.686)--(5.549,1.684)--(5.549,1.683)--(5.550,1.681)--(5.550,1.680)%
  --(5.551,1.680)--(5.551,1.678)--(5.552,1.676)--(5.553,1.675)--(5.553,1.674)--(5.554,1.674)%
  --(5.554,1.673)--(5.555,1.671)--(5.556,1.669)--(5.557,1.668)--(5.557,1.666)--(5.557,1.665)%
  --(5.558,1.665)--(5.558,1.664)--(5.559,1.663)--(5.559,1.662)--(5.560,1.661)--(5.560,1.660)%
  --(5.561,1.660)--(5.561,1.658)--(5.562,1.656)--(5.562,1.655)--(5.563,1.654)--(5.563,1.653)%
  --(5.563,1.652)--(5.564,1.651)--(5.564,1.649)--(5.565,1.649)--(5.566,1.648)--(5.566,1.647)%
  --(5.566,1.646)--(5.567,1.645)--(5.568,1.643)--(5.568,1.644)--(5.568,1.642)--(5.569,1.641)%
  --(5.570,1.639)--(5.570,1.638)--(5.571,1.637)--(5.571,1.636)--(5.571,1.635)--(5.572,1.635)%
  --(5.572,1.633)--(5.573,1.632)--(5.573,1.631)--(5.573,1.629)--(5.574,1.629)--(5.574,1.628)%
  --(5.575,1.627)--(5.575,1.626)--(5.575,1.625)--(5.576,1.624)--(5.576,1.622)--(5.576,1.621)%
  --(5.577,1.620)--(5.578,1.619)--(5.578,1.617)--(5.579,1.617)--(5.579,1.618)--(5.580,1.616)%
  --(5.580,1.615)--(5.580,1.614)--(5.581,1.612)--(5.581,1.611)--(5.582,1.609)--(5.582,1.608)%
  --(5.583,1.607)--(5.583,1.605)--(5.584,1.605)--(5.584,1.604)--(5.585,1.604)--(5.585,1.603)%
  --(5.585,1.602)--(5.586,1.601)--(5.586,1.600)--(5.587,1.599)--(5.587,1.598)--(5.588,1.598)%
  --(5.588,1.595)--(5.589,1.593)--(5.589,1.592)--(5.590,1.592)--(5.590,1.591)--(5.591,1.590)%
  --(5.592,1.589)--(5.592,1.588)--(5.592,1.587)--(5.593,1.586)--(5.594,1.585)--(5.594,1.584)%
  --(5.594,1.582)--(5.595,1.582)--(5.596,1.580)--(5.596,1.579)--(5.597,1.578)--(5.597,1.576)%
  --(5.598,1.576)--(5.598,1.574)--(5.599,1.573)--(5.599,1.572)--(5.599,1.570)--(5.600,1.569)%
  --(5.600,1.568)--(5.601,1.568)--(5.601,1.567)--(5.601,1.566)--(5.602,1.565)--(5.602,1.564)%
  --(5.603,1.563)--(5.603,1.562)--(5.603,1.560)--(5.604,1.558)--(5.605,1.557)--(5.606,1.557)%
  --(5.606,1.555)--(5.607,1.555)--(5.607,1.553)--(5.608,1.552)--(5.608,1.551)--(5.608,1.550)%
  --(5.609,1.549)--(5.609,1.548)--(5.609,1.547)--(5.610,1.546)--(5.610,1.545)--(5.611,1.544)%
  --(5.611,1.542)--(5.611,1.541)--(5.612,1.540)--(5.612,1.539)--(5.613,1.540)--(5.613,1.538)%
  --(5.613,1.536)--(5.614,1.535)--(5.614,1.534)--(5.615,1.533)--(5.615,1.534)--(5.616,1.534)%
  --(5.616,1.533)--(5.616,1.532)--(5.617,1.532)--(5.618,1.531)--(5.618,1.530)--(5.618,1.531)%
  --(5.619,1.530)--(5.619,1.529)--(5.620,1.528)--(5.620,1.527)--(5.621,1.527)--(5.622,1.526)%
  --(5.623,1.526)--(5.624,1.525)--(5.624,1.524)--(5.625,1.523)--(5.625,1.522)--(5.626,1.521)%
  --(5.627,1.520)--(5.627,1.519)--(5.628,1.519)--(5.628,1.518)--(5.629,1.518)--(5.629,1.517)%
  --(5.629,1.516)--(5.630,1.516)--(5.630,1.514)--(5.630,1.513)--(5.631,1.513)--(5.631,1.512)%
  --(5.632,1.511)--(5.633,1.510)--(5.634,1.509)--(5.634,1.508)--(5.635,1.507)--(5.635,1.506)%
  --(5.636,1.504)--(5.636,1.503)--(5.637,1.548)--(5.637,1.550)--(5.637,1.551)--(5.638,1.552)%
  --(5.638,1.553)--(5.639,1.554)--(5.639,1.556)--(5.639,1.557)--(5.640,1.558)--(5.640,1.559)%
  --(5.641,1.560)--(5.641,1.561)--(5.642,1.562)--(5.643,1.562)--(5.643,1.563)--(5.644,1.564)%
  --(5.644,1.565)--(5.644,1.567)--(5.645,1.567)--(5.645,1.568)--(5.646,1.569)--(5.646,1.570)%
  --(5.647,1.571)--(5.647,1.572)--(5.648,1.573)--(5.648,1.575)--(5.649,1.576)--(5.649,1.577)%
  --(5.649,1.578)--(5.650,1.579)--(5.650,1.580)--(5.651,1.580)--(5.651,1.581)--(5.652,1.582)%
  --(5.652,1.583)--(5.653,1.584)--(5.653,1.583)--(5.653,1.584)--(5.654,1.585)--(5.654,1.586)%
  --(5.655,1.588)--(5.655,1.590)--(5.655,1.591)--(5.656,1.591)--(5.656,1.592)--(5.656,1.593)%
  --(5.657,1.594)--(5.658,1.595)--(5.658,1.596)--(5.658,1.597)--(5.659,1.598)--(5.659,1.599)%
  --(5.660,1.599)--(5.661,1.601)--(5.661,1.602)--(5.662,1.602)--(5.662,1.604)--(5.663,1.604)%
  --(5.663,1.606)--(5.663,1.607)--(5.664,1.608)--(5.664,1.610)--(5.665,1.611)--(5.666,1.611)%
  --(5.666,1.613)--(5.667,1.614)--(5.667,1.615)--(5.668,1.615)--(5.668,1.616)--(5.669,1.618)%
  --(5.669,1.619)--(5.669,1.620)--(5.670,1.621)--(5.670,1.622)--(5.670,1.624)--(5.671,1.624)%
  --(5.672,1.625)--(5.672,1.627)--(5.672,1.629)--(5.673,1.629)--(5.674,1.630)--(5.675,1.631)%
  --(5.675,1.633)--(5.676,1.634)--(5.677,1.635)--(5.677,1.636)--(5.677,1.638)--(5.678,1.639)%
  --(5.679,1.640)--(5.680,1.642)--(5.680,1.644)--(5.681,1.645)--(5.681,1.647)--(5.681,1.649)%
  --(5.682,1.650)--(5.682,1.651)--(5.682,1.652)--(5.683,1.653)--(5.684,1.653)--(5.684,1.654)%
  --(5.684,1.656)--(5.685,1.657)--(5.685,1.659)--(5.686,1.659)--(5.687,1.660)--(5.688,1.662)%
  --(5.688,1.663)--(5.688,1.664)--(5.689,1.664)--(5.689,1.665)--(5.690,1.667)--(5.691,1.668)%
  --(5.691,1.669)--(5.692,1.670)--(5.692,1.671)--(5.693,1.674)--(5.693,1.676)--(5.694,1.676)%
  --(5.694,1.677)--(5.695,1.678)--(5.695,1.679)--(5.695,1.680)--(5.696,1.680)--(5.696,1.681)%
  --(5.696,1.682)--(5.697,1.683)--(5.698,1.684)--(5.699,1.685)--(5.700,1.685)--(5.700,1.687)%
  --(5.701,1.687)--(5.701,1.688)--(5.702,1.688)--(5.702,1.689)--(5.703,1.689)--(5.703,1.691)%
  --(5.703,1.692)--(5.704,1.694)--(5.704,1.695)--(5.705,1.695)--(5.705,1.696)--(5.705,1.697)%
  --(5.706,1.697)--(5.706,1.698)--(5.707,1.699)--(5.707,1.700)--(5.707,1.701)--(5.708,1.702)%
  --(5.708,1.703)--(5.709,1.704)--(5.710,1.705)--(5.710,1.706)--(5.710,1.707)--(5.711,1.709)%
  --(5.711,1.711)--(5.712,1.713)--(5.712,1.714)--(5.712,1.715)--(5.713,1.716)--(5.713,1.717)%
  --(5.714,1.718)--(5.714,1.721)--(5.714,1.722)--(5.715,1.724)--(5.715,1.723)--(5.716,1.724)%
  --(5.717,1.725)--(5.717,1.726)--(5.718,1.726)--(5.718,1.727)--(5.719,1.728)--(5.720,1.728)%
  --(5.721,1.728)--(5.721,1.730)--(5.721,1.731)--(5.722,1.731)--(5.722,1.734)--(5.722,1.736)%
  --(5.723,1.739)--(5.723,1.740)--(5.724,1.741)--(5.724,1.743)--(5.724,1.744)--(5.725,1.747)%
  --(5.725,1.748)--(5.726,1.750)--(5.726,1.751)--(5.727,1.752)--(5.727,1.753)--(5.728,1.753)%
  --(5.728,1.754)--(5.728,1.756)--(5.729,1.758)--(5.729,1.762)--(5.729,1.764)--(5.730,1.764)%
  --(5.731,1.765)--(5.731,1.767)--(5.732,1.768)--(5.732,1.770)--(5.733,1.771)--(5.733,1.772)%
  --(5.733,1.774)--(5.734,1.775)--(5.734,1.776)--(5.735,1.776)--(5.735,1.777)--(5.736,1.779)%
  --(5.736,1.781)--(5.736,1.782)--(5.737,1.785)--(5.737,1.788)--(5.738,1.789)--(5.738,1.790)%
  --(5.739,1.791)--(5.739,1.792)--(5.740,1.793)--(5.740,1.794)--(5.740,1.796)--(5.741,1.798)%
  --(5.741,1.801)--(5.742,1.801)--(5.743,1.803)--(5.743,1.805)--(5.744,1.806)--(5.744,1.807)%
  --(5.745,1.810)--(5.745,1.812)--(5.746,1.813)--(5.747,1.814)--(5.747,1.816)--(5.747,1.817)%
  --(5.748,1.820)--(5.748,1.821)--(5.748,1.823)--(5.749,1.823)--(5.749,1.826)--(5.750,1.827)%
  --(5.750,1.828)--(5.751,1.829)--(5.751,1.830)--(5.752,1.831)--(5.752,1.833)--(5.752,1.834)%
  --(5.753,1.835)--(5.753,1.836)--(5.754,1.836)--(5.754,1.838)--(5.754,1.839)--(5.755,1.839)%
  --(5.755,1.840)--(5.755,1.842)--(5.756,1.842)--(5.756,1.843)--(5.757,1.844)--(5.757,1.845)%
  --(5.758,1.847)--(5.758,1.848)--(5.759,1.849)--(5.759,1.850)--(5.759,1.851)--(5.760,1.851)%
  --(5.761,1.851)--(5.762,1.851)--(5.762,1.852)--(5.762,1.854)--(5.763,1.861)--(5.763,1.864)%
  --(5.764,1.865)--(5.764,1.867)--(5.765,1.869)--(5.766,1.870)--(5.766,1.871)--(5.766,1.873)%
  --(5.767,1.873)--(5.767,1.874)--(5.768,1.875)--(5.768,1.877)--(5.768,1.878)--(5.769,1.879)%
  --(5.769,1.880)--(5.769,1.881)--(5.770,1.883)--(5.770,1.885)--(5.771,1.887)--(5.771,1.888)%
  --(5.771,1.889)--(5.772,1.890)--(5.773,1.892)--(5.773,1.895)--(5.774,1.896)--(5.775,1.896)%
  --(5.775,1.897)--(5.775,1.898)--(5.776,1.899)--(5.776,1.900)--(5.777,1.902)--(5.777,1.903)%
  --(5.778,1.906)--(5.778,1.908)--(5.778,1.910)--(5.779,1.910)--(5.779,1.909)--(5.780,1.909)%
  --(5.780,1.911)--(5.781,1.913)--(5.781,1.915)--(5.782,1.916)--(5.782,1.917)--(5.782,1.918)%
  --(5.783,1.918)--(5.783,1.917)--(5.783,1.918)--(5.784,1.918)--(5.784,1.919)--(5.785,1.920)%
  --(5.785,1.923)--(5.785,1.924)--(5.786,1.927)--(5.786,1.930)--(5.787,1.932)--(5.787,1.933)%
  --(5.787,1.934)--(5.788,1.935)--(5.788,1.937)--(5.788,1.938)--(5.789,1.940)--(5.790,1.942)%
  --(5.790,1.945)--(5.790,1.946)--(5.791,1.948)--(5.791,1.951)--(5.792,1.952)--(5.792,1.954)%
  --(5.792,1.957)--(5.793,1.957)--(5.793,1.959)--(5.794,1.962)--(5.794,1.964)--(5.794,1.965)%
  --(5.795,1.966)--(5.795,1.968)--(5.795,1.969)--(5.796,1.970)--(5.796,1.971)--(5.797,1.974)%
  --(5.797,1.973)--(5.798,1.975)--(5.799,1.975)--(5.799,1.976)--(5.799,1.977)--(5.800,1.978)%
  --(5.801,1.979)--(5.801,1.981)--(5.802,1.983)--(5.802,1.984)--(5.802,1.986)--(5.803,1.987)%
  --(5.803,1.989)--(5.804,1.990)--(5.804,1.991)--(5.804,1.992)--(5.805,1.993)--(5.805,1.994)%
  --(5.806,1.995)--(5.806,1.996)--(5.806,1.997)--(5.807,2.000)--(5.808,2.003)--(5.809,2.003)%
  --(5.809,2.005)--(5.809,2.006)--(5.810,2.007)--(5.810,2.009)--(5.811,2.010)--(5.811,2.013)%
  --(5.811,2.012)--(5.812,2.015)--(5.812,2.017)--(5.813,2.018)--(5.813,2.021)--(5.813,2.022)%
  --(5.814,2.022)--(5.814,2.023)--(5.815,2.024)--(5.815,2.026)--(5.815,2.027)--(5.816,2.028)%
  --(5.816,2.030)--(5.816,2.032)--(5.817,2.034)--(5.817,2.036)--(5.818,2.037)--(5.818,2.038)%
  --(5.818,2.040)--(5.819,2.040)--(5.819,2.041)--(5.820,2.041)--(5.820,2.043)--(5.821,2.043)%
  --(5.821,2.045)--(5.821,2.046)--(5.822,2.047)--(5.822,2.049)--(5.823,2.050)--(5.823,2.051)%
  --(5.824,2.051)--(5.824,2.052)--(5.825,2.052)--(5.826,2.055)--(5.826,2.056)--(5.827,2.057)%
  --(5.827,2.058)--(5.828,2.058)--(5.828,2.060)--(5.829,2.060)--(5.830,2.061)--(5.830,2.062)%
  --(5.830,2.063)--(5.831,2.065)--(5.831,2.067)--(5.832,2.070)--(5.832,2.071)--(5.832,2.072)%
  --(5.833,2.074)--(5.834,2.075)--(5.834,2.078)--(5.834,2.079)--(5.835,2.080)--(5.835,2.083)%
  --(5.835,2.084)--(5.836,2.086)--(5.836,2.089)--(5.837,2.089)--(5.837,2.092)--(5.837,2.093)%
  --(5.838,2.094)--(5.839,2.096)--(5.839,2.098)--(5.840,2.099)--(5.840,2.101)--(5.841,2.104)%
  --(5.841,2.106)--(5.842,2.108)--(5.842,2.109)--(5.842,2.108)--(5.843,2.110)--(5.843,2.111)%
  --(5.844,2.113)--(5.844,2.116)--(5.845,2.117)--(5.845,2.119)--(5.846,2.119)--(5.847,2.120)%
  --(5.847,2.121)--(5.848,2.121)--(5.848,2.123)--(5.848,2.124)--(5.849,2.124)--(5.849,2.125)%
  --(5.849,2.126)--(5.850,2.126)--(5.850,2.127)--(5.851,2.127)--(5.851,2.128)--(5.851,2.130)%
  --(5.852,2.130)--(5.852,2.131)--(5.853,2.132)--(5.853,2.135)--(5.853,2.136)--(5.854,2.138)%
  --(5.854,2.139)--(5.854,2.141)--(5.855,2.142)--(5.855,2.143)--(5.856,2.145)--(5.856,2.146)%
  --(5.856,2.148)--(5.857,2.149)--(5.857,2.150)--(5.858,2.150)--(5.858,2.151)--(5.858,2.152)%
  --(5.859,2.154)--(5.859,2.156)--(5.860,2.158)--(5.860,2.160)--(5.861,2.160)--(5.861,2.161)%
  --(5.862,2.162)--(5.862,2.163)--(5.863,2.164)--(5.863,2.165)--(5.863,2.166)--(5.864,2.167)%
  --(5.864,2.168)--(5.865,2.170)--(5.865,2.172)--(5.866,2.172)--(5.866,2.173)--(5.867,2.175)%
  --(5.867,2.176)--(5.867,2.177)--(5.868,2.178)--(5.868,2.179)--(5.869,2.180)--(5.870,2.181)%
  --(5.871,2.183)--(5.872,2.185)--(5.873,2.186)--(5.874,2.188)--(5.874,2.189)--(5.874,2.190)%
  --(5.875,2.192)--(5.875,2.193)--(5.875,2.194)--(5.876,2.194)--(5.877,2.195)--(5.877,2.196)%
  --(5.878,2.198)--(5.878,2.199)--(5.879,2.201)--(5.879,2.203)--(5.879,2.204)--(5.880,2.205)%
  --(5.880,2.208)--(5.881,2.208)--(5.881,2.209)--(5.881,2.211)--(5.882,2.215)--(5.882,2.218)%
  --(5.882,2.220)--(5.883,2.220)--(5.883,2.222)--(5.884,2.222)--(5.885,2.222)--(5.885,2.223)%
  --(5.886,2.225)--(5.886,2.226)--(5.887,2.228)--(5.887,2.229)--(5.888,2.229)--(5.888,2.231)%
  --(5.889,2.231)--(5.889,2.233)--(5.890,2.236)--(5.890,2.237)--(5.891,2.238)--(5.891,2.240)%
  --(5.892,2.242)--(5.892,2.243)--(5.893,2.244)--(5.893,2.245)--(5.893,2.247)--(5.894,2.248)%
  --(5.894,2.249)--(5.895,2.249)--(5.895,2.250)--(5.896,2.251)--(5.896,2.253)--(5.896,2.255)%
  --(5.897,2.256)--(5.897,2.257)--(5.898,2.258)--(5.898,2.259)--(5.898,2.260)--(5.899,2.261)%
  --(5.899,2.262)--(5.900,2.263)--(5.900,2.266)--(5.900,2.267)--(5.901,2.268)--(5.901,2.269)%
  --(5.902,2.269)--(5.903,2.269)--(5.903,2.270)--(5.903,2.271)--(5.904,2.272)--(5.904,2.271)%
  --(5.905,2.273)--(5.905,2.275)--(5.906,2.277)--(5.907,2.279)--(5.907,2.280)--(5.907,2.282)%
  --(5.908,2.283)--(5.908,2.284)--(5.908,2.287)--(5.909,2.288)--(5.910,2.289)--(5.910,2.290)%
  --(5.910,2.292)--(5.911,2.292)--(5.912,2.292)--(5.912,2.294)--(5.913,2.294)--(5.914,2.294)%
  --(5.914,2.295)--(5.915,2.296)--(5.915,2.297)--(5.915,2.298)--(5.916,2.298)--(5.917,2.297)%
  --(5.917,2.298)--(5.917,2.300)--(5.918,2.302)--(5.918,2.303)--(5.919,2.306)--(5.919,2.308)%
  --(5.919,2.311)--(5.920,2.313)--(5.920,2.314)--(5.921,2.314)--(5.922,2.316)--(5.922,2.318)%
  --(5.922,2.320)--(5.923,2.320)--(5.924,2.322)--(5.924,2.324)--(5.925,2.325)--(5.925,2.327)%
  --(5.926,2.327)--(5.926,2.329)--(5.927,2.331)--(5.927,2.332)--(5.927,2.333)--(5.928,2.334)%
  --(5.929,2.335)--(5.929,2.336)--(5.929,2.337)--(5.930,2.339)--(5.930,2.343)--(5.931,2.345)%
  --(5.931,2.346)--(5.932,2.349)--(5.932,2.348)--(5.933,2.349)--(5.933,2.350)--(5.933,2.351)%
  --(5.934,2.353)--(5.934,2.356)--(5.934,2.358)--(5.935,2.360)--(5.935,2.361)--(5.936,2.363)%
  --(5.936,2.364)--(5.937,2.366)--(5.937,2.368)--(5.938,2.369)--(5.938,2.370)--(5.938,2.371)%
  --(5.939,2.373)--(5.939,2.374)--(5.940,2.374)--(5.940,2.375)--(5.940,2.376)--(5.941,2.379)%
  --(5.941,2.378)--(5.941,2.379)--(5.942,2.383)--(5.942,2.386)--(5.943,2.387)--(5.943,2.389)%
  --(5.943,2.390)--(5.944,2.390)--(5.944,2.391)--(5.945,2.393)--(5.945,2.395)--(5.945,2.396)%
  --(5.946,2.397)--(5.946,2.398)--(5.947,2.400)--(5.947,2.401)--(5.947,2.402)--(5.948,2.404)%
  --(5.948,2.408)--(5.949,2.411)--(5.949,2.412)--(5.950,2.414)--(5.950,2.416)--(5.950,2.418)%
  --(5.951,2.419)--(5.951,2.421)--(5.952,2.423)--(5.952,2.425)--(5.952,2.424)--(5.953,2.426)%
  --(5.953,2.427)--(5.954,2.428)--(5.954,2.430)--(5.954,2.432)--(5.955,2.434)--(5.955,2.435)%
  --(5.955,2.437)--(5.956,2.439)--(5.956,2.441)--(5.957,2.443)--(5.957,2.444)--(5.957,2.447)%
  --(5.958,2.448)--(5.958,2.450)--(5.959,2.452)--(5.959,2.451)--(5.959,2.456)--(5.960,2.456)%
  --(5.960,2.458)--(5.961,2.459)--(5.961,2.462)--(5.962,2.464)--(5.962,2.467)--(5.963,2.468)%
  --(5.963,2.470)--(5.964,2.472)--(5.964,2.475)--(5.964,2.477)--(5.965,2.480)--(5.965,2.482)%
  --(5.966,2.483)--(5.966,2.485)--(5.967,2.488)--(5.967,2.489)--(5.967,2.491)--(5.968,2.492)%
  --(5.968,2.494)--(5.969,2.495)--(5.969,2.496)--(5.970,2.498)--(5.970,2.499)--(5.971,2.502)%
  --(5.971,2.503)--(5.971,2.504)--(5.972,2.505)--(5.972,2.506)--(5.973,2.506)--(5.973,2.508)%
  --(5.973,2.511)--(5.974,2.514)--(5.974,2.516)--(5.974,2.519)--(5.975,2.522)--(5.975,2.525)%
  --(5.976,2.528)--(5.976,2.532)--(5.976,2.535)--(5.977,2.534)--(5.977,2.536)--(5.978,2.537)%
  --(5.978,2.541)--(5.978,2.543)--(5.979,2.546)--(5.979,2.549)--(5.980,2.549)--(5.980,2.551)%
  --(5.980,2.553)--(5.981,2.554)--(5.981,2.555)--(5.981,2.558)--(5.982,2.559)--(5.982,2.561)%
  --(5.983,2.564)--(5.983,2.565)--(5.983,2.566)--(5.984,2.568)--(5.984,2.569)--(5.985,2.572)%
  --(5.985,2.574)--(5.985,2.576)--(5.986,2.579)--(5.986,2.581)--(5.987,2.583)--(5.987,2.584)%
  --(5.988,2.585)--(5.988,2.586)--(5.988,2.591)--(5.989,2.595)--(5.989,2.598)--(5.990,2.601)%
  --(5.990,2.604)--(5.990,2.605)--(5.991,2.606)--(5.991,2.607)--(5.992,2.608)--(5.992,2.609)%
  --(5.992,2.610)--(5.993,2.613)--(5.994,2.614)--(5.994,2.617)--(5.994,2.619)--(5.995,2.622)%
  --(5.995,2.625)--(5.995,2.629)--(5.996,2.632)--(5.996,2.636)--(5.997,2.639)--(5.997,2.640)%
  --(5.997,2.643)--(5.998,2.646)--(5.998,2.650)--(5.999,2.651)--(5.999,2.655)--(5.999,2.658)%
  --(6.000,2.662)--(6.000,2.665)--(6.000,2.667)--(6.001,2.669)--(6.001,2.671)--(6.002,2.675)%
  --(6.002,2.681)--(6.002,2.685)--(6.003,2.687)--(6.003,2.691)--(6.004,2.696)--(6.004,2.697)%
  --(6.004,2.698)--(6.005,2.702)--(6.006,2.702)--(6.006,2.703)--(6.006,2.706)--(6.007,2.708)%
  --(6.007,2.711)--(6.007,2.713)--(6.008,2.715)--(6.008,2.717)--(6.009,2.718)--(6.009,2.719)%
  --(6.010,2.719)--(6.010,2.721)--(6.011,2.723)--(6.011,2.724)--(6.011,2.726)--(6.012,2.728)%
  --(6.012,2.729)--(6.013,2.731)--(6.013,2.733)--(6.014,2.734)--(6.014,2.737)--(6.014,2.740)%
  --(6.015,2.746)--(6.015,2.748)--(6.016,2.749)--(6.016,2.752)--(6.017,2.755)--(6.017,2.756)%
  --(6.018,2.757)--(6.018,2.758)--(6.019,2.757)--(6.019,2.760)--(6.020,2.762)--(6.020,2.763)%
  --(6.020,2.766)--(6.021,2.768)--(6.021,2.771)--(6.021,2.772)--(6.022,2.775)--(6.022,2.778)%
  --(6.023,2.780)--(6.023,2.784)--(6.023,2.789)--(6.024,2.793)--(6.024,2.795)--(6.025,2.799)%
  --(6.025,2.801)--(6.025,2.805)--(6.026,2.809)--(6.026,2.815)--(6.027,2.818)--(6.027,2.822)%
  --(6.027,2.824)--(6.028,2.828)--(6.028,2.831)--(6.028,2.832)--(6.029,2.833)--(6.029,2.835)%
  --(6.030,2.836)--(6.030,2.840)--(6.030,2.842)--(6.031,2.843)--(6.031,2.844)--(6.032,2.847)%
  --(6.032,2.850)--(6.032,2.851)--(6.033,2.854)--(6.033,2.856)--(6.033,2.857)--(6.034,2.858)%
  --(6.034,2.859)--(6.035,2.862)--(6.035,2.864)--(6.035,2.868)--(6.036,2.869)--(6.036,2.871)%
  --(6.037,2.873)--(6.037,2.877)--(6.037,2.878)--(6.038,2.882)--(6.038,2.885)--(6.039,2.889)%
  --(6.039,2.891)--(6.039,2.896)--(6.040,2.899)--(6.040,2.901)--(6.040,2.903)--(6.041,2.906)%
  --(6.041,2.908)--(6.042,2.911)--(6.042,2.913)--(6.042,2.915)--(6.043,2.918)--(6.043,2.919)%
  --(6.044,2.923)--(6.044,2.927)--(6.044,2.931)--(6.045,2.934)--(6.045,2.937)--(6.046,2.939)%
  --(6.046,2.943)--(6.046,2.949)--(6.047,2.955)--(6.047,2.959)--(6.047,2.963)--(6.048,2.967)%
  --(6.048,2.969)--(6.049,2.969)--(6.049,2.971)--(6.049,2.975)--(6.050,2.980)--(6.050,2.983)%
  --(6.051,2.986)--(6.051,2.992)--(6.051,2.995)--(6.052,2.998)--(6.052,3.001)--(6.053,3.006)%
  --(6.053,3.011)--(6.053,3.016)--(6.054,3.020)--(6.054,3.023)--(6.054,3.027)--(6.055,3.030)%
  --(6.055,3.034)--(6.056,3.038)--(6.056,3.043)--(6.056,3.049)--(6.057,3.055)--(6.057,3.060)%
  --(6.058,3.062)--(6.058,3.066)--(6.058,3.070)--(6.059,3.074)--(6.059,3.079)--(6.060,3.085)%
  --(6.060,3.089)--(6.060,3.091)--(6.061,3.095)--(6.061,3.100)--(6.061,3.101)--(6.062,3.103)%
  --(6.062,3.105)--(6.063,3.110)--(6.063,3.113)--(6.063,3.117)--(6.064,3.121)--(6.064,3.125)%
  --(6.065,3.130)--(6.065,3.138)--(6.065,3.143)--(6.066,3.151)--(6.066,3.159)--(6.066,3.161)%
  --(6.067,3.162)--(6.067,3.165)--(6.068,3.169)--(6.068,3.173)--(6.068,3.175)--(6.069,3.179)%
  --(6.069,3.181)--(6.070,3.185)--(6.070,3.188)--(6.070,3.192)--(6.071,3.193)--(6.071,3.196)%
  --(6.072,3.199)--(6.072,3.201)--(6.072,3.205)--(6.073,3.208)--(6.073,3.212)--(6.073,3.215)%
  --(6.074,3.218)--(6.074,3.219)--(6.075,3.223)--(6.075,3.226)--(6.075,3.227)--(6.076,3.228)%
  --(6.076,3.230)--(6.077,3.234)--(6.077,3.237)--(6.077,3.244)--(6.078,3.247)--(6.078,3.251)%
  --(6.079,3.255)--(6.079,3.260)--(6.079,3.267)--(6.080,3.273)--(6.080,3.277)--(6.080,3.282)%
  --(6.081,3.286)--(6.081,3.289)--(6.082,3.293)--(6.082,3.295)--(6.082,3.296)--(6.083,3.299)%
  --(6.083,3.304)--(6.084,3.308)--(6.084,3.314)--(6.084,3.319)--(6.085,3.323)--(6.085,3.327)%
  --(6.086,3.333)--(6.086,3.338)--(6.086,3.342)--(6.087,3.347)--(6.087,3.350)--(6.087,3.353)%
  --(6.088,3.355)--(6.088,3.359)--(6.089,3.358)--(6.089,3.359)--(6.089,3.361)--(6.090,3.365)%
  --(6.090,3.366)--(6.091,3.366)--(6.091,3.369)--(6.091,3.373)--(6.092,3.378)--(6.092,3.382)%
  --(6.093,3.383)--(6.093,3.388)--(6.093,3.393)--(6.094,3.400)--(6.094,3.404)--(6.094,3.406)%
  --(6.095,3.407)--(6.095,3.413)--(6.096,3.419)--(6.096,3.423)--(6.096,3.427)--(6.097,3.433)%
  --(6.097,3.438)--(6.098,3.442)--(6.098,3.449)--(6.098,3.454)--(6.099,3.458)--(6.100,3.465)%
  --(6.100,3.469)--(6.100,3.478)--(6.101,3.483)--(6.101,3.485)--(6.101,3.488)--(6.102,3.494)%
  --(6.102,3.498)--(6.103,3.502)--(6.103,3.507)--(6.103,3.512)--(6.104,3.517)--(6.104,3.521)%
  --(6.105,3.524)--(6.105,3.528)--(6.105,3.533)--(6.106,3.535)--(6.106,3.538)--(6.106,3.541)%
  --(6.107,3.546)--(6.107,3.547)--(6.108,3.556)--(6.108,3.558)--(6.109,3.560)--(6.110,3.563)%
  --(6.110,3.565)--(6.110,3.570)--(6.111,3.575)--(6.111,3.578)--(6.112,3.580)--(6.112,3.582)%
  --(6.112,3.585)--(6.113,3.595)--(6.113,3.600)--(6.113,3.605)--(6.114,3.610)--(6.114,3.614)%
  --(6.115,3.622)--(6.115,3.629)--(6.115,3.632)--(6.116,3.636)--(6.117,3.640)--(6.117,3.645)%
  --(6.118,3.648)--(6.118,3.654)--(6.119,3.657)--(6.119,3.661)--(6.119,3.666)--(6.120,3.673)%
  --(6.120,3.679)--(6.120,3.687)--(6.121,3.691)--(6.121,3.694)--(6.122,3.701)--(6.122,3.705)%
  --(6.122,3.712)--(6.123,3.715)--(6.123,3.719)--(6.124,3.723)--(6.124,3.727)--(6.125,3.730)%
  --(6.125,3.738)--(6.126,3.742)--(6.126,3.745)--(6.126,3.747)--(6.127,3.751)--(6.127,3.754)%
  --(6.127,3.756)--(6.128,3.762)--(6.128,3.769)--(6.129,3.772)--(6.129,3.778)--(6.129,3.783)%
  --(6.130,3.790)--(6.130,3.796)--(6.131,3.801)--(6.131,3.805)--(6.132,3.809)--(6.132,3.812)%
  --(6.133,3.815)--(6.133,3.820)--(6.133,3.826)--(6.134,3.831)--(6.134,3.837)--(6.134,3.842)%
  --(6.135,3.846)--(6.135,3.851)--(6.136,3.857)--(6.136,3.863)--(6.136,3.873)--(6.137,3.883)%
  --(6.137,3.890)--(6.138,3.895)--(6.138,3.900)--(6.138,3.903)--(6.139,3.908)--(6.139,3.913)%
  --(6.139,3.916)--(6.140,3.922)--(6.140,3.931)--(6.141,3.933)--(6.141,3.936)--(6.141,3.939)%
  --(6.142,3.944)--(6.142,3.947)--(6.143,3.951)--(6.143,3.955)--(6.143,3.960)--(6.144,3.962)%
  --(6.144,3.964)--(6.145,3.965)--(6.145,3.969)--(6.145,3.981)--(6.146,3.988)--(6.146,3.990)%
  --(6.146,3.993)--(6.147,3.995)--(6.148,3.999)--(6.148,4.010)--(6.148,4.017)--(6.149,4.023)%
  --(6.149,4.026)--(6.150,4.031)--(6.150,4.035)--(6.150,4.040)--(6.151,4.045)--(6.151,4.049)%
  --(6.152,4.054)--(6.152,4.059)--(6.152,4.066)--(6.153,4.073)--(6.153,4.074)--(6.153,4.080)%
  --(6.154,4.089)--(6.154,4.091)--(6.155,4.100)--(6.155,4.111)--(6.155,4.117)--(6.156,4.122)%
  --(6.156,4.129)--(6.157,4.133)--(6.157,4.135)--(6.157,4.137)--(6.158,4.137)--(6.158,4.141)%
  --(6.159,4.144)--(6.159,4.148)--(6.159,4.151)--(6.160,4.153)--(6.160,4.155)--(6.160,4.161)%
  --(6.161,4.164)--(6.161,4.169)--(6.162,4.169)--(6.162,4.173)--(6.163,4.179)--(6.163,4.180)%
  --(6.164,4.185)--(6.164,4.186)--(6.164,4.188)--(6.165,4.192)--(6.165,4.196)--(6.166,4.200)%
  --(6.166,4.203)--(6.166,4.209)--(6.167,4.212)--(6.167,4.218)--(6.167,4.221)--(6.168,4.223)%
  --(6.168,4.227)--(6.169,4.232)--(6.169,4.237)--(6.169,4.244)--(6.170,4.252)--(6.170,4.260)%
  --(6.171,4.266)--(6.171,4.270)--(6.171,4.275)--(6.172,4.277)--(6.172,4.283)--(6.172,4.293)%
  --(6.173,4.304)--(6.173,4.313)--(6.174,4.317)--(6.174,4.320)--(6.174,4.328)--(6.175,4.331)%
  --(6.175,4.335)--(6.176,4.338)--(6.176,4.343)--(6.176,4.346)--(6.177,4.349)--(6.177,4.353)%
  --(6.178,4.359)--(6.178,4.363)--(6.178,4.368)--(6.179,4.374)--(6.179,4.380)--(6.179,4.386)%
  --(6.180,4.391)--(6.180,4.395)--(6.181,4.396)--(6.181,4.402)--(6.181,4.408)--(6.182,4.413)%
  --(6.182,4.415)--(6.183,4.422)--(6.183,4.428)--(6.183,4.431)--(6.184,4.432)--(6.184,4.434)%
  --(6.185,4.435)--(6.185,4.444)--(6.185,4.446)--(6.186,4.448)--(6.186,4.452)--(6.186,4.454)%
  --(6.187,4.460)--(6.187,4.464)--(6.188,4.469)--(6.188,4.474)--(6.188,4.478)--(6.189,4.487)%
  --(6.189,4.492)--(6.190,4.497)--(6.190,4.503)--(6.190,4.506)--(6.191,4.516)--(6.191,4.517)%
  --(6.192,4.524)--(6.192,4.529)--(6.192,4.537)--(6.193,4.541)--(6.193,4.544)--(6.193,4.550)%
  --(6.194,4.558)--(6.194,4.565)--(6.195,4.567)--(6.195,4.573)--(6.195,4.576)--(6.196,4.584)%
  --(6.196,4.590)--(6.197,4.590)--(6.197,4.597)--(6.197,4.598)--(6.198,4.600)--(6.198,4.608)%
  --(6.199,4.614)--(6.199,4.619)--(6.199,4.622)--(6.200,4.627)--(6.200,4.633)--(6.200,4.640)%
  --(6.201,4.651)--(6.201,4.658)--(6.202,4.663)--(6.202,4.666)--(6.202,4.671)--(6.203,4.677)%
  --(6.203,4.686)--(6.204,4.694)--(6.204,4.698)--(6.204,4.705)--(6.205,4.706)--(6.205,4.713)%
  --(6.206,4.718)--(6.206,4.721)--(6.206,4.722)--(6.207,4.725)--(6.207,4.728)--(6.207,4.729)%
  --(6.208,4.736)--(6.208,4.743)--(6.209,4.750)--(6.209,4.759)--(6.209,4.763)--(6.210,4.768)%
  --(6.210,4.772)--(6.211,4.779)--(6.211,4.789)--(6.211,4.790)--(6.212,4.797)--(6.212,4.798)%
  --(6.212,4.805)--(6.213,4.816)--(6.213,4.824)--(6.214,4.833)--(6.214,4.838)--(6.214,4.843)%
  --(6.215,4.842)--(6.215,4.847)--(6.216,4.852)--(6.216,4.857)--(6.216,4.864)--(6.217,4.870)%
  --(6.217,4.871)--(6.218,4.873)--(6.218,4.878)--(6.218,4.879)--(6.219,4.882)--(6.219,4.887)%
  --(6.219,4.893)--(6.220,4.899)--(6.220,4.904)--(6.221,4.906)--(6.221,4.910)--(6.221,4.914)%
  --(6.222,4.920)--(6.222,4.924)--(6.223,4.928)--(6.223,4.936)--(6.223,4.940)--(6.224,4.943)%
  --(6.224,4.953)--(6.225,4.957)--(6.225,4.962)--(6.225,4.964)--(6.226,4.976)--(6.226,4.978)%
  --(6.226,4.983)--(6.227,4.989)--(6.227,4.995)--(6.228,5.000)--(6.228,5.008)--(6.228,5.015)%
  --(6.229,5.019)--(6.229,5.027)--(6.230,5.035)--(6.230,5.045)--(6.230,5.050)--(6.231,5.056)%
  --(6.231,5.063)--(6.232,5.068)--(6.232,5.073)--(6.232,5.078)--(6.233,5.087)--(6.233,5.093)%
  --(6.233,5.097)--(6.234,5.105)--(6.234,5.112)--(6.235,5.112)--(6.235,5.114)--(6.235,5.121)%
  --(6.236,5.127)--(6.236,5.130)--(6.237,5.134)--(6.237,5.139)--(6.237,5.144)--(6.238,5.148)%
  --(6.238,5.156)--(6.239,5.157)--(6.239,5.168)--(6.239,5.174)--(6.240,5.179)--(6.240,5.181)%
  --(6.240,5.185)--(6.241,5.189)--(6.241,5.197)--(6.242,5.201)--(6.242,5.210)--(6.242,5.215)%
  --(6.243,5.222)--(6.243,5.228)--(6.244,5.238)--(6.244,5.242)--(6.244,5.249)--(6.245,5.258)%
  --(6.245,5.266)--(6.245,5.271)--(6.246,5.282)--(6.246,5.289)--(6.247,5.296)--(6.247,5.302)%
  --(6.247,5.305)--(6.248,5.314)--(6.248,5.320)--(6.249,5.325)--(6.249,5.330)--(6.249,5.334)%
  --(6.250,5.337)--(6.250,5.345)--(6.251,5.351)--(6.251,5.359)--(6.251,5.364)--(6.252,5.371)%
  --(6.252,5.381)--(6.252,5.391)--(6.253,5.397)--(6.253,5.405)--(6.254,5.416)--(6.254,5.426)%
  --(6.254,5.433)--(6.255,5.440)--(6.255,5.448)--(6.256,5.459)--(6.256,5.470)--(6.256,5.472)%
  --(6.257,5.475)--(6.257,5.482)--(6.258,5.489)--(6.258,5.496)--(6.258,5.498)--(6.259,5.509)%
  --(6.259,5.514)--(6.259,5.519)--(6.260,5.521)--(6.260,5.528)--(6.261,5.532)--(6.261,5.545)%
  --(6.261,5.555)--(6.262,5.560)--(6.262,5.566)--(6.263,5.576)--(6.263,5.585)--(6.263,5.595)%
  --(6.264,5.604)--(6.264,5.611)--(6.265,5.624)--(6.265,5.626)--(6.265,5.629)--(6.266,5.632)%
  --(6.266,5.640)--(6.266,5.648)--(6.267,5.651)--(6.267,5.669)--(6.268,5.675)--(6.268,5.679)%
  --(6.268,5.683)--(6.269,5.693)--(6.269,5.707)--(6.270,5.716)--(6.270,5.726)--(6.270,5.733)%
  --(6.271,5.739)--(6.271,5.744)--(6.272,5.746)--(6.272,5.747)--(6.272,5.758)--(6.273,5.761)%
  --(6.273,5.769)--(6.273,5.779)--(6.274,5.792)--(6.275,5.800)--(6.275,5.812)--(6.275,5.823)%
  --(6.276,5.824)--(6.276,5.830)--(6.277,5.835)--(6.277,5.836)--(6.277,5.842)--(6.278,5.850)%
  --(6.278,5.857)--(6.279,5.862)--(6.279,5.869)--(6.279,5.876)--(6.280,5.890)--(6.280,5.903)%
  --(6.280,5.912)--(6.281,5.922)--(6.281,5.924)--(6.282,5.935)--(6.282,5.945)--(6.282,5.956)%
  --(6.283,5.968)--(6.283,5.978)--(6.284,5.985)--(6.284,5.994)--(6.284,6.007)--(6.285,6.030)%
  --(6.285,6.037)--(6.285,6.045)--(6.286,6.053)--(6.286,6.060)--(6.287,6.062)--(6.287,6.066)%
  --(6.287,6.071)--(6.288,6.078)--(6.288,6.087)--(6.289,6.097)--(6.289,6.104)--(6.289,6.115)%
  --(6.290,6.129)--(6.290,6.137)--(6.291,6.149)--(6.291,6.160)--(6.291,6.168)--(6.292,6.182)%
  --(6.292,6.191)--(6.292,6.203)--(6.293,6.218)--(6.293,6.225)--(6.294,6.231)--(6.294,6.236)%
  --(6.294,6.244)--(6.295,6.246)--(6.295,6.262)--(6.296,6.270)--(6.296,6.273)--(6.296,6.283)%
  --(6.297,6.287)--(6.297,6.296)--(6.298,6.299)--(6.298,6.307)--(6.298,6.312)--(6.299,6.326)%
  --(6.299,6.336)--(6.299,6.345)--(6.300,6.353)--(6.300,6.362)--(6.301,6.370)--(6.301,6.376)%
  --(6.301,6.383)--(6.302,6.383)--(6.302,6.384)--(6.303,6.390)--(6.303,6.399)--(6.303,6.414)%
  --(6.304,6.424)--(6.304,6.432)--(6.305,6.437)--(6.305,6.442)--(6.305,6.454)--(6.306,6.473)%
  --(6.306,6.482)--(6.306,6.490)--(6.307,6.508)--(6.307,6.515)--(6.308,6.519)--(6.308,6.527)%
  --(6.308,6.535)--(6.309,6.549)--(6.309,6.560)--(6.310,6.569)--(6.310,6.580)--(6.310,6.595)%
  --(6.311,6.609)--(6.311,6.632)--(6.312,6.652)--(6.312,6.663)--(6.313,6.661)--(6.313,6.668)%
  --(6.313,6.679)--(6.314,6.691)--(6.314,6.700)--(6.315,6.711)--(6.315,6.728)--(6.315,6.750)%
  --(6.316,6.768)--(6.316,6.782)--(6.317,6.793)--(6.317,6.804)--(6.317,6.808)--(6.318,6.501)%
  --(6.318,6.484)--(6.318,6.463)--(6.319,6.456)--(6.319,6.443)--(6.320,6.432)--(6.320,6.418)%
  --(6.320,6.402)--(6.321,6.387)--(6.321,6.365)--(6.322,6.362)--(6.322,6.357)--(6.322,6.349)%
  --(6.323,6.343)--(6.323,6.329)--(6.324,6.323)--(6.324,6.313)--(6.324,6.292)--(6.325,6.277)%
  --(6.325,6.262)--(6.325,6.239)--(6.326,6.216)--(6.326,6.195)--(6.327,6.184)--(6.327,6.162)%
  --(6.327,6.151)--(6.328,6.129)--(6.328,6.113)--(6.329,6.087)--(6.329,6.070)--(6.329,6.058)%
  --(6.330,6.043)--(6.330,6.034)--(6.331,6.018)--(6.331,5.998)--(6.331,5.988)--(6.332,5.973)%
  --(6.332,5.958)--(6.332,5.945)--(6.333,5.936)--(6.333,5.926)--(6.334,5.906)--(6.334,5.881)%
  --(6.334,5.864)--(6.335,5.857)--(6.335,5.854)--(6.336,5.846)--(6.336,5.836)--(6.336,5.829)%
  --(6.337,5.826)--(6.337,5.818)--(6.338,5.817)--(6.338,5.809)--(6.338,5.808)--(6.339,5.793)%
  --(6.339,5.783)--(6.339,5.778)--(6.340,5.770)--(6.340,5.760)--(6.341,5.742)--(6.341,5.726)%
  --(6.341,5.707)--(6.342,5.696)--(6.342,5.687)--(6.343,5.677)--(6.343,5.664)--(6.343,5.658)%
  --(6.344,5.654)--(6.344,5.643)--(6.345,5.629)--(6.345,5.621)--(6.345,5.613)--(6.346,5.603)%
  --(6.346,5.595)--(6.346,5.590)--(6.347,5.583)--(6.347,5.575)--(6.348,5.562)--(6.348,5.541)%
  --(6.348,5.529)--(6.349,5.523)--(6.349,5.516)--(6.350,5.502)--(6.350,5.492)--(6.350,5.488)%
  --(6.351,5.474)--(6.351,5.458)--(6.351,5.449)--(6.352,5.440)--(6.352,5.438)--(6.353,5.430)%
  --(6.353,5.423)--(6.353,5.416)--(6.354,5.410)--(6.354,5.399)--(6.355,5.386)--(6.355,5.373)%
  --(6.355,5.358)--(6.356,5.341)--(6.356,5.323)--(6.357,5.310)--(6.357,5.296)--(6.357,5.286)%
  --(6.358,5.271)--(6.358,5.254)--(6.358,5.245)--(6.359,5.236)--(6.359,5.224)--(6.360,5.217)%
  --(6.360,5.209)--(6.360,5.205)--(6.361,5.200)--(6.361,5.198)--(6.362,5.194)--(6.362,5.190)%
  --(6.362,5.185)--(6.363,5.174)--(6.363,5.163)--(6.364,5.152)--(6.364,5.139)--(6.364,5.132)%
  --(6.365,5.119)--(6.365,5.112)--(6.365,5.104)--(6.366,5.097)--(6.366,5.088)--(6.367,5.079)%
  --(6.367,5.066)--(6.367,5.052)--(6.368,5.044)--(6.368,5.036)--(6.369,5.023)--(6.369,5.009)%
  --(6.369,5.004)--(6.370,5.001)--(6.370,4.994)--(6.371,4.991)--(6.371,4.985)--(6.371,4.975)%
  --(6.372,4.967)--(6.372,4.956)--(6.372,4.950)--(6.373,4.941)--(6.373,4.937)--(6.374,4.934)%
  --(6.374,4.931)--(6.374,4.929)--(6.375,4.926)--(6.375,4.920)--(6.376,4.917)--(6.376,4.911)%
  --(6.376,4.905)--(6.377,4.895)--(6.377,4.887)--(6.378,4.885)--(6.378,4.881)--(6.378,4.882)%
  --(6.379,4.877)--(6.379,4.876)--(6.379,4.868)--(6.380,4.863)--(6.380,4.855)--(6.381,4.853)%
  --(6.381,4.849)--(6.381,4.841)--(6.382,4.837)--(6.382,4.832)--(6.383,4.830)--(6.383,4.822)%
  --(6.384,4.819)--(6.384,4.814)--(6.385,4.810)--(6.385,4.808)--(6.385,4.800)--(6.386,4.796)%
  --(6.386,4.789)--(6.386,4.785)--(6.387,4.780)--(6.387,4.771)--(6.388,4.768)--(6.388,4.756)%
  --(6.388,4.753)--(6.389,4.744)--(6.390,4.739)--(6.390,4.729)--(6.390,4.721)--(6.391,4.713)%
  --(6.391,4.703)--(6.391,4.693)--(6.392,4.684)--(6.392,4.681)--(6.393,4.673)--(6.393,4.668)%
  --(6.393,4.661)--(6.394,4.653)--(6.394,4.642)--(6.395,4.632)--(6.395,4.622)--(6.395,4.616)%
  --(6.396,4.610)--(6.396,4.603)--(6.397,4.596)--(6.397,4.588)--(6.397,4.581)--(6.398,4.577)%
  --(6.398,4.572)--(6.398,4.565)--(6.399,4.559)--(6.399,4.555)--(6.400,4.550)--(6.400,4.546)%
  --(6.400,4.541)--(6.401,4.528)--(6.401,4.518)--(6.402,4.516)--(6.402,4.511)--(6.402,4.509)%
  --(6.403,4.504)--(6.403,4.498)--(6.404,4.487)--(6.404,4.478)--(6.404,4.465)--(6.405,4.459)%
  --(6.405,4.455)--(6.405,4.446)--(6.406,4.440)--(6.406,4.433)--(6.407,4.425)--(6.407,4.419)%
  --(6.407,4.412)--(6.408,4.405)--(6.408,4.396)--(6.409,4.390)--(6.409,4.384)--(6.409,4.379)%
  --(6.410,4.373)--(6.410,4.364)--(6.411,4.354)--(6.411,4.348)--(6.411,4.342)--(6.412,4.335)%
  --(6.412,4.332)--(6.412,4.328)--(6.413,4.320)--(6.413,4.312)--(6.414,4.304)--(6.414,4.300)%
  --(6.414,4.291)--(6.415,4.283)--(6.415,4.280)--(6.416,4.275)--(6.416,4.262)--(6.416,4.248)%
  --(6.417,4.238)--(6.417,4.226)--(6.418,4.215)--(6.418,4.201)--(6.418,4.190)--(6.419,4.188)%
  --(6.419,4.186)--(6.420,4.179)--(6.420,4.173)--(6.421,4.164)--(6.421,4.157)--(6.421,4.150)%
  --(6.422,4.145)--(6.422,4.140)--(6.423,4.133)--(6.423,4.129)--(6.423,4.122)--(6.424,4.117)%
  --(6.424,4.115)--(6.424,4.113)--(6.425,4.111)--(6.425,4.105)--(6.426,4.103)--(6.426,4.099)%
  --(6.426,4.094)--(6.427,4.091)--(6.427,4.086)--(6.428,4.083)--(6.428,4.081)--(6.428,4.076)%
  --(6.429,4.068)--(6.429,4.065)--(6.430,4.062)--(6.430,4.056)--(6.430,4.051)--(6.431,4.045)%
  --(6.431,4.037)--(6.431,4.026)--(6.432,4.020)--(6.432,4.012)--(6.433,4.004)--(6.433,3.997)%
  --(6.433,3.991)--(6.434,3.989)--(6.434,3.984)--(6.435,3.980)--(6.435,3.978)--(6.435,3.973)%
  --(6.436,3.970)--(6.436,3.967)--(6.437,3.962)--(6.437,3.959)--(6.437,3.951)--(6.438,3.946)%
  --(6.438,3.931)--(6.438,3.924)--(6.439,3.920)--(6.439,3.916)--(6.440,3.914)--(6.440,3.907)%
  --(6.440,3.904)--(6.441,3.902)--(6.441,3.897)--(6.442,3.896)--(6.442,3.893)--(6.443,3.891)%
  --(6.443,3.889)--(6.444,3.886)--(6.444,3.884)--(6.444,3.886)--(6.445,3.884)--(6.445,3.882)%
  --(6.445,3.880)--(6.446,3.879)--(6.446,3.876)--(6.447,3.872)--(6.447,3.866)--(6.447,3.864)%
  --(6.448,3.863)--(6.448,3.860)--(6.449,3.856)--(6.449,3.850)--(6.449,3.843)--(6.450,3.833)%
  --(6.450,3.823)--(6.451,3.812)--(6.451,3.806)--(6.451,3.799)--(6.452,3.793)--(6.452,3.789)%
  --(6.452,3.781)--(6.453,3.774)--(6.453,3.767)--(6.454,3.761)--(6.454,3.754)--(6.454,3.749)%
  --(6.455,3.742)--(6.455,3.737)--(6.456,3.732)--(6.456,3.729)--(6.456,3.727)--(6.457,3.726)%
  --(6.457,3.723)--(6.457,3.719)--(6.458,3.715)--(6.458,3.711)--(6.459,3.710)--(6.459,3.708)%
  --(6.459,3.706)--(6.460,3.703)--(6.460,3.701)--(6.461,3.696)--(6.461,3.695)--(6.461,3.692)%
  --(6.462,3.689)--(6.462,3.685)--(6.463,3.684)--(6.463,3.680)--(6.463,3.679)--(6.464,3.676)%
  --(6.464,3.673)--(6.464,3.672)--(6.465,3.669)--(6.465,3.668)--(6.466,3.664)--(6.466,3.659)%
  --(6.467,3.655)--(6.467,3.650)--(6.468,3.646)--(6.468,3.643)--(6.468,3.641)--(6.469,3.635)%
  --(6.469,3.632)--(6.470,3.626)--(6.470,3.623)--(6.470,3.617)--(6.471,3.610)--(6.471,3.604)%
  --(6.471,3.598)--(6.472,3.592)--(6.472,3.589)--(6.473,3.583)--(6.473,3.576)--(6.473,3.572)%
  --(6.474,3.567)--(6.474,3.563)--(6.475,3.558)--(6.475,3.555)--(6.475,3.549)--(6.476,3.545)%
  --(6.476,3.544)--(6.477,3.540)--(6.477,3.538)--(6.477,3.535)--(6.478,3.532)--(6.478,3.527)%
  --(6.478,3.522)--(6.479,3.517)--(6.479,3.514)--(6.480,3.508)--(6.480,3.506)--(6.480,3.505)%
  --(6.481,3.502)--(6.481,3.499)--(6.482,3.496)--(6.482,3.492)--(6.482,3.489)--(6.483,3.486)%
  --(6.483,3.483)--(6.484,3.482)--(6.484,3.478)--(6.484,3.477)--(6.485,3.474)--(6.485,3.473)%
  --(6.485,3.472)--(6.486,3.467)--(6.486,3.462)--(6.487,3.458)--(6.487,3.451)--(6.487,3.450)%
  --(6.488,3.446)--(6.488,3.438)--(6.489,3.434)--(6.489,3.430)--(6.489,3.425)--(6.490,3.423)%
  --(6.490,3.416)--(6.491,3.412)--(6.491,3.405)--(6.491,3.396)--(6.492,3.387)--(6.492,3.382)%
  --(6.492,3.378)--(6.493,3.375)--(6.493,3.372)--(6.494,3.371)--(6.494,3.369)--(6.494,3.365)%
  --(6.495,3.361)--(6.495,3.357)--(6.496,3.356)--(6.496,3.352)--(6.496,3.347)--(6.497,3.344)%
  --(6.497,3.340)--(6.498,3.335)--(6.498,3.330)--(6.499,3.327)--(6.499,3.323)--(6.499,3.318)%
  --(6.500,3.315)--(6.500,3.312)--(6.501,3.308)--(6.501,3.304)--(6.501,3.299)--(6.502,3.292)%
  --(6.502,3.284)--(6.503,3.280)--(6.503,3.274)--(6.503,3.267)--(6.504,3.260)--(6.504,3.250)%
  --(6.504,3.237)--(6.505,3.228)--(6.505,3.223)--(6.506,3.217)--(6.506,3.211)--(6.506,3.204)%
  --(6.507,3.198)--(6.507,3.195)--(6.508,3.194)--(6.508,3.191)--(6.508,3.192)--(6.509,3.190)%
  --(6.509,3.191)--(6.510,3.191)--(6.510,3.189)--(6.511,3.187)--(6.511,3.185)--(6.512,3.183)%
  --(6.512,3.179)--(6.513,3.178)--(6.513,3.173)--(6.513,3.170)--(6.514,3.169)--(6.514,3.167)%
  --(6.515,3.169)--(6.515,3.164)--(6.515,3.160)--(6.516,3.158)--(6.516,3.155)--(6.517,3.152)%
  --(6.517,3.147)--(6.517,3.146)--(6.518,3.144)--(6.518,3.139)--(6.518,3.140)--(6.519,3.137)%
  --(6.519,3.136)--(6.520,3.133)--(6.520,3.131)--(6.520,3.129)--(6.521,3.125)--(6.521,3.121)%
  --(6.522,3.118)--(6.522,3.116)--(6.522,3.112)--(6.523,3.108)--(6.523,3.104)--(6.524,3.102)%
  --(6.524,3.099)--(6.524,3.098)--(6.525,3.097)--(6.525,3.096)--(6.526,3.094)--(6.526,3.091)%
  --(6.527,3.086)--(6.527,3.084)--(6.527,3.082)--(6.528,3.081)--(6.528,3.079)--(6.529,3.076)%
  --(6.529,3.073)--(6.529,3.071)--(6.530,3.070)--(6.530,3.066)--(6.530,3.062)--(6.531,3.062)%
  --(6.531,3.056)--(6.532,3.052)--(6.532,3.047)--(6.532,3.045)--(6.533,3.043)--(6.533,3.042)%
  --(6.534,3.038)--(6.534,3.035)--(6.534,3.033)--(6.535,3.028)--(6.535,3.026)--(6.536,3.023)%
  --(6.536,3.019)--(6.536,3.015)--(6.537,3.013)--(6.537,3.011)--(6.537,3.007)--(6.538,3.005)%
  --(6.538,3.001)--(6.539,3.001)--(6.539,3.000)--(6.539,2.997)--(6.540,2.995)--(6.540,2.992)%
  --(6.541,2.991)--(6.541,2.989)--(6.541,2.987)--(6.542,2.986)--(6.542,2.985)--(6.543,2.982)%
  --(6.543,2.981)--(6.543,2.980)--(6.544,2.979)--(6.544,2.977)--(6.544,2.975)--(6.545,2.974)%
  --(6.545,2.973)--(6.546,2.971)--(6.546,2.969)--(6.547,2.968)--(6.547,2.966)--(6.548,2.964)%
  --(6.548,2.963)--(6.548,2.961)--(6.549,2.959)--(6.549,2.956)--(6.550,2.955)--(6.550,2.952)%
  --(6.550,2.951)--(6.551,2.950)--(6.551,2.949)--(6.552,2.949)--(6.553,2.947)--(6.553,2.945)%
  --(6.553,2.940)--(6.554,2.933)--(6.554,2.930)--(6.555,2.919)--(6.555,2.914)--(6.555,2.913)%
  --(6.556,2.909)--(6.557,2.906)--(6.557,2.904)--(6.557,2.900)--(6.558,2.896)--(6.558,2.894)%
  --(6.558,2.890)--(6.559,2.887)--(6.559,2.882)--(6.560,2.877)--(6.560,2.876)--(6.560,2.872)%
  --(6.561,2.870)--(6.561,2.868)--(6.562,2.866)--(6.562,2.864)--(6.562,2.863)--(6.563,2.863)%
  --(6.563,2.862)--(6.563,2.861)--(6.564,2.860)--(6.565,2.857)--(6.565,2.854)--(6.566,2.853)%
  --(6.567,2.852)--(6.567,2.851)--(6.568,2.851)--(6.568,2.850)--(6.569,2.848)--(6.569,2.847)%
  --(6.570,2.845)--(6.570,2.844)--(6.570,2.843)--(6.571,2.841)--(6.572,2.837)--(6.572,2.836)%
  --(6.572,2.835)--(6.573,2.832)--(6.573,2.831)--(6.574,2.830)--(6.574,2.827)--(6.574,2.826)%
  --(6.575,2.823)--(6.575,2.820)--(6.576,2.819)--(6.576,2.816)--(6.576,2.813)--(6.577,2.811)%
  --(6.577,2.810)--(6.577,2.807)--(6.578,2.806)--(6.579,2.804)--(6.579,2.802)--(6.579,2.801)%
  --(6.580,2.798)--(6.580,2.795)--(6.581,2.794)--(6.581,2.791)--(6.581,2.789)--(6.582,2.785)%
  --(6.582,2.782)--(6.583,2.776)--(6.583,2.768)--(6.583,2.761)--(6.584,2.757)--(6.584,2.755)%
  --(6.584,2.754)--(6.585,2.754)--(6.585,2.753)--(6.586,2.752)--(6.587,2.750)--(6.587,2.748)%
  --(6.588,2.747)--(6.588,2.746)--(6.588,2.745)--(6.589,2.745)--(6.590,2.744)--(6.590,2.743)%
  --(6.590,2.741)--(6.591,2.740)--(6.591,2.738)--(6.592,2.737)--(6.592,2.735)--(6.593,2.734)%
  --(6.593,2.732)--(6.593,2.728)--(6.594,2.726)--(6.594,2.724)--(6.595,2.722)--(6.595,2.719)%
  --(6.596,2.719)--(6.597,2.719)--(6.597,2.716)--(6.597,2.714)--(6.598,2.714)--(6.598,2.713)%
  --(6.599,2.712)--(6.600,2.712)--(6.600,2.710)--(6.600,2.708)--(6.601,2.708)--(6.601,2.707)%
  --(6.602,2.705)--(6.602,2.706)--(6.602,2.704)--(6.603,2.703)--(6.603,2.698)--(6.604,2.695)%
  --(6.604,2.694)--(6.605,2.694)--(6.605,2.691)--(6.605,2.690)--(6.606,2.688)--(6.606,2.684)%
  --(6.607,2.682)--(6.607,2.680)--(6.607,2.677)--(6.608,2.675)--(6.608,2.673)--(6.609,2.670)%
  --(6.609,2.667)--(6.610,2.665)--(6.610,2.662)--(6.610,2.660)--(6.611,2.657)--(6.611,2.658)%
  --(6.612,2.654)--(6.612,2.653)--(6.612,2.650)--(6.613,2.647)--(6.613,2.643)--(6.614,2.640)%
  --(6.614,2.639)--(6.615,2.640)--(6.615,2.636)--(6.616,2.633)--(6.616,2.630)--(6.616,2.627)%
  --(6.617,2.625)--(6.617,2.623)--(6.617,2.620)--(6.618,2.619)--(6.618,2.618)--(6.619,2.618)%
  --(6.619,2.615)--(6.619,2.614)--(6.620,2.610)--(6.620,2.607)--(6.621,2.604)--(6.621,2.601)%
  --(6.621,2.598)--(6.622,2.595)--(6.622,2.594)--(6.623,2.589)--(6.623,2.588)--(6.623,2.585)%
  --(6.624,2.584)--(6.624,2.583)--(6.625,2.580)--(6.626,2.578)--(6.626,2.577)--(6.626,2.573)%
  --(6.627,2.573)--(6.627,2.570)--(6.628,2.570)--(6.628,2.569)--(6.628,2.567)--(6.629,2.567)%
  --(6.629,2.565)--(6.630,2.565)--(6.630,2.564)--(6.630,2.561)--(6.631,2.560)--(6.631,2.558)%
  --(6.631,2.554)--(6.632,2.552)--(6.632,2.549)--(6.633,2.549)--(6.633,2.547)--(6.633,2.546)%
  --(6.634,2.544)--(6.634,2.540)--(6.635,2.539)--(6.635,2.536)--(6.635,2.534)--(6.636,2.531)%
  --(6.636,2.530)--(6.636,2.528)--(6.637,2.525)--(6.637,2.523)--(6.638,2.523)--(6.638,2.521)%
  --(6.638,2.519)--(6.639,2.518)--(6.639,2.516)--(6.640,2.513)--(6.640,2.511)--(6.640,2.510)%
  --(6.641,2.509)--(6.641,2.507)--(6.642,2.506)--(6.642,2.505)--(6.642,2.504)--(6.643,2.501)%
  --(6.643,2.499)--(6.644,2.498)--(6.644,2.497)--(6.645,2.497)--(6.645,2.495)--(6.645,2.493)%
  --(6.646,2.490)--(6.646,2.489)--(6.647,2.487)--(6.647,2.485)--(6.647,2.483)--(6.648,2.483)%
  --(6.648,2.482)--(6.649,2.480)--(6.649,2.478)--(6.649,2.475)--(6.650,2.473)--(6.650,2.471)%
  --(6.651,2.469)--(6.651,2.468)--(6.652,2.466)--(6.653,2.464)--(6.653,2.463)--(6.654,2.461)%
  --(6.654,2.460)--(6.654,2.458)--(6.655,2.458)--(6.655,2.457)--(6.656,2.454)--(6.656,2.452)%
  --(6.657,2.450)--(6.657,2.448)--(6.658,2.448)--(6.658,2.447)--(6.659,2.446)--(6.659,2.444)%
  --(6.659,2.443)--(6.660,2.443)--(6.660,2.441)--(6.661,2.439)--(6.661,2.437)--(6.661,2.435)%
  --(6.662,2.434)--(6.662,2.433)--(6.663,2.433)--(6.663,2.432)--(6.663,2.428)--(6.664,2.426)%
  --(6.664,2.425)--(6.664,2.424)--(6.665,2.423)--(6.666,2.423)--(6.666,2.421)--(6.667,2.420)%
  --(6.668,2.419)--(6.668,2.417)--(6.668,2.415)--(6.669,2.414)--(6.669,2.413)--(6.669,2.410)%
  --(6.670,2.410)--(6.670,2.409)--(6.671,2.409)--(6.671,2.408)--(6.671,2.407)--(6.672,2.406)%
  --(6.672,2.404)--(6.673,2.400)--(6.673,2.398)--(6.674,2.398)--(6.674,2.397)--(6.675,2.394)%
  --(6.675,2.393)--(6.676,2.391)--(6.676,2.390)--(6.677,2.388)--(6.677,2.387)--(6.678,2.388)%
  --(6.678,2.387)--(6.678,2.385)--(6.679,2.384)--(6.679,2.383)--(6.680,2.381)--(6.680,2.380)%
  --(6.680,2.378)--(6.681,2.377)--(6.681,2.375)--(6.682,2.375)--(6.682,2.371)--(6.682,2.369)%
  --(6.683,2.368)--(6.683,2.367)--(6.683,2.366)--(6.684,2.363)--(6.685,2.361)--(6.685,2.360)%
  --(6.685,2.358)--(6.686,2.356)--(6.686,2.355)--(6.687,2.353)--(6.687,2.351)--(6.687,2.350)%
  --(6.688,2.350)--(6.688,2.348)--(6.689,2.348)--(6.689,2.347)--(6.689,2.345)--(6.690,2.344)%
  --(6.690,2.343)--(6.691,2.342)--(6.691,2.340)--(6.692,2.340)--(6.692,2.338)--(6.693,2.335)%
  --(6.693,2.333)--(6.694,2.331)--(6.694,2.329)--(6.694,2.328)--(6.695,2.327)--(6.695,2.325)%
  --(6.696,2.324)--(6.696,2.322)--(6.697,2.321)--(6.697,2.319)--(6.698,2.317)--(6.698,2.316)%
  --(6.699,2.315)--(6.699,2.314)--(6.699,2.311)--(6.700,2.310)--(6.700,2.307)--(6.701,2.305)%
  --(6.701,2.304)--(6.701,2.305)--(6.702,2.305)--(6.702,2.303)--(6.703,2.302)--(6.703,2.301)%
  --(6.704,2.300)--(6.704,2.299)--(6.705,2.299)--(6.705,2.297)--(6.706,2.297)--(6.706,2.295)%
  --(6.707,2.294)--(6.707,2.296)--(6.708,2.295)--(6.708,2.294)--(6.709,2.293)--(6.709,2.289)%
  --(6.710,2.287)--(6.710,2.286)--(6.711,2.282)--(6.711,2.283)--(6.711,2.281)--(6.712,2.279)%
  --(6.712,2.277)--(6.713,2.276)--(6.713,2.275)--(6.713,2.274)--(6.714,2.271)--(6.714,2.270)%
  --(6.715,2.270)--(6.715,2.268)--(6.715,2.267)--(6.716,2.267)--(6.716,2.265)--(6.716,2.263)%
  --(6.717,2.261)--(6.717,2.259)--(6.718,2.258)--(6.718,2.256)--(6.718,2.255)--(6.719,2.251)%
  --(6.719,2.248)--(6.720,2.246)--(6.720,2.245)--(6.721,2.245)--(6.721,2.244)--(6.722,2.244)%
  --(6.722,2.243)--(6.722,2.242)--(6.723,2.242)--(6.723,2.240)--(6.724,2.238)--(6.724,2.236)%
  --(6.725,2.233)--(6.725,2.231)--(6.726,2.232)--(6.726,2.231)--(6.727,2.229)--(6.727,2.228)%
  --(6.727,2.226)--(6.728,2.226)--(6.728,2.225)--(6.729,2.224)--(6.729,2.223)--(6.730,2.222)%
  --(6.730,2.220)--(6.730,2.219)--(6.731,2.217)--(6.731,2.216)--(6.732,2.214)--(6.732,2.213)%
  --(6.733,2.213)--(6.733,2.212)--(6.734,2.212)--(6.734,2.211)--(6.735,2.210)--(6.735,2.208)%
  --(6.736,2.208)--(6.736,2.206)--(6.736,2.204)--(6.737,2.203)--(6.737,2.202)--(6.737,2.201)%
  --(6.738,2.199)--(6.739,2.195)--(6.739,2.194)--(6.739,2.193)--(6.740,2.191)--(6.740,2.190)%
  --(6.741,2.189)--(6.741,2.187)--(6.741,2.186)--(6.742,2.185)--(6.742,2.183)--(6.743,2.182)%
  --(6.743,2.181)--(6.744,2.179)--(6.744,2.177)--(6.745,2.177)--(6.745,2.174)--(6.746,2.173)%
  --(6.746,2.172)--(6.746,2.171)--(6.747,2.171)--(6.747,2.169)--(6.748,2.167)--(6.748,2.166)%
  --(6.749,2.164)--(6.749,2.163)--(6.749,2.162)--(6.750,2.162)--(6.751,2.162)--(6.751,2.160)%
  --(6.752,2.159)--(6.753,2.155)--(6.753,2.154)--(6.753,2.153)--(6.754,2.152)--(6.755,2.152)%
  --(6.755,2.150)--(6.756,2.149)--(6.756,2.148)--(6.757,2.147)--(6.758,2.145)--(6.758,2.144)%
  --(6.758,2.142)--(6.759,2.142)--(6.759,2.143)--(6.760,2.143)--(6.760,2.141)--(6.760,2.140)%
  --(6.761,2.140)--(6.762,2.139)--(6.763,2.139)--(6.763,2.138)--(6.764,2.137)--(6.764,2.136)%
  --(6.765,2.136)--(6.765,2.134)--(6.766,2.133)--(6.767,2.132)--(6.767,2.131)--(6.767,2.130)%
  --(6.768,2.128)--(6.768,2.126)--(6.769,2.126)--(6.769,2.124)--(6.769,2.123)--(6.770,2.122)%
  --(6.771,2.120)--(6.772,2.120)--(6.772,2.118)--(6.772,2.116)--(6.773,2.114)--(6.773,2.112)%
  --(6.774,2.111)--(6.774,2.109)--(6.775,2.108)--(6.775,2.106)--(6.775,2.105)--(6.776,2.104)%
  --(6.776,2.103)--(6.777,2.102)--(6.777,2.100)--(6.778,2.098)--(6.778,2.096)--(6.779,2.096)%
  --(6.779,2.093)--(6.779,2.092)--(6.780,2.090)--(6.780,2.089)--(6.781,2.087)--(6.781,2.086)%
  --(6.781,2.085)--(6.782,2.083)--(6.782,2.082)--(6.782,2.079)--(6.783,2.078)--(6.783,2.077)%
  --(6.784,2.075)--(6.784,2.073)--(6.785,2.072)--(6.785,2.070)--(6.786,2.069)--(6.786,2.067)%
  --(6.786,2.064)--(6.787,2.063)--(6.787,2.061)--(6.788,2.061)--(6.788,2.060)--(6.789,2.058)%
  --(6.789,2.055)--(6.790,2.054)--(6.791,2.051)--(6.791,2.050)--(6.791,2.049)--(6.792,2.048)%
  --(6.792,2.047)--(6.793,2.046)--(6.793,2.043)--(6.793,2.041)--(6.794,2.041)--(6.794,2.040)%
  --(6.795,2.039)--(6.795,2.037)--(6.795,2.035)--(6.796,2.035)--(6.796,2.033)--(6.797,2.032)%
  --(6.797,2.030)--(6.798,2.029)--(6.798,2.027)--(6.798,2.026)--(6.799,2.024)--(6.800,2.024)%
  --(6.800,2.022)--(6.800,2.020)--(6.801,2.019)--(6.802,2.017)--(6.802,2.015)--(6.803,2.015)%
  --(6.803,2.014)--(6.803,2.012)--(6.804,2.012)--(6.805,2.010)--(6.805,2.009)--(6.806,2.008)%
  --(6.806,2.007)--(6.807,2.005)--(6.807,2.003)--(6.808,2.002)--(6.809,1.999)--(6.809,1.998)%
  --(6.810,1.995)--(6.810,1.992)--(6.810,1.990)--(6.811,1.987)--(6.812,1.987)--(6.812,1.984)%
  --(6.813,1.983)--(6.814,1.983)--(6.814,1.981)--(6.814,1.980)--(6.815,1.980)--(6.815,1.978)%
  --(6.815,1.977)--(6.816,1.975)--(6.817,1.974)--(6.817,1.973)--(6.817,1.972)--(6.818,1.971)%
  --(6.819,1.970)--(6.819,1.968)--(6.819,1.966)--(6.820,1.965)--(6.821,1.963)--(6.821,1.962)%
  --(6.821,1.960)--(6.822,1.960)--(6.822,1.959)--(6.822,1.958)--(6.823,1.957)--(6.823,1.956)%
  --(6.824,1.954)--(6.824,1.952)--(6.825,1.952)--(6.825,1.950)--(6.826,1.950)--(6.826,1.949)%
  --(6.826,1.948)--(6.827,1.947)--(6.827,1.945)--(6.828,1.944)--(6.828,1.942)--(6.828,1.940)%
  --(6.829,1.939)--(6.829,1.938)--(6.830,1.936)--(6.830,1.935)--(6.831,1.933)--(6.831,1.932)%
  --(6.831,1.931)--(6.832,1.928)--(6.832,1.927)--(6.833,1.924)--(6.833,1.923)--(6.834,1.920)%
  --(6.834,1.919)--(6.835,1.916)--(6.835,1.914)--(6.835,1.911)--(6.836,1.911)--(6.836,1.909)%
  --(6.837,1.908)--(6.838,1.907)--(6.838,1.905)--(6.838,1.904)--(6.839,1.903)--(6.839,1.902)%
  --(6.840,1.902)--(6.840,1.900)--(6.841,1.898)--(6.842,1.897)--(6.842,1.896)--(6.843,1.895)%
  --(6.843,1.893)--(6.844,1.893)--(6.844,1.891)--(6.845,1.891)--(6.846,1.887)--(6.847,1.885)%
  --(6.848,1.884)--(6.848,1.883)--(6.849,1.880)--(6.850,1.880)--(6.850,1.879)--(6.850,1.878)%
  --(6.851,1.878)--(6.851,1.876)--(6.852,1.876)--(6.852,1.872)--(6.852,1.871)--(6.853,1.869)%
  --(6.853,1.868)--(6.854,1.867)--(6.854,1.865)--(6.855,1.863)--(6.855,1.862)--(6.855,1.859)%
  --(6.856,1.857)--(6.857,1.856)--(6.857,1.855)--(6.858,1.853)--(6.858,1.852)--(6.859,1.851)%
  --(6.859,1.850)--(6.859,1.849)--(6.860,1.848)--(6.860,1.846)--(6.861,1.846)--(6.861,1.844)%
  --(6.861,1.841)--(6.862,1.840)--(6.862,1.839)--(6.862,1.838)--(6.863,1.836)--(6.863,1.835)%
  --(6.864,1.834)--(6.864,1.832)--(6.864,1.830)--(6.865,1.829)--(6.865,1.828)--(6.866,1.826)%
  --(6.866,1.824)--(6.866,1.822)--(6.867,1.821)--(6.867,1.819)--(6.868,1.819)--(6.869,1.819)%
  --(6.869,1.818)--(6.870,1.816)--(6.870,1.814)--(6.871,1.813)--(6.871,1.812)--(6.872,1.812)%
  --(6.872,1.809)--(6.873,1.808)--(6.873,1.807)--(6.874,1.806)--(6.874,1.805)--(6.875,1.804)%
  --(6.875,1.802)--(6.876,1.800)--(6.876,1.799)--(6.876,1.798)--(6.877,1.797)--(6.878,1.795)%
  --(6.878,1.793)--(6.878,1.792)--(6.879,1.792)--(6.879,1.790)--(6.880,1.788)--(6.880,1.787)%
  --(6.880,1.786)--(6.881,1.785)--(6.881,1.784)--(6.881,1.782)--(6.882,1.783)--(6.882,1.780)%
  --(6.883,1.779)--(6.883,1.777)--(6.883,1.776)--(6.884,1.775)--(6.884,1.774)--(6.885,1.774)%
  --(6.885,1.773)--(6.886,1.771)--(6.886,1.770)--(6.887,1.769)--(6.887,1.768)--(6.887,1.766)%
  --(6.888,1.766)--(6.888,1.764)--(6.888,1.765)--(6.889,1.764)--(6.889,1.762)--(6.890,1.762)%
  --(6.890,1.761)--(6.891,1.760)--(6.891,1.759)--(6.892,1.758)--(6.892,1.757)--(6.893,1.756)%
  --(6.893,1.755)--(6.894,1.755)--(6.894,1.754)--(6.894,1.753)--(6.895,1.752)--(6.895,1.751)%
  --(6.895,1.750)--(6.896,1.749)--(6.896,1.748)--(6.897,1.747)--(6.897,1.746)--(6.897,1.744)%
  --(6.898,1.742)--(6.899,1.741)--(6.899,1.738)--(6.899,1.737)--(6.900,1.737)--(6.900,1.735)%
  --(6.901,1.735)--(6.901,1.733)--(6.901,1.731)--(6.902,1.730)--(6.902,1.728)--(6.903,1.726)%
  --(6.903,1.725)--(6.904,1.724)--(6.904,1.722)--(6.905,1.721)--(6.905,1.719)--(6.906,1.719)%
  --(6.906,1.717)--(6.907,1.716)--(6.907,1.715)--(6.908,1.714)--(6.908,1.713)--(6.908,1.711)%
  --(6.909,1.710)--(6.909,1.709)--(6.909,1.708)--(6.910,1.706)--(6.910,1.705)--(6.911,1.704)%
  --(6.911,1.703)--(6.911,1.702)--(6.912,1.701)--(6.912,1.699)--(6.913,1.699)--(6.913,1.698)%
  --(6.913,1.696)--(6.914,1.696)--(6.914,1.694)--(6.915,1.693)--(6.915,1.691)--(6.916,1.689)%
  --(6.916,1.688)--(6.917,1.686)--(6.917,1.684)--(6.918,1.684)--(6.918,1.683)--(6.919,1.683)%
  --(6.920,1.683)--(6.920,1.681)--(6.920,1.680)--(6.921,1.679)--(6.921,1.678)--(6.921,1.677)%
  --(6.922,1.677)--(6.922,1.676)--(6.923,1.675)--(6.923,1.674)--(6.923,1.672)--(6.924,1.671)%
  --(6.924,1.669)--(6.925,1.668)--(6.925,1.666)--(6.926,1.665)--(6.926,1.662)--(6.927,1.662)%
  --(6.927,1.661)--(6.928,1.660)--(6.928,1.659)--(6.929,1.657)--(6.930,1.655)--(6.930,1.654)%
  --(6.930,1.652)--(6.931,1.651)--(6.931,1.650)--(6.932,1.650)--(6.932,1.649)--(6.933,1.649)%
  --(6.933,1.647)--(6.934,1.646)--(6.934,1.645)--(6.935,1.645)--(6.935,1.643)--(6.935,1.641)%
  --(6.936,1.640)--(6.936,1.639)--(6.937,1.638)--(6.937,1.637)--(6.937,1.636)--(6.938,1.636)%
  --(6.938,1.634)--(6.939,1.632)--(6.939,1.630)--(6.940,1.629)--(6.940,1.628)--(6.941,1.627)%
  --(6.941,1.625)--(6.942,1.624)--(6.942,1.622)--(6.943,1.621)--(6.943,1.620)--(6.944,1.619)%
  --(6.944,1.616)--(6.945,1.615)--(6.945,1.614)--(6.946,1.614)--(6.946,1.613)--(6.946,1.612)%
  --(6.947,1.609)--(6.948,1.608)--(6.948,1.607)--(6.948,1.606)--(6.949,1.604)--(6.949,1.602)%
  --(6.950,1.602)--(6.950,1.601)--(6.951,1.599)--(6.952,1.598)--(6.952,1.596)--(6.953,1.595)%
  --(6.953,1.594)--(6.953,1.593)--(6.954,1.592)--(6.954,1.591)--(6.955,1.590)--(6.955,1.589)%
  --(6.956,1.588)--(6.956,1.586)--(6.957,1.585)--(6.958,1.583)--(6.958,1.582)--(6.959,1.582)%
  --(6.959,1.581)--(6.960,1.580)--(6.960,1.578)--(6.961,1.577)--(6.961,1.576)--(6.962,1.576)%
  --(6.963,1.575)--(6.963,1.574)--(6.964,1.573)--(6.964,1.572)--(6.965,1.571)--(6.965,1.570)%
  --(6.966,1.570)--(6.967,1.569)--(6.967,1.567)--(6.967,1.566)--(6.968,1.566)--(6.968,1.564)%
  --(6.968,1.563)--(6.969,1.562)--(6.970,1.561)--(6.970,1.560)--(6.971,1.560)--(6.971,1.558)%
  --(6.972,1.557)--(6.972,1.556)--(6.973,1.556)--(6.973,1.554)--(6.974,1.553)--(6.974,1.552)%
  --(6.975,1.551)--(6.975,1.550)--(6.976,1.548)--(6.977,1.547)--(6.977,1.546)--(6.978,1.546)%
  --(6.978,1.557)--(6.979,1.557)--(6.979,1.558)--(6.979,1.559)--(6.980,1.560)--(6.981,1.561)%
  --(6.981,1.562)--(6.981,1.564)--(6.982,1.565)--(6.983,1.565)--(6.984,1.566)--(6.984,1.567)%
  --(6.984,1.568)--(6.985,1.570)--(6.985,1.571)--(6.986,1.572)--(6.986,1.574)--(6.986,1.575)%
  --(6.987,1.577)--(6.987,1.578)--(6.987,1.579)--(6.988,1.580)--(6.988,1.581)--(6.989,1.584)%
  --(6.989,1.586)--(6.989,1.587)--(6.990,1.588)--(6.990,1.590)--(6.991,1.590)--(6.991,1.591)%
  --(6.991,1.592)--(6.992,1.593)--(6.992,1.594)--(6.993,1.595)--(6.993,1.596)--(6.994,1.597)%
  --(6.994,1.598)--(6.994,1.600)--(6.995,1.602)--(6.996,1.602)--(6.996,1.603)--(6.997,1.604)%
  --(6.998,1.603)--(6.999,1.604)--(6.999,1.605)--(7.000,1.605)--(7.000,1.606)--(7.000,1.607)%
  --(7.001,1.608)--(7.001,1.609)--(7.001,1.610)--(7.002,1.612)--(7.002,1.613)--(7.003,1.615)%
  --(7.003,1.616)--(7.004,1.617)--(7.004,1.616)--(7.005,1.617)--(7.005,1.618)--(7.005,1.620)%
  --(7.006,1.620)--(7.006,1.621)--(7.007,1.622)--(7.007,1.623)--(7.007,1.622)--(7.008,1.623)%
  --(7.008,1.625)--(7.008,1.627)--(7.009,1.628)--(7.010,1.630)--(7.010,1.631)--(7.011,1.632)%
  --(7.011,1.633)--(7.012,1.635)--(7.012,1.636)--(7.012,1.637)--(7.013,1.637)--(7.013,1.636)%
  --(7.014,1.637)--(7.014,1.639)--(7.015,1.639)--(7.015,1.640)--(7.015,1.642)--(7.016,1.645)%
  --(7.016,1.647)--(7.017,1.649)--(7.017,1.650)--(7.018,1.651)--(7.018,1.654)--(7.019,1.655)%
  --(7.020,1.657)--(7.021,1.658)--(7.021,1.657)--(7.022,1.659)--(7.022,1.658)--(7.023,1.659)%
  --(7.023,1.660)--(7.024,1.662)--(7.024,1.664)--(7.025,1.665)--(7.026,1.667)--(7.026,1.668)%
  --(7.027,1.668)--(7.027,1.669)--(7.027,1.671)--(7.028,1.672)--(7.028,1.673)--(7.029,1.674)%
  --(7.029,1.676)--(7.030,1.676)--(7.031,1.676)--(7.031,1.677)--(7.032,1.677)--(7.032,1.678)%
  --(7.033,1.679)--(7.033,1.680)--(7.033,1.681)--(7.034,1.680)--(7.034,1.681)--(7.035,1.680)%
  --(7.036,1.679)--(7.036,1.680)--(7.037,1.680)--(7.037,1.681)--(7.038,1.682)--(7.039,1.682)%
  --(7.039,1.683)--(7.040,1.683)--(7.040,1.684)--(7.040,1.685)--(7.041,1.685)--(7.042,1.685)%
  --(7.042,1.686)--(7.043,1.686)--(7.043,1.685)--(7.044,1.686)--(7.044,1.687)--(7.045,1.687)%
  --(7.045,1.688)--(7.045,1.687)--(7.046,1.688)--(7.046,1.689)--(7.047,1.688)--(7.047,1.689)%
  --(7.047,1.690)--(7.048,1.690)--(7.048,1.691)--(7.049,1.691)--(7.049,1.692)--(7.050,1.691)%
  --(7.050,1.692)--(7.050,1.693)--(7.051,1.694)--(7.051,1.695)--(7.052,1.697)--(7.052,1.698)%
  --(7.053,1.699)--(7.054,1.699)--(7.054,1.700)--(7.055,1.700)--(7.055,1.702)--(7.055,1.703)%
  --(7.056,1.704)--(7.056,1.703)--(7.057,1.704)--(7.057,1.705)--(7.058,1.706)--(7.059,1.706)%
  --(7.059,1.707)--(7.060,1.707)--(7.060,1.706)--(7.060,1.707)--(7.061,1.707)--(7.061,1.708)%
  --(7.062,1.709)--(7.062,1.710)--(7.062,1.711)--(7.063,1.711)--(7.063,1.712)--(7.064,1.712)%
  --(7.064,1.714)--(7.065,1.716)--(7.065,1.717)--(7.066,1.716)--(7.066,1.717)--(7.066,1.718)%
  --(7.067,1.719)--(7.067,1.720)--(7.067,1.721)--(7.068,1.721)--(7.068,1.722)--(7.069,1.723)%
  --(7.069,1.724)--(7.070,1.725)--(7.070,1.726)--(7.071,1.727)--(7.072,1.728)--(7.072,1.729)%
  --(7.073,1.728)--(7.073,1.729)--(7.073,1.730)--(7.074,1.731)--(7.074,1.730)--(7.074,1.731)%
  --(7.075,1.732)--(7.076,1.732)--(7.076,1.734)--(7.076,1.735)--(7.077,1.737)--(7.077,1.739)%
  --(7.078,1.741)--(7.078,1.743)--(7.078,1.744)--(7.079,1.745)--(7.079,1.744)--(7.080,1.744)%
  --(7.080,1.745)--(7.081,1.746)--(7.081,1.747)--(7.081,1.748)--(7.082,1.749)--(7.082,1.750)%
  --(7.083,1.751)--(7.084,1.751)--(7.084,1.752)--(7.085,1.753)--(7.086,1.754)--(7.086,1.755)%
  --(7.087,1.756)--(7.087,1.755)--(7.087,1.756)--(7.088,1.757)--(7.088,1.758)--(7.089,1.758)%
  --(7.090,1.758)--(7.090,1.759)--(7.090,1.761)--(7.091,1.762)--(7.092,1.764)--(7.092,1.767)%
  --(7.092,1.768)--(7.093,1.768)--(7.093,1.770)--(7.094,1.770)--(7.094,1.771)--(7.095,1.770)%
  --(7.095,1.769)--(7.096,1.768)--(7.097,1.768)--(7.097,1.769)--(7.098,1.769)--(7.098,1.770)%
  --(7.099,1.774)--(7.099,1.775)--(7.099,1.776)--(7.100,1.776)--(7.100,1.777)--(7.101,1.777)%
  --(7.102,1.779)--(7.102,1.778)--(7.103,1.778)--(7.104,1.777)--(7.105,1.776)--(7.106,1.776)%
  --(7.106,1.775)--(7.107,1.776)--(7.108,1.776)--(7.108,1.777)--(7.109,1.776)--(7.109,1.777)%
  --(7.109,1.778)--(7.110,1.779)--(7.110,1.781)--(7.111,1.783)--(7.111,1.785)--(7.111,1.786)%
  --(7.112,1.787)--(7.113,1.788)--(7.113,1.789)--(7.114,1.791)--(7.114,1.792)--(7.115,1.793)%
  --(7.116,1.794)--(7.117,1.795)--(7.118,1.796)--(7.118,1.797)--(7.118,1.796)--(7.119,1.796)%
  --(7.119,1.797)--(7.120,1.796)--(7.120,1.797)--(7.120,1.796)--(7.121,1.796)--(7.121,1.797)%
  --(7.121,1.798)--(7.122,1.797)--(7.122,1.798)--(7.123,1.798)--(7.123,1.797)--(7.124,1.797)%
  --(7.125,1.798)--(7.125,1.799)--(7.126,1.800)--(7.126,1.799)--(7.127,1.799)--(7.127,1.800)%
  --(7.128,1.800)--(7.128,1.801)--(7.129,1.803)--(7.129,1.802)--(7.130,1.811)--(7.130,1.813)%
  --(7.130,1.815)--(7.131,1.816)--(7.131,1.818)--(7.132,1.819)--(7.132,1.820)--(7.132,1.822)%
  --(7.133,1.823)--(7.133,1.825)--(7.133,1.826)--(7.134,1.827)--(7.134,1.828)--(7.135,1.829)%
  --(7.135,1.831)--(7.136,1.833)--(7.136,1.834)--(7.137,1.835)--(7.138,1.836)--(7.138,1.838)%
  --(7.139,1.839)--(7.139,1.840)--(7.139,1.842)--(7.140,1.843)--(7.140,1.844)--(7.140,1.845)%
  --(7.141,1.845)--(7.141,1.846)--(7.142,1.846)--(7.142,1.847)--(7.142,1.848)--(7.143,1.848)%
  --(7.143,1.850)--(7.144,1.850)--(7.144,1.851)--(7.145,1.851)--(7.146,1.851)--(7.146,1.852)%
  --(7.146,1.853)--(7.147,1.854)--(7.147,1.855)--(7.147,1.856)--(7.148,1.857)--(7.148,1.859)%
  --(7.149,1.859)--(7.149,1.861)--(7.149,1.862)--(7.150,1.863)--(7.150,1.865)--(7.151,1.866)%
  --(7.151,1.868)--(7.151,1.869)--(7.152,1.873)--(7.152,1.876)--(7.153,1.878)--(7.154,1.879)%
  --(7.154,1.881)--(7.155,1.882)--(7.155,1.883)--(7.156,1.884)--(7.156,1.882)--(7.156,1.884)%
  --(7.157,1.885)--(7.157,1.886)--(7.158,1.887)--(7.158,1.889)--(7.159,1.889)--(7.159,1.890)%
  --(7.160,1.890)--(7.160,1.891)--(7.160,1.893)--(7.161,1.894)--(7.161,1.895)--(7.161,1.898)%
  --(7.162,1.899)--(7.162,1.901)--(7.163,1.903)--(7.163,1.904)--(7.163,1.906)--(7.164,1.909)%
  --(7.164,1.912)--(7.165,1.916)--(7.165,1.921)--(7.165,1.925)--(7.166,1.926)--(7.166,1.930)%
  --(7.166,1.931)--(7.167,1.932)--(7.167,1.933)--(7.168,1.932)--(7.168,1.934)--(7.169,1.934)%
  --(7.169,1.935)--(7.170,1.935)--(7.170,1.936)--(7.171,1.937)--(7.172,1.937)--(7.172,1.939)%
  --(7.173,1.941)--(7.173,1.942)--(7.173,1.943)--(7.174,1.943)--(7.174,1.944)--(7.175,1.944)%
  --(7.175,1.946)--(7.176,1.947)--(7.176,1.949)--(7.177,1.950)--(7.177,1.951)--(7.177,1.952)%
  --(7.178,1.952)--(7.178,1.953)--(7.179,1.953)--(7.179,1.955)--(7.180,1.956)--(7.180,1.959)%
  --(7.181,1.961)--(7.182,1.963)--(7.182,1.965)--(7.182,1.966)--(7.183,1.967)--(7.183,1.969)%
  --(7.184,1.970)--(7.184,1.972)--(7.184,1.974)--(7.185,1.974)--(7.185,1.975)--(7.186,1.976)%
  --(7.186,1.978)--(7.187,1.978)--(7.187,1.980)--(7.187,1.981)--(7.188,1.981)--(7.188,1.982)%
  --(7.189,1.983)--(7.189,1.985)--(7.189,1.987)--(7.190,1.987)--(7.190,1.989)--(7.191,1.989)%
  --(7.191,1.990)--(7.191,1.991)--(7.192,1.992)--(7.193,1.994)--(7.194,1.996)--(7.194,1.998)%
  --(7.195,2.000)--(7.195,2.001)--(7.196,2.000)--(7.196,2.001)--(7.196,2.002)--(7.197,2.003)%
  --(7.197,2.005)--(7.198,2.005)--(7.198,2.006)--(7.198,2.009)--(7.199,2.010)--(7.200,2.012)%
  --(7.200,2.014)--(7.201,2.016)--(7.201,2.018)--(7.202,2.019)--(7.202,2.021)--(7.203,2.020)%
  --(7.203,2.021)--(7.203,2.022)--(7.204,2.023)--(7.205,2.025)--(7.205,2.026)--(7.205,2.027)%
  --(7.206,2.028)--(7.206,2.029)--(7.206,2.031)--(7.207,2.033)--(7.207,2.034)--(7.208,2.035)%
  --(7.208,2.037)--(7.208,2.038)--(7.209,2.038)--(7.209,2.040)--(7.210,2.043)--(7.210,2.044)%
  --(7.210,2.046)--(7.211,2.047)--(7.211,2.050)--(7.212,2.051)--(7.212,2.053)--(7.212,2.054)%
  --(7.213,2.056)--(7.213,2.057)--(7.213,2.058)--(7.214,2.059)--(7.214,2.060)--(7.215,2.062)%
  --(7.215,2.064)--(7.216,2.066)--(7.216,2.068)--(7.217,2.068)--(7.217,2.070)--(7.218,2.071)%
  --(7.218,2.073)--(7.219,2.073)--(7.219,2.074)--(7.219,2.075)--(7.220,2.077)--(7.220,2.079)%
  --(7.220,2.080)--(7.221,2.083)--(7.222,2.085)--(7.222,2.087)--(7.222,2.089)--(7.223,2.090)%
  --(7.223,2.091)--(7.224,2.092)--(7.224,2.094)--(7.224,2.097)--(7.225,2.096)--(7.225,2.098)%
  --(7.226,2.099)--(7.226,2.101)--(7.226,2.103)--(7.227,2.104)--(7.227,2.107)--(7.227,2.108)%
  --(7.228,2.109)--(7.228,2.112)--(7.229,2.112)--(7.229,2.114)--(7.229,2.115)--(7.230,2.115)%
  --(7.230,2.117)--(7.231,2.118)--(7.231,2.119)--(7.232,2.120)--(7.232,2.123)--(7.233,2.122)%
  --(7.233,2.124)--(7.234,2.125)--(7.234,2.126)--(7.234,2.127)--(7.235,2.130)--(7.235,2.132)%
  --(7.236,2.136)--(7.236,2.137)--(7.237,2.138)--(7.238,2.139)--(7.238,2.140)--(7.239,2.141)%
  --(7.239,2.142)--(7.239,2.143)--(7.240,2.143)--(7.240,2.145)--(7.241,2.146)--(7.241,2.145)%
  --(7.241,2.148)--(7.242,2.150)--(7.242,2.151)--(7.243,2.151)--(7.243,2.154)--(7.243,2.156)%
  --(7.244,2.157)--(7.244,2.158)--(7.245,2.159)--(7.245,2.161)--(7.245,2.163)--(7.246,2.165)%
  --(7.246,2.166)--(7.246,2.168)--(7.247,2.169)--(7.247,2.171)--(7.248,2.174)--(7.248,2.175)%
  --(7.248,2.177)--(7.249,2.178)--(7.250,2.180)--(7.250,2.182)--(7.250,2.184)--(7.251,2.184)%
  --(7.251,2.185)--(7.252,2.187)--(7.252,2.186)--(7.252,2.187)--(7.253,2.187)--(7.253,2.189)%
  --(7.254,2.190)--(7.254,2.191)--(7.255,2.191)--(7.255,2.193)--(7.255,2.195)--(7.256,2.196)%
  --(7.257,2.196)--(7.257,2.198)--(7.257,2.199)--(7.258,2.201)--(7.258,2.202)--(7.259,2.202)%
  --(7.259,2.203)--(7.260,2.203)--(7.260,2.205)--(7.261,2.206)--(7.261,2.208)--(7.262,2.208)%
  --(7.262,2.209)--(7.262,2.212)--(7.263,2.213)--(7.263,2.214)--(7.264,2.215)--(7.264,2.217)%
  --(7.264,2.220)--(7.265,2.221)--(7.265,2.222)--(7.266,2.225)--(7.266,2.226)--(7.266,2.229)%
  --(7.267,2.229)--(7.267,2.230)--(7.267,2.232)--(7.268,2.233)--(7.269,2.234)--(7.269,2.237)%
  --(7.270,2.238)--(7.270,2.240)--(7.271,2.240)--(7.271,2.242)--(7.271,2.244)--(7.272,2.244)%
  --(7.272,2.246)--(7.272,2.248)--(7.273,2.249)--(7.273,2.250)--(7.274,2.250)--(7.274,2.251)%
  --(7.274,2.253)--(7.275,2.254)--(7.275,2.256)--(7.276,2.258)--(7.276,2.259)--(7.276,2.262)%
  --(7.277,2.262)--(7.278,2.264)--(7.278,2.263)--(7.278,2.264)--(7.279,2.265)--(7.279,2.267)%
  --(7.279,2.269)--(7.280,2.270)--(7.280,2.272)--(7.281,2.273)--(7.281,2.275)--(7.281,2.277)%
  --(7.282,2.279)--(7.282,2.281)--(7.283,2.281)--(7.283,2.282)--(7.283,2.285)--(7.284,2.287)%
  --(7.284,2.288)--(7.285,2.289)--(7.285,2.291)--(7.285,2.292)--(7.286,2.293)--(7.286,2.294)%
  --(7.286,2.296)--(7.287,2.298)--(7.287,2.300)--(7.288,2.301)--(7.288,2.303)--(7.289,2.302)%
  --(7.289,2.304)--(7.290,2.304)--(7.290,2.305)--(7.290,2.309)--(7.291,2.310)--(7.291,2.312)%
  --(7.292,2.315)--(7.292,2.317)--(7.292,2.320)--(7.293,2.322)--(7.293,2.325)--(7.293,2.327)%
  --(7.294,2.328)--(7.294,2.330)--(7.295,2.333)--(7.295,2.334)--(7.295,2.336)--(7.296,2.336)%
  --(7.296,2.338)--(7.297,2.342)--(7.297,2.343)--(7.298,2.345)--(7.298,2.347)--(7.299,2.349)%
  --(7.299,2.351)--(7.299,2.352)--(7.300,2.356)--(7.300,2.358)--(7.300,2.359)--(7.301,2.361)%
  --(7.302,2.363)--(7.302,2.366)--(7.302,2.365)--(7.303,2.368)--(7.303,2.369)--(7.304,2.370)%
  --(7.304,2.373)--(7.304,2.375)--(7.305,2.376)--(7.305,2.377)--(7.306,2.379)--(7.306,2.380)%
  --(7.306,2.381)--(7.307,2.383)--(7.307,2.385)--(7.308,2.386)--(7.308,2.389)--(7.309,2.390)%
  --(7.309,2.392)--(7.309,2.395)--(7.310,2.395)--(7.310,2.397)--(7.311,2.399)--(7.311,2.401)%
  --(7.311,2.403)--(7.312,2.403)--(7.312,2.404)--(7.312,2.405)--(7.313,2.407)--(7.313,2.409)%
  --(7.314,2.410)--(7.314,2.411)--(7.315,2.413)--(7.315,2.414)--(7.316,2.415)--(7.316,2.418)%
  --(7.316,2.419)--(7.317,2.420)--(7.317,2.422)--(7.318,2.424)--(7.318,2.425)--(7.318,2.426)%
  --(7.319,2.428)--(7.319,2.430)--(7.319,2.431)--(7.320,2.432)--(7.320,2.434)--(7.321,2.435)%
  --(7.321,2.436)--(7.322,2.436)--(7.322,2.437)--(7.323,2.437)--(7.323,2.440)--(7.324,2.442)%
  --(7.324,2.444)--(7.325,2.446)--(7.325,2.447)--(7.325,2.449)--(7.326,2.450)--(7.326,2.451)%
  --(7.327,2.453)--(7.327,2.454)--(7.328,2.456)--(7.328,2.458)--(7.328,2.461)--(7.329,2.461)%
  --(7.329,2.462)--(7.330,2.465)--(7.330,2.466)--(7.330,2.467)--(7.331,2.467)--(7.331,2.469)%
  --(7.332,2.470)--(7.332,2.472)--(7.332,2.473)--(7.333,2.475)--(7.333,2.476)--(7.334,2.478)%
  --(7.334,2.480)--(7.335,2.482)--(7.335,2.483)--(7.335,2.486)--(7.336,2.487)--(7.336,2.489)%
  --(7.337,2.489)--(7.337,2.487)--(7.337,2.489)--(7.338,2.491)--(7.338,2.492)--(7.339,2.494)%
  --(7.339,2.496)--(7.339,2.499)--(7.340,2.499)--(7.340,2.501)--(7.340,2.502)--(7.341,2.504)%
  --(7.341,2.505)--(7.342,2.506)--(7.342,2.509)--(7.342,2.512)--(7.343,2.513)--(7.343,2.518)%
  --(7.344,2.520)--(7.344,2.524)--(7.344,2.528)--(7.345,2.528)--(7.345,2.530)--(7.345,2.531)%
  --(7.346,2.533)--(7.346,2.536)--(7.347,2.540)--(7.347,2.541)--(7.347,2.543)--(7.348,2.544)%
  --(7.348,2.546)--(7.349,2.549)--(7.349,2.550)--(7.349,2.552)--(7.350,2.553)--(7.350,2.555)%
  --(7.351,2.558)--(7.351,2.560)--(7.352,2.561)--(7.352,2.562)--(7.353,2.564)--(7.354,2.566)%
  --(7.354,2.569)--(7.355,2.571)--(7.355,2.572)--(7.356,2.573)--(7.356,2.575)--(7.356,2.578)%
  --(7.357,2.578)--(7.357,2.580)--(7.358,2.581)--(7.358,2.582)--(7.358,2.583)--(7.359,2.584)%
  --(7.359,2.585)--(7.359,2.586)--(7.360,2.587)--(7.360,2.588)--(7.361,2.590)--(7.361,2.591)%
  --(7.361,2.592)--(7.362,2.592)--(7.362,2.593)--(7.363,2.594)--(7.363,2.593)--(7.364,2.595)%
  --(7.364,2.593)--(7.365,2.593)--(7.365,2.595)--(7.365,2.596)--(7.366,2.598)--(7.366,2.600)%
  --(7.366,2.607)--(7.367,2.610)--(7.367,2.611)--(7.368,2.612)--(7.368,2.613)--(7.368,2.615)%
  --(7.369,2.616)--(7.369,2.617)--(7.370,2.619)--(7.370,2.623)--(7.370,2.626)--(7.371,2.628)%
  --(7.371,2.631)--(7.372,2.634)--(7.372,2.636)--(7.372,2.639)--(7.373,2.641)--(7.373,2.643)%
  --(7.373,2.644)--(7.374,2.646)--(7.374,2.645)--(7.375,2.645)--(7.375,2.648)--(7.375,2.649)%
  --(7.376,2.649)--(7.376,2.650)--(7.377,2.651)--(7.377,2.652)--(7.378,2.653)--(7.379,2.653)%
  --(7.379,2.654)--(7.380,2.654)--(7.380,2.656)--(7.381,2.655)--(7.381,2.656)--(7.382,2.657)%
  --(7.382,2.659)--(7.383,2.661)--(7.383,2.662)--(7.384,2.663)--(7.384,2.665)--(7.385,2.666)%
  --(7.385,2.667)--(7.385,2.668)--(7.386,2.669)--(7.386,2.671)--(7.387,2.672)--(7.387,2.673)%
  --(7.387,2.675)--(7.388,2.676)--(7.388,2.677)--(7.389,2.679)--(7.389,2.681)--(7.390,2.682)%
  --(7.390,2.684)--(7.391,2.686)--(7.391,2.689)--(7.391,2.691)--(7.392,2.694)--(7.392,2.696)%
  --(7.392,2.697)--(7.393,2.698)--(7.393,2.699)--(7.394,2.700)--(7.394,2.703)--(7.394,2.704)%
  --(7.395,2.712)--(7.395,2.717)--(7.396,2.721)--(7.396,2.723)--(7.396,2.725)--(7.397,2.725)%
  --(7.397,2.727)--(7.398,2.727)--(7.398,2.729)--(7.399,2.729)--(7.399,2.730)--(7.400,2.730)%
  --(7.401,2.732)--(7.401,2.731)--(7.402,2.733)--(7.402,2.735)--(7.403,2.734)--(7.403,2.735)%
  --(7.403,2.737)--(7.404,2.737)--(7.404,2.740)--(7.405,2.741)--(7.405,2.744)--(7.405,2.745)%
  --(7.406,2.749)--(7.406,2.751)--(7.406,2.753)--(7.407,2.755)--(7.407,2.756)--(7.408,2.756)%
  --(7.408,2.758)--(7.409,2.759)--(7.409,2.760)--(7.410,2.761)--(7.410,2.762)--(7.410,2.763)%
  --(7.411,2.764)--(7.412,2.766)--(7.412,2.767)--(7.412,2.768)--(7.413,2.768)--(7.413,2.769)%
  --(7.413,2.772)--(7.414,2.771)--(7.414,2.772)--(7.415,2.772)--(7.415,2.773)--(7.415,2.775)%
  --(7.416,2.777)--(7.416,2.781)--(7.417,2.782)--(7.417,2.784)--(7.417,2.785)--(7.418,2.788)%
  --(7.418,2.791)--(7.418,2.793)--(7.419,2.793)--(7.419,2.794)--(7.420,2.798)--(7.420,2.800)%
  --(7.420,2.801)--(7.421,2.804)--(7.421,2.806)--(7.422,2.807)--(7.422,2.808)--(7.423,2.809)%
  --(7.423,2.811)--(7.424,2.813)--(7.424,2.816)--(7.424,2.818)--(7.425,2.821)--(7.425,2.827)%
  --(7.425,2.829)--(7.426,2.829)--(7.426,2.828)--(7.427,2.830)--(7.427,2.832)--(7.427,2.833)%
  --(7.428,2.836)--(7.428,2.837)--(7.429,2.841)--(7.429,2.844)--(7.429,2.846)--(7.430,2.847)%
  --(7.430,2.848)--(7.431,2.849)--(7.431,2.850)--(7.431,2.851)--(7.432,2.855)--(7.432,2.858)%
  --(7.432,2.862)--(7.433,2.865)--(7.433,2.866)--(7.434,2.868)--(7.434,2.871)--(7.434,2.874)%
  --(7.435,2.876)--(7.435,2.881)--(7.436,2.883)--(7.436,2.885)--(7.436,2.887)--(7.437,2.888)%
  --(7.437,2.889)--(7.438,2.891)--(7.438,2.892)--(7.438,2.896)--(7.439,2.897)--(7.439,2.900)%
  --(7.439,2.901)--(7.440,2.903)--(7.440,2.904)--(7.441,2.904)--(7.441,2.907)--(7.442,2.908)%
  --(7.442,2.911)--(7.443,2.912)--(7.443,2.915)--(7.443,2.917)--(7.444,2.917)--(7.444,2.920)%
  --(7.445,2.920)--(7.445,2.922)--(7.446,2.923)--(7.446,2.924)--(7.446,2.925)--(7.447,2.927)%
  --(7.447,2.928)--(7.448,2.931)--(7.448,2.934)--(7.449,2.935)--(7.449,2.937)--(7.450,2.939)%
  --(7.450,2.940)--(7.451,2.943)--(7.451,2.945)--(7.451,2.950)--(7.452,2.950)--(7.453,2.951)%
  --(7.453,2.954)--(7.453,2.958)--(7.454,2.961)--(7.454,2.964)--(7.455,2.967)--(7.455,2.968)%
  --(7.456,2.968)--(7.456,2.970)--(7.457,2.973)--(7.457,2.974)--(7.457,2.977)--(7.458,2.981)%
  --(7.458,2.982)--(7.458,2.985)--(7.459,2.985)--(7.459,2.988)--(7.460,2.990)--(7.460,2.991)%
  --(7.460,2.994)--(7.461,2.997)--(7.461,3.000)--(7.462,3.004)--(7.462,3.005)--(7.463,3.008)%
  --(7.463,3.011)--(7.464,3.012)--(7.464,3.015)--(7.464,3.017)--(7.465,3.019)--(7.465,3.021)%
  --(7.466,3.024)--(7.466,3.025)--(7.467,3.026)--(7.467,3.030)--(7.467,3.031)--(7.468,3.034)%
  --(7.468,3.036)--(7.469,3.039)--(7.469,3.041)--(7.469,3.043)--(7.470,3.042)--(7.470,3.045)%
  --(7.471,3.045)--(7.471,3.048)--(7.471,3.049)--(7.472,3.052)--(7.472,3.057)--(7.472,3.059)%
  --(7.473,3.059)--(7.473,3.063)--(7.474,3.067)--(7.474,3.069)--(7.474,3.071)--(7.475,3.073)%
  --(7.475,3.078)--(7.476,3.082)--(7.476,3.083)--(7.476,3.088)--(7.477,3.089)--(7.477,3.090)%
  --(7.478,3.092)--(7.478,3.093)--(7.478,3.095)--(7.479,3.095)--(7.479,3.096)--(7.479,3.098)%
  --(7.480,3.102)--(7.480,3.104)--(7.481,3.107)--(7.481,3.108)--(7.481,3.109)--(7.482,3.111)%
  --(7.482,3.113)--(7.483,3.116)--(7.483,3.118)--(7.483,3.119)--(7.484,3.123)--(7.484,3.128)%
  --(7.484,3.129)--(7.485,3.130)--(7.485,3.132)--(7.486,3.134)--(7.486,3.135)--(7.487,3.137)%
  --(7.487,3.139)--(7.488,3.140)--(7.488,3.142)--(7.488,3.143)--(7.489,3.144)--(7.489,3.145)%
  --(7.490,3.148)--(7.490,3.149)--(7.491,3.150)--(7.491,3.153)--(7.492,3.155)--(7.492,3.158)%
  --(7.493,3.159)--(7.493,3.162)--(7.493,3.164)--(7.494,3.166)--(7.494,3.167)--(7.495,3.168)%
  --(7.495,3.172)--(7.496,3.172)--(7.496,3.175)--(7.497,3.176)--(7.497,3.178)--(7.497,3.180)%
  --(7.498,3.182)--(7.498,3.183)--(7.498,3.188)--(7.499,3.188)--(7.499,3.191)--(7.500,3.192)%
  --(7.500,3.194)--(7.500,3.195)--(7.501,3.197)--(7.501,3.199)--(7.502,3.199)--(7.502,3.201)%
  --(7.502,3.202)--(7.503,3.205)--(7.503,3.206)--(7.504,3.208)--(7.504,3.209)--(7.504,3.211)%
  --(7.505,3.214)--(7.505,3.216)--(7.505,3.220)--(7.506,3.222)--(7.506,3.224)--(7.507,3.227)%
  --(7.507,3.229)--(7.507,3.231)--(7.508,3.231)--(7.508,3.233)--(7.509,3.235)--(7.509,3.238)%
  --(7.509,3.239)--(7.510,3.243)--(7.510,3.245)--(7.511,3.246)--(7.511,3.251)--(7.511,3.252)%
  --(7.512,3.257)--(7.512,3.260)--(7.512,3.265)--(7.513,3.268)--(7.514,3.268)--(7.514,3.270)%
  --(7.514,3.271)--(7.515,3.271)--(7.515,3.273)--(7.516,3.275)--(7.516,3.279)--(7.516,3.282)%
  --(7.517,3.284)--(7.517,3.287)--(7.518,3.289)--(7.518,3.291)--(7.518,3.295)--(7.519,3.294)%
  --(7.520,3.296)--(7.521,3.300)--(7.521,3.301)--(7.521,3.306)--(7.522,3.311)--(7.522,3.313)%
  --(7.523,3.316)--(7.523,3.320)--(7.524,3.321)--(7.524,3.323)--(7.524,3.326)--(7.525,3.327)%
  --(7.525,3.331)--(7.526,3.333)--(7.526,3.335)--(7.527,3.337)--(7.527,3.339)--(7.528,3.343)%
  --(7.528,3.346)--(7.528,3.349)--(7.529,3.349)--(7.529,3.351)--(7.530,3.358)--(7.530,3.360)%
  --(7.530,3.367)--(7.531,3.372)--(7.531,3.376)--(7.531,3.379)--(7.532,3.384)--(7.532,3.386)%
  --(7.533,3.388)--(7.533,3.391)--(7.533,3.388)--(7.534,3.389)--(7.534,3.390)--(7.535,3.393)%
  --(7.535,3.395)--(7.535,3.398)--(7.536,3.401)--(7.536,3.406)--(7.537,3.412)--(7.537,3.414)%
  --(7.537,3.416)--(7.538,3.417)--(7.538,3.421)--(7.538,3.424)--(7.539,3.426)--(7.539,3.429)%
  --(7.540,3.432)--(7.540,3.435)--(7.540,3.438)--(7.541,3.438)--(7.541,3.441)--(7.542,3.444)%
  --(7.542,3.445)--(7.542,3.446)--(7.543,3.448)--(7.543,3.451)--(7.544,3.455)--(7.544,3.457)%
  --(7.544,3.458)--(7.545,3.458)--(7.545,3.459)--(7.545,3.463)--(7.546,3.464)--(7.546,3.466)%
  --(7.547,3.470)--(7.547,3.469)--(7.547,3.471)--(7.548,3.474)--(7.548,3.476)--(7.549,3.478)%
  --(7.549,3.480)--(7.549,3.481)--(7.550,3.484)--(7.550,3.488)--(7.551,3.488)--(7.551,3.495)%
  --(7.551,3.497)--(7.552,3.502)--(7.552,3.507)--(7.552,3.513)--(7.553,3.517)--(7.553,3.520)%
  --(7.554,3.521)--(7.554,3.525)--(7.554,3.524)--(7.555,3.525)--(7.555,3.528)--(7.556,3.531)%
  --(7.556,3.535)--(7.556,3.537)--(7.557,3.538)--(7.557,3.541)--(7.558,3.545)--(7.558,3.546)%
  --(7.559,3.550)--(7.559,3.553)--(7.560,3.557)--(7.560,3.563)--(7.561,3.568)--(7.561,3.570)%
  --(7.561,3.573)--(7.562,3.576)--(7.562,3.578)--(7.563,3.581)--(7.563,3.583)--(7.563,3.584)%
  --(7.564,3.588)--(7.564,3.591)--(7.564,3.593)--(7.565,3.600)--(7.565,3.602)--(7.566,3.602)%
  --(7.566,3.605)--(7.566,3.606)--(7.567,3.610)--(7.567,3.613)--(7.568,3.613)--(7.568,3.614)%
  --(7.568,3.617)--(7.569,3.621)--(7.570,3.625)--(7.570,3.629)--(7.570,3.633)--(7.571,3.634)%
  --(7.571,3.636)--(7.571,3.638)--(7.572,3.641)--(7.572,3.643)--(7.573,3.648)--(7.573,3.649)%
  --(7.574,3.651)--(7.574,3.655)--(7.575,3.660)--(7.575,3.661)--(7.575,3.665)--(7.576,3.668)%
  --(7.576,3.670)--(7.577,3.672)--(7.577,3.677)--(7.578,3.678)--(7.578,3.680)--(7.579,3.681)%
  --(7.579,3.683)--(7.580,3.685)--(7.580,3.688)--(7.580,3.690)--(7.581,3.692)--(7.581,3.695)%
  --(7.582,3.699)--(7.582,3.701)--(7.582,3.704)--(7.583,3.706)--(7.583,3.708)--(7.584,3.713)%
  --(7.584,3.715)--(7.584,3.719)--(7.585,3.723)--(7.585,3.726)--(7.586,3.729)--(7.586,3.733)%
  --(7.587,3.738)--(7.587,3.741)--(7.587,3.745)--(7.588,3.747)--(7.588,3.751)--(7.589,3.752)%
  --(7.589,3.754)--(7.589,3.759)--(7.590,3.766)--(7.590,3.770)--(7.590,3.775)--(7.591,3.779)%
  --(7.591,3.781)--(7.592,3.786)--(7.592,3.791)--(7.593,3.797)--(7.593,3.798)--(7.594,3.805)%
  --(7.594,3.810)--(7.594,3.816)--(7.595,3.820)--(7.595,3.822)--(7.596,3.826)--(7.596,3.833)%
  --(7.596,3.834)--(7.597,3.842)--(7.597,3.845)--(7.598,3.851)--(7.598,3.859)--(7.599,3.860)%
  --(7.599,3.864)--(7.599,3.865)--(7.600,3.867)--(7.600,3.871)--(7.601,3.874)--(7.601,3.878)%
  --(7.601,3.882)--(7.602,3.888)--(7.602,3.892)--(7.603,3.898)--(7.603,3.900)--(7.603,3.904)%
  --(7.604,3.904)--(7.604,3.908)--(7.604,3.912)--(7.605,3.916)--(7.605,3.921)--(7.606,3.923)%
  --(7.606,3.926)--(7.606,3.931)--(7.607,3.937)--(7.607,3.938)--(7.608,3.943)--(7.608,3.944)%
  --(7.608,3.949)--(7.609,3.955)--(7.609,3.961)--(7.610,3.967)--(7.610,3.971)--(7.610,3.977)%
  --(7.611,3.980)--(7.611,3.983)--(7.611,3.988)--(7.612,3.992)--(7.612,3.996)--(7.613,4.000)%
  --(7.613,4.007)--(7.614,4.012)--(7.614,4.013)--(7.615,4.018)--(7.615,4.019)--(7.615,4.022)%
  --(7.616,4.026)--(7.617,4.028)--(7.617,4.031)--(7.617,4.034)--(7.618,4.038)--(7.618,4.042)%
  --(7.618,4.044)--(7.619,4.046)--(7.619,4.048)--(7.620,4.052)--(7.620,4.054)--(7.620,4.058)%
  --(7.621,4.062)--(7.621,4.065)--(7.622,4.070)--(7.622,4.078)--(7.622,4.090)--(7.623,4.095)%
  --(7.623,4.099)--(7.624,4.097)--(7.624,4.100)--(7.624,4.103)--(7.625,4.102)--(7.625,4.104)%
  --(7.625,4.105)--(7.626,4.109)--(7.626,4.110)--(7.627,4.112)--(7.627,4.113)--(7.627,4.114)%
  --(7.628,4.117)--(7.628,4.116)--(7.629,4.119)--(7.629,4.123)--(7.629,4.126)--(7.630,4.132)%
  --(7.630,4.137)--(7.630,4.141)--(7.631,4.142)--(7.631,4.146)--(7.632,4.148)--(7.632,4.151)%
  --(7.632,4.158)--(7.633,4.159)--(7.633,4.164)--(7.634,4.166)--(7.634,4.169)--(7.634,4.171)%
  --(7.635,4.175)--(7.635,4.181)--(7.636,4.185)--(7.636,4.188)--(7.636,4.191)--(7.637,4.193)%
  --(7.637,4.198)--(7.637,4.202)--(7.638,4.205)--(7.638,4.209)--(7.639,4.212)--(7.639,4.218)%
  --(7.639,4.222)--(7.640,4.226)--(7.640,4.233)--(7.641,4.237)--(7.641,4.242)--(7.641,4.244)%
  --(7.642,4.246)--(7.642,4.250)--(7.643,4.257)--(7.643,4.262)--(7.643,4.267)--(7.644,4.273)%
  --(7.644,4.281)--(7.644,4.283)--(7.645,4.289)--(7.645,4.292)--(7.646,4.302)--(7.646,4.311)%
  --(7.646,4.316)--(7.647,4.329)--(7.647,4.342)--(7.648,4.343)--(7.648,4.347)--(7.648,4.354)%
  --(7.649,4.357)--(7.649,4.360)--(7.650,4.364)--(7.650,4.368)--(7.650,4.373)--(7.651,4.376)%
  --(7.651,4.380)--(7.651,4.382)--(7.652,4.388)--(7.652,4.391)--(7.653,4.395)--(7.653,4.406)%
  --(7.653,4.405)--(7.654,4.410)--(7.654,4.413)--(7.655,4.418)--(7.655,4.424)--(7.656,4.424)%
  --(7.656,4.425)--(7.657,4.429)--(7.657,4.432)--(7.657,4.437)--(7.658,4.441)--(7.658,4.447)%
  --(7.658,4.449)--(7.659,4.450)--(7.659,4.453)--(7.660,4.454)--(7.660,4.456)--(7.660,4.457)%
  --(7.661,4.460)--(7.661,4.462)--(7.662,4.464)--(7.662,4.469)--(7.662,4.470)--(7.663,4.470)%
  --(7.663,4.474)--(7.663,4.477)--(7.664,4.484)--(7.664,4.492)--(7.665,4.500)--(7.665,4.502)%
  --(7.665,4.503)--(7.666,4.510)--(7.666,4.520)--(7.667,4.525)--(7.667,4.532)--(7.667,4.536)%
  --(7.668,4.535)--(7.668,4.540)--(7.669,4.543)--(7.669,4.548)--(7.669,4.551)--(7.670,4.555)%
  --(7.670,4.561)--(7.670,4.566)--(7.671,4.574)--(7.671,4.580)--(7.672,4.581)--(7.672,4.585)%
  --(7.673,4.594)--(7.673,4.608)--(7.674,4.610)--(7.674,4.615)--(7.674,4.622)--(7.675,4.628)%
  --(7.675,4.629)--(7.676,4.635)--(7.676,4.646)--(7.676,4.652)--(7.677,4.657)--(7.677,4.662)%
  --(7.677,4.675)--(7.678,4.683)--(7.678,4.690)--(7.679,4.696)--(7.679,4.702)--(7.679,4.708)%
  --(7.680,4.710)--(7.680,4.713)--(7.681,4.713)--(7.681,4.711)--(7.681,4.713)--(7.682,4.724)%
  --(7.682,4.732)--(7.683,4.736)--(7.683,4.740)--(7.683,4.738)--(7.684,4.741)--(7.684,4.749)%
  --(7.684,4.751)--(7.685,4.756)--(7.685,4.764)--(7.686,4.766)--(7.686,4.770)--(7.686,4.774)%
  --(7.687,4.782)--(7.687,4.789)--(7.688,4.795)--(7.688,4.800)--(7.688,4.801)--(7.689,4.808)%
  --(7.689,4.814)--(7.690,4.814)--(7.690,4.824)--(7.690,4.833)--(7.691,4.837)--(7.691,4.845)%
  --(7.691,4.846)--(7.692,4.854)--(7.692,4.862)--(7.693,4.869)--(7.693,4.874)--(7.693,4.877)%
  --(7.694,4.884)--(7.694,4.885)--(7.695,4.895)--(7.695,4.901)--(7.695,4.907)--(7.696,4.914)%
  --(7.696,4.921)--(7.696,4.925)--(7.697,4.927)--(7.697,4.932)--(7.698,4.941)--(7.698,4.946)%
  --(7.698,4.952)--(7.699,4.957)--(7.699,4.966)--(7.700,4.971)--(7.700,4.974)--(7.701,4.980)%
  --(7.701,4.989)--(7.702,4.993)--(7.702,4.996)--(7.702,4.997)--(7.703,5.000)--(7.703,5.003)%
  --(7.704,5.011)--(7.704,5.018)--(7.705,5.018)--(7.705,5.021)--(7.705,5.022)--(7.706,5.023)%
  --(7.706,5.030)--(7.707,5.034)--(7.707,5.038)--(7.707,5.043)--(7.708,5.047)--(7.708,5.053)%
  --(7.709,5.061)--(7.709,5.070)--(7.709,5.078)--(7.710,5.084)--(7.710,5.090)--(7.710,5.095)%
  --(7.711,5.106)--(7.711,5.117)--(7.712,5.122)--(7.712,5.129)--(7.712,5.133)--(7.713,5.142)%
  --(7.713,5.148)--(7.714,5.154)--(7.714,5.167)--(7.714,5.172)--(7.715,5.179)--(7.715,5.186)%
  --(7.716,5.196)--(7.716,5.204)--(7.716,5.210)--(7.717,5.211)--(7.717,5.216)--(7.717,5.223)%
  --(7.718,5.226)--(7.718,5.234)--(7.719,5.239)--(7.719,5.247)--(7.719,5.259)--(7.720,5.264)%
  --(7.720,5.272)--(7.721,5.276)--(7.721,5.285)--(7.721,5.292)--(7.722,5.296)--(7.722,5.307)%
  --(7.723,5.315)--(7.723,5.319)--(7.723,5.324)--(7.724,5.329)--(7.724,5.333)--(7.724,5.336)%
  --(7.725,5.344)--(7.725,5.342)--(7.726,5.347)--(7.726,5.355)--(7.726,5.358)--(7.727,5.364)%
  --(7.727,5.370)--(7.728,5.375)--(7.728,5.379)--(7.728,5.387)--(7.729,5.398)--(7.729,5.413)%
  --(7.730,5.424)--(7.730,5.429)--(7.730,5.430)--(7.731,5.433)--(7.731,5.435)--(7.731,5.440)%
  --(7.732,5.447)--(7.732,5.456)--(7.733,5.464)--(7.733,5.473)--(7.733,5.477)--(7.734,5.486)%
  --(7.734,5.495)--(7.735,5.505)--(7.735,5.509)--(7.735,5.514)--(7.736,5.520)--(7.736,5.528)%
  --(7.736,5.537)--(7.737,5.543)--(7.737,5.553)--(7.738,5.569)--(7.738,5.574)--(7.738,5.584)%
  --(7.739,5.584)--(7.740,5.594)--(7.740,5.600)--(7.740,5.606)--(7.741,5.609)--(7.741,5.618)%
  --(7.742,5.623)--(7.742,5.633)--(7.742,5.647)--(7.743,5.657)--(7.743,5.666)--(7.743,5.665)%
  --(7.744,5.670)--(7.744,5.675)--(7.745,5.679)--(7.745,5.685)--(7.745,5.689)--(7.746,5.693)%
  --(7.747,5.698)--(7.747,5.702)--(7.747,5.711)--(7.748,5.718)--(7.748,5.728)--(7.749,5.734)%
  --(7.749,5.740)--(7.749,5.746)--(7.750,5.753)--(7.750,5.766)--(7.750,5.774)--(7.751,5.784)%
  --(7.751,5.794)--(7.752,5.800)--(7.752,5.808)--(7.752,5.818)--(7.753,5.827)--(7.753,5.832)%
  --(7.754,5.838)--(7.754,5.844)--(7.754,5.858)--(7.755,5.878)--(7.755,5.892)--(7.756,5.897)%
  --(7.756,5.911)--(7.756,5.929)--(7.757,5.933)--(7.757,5.940)--(7.757,5.948)--(7.758,5.956)%
  --(7.758,5.968)--(7.759,5.980)--(7.759,5.984)--(7.759,5.994)--(7.760,6.004)--(7.760,6.014)%
  --(7.761,6.025)--(7.761,6.029)--(7.761,6.040)--(7.762,6.042)--(7.762,6.053)--(7.763,6.063)%
  --(7.763,6.071)--(7.763,6.073)--(7.764,6.075)--(7.764,6.082)--(7.764,6.090)--(7.765,6.111)%
  --(7.765,6.117)--(7.766,6.124)--(7.766,6.130)--(7.766,6.134)--(7.767,6.141)--(7.767,6.160)%
  --(7.768,6.173)--(7.768,6.175)--(7.768,6.182)--(7.769,6.188)--(7.769,6.189)--(7.769,6.198)%
  --(7.770,6.201)--(7.770,6.206)--(7.771,6.212)--(7.771,6.221)--(7.771,6.225)--(7.772,6.232)%
  --(7.772,6.241)--(7.773,6.248)--(7.773,6.261)--(7.773,6.270)--(7.774,6.278)--(7.774,6.282)%
  --(7.775,6.291)--(7.775,6.294)--(7.775,6.307)--(7.776,6.308)--(7.776,6.326)--(7.776,6.337)%
  --(7.777,6.337)--(7.777,6.341)--(7.778,6.342)--(7.778,6.348)--(7.778,6.362)--(7.779,6.378)%
  --(7.779,6.393)--(7.780,6.395)--(7.780,6.408)--(7.780,6.414)--(7.781,6.418)--(7.781,6.423)%
  --(7.782,6.430)--(7.782,6.446)--(7.782,6.447)--(7.783,6.458)--(7.783,6.465)--(7.783,6.476)%
  --(7.784,6.488)--(7.784,6.495)--(7.785,6.499)--(7.785,6.511)--(7.785,6.524)--(7.786,6.532)%
  --(7.786,6.539)--(7.787,6.549)--(7.787,6.560)--(7.787,6.576)--(7.788,6.583)--(7.788,6.598)%
  --(7.789,6.610)--(7.789,6.615)--(7.789,6.619)--(7.790,6.625)--(7.790,6.398)--(7.790,6.387)%
  --(7.791,6.376)--(7.791,6.363)--(7.792,6.359)--(7.792,6.354)--(7.792,6.344)--(7.793,6.332)%
  --(7.793,6.321)--(7.794,6.312)--(7.794,6.307)--(7.794,6.297)--(7.795,6.278)--(7.795,6.265)%
  --(7.796,6.256)--(7.796,6.249)--(7.796,6.241)--(7.797,6.233)--(7.797,6.224)--(7.797,6.223)%
  --(7.798,6.211)--(7.798,6.201)--(7.799,6.190)--(7.799,6.180)--(7.799,6.171)--(7.800,6.170)%
  --(7.800,6.154)--(7.801,6.130)--(7.801,6.116)--(7.801,6.100)--(7.802,6.078)--(7.802,6.057)%
  --(7.802,6.045)--(7.803,6.037)--(7.803,6.020)--(7.804,6.012)--(7.804,6.003)--(7.804,6.000)%
  --(7.805,5.997)--(7.805,5.986)--(7.806,5.979)--(7.806,5.971)--(7.806,5.966)--(7.807,5.956)%
  --(7.807,5.946)--(7.808,5.933)--(7.808,5.922)--(7.808,5.907)--(7.809,5.886)--(7.809,5.875)%
  --(7.809,5.869)--(7.810,5.861)--(7.810,5.854)--(7.811,5.842)--(7.811,5.826)--(7.811,5.817)%
  --(7.812,5.812)--(7.812,5.806)--(7.813,5.797)--(7.813,5.785)--(7.813,5.780)--(7.814,5.775)%
  --(7.814,5.767)--(7.815,5.749)--(7.815,5.741)--(7.815,5.733)--(7.816,5.726)--(7.816,5.720)%
  --(7.816,5.714)--(7.817,5.701)--(7.817,5.693)--(7.818,5.683)--(7.818,5.670)--(7.818,5.664)%
  --(7.819,5.660)--(7.819,5.649)--(7.820,5.639)--(7.820,5.623)--(7.820,5.613)--(7.821,5.606)%
  --(7.821,5.600)--(7.822,5.589)--(7.822,5.582)--(7.822,5.578)--(7.823,5.576)--(7.823,5.564)%
  --(7.823,5.551)--(7.824,5.542)--(7.824,5.529)--(7.825,5.511)--(7.825,5.503)--(7.825,5.493)%
  --(7.826,5.480)--(7.826,5.474)--(7.827,5.462)--(7.827,5.451)--(7.827,5.445)--(7.828,5.436)%
  --(7.828,5.430)--(7.829,5.421)--(7.829,5.406)--(7.829,5.397)--(7.830,5.392)--(7.830,5.388)%
  --(7.830,5.384)--(7.831,5.377)--(7.831,5.370)--(7.832,5.363)--(7.832,5.359)--(7.832,5.353)%
  --(7.833,5.346)--(7.833,5.339)--(7.834,5.334)--(7.834,5.324)--(7.834,5.318)--(7.835,5.306)%
  --(7.835,5.299)--(7.836,5.294)--(7.836,5.291)--(7.836,5.279)--(7.837,5.272)--(7.837,5.266)%
  --(7.837,5.256)--(7.838,5.246)--(7.838,5.235)--(7.839,5.227)--(7.839,5.215)--(7.839,5.199)%
  --(7.840,5.191)--(7.840,5.183)--(7.841,5.173)--(7.841,5.172)--(7.841,5.164)--(7.842,5.162)%
  --(7.842,5.155)--(7.842,5.154)--(7.843,5.153)--(7.843,5.150)--(7.844,5.141)--(7.844,5.132)%
  --(7.844,5.128)--(7.845,5.123)--(7.845,5.120)--(7.846,5.114)--(7.846,5.107)--(7.846,5.102)%
  --(7.847,5.095)--(7.847,5.089)--(7.848,5.081)--(7.848,5.073)--(7.848,5.065)--(7.849,5.057)%
  --(7.849,5.045)--(7.849,5.037)--(7.850,5.026)--(7.850,5.014)--(7.851,5.011)--(7.851,5.007)%
  --(7.851,4.997)--(7.852,4.991)--(7.852,4.990)--(7.853,4.987)--(7.853,4.980)--(7.853,4.970)%
  --(7.854,4.962)--(7.854,4.952)--(7.855,4.943)--(7.855,4.938)--(7.855,4.933)--(7.856,4.926)%
  --(7.856,4.917)--(7.856,4.909)--(7.857,4.900)--(7.857,4.885)--(7.858,4.873)--(7.858,4.864)%
  --(7.858,4.858)--(7.859,4.854)--(7.859,4.848)--(7.860,4.838)--(7.860,4.833)--(7.860,4.825)%
  --(7.861,4.821)--(7.861,4.811)--(7.862,4.804)--(7.862,4.799)--(7.862,4.796)--(7.863,4.788)%
  --(7.863,4.781)--(7.863,4.777)--(7.864,4.775)--(7.864,4.767)--(7.865,4.748)--(7.865,4.736)%
  --(7.865,4.730)--(7.866,4.725)--(7.866,4.722)--(7.867,4.718)--(7.867,4.708)--(7.867,4.698)%
  --(7.868,4.692)--(7.868,4.689)--(7.869,4.684)--(7.869,4.674)--(7.869,4.669)--(7.870,4.664)%
  --(7.870,4.660)--(7.870,4.656)--(7.871,4.654)--(7.871,4.648)--(7.872,4.641)--(7.872,4.637)%
  --(7.872,4.634)--(7.873,4.632)--(7.873,4.625)--(7.874,4.601)--(7.874,4.597)--(7.874,4.589)%
  --(7.875,4.584)--(7.875,4.581)--(7.875,4.572)--(7.876,4.563)--(7.876,4.558)--(7.877,4.557)%
  --(7.877,4.558)--(7.877,4.557)--(7.878,4.556)--(7.879,4.552)--(7.879,4.549)--(7.879,4.546)%
  --(7.880,4.537)--(7.880,4.525)--(7.881,4.515)--(7.881,4.508)--(7.881,4.501)--(7.882,4.493)%
  --(7.882,4.491)--(7.882,4.478)--(7.883,4.462)--(7.883,4.458)--(7.884,4.453)--(7.884,4.452)%
  --(7.884,4.448)--(7.885,4.439)--(7.885,4.432)--(7.886,4.429)--(7.886,4.427)--(7.886,4.424)%
  --(7.887,4.422)--(7.887,4.423)--(7.888,4.414)--(7.888,4.416)--(7.888,4.412)--(7.889,4.409)%
  --(7.889,4.405)--(7.889,4.400)--(7.890,4.397)--(7.891,4.388)--(7.891,4.383)--(7.891,4.376)%
  --(7.892,4.373)--(7.892,4.366)--(7.893,4.356)--(7.893,4.346)--(7.893,4.340)--(7.894,4.331)%
  --(7.894,4.323)--(7.895,4.316)--(7.895,4.311)--(7.895,4.306)--(7.896,4.301)--(7.896,4.298)%
  --(7.896,4.290)--(7.897,4.283)--(7.897,4.278)--(7.898,4.278)--(7.898,4.271)--(7.898,4.268)%
  --(7.899,4.265)--(7.899,4.260)--(7.900,4.251)--(7.900,4.250)--(7.900,4.247)--(7.901,4.242)%
  --(7.901,4.234)--(7.902,4.225)--(7.902,4.217)--(7.902,4.213)--(7.903,4.209)--(7.903,4.206)%
  --(7.903,4.202)--(7.904,4.193)--(7.904,4.187)--(7.905,4.180)--(7.905,4.173)--(7.905,4.169)%
  --(7.906,4.166)--(7.906,4.161)--(7.907,4.150)--(7.907,4.128)--(7.907,4.126)--(7.908,4.124)%
  --(7.908,4.126)--(7.909,4.121)--(7.909,4.115)--(7.910,4.109)--(7.910,4.102)--(7.910,4.091)%
  --(7.911,4.082)--(7.911,4.074)--(7.912,4.065)--(7.912,4.063)--(7.912,4.060)--(7.913,4.053)%
  --(7.913,4.045)--(7.914,4.038)--(7.914,4.032)--(7.914,4.028)--(7.915,4.022)--(7.915,4.015)%
  --(7.915,4.011)--(7.916,4.006)--(7.916,4.003)--(7.917,4.001)--(7.917,3.997)--(7.917,3.987)%
  --(7.918,3.981)--(7.918,3.975)--(7.919,3.966)--(7.919,3.958)--(7.919,3.951)--(7.920,3.945)%
  --(7.920,3.942)--(7.921,3.937)--(7.921,3.930)--(7.921,3.921)--(7.922,3.916)--(7.922,3.913)%
  --(7.922,3.912)--(7.923,3.910)--(7.923,3.901)--(7.924,3.893)--(7.924,3.889)--(7.924,3.884)%
  --(7.925,3.882)--(7.925,3.875)--(7.926,3.868)--(7.926,3.860)--(7.926,3.853)--(7.927,3.851)%
  --(7.927,3.842)--(7.928,3.837)--(7.928,3.829)--(7.929,3.829)--(7.929,3.825)--(7.930,3.825)%
  --(7.930,3.822)--(7.931,3.821)--(7.931,3.816)--(7.932,3.807)--(7.932,3.800)--(7.933,3.795)%
  --(7.933,3.794)--(7.933,3.789)--(7.934,3.786)--(7.934,3.785)--(7.935,3.778)--(7.935,3.775)%
  --(7.935,3.772)--(7.936,3.770)--(7.936,3.766)--(7.936,3.764)--(7.937,3.761)--(7.937,3.756)%
  --(7.938,3.749)--(7.938,3.744)--(7.938,3.739)--(7.939,3.733)--(7.939,3.725)--(7.940,3.711)%
  --(7.940,3.698)--(7.940,3.693)--(7.941,3.690)--(7.941,3.684)--(7.942,3.680)--(7.942,3.678)%
  --(7.942,3.677)--(7.943,3.673)--(7.943,3.668)--(7.943,3.661)--(7.944,3.657)--(7.944,3.656)%
  --(7.945,3.654)--(7.945,3.652)--(7.945,3.649)--(7.946,3.644)--(7.946,3.641)--(7.947,3.638)%
  --(7.947,3.635)--(7.947,3.633)--(7.948,3.630)--(7.948,3.628)--(7.948,3.622)--(7.949,3.619)%
  --(7.949,3.612)--(7.950,3.610)--(7.950,3.607)--(7.950,3.605)--(7.951,3.603)--(7.951,3.600)%
  --(7.952,3.599)--(7.952,3.597)--(7.952,3.595)--(7.953,3.596)--(7.953,3.594)--(7.954,3.592)%
  --(7.954,3.590)--(7.955,3.591)--(7.955,3.589)--(7.955,3.588)--(7.956,3.586)--(7.956,3.587)%
  --(7.957,3.587)--(7.957,3.585)--(7.958,3.583)--(7.958,3.579)--(7.959,3.576)--(7.959,3.570)%
  --(7.959,3.567)--(7.960,3.561)--(7.960,3.554)--(7.961,3.551)--(7.961,3.547)--(7.961,3.544)%
  --(7.962,3.537)--(7.962,3.534)--(7.962,3.528)--(7.963,3.525)--(7.963,3.522)--(7.964,3.516)%
  --(7.964,3.511)--(7.964,3.509)--(7.965,3.506)--(7.965,3.499)--(7.966,3.493)--(7.966,3.490)%
  --(7.966,3.489)--(7.967,3.486)--(7.967,3.484)--(7.968,3.483)--(7.968,3.479)--(7.968,3.477)%
  --(7.969,3.475)--(7.969,3.470)--(7.969,3.468)--(7.970,3.464)--(7.970,3.463)--(7.971,3.459)%
  --(7.971,3.455)--(7.972,3.455)--(7.972,3.453)--(7.973,3.450)--(7.973,3.448)--(7.973,3.444)%
  --(7.974,3.442)--(7.974,3.440)--(7.975,3.438)--(7.975,3.437)--(7.975,3.434)--(7.976,3.433)%
  --(7.976,3.432)--(7.976,3.429)--(7.977,3.425)--(7.977,3.421)--(7.978,3.420)--(7.978,3.417)%
  --(7.978,3.412)--(7.979,3.409)--(7.979,3.404)--(7.980,3.400)--(7.980,3.393)--(7.980,3.388)%
  --(7.981,3.384)--(7.981,3.381)--(7.981,3.379)--(7.982,3.379)--(7.982,3.377)--(7.983,3.377)%
  --(7.983,3.375)--(7.983,3.374)--(7.984,3.370)--(7.984,3.369)--(7.985,3.366)--(7.985,3.360)%
  --(7.985,3.358)--(7.986,3.355)--(7.986,3.352)--(7.987,3.348)--(7.987,3.341)--(7.987,3.337)%
  --(7.988,3.331)--(7.988,3.328)--(7.988,3.325)--(7.989,3.321)--(7.989,3.319)--(7.990,3.317)%
  --(7.990,3.314)--(7.990,3.308)--(7.991,3.306)--(7.991,3.304)--(7.992,3.302)--(7.992,3.300)%
  --(7.992,3.298)--(7.993,3.297)--(7.993,3.296)--(7.994,3.291)--(7.994,3.290)--(7.995,3.287)%
  --(7.995,3.284)--(7.995,3.282)--(7.996,3.280)--(7.996,3.279)--(7.997,3.275)--(7.997,3.272)%
  --(7.997,3.271)--(7.998,3.267)--(7.998,3.266)--(7.999,3.261)--(7.999,3.257)--(7.999,3.253)%
  --(8.000,3.251)--(8.000,3.248)--(8.001,3.247)--(8.001,3.244)--(8.001,3.239)--(8.002,3.235)%
  --(8.002,3.231)--(8.002,3.230)--(8.003,3.226)--(8.003,3.223)--(8.004,3.219)--(8.004,3.214)%
  --(8.004,3.213)--(8.005,3.212)--(8.005,3.210)--(8.006,3.207)--(8.006,3.203)--(8.006,3.201)%
  --(8.007,3.200)--(8.007,3.198)--(8.008,3.195)--(8.008,3.193)--(8.008,3.188)--(8.009,3.184)%
  --(8.009,3.183)--(8.009,3.180)--(8.010,3.171)--(8.010,3.166)--(8.011,3.164)--(8.011,3.165)%
  --(8.011,3.164)--(8.012,3.162)--(8.012,3.161)--(8.013,3.157)--(8.013,3.154)--(8.013,3.151)%
  --(8.014,3.149)--(8.014,3.148)--(8.015,3.147)--(8.015,3.145)--(8.016,3.139)--(8.016,3.134)%
  --(8.016,3.132)--(8.017,3.130)--(8.017,3.128)--(8.018,3.128)--(8.018,3.125)--(8.018,3.124)%
  --(8.019,3.121)--(8.019,3.119)--(8.020,3.114)--(8.020,3.109)--(8.020,3.105)--(8.021,3.099)%
  --(8.021,3.096)--(8.021,3.094)--(8.022,3.092)--(8.022,3.091)--(8.023,3.089)--(8.023,3.087)%
  --(8.023,3.084)--(8.024,3.080)--(8.024,3.076)--(8.025,3.075)--(8.025,3.072)--(8.025,3.071)%
  --(8.026,3.071)--(8.026,3.068)--(8.027,3.065)--(8.027,3.064)--(8.027,3.060)--(8.028,3.056)%
  --(8.028,3.054)--(8.028,3.052)--(8.029,3.051)--(8.029,3.047)--(8.030,3.035)--(8.030,3.034)%
  --(8.030,3.031)--(8.031,3.032)--(8.031,3.033)--(8.032,3.029)--(8.032,3.028)--(8.032,3.025)%
  --(8.033,3.019)--(8.033,3.016)--(8.034,3.014)--(8.034,3.011)--(8.034,3.008)--(8.035,3.006)%
  --(8.035,3.002)--(8.035,3.000)--(8.036,2.996)--(8.036,2.993)--(8.037,2.991)--(8.037,2.988)%
  --(8.037,2.986)--(8.038,2.981)--(8.039,2.979)--(8.039,2.976)--(8.039,2.975)--(8.040,2.973)%
  --(8.040,2.967)--(8.041,2.956)--(8.041,2.952)--(8.042,2.950)--(8.042,2.948)--(8.043,2.946)%
  --(8.043,2.945)--(8.044,2.943)--(8.044,2.941)--(8.045,2.942)--(8.045,2.940)--(8.046,2.939)%
  --(8.046,2.937)--(8.046,2.936)--(8.047,2.934)--(8.047,2.932)--(8.048,2.932)--(8.048,2.930)%
  --(8.048,2.925)--(8.049,2.920)--(8.049,2.917)--(8.049,2.915)--(8.050,2.912)--(8.050,2.907)%
  --(8.051,2.896)--(8.051,2.890)--(8.051,2.888)--(8.052,2.886)--(8.052,2.883)--(8.053,2.880)%
  --(8.053,2.877)--(8.053,2.875)--(8.054,2.872)--(8.054,2.870)--(8.054,2.862)--(8.055,2.851)%
  --(8.055,2.849)--(8.056,2.847)--(8.056,2.846)--(8.056,2.845)--(8.057,2.844)--(8.057,2.843)%
  --(8.058,2.841)--(8.058,2.837)--(8.058,2.835)--(8.059,2.834)--(8.059,2.829)--(8.060,2.826)%
  --(8.060,2.825)--(8.060,2.823)--(8.061,2.821)--(8.061,2.819)--(8.061,2.818)--(8.062,2.817)%
  --(8.062,2.813)--(8.063,2.813)--(8.063,2.814)--(8.064,2.811)--(8.064,2.809)--(8.065,2.807)%
  --(8.065,2.805)--(8.066,2.803)--(8.067,2.801)--(8.067,2.800)--(8.067,2.799)--(8.068,2.797)%
  --(8.068,2.795)--(8.069,2.791)--(8.069,2.790)--(8.070,2.787)--(8.070,2.784)--(8.070,2.781)%
  --(8.071,2.777)--(8.071,2.775)--(8.072,2.774)--(8.072,2.771)--(8.073,2.767)--(8.073,2.766)%
  --(8.074,2.764)--(8.074,2.760)--(8.074,2.746)--(8.075,2.743)--(8.075,2.742)--(8.075,2.738)%
  --(8.076,2.735)--(8.076,2.734)--(8.077,2.732)--(8.077,2.731)--(8.077,2.730)--(8.078,2.728)%
  --(8.078,2.729)--(8.079,2.727)--(8.079,2.726)--(8.079,2.724)--(8.080,2.712)--(8.080,2.706)%
  --(8.081,2.703)--(8.081,2.701)--(8.081,2.699)--(8.082,2.700)--(8.082,2.698)--(8.083,2.692)%
  --(8.083,2.688)--(8.084,2.683)--(8.084,2.680)--(8.084,2.676)--(8.085,2.673)--(8.085,2.667)%
  --(8.086,2.664)--(8.086,2.661)--(8.086,2.656)--(8.087,2.651)--(8.087,2.645)--(8.087,2.644)%
  --(8.088,2.640)--(8.088,2.639)--(8.089,2.633)--(8.089,2.629)--(8.089,2.627)--(8.090,2.625)%
  --(8.090,2.622)--(8.091,2.617)--(8.091,2.614)--(8.091,2.610)--(8.092,2.608)--(8.092,2.606)%
  --(8.093,2.604)--(8.093,2.600)--(8.094,2.600)--(8.094,2.597)--(8.094,2.594)--(8.095,2.590)%
  --(8.095,2.588)--(8.096,2.586)--(8.096,2.585)--(8.096,2.583)--(8.097,2.580)--(8.097,2.567)%
  --(8.098,2.566)--(8.098,2.564)--(8.099,2.564)--(8.099,2.562)--(8.100,2.559)--(8.100,2.557)%
  --(8.100,2.555)--(8.101,2.554)--(8.101,2.553)--(8.101,2.551)--(8.102,2.550)--(8.102,2.549)%
  --(8.103,2.548)--(8.103,2.547)--(8.103,2.541)--(8.104,2.532)--(8.104,2.527)--(8.105,2.527)%
  --(8.105,2.526)--(8.105,2.524)--(8.106,2.520)--(8.106,2.517)--(8.107,2.513)--(8.107,2.509)%
  --(8.107,2.506)--(8.108,2.504)--(8.108,2.498)--(8.108,2.495)--(8.109,2.493)--(8.109,2.490)%
  --(8.110,2.486)--(8.110,2.480)--(8.110,2.475)--(8.111,2.469)--(8.111,2.467)--(8.112,2.464)%
  --(8.112,2.458)--(8.112,2.454)--(8.113,2.452)--(8.113,2.447)--(8.114,2.444)--(8.114,2.440)%
  --(8.114,2.438)--(8.115,2.435)--(8.115,2.433)--(8.116,2.431)--(8.116,2.427)--(8.117,2.426)%
  --(8.117,2.425)--(8.117,2.423)--(8.118,2.422)--(8.118,2.421)--(8.119,2.421)--(8.119,2.418)%
  --(8.120,2.410)--(8.120,2.405)--(8.121,2.404)--(8.121,2.402)--(8.121,2.401)--(8.122,2.401)%
  --(8.122,2.397)--(8.122,2.396)--(8.123,2.392)--(8.123,2.391)--(8.124,2.390)--(8.124,2.389)%
  --(8.125,2.389)--(8.125,2.388)--(8.126,2.387)--(8.126,2.386)--(8.127,2.384)--(8.127,2.382)%
  --(8.127,2.380)--(8.128,2.377)--(8.128,2.375)--(8.129,2.373)--(8.129,2.371)--(8.129,2.369)%
  --(8.130,2.368)--(8.130,2.366)--(8.131,2.363)--(8.131,2.360)--(8.131,2.358)--(8.132,2.358)%
  --(8.132,2.356)--(8.133,2.355)--(8.133,2.353)--(8.133,2.351)--(8.134,2.350)--(8.134,2.347)%
  --(8.134,2.344)--(8.135,2.341)--(8.135,2.339)--(8.136,2.337)--(8.136,2.336)--(8.136,2.334)%
  --(8.137,2.332)--(8.137,2.331)--(8.138,2.328)--(8.138,2.326)--(8.138,2.323)--(8.139,2.321)%
  --(8.139,2.318)--(8.140,2.315)--(8.140,2.312)--(8.141,2.310)--(8.141,2.308)--(8.141,2.305)%
  --(8.142,2.303)--(8.142,2.301)--(8.143,2.299)--(8.143,2.297)--(8.143,2.296)--(8.144,2.292)%
  --(8.144,2.290)--(8.145,2.287)--(8.145,2.286)--(8.145,2.284)--(8.146,2.283)--(8.146,2.281)%
  --(8.147,2.278)--(8.147,2.275)--(8.147,2.273)--(8.148,2.271)--(8.148,2.270)--(8.148,2.269)%
  --(8.149,2.267)--(8.149,2.265)--(8.150,2.262)--(8.150,2.261)--(8.150,2.260)--(8.151,2.256)%
  --(8.151,2.254)--(8.152,2.250)--(8.152,2.249)--(8.152,2.245)--(8.153,2.242)--(8.153,2.241)%
  --(8.154,2.240)--(8.154,2.237)--(8.154,2.236)--(8.155,2.234)--(8.155,2.233)--(8.155,2.232)%
  --(8.156,2.230)--(8.156,2.226)--(8.157,2.224)--(8.157,2.220)--(8.157,2.219)--(8.158,2.216)%
  --(8.158,2.214)--(8.159,2.212)--(8.159,2.209)--(8.159,2.207)--(8.160,2.204)--(8.160,2.202)%
  --(8.161,2.201)--(8.161,2.200)--(8.162,2.198)--(8.162,2.196)--(8.162,2.194)--(8.163,2.191)%
  --(8.163,2.189)--(8.164,2.187)--(8.164,2.185)--(8.164,2.184)--(8.165,2.182)--(8.165,2.179)%
  --(8.166,2.176)--(8.166,2.173)--(8.166,2.170)--(8.167,2.168)--(8.167,2.166)--(8.167,2.164)%
  --(8.168,2.162)--(8.168,2.159)--(8.169,2.158)--(8.169,2.157)--(8.169,2.155)--(8.170,2.150)%
  --(8.170,2.149)--(8.171,2.146)--(8.171,2.147)--(8.171,2.145)--(8.172,2.143)--(8.172,2.140)%
  --(8.173,2.139)--(8.173,2.136)--(8.173,2.134)--(8.174,2.134)--(8.174,2.132)--(8.175,2.130)%
  --(8.175,2.125)--(8.176,2.123)--(8.176,2.122)--(8.176,2.120)--(8.177,2.119)--(8.177,2.117)%
  --(8.178,2.114)--(8.178,2.113)--(8.178,2.110)--(8.179,2.107)--(8.179,2.105)--(8.180,2.102)%
  --(8.180,2.100)--(8.181,2.098)--(8.181,2.095)--(8.181,2.094)--(8.182,2.092)--(8.182,2.088)%
  --(8.183,2.087)--(8.183,2.085)--(8.183,2.084)--(8.184,2.081)--(8.184,2.080)--(8.185,2.079)%
  --(8.185,2.077)--(8.185,2.078)--(8.186,2.076)--(8.186,2.074)--(8.187,2.072)--(8.187,2.071)%
  --(8.188,2.069)--(8.188,2.067)--(8.188,2.065)--(8.189,2.062)--(8.189,2.061)--(8.190,2.057)%
  --(8.190,2.055)--(8.190,2.053)--(8.191,2.051)--(8.191,2.049)--(8.192,2.048)--(8.192,2.045)%
  --(8.192,2.042)--(8.193,2.041)--(8.193,2.038)--(8.193,2.036)--(8.194,2.034)--(8.195,2.032)%
  --(8.195,2.030)--(8.195,2.026)--(8.196,2.023)--(8.196,2.020)--(8.197,2.019)--(8.197,2.016)%
  --(8.197,2.015)--(8.198,2.015)--(8.198,2.013)--(8.199,2.011)--(8.199,2.009)--(8.199,2.007)%
  --(8.200,2.005)--(8.200,2.004)--(8.200,2.003)--(8.201,2.001)--(8.201,2.000)--(8.202,1.997)%
  --(8.202,1.995)--(8.203,1.993)--(8.203,1.992)--(8.204,1.990)--(8.204,1.987)--(8.205,1.984)%
  --(8.205,1.983)--(8.206,1.982)--(8.206,1.979)--(8.206,1.976)--(8.207,1.975)--(8.207,1.974)%
  --(8.207,1.973)--(8.208,1.973)--(8.208,1.969)--(8.209,1.966)--(8.209,1.965)--(8.209,1.962)%
  --(8.210,1.962)--(8.210,1.959)--(8.211,1.958)--(8.211,1.956)--(8.211,1.955)--(8.212,1.954)%
  --(8.212,1.953)--(8.213,1.950)--(8.213,1.947)--(8.213,1.945)--(8.214,1.944)--(8.214,1.943)%
  --(8.214,1.942)--(8.215,1.942)--(8.215,1.941)--(8.216,1.941)--(8.216,1.937)--(8.216,1.936)%
  --(8.217,1.936)--(8.217,1.934)--(8.218,1.931)--(8.218,1.929)--(8.218,1.927)--(8.219,1.925)%
  --(8.219,1.923)--(8.220,1.922)--(8.220,1.920)--(8.221,1.918)--(8.221,1.917)--(8.222,1.915)%
  --(8.222,1.913)--(8.223,1.911)--(8.223,1.909)--(8.223,1.907)--(8.224,1.906)--(8.224,1.904)%
  --(8.225,1.904)--(8.225,1.903)--(8.225,1.902)--(8.226,1.900)--(8.227,1.898)--(8.228,1.896)%
  --(8.228,1.895)--(8.228,1.894)--(8.229,1.892)--(8.229,1.889)--(8.230,1.888)--(8.230,1.887)%
  --(8.230,1.886)--(8.231,1.884)--(8.231,1.883)--(8.232,1.881)--(8.232,1.879)--(8.232,1.878)%
  --(8.233,1.878)--(8.234,1.877)--(8.234,1.876)--(8.235,1.874)--(8.235,1.873)--(8.235,1.871)%
  --(8.236,1.870)--(8.236,1.868)--(8.237,1.868)--(8.237,1.866)--(8.237,1.867)--(8.238,1.867)%
  --(8.238,1.866)--(8.239,1.864)--(8.239,1.862)--(8.239,1.860)--(8.240,1.858)--(8.240,1.857)%
  --(8.241,1.856)--(8.241,1.855)--(8.242,1.854)--(8.242,1.851)--(8.242,1.849)--(8.243,1.847)%
  --(8.243,1.845)--(8.244,1.842)--(8.244,1.840)--(8.244,1.838)--(8.245,1.837)--(8.245,1.834)%
  --(8.246,1.831)--(8.246,1.828)--(8.246,1.827)--(8.247,1.825)--(8.247,1.823)--(8.247,1.822)%
  --(8.248,1.821)--(8.248,1.820)--(8.249,1.820)--(8.249,1.817)--(8.249,1.816)--(8.250,1.815)%
  --(8.251,1.814)--(8.251,1.813)--(8.252,1.812)--(8.253,1.810)--(8.253,1.808)--(8.253,1.807)%
  --(8.254,1.803)--(8.254,1.802)--(8.254,1.801)--(8.255,1.800)--(8.255,1.799)--(8.256,1.798)%
  --(8.256,1.797)--(8.256,1.796)--(8.257,1.794)--(8.257,1.793)--(8.258,1.791)--(8.258,1.789)%
  --(8.258,1.787)--(8.259,1.785)--(8.259,1.784)--(8.260,1.783)--(8.260,1.782)--(8.260,1.780)%
  --(8.261,1.780)--(8.262,1.778)--(8.262,1.777)--(8.263,1.776)--(8.263,1.775)--(8.263,1.773)%
  --(8.264,1.771)--(8.264,1.768)--(8.265,1.766)--(8.265,1.764)--(8.266,1.764)--(8.266,1.762)%
  --(8.267,1.761)--(8.267,1.759)--(8.268,1.758)--(8.268,1.757)--(8.268,1.756)--(8.269,1.756)%
  --(8.269,1.755)--(8.270,1.753)--(8.270,1.751)--(8.271,1.751)--(8.271,1.749)--(8.272,1.746)%
  --(8.272,1.745)--(8.272,1.743)--(8.273,1.742)--(8.273,1.741)--(8.273,1.739)--(8.274,1.738)%
  --(8.274,1.735)--(8.275,1.734)--(8.275,1.732)--(8.275,1.731)--(8.276,1.730)--(8.276,1.728)%
  --(8.277,1.728)--(8.277,1.727)--(8.278,1.727)--(8.278,1.728)--(8.279,1.725)--(8.280,1.723)%
  --(8.280,1.722)--(8.281,1.722)--(8.281,1.720)--(8.282,1.720)--(8.282,1.719)--(8.282,1.717)%
  --(8.283,1.716)--(8.283,1.714)--(8.284,1.713)--(8.284,1.712)--(8.284,1.711)--(8.285,1.709)%
  --(8.285,1.708)--(8.286,1.708)--(8.286,1.707)--(8.286,1.706)--(8.287,1.704)--(8.287,1.701)%
  --(8.287,1.700)--(8.288,1.699)--(8.288,1.698)--(8.289,1.697)--(8.289,1.695)--(8.290,1.694)%
  --(8.290,1.691)--(8.291,1.690)--(8.291,1.689)--(8.292,1.688)--(8.292,1.686)--(8.293,1.685)%
  --(8.293,1.684)--(8.293,1.683)--(8.294,1.683)--(8.294,1.682)--(8.294,1.680)--(8.295,1.679)%
  --(8.295,1.678)--(8.296,1.677)--(8.296,1.675)--(8.296,1.674)--(8.297,1.673)--(8.298,1.673)%
  --(8.298,1.671)--(8.298,1.670)--(8.299,1.668)--(8.299,1.666)--(8.299,1.665)--(8.300,1.664)%
  --(8.300,1.663)--(8.301,1.662)--(8.301,1.661)--(8.301,1.658)--(8.302,1.657)--(8.302,1.655)%
  --(8.303,1.654)--(8.303,1.653)--(8.303,1.652)--(8.304,1.652)--(8.305,1.651)--(8.305,1.648)%
  --(8.305,1.646)--(8.306,1.644)--(8.306,1.643)--(8.306,1.642)--(8.307,1.641)--(8.308,1.640)%
  --(8.308,1.639)--(8.308,1.637)--(8.309,1.636)--(8.309,1.634)--(8.310,1.633)--(8.310,1.631)%
  --(8.310,1.630)--(8.311,1.629)--(8.312,1.628)--(8.312,1.627)--(8.312,1.625)--(8.313,1.623)%
  --(8.313,1.622)--(8.313,1.621)--(8.314,1.621)--(8.314,1.620)--(8.315,1.618)--(8.315,1.617)%
  --(8.316,1.615)--(8.316,1.614)--(8.317,1.612)--(8.317,1.611)--(8.317,1.609)--(8.318,1.608)%
  --(8.318,1.607)--(8.319,1.606)--(8.319,1.605)--(8.319,1.604)--(8.320,1.602)--(8.320,1.599)%
  --(8.320,1.597)--(8.321,1.597)--(8.321,1.595)--(8.322,1.594)--(8.322,1.592)--(8.323,1.590)%
  --(8.323,1.588)--(8.324,1.586)--(8.324,1.583)--(8.324,1.580)--(8.325,1.579)--(8.325,1.578)%
  --(8.326,1.578)--(8.326,1.577)--(8.327,1.574)--(8.327,1.573)--(8.327,1.572)--(8.328,1.568)%
  --(8.328,1.567)--(8.329,1.567)--(8.329,1.566)--(8.329,1.565)--(8.330,1.563)--(8.330,1.562)%
  --(8.331,1.562)--(8.331,1.560)--(8.332,1.560)--(8.333,1.559)--(8.333,1.557)--(8.334,1.555)%
  --(8.334,1.553)--(8.335,1.551)--(8.335,1.549)--(8.336,1.549)--(8.336,1.547)--(8.336,1.545)%
  --(8.337,1.545)--(8.337,1.544)--(8.338,1.541)--(8.338,1.540)--(8.338,1.538)--(8.339,1.537)%
  --(8.339,1.535)--(8.340,1.535)--(8.340,1.534)--(8.341,1.533)--(8.341,1.532)--(8.341,1.531)%
  --(8.342,1.529)--(8.342,1.528)--(8.343,1.527)--(8.343,1.525)--(8.344,1.523)--(8.344,1.522)%
  --(8.345,1.522)--(8.345,1.521)--(8.345,1.519)--(8.346,1.518)--(8.346,1.517)--(8.346,1.516)%
  --(8.347,1.515)--(8.347,1.514)--(8.348,1.512)--(8.348,1.511)--(8.348,1.510)--(8.349,1.509)%
  --(8.349,1.508)--(8.350,1.507)--(8.350,1.506)--(8.351,1.504)--(8.351,1.503)--(8.352,1.501)%
  --(8.352,1.499)--(8.353,1.498)--(8.353,1.497)--(8.353,1.496)--(8.354,1.495)--(8.354,1.494)%
  --(8.355,1.494)--(8.355,1.493)--(8.356,1.492)--(8.356,1.490)--(8.357,1.490)--(8.357,1.489)%
  --(8.357,1.488)--(8.358,1.487)--(8.359,1.486)--(8.359,1.485)--(8.360,1.482)--(8.360,1.481)%
  --(8.361,1.481)--(8.361,1.486)--(8.362,1.486)--(8.362,1.487)--(8.363,1.486)--(8.363,1.488)%
  --(8.364,1.487)--(8.364,1.488)--(8.364,1.489)--(8.365,1.488)--(8.365,1.489)--(8.366,1.489)%
  --(8.366,1.490)--(8.367,1.491)--(8.367,1.492)--(8.368,1.493)--(8.369,1.492)--(8.369,1.493)%
  --(8.370,1.493)--(8.371,1.494)--(8.372,1.495)--(8.372,1.497)--(8.373,1.499)--(8.374,1.500)%
  --(8.374,1.501)--(8.374,1.502)--(8.375,1.502)--(8.376,1.502)--(8.376,1.501)--(8.376,1.503)%
  --(8.377,1.503)--(8.378,1.504)--(8.379,1.505)--(8.379,1.506)--(8.380,1.508)--(8.381,1.508)%
  --(8.381,1.511)--(8.381,1.512)--(8.382,1.513)--(8.382,1.515)--(8.383,1.516)--(8.384,1.518)%
  --(8.385,1.519)--(8.385,1.520)--(8.386,1.522)--(8.386,1.523)--(8.386,1.524)--(8.387,1.524)%
  --(8.387,1.525)--(8.388,1.525)--(8.388,1.526)--(8.388,1.527)--(8.389,1.528)--(8.389,1.529)%
  --(8.390,1.530)--(8.390,1.532)--(8.391,1.533)--(8.391,1.534)--(8.392,1.535)--(8.392,1.536)%
  --(8.392,1.537)--(8.393,1.539)--(8.394,1.540)--(8.395,1.542)--(8.395,1.543)--(8.395,1.544)%
  --(8.396,1.545)--(8.396,1.546)--(8.397,1.547)--(8.397,1.549)--(8.397,1.551)--(8.398,1.552)%
  --(8.398,1.553)--(8.399,1.553)--(8.399,1.554)--(8.399,1.555)--(8.400,1.556)--(8.400,1.557)%
  --(8.400,1.558)--(8.401,1.559)--(8.401,1.560)--(8.402,1.560)--(8.402,1.561)--(8.402,1.563)%
  --(8.403,1.563)--(8.403,1.564)--(8.404,1.564)--(8.404,1.566)--(8.404,1.567)--(8.405,1.568)%
  --(8.405,1.569)--(8.405,1.570)--(8.406,1.571)--(8.406,1.572)--(8.407,1.573)--(8.407,1.574)%
  --(8.407,1.575)--(8.408,1.576)--(8.408,1.579)--(8.409,1.580)--(8.409,1.582)--(8.409,1.583)%
  --(8.410,1.585)--(8.410,1.586)--(8.411,1.588)--(8.411,1.589)--(8.411,1.590)--(8.412,1.592)%
  --(8.412,1.594)--(8.413,1.595)--(8.414,1.596)--(8.414,1.595)--(8.414,1.596)--(8.415,1.597)%
  --(8.415,1.598)--(8.416,1.599)--(8.416,1.600)--(8.417,1.600)--(8.418,1.602)--(8.418,1.603)%
  --(8.419,1.604)--(8.419,1.606)--(8.419,1.607)--(8.420,1.609)--(8.420,1.610)--(8.421,1.611)%
  --(8.421,1.612)--(8.421,1.614)--(8.422,1.614)--(8.422,1.615)--(8.423,1.617)--(8.424,1.619)%
  --(8.424,1.620)--(8.425,1.621)--(8.425,1.622)--(8.426,1.623)--(8.426,1.624)--(8.426,1.625)%
  --(8.427,1.626)--(8.427,1.627)--(8.428,1.629)--(8.428,1.631)--(8.428,1.633)--(8.429,1.634)%
  --(8.429,1.635)--(8.430,1.635)--(8.430,1.636)--(8.431,1.638)--(8.431,1.639)--(8.432,1.640)%
  --(8.432,1.641)--(8.433,1.642)--(8.434,1.644)--(8.434,1.645)--(8.435,1.646)--(8.435,1.647)%
  --(8.435,1.648)--(8.436,1.648)--(8.436,1.649)--(8.437,1.650)--(8.437,1.652)--(8.438,1.652)%
  --(8.438,1.654)--(8.439,1.656)--(8.439,1.657)--(8.439,1.658)--(8.440,1.660)--(8.440,1.661)%
  --(8.440,1.662)--(8.441,1.663)--(8.441,1.665)--(8.442,1.666)--(8.442,1.668)--(8.443,1.668)%
  --(8.443,1.669)--(8.444,1.669)--(8.444,1.670)--(8.444,1.671)--(8.445,1.672)--(8.445,1.673)%
  --(8.445,1.674)--(8.446,1.675)--(8.446,1.677)--(8.447,1.678)--(8.447,1.679)--(8.447,1.681)%
  --(8.448,1.681)--(8.448,1.682)--(8.449,1.682)--(8.449,1.681)--(8.449,1.684)--(8.450,1.683)%
  --(8.451,1.684)--(8.451,1.686)--(8.451,1.687)--(8.452,1.689)--(8.452,1.690)--(8.452,1.692)%
  --(8.453,1.693)--(8.453,1.694)--(8.454,1.694)--(8.454,1.695)--(8.454,1.697)--(8.455,1.697)%
  --(8.455,1.698)--(8.456,1.701)--(8.456,1.704)--(8.456,1.705)--(8.457,1.708)--(8.457,1.707)%
  --(8.458,1.708)--(8.458,1.710)--(8.459,1.712)--(8.459,1.713)--(8.460,1.713)--(8.460,1.716)%
  --(8.461,1.716)--(8.462,1.717)--(8.462,1.718)--(8.463,1.719)--(8.463,1.721)--(8.463,1.722)%
  --(8.464,1.723)--(8.464,1.725)--(8.465,1.729)--(8.465,1.731)--(8.466,1.733)--(8.467,1.733)%
  --(8.467,1.734)--(8.468,1.734)--(8.468,1.736)--(8.469,1.736)--(8.470,1.736)--(8.470,1.737)%
  --(8.471,1.739)--(8.471,1.740)--(8.472,1.741)--(8.472,1.742)--(8.473,1.745)--(8.473,1.747)%
  --(8.473,1.751)--(8.474,1.753)--(8.474,1.755)--(8.475,1.755)--(8.475,1.756)--(8.475,1.758)%
  --(8.476,1.759)--(8.476,1.762)--(8.477,1.763)--(8.477,1.762)--(8.477,1.764)--(8.478,1.765)%
  --(8.478,1.766)--(8.479,1.767)--(8.480,1.768)--(8.480,1.769)--(8.480,1.771)--(8.481,1.772)%
  --(8.481,1.774)--(8.482,1.778)--(8.482,1.780)--(8.482,1.781)--(8.483,1.783)--(8.484,1.786)%
  --(8.484,1.789)--(8.484,1.790)--(8.485,1.792)--(8.485,1.794)--(8.486,1.796)--(8.486,1.798)%
  --(8.487,1.801)--(8.487,1.802)--(8.488,1.803)--(8.488,1.805)--(8.489,1.807)--(8.489,1.810)%
  --(8.489,1.812)--(8.490,1.813)--(8.490,1.816)--(8.491,1.818)--(8.491,1.819)--(8.492,1.820)%
  --(8.492,1.824)--(8.492,1.827)--(8.493,1.828)--(8.493,1.829)--(8.494,1.829)--(8.494,1.830)%
  --(8.494,1.832)--(8.495,1.836)--(8.495,1.838)--(8.496,1.838)--(8.496,1.839)--(8.496,1.840)%
  --(8.497,1.843)--(8.497,1.844)--(8.498,1.847)--(8.498,1.853)--(8.498,1.854)--(8.499,1.855)%
  --(8.500,1.855)--(8.501,1.855)--(8.502,1.856)--(8.503,1.857)--(8.503,1.858)--(8.503,1.859)%
  --(8.504,1.861)--(8.504,1.862)--(8.505,1.863)--(8.505,1.864)--(8.505,1.865)--(8.506,1.865)%
  --(8.506,1.866)--(8.507,1.866)--(8.507,1.869)--(8.508,1.869)--(8.508,1.871)--(8.509,1.874)%
  --(8.509,1.877)--(8.510,1.879)--(8.510,1.882)--(8.510,1.884)--(8.511,1.886)--(8.511,1.887)%
  --(8.511,1.889)--(8.512,1.890)--(8.512,1.893)--(8.513,1.894)--(8.513,1.896)--(8.514,1.897)%
  --(8.514,1.900)--(8.515,1.902)--(8.515,1.904)--(8.516,1.906)--(8.517,1.907)--(8.517,1.908)%
  --(8.517,1.910)--(8.518,1.910)--(8.518,1.912)--(8.518,1.913)--(8.519,1.914)--(8.519,1.916)%
  --(8.520,1.917)--(8.520,1.918)--(8.521,1.920)--(8.521,1.922)--(8.522,1.922)--(8.522,1.924)%
  --(8.523,1.925)--(8.523,1.926)--(8.524,1.927)--(8.524,1.928)--(8.525,1.929)--(8.525,1.928)%
  --(8.526,1.928)--(8.526,1.929)--(8.527,1.929)--(8.527,1.930)--(8.527,1.929)--(8.528,1.930)%
  --(8.529,1.930)--(8.529,1.932)--(8.530,1.933)--(8.530,1.935)--(8.531,1.940)--(8.531,1.945)%
  --(8.531,1.946)--(8.532,1.950)--(8.532,1.951)--(8.533,1.953)--(8.533,1.954)--(8.534,1.955)%
  --(8.534,1.956)--(8.534,1.957)--(8.535,1.958)--(8.535,1.959)--(8.536,1.961)--(8.536,1.962)%
  --(8.536,1.964)--(8.537,1.965)--(8.537,1.966)--(8.538,1.968)--(8.539,1.968)--(8.539,1.970)%
  --(8.539,1.971)--(8.540,1.972)--(8.540,1.974)--(8.541,1.974)--(8.541,1.975)--(8.541,1.976)%
  --(8.542,1.979)--(8.542,1.978)--(8.543,1.979)--(8.543,1.980)--(8.544,1.981)--(8.544,1.983)%
  --(8.545,1.984)--(8.545,1.985)--(8.546,1.986)--(8.546,1.987)--(8.547,1.988)--(8.547,1.989)%
  --(8.548,1.991)--(8.548,1.993)--(8.549,1.994)--(8.549,1.996)--(8.550,1.997)--(8.550,1.999)%
  --(8.550,2.002)--(8.551,2.004)--(8.551,2.006)--(8.551,2.009)--(8.552,2.010)--(8.552,2.012)%
  --(8.553,2.012)--(8.553,2.013)--(8.553,2.014)--(8.554,2.015)--(8.555,2.015)--(8.555,2.017)%
  --(8.556,2.019)--(8.556,2.021)--(8.557,2.023)--(8.557,2.024)--(8.557,2.025)--(8.558,2.028)%
  --(8.558,2.029)--(8.558,2.030)--(8.559,2.032)--(8.560,2.034)--(8.560,2.035)--(8.560,2.037)%
  --(8.561,2.039)--(8.561,2.040)--(8.562,2.042)--(8.562,2.043)--(8.563,2.045)--(8.564,2.046)%
  --(8.564,2.048)--(8.565,2.049)--(8.565,2.050)--(8.566,2.050)--(8.566,2.052)--(8.567,2.052)%
  --(8.567,2.053)--(8.567,2.055)--(8.568,2.058)--(8.568,2.059)--(8.569,2.060)--(8.569,2.062)%
  --(8.569,2.063)--(8.570,2.064)--(8.570,2.066)--(8.571,2.068)--(8.571,2.070)--(8.571,2.071)%
  --(8.572,2.073)--(8.572,2.074)--(8.572,2.077)--(8.573,2.078)--(8.573,2.080)--(8.574,2.081)%
  --(8.574,2.083)--(8.574,2.084)--(8.575,2.085)--(8.575,2.087)--(8.576,2.087)--(8.576,2.090)%
  --(8.576,2.091)--(8.577,2.092)--(8.577,2.093)--(8.578,2.095)--(8.578,2.097)--(8.579,2.098)%
  --(8.579,2.100)--(8.579,2.102)--(8.580,2.104)--(8.581,2.105)--(8.581,2.106)--(8.582,2.107)%
  --(8.583,2.108)--(8.583,2.110)--(8.583,2.112)--(8.584,2.113)--(8.584,2.115)--(8.584,2.117)%
  --(8.585,2.119)--(8.585,2.121)--(8.586,2.122)--(8.586,2.123)--(8.586,2.124)--(8.587,2.125)%
  --(8.587,2.128)--(8.588,2.130)--(8.589,2.131)--(8.589,2.133)--(8.590,2.135)--(8.590,2.136)%
  --(8.591,2.137)--(8.591,2.140)--(8.591,2.142)--(8.592,2.145)--(8.592,2.146)--(8.593,2.148)%
  --(8.593,2.151)--(8.594,2.151)--(8.594,2.152)--(8.595,2.154)--(8.595,2.155)--(8.596,2.156)%
  --(8.596,2.158)--(8.597,2.158)--(8.597,2.159)--(8.598,2.160)--(8.598,2.159)--(8.598,2.161)%
  --(8.599,2.163)--(8.599,2.165)--(8.600,2.167)--(8.600,2.168)--(8.601,2.173)--(8.601,2.175)%
  --(8.602,2.176)--(8.602,2.177)--(8.602,2.178)--(8.603,2.179)--(8.604,2.180)--(8.604,2.181)%
  --(8.604,2.183)--(8.605,2.183)--(8.605,2.184)--(8.605,2.188)--(8.606,2.188)--(8.606,2.189)%
  --(8.607,2.190)--(8.607,2.193)--(8.607,2.194)--(8.608,2.195)--(8.608,2.197)--(8.609,2.199)%
  --(8.609,2.201)--(8.609,2.202)--(8.610,2.203)--(8.611,2.205)--(8.611,2.207)--(8.611,2.209)%
  --(8.612,2.210)--(8.612,2.212)--(8.612,2.213)--(8.613,2.213)--(8.613,2.214)--(8.614,2.214)%
  --(8.614,2.215)--(8.614,2.217)--(8.615,2.217)--(8.615,2.219)--(8.616,2.222)--(8.616,2.220)%
  --(8.616,2.221)--(8.617,2.223)--(8.617,2.224)--(8.617,2.226)--(8.618,2.227)--(8.619,2.229)%
  --(8.619,2.230)--(8.619,2.232)--(8.620,2.234)--(8.620,2.235)--(8.621,2.243)--(8.621,2.245)%
  --(8.621,2.246)--(8.622,2.248)--(8.622,2.249)--(8.623,2.251)--(8.623,2.253)--(8.623,2.255)%
  --(8.624,2.258)--(8.624,2.261)--(8.624,2.263)--(8.625,2.264)--(8.625,2.266)--(8.626,2.267)%
  --(8.626,2.270)--(8.626,2.271)--(8.627,2.273)--(8.627,2.274)--(8.628,2.275)--(8.628,2.276)%
  --(8.629,2.277)--(8.629,2.278)--(8.630,2.279)--(8.630,2.281)--(8.631,2.283)--(8.631,2.285)%
  --(8.631,2.292)--(8.632,2.293)--(8.632,2.295)--(8.633,2.295)--(8.633,2.297)--(8.634,2.298)%
  --(8.634,2.299)--(8.635,2.301)--(8.636,2.301)--(8.636,2.303)--(8.637,2.304)--(8.637,2.305)%
  --(8.637,2.307)--(8.638,2.309)--(8.638,2.310)--(8.638,2.311)--(8.639,2.313)--(8.639,2.314)%
  --(8.640,2.315)--(8.640,2.317)--(8.640,2.318)--(8.641,2.320)--(8.641,2.323)--(8.642,2.330)%
  --(8.642,2.332)--(8.642,2.335)--(8.643,2.335)--(8.643,2.336)--(8.644,2.337)--(8.644,2.338)%
  --(8.644,2.340)--(8.645,2.341)--(8.645,2.343)--(8.645,2.348)--(8.646,2.356)--(8.646,2.357)%
  --(8.647,2.358)--(8.647,2.359)--(8.647,2.360)--(8.648,2.362)--(8.649,2.363)--(8.649,2.366)%
  --(8.650,2.368)--(8.650,2.370)--(8.651,2.372)--(8.651,2.374)--(8.651,2.376)--(8.652,2.378)%
  --(8.652,2.380)--(8.652,2.382)--(8.653,2.385)--(8.653,2.386)--(8.654,2.387)--(8.654,2.391)%
  --(8.654,2.392)--(8.655,2.394)--(8.655,2.397)--(8.656,2.398)--(8.656,2.401)--(8.656,2.403)%
  --(8.657,2.404)--(8.657,2.406)--(8.657,2.407)--(8.658,2.408)--(8.658,2.407)--(8.659,2.409)%
  --(8.659,2.410)--(8.659,2.411)--(8.660,2.413)--(8.660,2.415)--(8.661,2.416)--(8.661,2.420)%
  --(8.661,2.423)--(8.662,2.426)--(8.662,2.427)--(8.663,2.431)--(8.663,2.432)--(8.664,2.434)%
  --(8.665,2.436)--(8.665,2.445)--(8.666,2.447)--(8.666,2.448)--(8.666,2.450)--(8.667,2.451)%
  --(8.667,2.452)--(8.668,2.453)--(8.668,2.454)--(8.668,2.456)--(8.669,2.456)--(8.669,2.458)%
  --(8.670,2.458)--(8.670,2.459)--(8.670,2.460)--(8.671,2.469)--(8.671,2.475)--(8.672,2.477)%
  --(8.672,2.478)--(8.673,2.478)--(8.673,2.481)--(8.673,2.482)--(8.674,2.486)--(8.674,2.490)%
  --(8.675,2.493)--(8.675,2.497)--(8.675,2.499)--(8.676,2.501)--(8.676,2.504)--(8.677,2.506)%
  --(8.677,2.510)--(8.677,2.514)--(8.678,2.519)--(8.678,2.524)--(8.678,2.528)--(8.679,2.530)%
  --(8.679,2.531)--(8.680,2.537)--(8.680,2.538)--(8.680,2.541)--(8.681,2.544)--(8.681,2.547)%
  --(8.682,2.551)--(8.682,2.554)--(8.682,2.559)--(8.683,2.560)--(8.683,2.565)--(8.684,2.566)%
  --(8.684,2.567)--(8.684,2.569)--(8.685,2.569)--(8.685,2.572)--(8.685,2.577)--(8.686,2.582)%
  --(8.686,2.585)--(8.687,2.585)--(8.687,2.589)--(8.687,2.591)--(8.688,2.594)--(8.688,2.603)%
  --(8.689,2.606)--(8.689,2.608)--(8.689,2.609)--(8.690,2.611)--(8.690,2.612)--(8.690,2.615)%
  --(8.691,2.618)--(8.691,2.620)--(8.692,2.621)--(8.692,2.622)--(8.693,2.624)--(8.693,2.627)%
  --(8.694,2.632)--(8.694,2.635)--(8.694,2.642)--(8.695,2.650)--(8.695,2.652)--(8.696,2.654)%
  --(8.696,2.655)--(8.697,2.660)--(8.697,2.662)--(8.697,2.666)--(8.698,2.668)--(8.698,2.674)%
  --(8.699,2.677)--(8.699,2.678)--(8.699,2.681)--(8.700,2.684)--(8.700,2.687)--(8.701,2.691)%
  --(8.701,2.696)--(8.701,2.702)--(8.702,2.707)--(8.702,2.711)--(8.703,2.715)--(8.703,2.720)%
  --(8.703,2.724)--(8.704,2.729)--(8.704,2.732)--(8.704,2.735)--(8.705,2.740)--(8.705,2.746)%
  --(8.706,2.751)--(8.706,2.753)--(8.706,2.752)--(8.707,2.755)--(8.707,2.757)--(8.708,2.759)%
  --(8.708,2.761)--(8.708,2.765)--(8.709,2.767)--(8.709,2.769)--(8.710,2.770)--(8.710,2.772)%
  --(8.710,2.774)--(8.711,2.781)--(8.711,2.788)--(8.711,2.790)--(8.712,2.793)--(8.712,2.796)%
  --(8.713,2.798)--(8.713,2.799)--(8.713,2.801)--(8.714,2.802)--(8.715,2.802)--(8.715,2.804)%
  --(8.716,2.807)--(8.717,2.809)--(8.717,2.813)--(8.717,2.812)--(8.718,2.814)--(8.718,2.817)%
  --(8.718,2.819)--(8.719,2.822)--(8.719,2.824)--(8.720,2.829)--(8.720,2.832)--(8.720,2.834)%
  --(8.721,2.836)--(8.721,2.838)--(8.722,2.839)--(8.722,2.842)--(8.722,2.845)--(8.723,2.848)%
  --(8.723,2.851)--(8.723,2.855)--(8.724,2.859)--(8.724,2.860)--(8.725,2.861)--(8.725,2.863)%
  --(8.725,2.865)--(8.726,2.868)--(8.726,2.872)--(8.727,2.874)--(8.727,2.875)--(8.727,2.879)%
  --(8.728,2.882)--(8.728,2.887)--(8.729,2.888)--(8.729,2.892)--(8.729,2.896)--(8.730,2.900)%
  --(8.730,2.903)--(8.730,2.910)--(8.731,2.911)--(8.731,2.915)--(8.732,2.915)--(8.732,2.918)%
  --(8.732,2.919)--(8.733,2.925)--(8.733,2.928)--(8.734,2.930)--(8.734,2.934)--(8.734,2.937)%
  --(8.735,2.943)--(8.735,2.945)--(8.736,2.949)--(8.736,2.951)--(8.736,2.953)--(8.737,2.957)%
  --(8.737,2.959)--(8.737,2.963)--(8.738,2.969)--(8.738,2.971)--(8.739,2.972)--(8.739,2.976)%
  --(8.739,2.979)--(8.740,2.982)--(8.740,2.984)--(8.741,2.988)--(8.741,2.994)--(8.741,2.999)%
  --(8.742,3.001)--(8.742,3.005)--(8.743,3.010)--(8.743,3.014)--(8.743,3.018)--(8.744,3.022)%
  --(8.744,3.026)--(8.744,3.030)--(8.745,3.033)--(8.745,3.037)--(8.746,3.038)--(8.746,3.042)%
  --(8.746,3.049)--(8.747,3.055)--(8.747,3.060)--(8.748,3.064)--(8.748,3.066)--(8.748,3.070)%
  --(8.749,3.074)--(8.749,3.076)--(8.750,3.079)--(8.750,3.083)--(8.750,3.087)--(8.751,3.091)%
  --(8.751,3.094)--(8.751,3.096)--(8.752,3.098)--(8.753,3.101)--(8.753,3.103)--(8.753,3.106)%
  --(8.754,3.110)--(8.754,3.114)--(8.755,3.117)--(8.755,3.121)--(8.755,3.124)--(8.756,3.126)%
  --(8.756,3.128)--(8.757,3.134)--(8.757,3.138)--(8.757,3.143)--(8.758,3.146)--(8.758,3.152)%
  --(8.758,3.154)--(8.759,3.159)--(8.759,3.162)--(8.760,3.166)--(8.760,3.171)--(8.760,3.174)%
  --(8.761,3.178)--(8.761,3.182)--(8.762,3.187)--(8.762,3.191)--(8.762,3.196)--(8.763,3.198)%
  --(8.763,3.202)--(8.763,3.207)--(8.764,3.209)--(8.764,3.213)--(8.765,3.216)--(8.765,3.217)%
  --(8.765,3.220)--(8.766,3.224)--(8.766,3.225)--(8.767,3.231)--(8.767,3.237)--(8.767,3.241)%
  --(8.768,3.245)--(8.768,3.248)--(8.769,3.253)--(8.769,3.256)--(8.769,3.261)--(8.770,3.267)%
  --(8.770,3.270)--(8.770,3.272)--(8.771,3.277)--(8.771,3.280)--(8.772,3.282)--(8.772,3.287)%
  --(8.772,3.291)--(8.773,3.299)--(8.773,3.302)--(8.774,3.308)--(8.774,3.310)--(8.774,3.312)%
  --(8.775,3.319)--(8.775,3.324)--(8.776,3.326)--(8.776,3.328)--(8.776,3.331)--(8.777,3.333)%
  --(8.777,3.335)--(8.777,3.338)--(8.778,3.340)--(8.778,3.341)--(8.779,3.344)--(8.779,3.348)%
  --(8.780,3.355)--(8.780,3.358)--(8.781,3.364)--(8.781,3.371)--(8.781,3.374)--(8.782,3.380)%
  --(8.782,3.387)--(8.783,3.392)--(8.783,3.397)--(8.783,3.402)--(8.784,3.409)--(8.784,3.412)%
  --(8.784,3.420)--(8.785,3.425)--(8.785,3.428)--(8.786,3.430)--(8.786,3.439)--(8.786,3.447)%
  --(8.787,3.453)--(8.787,3.459)--(8.788,3.464)--(8.788,3.467)--(8.788,3.469)--(8.789,3.473)%
  --(8.789,3.476)--(8.790,3.479)--(8.790,3.485)--(8.790,3.490)--(8.791,3.494)--(8.791,3.497)%
  --(8.791,3.503)--(8.792,3.506)--(8.792,3.510)--(8.793,3.515)--(8.793,3.519)--(8.793,3.522)%
  --(8.794,3.527)--(8.794,3.533)--(8.795,3.535)--(8.795,3.534)--(8.795,3.539)--(8.796,3.545)%
  --(8.796,3.551)--(8.796,3.557)--(8.797,3.562)--(8.797,3.568)--(8.798,3.575)--(8.798,3.576)%
  --(8.798,3.579)--(8.799,3.584)--(8.799,3.590)--(8.800,3.596)--(8.800,3.603)--(8.800,3.609)%
  --(8.801,3.610)--(8.801,3.614)--(8.802,3.620)--(8.802,3.626)--(8.802,3.632)--(8.803,3.636)%
  --(8.803,3.641)--(8.803,3.647)--(8.804,3.652)--(8.804,3.659)--(8.805,3.662)--(8.805,3.665)%
  --(8.806,3.668)--(8.806,3.672)--(8.807,3.676)--(8.807,3.682)--(8.807,3.687)--(8.808,3.690)%
  --(8.808,3.694)--(8.809,3.700)--(8.809,3.704)--(8.809,3.708)--(8.810,3.714)--(8.810,3.721)%
  --(8.810,3.727)--(8.811,3.734)--(8.811,3.739)--(8.812,3.741)--(8.812,3.744)--(8.812,3.746)%
  --(8.813,3.748)--(8.813,3.754)--(8.814,3.762)--(8.814,3.768)--(8.814,3.775)--(8.815,3.778)%
  --(8.815,3.780)--(8.816,3.784)--(8.816,3.785)--(8.816,3.789)--(8.817,3.795)--(8.817,3.797)%
  --(8.817,3.800)--(8.818,3.799)--(8.818,3.802)--(8.819,3.803)--(8.819,3.806)--(8.819,3.812)%
  --(8.820,3.816)--(8.820,3.823)--(8.821,3.828)--(8.821,3.831)--(8.822,3.839)--(8.822,3.843)%
  --(8.823,3.847)--(8.823,3.853)--(8.824,3.855)--(8.824,3.856)--(8.824,3.858)--(8.825,3.861)%
  --(8.825,3.864)--(8.826,3.869)--(8.826,3.872)--(8.826,3.878)--(8.827,3.881)--(8.827,3.889)%
  --(8.828,3.895)--(8.828,3.899)--(8.828,3.902)--(8.829,3.903)--(8.829,3.905)--(8.829,3.910)%
  --(8.830,3.917)--(8.830,3.924)--(8.831,3.927)--(8.831,3.931)--(8.831,3.934)--(8.832,3.936)%
  --(8.832,3.941)--(8.833,3.949)--(8.833,3.958)--(8.833,3.968)--(8.834,3.977)--(8.834,3.983)%
  --(8.835,3.990)--(8.835,3.998)--(8.835,4.003)--(8.836,4.009)--(8.836,4.023)--(8.836,4.032)%
  --(8.837,4.037)--(8.837,4.043)--(8.838,4.051)--(8.838,4.056)--(8.838,4.064)--(8.839,4.071)%
  --(8.839,4.076)--(8.840,4.081)--(8.840,4.088)--(8.840,4.096)--(8.841,4.099)--(8.842,4.103)%
  --(8.842,4.104)--(8.842,4.108)--(8.843,4.112)--(8.843,4.116)--(8.843,4.126)--(8.844,4.128)%
  --(8.844,4.133)--(8.845,4.141)--(8.845,4.147)--(8.845,4.156)--(8.846,4.157)--(8.846,4.160)%
  --(8.847,4.166)--(8.847,4.171)--(8.847,4.175)--(8.848,4.179)--(8.848,4.187)--(8.849,4.194)%
  --(8.849,4.203)--(8.849,4.212)--(8.850,4.218)--(8.850,4.222)--(8.850,4.227)--(8.851,4.235)%
  --(8.851,4.238)--(8.852,4.240)--(8.852,4.245)--(8.852,4.252)--(8.853,4.251)--(8.853,4.254)%
  --(8.854,4.255)--(8.854,4.259)--(8.854,4.266)--(8.855,4.274)--(8.855,4.283)--(8.856,4.296)%
  --(8.856,4.304)--(8.856,4.310)--(8.857,4.311)--(8.857,4.316)--(8.857,4.321)--(8.858,4.330)%
  --(8.858,4.333)--(8.859,4.337)--(8.859,4.344)--(8.859,4.349)--(8.860,4.358)--(8.860,4.360)%
  --(8.861,4.366)--(8.861,4.369)--(8.861,4.376)--(8.862,4.382)--(8.862,4.385)--(8.863,4.394)%
  --(8.863,4.404)--(8.863,4.412)--(8.864,4.420)--(8.864,4.427)--(8.864,4.432)--(8.865,4.440)%
  --(8.865,4.451)--(8.866,4.460)--(8.866,4.468)--(8.866,4.473)--(8.867,4.481)--(8.867,4.488)%
  --(8.868,4.490)--(8.868,4.492)--(8.868,4.494)--(8.869,4.495)--(8.869,4.500)--(8.869,4.507)%
  --(8.870,4.509)--(8.870,4.513)--(8.871,4.514)--(8.871,4.519)--(8.871,4.523)--(8.872,4.528)%
  --(8.872,4.537)--(8.873,4.540)--(8.873,4.545)--(8.873,4.551)--(8.874,4.558)--(8.874,4.564)%
  --(8.875,4.573)--(8.875,4.577)--(8.875,4.582)--(8.876,4.589)--(8.876,4.597)--(8.876,4.601)%
  --(8.877,4.606)--(8.877,4.618)--(8.878,4.626)--(8.878,4.638)--(8.878,4.649)--(8.879,4.653)%
  --(8.879,4.657)--(8.880,4.661)--(8.880,4.669)--(8.880,4.675)--(8.881,4.682)--(8.881,4.687)%
  --(8.882,4.692)--(8.882,4.697)--(8.882,4.702)--(8.883,4.708)--(8.883,4.713)--(8.883,4.718)%
  --(8.884,4.726)--(8.884,4.729)--(8.885,4.733)--(8.885,4.735)--(8.885,4.740)--(8.886,4.748)%
  --(8.886,4.756)--(8.887,4.762)--(8.887,4.766)--(8.887,4.773)--(8.888,4.780)--(8.888,4.785)%
  --(8.889,4.788)--(8.889,4.791)--(8.889,4.794)--(8.890,4.805)--(8.890,4.817)--(8.890,4.827)%
  --(8.891,4.834)--(8.891,4.843)--(8.892,4.855)--(8.892,4.860)--(8.892,4.875)--(8.893,4.885)%
  --(8.893,4.895)--(8.894,4.906)--(8.894,4.916)--(8.894,4.928)--(8.895,4.933)--(8.895,4.940)%
  --(8.896,4.945)--(8.896,4.955)--(8.896,4.969)--(8.897,4.978)--(8.897,4.987)--(8.897,4.997)%
  --(8.898,5.002)--(8.898,5.011)--(8.899,5.018)--(8.899,5.027)--(8.899,5.038)--(8.900,5.045)%
  --(8.900,5.056)--(8.901,5.064)--(8.901,5.074)--(8.901,5.082)--(8.902,5.091)--(8.902,5.101)%
  --(8.902,5.109)--(8.903,5.118)--(8.903,5.122)--(8.904,5.129)--(8.904,5.135)--(8.904,5.141)%
  --(8.905,5.142)--(8.905,5.149)--(8.906,5.162)--(8.906,5.170)--(8.906,5.179)--(8.907,5.185)%
  --(8.907,5.196)--(8.908,5.208)--(8.908,5.220)--(8.908,5.227)--(8.909,5.241)--(8.909,5.256)%
  --(8.909,5.264)--(8.910,5.271)--(8.910,5.279)--(8.911,5.292)--(8.911,5.308)--(8.911,5.317)%
  --(8.912,5.330)--(8.912,5.338)--(8.913,5.350)--(8.913,5.358)--(8.913,5.364)--(8.914,5.373)%
  --(8.914,5.382)--(8.915,5.398)--(8.915,5.417)--(8.915,5.437)--(8.916,5.450)--(8.916,5.458)%
  --(8.916,5.464)--(8.917,5.468)--(8.917,5.472)--(8.918,5.477)--(8.918,5.488)--(8.918,5.501)%
  --(8.919,5.514)--(8.919,5.529)--(8.920,5.539)--(8.920,5.547)--(8.920,5.558)--(8.921,5.567)%
  --(8.921,5.573)--(8.922,5.586)--(8.922,5.593)--(8.922,5.598)--(8.923,5.603)--(8.923,5.602)%
  --(8.923,5.611)--(8.924,5.626)--(8.924,5.635)--(8.925,5.645)--(8.925,5.660)--(8.925,5.666)%
  --(8.926,5.681)--(8.926,5.702)--(8.927,5.706)--(8.927,5.715)--(8.927,5.728)--(8.928,5.735)%
  --(8.928,5.746)--(8.929,5.765)--(8.929,5.784)--(8.929,5.795)--(8.930,5.805)--(8.930,5.809)%
  --(8.930,5.818)--(8.931,5.829)--(8.931,5.833)--(8.932,5.838)--(8.932,5.844)--(8.932,5.853)%
  --(8.933,5.868)--(8.933,5.878)--(8.934,5.887)--(8.934,5.897)--(8.934,5.905)--(8.935,5.919)%
  --(8.935,5.926)--(8.936,5.936)--(8.936,5.943)--(8.936,5.951)--(8.937,5.962)--(8.937,5.970)%
  --(8.937,5.980)--(8.938,5.992)--(8.938,6.006)--(8.939,6.019)--(8.939,6.035)--(8.939,6.056)%
  --(8.940,6.074)--(8.940,6.091)--(8.941,6.102)--(8.941,6.109)--(8.941,6.111)--(8.942,6.122)%
  --(8.942,6.144)--(8.942,6.169)--(8.943,6.181)--(8.943,6.194)--(8.944,6.207)--(8.944,6.223)%
  --(8.944,6.240)--(8.945,6.248)--(8.946,6.253)--(8.946,6.258)--(8.946,6.265)--(8.947,6.278)%
  --(8.947,6.295)--(8.948,6.306)--(8.948,6.316)--(8.948,6.332)--(8.949,6.348)--(8.949,6.359)%
  --(8.949,6.366)--(8.950,6.374)--(8.950,6.385)--(8.951,6.396)--(8.951,6.400)--(8.951,6.407)%
  --(8.952,6.415)--(8.952,6.255)--(8.953,6.251)--(8.953,6.247)--(8.953,6.236)--(8.954,6.225)%
  --(8.954,6.215)--(8.955,6.209)--(8.955,6.200)--(8.955,6.196)--(8.956,6.189)--(8.956,6.188)%
  --(8.956,6.183)--(8.957,6.173)--(8.957,6.163)--(8.958,6.153)--(8.958,6.149)--(8.958,6.134)%
  --(8.959,6.127)--(8.959,6.121)--(8.960,6.108)--(8.960,6.092)--(8.960,6.088)--(8.961,6.083)%
  --(8.961,6.078)--(8.962,6.071)--(8.962,6.064)--(8.962,6.054)--(8.963,6.046)--(8.963,6.038)%
  --(8.963,6.031)--(8.964,6.023)--(8.964,6.018)--(8.965,6.014)--(8.965,6.007)--(8.965,6.002)%
  --(8.966,5.996)--(8.966,5.978)--(8.967,5.962)--(8.967,5.949)--(8.967,5.941)--(8.968,5.931)%
  --(8.968,5.920)--(8.969,5.904)--(8.969,5.895)--(8.969,5.883)--(8.970,5.869)--(8.970,5.859)%
  --(8.970,5.846)--(8.971,5.834)--(8.971,5.825)--(8.972,5.802)--(8.972,5.781)--(8.972,5.765)%
  --(8.973,5.753)--(8.973,5.745)--(8.974,5.736)--(8.974,5.727)--(8.974,5.715)--(8.975,5.709)%
  --(8.975,5.702)--(8.975,5.688)--(8.976,5.674)--(8.976,5.662)--(8.977,5.654)--(8.977,5.646)%
  --(8.977,5.632)--(8.978,5.623)--(8.978,5.620)--(8.979,5.618)--(8.979,5.613)--(8.979,5.603)%
  --(8.980,5.578)--(8.980,5.561)--(8.981,5.558)--(8.981,5.549)--(8.981,5.539)--(8.982,5.523)%
  --(8.982,5.512)--(8.982,5.497)--(8.983,5.489)--(8.983,5.482)--(8.984,5.476)--(8.984,5.464)%
  --(8.984,5.449)--(8.985,5.434)--(8.985,5.418)--(8.986,5.407)--(8.986,5.398)--(8.986,5.388)%
  --(8.987,5.379)--(8.987,5.368)--(8.988,5.355)--(8.988,5.343)--(8.988,5.329)--(8.989,5.316)%
  --(8.989,5.309)--(8.989,5.300)--(8.990,5.289)--(8.990,5.283)--(8.991,5.280)--(8.991,5.275)%
  --(8.991,5.272)--(8.992,5.265)--(8.992,5.267)--(8.993,5.259)--(8.993,5.249)--(8.993,5.237)%
  --(8.994,5.225)--(8.994,5.218)--(8.995,5.209)--(8.995,5.199)--(8.995,5.192)--(8.996,5.183)%
  --(8.996,5.173)--(8.996,5.161)--(8.997,5.145)--(8.997,5.127)--(8.998,5.118)--(8.998,5.109)%
  --(8.998,5.103)--(8.999,5.094)--(8.999,5.086)--(9.000,5.075)--(9.000,5.061)--(9.000,5.049)%
  --(9.001,5.039)--(9.001,5.018)--(9.002,5.006)--(9.002,4.991)--(9.002,4.980)--(9.003,4.976)%
  --(9.003,4.971)--(9.003,4.963)--(9.004,4.957)--(9.004,4.955)--(9.005,4.952)--(9.005,4.946)%
  --(9.005,4.932)--(9.006,4.919)--(9.006,4.909)--(9.007,4.903)--(9.007,4.896)--(9.007,4.889)%
  --(9.008,4.879)--(9.008,4.869)--(9.008,4.861)--(9.009,4.855)--(9.009,4.844)--(9.010,4.834)%
  --(9.010,4.825)--(9.010,4.814)--(9.011,4.805)--(9.011,4.795)--(9.012,4.785)--(9.012,4.778)%
  --(9.012,4.771)--(9.013,4.763)--(9.013,4.753)--(9.014,4.747)--(9.014,4.743)--(9.014,4.742)%
  --(9.015,4.741)--(9.015,4.737)--(9.015,4.730)--(9.016,4.724)--(9.016,4.721)--(9.017,4.717)%
  --(9.017,4.719)--(9.017,4.715)--(9.018,4.712)--(9.018,4.705)--(9.019,4.695)--(9.019,4.692)%
  --(9.020,4.691)--(9.020,4.690)--(9.021,4.684)--(9.021,4.681)--(9.021,4.673)--(9.022,4.661)%
  --(9.022,4.655)--(9.022,4.649)--(9.023,4.640)--(9.023,4.636)--(9.024,4.627)--(9.024,4.620)%
  --(9.024,4.615)--(9.025,4.606)--(9.025,4.599)--(9.026,4.590)--(9.026,4.582)--(9.026,4.570)%
  --(9.027,4.563)--(9.027,4.560)--(9.028,4.551)--(9.028,4.547)--(9.028,4.544)--(9.029,4.540)%
  --(9.029,4.535)--(9.029,4.530)--(9.030,4.526)--(9.030,4.519)--(9.031,4.513)--(9.031,4.511)%
  --(9.031,4.504)--(9.032,4.497)--(9.032,4.489)--(9.033,4.485)--(9.033,4.482)--(9.033,4.475)%
  --(9.034,4.471)--(9.034,4.468)--(9.035,4.465)--(9.035,4.461)--(9.035,4.462)--(9.036,4.454)%
  --(9.036,4.453)--(9.036,4.449)--(9.037,4.445)--(9.037,4.443)--(9.038,4.444)--(9.038,4.442)%
  --(9.038,4.438)--(9.039,4.434)--(9.039,4.430)--(9.040,4.429)--(9.040,4.425)--(9.040,4.421)%
  --(9.041,4.412)--(9.041,4.406)--(9.042,4.402)--(9.042,4.401)--(9.043,4.395)--(9.043,4.385)%
  --(9.043,4.379)--(9.044,4.373)--(9.044,4.369)--(9.045,4.366)--(9.045,4.360)--(9.045,4.355)%
  --(9.046,4.353)--(9.046,4.348)--(9.047,4.344)--(9.047,4.336)--(9.047,4.329)--(9.048,4.323)%
  --(9.048,4.315)--(9.048,4.313)--(9.049,4.302)--(9.049,4.297)--(9.050,4.291)--(9.050,4.286)%
  --(9.050,4.283)--(9.051,4.277)--(9.051,4.275)--(9.052,4.270)--(9.052,4.263)--(9.052,4.252)%
  --(9.053,4.248)--(9.053,4.243)--(9.054,4.237)--(9.054,4.236)--(9.054,4.233)--(9.055,4.227)%
  --(9.055,4.226)--(9.055,4.219)--(9.056,4.216)--(9.056,4.212)--(9.057,4.209)--(9.057,4.204)%
  --(9.057,4.202)--(9.058,4.197)--(9.058,4.194)--(9.059,4.187)--(9.059,4.180)--(9.059,4.175)%
  --(9.060,4.173)--(9.060,4.172)--(9.061,4.169)--(9.061,4.163)--(9.061,4.153)--(9.062,4.144)%
  --(9.062,4.138)--(9.062,4.130)--(9.063,4.122)--(9.063,4.111)--(9.064,4.106)--(9.064,4.105)%
  --(9.065,4.102)--(9.065,4.097)--(9.066,4.089)--(9.066,4.082)--(9.066,4.075)--(9.067,4.070)%
  --(9.067,4.065)--(9.068,4.056)--(9.068,4.054)--(9.068,4.049)--(9.069,4.045)--(9.069,4.038)%
  --(9.069,4.031)--(9.070,4.024)--(9.070,4.020)--(9.071,4.012)--(9.071,4.008)--(9.072,4.006)%
  --(9.072,4.004)--(9.073,4.001)--(9.073,3.995)--(9.073,3.991)--(9.074,3.986)--(9.074,3.979)%
  --(9.075,3.972)--(9.075,3.966)--(9.075,3.965)--(9.076,3.964)--(9.077,3.962)--(9.078,3.962)%
  --(9.078,3.960)--(9.078,3.962)--(9.079,3.959)--(9.079,3.956)--(9.080,3.955)--(9.080,3.952)%
  --(9.080,3.947)--(9.081,3.947)--(9.081,3.945)--(9.081,3.929)--(9.082,3.916)--(9.082,3.905)%
  --(9.083,3.897)--(9.083,3.894)--(9.083,3.893)--(9.084,3.891)--(9.085,3.889)--(9.085,3.887)%
  --(9.086,3.887)--(9.086,3.884)--(9.087,3.883)--(9.087,3.880)--(9.087,3.874)--(9.088,3.870)%
  --(9.088,3.869)--(9.088,3.866)--(9.089,3.865)--(9.089,3.863)--(9.090,3.861)--(9.090,3.859)%
  --(9.090,3.857)--(9.091,3.856)--(9.091,3.855)--(9.092,3.852)--(9.092,3.851)--(9.092,3.846)%
  --(9.093,3.845)--(9.093,3.843)--(9.094,3.843)--(9.094,3.842)--(9.094,3.836)--(9.095,3.815)%
  --(9.095,3.797)--(9.095,3.790)--(9.096,3.786)--(9.096,3.781)--(9.097,3.776)--(9.097,3.770)%
  --(9.097,3.765)--(9.098,3.757)--(9.098,3.749)--(9.099,3.745)--(9.099,3.740)--(9.100,3.739)%
  --(9.100,3.736)--(9.101,3.736)--(9.101,3.737)--(9.101,3.733)--(9.102,3.730)--(9.102,3.727)%
  --(9.102,3.722)--(9.103,3.716)--(9.103,3.710)--(9.104,3.704)--(9.104,3.702)--(9.104,3.700)%
  --(9.105,3.697)--(9.105,3.692)--(9.106,3.687)--(9.106,3.683)--(9.106,3.678)--(9.107,3.672)%
  --(9.107,3.667)--(9.108,3.661)--(9.108,3.656)--(9.108,3.650)--(9.109,3.642)--(9.109,3.636)%
  --(9.109,3.626)--(9.110,3.620)--(9.110,3.614)--(9.111,3.609)--(9.111,3.604)--(9.111,3.600)%
  --(9.112,3.592)--(9.112,3.591)--(9.113,3.586)--(9.113,3.582)--(9.113,3.580)--(9.114,3.578)%
  --(9.115,3.578)--(9.115,3.576)--(9.116,3.576)--(9.116,3.575)--(9.116,3.573)--(9.117,3.568)%
  --(9.117,3.567)--(9.118,3.564)--(9.118,3.561)--(9.118,3.557)--(9.119,3.554)--(9.119,3.552)%
  --(9.120,3.552)--(9.120,3.549)--(9.120,3.547)--(9.121,3.543)--(9.121,3.540)--(9.122,3.538)%
  --(9.122,3.534)--(9.123,3.530)--(9.123,3.525)--(9.123,3.520)--(9.124,3.515)--(9.124,3.513)%
  --(9.125,3.512)--(9.125,3.507)--(9.125,3.504)--(9.126,3.501)--(9.126,3.497)--(9.127,3.493)%
  --(9.127,3.492)--(9.127,3.490)--(9.128,3.490)--(9.128,3.486)--(9.128,3.484)--(9.129,3.483)%
  --(9.129,3.481)--(9.130,3.479)--(9.130,3.475)--(9.130,3.472)--(9.131,3.470)--(9.131,3.466)%
  --(9.132,3.463)--(9.132,3.461)--(9.132,3.457)--(9.133,3.453)--(9.133,3.451)--(9.134,3.446)%
  --(9.134,3.443)--(9.134,3.441)--(9.135,3.437)--(9.135,3.434)--(9.135,3.430)--(9.136,3.427)%
  --(9.136,3.424)--(9.137,3.421)--(9.137,3.420)--(9.137,3.418)--(9.138,3.415)--(9.139,3.414)%
  --(9.139,3.412)--(9.139,3.406)--(9.140,3.405)--(9.140,3.402)--(9.141,3.399)--(9.141,3.398)%
  --(9.141,3.395)--(9.142,3.394)--(9.142,3.393)--(9.142,3.392)--(9.143,3.391)--(9.143,3.388)%
  --(9.144,3.386)--(9.144,3.383)--(9.144,3.381)--(9.145,3.375)--(9.145,3.372)--(9.146,3.369)%
  --(9.146,3.367)--(9.146,3.365)--(9.147,3.361)--(9.147,3.357)--(9.148,3.352)--(9.148,3.345)%
  --(9.148,3.342)--(9.149,3.339)--(9.149,3.335)--(9.149,3.331)--(9.150,3.326)--(9.150,3.325)%
  --(9.151,3.323)--(9.151,3.322)--(9.151,3.318)--(9.152,3.314)--(9.152,3.308)--(9.153,3.304)%
  --(9.153,3.303)--(9.153,3.299)--(9.154,3.295)--(9.154,3.294)--(9.154,3.292)--(9.155,3.290)%
  --(9.155,3.287)--(9.156,3.283)--(9.156,3.276)--(9.156,3.273)--(9.157,3.273)--(9.157,3.270)%
  --(9.158,3.269)--(9.158,3.268)--(9.158,3.267)--(9.159,3.265)--(9.159,3.263)--(9.160,3.259)%
  --(9.160,3.253)--(9.160,3.252)--(9.161,3.250)--(9.161,3.246)--(9.161,3.245)--(9.162,3.241)%
  --(9.162,3.237)--(9.163,3.234)--(9.163,3.230)--(9.163,3.229)--(9.164,3.227)--(9.164,3.225)%
  --(9.165,3.219)--(9.165,3.217)--(9.165,3.213)--(9.166,3.210)--(9.166,3.208)--(9.167,3.206)%
  --(9.167,3.202)--(9.167,3.200)--(9.168,3.198)--(9.168,3.197)--(9.168,3.192)--(9.169,3.191)%
  --(9.169,3.190)--(9.170,3.189)--(9.170,3.188)--(9.170,3.186)--(9.171,3.184)--(9.171,3.181)%
  --(9.172,3.181)--(9.172,3.179)--(9.172,3.176)--(9.173,3.175)--(9.173,3.173)--(9.174,3.172)%
  --(9.174,3.169)--(9.174,3.167)--(9.175,3.165)--(9.175,3.163)--(9.175,3.164)--(9.176,3.161)%
  --(9.176,3.158)--(9.177,3.155)--(9.177,3.153)--(9.178,3.151)--(9.178,3.147)--(9.179,3.142)%
  --(9.179,3.141)--(9.179,3.134)--(9.180,3.133)--(9.180,3.131)--(9.181,3.126)--(9.181,3.124)%
  --(9.181,3.121)--(9.182,3.120)--(9.182,3.118)--(9.182,3.115)--(9.183,3.112)--(9.183,3.108)%
  --(9.184,3.106)--(9.184,3.105)--(9.184,3.104)--(9.185,3.101)--(9.185,3.099)--(9.186,3.096)%
  --(9.186,3.092)--(9.186,3.091)--(9.187,3.087)--(9.187,3.084)--(9.187,3.083)--(9.188,3.080)%
  --(9.188,3.079)--(9.189,3.076)--(9.189,3.075)--(9.189,3.072)--(9.190,3.071)--(9.190,3.070)%
  --(9.191,3.066)--(9.191,3.064)--(9.191,3.063)--(9.192,3.060)--(9.193,3.058)--(9.193,3.056)%
  --(9.193,3.054)--(9.194,3.054)--(9.194,3.053)--(9.194,3.049)--(9.195,3.047)--(9.195,3.042)%
  --(9.196,3.038)--(9.196,3.035)--(9.196,3.033)--(9.197,3.031)--(9.197,3.029)--(9.198,3.027)%
  --(9.198,3.024)--(9.198,3.019)--(9.199,3.016)--(9.199,3.015)--(9.200,3.013)--(9.200,3.010)%
  --(9.200,3.007)--(9.201,3.005)--(9.201,3.002)--(9.202,3.000)--(9.202,2.996)--(9.203,2.994)%
  --(9.203,2.992)--(9.203,2.989)--(9.204,2.984)--(9.204,2.981)--(9.205,2.977)--(9.205,2.975)%
  --(9.205,2.972)--(9.206,2.969)--(9.206,2.967)--(9.207,2.966)--(9.207,2.963)--(9.207,2.964)%
  --(9.208,2.962)--(9.209,2.959)--(9.209,2.957)--(9.210,2.952)--(9.210,2.949)--(9.210,2.948)%
  --(9.211,2.947)--(9.211,2.946)--(9.212,2.942)--(9.212,2.941)--(9.212,2.940)--(9.213,2.937)%
  --(9.213,2.934)--(9.214,2.931)--(9.214,2.929)--(9.214,2.927)--(9.215,2.925)--(9.215,2.922)%
  --(9.215,2.920)--(9.216,2.917)--(9.216,2.915)--(9.217,2.911)--(9.217,2.909)--(9.217,2.906)%
  --(9.218,2.902)--(9.218,2.899)--(9.219,2.896)--(9.219,2.893)--(9.220,2.891)--(9.220,2.888)%
  --(9.220,2.886)--(9.221,2.885)--(9.221,2.881)--(9.222,2.881)--(9.222,2.880)--(9.222,2.879)%
  --(9.223,2.878)--(9.224,2.875)--(9.224,2.869)--(9.224,2.867)--(9.225,2.865)--(9.225,2.862)%
  --(9.226,2.860)--(9.226,2.858)--(9.226,2.857)--(9.227,2.855)--(9.227,2.854)--(9.227,2.852)%
  --(9.228,2.851)--(9.228,2.849)--(9.229,2.846)--(9.229,2.843)--(9.229,2.840)--(9.230,2.839)%
  --(9.231,2.836)--(9.231,2.833)--(9.232,2.832)--(9.232,2.830)--(9.233,2.829)--(9.233,2.826)%
  --(9.233,2.824)--(9.234,2.822)--(9.234,2.820)--(9.235,2.817)--(9.235,2.813)--(9.236,2.813)%
  --(9.236,2.810)--(9.236,2.809)--(9.237,2.808)--(9.237,2.807)--(9.238,2.804)--(9.238,2.802)%
  --(9.239,2.801)--(9.239,2.800)--(9.240,2.799)--(9.240,2.797)--(9.241,2.796)--(9.241,2.795)%
  --(9.241,2.794)--(9.242,2.794)--(9.242,2.790)--(9.243,2.788)--(9.243,2.789)--(9.243,2.787)%
  --(9.244,2.787)--(9.244,2.784)--(9.245,2.784)--(9.245,2.780)--(9.245,2.779)--(9.246,2.776)%
  --(9.246,2.774)--(9.247,2.773)--(9.247,2.772)--(9.247,2.771)--(9.248,2.769)--(9.248,2.768)%
  --(9.248,2.767)--(9.249,2.766)--(9.250,2.763)--(9.250,2.761)--(9.250,2.759)--(9.251,2.757)%
  --(9.251,2.755)--(9.252,2.752)--(9.252,2.750)--(9.252,2.746)--(9.253,2.744)--(9.253,2.741)%
  --(9.254,2.739)--(9.254,2.736)--(9.254,2.734)--(9.255,2.733)--(9.255,2.732)--(9.256,2.730)%
  --(9.256,2.729)--(9.257,2.730)--(9.257,2.728)--(9.257,2.726)--(9.258,2.725)--(9.259,2.722)%
  --(9.259,2.720)--(9.260,2.718)--(9.260,2.715)--(9.261,2.715)--(9.261,2.714)--(9.262,2.712)%
  --(9.262,2.711)--(9.263,2.711)--(9.263,2.710)--(9.264,2.709)--(9.264,2.710)--(9.264,2.708)%
  --(9.265,2.705)--(9.265,2.702)--(9.266,2.702)--(9.266,2.699)--(9.267,2.698)--(9.267,2.696)%
  --(9.267,2.695)--(9.268,2.695)--(9.268,2.694)--(9.269,2.692)--(9.269,2.689)--(9.270,2.688)%
  --(9.270,2.686)--(9.271,2.685)--(9.271,2.683)--(9.271,2.682)--(9.272,2.681)--(9.272,2.679)%
  --(9.273,2.678)--(9.273,2.676)--(9.273,2.673)--(9.274,2.671)--(9.274,2.670)--(9.274,2.667)%
  --(9.275,2.666)--(9.275,2.664)--(9.276,2.660)--(9.276,2.659)--(9.276,2.657)--(9.277,2.656)%
  --(9.277,2.653)--(9.278,2.652)--(9.278,2.650)--(9.278,2.648)--(9.279,2.648)--(9.279,2.647)%
  --(9.280,2.646)--(9.280,2.643)--(9.280,2.641)--(9.281,2.640)--(9.281,2.638)--(9.281,2.635)%
  --(9.282,2.634)--(9.282,2.632)--(9.283,2.632)--(9.283,2.631)--(9.283,2.630)--(9.284,2.629)%
  --(9.284,2.626)--(9.285,2.625)--(9.285,2.624)--(9.285,2.623)--(9.286,2.622)--(9.286,2.621)%
  --(9.287,2.618)--(9.287,2.616)--(9.287,2.612)--(9.288,2.610)--(9.288,2.608)--(9.288,2.606)%
  --(9.289,2.605)--(9.289,2.603)--(9.290,2.602)--(9.290,2.599)--(9.290,2.597)--(9.291,2.596)%
  --(9.291,2.595)--(9.292,2.591)--(9.292,2.590)--(9.292,2.587)--(9.293,2.586)--(9.293,2.585)%
  --(9.294,2.584)--(9.294,2.582)--(9.295,2.582)--(9.295,2.581)--(9.295,2.578)--(9.296,2.578)%
  --(9.297,2.577)--(9.297,2.576)--(9.297,2.574)--(9.298,2.570)--(9.298,2.569)--(9.299,2.567)%
  --(9.299,2.566)--(9.299,2.564)--(9.300,2.564)--(9.300,2.563)--(9.300,2.562)--(9.301,2.562)%
  --(9.302,2.561)--(9.302,2.560)--(9.302,2.559)--(9.303,2.556)--(9.303,2.555)--(9.304,2.555)%
  --(9.304,2.554)--(9.304,2.553)--(9.305,2.552)--(9.305,2.550)--(9.306,2.550)--(9.306,2.547)%
  --(9.306,2.546)--(9.307,2.543)--(9.307,2.542)--(9.308,2.541)--(9.308,2.540)--(9.309,2.537)%
  --(9.309,2.535)--(9.309,2.534)--(9.310,2.534)--(9.310,2.530)--(9.311,2.523)--(9.311,2.522)%
  --(9.311,2.521)--(9.312,2.520)--(9.313,2.517)--(9.313,2.515)--(9.314,2.513)--(9.314,2.511)%
  --(9.315,2.508)--(9.315,2.505)--(9.316,2.504)--(9.316,2.497)--(9.316,2.493)--(9.317,2.491)%
  --(9.317,2.487)--(9.318,2.486)--(9.318,2.485)--(9.318,2.483)--(9.319,2.482)--(9.319,2.480)%
  --(9.320,2.482)--(9.320,2.479)--(9.320,2.478)--(9.321,2.477)--(9.321,2.474)--(9.321,2.473)%
  --(9.322,2.471)--(9.323,2.469)--(9.323,2.466)--(9.324,2.465)--(9.324,2.462)--(9.325,2.460)%
  --(9.325,2.458)--(9.325,2.457)--(9.326,2.457)--(9.326,2.456)--(9.327,2.455)--(9.327,2.453)%
  --(9.328,2.452)--(9.328,2.450)--(9.329,2.449)--(9.330,2.447)--(9.331,2.444)--(9.331,2.443)%
  --(9.332,2.441)--(9.332,2.439)--(9.332,2.438)--(9.333,2.437)--(9.333,2.436)--(9.333,2.435)%
  --(9.334,2.435)--(9.334,2.433)--(9.335,2.431)--(9.335,2.430)--(9.336,2.429)--(9.336,2.428)%
  --(9.337,2.427)--(9.337,2.426)--(9.337,2.425)--(9.338,2.425)--(9.338,2.424)--(9.339,2.422)%
  --(9.339,2.421)--(9.339,2.420)--(9.340,2.420)--(9.340,2.419)--(9.340,2.418)--(9.341,2.416)%
  --(9.341,2.417)--(9.342,2.416)--(9.342,2.414)--(9.343,2.410)--(9.344,2.408)--(9.344,2.406)%
  --(9.345,2.405)--(9.345,2.404)--(9.346,2.403)--(9.346,2.402)--(9.346,2.401)--(9.347,2.399)%
  --(9.347,2.398)--(9.348,2.397)--(9.348,2.396)--(9.349,2.395)--(9.349,2.394)--(9.349,2.392)%
  --(9.350,2.390)--(9.350,2.388)--(9.351,2.387)--(9.351,2.386)--(9.351,2.384)--(9.352,2.383)%
  --(9.352,2.381)--(9.353,2.381)--(9.353,2.380)--(9.353,2.378)--(9.354,2.378)--(9.354,2.377)%
  --(9.354,2.374)--(9.355,2.374)--(9.355,2.373)--(9.356,2.372)--(9.356,2.371)--(9.357,2.371)%
  --(9.357,2.369)--(9.358,2.366)--(9.358,2.364)--(9.358,2.363)--(9.359,2.361)--(9.359,2.360)%
  --(9.360,2.360)--(9.360,2.359)--(9.360,2.358)--(9.361,2.358)--(9.361,2.357)--(9.361,2.356)%
  --(9.362,2.355)--(9.362,2.353)--(9.363,2.353)--(9.363,2.351)--(9.363,2.352)--(9.364,2.350)%
  --(9.364,2.349)--(9.365,2.350)--(9.365,2.349)--(9.365,2.348)--(9.366,2.346)--(9.366,2.344)%
  --(9.366,2.343)--(9.367,2.342)--(9.367,2.343)--(9.368,2.343)--(9.368,2.340)--(9.368,2.337)%
  --(9.369,2.336)--(9.369,2.334)--(9.370,2.333)--(9.370,2.331)--(9.370,2.332)--(9.371,2.332)%
  --(9.371,2.331)--(9.372,2.329)--(9.372,2.327)--(9.372,2.325)--(9.373,2.324)--(9.373,2.322)%
  --(9.374,2.321)--(9.374,2.315)--(9.375,2.313)--(9.376,2.311)--(9.376,2.308)--(9.377,2.305)%
  --(9.377,2.303)--(9.377,2.301)--(9.378,2.301)--(9.378,2.298)--(9.379,2.298)--(9.379,2.296)%
  --(9.379,2.295)--(9.380,2.294)--(9.380,2.292)--(9.381,2.292)--(9.381,2.291)--(9.382,2.290)%
  --(9.382,2.291)--(9.382,2.288)--(9.383,2.287)--(9.384,2.287)--(9.384,2.285)--(9.384,2.284)%
  --(9.385,2.282)--(9.385,2.280)--(9.386,2.279)--(9.386,2.276)--(9.387,2.275)--(9.387,2.274)%
  --(9.387,2.272)--(9.388,2.271)--(9.388,2.270)--(9.389,2.268)--(9.389,2.267)--(9.389,2.265)%
  --(9.390,2.263)--(9.390,2.262)--(9.391,2.260)--(9.391,2.259)--(9.391,2.257)--(9.392,2.256)%
  --(9.392,2.255)--(9.393,2.254)--(9.393,2.252)--(9.393,2.251)--(9.394,2.250)--(9.394,2.248)%
  --(9.395,2.246)--(9.395,2.244)--(9.396,2.243)--(9.396,2.242)--(9.396,2.240)--(9.397,2.240)%
  --(9.397,2.239)--(9.398,2.238)--(9.398,2.237)--(9.398,2.234)--(9.399,2.234)--(9.399,2.233)%
  --(9.399,2.231)--(9.400,2.230)--(9.400,2.229)--(9.401,2.229)--(9.401,2.227)--(9.401,2.226)%
  --(9.402,2.225)--(9.402,2.224)--(9.403,2.224)--(9.403,2.223)--(9.403,2.221)--(9.404,2.221)%
  --(9.404,2.222)--(9.405,2.219)--(9.405,2.218)--(9.405,2.217)--(9.406,2.217)--(9.406,2.216)%
  --(9.406,2.214)--(9.407,2.214)--(9.407,2.211)--(9.408,2.212)--(9.408,2.211)--(9.408,2.210)%
  --(9.409,2.210)--(9.409,2.209)--(9.410,2.208)--(9.410,2.207)--(9.411,2.206)--(9.411,2.205)%
  --(9.412,2.204)--(9.412,2.203)--(9.413,2.203)--(9.413,2.201)--(9.414,2.200)--(9.414,2.199)%
  --(9.415,2.198)--(9.415,2.197)--(9.416,2.197)--(9.416,2.196)--(9.417,2.195)--(9.417,2.193)%
  --(9.417,2.191)--(9.418,2.190)--(9.418,2.189)--(9.419,2.188)--(9.419,2.186)--(9.419,2.185)%
  --(9.420,2.183)--(9.420,2.184)--(9.420,2.183)--(9.421,2.181)--(9.421,2.180)--(9.422,2.178)%
  --(9.422,2.177)--(9.423,2.177)--(9.423,2.175)--(9.424,2.175)--(9.424,2.173)--(9.425,2.171)%
  --(9.425,2.170)--(9.426,2.170)--(9.426,2.168)--(9.427,2.167)--(9.427,2.165)--(9.428,2.164)%
  --(9.429,2.163)--(9.429,2.162)--(9.430,2.161)--(9.430,2.160)--(9.431,2.160)--(9.431,2.158)%
  --(9.432,2.158)--(9.432,2.157)--(9.432,2.156)--(9.433,2.155)--(9.434,2.155)--(9.434,2.154)%
  --(9.435,2.152)--(9.436,2.148)--(9.436,2.147)--(9.437,2.145)--(9.437,2.143)--(9.438,2.141)%
  --(9.438,2.140)--(9.438,2.139)--(9.439,2.139)--(9.439,2.138)--(9.439,2.136)--(9.440,2.134)%
  --(9.440,2.132)--(9.441,2.132)--(9.441,2.130)--(9.442,2.129)--(9.443,2.127)--(9.443,2.126)%
  --(9.444,2.126)--(9.444,2.124)--(9.445,2.123)--(9.446,2.122)--(9.446,2.119)--(9.446,2.120)%
  --(9.447,2.119)--(9.448,2.117)--(9.448,2.116)--(9.449,2.115)--(9.450,2.112)--(9.450,2.113)%
  --(9.450,2.110)--(9.451,2.109)--(9.451,2.108)--(9.452,2.107)--(9.452,2.106)--(9.453,2.106)%
  --(9.454,2.105)--(9.454,2.103)--(9.455,2.102)--(9.455,2.100)--(9.456,2.100)--(9.457,2.098)%
  --(9.457,2.097)--(9.458,2.096)--(9.459,2.095)--(9.459,2.093)--(9.460,2.093)--(9.460,2.091)%
  --(9.460,2.090)--(9.461,2.089)--(9.461,2.087)--(9.462,2.086)--(9.462,2.084)--(9.462,2.083)%
  --(9.463,2.083)--(9.463,2.080)--(9.464,2.081)--(9.464,2.079)--(9.464,2.077)--(9.465,2.076)%
  --(9.465,2.074)--(9.466,2.073)--(9.466,2.072)--(9.467,2.071)--(9.467,2.070)--(9.468,2.070)%
  --(9.468,2.067)--(9.469,2.068)--(9.469,2.067)--(9.469,2.066)--(9.470,2.065)--(9.470,2.064)%
  --(9.471,2.063)--(9.471,2.062)--(9.471,2.059)--(9.472,2.059)--(9.472,2.057)--(9.473,2.056)%
  --(9.473,2.055)--(9.474,2.053)--(9.474,2.052)--(9.475,2.051)--(9.475,2.052)--(9.476,2.051)%
  --(9.476,2.049)--(9.477,2.047)--(9.478,2.046)--(9.478,2.044)--(9.478,2.045)--(9.479,2.042)%
  --(9.479,2.041)--(9.479,2.040)--(9.480,2.040)--(9.480,2.039)--(9.481,2.039)--(9.481,2.038)%
  --(9.481,2.037)--(9.482,2.036)--(9.482,2.034)--(9.483,2.032)--(9.483,2.029)--(9.483,2.028)%
  --(9.484,2.027)--(9.484,2.025)--(9.485,2.025)--(9.485,2.024)--(9.486,2.023)--(9.486,2.022)%
  --(9.487,2.022)--(9.487,2.020)--(9.488,2.020)--(9.488,2.019)--(9.488,2.017)--(9.489,2.016)%
  --(9.489,2.015)--(9.490,2.014)--(9.490,2.012)--(9.490,2.010)--(9.491,2.010)--(9.492,2.009)%
  --(9.492,2.008)--(9.492,2.006)--(9.493,2.006)--(9.493,2.005)--(9.493,2.004)--(9.494,2.004)%
  --(9.494,2.002)--(9.495,2.002)--(9.495,2.000)--(9.496,2.000)--(9.496,1.999)--(9.497,1.998)%
  --(9.497,1.997)--(9.498,1.997)--(9.498,1.996)--(9.499,1.994)--(9.499,1.993)--(9.500,1.992)%
  --(9.500,1.991)--(9.501,1.990)--(9.501,1.989)--(9.502,1.987)--(9.502,1.986)--(9.502,1.984)%
  --(9.503,1.983)--(9.504,1.982)--(9.504,1.981)--(9.505,1.980)--(9.505,1.979)--(9.506,1.978)%
  --(9.506,1.977)--(9.507,1.976)--(9.507,1.975)--(9.507,1.974)--(9.508,1.974)--(9.509,1.974)%
  --(9.509,1.973)--(9.509,1.971)--(9.510,1.969)--(9.510,1.968)--(9.511,1.968)--(9.511,1.966)%
  --(9.512,1.966)--(9.512,1.965)--(9.513,1.964)--(9.513,1.961)--(9.514,1.960)--(9.514,1.959)%
  --(9.515,1.958)--(9.515,1.957)--(9.516,1.957)--(9.516,1.956)--(9.516,1.955)--(9.517,1.954)%
  --(9.517,1.953)--(9.518,1.953)--(9.518,1.951)--(9.518,1.949)--(9.519,1.949)--(9.519,1.948)%
  --(9.519,1.947)--(9.520,1.946)--(9.520,1.945)--(9.521,1.944)--(9.521,1.943)--(9.521,1.941)%
  --(9.522,1.940)--(9.522,1.938)--(9.523,1.936)--(9.523,1.935)--(9.523,1.934)--(9.524,1.934)%
  --(9.524,1.933)--(9.525,1.933)--(9.525,1.932)--(9.526,1.931)--(9.526,1.930)--(9.526,1.931)%
  --(9.527,1.931)--(9.527,1.929)--(9.528,1.929)--(9.528,1.927)--(9.528,1.925)--(9.529,1.923)%
  --(9.529,1.922)--(9.530,1.922)--(9.530,1.920)--(9.530,1.919)--(9.531,1.919)--(9.531,1.918)%
  --(9.532,1.918)--(9.532,1.916)--(9.532,1.915)--(9.533,1.914)--(9.533,1.913)--(9.534,1.912)%
  --(9.534,1.911)--(9.535,1.910)--(9.535,1.908)--(9.535,1.907)--(9.536,1.906)--(9.536,1.905)%
  --(9.537,1.904)--(9.537,1.903)--(9.537,1.901)--(9.538,1.900)--(9.538,1.898)--(9.539,1.898)%
  --(9.539,1.897)--(9.540,1.895)--(9.540,1.894)--(9.540,1.893)--(9.541,1.892)--(9.541,1.890)%
  --(9.542,1.890)--(9.542,1.889)--(9.543,1.887)--(9.544,1.886)--(9.545,1.884)--(9.545,1.885)%
  --(9.546,1.883)--(9.546,1.882)--(9.547,1.880)--(9.547,1.878)--(9.548,1.878)--(9.549,1.877)%
  --(9.549,1.876)--(9.549,1.875)--(9.550,1.876)--(9.550,1.874)--(9.551,1.873)--(9.551,1.872)%
  --(9.552,1.872)--(9.552,1.870)--(9.552,1.871)--(9.553,1.870)--(9.553,1.869)--(9.554,1.869)%
  --(9.554,1.868)--(9.555,1.868)--(9.556,1.867)--(9.556,1.866)--(9.557,1.865)--(9.557,1.864)%
  --(9.558,1.863)--(9.559,1.862)--(9.559,1.860)--(9.560,1.860)--(9.560,1.859)--(9.561,1.858)%
  --(9.561,1.857)--(9.562,1.856)--(9.562,1.855)--(9.563,1.855)--(9.563,1.854)--(9.564,1.853)%
  --(9.565,1.852)--(9.565,1.850)--(9.566,1.850)--(9.566,1.849)--(9.567,1.848)--(9.568,1.848)%
  --(9.568,1.847)--(9.568,1.846)--(9.569,1.845)--(9.569,1.844)--(9.570,1.844)--(9.570,1.843)%
  --(9.571,1.843)--(9.571,1.842)--(9.572,1.840)--(9.572,1.839)--(9.573,1.837)--(9.573,1.836)%
  --(9.574,1.834)--(9.574,1.833)--(9.575,1.834)--(9.575,1.833)--(9.576,1.832)--(9.576,1.831)%
  --(9.577,1.831)--(9.577,1.830)--(9.578,1.829)--(9.578,1.828)--(9.579,1.827)--(9.579,1.826)%
  --(9.580,1.825)--(9.580,1.823)--(9.581,1.823)--(9.581,1.824)--(9.582,1.824)--(9.583,1.823)%
  --(9.584,1.823)--(9.584,1.822)--(9.585,1.822)--(9.585,1.820)--(9.585,1.821)--(9.586,1.821)%
  --(9.586,1.820)--(9.587,1.821)--(9.587,1.820)--(9.588,1.820)--(9.588,1.821)--(9.589,1.820)%
  --(9.589,1.819)--(9.590,1.819)--(9.590,1.818)--(9.591,1.819)--(9.591,1.818)--(9.592,1.818)%
  --(9.592,1.819)--(9.592,1.818)--(9.593,1.817)--(9.593,1.818)--(9.594,1.817)--(9.595,1.817)%
  --(9.595,1.816)--(9.596,1.816)--(9.596,1.817)--(9.597,1.817)--(9.598,1.817)--(9.598,1.816)%
  --(9.599,1.815)--(9.599,1.816)--(9.600,1.816)--(9.601,1.817)--(9.601,1.815)--(9.602,1.814)%
  --(9.602,1.815)--(9.603,1.815)--(9.603,1.816)--(9.604,1.816)--(9.604,1.815)--(9.605,1.813)%
  --(9.605,1.812)--(9.606,1.812)--(9.606,1.813)--(9.607,1.812)--(9.608,1.813)--(9.608,1.812)%
  --(9.609,1.813)--(9.609,1.812)--(9.610,1.812)--(9.611,1.812)--(9.611,1.811)--(9.612,1.811)%
  --(9.612,1.810)--(9.613,1.810)--(9.613,1.809)--(9.614,1.810)--(9.614,1.809)--(9.615,1.810)%
  --(9.616,1.809)--(9.617,1.809)--(9.617,1.808)--(9.618,1.809)--(9.618,1.810)--(9.619,1.810)%
  --(9.620,1.810)--(9.620,1.809)--(9.620,1.810)--(9.621,1.809)--(9.622,1.809)--(9.622,1.808)%
  --(9.622,1.809)--(9.623,1.808)--(9.624,1.808)--(9.625,1.808)--(9.625,1.809)--(9.626,1.808)%
  --(9.626,1.810)--(9.627,1.810)--(9.628,1.811)--(9.628,1.810)--(9.629,1.811)--(9.629,1.812)%
  --(9.630,1.812)--(9.630,1.813)--(9.631,1.813)--(9.632,1.813)--(9.633,1.813)--(9.633,1.812)%
  --(9.634,1.812)--(9.634,1.810)--(9.634,1.811)--(9.635,1.810)--(9.636,1.810)--(9.636,1.809)%
  --(9.636,1.810)--(9.637,1.810)--(9.638,1.810)--(9.639,1.810)--(9.639,1.811)--(9.640,1.812)%
  --(9.640,1.811)--(9.641,1.811)--(9.642,1.811)--(9.642,1.812)--(9.643,1.811)--(9.643,1.810)%
  --(9.644,1.811)--(9.644,1.810)--(9.644,1.809)--(9.645,1.809)--(9.645,1.808)--(9.646,1.808)%
  --(9.647,1.808)--(9.648,1.807)--(9.649,1.807)--(9.649,1.806)--(9.650,1.806)--(9.651,1.805)%
  --(9.651,1.806)--(9.652,1.806)--(9.653,1.806)--(9.653,1.805)--(9.653,1.806)--(9.654,1.805)%
  --(9.654,1.804)--(9.655,1.805)--(9.655,1.803)--(9.656,1.803)--(9.657,1.802)--(9.657,1.803)%
  --(9.658,1.803)--(9.659,1.802)--(9.660,1.801)--(9.661,1.800)--(9.662,1.800)--(9.662,1.801)%
  --(9.662,1.800)--(9.663,1.800)--(9.663,1.799)--(9.664,1.800)--(9.664,1.799)--(9.665,1.798)%
  --(9.665,1.799)--(9.665,1.798)--(9.666,1.798)--(9.666,1.799)--(9.667,1.799)--(9.667,1.798)%
  --(9.668,1.798)--(9.669,1.798)--(9.670,1.798)--(9.670,1.799)--(9.671,1.799)--(9.671,1.798)%
  --(9.672,1.798)--(9.672,1.799)--(9.673,1.799)--(9.673,1.800)--(9.674,1.800)--(9.675,1.800)%
  --(9.676,1.799)--(9.676,1.800)--(9.677,1.800)--(9.678,1.800)--(9.678,1.801)--(9.679,1.801)%
  --(9.680,1.801)--(9.681,1.802)--(9.681,1.801)--(9.682,1.801)--(9.682,1.800)--(9.683,1.800)%
  --(9.683,1.799)--(9.684,1.798)--(9.684,1.799)--(9.685,1.800)--(9.685,1.799)--(9.686,1.799)%
  --(9.687,1.799)--(9.688,1.799)--(9.688,1.800)--(9.688,1.799)--(9.689,1.799)--(9.689,1.800)%
  --(9.690,1.799)--(9.690,1.798)--(9.691,1.799)--(9.691,1.798)--(9.691,1.797)--(9.692,1.797)%
  --(9.692,1.798)--(9.693,1.797)--(9.693,1.796)--(9.694,1.796)--(9.694,1.797)--(9.695,1.796)%
  --(9.695,1.797)--(9.696,1.797)--(9.697,1.796)--(9.697,1.797)--(9.697,1.796)--(9.698,1.796)%
  --(9.699,1.795)--(9.699,1.796)--(9.700,1.796)--(9.700,1.795)--(9.701,1.794)--(9.702,1.794)%
  --(9.702,1.793)--(9.703,1.793)--(9.703,1.794)--(9.704,1.794)--(9.704,1.793)--(9.705,1.793)%
  --(9.705,1.792)--(9.706,1.792)--(9.707,1.792)--(9.707,1.791)--(9.708,1.791)--(9.709,1.791)%
  --(9.709,1.790)--(9.710,1.790)--(9.710,1.791)--(9.711,1.791)--(9.711,1.790)--(9.712,1.791)%
  --(9.713,1.791)--(9.714,1.792)--(9.714,1.791)--(9.715,1.791)--(9.716,1.791)--(9.717,1.792)%
  --(9.717,1.791)--(9.718,1.792)--(9.718,1.791)--(9.719,1.791)--(9.720,1.790)--(9.721,1.789)%
  --(9.721,1.790)--(9.722,1.790)--(9.723,1.789)--(9.723,1.788)--(9.724,1.787)--(9.724,1.788)%
  --(9.724,1.787)--(9.725,1.786)--(9.726,1.787)--(9.726,1.786)--(9.727,1.786)--(9.727,1.787)%
  --(9.728,1.787)--(9.728,1.788)--(9.729,1.786)--(9.730,1.786)--(9.731,1.787)--(9.731,1.786)%
  --(9.731,1.787)--(9.732,1.787)--(9.732,1.788)--(9.733,1.786)--(9.734,1.786)--(9.734,1.785)%
  --(9.735,1.785)--(9.735,1.784)--(9.735,1.785)--(9.736,1.785)--(9.736,1.783)--(9.737,1.783)%
  --(9.737,1.782)--(9.738,1.782)--(9.739,1.781)--(9.740,1.781)--(9.740,1.780)--(9.741,1.781)%
  --(9.741,1.780)--(9.742,1.780)--(9.742,1.779)--(9.743,1.779)--(9.743,1.778)--(9.744,1.778)%
  --(9.744,1.777)--(9.744,1.778)--(9.745,1.777)--(9.746,1.778)--(9.747,1.779)--(9.748,1.780)%
  --(9.749,1.779)--(9.749,1.780)--(9.750,1.780)--(9.751,1.781)--(9.752,1.781)--(9.752,1.780)%
  --(9.753,1.780)--(9.754,1.780)--(9.754,1.781)--(9.755,1.781)--(9.755,1.780)--(9.756,1.780)%
  --(9.756,1.779)--(9.756,1.778)--(9.757,1.778)--(9.757,1.777)--(9.758,1.777)--(9.758,1.776)%
  --(9.759,1.775)--(9.759,1.776)--(9.759,1.777)--(9.760,1.775)--(9.760,1.776)--(9.761,1.775)%
  --(9.761,1.774)--(9.762,1.774)--(9.762,1.775)--(9.763,1.774)--(9.764,1.773)--(9.764,1.774)%
  --(9.765,1.774)--(9.766,1.773)--(9.767,1.772)--(9.767,1.771)--(9.768,1.771)--(9.768,1.770)%
  --(9.769,1.770)--(9.769,1.771)--(9.770,1.770)--(9.770,1.771)--(9.771,1.770)--(9.772,1.770)%
  --(9.773,1.770)--(9.773,1.771)--(9.773,1.770)--(9.774,1.769)--(9.775,1.769)--(9.775,1.770)%
  --(9.776,1.770)--(9.776,1.769)--(9.777,1.770)--(9.777,1.771)--(9.778,1.771)--(9.779,1.771)%
  --(9.780,1.771)--(9.780,1.772)--(9.781,1.773)--(9.781,1.772)--(9.782,1.772)--(9.783,1.773)%
  --(9.783,1.772)--(9.784,1.772)--(9.784,1.773)--(9.785,1.772)--(9.786,1.772)--(9.787,1.772)%
  --(9.787,1.773)--(9.787,1.774)--(9.788,1.773)--(9.789,1.772)--(9.789,1.773)--(9.790,1.773)%
  --(9.790,1.772)--(9.791,1.772)--(9.792,1.772)--(9.792,1.773)--(9.793,1.773)--(9.793,1.774)%
  --(9.794,1.773)--(9.794,1.774)--(9.795,1.774)--(9.795,1.773)--(9.796,1.774)--(9.796,1.773)%
  --(9.797,1.773)--(9.797,1.772)--(9.797,1.771)--(9.798,1.771)--(9.799,1.772)--(9.799,1.771)%
  --(9.799,1.772)--(9.800,1.771)--(9.800,1.770)--(9.801,1.771)--(9.802,1.771)--(9.802,1.773)%
  --(9.803,1.771)--(9.803,1.770)--(9.804,1.770)--(9.805,1.771)--(9.806,1.770)--(9.806,1.771)%
  --(9.806,1.769)--(9.807,1.770)--(9.808,1.770)--(9.809,1.770)--(9.809,1.771)--(9.810,1.770)%
  --(9.811,1.770)--(9.811,1.771)--(9.812,1.771)--(9.812,1.770)--(9.813,1.770)--(9.814,1.769)%
  --(9.815,1.770)--(9.816,1.769)--(9.817,1.768)--(9.817,1.769)--(9.818,1.768)--(9.818,1.769)%
  --(9.818,1.768)--(9.819,1.769)--(9.820,1.769)--(9.820,1.770)--(9.821,1.770)--(9.821,1.769)%
  --(9.822,1.769)--(9.823,1.769)--(9.823,1.768)--(9.824,1.768)--(9.824,1.767)--(9.825,1.766)%
  --(9.825,1.767)--(9.825,1.766)--(9.826,1.767)--(9.826,1.766)--(9.827,1.765)--(9.828,1.764)%
  --(9.829,1.764)--(9.830,1.764)--(9.830,1.763)--(9.831,1.762)--(9.832,1.761)--(9.833,1.762)%
  --(9.834,1.762)--(9.834,1.761)--(9.834,1.762)--(9.835,1.762)--(9.836,1.761)--(9.836,1.762)%
  --(9.837,1.762)--(9.837,1.761)--(9.838,1.761)--(9.839,1.761)--(9.839,1.762)--(9.840,1.761)%
  --(9.840,1.762)--(9.841,1.762)--(9.842,1.761)--(9.843,1.760)--(9.844,1.761)--(9.844,1.760)%
  --(9.845,1.760)--(9.845,1.761)--(9.846,1.761)--(9.846,1.760)--(9.847,1.762)--(9.847,1.761)%
  --(9.848,1.762)--(9.849,1.761)--(9.850,1.761)--(9.850,1.760)--(9.851,1.761)--(9.851,1.760)%
  --(9.852,1.760)--(9.853,1.760)--(9.853,1.759)--(9.853,1.760)--(9.854,1.759)--(9.854,1.758)%
  --(9.855,1.759)--(9.855,1.758)--(9.856,1.759)--(9.856,1.758)--(9.857,1.758)--(9.857,1.757)%
  --(9.858,1.758)--(9.858,1.757)--(9.858,1.758)--(9.859,1.758)--(9.859,1.757)--(9.860,1.757)%
  --(9.860,1.756)--(9.860,1.755)--(9.861,1.756)--(9.861,1.755)--(9.862,1.755)--(9.862,1.756)%
  --(9.862,1.755)--(9.863,1.754)--(9.863,1.753)--(9.864,1.753)--(9.864,1.752)--(9.865,1.752)%
  --(9.865,1.751)--(9.865,1.752)--(9.866,1.752)--(9.867,1.752)--(9.868,1.752)--(9.869,1.751)%
  --(9.869,1.750)--(9.870,1.750)--(9.870,1.749)--(9.871,1.749)--(9.871,1.748)--(9.872,1.748)%
  --(9.873,1.749)--(9.874,1.748)--(9.875,1.748)--(9.875,1.747)--(9.876,1.747)--(9.876,1.748)%
  --(9.877,1.747)--(9.878,1.746)--(9.878,1.745)--(9.879,1.744)--(9.879,1.743)--(9.880,1.744)%
  --(9.880,1.745)--(9.881,1.743)--(9.881,1.742)--(9.882,1.742)--(9.883,1.741)--(9.883,1.739)%
  --(9.884,1.740)--(9.884,1.739)--(9.885,1.739)--(9.886,1.738)--(9.886,1.740)--(9.887,1.739)%
  --(9.888,1.740)--(9.888,1.739)--(9.888,1.740)--(9.889,1.739)--(9.890,1.739)--(9.891,1.740)%
  --(9.891,1.739)--(9.892,1.739)--(9.892,1.738)--(9.893,1.737)--(9.893,1.736)--(9.894,1.736)%
  --(9.895,1.735)--(9.895,1.736)--(9.895,1.734)--(9.896,1.734)--(9.896,1.733)--(9.897,1.732)%
  --(9.897,1.733)--(9.898,1.733)--(9.898,1.732)--(9.899,1.733)--(9.899,1.731)--(9.900,1.731)%
  --(9.900,1.730)--(9.901,1.730)--(9.902,1.729)--(9.902,1.730)--(9.903,1.729)--(9.903,1.728)%
  --(9.904,1.727)--(9.904,1.726)--(9.905,1.725)--(9.905,1.726)--(9.906,1.725)--(9.907,1.724)%
  --(9.907,1.723)--(9.907,1.722)--(9.908,1.721)--(9.909,1.722)--(9.910,1.721)--(9.910,1.722)%
  --(9.910,1.720)--(9.911,1.720)--(9.911,1.719)--(9.912,1.719)--(9.912,1.718)--(9.913,1.718)%
  --(9.913,1.717)--(9.914,1.717)--(9.914,1.715)--(9.915,1.714)--(9.916,1.713)--(9.916,1.714)%
  --(9.917,1.713)--(9.917,1.712)--(9.917,1.711)--(9.918,1.709)--(9.918,1.708)--(9.919,1.708)%
  --(9.919,1.707)--(9.920,1.707)--(9.920,1.706)--(9.921,1.706)--(9.921,1.705)--(9.921,1.704)%
  --(9.922,1.704)--(9.922,1.703)--(9.923,1.703)--(9.923,1.702)--(9.924,1.702)--(9.925,1.700)%
  --(9.925,1.699)--(9.926,1.700)--(9.926,1.699)--(9.927,1.698)--(9.927,1.697)--(9.928,1.698)%
  --(9.928,1.697)--(9.929,1.695)--(9.929,1.696)--(9.929,1.695)--(9.930,1.695)--(9.930,1.693)%
  --(9.931,1.692)--(9.931,1.691)--(9.932,1.691)--(9.933,1.691)--(9.933,1.690)--(9.934,1.690)%
  --(9.934,1.689)--(9.935,1.690)--(9.935,1.688)--(9.935,1.689)--(9.936,1.688)--(9.937,1.687)%
  --(9.938,1.686)--(9.938,1.685)--(9.939,1.684)--(9.939,1.683)--(9.940,1.686)--(9.940,1.685)%
  --(9.940,1.686)--(9.941,1.685)--(9.941,1.683)--(9.942,1.683)--(9.942,1.684)--(9.942,1.683)%
  --(9.943,1.681)--(9.944,1.680)--(9.944,1.681)--(9.945,1.680)--(9.945,1.679)--(9.945,1.680)%
  --(9.946,1.681)--(9.946,1.680)--(9.947,1.681)--(9.947,1.680)--(9.948,1.681)--(9.948,1.680)%
  --(9.949,1.680)--(9.949,1.678)--(9.949,1.679)--(9.950,1.679)--(9.950,1.678)--(9.950,1.677)%
  --(9.951,1.677)--(9.951,1.676)--(9.952,1.675)--(9.952,1.674)--(9.953,1.672)--(9.954,1.671)%
  --(9.954,1.672)--(9.955,1.671)--(9.956,1.670)--(9.957,1.670)--(9.957,1.668)--(9.957,1.669)%
  --(9.958,1.669)--(9.958,1.668)--(9.959,1.669)--(9.959,1.667)--(9.960,1.668)--(9.961,1.668)%
  --(9.961,1.667)--(9.962,1.668)--(9.962,1.667)--(9.963,1.666)--(9.963,1.665)--(9.964,1.665)%
  --(9.964,1.666)--(9.965,1.666)--(9.965,1.664)--(9.966,1.664)--(9.966,1.665)--(9.967,1.664)%
  --(9.967,1.663)--(9.968,1.663)--(9.968,1.662)--(9.969,1.662)--(9.969,1.663)--(9.970,1.662)%
  --(9.971,1.660)--(9.971,1.661)--(9.972,1.661)--(9.972,1.660)--(9.973,1.661)--(9.974,1.661)%
  --(9.975,1.661)--(9.975,1.660)--(9.975,1.661)--(9.976,1.661)--(9.976,1.660)--(9.977,1.659)%
  --(9.978,1.659)--(9.978,1.660)--(9.979,1.659)--(9.980,1.658)--(9.980,1.659)--(9.980,1.658)%
  --(9.981,1.658)--(9.981,1.657)--(9.982,1.657)--(9.983,1.656)--(9.983,1.655)--(9.984,1.654)%
  --(9.985,1.654)--(9.985,1.653)--(9.985,1.652)--(9.986,1.652)--(9.986,1.651)--(9.987,1.651)%
  --(9.987,1.652)--(9.988,1.651)--(9.989,1.651)--(9.989,1.650)--(9.990,1.651)--(9.991,1.651)%
  --(9.991,1.650)--(9.992,1.649)--(9.992,1.647)--(9.993,1.648)--(9.994,1.647)--(9.994,1.646)%
  --(9.995,1.644)--(9.995,1.643)--(9.996,1.643)--(9.996,1.642)--(9.997,1.641)--(9.997,1.642)%
  --(9.998,1.641)--(9.999,1.640)--(9.999,1.639)--(9.999,1.636)--(10.000,1.635)--(10.001,1.636)%
  --(10.001,1.637)--(10.001,1.628)--(10.002,1.630)--(10.002,1.631)--(10.002,1.634)--(10.003,1.636)%
  --(10.003,1.640)--(10.004,1.642)--(10.004,1.643)--(10.004,1.644)--(10.005,1.644)--(10.005,1.646)%
  --(10.006,1.646)--(10.006,1.648)--(10.006,1.650)--(10.007,1.651)--(10.007,1.652)--(10.008,1.653)%
  --(10.008,1.654)--(10.009,1.656)--(10.009,1.657)--(10.009,1.658)--(10.010,1.659)--(10.011,1.660)%
  --(10.011,1.662)--(10.012,1.663)--(10.013,1.664)--(10.013,1.666)--(10.013,1.667)--(10.014,1.668)%
  --(10.014,1.670)--(10.015,1.671)--(10.015,1.672)--(10.016,1.675)--(10.016,1.678)--(10.016,1.679)%
  --(10.017,1.681)--(10.017,1.683)--(10.018,1.685)--(10.018,1.687)--(10.018,1.688)--(10.019,1.690)%
  --(10.019,1.693)--(10.020,1.696)--(10.020,1.698)--(10.020,1.700)--(10.021,1.703)--(10.021,1.705)%
  --(10.022,1.707)--(10.022,1.711)--(10.022,1.713)--(10.023,1.714)--(10.023,1.716)--(10.023,1.717)%
  --(10.024,1.718)--(10.024,1.721)--(10.025,1.723)--(10.025,1.726)--(10.025,1.728)--(10.026,1.728)%
  --(10.026,1.730)--(10.027,1.733)--(10.027,1.735)--(10.028,1.736)--(10.029,1.738)--(10.029,1.742)%
  --(10.029,1.743)--(10.030,1.745)--(10.030,1.748)--(10.031,1.750)--(10.031,1.753)--(10.032,1.755)%
  --(10.032,1.757)--(10.032,1.758)--(10.033,1.759)--(10.033,1.761)--(10.034,1.763)--(10.034,1.766)%
  --(10.034,1.767)--(10.035,1.769)--(10.035,1.771)--(10.035,1.773)--(10.036,1.775)--(10.036,1.777)%
  --(10.037,1.778)--(10.037,1.780)--(10.037,1.782)--(10.038,1.785)--(10.038,1.787)--(10.039,1.787)%
  --(10.039,1.789)--(10.040,1.790)--(10.040,1.791)--(10.041,1.792)--(10.042,1.794)--(10.042,1.797)%
  --(10.042,1.798)--(10.043,1.800)--(10.043,1.802)--(10.044,1.805)--(10.044,1.806)--(10.044,1.807)%
  --(10.045,1.808)--(10.045,1.811)--(10.046,1.812)--(10.046,1.816)--(10.046,1.818)--(10.047,1.820)%
  --(10.047,1.821)--(10.048,1.823)--(10.048,1.826)--(10.048,1.827)--(10.049,1.829)--(10.049,1.831)%
  --(10.049,1.833)--(10.050,1.836)--(10.050,1.838)--(10.051,1.841)--(10.051,1.844)--(10.051,1.846)%
  --(10.052,1.847)--(10.052,1.848)--(10.053,1.852)--(10.053,1.854)--(10.054,1.855)--(10.054,1.857)%
  --(10.055,1.860)--(10.055,1.863)--(10.055,1.865)--(10.056,1.867)--(10.056,1.868)--(10.056,1.870)%
  --(10.057,1.872)--(10.057,1.875)--(10.058,1.877)--(10.058,1.879)--(10.059,1.881)--(10.059,1.883)%
  --(10.060,1.885)--(10.060,1.886)--(10.060,1.890)--(10.061,1.891)--(10.061,1.893)--(10.062,1.894)%
  --(10.062,1.895)--(10.062,1.896)--(10.063,1.898)--(10.063,1.900)--(10.064,1.900)--(10.064,1.902)%
  --(10.065,1.902)--(10.065,1.905)--(10.065,1.907)--(10.066,1.908)--(10.066,1.910)--(10.067,1.910)%
  --(10.067,1.912)--(10.067,1.915)--(10.068,1.917)--(10.068,1.918)--(10.069,1.920)--(10.069,1.922)%
  --(10.069,1.921)--(10.070,1.922)--(10.070,1.923)--(10.070,1.925)--(10.071,1.928)--(10.071,1.930)%
  --(10.072,1.932)--(10.072,1.933)--(10.072,1.935)--(10.073,1.938)--(10.073,1.939)--(10.074,1.941)%
  --(10.074,1.943)--(10.074,1.945)--(10.075,1.947)--(10.075,1.950)--(10.075,1.951)--(10.076,1.954)%
  --(10.076,1.957)--(10.077,1.959)--(10.077,1.960)--(10.077,1.961)--(10.078,1.963)--(10.079,1.965)%
  --(10.079,1.967)--(10.079,1.970)--(10.080,1.972)--(10.080,1.974)--(10.081,1.975)--(10.081,1.976)%
  --(10.081,1.978)--(10.082,1.978)--(10.082,1.980)--(10.083,1.982)--(10.083,1.983)--(10.084,1.984)%
  --(10.084,1.986)--(10.085,1.987)--(10.085,1.988)--(10.086,1.988)--(10.086,1.990)--(10.087,1.991)%
  --(10.088,1.994)--(10.088,1.995)--(10.089,1.997)--(10.089,2.000)--(10.090,2.000)--(10.090,2.002)%
  --(10.091,2.004)--(10.091,2.005)--(10.091,2.008)--(10.092,2.011)--(10.092,2.013)--(10.093,2.015)%
  --(10.093,2.016)--(10.093,2.017)--(10.094,2.019)--(10.094,2.022)--(10.095,2.023)--(10.095,2.025)%
  --(10.095,2.027)--(10.096,2.027)--(10.096,2.029)--(10.096,2.031)--(10.097,2.034)--(10.097,2.036)%
  --(10.098,2.038)--(10.098,2.039)--(10.098,2.041)--(10.099,2.043)--(10.099,2.045)--(10.100,2.047)%
  --(10.100,2.048)--(10.101,2.050)--(10.101,2.052)--(10.102,2.055)--(10.102,2.057)--(10.102,2.060)%
  --(10.103,2.063)--(10.103,2.065)--(10.104,2.065)--(10.105,2.067)--(10.105,2.069)--(10.105,2.070)%
  --(10.106,2.071)--(10.106,2.072)--(10.107,2.074)--(10.107,2.075)--(10.107,2.078)--(10.108,2.080)%
  --(10.108,2.083)--(10.108,2.085)--(10.109,2.086)--(10.109,2.088)--(10.110,2.090)--(10.110,2.093)%
  --(10.110,2.095)--(10.111,2.097)--(10.111,2.099)--(10.112,2.103)--(10.112,2.107)--(10.112,2.111)%
  --(10.113,2.113)--(10.114,2.114)--(10.114,2.117)--(10.115,2.120)--(10.115,2.122)--(10.115,2.123)%
  --(10.116,2.125)--(10.116,2.127)--(10.117,2.129)--(10.117,2.132)--(10.117,2.136)--(10.118,2.138)%
  --(10.118,2.140)--(10.119,2.143)--(10.119,2.147)--(10.119,2.148)--(10.120,2.150)--(10.120,2.153)%
  --(10.121,2.154)--(10.121,2.155)--(10.121,2.156)--(10.122,2.158)--(10.122,2.160)--(10.122,2.163)%
  --(10.123,2.165)--(10.123,2.168)--(10.124,2.170)--(10.124,2.172)--(10.124,2.175)--(10.125,2.175)%
  --(10.125,2.176)--(10.126,2.176)--(10.126,2.177)--(10.127,2.177)--(10.127,2.178)--(10.128,2.179)%
  --(10.128,2.181)--(10.129,2.182)--(10.129,2.183)--(10.130,2.183)--(10.130,2.186)--(10.131,2.193)%
  --(10.131,2.200)--(10.131,2.204)--(10.132,2.206)--(10.132,2.209)--(10.133,2.209)--(10.133,2.210)%
  --(10.134,2.210)--(10.134,2.211)--(10.135,2.211)--(10.135,2.210)--(10.135,2.212)--(10.136,2.213)%
  --(10.136,2.212)--(10.136,2.214)--(10.137,2.215)--(10.138,2.218)--(10.138,2.220)--(10.139,2.223)%
  --(10.139,2.224)--(10.140,2.226)--(10.140,2.228)--(10.141,2.228)--(10.141,2.230)--(10.142,2.230)%
  --(10.142,2.232)--(10.143,2.232)--(10.143,2.237)--(10.144,2.247)--(10.144,2.255)--(10.145,2.257)%
  --(10.145,2.259)--(10.145,2.261)--(10.146,2.263)--(10.146,2.264)--(10.147,2.269)--(10.147,2.271)%
  --(10.147,2.275)--(10.148,2.277)--(10.148,2.280)--(10.148,2.283)--(10.149,2.285)--(10.150,2.285)%
  --(10.150,2.286)--(10.151,2.287)--(10.152,2.292)--(10.152,2.295)--(10.152,2.299)--(10.153,2.303)%
  --(10.153,2.304)--(10.154,2.306)--(10.154,2.309)--(10.154,2.312)--(10.155,2.316)--(10.155,2.318)%
  --(10.156,2.322)--(10.156,2.324)--(10.157,2.328)--(10.157,2.332)--(10.157,2.336)--(10.158,2.341)%
  --(10.158,2.345)--(10.159,2.350)--(10.159,2.352)--(10.159,2.356)--(10.160,2.358)--(10.160,2.360)%
  --(10.161,2.362)--(10.161,2.365)--(10.161,2.368)--(10.162,2.368)--(10.162,2.370)--(10.162,2.372)%
  --(10.163,2.373)--(10.164,2.375)--(10.164,2.377)--(10.165,2.378)--(10.165,2.379)--(10.166,2.380)%
  --(10.166,2.381)--(10.166,2.382)--(10.167,2.384)--(10.167,2.385)--(10.168,2.387)--(10.168,2.388)%
  --(10.168,2.390)--(10.169,2.390)--(10.169,2.393)--(10.170,2.395)--(10.170,2.396)--(10.171,2.398)%
  --(10.171,2.400)--(10.171,2.403)--(10.172,2.403)--(10.172,2.407)--(10.173,2.412)--(10.173,2.413)%
  --(10.173,2.414)--(10.174,2.415)--(10.174,2.416)--(10.175,2.419)--(10.175,2.420)--(10.175,2.421)%
  --(10.176,2.423)--(10.176,2.425)--(10.176,2.426)--(10.177,2.427)--(10.177,2.429)--(10.178,2.429)%
  --(10.178,2.430)--(10.178,2.431)--(10.179,2.432)--(10.179,2.434)--(10.180,2.437)--(10.180,2.440)%
  --(10.180,2.442)--(10.181,2.442)--(10.181,2.443)--(10.181,2.445)--(10.182,2.446)--(10.182,2.448)%
  --(10.183,2.450)--(10.183,2.452)--(10.183,2.454)--(10.184,2.457)--(10.184,2.459)--(10.185,2.459)%
  --(10.185,2.461)--(10.185,2.462)--(10.186,2.463)--(10.186,2.464)--(10.187,2.465)--(10.187,2.467)%
  --(10.188,2.469)--(10.188,2.472)--(10.189,2.472)--(10.189,2.474)--(10.190,2.475)--(10.190,2.476)%
  --(10.191,2.477)--(10.191,2.478)--(10.192,2.479)--(10.192,2.482)--(10.193,2.483)--(10.194,2.485)%
  --(10.194,2.487)--(10.194,2.492)--(10.195,2.493)--(10.195,2.494)--(10.195,2.496)--(10.196,2.498)%
  --(10.196,2.500)--(10.197,2.500)--(10.197,2.502)--(10.197,2.504)--(10.198,2.507)--(10.198,2.509)%
  --(10.199,2.512)--(10.199,2.513)--(10.200,2.515)--(10.200,2.518)--(10.201,2.520)--(10.201,2.524)%
  --(10.202,2.526)--(10.202,2.530)--(10.202,2.531)--(10.203,2.534)--(10.203,2.535)--(10.204,2.537)%
  --(10.204,2.539)--(10.205,2.541)--(10.205,2.545)--(10.206,2.546)--(10.206,2.547)--(10.206,2.550)%
  --(10.207,2.552)--(10.207,2.553)--(10.208,2.553)--(10.208,2.556)--(10.208,2.558)--(10.209,2.560)%
  --(10.209,2.564)--(10.209,2.566)--(10.210,2.569)--(10.210,2.572)--(10.211,2.573)--(10.211,2.574)%
  --(10.211,2.578)--(10.212,2.580)--(10.212,2.584)--(10.213,2.586)--(10.213,2.588)--(10.213,2.589)%
  --(10.214,2.592)--(10.214,2.594)--(10.214,2.597)--(10.215,2.600)--(10.216,2.603)--(10.216,2.608)%
  --(10.216,2.610)--(10.217,2.611)--(10.217,2.615)--(10.218,2.614)--(10.218,2.616)--(10.218,2.619)%
  --(10.219,2.620)--(10.219,2.622)--(10.220,2.624)--(10.220,2.627)--(10.221,2.630)--(10.222,2.633)%
  --(10.222,2.635)--(10.223,2.636)--(10.223,2.638)--(10.223,2.639)--(10.224,2.641)--(10.224,2.644)%
  --(10.225,2.645)--(10.225,2.647)--(10.225,2.649)--(10.226,2.650)--(10.226,2.649)--(10.227,2.651)%
  --(10.227,2.653)--(10.227,2.656)--(10.228,2.660)--(10.228,2.662)--(10.228,2.666)--(10.229,2.667)%
  --(10.229,2.670)--(10.230,2.674)--(10.230,2.675)--(10.230,2.678)--(10.231,2.678)--(10.231,2.679)%
  --(10.232,2.682)--(10.232,2.684)--(10.232,2.687)--(10.233,2.689)--(10.233,2.690)--(10.234,2.692)%
  --(10.234,2.695)--(10.234,2.696)--(10.235,2.698)--(10.235,2.700)--(10.235,2.702)--(10.236,2.705)%
  --(10.236,2.708)--(10.237,2.709)--(10.237,2.712)--(10.238,2.713)--(10.238,2.716)--(10.239,2.717)%
  --(10.239,2.718)--(10.239,2.723)--(10.240,2.725)--(10.240,2.727)--(10.241,2.728)--(10.241,2.731)%
  --(10.241,2.732)--(10.242,2.735)--(10.242,2.737)--(10.242,2.739)--(10.243,2.740)--(10.243,2.742)%
  --(10.244,2.744)--(10.244,2.747)--(10.244,2.751)--(10.245,2.752)--(10.245,2.754)--(10.246,2.759)%
  --(10.246,2.762)--(10.246,2.765)--(10.247,2.769)--(10.247,2.772)--(10.247,2.774)--(10.248,2.777)%
  --(10.248,2.778)--(10.249,2.778)--(10.249,2.781)--(10.249,2.784)--(10.250,2.787)--(10.250,2.789)%
  --(10.251,2.791)--(10.251,2.795)--(10.251,2.800)--(10.252,2.804)--(10.253,2.809)--(10.253,2.812)%
  --(10.253,2.816)--(10.254,2.820)--(10.254,2.823)--(10.254,2.827)--(10.255,2.828)--(10.255,2.829)%
  --(10.256,2.829)--(10.256,2.830)--(10.257,2.832)--(10.257,2.833)--(10.258,2.833)--(10.258,2.837)%
  --(10.258,2.841)--(10.259,2.844)--(10.259,2.847)--(10.260,2.849)--(10.260,2.851)--(10.260,2.853)%
  --(10.261,2.854)--(10.261,2.857)--(10.261,2.859)--(10.262,2.864)--(10.262,2.867)--(10.263,2.870)%
  --(10.263,2.874)--(10.263,2.876)--(10.264,2.877)--(10.264,2.880)--(10.265,2.881)--(10.265,2.883)%
  --(10.265,2.887)--(10.266,2.890)--(10.266,2.893)--(10.267,2.895)--(10.267,2.898)--(10.267,2.900)%
  --(10.268,2.902)--(10.268,2.906)--(10.269,2.908)--(10.269,2.910)--(10.270,2.913)--(10.270,2.917)%
  --(10.271,2.915)--(10.271,2.917)--(10.272,2.918)--(10.272,2.920)--(10.272,2.922)--(10.273,2.926)%
  --(10.273,2.931)--(10.274,2.935)--(10.274,2.939)--(10.274,2.941)--(10.275,2.943)--(10.275,2.946)%
  --(10.275,2.947)--(10.276,2.949)--(10.276,2.953)--(10.277,2.956)--(10.277,2.957)--(10.278,2.959)%
  --(10.278,2.961)--(10.279,2.965)--(10.279,2.968)--(10.280,2.971)--(10.280,2.974)--(10.281,2.977)%
  --(10.281,2.980)--(10.282,2.984)--(10.282,2.985)--(10.282,2.986)--(10.283,2.990)--(10.283,2.991)%
  --(10.284,2.995)--(10.284,2.999)--(10.284,3.000)--(10.285,3.001)--(10.285,3.003)--(10.286,3.006)%
  --(10.286,3.009)--(10.286,3.011)--(10.287,3.013)--(10.287,3.015)--(10.287,3.018)--(10.288,3.020)%
  --(10.288,3.022)--(10.289,3.026)--(10.289,3.029)--(10.290,3.032)--(10.290,3.034)--(10.291,3.036)%
  --(10.291,3.039)--(10.291,3.043)--(10.292,3.044)--(10.292,3.045)--(10.293,3.045)--(10.293,3.049)%
  --(10.293,3.051)--(10.294,3.055)--(10.294,3.060)--(10.294,3.062)--(10.295,3.068)--(10.295,3.069)%
  --(10.296,3.072)--(10.296,3.074)--(10.296,3.077)--(10.297,3.078)--(10.297,3.081)--(10.298,3.082)%
  --(10.298,3.085)--(10.298,3.087)--(10.299,3.090)--(10.299,3.093)--(10.300,3.097)--(10.300,3.100)%
  --(10.300,3.105)--(10.301,3.105)--(10.301,3.107)--(10.301,3.111)--(10.302,3.113)--(10.302,3.118)%
  --(10.303,3.117)--(10.303,3.119)--(10.303,3.122)--(10.304,3.124)--(10.304,3.125)--(10.305,3.126)%
  --(10.305,3.128)--(10.305,3.129)--(10.306,3.131)--(10.306,3.133)--(10.307,3.132)--(10.307,3.134)%
  --(10.308,3.136)--(10.308,3.138)--(10.309,3.138)--(10.309,3.141)--(10.310,3.144)--(10.310,3.146)%
  --(10.310,3.148)--(10.311,3.148)--(10.311,3.150)--(10.312,3.153)--(10.312,3.155)--(10.312,3.157)%
  --(10.313,3.159)--(10.313,3.162)--(10.314,3.166)--(10.314,3.167)--(10.314,3.171)--(10.315,3.175)%
  --(10.315,3.177)--(10.316,3.180)--(10.316,3.181)--(10.317,3.184)--(10.317,3.187)--(10.318,3.189)%
  --(10.318,3.191)--(10.319,3.194)--(10.319,3.198)--(10.319,3.201)--(10.320,3.202)--(10.320,3.205)%
  --(10.320,3.209)--(10.321,3.213)--(10.321,3.215)--(10.322,3.219)--(10.322,3.223)--(10.322,3.225)%
  --(10.323,3.225)--(10.323,3.230)--(10.324,3.233)--(10.324,3.235)--(10.324,3.239)--(10.325,3.243)%
  --(10.325,3.244)--(10.326,3.247)--(10.326,3.248)--(10.326,3.249)--(10.327,3.251)--(10.327,3.255)%
  --(10.327,3.256)--(10.328,3.258)--(10.328,3.262)--(10.329,3.264)--(10.329,3.267)--(10.329,3.269)%
  --(10.330,3.272)--(10.330,3.275)--(10.331,3.278)--(10.331,3.280)--(10.331,3.283)--(10.332,3.286)%
  --(10.332,3.290)--(10.333,3.293)--(10.333,3.295)--(10.333,3.300)--(10.334,3.300)--(10.334,3.303)%
  --(10.334,3.304)--(10.335,3.310)--(10.335,3.313)--(10.336,3.316)--(10.336,3.322)--(10.336,3.324)%
  --(10.337,3.330)--(10.337,3.334)--(10.338,3.335)--(10.338,3.337)--(10.338,3.340)--(10.339,3.343)%
  --(10.339,3.347)--(10.340,3.351)--(10.340,3.356)--(10.340,3.359)--(10.341,3.361)--(10.341,3.362)%
  --(10.341,3.365)--(10.342,3.366)--(10.342,3.368)--(10.343,3.372)--(10.343,3.374)--(10.343,3.376)%
  --(10.344,3.378)--(10.345,3.380)--(10.345,3.381)--(10.346,3.382)--(10.346,3.384)--(10.347,3.386)%
  --(10.347,3.389)--(10.347,3.393)--(10.348,3.396)--(10.348,3.400)--(10.348,3.399)--(10.349,3.403)%
  --(10.349,3.407)--(10.350,3.410)--(10.350,3.412)--(10.350,3.415)--(10.351,3.417)--(10.351,3.419)%
  --(10.352,3.422)--(10.352,3.426)--(10.352,3.427)--(10.353,3.432)--(10.353,3.434)--(10.353,3.438)%
  --(10.354,3.444)--(10.354,3.446)--(10.355,3.449)--(10.355,3.450)--(10.355,3.452)--(10.356,3.456)%
  --(10.356,3.457)--(10.357,3.458)--(10.357,3.460)--(10.357,3.462)--(10.358,3.467)--(10.358,3.471)%
  --(10.359,3.471)--(10.359,3.475)--(10.359,3.476)--(10.360,3.476)--(10.360,3.479)--(10.360,3.481)%
  --(10.361,3.482)--(10.361,3.483)--(10.362,3.487)--(10.362,3.488)--(10.362,3.493)--(10.363,3.494)%
  --(10.363,3.499)--(10.364,3.503)--(10.364,3.504)--(10.364,3.509)--(10.365,3.512)--(10.365,3.517)%
  --(10.366,3.523)--(10.366,3.528)--(10.366,3.533)--(10.367,3.537)--(10.367,3.540)--(10.367,3.543)%
  --(10.368,3.544)--(10.368,3.548)--(10.369,3.552)--(10.369,3.557)--(10.369,3.556)--(10.370,3.560)%
  --(10.370,3.565)--(10.371,3.567)--(10.371,3.572)--(10.371,3.577)--(10.372,3.581)--(10.372,3.588)%
  --(10.373,3.593)--(10.373,3.598)--(10.373,3.601)--(10.374,3.607)--(10.374,3.610)--(10.374,3.612)%
  --(10.375,3.611)--(10.375,3.615)--(10.376,3.615)--(10.376,3.618)--(10.376,3.621)--(10.377,3.621)%
  --(10.377,3.623)--(10.378,3.624)--(10.378,3.627)--(10.379,3.627)--(10.379,3.629)--(10.380,3.632)%
  --(10.380,3.636)--(10.380,3.638)--(10.381,3.638)--(10.381,3.642)--(10.381,3.645)--(10.382,3.650)%
  --(10.382,3.652)--(10.383,3.656)--(10.383,3.659)--(10.383,3.666)--(10.384,3.669)--(10.384,3.672)%
  --(10.385,3.675)--(10.385,3.681)--(10.385,3.684)--(10.386,3.686)--(10.386,3.688)--(10.387,3.694)%
  --(10.387,3.696)--(10.387,3.699)--(10.388,3.702)--(10.388,3.706)--(10.388,3.709)--(10.389,3.710)%
  --(10.389,3.715)--(10.390,3.717)--(10.390,3.720)--(10.390,3.725)--(10.391,3.729)--(10.392,3.733)%
  --(10.392,3.737)--(10.392,3.738)--(10.393,3.741)--(10.393,3.743)--(10.394,3.746)--(10.394,3.748)%
  --(10.395,3.753)--(10.395,3.757)--(10.395,3.760)--(10.396,3.763)--(10.396,3.765)--(10.397,3.768)%
  --(10.397,3.773)--(10.397,3.775)--(10.398,3.779)--(10.398,3.781)--(10.399,3.785)--(10.399,3.789)%
  --(10.399,3.793)--(10.400,3.796)--(10.400,3.797)--(10.400,3.798)--(10.401,3.801)--(10.401,3.802)%
  --(10.402,3.805)--(10.402,3.807)--(10.402,3.811)--(10.403,3.814)--(10.403,3.817)--(10.404,3.818)%
  --(10.404,3.821)--(10.404,3.823)--(10.405,3.823)--(10.405,3.827)--(10.406,3.829)--(10.406,3.836)%
  --(10.406,3.838)--(10.407,3.841)--(10.407,3.843)--(10.407,3.847)--(10.408,3.849)--(10.408,3.852)%
  --(10.409,3.854)--(10.409,3.858)--(10.410,3.858)--(10.411,3.862)--(10.411,3.864)--(10.412,3.866)%
  --(10.412,3.867)--(10.413,3.870)--(10.413,3.873)--(10.413,3.879)--(10.414,3.880)--(10.414,3.882)%
  --(10.414,3.883)--(10.415,3.883)--(10.415,3.884)--(10.416,3.889)--(10.416,3.887)--(10.416,3.889)%
  --(10.417,3.891)--(10.417,3.895)--(10.418,3.895)--(10.418,3.898)--(10.418,3.899)--(10.419,3.899)%
  --(10.419,3.900)--(10.420,3.901)--(10.420,3.903)--(10.420,3.905)--(10.421,3.909)--(10.421,3.912)%
  --(10.422,3.912)--(10.423,3.915)--(10.423,3.916)--(10.423,3.918)--(10.424,3.918)--(10.424,3.920)%
  --(10.425,3.922)--(10.425,3.927)--(10.426,3.932)--(10.426,3.933)--(10.426,3.935)--(10.427,3.939)%
  --(10.427,3.940)--(10.428,3.944)--(10.428,3.945)--(10.428,3.948)--(10.429,3.954)--(10.429,3.958)%
  --(10.430,3.961)--(10.430,3.962)--(10.430,3.964)--(10.431,3.969)--(10.431,3.972)--(10.432,3.977)%
  --(10.432,3.980)--(10.432,3.984)--(10.433,3.986)--(10.433,3.988)--(10.433,3.989)--(10.434,3.992)%
  --(10.435,3.995)--(10.435,3.999)--(10.435,4.002)--(10.436,4.005)--(10.436,4.007)--(10.437,4.009)%
  --(10.437,4.012)--(10.438,4.014)--(10.438,4.015)--(10.439,4.016)--(10.439,4.019)--(10.439,4.022)%
  --(10.440,4.022)--(10.440,4.025)--(10.441,4.027)--(10.441,4.028)--(10.442,4.030)--(10.442,4.033)%
  --(10.442,4.034)--(10.443,4.039)--(10.443,4.035)--(10.444,4.036)--(10.444,4.038)--(10.444,4.044)%
  --(10.445,4.046)--(10.445,4.049)--(10.446,4.048)--(10.446,4.051)--(10.446,4.052)--(10.447,4.055)%
  --(10.447,4.056)--(10.448,4.057)--(10.448,4.058)--(10.449,4.062)--(10.449,4.063)--(10.450,4.067)%
  --(10.450,4.069)--(10.451,4.068)--(10.451,4.072)--(10.451,4.074)--(10.452,4.077)--(10.452,4.078)%
  --(10.453,4.080)--(10.453,4.084)--(10.453,4.087)--(10.454,4.091)--(10.454,4.093)--(10.454,4.095)%
  --(10.455,4.096)--(10.455,4.103)--(10.456,4.110)--(10.456,4.113)--(10.457,4.117)--(10.457,4.118)%
  --(10.458,4.123)--(10.458,4.124)--(10.458,4.126)--(10.459,4.126)--(10.459,4.128)--(10.459,4.133)%
  --(10.460,4.134)--(10.460,4.141)--(10.461,4.142)--(10.461,4.146)--(10.461,4.149)--(10.462,4.149)%
  --(10.462,4.152)--(10.463,4.156)--(10.463,4.157)--(10.463,4.159)--(10.464,4.163)--(10.465,4.169)%
  --(10.465,4.172)--(10.466,4.176)--(10.466,4.177)--(10.467,4.181)--(10.467,4.185)--(10.468,4.188)%
  --(10.468,4.189)--(10.468,4.194)--(10.469,4.197)--(10.470,4.197)--(10.470,4.202)--(10.470,4.204)%
  --(10.471,4.207)--(10.471,4.209)--(10.472,4.211)--(10.472,4.214)--(10.472,4.217)--(10.473,4.217)%
  --(10.473,4.220)--(10.473,4.218)--(10.474,4.221)--(10.474,4.222)--(10.475,4.223)--(10.475,4.226)%
  --(10.475,4.230)--(10.476,4.230)--(10.477,4.232)--(10.477,4.230)--(10.477,4.237)--(10.478,4.242)%
  --(10.479,4.244)--(10.479,4.246)--(10.480,4.251)--(10.480,4.254)--(10.481,4.253)--(10.482,4.254)%
  --(10.482,4.259)--(10.482,4.262)--(10.483,4.266)--(10.483,4.272)--(10.484,4.273)--(10.484,4.278)%
  --(10.484,4.281)--(10.485,4.285)--(10.485,4.288)--(10.486,4.293)--(10.486,4.294)--(10.486,4.295)%
  --(10.487,4.300)--(10.487,4.303)--(10.488,4.306)--(10.489,4.308)--(10.489,4.309)--(10.489,4.313)%
  --(10.490,4.321)--(10.490,4.322)--(10.491,4.324)--(10.491,4.328)--(10.492,4.335)--(10.492,4.338)%
  --(10.493,4.346)--(10.493,4.352)--(10.493,4.355)--(10.494,4.360)--(10.494,4.364)--(10.494,4.368)%
  --(10.495,4.368)--(10.495,4.370)--(10.496,4.370)--(10.496,4.369)--(10.497,4.370)--(10.498,4.375)%
  --(10.498,4.377)--(10.498,4.382)--(10.499,4.385)--(10.499,4.391)--(10.499,4.393)--(10.500,4.395)%
  --(10.500,4.396)--(10.501,4.397)--(10.501,4.399)--(10.501,4.402)--(10.502,4.403)--(10.502,4.408)%
  --(10.503,4.407)--(10.503,4.410)--(10.503,4.413)--(10.504,4.412)--(10.504,4.417)--(10.505,4.421)%
  --(10.505,4.425)--(10.506,4.429)--(10.506,4.433)--(10.506,4.436)--(10.507,4.438)--(10.508,4.441)%
  --(10.508,4.443)--(10.508,4.446)--(10.509,4.448)--(10.509,4.451)--(10.510,4.457)--(10.510,4.459)%
  --(10.510,4.465)--(10.511,4.474)--(10.511,4.477)--(10.512,4.484)--(10.512,4.486)--(10.512,4.493)%
  --(10.513,4.496)--(10.513,4.501)--(10.513,4.503)--(10.514,4.507)--(10.514,4.513)--(10.515,4.516)%
  --(10.516,4.517)--(10.516,4.521)--(10.517,4.526)--(10.517,4.524)--(10.517,4.530)--(10.518,4.531)%
  --(10.518,4.532)--(10.519,4.534)--(10.519,4.535)--(10.519,4.540)--(10.520,4.543)--(10.520,4.544)%
  --(10.520,4.549)--(10.521,4.550)--(10.521,4.555)--(10.522,4.558)--(10.522,4.563)--(10.522,4.566)%
  --(10.523,4.566)--(10.523,4.567)--(10.524,4.569)--(10.524,4.570)--(10.524,4.574)--(10.525,4.580)%
  --(10.525,4.582)--(10.526,4.583)--(10.526,4.589)--(10.526,4.593)--(10.527,4.597)--(10.527,4.600)%
  --(10.527,4.601)--(10.528,4.607)--(10.528,4.608)--(10.529,4.610)--(10.529,4.617)--(10.529,4.618)%
  --(10.530,4.620)--(10.530,4.623)--(10.531,4.628)--(10.531,4.633)--(10.531,4.639)--(10.532,4.642)%
  --(10.532,4.648)--(10.532,4.653)--(10.533,4.662)--(10.533,4.673)--(10.534,4.675)--(10.534,4.679)%
  --(10.534,4.680)--(10.535,4.683)--(10.535,4.692)--(10.536,4.697)--(10.536,4.706)--(10.536,4.713)%
  --(10.537,4.720)--(10.537,4.726)--(10.538,4.728)--(10.538,4.729)--(10.538,4.732)--(10.539,4.740)%
  --(10.539,4.742)--(10.539,4.746)--(10.540,4.754)--(10.540,4.760)--(10.541,4.768)--(10.541,4.773)%
  --(10.541,4.780)--(10.542,4.782)--(10.542,4.787)--(10.543,4.793)--(10.543,4.799)--(10.543,4.807)%
  --(10.544,4.811)--(10.544,4.813)--(10.545,4.816)--(10.545,4.821)--(10.545,4.825)--(10.546,4.827)%
  --(10.546,4.837)--(10.546,4.848)--(10.547,4.854)--(10.547,4.859)--(10.548,4.860)--(10.548,4.864)%
  --(10.548,4.868)--(10.549,4.870)--(10.549,4.871)--(10.550,4.874)--(10.550,4.882)--(10.550,4.887)%
  --(10.551,4.893)--(10.551,4.902)--(10.552,4.908)--(10.552,4.909)--(10.552,4.914)--(10.553,4.920)%
  --(10.553,4.924)--(10.553,4.935)--(10.554,4.942)--(10.554,4.948)--(10.555,4.949)--(10.555,4.956)%
  --(10.555,4.960)--(10.556,4.970)--(10.556,4.980)--(10.557,4.984)--(10.557,4.986)--(10.557,4.993)%
  --(10.558,5.002)--(10.558,5.006)--(10.559,5.013)--(10.559,5.018)--(10.559,5.021)--(10.560,5.031)%
  --(10.560,5.036)--(10.560,5.044)--(10.561,5.046)--(10.561,5.052)--(10.562,5.054)--(10.562,5.065)%
  --(10.562,5.069)--(10.563,5.074)--(10.563,5.081)--(10.564,5.086)--(10.564,5.089)--(10.564,5.093)%
  --(10.565,5.101)--(10.565,5.105)--(10.565,5.110)--(10.566,5.112)--(10.566,5.117)--(10.567,5.121)%
  --(10.567,5.133)--(10.567,5.136)--(10.568,5.143)--(10.568,5.153)--(10.569,5.159)--(10.569,5.161)%
  --(10.569,5.169)--(10.570,5.179)--(10.570,5.186)--(10.571,5.190)--(10.571,5.195)--(10.571,5.205)%
  --(10.572,5.213)--(10.572,5.225)--(10.572,5.228)--(10.573,5.235)--(10.573,5.246)--(10.574,5.256)%
  --(10.574,5.266)--(10.574,5.273)--(10.575,5.276)--(10.575,5.277)--(10.576,5.282)--(10.576,5.287)%
  --(10.576,5.285)--(10.577,5.293)--(10.577,5.304)--(10.578,5.310)--(10.578,5.317)--(10.578,5.320)%
  --(10.579,5.324)--(10.579,5.334)--(10.579,5.345)--(10.580,5.352)--(10.580,5.360)--(10.581,5.369)%
  --(10.581,5.377)--(10.581,5.380)--(10.582,5.385)--(10.582,5.393)--(10.583,5.404)--(10.583,5.408)%
  --(10.583,5.411)--(10.584,5.416)--(10.584,5.425)--(10.585,5.431)--(10.585,5.432)--(10.585,5.439)%
  --(10.586,5.442)--(10.586,5.449)--(10.586,5.452)--(10.587,5.459)--(10.587,5.461)--(10.588,5.467)%
  --(10.588,5.473)--(10.588,5.477)--(10.589,5.483)--(10.589,5.490)--(10.590,5.493)--(10.590,5.501)%
  --(10.590,5.504)--(10.591,5.509)--(10.591,5.512)--(10.592,5.519)--(10.592,5.523)--(10.592,5.526)%
  --(10.593,5.532)--(10.593,5.539)--(10.593,5.545)--(10.594,5.549)--(10.594,5.552)--(10.595,5.559)%
  --(10.595,5.567)--(10.595,5.574)--(10.596,5.585)--(10.596,5.589)--(10.597,5.593)--(10.597,5.596)%
  --(10.597,5.600)--(10.598,5.602)--(10.598,5.606)--(10.599,5.614)--(10.599,5.619)--(10.599,5.627)%
  --(10.600,5.633)--(10.600,5.634)--(10.600,5.640)--(10.601,5.645)--(10.601,5.650)--(10.602,5.650)%
  --(10.602,5.651)--(10.602,5.650)--(10.603,5.650)--(10.603,5.655)--(10.604,5.659)--(10.604,5.668)%
  --(10.604,5.669)--(10.605,5.673)--(10.605,5.682)--(10.605,5.692)--(10.606,5.698)--(10.606,5.701)%
  --(10.607,5.701)--(10.607,5.705)--(10.607,5.710)--(10.608,5.718)--(10.608,5.719)--(10.609,5.724)%
  --(10.609,5.730)--(10.609,5.741)--(10.610,5.752)--(10.610,5.758)--(10.611,5.767)--(10.611,5.773)%
  --(10.611,5.776)--(10.612,5.785)--(10.612,5.791)--(10.612,5.798)--(10.613,5.804)--(10.613,5.810)%
  --(10.614,5.822)--(10.614,5.831)--(10.614,5.842)--(10.615,5.856)--(10.615,5.864)--(10.616,5.873)%
  --(10.616,5.874)--(10.616,5.885)--(10.617,5.895)--(10.617,5.902)--(10.618,5.908)--(10.618,5.916)%
  --(10.618,5.918)--(10.619,5.918)--(10.619,5.929)--(10.619,5.933)--(10.620,5.940)--(10.620,5.951)%
  --(10.621,5.959)--(10.621,5.963)--(10.621,5.966)--(10.622,5.975)--(10.622,5.984)--(10.623,5.988)%
  --(10.623,5.994)--(10.623,6.010)--(10.624,6.028)--(10.624,6.037)--(10.625,6.039)--(10.625,6.043)%
  --(10.625,6.050)--(10.626,6.062)--(10.626,6.070)--(10.626,6.077)--(10.627,6.084)--(10.627,6.095)%
  --(10.628,6.105)--(10.628,6.121)--(10.628,6.145)--(10.629,6.160)--(10.629,6.167)--(10.630,6.178)%
  --(10.630,6.179)--(10.630,6.088)--(10.631,6.079)--(10.631,6.070)--(10.632,6.060)--(10.632,6.054)%
  --(10.632,6.046)--(10.633,6.041)--(10.633,6.030)--(10.633,6.022)--(10.634,6.013)--(10.634,6.006)%
  --(10.635,5.994)--(10.635,5.977)--(10.635,5.969)--(10.636,5.968)--(10.636,5.964)--(10.637,5.961)%
  --(10.638,5.948)--(10.638,5.938)--(10.638,5.933)--(10.639,5.928)--(10.639,5.924)--(10.640,5.922)%
  --(10.640,5.910)--(10.640,5.893)--(10.641,5.866)--(10.641,5.856)--(10.642,5.851)--(10.642,5.846)%
  --(10.642,5.838)--(10.643,5.818)--(10.643,5.812)--(10.644,5.809)--(10.644,5.803)--(10.644,5.799)%
  --(10.645,5.786)--(10.645,5.777)--(10.645,5.774)--(10.646,5.763)--(10.646,5.750)--(10.647,5.738)%
  --(10.647,5.735)--(10.647,5.726)--(10.648,5.715)--(10.648,5.707)--(10.649,5.693)--(10.649,5.682)%
  --(10.649,5.667)--(10.650,5.658)--(10.650,5.644)--(10.651,5.639)--(10.651,5.621)--(10.651,5.608)%
  --(10.652,5.598)--(10.652,5.592)--(10.652,5.584)--(10.653,5.579)--(10.653,5.569)--(10.654,5.556)%
  --(10.654,5.548)--(10.655,5.544)--(10.655,5.530)--(10.656,5.515)--(10.656,5.511)--(10.656,5.508)%
  --(10.657,5.506)--(10.657,5.496)--(10.658,5.481)--(10.658,5.463)--(10.658,5.453)--(10.659,5.442)%
  --(10.659,5.432)--(10.659,5.431)--(10.660,5.421)--(10.660,5.407)--(10.661,5.402)--(10.661,5.397)%
  --(10.661,5.394)--(10.662,5.384)--(10.662,5.378)--(10.663,5.374)--(10.663,5.363)--(10.663,5.342)%
  --(10.664,5.341)--(10.664,5.337)--(10.665,5.332)--(10.665,5.330)--(10.665,5.323)--(10.666,5.312)%
  --(10.666,5.303)--(10.666,5.293)--(10.667,5.281)--(10.667,5.267)--(10.668,5.254)--(10.668,5.243)%
  --(10.668,5.242)--(10.669,5.237)--(10.670,5.231)--(10.670,5.221)--(10.670,5.220)--(10.671,5.212)%
  --(10.671,5.198)--(10.672,5.185)--(10.672,5.184)--(10.672,5.173)--(10.673,5.161)--(10.673,5.154)%
  --(10.673,5.145)--(10.674,5.138)--(10.674,5.127)--(10.675,5.120)--(10.675,5.110)--(10.675,5.109)%
  --(10.676,5.101)--(10.676,5.096)--(10.677,5.093)--(10.677,5.090)--(10.677,5.084)--(10.678,5.082)%
  --(10.678,5.075)--(10.678,5.072)--(10.679,5.066)--(10.679,5.056)--(10.680,5.047)--(10.680,5.042)%
  --(10.680,5.036)--(10.681,5.035)--(10.681,5.023)--(10.682,5.010)--(10.682,5.002)--(10.682,4.995)%
  --(10.683,4.993)--(10.683,4.983)--(10.684,4.978)--(10.684,4.976)--(10.684,4.968)--(10.685,4.962)%
  --(10.685,4.955)--(10.685,4.948)--(10.686,4.946)--(10.686,4.943)--(10.687,4.934)--(10.687,4.926)%
  --(10.687,4.913)--(10.688,4.898)--(10.688,4.884)--(10.689,4.875)--(10.689,4.871)--(10.689,4.866)%
  --(10.690,4.860)--(10.690,4.853)--(10.691,4.843)--(10.691,4.838)--(10.691,4.830)--(10.692,4.826)%
  --(10.692,4.821)--(10.692,4.822)--(10.693,4.812)--(10.693,4.810)--(10.694,4.802)--(10.694,4.784)%
  --(10.694,4.766)--(10.695,4.750)--(10.695,4.743)--(10.696,4.733)--(10.696,4.728)--(10.696,4.724)%
  --(10.697,4.722)--(10.697,4.717)--(10.698,4.711)--(10.698,4.706)--(10.698,4.700)--(10.699,4.694)%
  --(10.699,4.685)--(10.699,4.680)--(10.700,4.674)--(10.700,4.672)--(10.701,4.668)--(10.701,4.666)%
  --(10.701,4.658)--(10.702,4.650)--(10.702,4.637)--(10.703,4.632)--(10.703,4.626)--(10.703,4.617)%
  --(10.704,4.612)--(10.704,4.610)--(10.705,4.605)--(10.705,4.596)--(10.705,4.592)--(10.706,4.586)%
  --(10.706,4.576)--(10.706,4.568)--(10.707,4.564)--(10.707,4.560)--(10.708,4.554)--(10.708,4.549)%
  --(10.708,4.548)--(10.709,4.540)--(10.709,4.539)--(10.710,4.533)--(10.710,4.526)--(10.710,4.517)%
  --(10.711,4.508)--(10.711,4.502)--(10.711,4.493)--(10.712,4.489)--(10.712,4.482)--(10.713,4.476)%
  --(10.713,4.468)--(10.713,4.460)--(10.714,4.452)--(10.714,4.442)--(10.715,4.437)--(10.715,4.429)%
  --(10.715,4.424)--(10.716,4.419)--(10.716,4.414)--(10.717,4.405)--(10.717,4.400)--(10.717,4.395)%
  --(10.718,4.388)--(10.718,4.383)--(10.718,4.378)--(10.719,4.371)--(10.719,4.366)--(10.720,4.362)%
  --(10.720,4.354)--(10.720,4.346)--(10.721,4.339)--(10.721,4.338)--(10.722,4.333)--(10.722,4.330)%
  --(10.722,4.321)--(10.723,4.316)--(10.723,4.310)--(10.724,4.302)--(10.724,4.297)--(10.724,4.293)%
  --(10.725,4.289)--(10.725,4.281)--(10.725,4.272)--(10.726,4.262)--(10.726,4.251)--(10.727,4.248)%
  --(10.727,4.242)--(10.727,4.235)--(10.728,4.228)--(10.728,4.222)--(10.729,4.214)--(10.729,4.204)%
  --(10.729,4.197)--(10.730,4.194)--(10.730,4.189)--(10.731,4.187)--(10.731,4.179)--(10.731,4.170)%
  --(10.732,4.166)--(10.732,4.159)--(10.732,4.154)--(10.733,4.148)--(10.733,4.143)--(10.734,4.138)%
  --(10.734,4.131)--(10.734,4.127)--(10.735,4.120)--(10.735,4.114)--(10.736,4.109)--(10.736,4.104)%
  --(10.736,4.097)--(10.737,4.092)--(10.737,4.086)--(10.738,4.070)--(10.738,4.059)--(10.738,4.049)%
  --(10.739,4.046)--(10.739,4.038)--(10.739,4.030)--(10.740,4.023)--(10.740,4.016)--(10.741,4.008)%
  --(10.741,4.002)--(10.741,3.996)--(10.742,3.990)--(10.742,3.984)--(10.743,3.977)--(10.743,3.971)%
  --(10.743,3.968)--(10.744,3.964)--(10.744,3.959)--(10.744,3.952)--(10.745,3.936)--(10.745,3.928)%
  --(10.746,3.922)--(10.746,3.916)--(10.746,3.911)--(10.747,3.906)--(10.747,3.901)--(10.748,3.892)%
  --(10.748,3.875)--(10.748,3.868)--(10.749,3.863)--(10.749,3.857)--(10.750,3.847)--(10.750,3.839)%
  --(10.750,3.833)--(10.751,3.826)--(10.751,3.819)--(10.751,3.815)--(10.752,3.808)--(10.752,3.806)%
  --(10.753,3.805)--(10.753,3.802)--(10.754,3.800)--(10.754,3.796)--(10.755,3.793)--(10.755,3.791)%
  --(10.755,3.790)--(10.756,3.788)--(10.757,3.780)--(10.757,3.776)--(10.757,3.769)--(10.758,3.763)%
  --(10.758,3.760)--(10.758,3.755)--(10.759,3.752)--(10.759,3.748)--(10.760,3.747)--(10.760,3.743)%
  --(10.760,3.736)--(10.761,3.733)--(10.761,3.728)--(10.762,3.716)--(10.762,3.710)--(10.762,3.704)%
  --(10.763,3.700)--(10.763,3.692)--(10.764,3.688)--(10.764,3.680)--(10.764,3.675)--(10.765,3.667)%
  --(10.765,3.660)--(10.765,3.656)--(10.766,3.653)--(10.766,3.647)--(10.767,3.644)--(10.767,3.641)%
  --(10.767,3.639)--(10.768,3.636)--(10.768,3.632)--(10.769,3.627)--(10.769,3.624)--(10.769,3.619)%
  --(10.770,3.614)--(10.770,3.612)--(10.771,3.606)--(10.771,3.604)--(10.771,3.599)--(10.772,3.597)%
  --(10.772,3.594)--(10.772,3.590)--(10.773,3.588)--(10.773,3.584)--(10.774,3.582)--(10.774,3.580)%
  --(10.774,3.574)--(10.775,3.570)--(10.775,3.566)--(10.776,3.567)--(10.776,3.563)--(10.776,3.559)%
  --(10.777,3.557)--(10.777,3.554)--(10.778,3.549)--(10.778,3.546)--(10.779,3.540)--(10.779,3.532)%
  --(10.779,3.528)--(10.780,3.520)--(10.780,3.509)--(10.781,3.498)--(10.781,3.488)--(10.781,3.484)%
  --(10.782,3.481)--(10.782,3.476)--(10.783,3.471)--(10.783,3.467)--(10.783,3.457)--(10.784,3.453)%
  --(10.784,3.448)--(10.784,3.446)--(10.785,3.443)--(10.785,3.442)--(10.786,3.441)--(10.786,3.436)%
  --(10.786,3.429)--(10.787,3.425)--(10.787,3.417)--(10.788,3.414)--(10.788,3.409)--(10.788,3.404)%
  --(10.789,3.396)--(10.789,3.390)--(10.790,3.382)--(10.790,3.373)--(10.790,3.365)--(10.791,3.360)%
  --(10.791,3.353)--(10.791,3.349)--(10.792,3.342)--(10.792,3.339)--(10.793,3.333)--(10.793,3.332)%
  --(10.793,3.324)--(10.794,3.321)--(10.794,3.315)--(10.795,3.311)--(10.795,3.305)--(10.795,3.302)%
  --(10.796,3.297)--(10.796,3.293)--(10.797,3.291)--(10.797,3.287)--(10.797,3.283)--(10.798,3.279)%
  --(10.798,3.274)--(10.798,3.273)--(10.799,3.270)--(10.799,3.269)--(10.800,3.268)--(10.800,3.267)%
  --(10.800,3.268)--(10.801,3.264)--(10.801,3.261)--(10.802,3.257)--(10.802,3.256)--(10.802,3.252)%
  --(10.803,3.237)--(10.803,3.229)--(10.804,3.223)--(10.804,3.221)--(10.804,3.218)--(10.805,3.214)%
  --(10.805,3.209)--(10.805,3.203)--(10.806,3.200)--(10.806,3.199)--(10.807,3.194)--(10.807,3.193)%
  --(10.807,3.189)--(10.808,3.185)--(10.808,3.179)--(10.809,3.175)--(10.809,3.168)--(10.809,3.161)%
  --(10.810,3.157)--(10.810,3.154)--(10.811,3.150)--(10.811,3.149)--(10.811,3.144)--(10.812,3.139)%
  --(10.812,3.137)--(10.812,3.135)--(10.813,3.133)--(10.813,3.131)--(10.814,3.129)--(10.814,3.125)%
  --(10.814,3.121)--(10.815,3.117)--(10.815,3.114)--(10.816,3.101)--(10.816,3.097)--(10.816,3.091)%
  --(10.817,3.085)--(10.817,3.083)--(10.817,3.080)--(10.818,3.077)--(10.818,3.073)--(10.819,3.070)%
  --(10.819,3.067)--(10.819,3.063)--(10.820,3.060)--(10.820,3.056)--(10.821,3.050)--(10.821,3.049)%
  --(10.821,3.047)--(10.822,3.045)--(10.822,3.041)--(10.823,3.041)--(10.823,3.036)--(10.823,3.034)%
  --(10.824,3.031)--(10.824,3.029)--(10.824,3.027)--(10.825,3.024)--(10.825,3.019)--(10.826,3.016)%
  --(10.826,3.014)--(10.826,3.008)--(10.827,3.008)--(10.827,3.006)--(10.828,3.003)--(10.828,3.002)%
  --(10.828,3.000)--(10.829,2.998)--(10.829,2.995)--(10.830,2.991)--(10.830,2.986)--(10.830,2.981)%
  --(10.831,2.980)--(10.831,2.979)--(10.831,2.974)--(10.832,2.969)--(10.832,2.966)--(10.833,2.962)%
  --(10.833,2.959)--(10.833,2.958)--(10.834,2.956)--(10.834,2.954)--(10.835,2.952)--(10.835,2.951)%
  --(10.835,2.948)--(10.836,2.946)--(10.836,2.943)--(10.837,2.941)--(10.837,2.938)--(10.838,2.934)%
  --(10.838,2.931)--(10.838,2.927)--(10.839,2.927)--(10.839,2.923)--(10.840,2.921)--(10.840,2.917)%
  --(10.840,2.913)--(10.841,2.910)--(10.841,2.907)--(10.842,2.905)--(10.842,2.900)--(10.842,2.896)%
  --(10.843,2.893)--(10.843,2.891)--(10.844,2.885)--(10.844,2.880)--(10.844,2.872)--(10.845,2.869)%
  --(10.845,2.865)--(10.845,2.861)--(10.846,2.854)--(10.846,2.850)--(10.847,2.845)--(10.847,2.843)%
  --(10.847,2.839)--(10.848,2.836)--(10.848,2.833)--(10.849,2.829)--(10.849,2.827)--(10.849,2.825)%
  --(10.850,2.824)--(10.850,2.822)--(10.850,2.820)--(10.851,2.819)--(10.851,2.817)--(10.852,2.816)%
  --(10.852,2.814)--(10.853,2.809)--(10.853,2.806)--(10.854,2.804)--(10.854,2.801)--(10.855,2.799)%
  --(10.855,2.797)--(10.856,2.796)--(10.856,2.795)--(10.856,2.794)--(10.857,2.785)--(10.857,2.774)%
  --(10.857,2.771)--(10.858,2.768)--(10.858,2.767)--(10.859,2.765)--(10.859,2.762)--(10.859,2.757)%
  --(10.860,2.756)--(10.860,2.754)--(10.861,2.750)--(10.861,2.746)--(10.861,2.741)--(10.862,2.739)%
  --(10.862,2.738)--(10.863,2.733)--(10.863,2.731)--(10.863,2.729)--(10.864,2.728)--(10.864,2.727)%
  --(10.864,2.723)--(10.865,2.719)--(10.865,2.715)--(10.866,2.711)--(10.866,2.708)--(10.866,2.706)%
  --(10.867,2.704)--(10.867,2.702)--(10.868,2.698)--(10.868,2.695)--(10.868,2.693)--(10.869,2.691)%
  --(10.869,2.689)--(10.870,2.688)--(10.870,2.684)--(10.870,2.681)--(10.871,2.682)--(10.871,2.681)%
  --(10.871,2.679)--(10.872,2.676)--(10.872,2.673)--(10.873,2.671)--(10.873,2.667)--(10.873,2.663)%
  --(10.874,2.660)--(10.874,2.657)--(10.875,2.657)--(10.875,2.656)--(10.875,2.652)--(10.876,2.649)%
  --(10.876,2.645)--(10.877,2.643)--(10.877,2.640)--(10.878,2.637)--(10.878,2.633)--(10.879,2.630)%
  --(10.879,2.625)--(10.880,2.621)--(10.880,2.618)--(10.880,2.614)--(10.881,2.610)--(10.881,2.608)%
  --(10.882,2.606)--(10.882,2.607)--(10.882,2.605)--(10.883,2.603)--(10.883,2.600)--(10.884,2.599)%
  --(10.884,2.596)--(10.885,2.594)--(10.886,2.592)--(10.886,2.590)--(10.887,2.588)--(10.887,2.587)%
  --(10.887,2.584)--(10.888,2.581)--(10.888,2.579)--(10.889,2.575)--(10.889,2.570)--(10.889,2.569)%
  --(10.890,2.565)--(10.890,2.563)--(10.890,2.559)--(10.891,2.555)--(10.891,2.542)--(10.892,2.537)%
  --(10.892,2.536)--(10.892,2.534)--(10.893,2.532)--(10.893,2.533)--(10.894,2.527)--(10.894,2.523)%
  --(10.894,2.519)--(10.895,2.516)--(10.895,2.514)--(10.896,2.512)--(10.896,2.510)--(10.896,2.504)%
  --(10.897,2.502)--(10.897,2.499)--(10.897,2.496)--(10.898,2.495)--(10.899,2.494)--(10.899,2.493)%
  --(10.900,2.492)--(10.900,2.490)--(10.901,2.488)--(10.901,2.487)--(10.901,2.486)--(10.902,2.486)%
  --(10.902,2.483)--(10.903,2.479)--(10.903,2.478)--(10.903,2.477)--(10.904,2.475)--(10.904,2.473)%
  --(10.904,2.470)--(10.905,2.468)--(10.905,2.464)--(10.906,2.459)--(10.906,2.455)--(10.906,2.449)%
  --(10.907,2.447)--(10.907,2.442)--(10.908,2.440)--(10.908,2.436)--(10.908,2.432)--(10.909,2.431)%
  --(10.909,2.429)--(10.910,2.426)--(10.910,2.423)--(10.910,2.422)--(10.911,2.421)--(10.911,2.419)%
  --(10.911,2.417)--(10.912,2.417)--(10.912,2.413)--(10.913,2.410)--(10.913,2.408)--(10.913,2.405)%
  --(10.914,2.404)--(10.914,2.402)--(10.915,2.400)--(10.915,2.398)--(10.915,2.397)--(10.916,2.394)%
  --(10.916,2.393)--(10.917,2.391)--(10.917,2.389)--(10.917,2.388)--(10.918,2.384)--(10.918,2.381)%
  --(10.918,2.379)--(10.919,2.376)--(10.919,2.372)--(10.920,2.371)--(10.920,2.370)--(10.921,2.368)%
  --(10.921,2.364)--(10.922,2.361)--(10.922,2.360)--(10.922,2.358)--(10.923,2.356)--(10.923,2.352)%
  --(10.923,2.348)--(10.924,2.345)--(10.924,2.342)--(10.925,2.340)--(10.925,2.338)--(10.925,2.336)%
  --(10.926,2.334)--(10.927,2.332)--(10.927,2.329)--(10.927,2.327)--(10.928,2.324)--(10.928,2.322)%
  --(10.929,2.322)--(10.929,2.320)--(10.929,2.317)--(10.930,2.313)--(10.930,2.309)--(10.931,2.308)%
  --(10.931,2.307)--(10.932,2.305)--(10.932,2.303)--(10.932,2.301)--(10.933,2.298)--(10.933,2.295)%
  --(10.934,2.293)--(10.934,2.290)--(10.934,2.287)--(10.935,2.284)--(10.935,2.283)--(10.936,2.281)%
  --(10.936,2.280)--(10.936,2.277)--(10.937,2.273)--(10.937,2.271)--(10.937,2.268)--(10.938,2.265)%
  --(10.938,2.260)--(10.939,2.259)--(10.939,2.255)--(10.939,2.252)--(10.940,2.250)--(10.940,2.248)%
  --(10.941,2.246)--(10.941,2.244)--(10.941,2.242)--(10.942,2.239)--(10.942,2.238)--(10.943,2.237)%
  --(10.943,2.234)--(10.943,2.230)--(10.944,2.228)--(10.944,2.223)--(10.944,2.220)--(10.945,2.220)%
  --(10.945,2.218)--(10.946,2.216)--(10.946,2.213)--(10.946,2.211)--(10.947,2.208)--(10.947,2.206)%
  --(10.948,2.202)--(10.948,2.198)--(10.948,2.196)--(10.949,2.194)--(10.949,2.193)--(10.950,2.191)%
  --(10.950,2.186)--(10.950,2.184)--(10.951,2.182)--(10.951,2.180)--(10.951,2.179)--(10.952,2.178)%
  --(10.952,2.176)--(10.953,2.175)--(10.953,2.174)--(10.953,2.170)--(10.954,2.168)--(10.954,2.166)%
  --(10.955,2.164)--(10.955,2.162)--(10.955,2.161)--(10.956,2.159)--(10.956,2.157)--(10.956,2.155)%
  --(10.957,2.153)--(10.957,2.149)--(10.958,2.148)--(10.958,2.147)--(10.958,2.146)--(10.959,2.144)%
  --(10.959,2.143)--(10.960,2.142)--(10.960,2.139)--(10.960,2.137)--(10.961,2.133)--(10.961,2.131)%
  --(10.962,2.128)--(10.962,2.126)--(10.962,2.124)--(10.963,2.121)--(10.963,2.122)--(10.963,2.118)%
  --(10.964,2.114)--(10.964,2.112)--(10.965,2.110)--(10.965,2.106)--(10.965,2.105)--(10.966,2.104)%
  --(10.966,2.102)--(10.967,2.099)--(10.967,2.097)--(10.967,2.096)--(10.968,2.094)--(10.968,2.092)%
  --(10.969,2.091)--(10.969,2.089)--(10.969,2.087)--(10.970,2.087)--(10.970,2.086)--(10.970,2.085)%
  --(10.971,2.081)--(10.971,2.079)--(10.972,2.078)--(10.972,2.076)--(10.972,2.074)--(10.973,2.073)%
  --(10.973,2.070)--(10.974,2.069)--(10.974,2.067)--(10.974,2.066)--(10.975,2.062)--(10.975,2.059)%
  --(10.976,2.057)--(10.976,2.055)--(10.976,2.052)--(10.977,2.050)--(10.977,2.046)--(10.977,2.044)%
  --(10.978,2.043)--(10.978,2.038)--(10.979,2.038)--(10.979,2.036)--(10.979,2.034)--(10.980,2.031)%
  --(10.980,2.029)--(10.981,2.028)--(10.981,2.025)--(10.981,2.023)--(10.982,2.023)--(10.982,2.020)%
  --(10.983,2.019)--(10.983,2.017)--(10.983,2.015)--(10.984,2.012)--(10.984,2.010)--(10.984,2.009)%
  --(10.985,2.008)--(10.985,2.006)--(10.986,2.004)--(10.986,2.003)--(10.987,1.999)--(10.987,1.997)%
  --(10.988,1.995)--(10.988,1.992)--(10.988,1.990)--(10.989,1.988)--(10.989,1.987)--(10.990,1.982)%
  --(10.990,1.980)--(10.991,1.977)--(10.991,1.976)--(10.991,1.973)--(10.992,1.971)--(10.992,1.969)%
  --(10.993,1.969)--(10.993,1.967)--(10.994,1.965)--(10.995,1.963)--(10.995,1.962)--(10.996,1.960)%
  --(10.996,1.959)--(10.996,1.958)--(10.997,1.956)--(10.997,1.954)--(10.998,1.953)--(10.998,1.951)%
  --(10.998,1.949)--(10.999,1.949)--(10.999,1.947)--(11.000,1.946)--(11.000,1.944)--(11.001,1.942)%
  --(11.001,1.940)--(11.002,1.938)--(11.002,1.936)--(11.002,1.934)--(11.003,1.933)--(11.003,1.929)%
  --(11.003,1.927)--(11.004,1.925)--(11.005,1.922)--(11.005,1.921)--(11.005,1.919)--(11.006,1.916)%
  --(11.006,1.915)--(11.007,1.914)--(11.007,1.911)--(11.007,1.909)--(11.008,1.908)--(11.008,1.907)%
  --(11.009,1.904)--(11.009,1.901)--(11.009,1.900)--(11.010,1.899)--(11.010,1.898)--(11.010,1.896)%
  --(11.011,1.894)--(11.012,1.893)--(11.012,1.889)--(11.012,1.888)--(11.013,1.886)--(11.013,1.884)%
  --(11.014,1.882)--(11.014,1.880)--(11.014,1.878)--(11.015,1.876)--(11.015,1.874)--(11.016,1.871)%
  --(11.016,1.868)--(11.016,1.865)--(11.017,1.862)--(11.017,1.860)--(11.017,1.857)--(11.018,1.855)%
  --(11.018,1.854)--(11.019,1.851)--(11.019,1.850)--(11.019,1.848)--(11.020,1.847)--(11.020,1.846)%
  --(11.021,1.844)--(11.021,1.842)--(11.021,1.840)--(11.022,1.840)--(11.022,1.839)--(11.023,1.838)%
  --(11.023,1.836)--(11.023,1.834)--(11.024,1.834)--(11.024,1.831)--(11.025,1.829)--(11.025,1.827)%
  --(11.026,1.826)--(11.026,1.825)--(11.026,1.823)--(11.027,1.820)--(11.027,1.818)--(11.028,1.816)%
  --(11.028,1.815)--(11.028,1.814)--(11.029,1.813)--(11.029,1.811)--(11.029,1.810)--(11.030,1.809)%
  --(11.030,1.808)--(11.031,1.807)--(11.031,1.805)--(11.031,1.804)--(11.032,1.802)--(11.033,1.801)%
  --(11.033,1.800)--(11.033,1.797)--(11.034,1.796)--(11.034,1.795)--(11.035,1.794)--(11.035,1.792)%
  --(11.035,1.789)--(11.036,1.788)--(11.036,1.785)--(11.036,1.783)--(11.037,1.782)--(11.038,1.779)%
  --(11.038,1.777)--(11.039,1.775)--(11.039,1.774)--(11.040,1.772)--(11.040,1.770)--(11.040,1.769)%
  --(11.041,1.767)--(11.041,1.766)--(11.042,1.766)--(11.042,1.765)--(11.042,1.764)--(11.043,1.762)%
  --(11.043,1.761)--(11.043,1.759)--(11.044,1.758)--(11.044,1.757)--(11.045,1.756)--(11.045,1.754)%
  --(11.045,1.753)--(11.046,1.752)--(11.046,1.751)--(11.047,1.750)--(11.047,1.748)--(11.047,1.746)%
  --(11.048,1.744)--(11.048,1.743)--(11.049,1.741)--(11.049,1.739)--(11.049,1.737)--(11.050,1.735)%
  --(11.050,1.732)--(11.050,1.730)--(11.051,1.729)--(11.051,1.728)--(11.052,1.727)--(11.052,1.726)%
  --(11.052,1.724)--(11.053,1.722)--(11.053,1.721)--(11.054,1.721)--(11.054,1.719)--(11.054,1.717)%
  --(11.055,1.715)--(11.055,1.714)--(11.056,1.713)--(11.056,1.709)--(11.056,1.707)--(11.057,1.704)%
  --(11.057,1.702)--(11.057,1.700)--(11.058,1.699)--(11.059,1.697)--(11.059,1.695)--(11.059,1.694)%
  --(11.060,1.691)--(11.060,1.689)--(11.061,1.687)--(11.061,1.686)--(11.061,1.683)--(11.062,1.683)%
  --(11.062,1.682)--(11.062,1.679)--(11.063,1.677)--(11.063,1.676)--(11.064,1.674)--(11.064,1.673)%
  --(11.064,1.671)--(11.065,1.670)--(11.065,1.669)--(11.066,1.667)--(11.066,1.666)--(11.067,1.666)%
  --(11.068,1.664)--(11.068,1.662)--(11.068,1.661)--(11.069,1.660)--(11.069,1.658)--(11.069,1.656)%
  --(11.070,1.654)--(11.070,1.653)--(11.071,1.651)--(11.071,1.649)--(11.071,1.647)--(11.072,1.645)%
  --(11.072,1.643)--(11.073,1.643)--(11.073,1.641)--(11.074,1.640)--(11.075,1.639)--(11.075,1.637)%
  --(11.075,1.636)--(11.076,1.635)--(11.076,1.633)--(11.077,1.631)--(11.078,1.631)--(11.078,1.629)%
  --(11.079,1.628)--(11.080,1.626)--(11.080,1.624)--(11.080,1.623)--(11.081,1.623)--(11.081,1.620)%
  --(11.082,1.619)--(11.082,1.618)--(11.082,1.616)--(11.083,1.614)--(11.083,1.613)--(11.083,1.612)%
  --(11.084,1.612)--(11.084,1.610)--(11.085,1.607)--(11.085,1.606)--(11.085,1.605)--(11.086,1.605)%
  --(11.086,1.604)--(11.087,1.603)--(11.087,1.601)--(11.088,1.600)--(11.088,1.598)--(11.089,1.597)%
  --(11.089,1.596)--(11.090,1.594)--(11.090,1.593)--(11.090,1.591)--(11.091,1.590)--(11.091,1.588)%
  --(11.092,1.587)--(11.092,1.586)--(11.093,1.585)--(11.093,1.583)--(11.094,1.582)--(11.094,1.580)%
  --(11.095,1.578)--(11.095,1.576)--(11.096,1.574)--(11.096,1.573)--(11.096,1.571)--(11.097,1.569)%
  --(11.097,1.568)--(11.097,1.566)--(11.098,1.565)--(11.098,1.564)--(11.099,1.561)--(11.099,1.559)%
  --(11.099,1.558)--(11.100,1.557)--(11.100,1.556)--(11.101,1.556)--(11.101,1.555)--(11.101,1.554)%
  --(11.102,1.553)--(11.102,1.552)--(11.102,1.551)--(11.103,1.548)--(11.103,1.547)--(11.104,1.546)%
  --(11.104,1.545)--(11.105,1.543)--(11.106,1.542)--(11.106,1.541)--(11.106,1.540)--(11.107,1.538)%
  --(11.107,1.535)--(11.108,1.534)--(11.108,1.532)--(11.109,1.530)--(11.109,1.529)--(11.109,1.528)%
  --(11.110,1.526)--(11.111,1.525)--(11.111,1.522)--(11.112,1.521)--(11.112,1.520)--(11.113,1.519)%
  --(11.113,1.516)--(11.114,1.515)--(11.115,1.514)--(11.115,1.512)--(11.116,1.511)--(11.116,1.509)%
  --(11.116,1.508)--(11.117,1.507)--(11.117,1.506)--(11.118,1.506)--(11.118,1.505)--(11.119,1.504)%
  --(11.119,1.503)--(11.120,1.503)--(11.120,1.502)--(11.120,1.500)--(11.121,1.500)--(11.121,1.498)%
  --(11.122,1.497)--(11.122,1.496)--(11.123,1.494)--(11.123,1.493)--(11.124,1.492)--(11.124,1.491)%
  --(11.125,1.490)--(11.125,1.489)--(11.126,1.490)--(11.127,1.490)--(11.127,1.489)--(11.128,1.488)%
  --(11.128,1.487)--(11.129,1.487)--(11.129,1.488)--(11.130,1.488)--(11.131,1.486)--(11.131,1.484)%
  --(11.132,1.484)--(11.132,1.483)--(11.133,1.482)--(11.134,1.482)--(11.134,1.481)--(11.134,1.480)%
  --(11.135,1.478)--(11.135,1.476)--(11.135,1.475)--(11.136,1.474)--(11.136,1.473)--(11.137,1.474)%
  --(11.137,1.473)--(11.138,1.472)--(11.139,1.471)--(11.140,1.471)--(11.141,1.472)--(11.141,1.471)%
  --(11.141,1.470)--(11.142,1.469)--(11.142,1.468)--(11.143,1.467)--(11.143,1.465)--(11.144,1.464)%
  --(11.144,1.465)--(11.144,1.464)--(11.145,1.460)--(11.146,1.461)--(11.146,1.460)--(11.147,1.459)%
  --(11.148,1.459)--(11.148,1.458)--(11.148,1.457)--(11.149,1.456)--(11.149,1.455)--(11.149,1.454)%
  --(11.150,1.449)--(11.150,1.450)--(11.151,1.451)--(11.151,1.452)--(11.152,1.452)--(11.152,1.453)%
  --(11.153,1.454)--(11.154,1.455)--(11.154,1.456)--(11.155,1.457)--(11.155,1.458)--(11.156,1.458)%
  --(11.157,1.458)--(11.157,1.459)--(11.158,1.460)--(11.159,1.461)--(11.160,1.462)--(11.160,1.464)%
  --(11.160,1.466)--(11.161,1.467)--(11.161,1.468)--(11.162,1.469)--(11.162,1.470)--(11.162,1.471)%
  --(11.163,1.471)--(11.163,1.472)--(11.164,1.474)--(11.164,1.476)--(11.165,1.476)--(11.165,1.477)%
  --(11.165,1.478)--(11.166,1.478)--(11.166,1.479)--(11.167,1.480)--(11.167,1.481)--(11.168,1.482)%
  --(11.168,1.484)--(11.168,1.486)--(11.169,1.487)--(11.170,1.488)--(11.170,1.489)--(11.170,1.490)%
  --(11.171,1.491)--(11.171,1.492)--(11.172,1.493)--(11.172,1.494)--(11.173,1.496)--(11.173,1.497)%
  --(11.174,1.498)--(11.174,1.499)--(11.175,1.500)--(11.176,1.500)--(11.176,1.502)--(11.177,1.503)%
  --(11.177,1.505)--(11.177,1.507)--(11.178,1.508)--(11.178,1.509)--(11.179,1.510)--(11.179,1.511)%
  --(11.180,1.512)--(11.181,1.513)--(11.181,1.514)--(11.181,1.515)--(11.182,1.516)--(11.182,1.517)%
  --(11.182,1.518)--(11.183,1.520)--(11.183,1.521)--(11.184,1.521)--(11.184,1.522)--(11.185,1.523)%
  --(11.185,1.524)--(11.186,1.525)--(11.186,1.527)--(11.186,1.528)--(11.187,1.529)--(11.187,1.531)%
  --(11.188,1.532)--(11.188,1.533)--(11.189,1.533)--(11.189,1.534)--(11.190,1.536)--(11.190,1.537)%
  --(11.191,1.538)--(11.191,1.540)--(11.192,1.541)--(11.192,1.544)--(11.193,1.545)--(11.193,1.546)%
  --(11.193,1.547)--(11.194,1.547)--(11.194,1.548)--(11.195,1.549)--(11.195,1.550)--(11.196,1.552)%
  --(11.197,1.554)--(11.197,1.553)--(11.198,1.553)--(11.198,1.554)--(11.198,1.555)--(11.199,1.556)%
  --(11.199,1.557)--(11.200,1.558)--(11.200,1.560)--(11.201,1.562)--(11.201,1.563)--(11.202,1.564)%
  --(11.202,1.565)--(11.203,1.566)--(11.203,1.568)--(11.203,1.569)--(11.204,1.569)--(11.204,1.570)%
  --(11.205,1.571)--(11.205,1.572)--(11.206,1.573)--(11.207,1.574)--(11.207,1.576)--(11.207,1.579)%
  --(11.208,1.580)--(11.208,1.581)--(11.208,1.582)--(11.209,1.582)--(11.209,1.583)--(11.210,1.584)%
  --(11.210,1.585)--(11.210,1.586)--(11.211,1.588)--(11.211,1.589)--(11.212,1.589)--(11.212,1.590)%
  --(11.213,1.591)--(11.213,1.592)--(11.214,1.594)--(11.214,1.597)--(11.214,1.600)--(11.215,1.601)%
  --(11.215,1.603)--(11.216,1.603)--(11.217,1.605)--(11.218,1.607)--(11.218,1.608)--(11.219,1.608)%
  --(11.219,1.609)--(11.219,1.610)--(11.220,1.610)--(11.220,1.611)--(11.221,1.611)--(11.221,1.612)%
  --(11.221,1.613)--(11.222,1.616)--(11.222,1.617)--(11.223,1.619)--(11.223,1.620)--(11.224,1.620)%
  --(11.224,1.621)--(11.225,1.623)--(11.225,1.625)--(11.226,1.627)--(11.226,1.629)--(11.226,1.631)%
  --(11.227,1.630)--(11.228,1.631)--(11.228,1.632)--(11.228,1.633)--(11.229,1.634)--(11.229,1.636)%
  --(11.229,1.637)--(11.230,1.638)--(11.230,1.639)--(11.231,1.642)--(11.231,1.643)--(11.231,1.644)%
  --(11.232,1.645)--(11.232,1.646)--(11.233,1.647)--(11.233,1.649)--(11.233,1.651)--(11.234,1.652)%
  --(11.234,1.654)--(11.235,1.655)--(11.235,1.656)--(11.236,1.658)--(11.236,1.659)--(11.236,1.660)%
  --(11.237,1.661)--(11.237,1.662)--(11.238,1.663)--(11.238,1.664)--(11.238,1.666)--(11.239,1.667)%
  --(11.239,1.668)--(11.240,1.669)--(11.240,1.670)--(11.240,1.672)--(11.241,1.673)--(11.241,1.674)%
  --(11.241,1.675)--(11.242,1.677)--(11.242,1.678)--(11.243,1.679)--(11.243,1.681)--(11.243,1.683)%
  --(11.244,1.683)--(11.244,1.684)--(11.245,1.686)--(11.245,1.688)--(11.245,1.689)--(11.246,1.691)%
  --(11.246,1.693)--(11.247,1.693)--(11.247,1.695)--(11.247,1.697)--(11.248,1.699)--(11.248,1.701)%
  --(11.248,1.703)--(11.249,1.703)--(11.249,1.704)--(11.250,1.705)--(11.250,1.706)--(11.250,1.708)%
  --(11.251,1.709)--(11.251,1.711)--(11.252,1.713)--(11.252,1.714)--(11.252,1.716)--(11.253,1.718)%
  --(11.253,1.720)--(11.254,1.721)--(11.254,1.722)--(11.254,1.724)--(11.255,1.724)--(11.255,1.726)%
  --(11.255,1.728)--(11.256,1.729)--(11.256,1.730)--(11.257,1.732)--(11.257,1.735)--(11.257,1.739)%
  --(11.258,1.740)--(11.258,1.742)--(11.259,1.743)--(11.259,1.744)--(11.259,1.745)--(11.260,1.747)%
  --(11.260,1.749)--(11.261,1.751)--(11.261,1.752)--(11.261,1.754)--(11.262,1.755)--(11.262,1.757)%
  --(11.263,1.759)--(11.264,1.761)--(11.264,1.763)--(11.264,1.765)--(11.265,1.769)--(11.265,1.772)%
  --(11.266,1.773)--(11.266,1.775)--(11.266,1.777)--(11.267,1.778)--(11.267,1.781)--(11.268,1.785)%
  --(11.268,1.786)--(11.268,1.789)--(11.269,1.790)--(11.269,1.792)--(11.269,1.794)--(11.270,1.798)%
  --(11.270,1.799)--(11.271,1.802)--(11.271,1.803)--(11.271,1.804)--(11.272,1.805)--(11.272,1.806)%
  --(11.273,1.807)--(11.273,1.809)--(11.274,1.810)--(11.274,1.812)--(11.274,1.813)--(11.275,1.813)%
  --(11.275,1.815)--(11.276,1.815)--(11.276,1.817)--(11.276,1.818)--(11.277,1.820)--(11.277,1.821)%
  --(11.278,1.823)--(11.278,1.825)--(11.278,1.827)--(11.279,1.828)--(11.279,1.830)--(11.280,1.832)%
  --(11.280,1.833)--(11.280,1.835)--(11.281,1.838)--(11.281,1.840)--(11.281,1.843)--(11.282,1.845)%
  --(11.282,1.846)--(11.283,1.849)--(11.283,1.850)--(11.283,1.854)--(11.284,1.856)--(11.284,1.857)%
  --(11.285,1.859)--(11.285,1.861)--(11.285,1.860)--(11.286,1.861)--(11.287,1.862)--(11.288,1.863)%
  --(11.288,1.865)--(11.288,1.872)--(11.289,1.873)--(11.289,1.875)--(11.290,1.878)--(11.290,1.880)%
  --(11.290,1.882)--(11.291,1.883)--(11.291,1.885)--(11.292,1.886)--(11.292,1.889)--(11.293,1.890)%
  --(11.293,1.891)--(11.294,1.892)--(11.294,1.893)--(11.294,1.894)--(11.295,1.897)--(11.295,1.898)%
  --(11.295,1.900)--(11.296,1.902)--(11.296,1.904)--(11.297,1.905)--(11.297,1.906)--(11.297,1.907)%
  --(11.298,1.907)--(11.299,1.908)--(11.299,1.909)--(11.299,1.910)--(11.300,1.914)--(11.300,1.917)%
  --(11.301,1.919)--(11.301,1.922)--(11.301,1.927)--(11.302,1.929)--(11.302,1.933)--(11.302,1.935)%
  --(11.303,1.936)--(11.303,1.937)--(11.304,1.939)--(11.304,1.940)--(11.304,1.941)--(11.305,1.942)%
  --(11.305,1.943)--(11.306,1.945)--(11.306,1.946)--(11.306,1.949)--(11.307,1.951)--(11.307,1.953)%
  --(11.308,1.954)--(11.308,1.956)--(11.308,1.957)--(11.309,1.960)--(11.309,1.964)--(11.309,1.967)%
  --(11.310,1.969)--(11.310,1.971)--(11.311,1.974)--(11.311,1.975)--(11.311,1.978)--(11.312,1.978)%
  --(11.312,1.980)--(11.313,1.980)--(11.313,1.981)--(11.313,1.983)--(11.314,1.984)--(11.314,1.985)%
  --(11.314,1.986)--(11.315,1.987)--(11.315,1.988)--(11.316,1.991)--(11.316,1.993)--(11.316,1.994)%
  --(11.317,1.996)--(11.317,1.998)--(11.318,1.999)--(11.318,2.001)--(11.318,2.002)--(11.319,2.004)%
  --(11.319,2.005)--(11.320,2.005)--(11.320,2.007)--(11.321,2.009)--(11.321,2.010)--(11.321,2.011)%
  --(11.322,2.013)--(11.322,2.014)--(11.323,2.016)--(11.323,2.017)--(11.323,2.018)--(11.324,2.020)%
  --(11.324,2.022)--(11.325,2.024)--(11.325,2.027)--(11.326,2.027)--(11.326,2.030)--(11.327,2.031)%
  --(11.327,2.033)--(11.327,2.035)--(11.328,2.039)--(11.328,2.042)--(11.328,2.045)--(11.329,2.046)%
  --(11.329,2.050)--(11.330,2.051)--(11.330,2.052)--(11.330,2.054)--(11.331,2.058)--(11.331,2.061)%
  --(11.332,2.064)--(11.332,2.066)--(11.332,2.069)--(11.333,2.071)--(11.333,2.074)--(11.334,2.077)%
  --(11.334,2.078)--(11.334,2.080)--(11.335,2.081)--(11.335,2.083)--(11.335,2.085)--(11.336,2.087)%
  --(11.336,2.089)--(11.337,2.090)--(11.337,2.091)--(11.337,2.093)--(11.338,2.096)--(11.339,2.097)%
  --(11.339,2.099)--(11.339,2.102)--(11.340,2.105)--(11.340,2.108)--(11.341,2.109)--(11.341,2.110)%
  --(11.341,2.111)--(11.342,2.112)--(11.342,2.113)--(11.342,2.116)--(11.343,2.122)--(11.343,2.124)%
  --(11.344,2.125)--(11.344,2.128)--(11.344,2.129)--(11.345,2.129)--(11.345,2.132)--(11.346,2.134)%
  --(11.346,2.136)--(11.346,2.138)--(11.347,2.139)--(11.347,2.141)--(11.347,2.143)--(11.348,2.144)%
  --(11.348,2.146)--(11.349,2.148)--(11.349,2.149)--(11.349,2.152)--(11.350,2.153)--(11.350,2.154)%
  --(11.351,2.156)--(11.351,2.159)--(11.351,2.163)--(11.352,2.163)--(11.352,2.166)--(11.353,2.168)%
  --(11.353,2.169)--(11.353,2.171)--(11.354,2.173)--(11.355,2.175)--(11.356,2.178)--(11.356,2.183)%
  --(11.356,2.184)--(11.357,2.184)--(11.357,2.186)--(11.358,2.188)--(11.358,2.190)--(11.358,2.193)%
  --(11.359,2.195)--(11.359,2.197)--(11.360,2.198)--(11.360,2.199)--(11.360,2.201)--(11.361,2.202)%
  --(11.361,2.204)--(11.361,2.205)--(11.362,2.207)--(11.362,2.208)--(11.363,2.212)--(11.363,2.213)%
  --(11.363,2.216)--(11.364,2.215)--(11.364,2.218)--(11.365,2.220)--(11.365,2.223)--(11.365,2.225)%
  --(11.366,2.227)--(11.366,2.230)--(11.367,2.230)--(11.367,2.232)--(11.367,2.233)--(11.368,2.236)%
  --(11.368,2.237)--(11.368,2.238)--(11.369,2.241)--(11.369,2.242)--(11.370,2.243)--(11.370,2.244)%
  --(11.370,2.245)--(11.371,2.244)--(11.371,2.245)--(11.372,2.247)--(11.372,2.248)--(11.372,2.249)%
  --(11.373,2.252)--(11.373,2.253)--(11.374,2.254)--(11.374,2.255)--(11.374,2.256)--(11.375,2.258)%
  --(11.375,2.260)--(11.375,2.262)--(11.376,2.265)--(11.376,2.269)--(11.377,2.272)--(11.377,2.283)%
  --(11.377,2.284)--(11.378,2.285)--(11.378,2.286)--(11.379,2.288)--(11.379,2.290)--(11.380,2.292)%
  --(11.380,2.294)--(11.380,2.296)--(11.381,2.298)--(11.382,2.301)--(11.382,2.308)--(11.382,2.310)%
  --(11.383,2.312)--(11.383,2.313)--(11.384,2.315)--(11.384,2.316)--(11.384,2.319)--(11.385,2.321)%
  --(11.386,2.322)--(11.386,2.324)--(11.387,2.325)--(11.387,2.327)--(11.388,2.330)--(11.388,2.333)%
  --(11.389,2.332)--(11.389,2.334)--(11.389,2.335)--(11.390,2.337)--(11.390,2.338)--(11.391,2.340)%
  --(11.391,2.344)--(11.391,2.347)--(11.392,2.349)--(11.392,2.354)--(11.393,2.358)--(11.393,2.363)%
  --(11.393,2.364)--(11.394,2.366)--(11.394,2.368)--(11.394,2.370)--(11.395,2.372)--(11.395,2.373)%
  --(11.396,2.377)--(11.396,2.378)--(11.397,2.381)--(11.397,2.384)--(11.398,2.386)--(11.398,2.389)%
  --(11.398,2.391)--(11.399,2.394)--(11.399,2.397)--(11.400,2.401)--(11.400,2.403)--(11.400,2.406)%
  --(11.401,2.408)--(11.401,2.410)--(11.401,2.412)--(11.402,2.411)--(11.402,2.414)--(11.403,2.416)%
  --(11.403,2.419)--(11.403,2.422)--(11.404,2.422)--(11.404,2.424)--(11.405,2.427)--(11.405,2.429)%
  --(11.405,2.430)--(11.406,2.433)--(11.407,2.434)--(11.407,2.438)--(11.407,2.442)--(11.408,2.445)%
  --(11.408,2.447)--(11.408,2.450)--(11.409,2.453)--(11.409,2.456)--(11.410,2.458)--(11.410,2.462)%
  --(11.410,2.464)--(11.411,2.467)--(11.411,2.471)--(11.412,2.472)--(11.412,2.473)--(11.412,2.477)%
  --(11.413,2.480)--(11.413,2.483)--(11.414,2.487)--(11.414,2.491)--(11.414,2.497)--(11.415,2.498)%
  --(11.415,2.501)--(11.415,2.503)--(11.416,2.506)--(11.416,2.508)--(11.417,2.511)--(11.417,2.513)%
  --(11.418,2.517)--(11.418,2.518)--(11.419,2.521)--(11.419,2.522)--(11.419,2.524)--(11.420,2.528)%
  --(11.420,2.530)--(11.420,2.532)--(11.421,2.533)--(11.421,2.535)--(11.422,2.540)--(11.422,2.541)%
  --(11.422,2.543)--(11.423,2.547)--(11.423,2.550)--(11.424,2.553)--(11.424,2.556)--(11.424,2.559)%
  --(11.425,2.562)--(11.425,2.566)--(11.426,2.572)--(11.426,2.575)--(11.426,2.578)--(11.427,2.583)%
  --(11.427,2.589)--(11.427,2.594)--(11.428,2.596)--(11.428,2.599)--(11.429,2.603)--(11.429,2.608)%
  --(11.429,2.611)--(11.430,2.612)--(11.430,2.614)--(11.431,2.617)--(11.431,2.620)--(11.431,2.623)%
  --(11.432,2.625)--(11.432,2.627)--(11.433,2.631)--(11.433,2.634)--(11.433,2.638)--(11.434,2.643)%
  --(11.434,2.646)--(11.434,2.649)--(11.435,2.651)--(11.435,2.654)--(11.436,2.658)--(11.436,2.659)%
  --(11.436,2.663)--(11.437,2.665)--(11.437,2.667)--(11.438,2.669)--(11.438,2.674)--(11.438,2.676)%
  --(11.439,2.680)--(11.439,2.682)--(11.440,2.684)--(11.440,2.687)--(11.440,2.690)--(11.441,2.693)%
  --(11.441,2.697)--(11.441,2.700)--(11.442,2.703)--(11.442,2.705)--(11.443,2.708)--(11.443,2.714)%
  --(11.443,2.718)--(11.444,2.723)--(11.444,2.724)--(11.445,2.726)--(11.445,2.728)--(11.446,2.733)%
  --(11.446,2.736)--(11.447,2.739)--(11.447,2.743)--(11.447,2.748)--(11.448,2.751)--(11.448,2.756)%
  --(11.448,2.758)--(11.449,2.760)--(11.449,2.767)--(11.450,2.771)--(11.450,2.775)--(11.450,2.778)%
  --(11.451,2.784)--(11.451,2.787)--(11.452,2.787)--(11.452,2.790)--(11.452,2.794)--(11.453,2.797)%
  --(11.453,2.803)--(11.453,2.808)--(11.454,2.811)--(11.454,2.815)--(11.455,2.818)--(11.455,2.820)%
  --(11.456,2.821)--(11.456,2.827)--(11.457,2.831)--(11.457,2.835)--(11.457,2.837)--(11.458,2.842)%
  --(11.458,2.847)--(11.459,2.848)--(11.459,2.850)--(11.459,2.851)--(11.460,2.853)--(11.460,2.856)%
  --(11.460,2.861)--(11.461,2.864)--(11.461,2.866)--(11.462,2.869)--(11.462,2.874)--(11.462,2.878)%
  --(11.463,2.882)--(11.463,2.887)--(11.464,2.892)--(11.464,2.900)--(11.464,2.903)--(11.465,2.906)%
  --(11.465,2.912)--(11.466,2.918)--(11.466,2.921)--(11.466,2.924)--(11.467,2.930)--(11.467,2.936)%
  --(11.467,2.941)--(11.468,2.943)--(11.468,2.946)--(11.469,2.950)--(11.469,2.955)--(11.469,2.964)%
  --(11.470,2.967)--(11.470,2.970)--(11.471,2.971)--(11.471,2.974)--(11.471,2.975)--(11.472,2.977)%
  --(11.472,2.980)--(11.473,2.982)--(11.473,2.987)--(11.473,2.989)--(11.474,2.992)--(11.474,2.996)%
  --(11.474,3.002)--(11.475,3.006)--(11.475,3.011)--(11.476,3.014)--(11.476,3.019)--(11.476,3.024)%
  --(11.477,3.030)--(11.477,3.034)--(11.478,3.036)--(11.478,3.038)--(11.478,3.043)--(11.479,3.044)%
  --(11.479,3.050)--(11.480,3.053)--(11.480,3.054)--(11.480,3.058)--(11.481,3.065)--(11.481,3.068)%
  --(11.481,3.070)--(11.482,3.073)--(11.482,3.078)--(11.483,3.083)--(11.483,3.086)--(11.483,3.090)%
  --(11.484,3.095)--(11.484,3.098)--(11.485,3.101)--(11.485,3.104)--(11.485,3.109)--(11.486,3.112)%
  --(11.486,3.113)--(11.486,3.117)--(11.487,3.119)--(11.487,3.125)--(11.488,3.128)--(11.488,3.131)%
  --(11.488,3.134)--(11.489,3.136)--(11.489,3.142)--(11.490,3.144)--(11.490,3.149)--(11.490,3.153)%
  --(11.491,3.160)--(11.491,3.164)--(11.492,3.170)--(11.492,3.176)--(11.492,3.178)--(11.493,3.184)%
  --(11.493,3.188)--(11.493,3.193)--(11.494,3.199)--(11.494,3.207)--(11.495,3.212)--(11.495,3.216)%
  --(11.495,3.218)--(11.496,3.222)--(11.496,3.225)--(11.497,3.230)--(11.497,3.236)--(11.497,3.241)%
  --(11.498,3.246)--(11.498,3.250)--(11.499,3.254)--(11.499,3.262)--(11.499,3.265)--(11.500,3.269)%
  --(11.500,3.270)--(11.500,3.274)--(11.501,3.279)--(11.501,3.285)--(11.502,3.287)--(11.502,3.295)%
  --(11.502,3.302)--(11.503,3.308)--(11.503,3.313)--(11.504,3.320)--(11.504,3.327)--(11.504,3.333)%
  --(11.505,3.336)--(11.505,3.340)--(11.506,3.343)--(11.506,3.346)--(11.506,3.354)--(11.507,3.359)%
  --(11.507,3.361)--(11.507,3.366)--(11.508,3.369)--(11.508,3.372)--(11.509,3.373)--(11.509,3.377)%
  --(11.509,3.380)--(11.510,3.380)--(11.510,3.385)--(11.511,3.390)--(11.511,3.395)--(11.511,3.398)%
  --(11.512,3.403)--(11.512,3.407)--(11.513,3.414)--(11.513,3.418)--(11.513,3.424)--(11.514,3.429)%
  --(11.514,3.431)--(11.514,3.434)--(11.515,3.437)--(11.515,3.441)--(11.516,3.445)--(11.516,3.449)%
  --(11.516,3.450)--(11.517,3.457)--(11.517,3.463)--(11.518,3.463)--(11.518,3.466)--(11.518,3.467)%
  --(11.519,3.473)--(11.519,3.482)--(11.520,3.486)--(11.520,3.490)--(11.520,3.496)--(11.521,3.503)%
  --(11.521,3.508)--(11.521,3.517)--(11.522,3.524)--(11.522,3.534)--(11.523,3.543)--(11.523,3.549)%
  --(11.523,3.553)--(11.524,3.556)--(11.524,3.562)--(11.525,3.573)--(11.525,3.580)--(11.525,3.587)%
  --(11.526,3.592)--(11.526,3.598)--(11.526,3.603)--(11.527,3.607)--(11.527,3.610)--(11.528,3.613)%
  --(11.528,3.618)--(11.528,3.624)--(11.529,3.626)--(11.529,3.636)--(11.530,3.641)--(11.530,3.647)%
  --(11.530,3.653)--(11.531,3.658)--(11.531,3.663)--(11.532,3.668)--(11.532,3.677)--(11.532,3.685)%
  --(11.533,3.689)--(11.533,3.695)--(11.533,3.704)--(11.534,3.711)--(11.534,3.720)--(11.535,3.730)%
  --(11.535,3.740)--(11.535,3.749)--(11.536,3.758)--(11.536,3.766)--(11.537,3.769)--(11.537,3.776)%
  --(11.537,3.781)--(11.538,3.789)--(11.538,3.798)--(11.539,3.805)--(11.539,3.814)--(11.539,3.819)%
  --(11.540,3.827)--(11.540,3.834)--(11.540,3.838)--(11.541,3.846)--(11.541,3.856)--(11.542,3.867)%
  --(11.542,3.876)--(11.542,3.887)--(11.543,3.896)--(11.543,3.908)--(11.544,3.913)--(11.544,3.916)%
  --(11.544,3.922)--(11.545,3.927)--(11.545,3.937)--(11.546,3.947)--(11.546,3.959)--(11.546,3.969)%
  --(11.547,3.976)--(11.547,3.986)--(11.547,3.991)--(11.548,4.000)--(11.548,4.011)--(11.549,4.019)%
  --(11.549,4.027)--(11.549,4.031)--(11.550,4.033)--(11.550,4.042)--(11.551,4.051)--(11.551,4.054)%
  --(11.551,4.057)--(11.552,4.065)--(11.552,4.064)--(11.553,4.067)--(11.553,4.072)--(11.553,4.081)%
  --(11.554,4.084)--(11.554,4.088)--(11.554,4.094)--(11.555,4.106)--(11.555,4.113)--(11.556,4.121)%
  --(11.556,4.128)--(11.556,4.135)--(11.557,4.148)--(11.557,4.156)--(11.558,4.162)--(11.558,4.174)%
  --(11.558,4.180)--(11.559,4.185)--(11.559,4.190)--(11.559,4.192)--(11.560,4.198)--(11.560,4.203)%
  --(11.561,4.207)--(11.561,4.211)--(11.561,4.215)--(11.562,4.224)--(11.562,4.233)--(11.563,4.245)%
  --(11.563,4.252)--(11.563,4.254)--(11.564,4.259)--(11.564,4.262)--(11.565,4.266)--(11.565,4.273)%
  --(11.565,4.280)--(11.566,4.291)--(11.566,4.296)--(11.566,4.299)--(11.567,4.309)--(11.567,4.314)%
  --(11.568,4.320)--(11.568,4.324)--(11.568,4.333)--(11.569,4.339)--(11.569,4.350)--(11.570,4.352)%
  --(11.570,4.367)--(11.570,4.381)--(11.571,4.387)--(11.571,4.394)--(11.572,4.399)--(11.572,4.408)%
  --(11.572,4.414)--(11.573,4.421)--(11.573,4.423)--(11.573,4.431)--(11.574,4.443)--(11.574,4.452)%
  --(11.575,4.455)--(11.575,4.464)--(11.575,4.477)--(11.576,4.488)--(11.576,4.495)--(11.577,4.505)%
  --(11.577,4.513)--(11.577,4.524)--(11.578,4.527)--(11.578,4.533)--(11.579,4.541)--(11.579,4.554)%
  --(11.579,4.563)--(11.580,4.564)--(11.580,4.568)--(11.580,4.580)--(11.581,4.588)--(11.581,4.599)%
  --(11.582,4.604)--(11.582,4.614)--(11.582,4.624)--(11.583,4.634)--(11.583,4.641)--(11.584,4.645)%
  --(11.584,4.656)--(11.584,4.670)--(11.585,4.684)--(11.585,4.692)--(11.586,4.703)--(11.586,4.712)%
  --(11.586,4.715)--(11.587,4.722)--(11.587,4.734)--(11.587,4.738)--(11.588,4.743)--(11.588,4.752)%
  --(11.589,4.772)--(11.589,4.784)--(11.589,4.792)--(11.590,4.799)--(11.590,4.810)--(11.591,4.820)%
  --(11.591,4.825)--(11.591,4.833)--(11.592,4.837)--(11.592,4.843)--(11.593,4.864)--(11.593,4.880)%
  --(11.593,4.886)--(11.594,4.896)--(11.594,4.904)--(11.594,4.914)--(11.595,4.925)--(11.595,4.938)%
  --(11.596,4.942)--(11.596,4.949)--(11.596,4.951)--(11.597,4.958)--(11.597,4.969)--(11.598,4.982)%
  --(11.598,4.985)--(11.598,5.000)--(11.599,5.015)--(11.599,5.014)--(11.599,5.024)--(11.600,5.032)%
  --(11.600,5.043)--(11.601,5.052)--(11.601,5.061)--(11.601,5.070)--(11.602,5.078)--(11.602,5.085)%
  --(11.603,5.094)--(11.603,5.099)--(11.603,5.108)--(11.604,5.117)--(11.604,5.132)--(11.605,5.140)%
  --(11.605,5.152)--(11.605,5.158)--(11.606,5.165)--(11.606,5.179)--(11.606,5.195)--(11.607,5.209)%
  --(11.607,5.218)--(11.608,5.232)--(11.608,5.246)--(11.608,5.260)--(11.609,5.279)--(11.609,5.287)%
  --(11.610,5.300)--(11.610,5.315)--(11.610,5.320)--(11.611,5.329)--(11.611,5.334)--(11.612,5.338)%
  --(11.612,5.346)--(11.612,5.350)--(11.613,5.358)--(11.613,5.368)--(11.613,5.382)--(11.614,5.392)%
  --(11.614,5.394)--(11.615,5.400)--(11.615,5.397)--(11.615,5.401)--(11.616,5.406)--(11.616,5.416)%
  --(11.617,5.428)--(11.617,5.447)--(11.617,5.461)--(11.618,5.467)--(11.618,5.477)--(11.619,5.486)%
  --(11.619,5.497)--(11.619,5.506)--(11.620,5.510)--(11.620,5.525)--(11.620,5.541)--(11.621,5.558)%
  --(11.621,5.580)--(11.622,5.600)--(11.622,5.603)--(11.622,5.606)--(11.623,5.618)--(11.623,5.631)%
  --(11.624,5.642)--(11.624,5.649)--(11.624,5.652)--(11.625,5.655)--(11.625,5.656)--(11.626,5.656)%
  --(11.626,5.659)--(11.626,5.666)--(11.627,5.679)--(11.627,5.692)--(11.627,5.701)--(11.628,5.703)%
  --(11.628,5.709)--(11.629,5.729)--(11.629,5.760)--(11.629,5.767)--(11.630,5.780)--(11.630,5.796)%
  --(11.631,5.828)--(11.631,5.840)--(11.631,5.841)--(11.632,5.843)--(11.632,5.857)--(11.632,5.873)%
  --(11.633,5.881)--(11.633,5.893)--(11.634,5.901)--(11.634,5.919)--(11.634,5.931)--(11.635,5.942)%
  --(11.635,5.955);
\gpcolor{color=gp lt color border}
\gpsetlinetype{gp lt border}
\draw[gp path] (2.240,7.508)--(2.240,1.449);
\draw[gp path] (2.240,0.985)--(11.635,0.985);
%% coordinates of the plot area
\gpdefrectangularnode{gp plot 1}{\pgfpoint{2.240cm}{0.985cm}}{\pgfpoint{11.947cm}{7.825cm}}
\end{tikzpicture}
%% gnuplot variables

\caption{Le moli calcolate dai dati, in ordine temporale}
\label{img:moli}
\end{grafico}

\begin{grafico}
  \centering
\begin{tikzpicture}[gnuplot]
%% generated with GNUPLOT 4.6p3 (Lua 5.1; terminal rev. 99, script rev. 100)
%% mar 27 mag 2014 22:34:39 CEST
\path (0.000,0.000) rectangle (12.500,8.750);
\gpcolor{color=gp lt color border}
\gpsetlinetype{gp lt border}
\gpsetlinewidth{1.00}
\draw[gp path] (1.320,1.745)--(1.500,1.745);
\node[gp node right] at (1.136,1.745) { 6};
\draw[gp path] (1.320,2.505)--(1.500,2.505);
\node[gp node right] at (1.136,2.505) { 8};
\draw[gp path] (1.320,3.265)--(1.500,3.265);
\node[gp node right] at (1.136,3.265) { 10};
\draw[gp path] (1.320,4.025)--(1.500,4.025);
\node[gp node right] at (1.136,4.025) { 12};
\draw[gp path] (1.320,4.785)--(1.500,4.785);
\node[gp node right] at (1.136,4.785) { 14};
\draw[gp path] (1.320,5.545)--(1.500,5.545);
\node[gp node right] at (1.136,5.545) { 16};
\draw[gp path] (1.320,6.305)--(1.500,6.305);
\node[gp node right] at (1.136,6.305) { 18};
\draw[gp path] (1.320,7.065)--(1.500,7.065);
\node[gp node right] at (1.136,7.065) { 20};
\draw[gp path] (2.648,0.985)--(2.648,1.165);
\node[gp node center] at (2.648,0.677) { 0.6};
\draw[gp path] (3.977,0.985)--(3.977,1.165);
\node[gp node center] at (3.977,0.677) { 0.8};
\draw[gp path] (5.305,0.985)--(5.305,1.165);
\node[gp node center] at (5.305,0.677) { 1};
\draw[gp path] (6.634,0.985)--(6.634,1.165);
\node[gp node center] at (6.634,0.677) { 1.2};
\draw[gp path] (7.962,0.985)--(7.962,1.165);
\node[gp node center] at (7.962,0.677) { 1.4};
\draw[gp path] (9.290,0.985)--(9.290,1.165);
\node[gp node center] at (9.290,0.677) { 1.6};
\draw[gp path] (10.619,0.985)--(10.619,1.165);
\node[gp node center] at (10.619,0.677) { 1.8};
\draw[gp path] (1.320,7.524)--(1.320,1.705);
\draw[gp path] (2.097,0.985)--(10.898,0.985);
\node[gp node center,rotate=-270] at (0.246,4.405) {Volume $[cm^3]$};
\node[gp node center] at (6.633,0.215) {Inverso della pressione $[\frac{cm^2}{Kgf}]$};
\node[gp node center] at (6.633,8.287) {Prima giornata};
\gpcolor{color=gp lt color 0}
\gpsetlinetype{gp lt plot 0}
\draw[gp path] (10.898,7.524)--(10.898,7.523)--(10.891,7.523)--(10.884,7.523)--(10.884,7.520)%
  --(10.884,7.516)--(10.878,7.520)--(10.878,7.519)--(10.864,7.518)--(10.858,7.516)--(10.858,7.515)%
  --(10.844,7.513)--(10.838,7.512)--(10.838,7.509)--(10.831,7.508)--(10.825,7.507)--(10.818,7.500)%
  --(10.811,7.504)--(10.805,7.503)--(10.798,7.501)--(10.791,7.499)--(10.791,7.498)--(10.785,7.495)%
  --(10.778,7.488)--(10.778,7.489)--(10.765,7.488)--(10.765,7.485)--(10.758,7.492)--(10.738,7.491)%
  --(10.745,7.488)--(10.738,7.486)--(10.732,7.481)--(10.718,7.481)--(10.725,7.478)--(10.725,7.474)%
  --(10.718,7.469)--(10.718,7.463)--(10.712,7.466)--(10.705,7.460)--(10.698,7.455)--(10.712,7.453)%
  --(10.698,7.448)--(10.685,7.439)--(10.692,7.437)--(10.678,7.437)--(10.685,7.434)--(10.685,7.437)%
  --(10.678,7.435)--(10.678,7.431)--(10.672,7.429)--(10.665,7.421)--(10.665,7.417)--(10.658,7.415)%
  --(10.652,7.410)--(10.652,7.409)--(10.645,7.410)--(10.639,7.407)--(10.645,7.405)--(10.645,7.404)%
  --(10.645,7.402)--(10.632,7.402)--(10.632,7.399)--(10.625,7.397)--(10.619,7.387)--(10.625,7.390)%
  --(10.619,7.388)--(10.612,7.385)--(10.612,7.383)--(10.612,7.381)--(10.612,7.375)--(10.605,7.369)%
  --(10.605,7.368)--(10.599,7.365)--(10.585,7.362)--(10.579,7.363)--(10.572,7.358)--(10.572,7.354)%
  --(10.565,7.351)--(10.565,7.346)--(10.552,7.343)--(10.546,7.338)--(10.532,7.333)--(10.526,7.324)%
  --(10.526,7.327)--(10.526,7.326)--(10.519,7.324)--(10.512,7.323)--(10.512,7.320)--(10.506,7.313)%
  --(10.499,7.311)--(10.499,7.309)--(10.499,7.310)--(10.486,7.309)--(10.479,7.306)--(10.473,7.303)%
  --(10.473,7.298)--(10.473,7.295)--(10.466,7.293)--(10.459,7.289)--(10.453,7.279)--(10.453,7.275)%
  --(10.433,7.274)--(10.426,7.272)--(10.426,7.267)--(10.419,7.263)--(10.419,7.257)--(10.413,7.249)%
  --(10.406,7.238)--(10.399,7.235)--(10.393,7.227)--(10.386,7.228)--(10.386,7.226)--(10.373,7.224)%
  --(10.366,7.220)--(10.366,7.211)--(10.360,7.200)--(10.340,7.190)--(10.333,7.184)--(10.326,7.183)%
  --(10.320,7.185)--(10.313,7.182)--(10.313,7.180)--(10.320,7.179)--(10.313,7.175)--(10.306,7.171)%
  --(10.300,7.168)--(10.287,7.164)--(10.280,7.152)--(10.267,7.150)--(10.260,7.148)--(10.253,7.142)%
  --(10.240,7.137)--(10.227,7.132)--(10.213,7.117)--(10.200,7.109)--(10.187,7.101)--(10.187,7.097)%
  --(10.187,7.094)--(10.174,7.096)--(10.174,7.094)--(10.167,7.087)--(10.167,7.082)--(10.160,7.078)%
  --(10.160,7.074)--(10.160,7.071)--(10.154,7.060)--(10.154,7.054)--(10.134,7.058)--(10.127,7.051)%
  --(10.127,7.046)--(10.120,7.042)--(10.114,7.039)--(10.107,7.033)--(10.094,7.028)--(10.087,7.021)%
  --(10.074,7.010)--(10.074,7.006)--(10.067,7.003)--(10.054,6.998)--(10.054,6.993)--(10.047,6.990)%
  --(10.034,6.981)--(10.027,6.975)--(10.014,6.970)--(10.014,6.966)--(10.008,6.961)--(10.001,6.965)%
  --(9.994,6.963)--(9.994,6.961)--(9.988,6.959)--(9.988,6.957)--(9.981,6.956)--(9.974,6.953)%
  --(9.974,6.952)--(9.974,6.948)--(9.968,6.946)--(9.954,6.940)--(9.941,6.935)--(9.935,6.933)%
  --(9.928,6.928)--(9.921,6.919)--(9.921,6.914)--(9.921,6.904)--(9.915,6.897)--(9.921,6.893)%
  --(9.908,6.900)--(9.901,6.899)--(9.908,6.895)--(9.901,6.892)--(9.895,6.889)--(9.888,6.878)%
  --(9.881,6.873)--(9.881,6.870)--(9.875,6.865)--(9.868,6.866)--(9.868,6.864)--(9.868,6.862)%
  --(9.855,6.859)--(9.842,6.853)--(9.835,6.850)--(9.828,6.842)--(9.815,6.837)--(9.815,6.831)%
  --(9.808,6.819)--(9.795,6.818)--(9.788,6.814)--(9.775,6.810)--(9.768,6.808)--(9.762,6.804)%
  --(9.755,6.794)--(9.755,6.791)--(9.755,6.790)--(9.749,6.789)--(9.742,6.788)--(9.735,6.789)%
  --(9.729,6.785)--(9.722,6.783)--(9.709,6.780)--(9.709,6.779)--(9.715,6.777)--(9.702,6.773)%
  --(9.695,6.771)--(9.695,6.758)--(9.689,6.757)--(9.662,6.752)--(9.669,6.747)--(9.669,6.739)%
  --(9.662,6.734)--(9.656,6.720)--(9.649,6.715)--(9.649,6.710)--(9.642,6.705)--(9.636,6.702)%
  --(9.629,6.697)--(9.616,6.699)--(9.616,6.696)--(9.609,6.692)--(9.609,6.691)--(9.609,6.687)%
  --(9.602,6.682)--(9.589,6.669)--(9.582,6.663)--(9.576,6.670)--(9.582,6.668)--(9.569,6.664)%
  --(9.563,6.659)--(9.543,6.649)--(9.523,6.637)--(9.516,6.624)--(9.503,6.621)--(9.496,6.618)%
  --(9.490,6.613)--(9.483,6.611)--(9.483,6.604)--(9.483,6.601)--(9.476,6.599)--(9.470,6.596)%
  --(9.470,6.601)--(9.463,6.599)--(9.443,6.594)--(9.436,6.590)--(9.436,6.585)--(9.430,6.581)%
  --(9.423,6.578)--(9.423,6.576)--(9.430,6.567)--(9.423,6.574)--(9.410,6.571)--(9.403,6.568)%
  --(9.403,6.563)--(9.403,6.560)--(9.397,6.547)--(9.397,6.548)--(9.390,6.544)--(9.390,6.541)%
  --(9.383,6.538)--(9.383,6.535)--(9.377,6.532)--(9.370,6.527)--(9.363,6.524)--(9.357,6.517)%
  --(9.357,6.515)--(9.350,6.511)--(9.350,6.500)--(9.337,6.502)--(9.330,6.498)--(9.317,6.495)%
  --(9.323,6.490)--(9.317,6.488)--(9.310,6.484)--(9.304,6.478)--(9.297,6.469)--(9.284,6.457)%
  --(9.270,6.456)--(9.270,6.450)--(9.257,6.447)--(9.257,6.443)--(9.250,6.441)--(9.244,6.433)%
  --(9.237,6.431)--(9.230,6.429)--(9.224,6.428)--(9.217,6.426)--(9.211,6.422)--(9.204,6.418)%
  --(9.197,6.414)--(9.191,6.409)--(9.171,6.405)--(9.157,6.400)--(9.157,6.395)--(9.151,6.392)%
  --(9.151,6.387)--(9.151,6.382)--(9.144,6.381)--(9.137,6.378)--(9.137,6.374)--(9.137,6.371)%
  --(9.124,6.363)--(9.118,6.355)--(9.118,6.354)--(9.111,6.351)--(9.104,6.347)--(9.098,6.340)%
  --(9.098,6.335)--(9.091,6.332)--(9.084,6.329)--(9.071,6.326)--(9.071,6.315)--(9.071,6.319)%
  --(9.071,6.315)--(9.064,6.312)--(9.064,6.308)--(9.058,6.304)--(9.058,6.303)--(9.051,6.301)%
  --(9.045,6.298)--(9.031,6.295)--(9.031,6.291)--(9.025,6.289)--(9.025,6.284)--(9.018,6.272)%
  --(9.011,6.275)--(9.011,6.270)--(8.998,6.266)--(8.991,6.262)--(8.991,6.258)--(8.978,6.250)%
  --(8.978,6.245)--(8.965,6.242)--(8.958,6.244)--(8.952,6.242)--(8.945,6.237)--(8.932,6.232)%
  --(8.918,6.225)--(8.905,6.219)--(8.905,6.214)--(8.892,6.210)--(8.878,6.205)--(8.878,6.200)%
  --(8.872,6.189)--(8.865,6.186)--(8.872,6.183)--(8.865,6.177)--(8.859,6.172)--(8.845,6.167)%
  --(8.839,6.161)--(8.839,6.159)--(8.839,6.155)--(8.832,6.152)--(8.832,6.147)--(8.825,6.147)%
  --(8.819,6.142)--(8.805,6.139)--(8.799,6.136)--(8.792,6.130)--(8.785,6.121)--(8.785,6.115)%
  --(8.779,6.108)--(8.766,6.104)--(8.766,6.105)--(8.759,6.103)--(8.752,6.098)--(8.752,6.093)%
  --(8.732,6.089)--(8.732,6.087)--(8.726,6.084)--(8.719,6.079)--(8.712,6.069)--(8.699,6.062)%
  --(8.692,6.065)--(8.686,6.064)--(8.686,6.061)--(8.686,6.058)--(8.679,6.054)--(8.679,6.050)%
  --(8.679,6.047)--(8.679,6.046)--(8.673,6.044)--(8.659,6.047)--(8.653,6.043)--(8.653,6.039)%
  --(8.646,6.034)--(8.639,6.028)--(8.639,6.025)--(8.639,6.020)--(8.626,6.014)--(8.619,6.009)%
  --(8.613,6.006)--(8.613,6.004)--(8.606,6.001)--(8.599,5.997)--(8.593,5.993)--(8.586,5.988)%
  --(8.586,5.982)--(8.580,5.980)--(8.586,5.976)--(8.580,5.974)--(8.573,5.976)--(8.573,5.972)%
  --(8.560,5.969)--(8.546,5.965)--(8.546,5.959)--(8.540,5.955)--(8.533,5.949)--(8.520,5.936)%
  --(8.513,5.931)--(8.513,5.930)--(8.507,5.930)--(8.493,5.924)--(8.480,5.922)--(8.487,5.919)%
  --(8.473,5.915)--(8.473,5.912)--(8.467,5.906)--(8.460,5.901)--(8.453,5.893)--(8.447,5.894)%
  --(8.440,5.891)--(8.440,5.889)--(8.433,5.887)--(8.433,5.885)--(8.427,5.879)--(8.427,5.876)%
  --(8.420,5.872)--(8.420,5.869)--(8.414,5.868)--(8.407,5.868)--(8.394,5.865)--(8.394,5.862)%
  --(8.394,5.858)--(8.387,5.857)--(8.387,5.854)--(8.380,5.851)--(8.367,5.845)--(8.367,5.835)%
  --(8.360,5.836)--(8.360,5.833)--(8.360,5.832)--(8.354,5.830)--(8.354,5.826)--(8.347,5.819)%
  --(8.340,5.815)--(8.327,5.813)--(8.327,5.811)--(8.327,5.810)--(8.321,5.806)--(8.314,5.803)%
  --(8.301,5.797)--(8.294,5.792)--(8.294,5.790)--(8.287,5.788)--(8.287,5.786)--(8.287,5.778)%
  --(8.281,5.774)--(8.267,5.774)--(8.267,5.771)--(8.261,5.771)--(8.261,5.772)--(8.261,5.768)%
  --(8.254,5.764)--(8.254,5.760)--(8.234,5.754)--(8.234,5.748)--(8.228,5.743)--(8.214,5.741)%
  --(8.201,5.737)--(8.194,5.732)--(8.188,5.726)--(8.188,5.719)--(8.181,5.711)--(8.174,5.708)%
  --(8.168,5.705)--(8.161,5.704)--(8.154,5.705)--(8.148,5.699)--(8.135,5.695)--(8.128,5.690)%
  --(8.121,5.686)--(8.121,5.684)--(8.121,5.681)--(8.115,5.676)--(8.108,5.670)--(8.101,5.668)%
  --(8.101,5.669)--(8.095,5.667)--(8.095,5.665)--(8.095,5.661)--(8.088,5.653)--(8.081,5.647)%
  --(8.075,5.645)--(8.075,5.642)--(8.068,5.639)--(8.062,5.641)--(8.062,5.639)--(8.062,5.637)%
  --(8.055,5.634)--(8.048,5.629)--(8.042,5.624)--(8.042,5.621)--(8.035,5.616)--(8.028,5.610)%
  --(8.022,5.605)--(8.002,5.601)--(7.995,5.597)--(7.995,5.594)--(7.988,5.591)--(7.988,5.588)%
  --(7.975,5.586)--(7.969,5.579)--(7.962,5.569)--(7.955,5.561)--(7.942,5.559)--(7.935,5.556)%
  --(7.929,5.551)--(7.929,5.545)--(7.915,5.537)--(7.909,5.532)--(7.902,5.530)--(7.902,5.529)%
  --(7.895,5.532)--(7.889,5.529)--(7.882,5.526)--(7.876,5.524)--(7.869,5.519)--(7.862,5.514)%
  --(7.856,5.509)--(7.849,5.502)--(7.842,5.498)--(7.842,5.492)--(7.836,5.493)--(7.836,5.491)%
  --(7.836,5.488)--(7.829,5.487)--(7.829,5.484)--(7.829,5.477)--(7.816,5.471)--(7.809,5.467)%
  --(7.802,5.463)--(7.802,5.464)--(7.802,5.463)--(7.802,5.461)--(7.796,5.458)--(7.789,5.456)%
  --(7.783,5.447)--(7.776,5.440)--(7.769,5.441)--(7.749,5.434)--(7.736,5.426)--(7.723,5.421)%
  --(7.723,5.416)--(7.716,5.412)--(7.716,5.406)--(7.709,5.401)--(7.696,5.398)--(7.696,5.394)%
  --(7.690,5.389)--(7.683,5.379)--(7.676,5.381)--(7.670,5.377)--(7.663,5.373)--(7.656,5.369)%
  --(7.656,5.365)--(7.650,5.357)--(7.636,5.359)--(7.636,5.358)--(7.630,5.357)--(7.630,5.355)%
  --(7.623,5.350)--(7.616,5.342)--(7.610,5.338)--(7.603,5.336)--(7.597,5.333)--(7.590,5.330)%
  --(7.583,5.325)--(7.577,5.321)--(7.577,5.312)--(7.570,5.316)--(7.570,5.312)--(7.563,5.309)%
  --(7.557,5.305)--(7.557,5.303)--(7.550,5.294)--(7.550,5.296)--(7.543,5.294)--(7.537,5.289)%
  --(7.537,5.287)--(7.530,5.282)--(7.530,5.280)--(7.530,5.277)--(7.517,5.275)--(7.517,5.273)%
  --(7.510,5.269)--(7.504,5.263)--(7.490,5.252)--(7.484,5.255)--(7.477,5.249)--(7.470,5.244)%
  --(7.464,5.238)--(7.457,5.231)--(7.444,5.225)--(7.437,5.221)--(7.431,5.216)--(7.431,5.212)%
  --(7.424,5.201)--(7.417,5.202)--(7.404,5.198)--(7.397,5.194)--(7.391,5.193)--(7.384,5.187)%
  --(7.377,5.182)--(7.371,5.176)--(7.371,5.177)--(7.364,5.175)--(7.357,5.171)--(7.344,5.165)%
  --(7.338,5.160)--(7.331,5.154)--(7.331,5.151)--(7.331,5.146)--(7.324,5.141)--(7.311,5.138)%
  --(7.311,5.134)--(7.311,5.129)--(7.304,5.130)--(7.304,5.129)--(7.291,5.124)--(7.291,5.120)%
  --(7.284,5.116)--(7.284,5.109)--(7.278,5.110)--(7.271,5.106)--(7.264,5.101)--(7.258,5.095)%
  --(7.245,5.087)--(7.231,5.080)--(7.225,5.073)--(7.211,5.068)--(7.211,5.063)--(7.205,5.058)%
  --(7.198,5.056)--(7.185,5.050)--(7.185,5.048)--(7.178,5.044)--(7.171,5.040)--(7.171,5.036)%
  --(7.158,5.032)--(7.158,5.028)--(7.152,5.024)--(7.145,5.018)--(7.138,5.014)--(7.125,5.010)%
  --(7.125,5.005)--(7.118,5.004)--(7.105,4.999)--(7.105,4.995)--(7.098,4.993)--(7.092,4.988)%
  --(7.079,4.978)--(7.079,4.974)--(7.072,4.971)--(7.072,4.967)--(7.065,4.965)--(7.059,4.962)%
  --(7.052,4.957)--(7.052,4.953)--(7.045,4.948)--(7.039,4.946)--(7.039,4.943)--(7.025,4.941)%
  --(7.025,4.936)--(7.019,4.932)--(7.012,4.929)--(7.005,4.924)--(6.992,4.917)--(6.979,4.909)%
  --(6.972,4.902)--(6.966,4.891)--(6.946,4.881)--(6.946,4.875)--(6.926,4.868)--(6.919,4.865)%
  --(6.906,4.860)--(6.899,4.855)--(6.893,4.850)--(6.893,4.846)--(6.893,4.840)--(6.886,4.837)%
  --(6.879,4.834)--(6.873,4.829)--(6.866,4.828)--(6.859,4.826)--(6.853,4.823)--(6.846,4.819)%
  --(6.839,4.814)--(6.839,4.809)--(6.826,4.804)--(6.826,4.801)--(6.819,4.798)--(6.819,4.791)%
  --(6.813,4.790)--(6.806,4.787)--(6.800,4.782)--(6.800,4.777)--(6.786,4.774)--(6.786,4.767)%
  --(6.780,4.764)--(6.780,4.760)--(6.773,4.756)--(6.766,4.753)--(6.753,4.752)--(6.746,4.746)%
  --(6.733,4.738)--(6.720,4.728)--(6.707,4.722)--(6.707,4.720)--(6.707,4.717)--(6.700,4.709)%
  --(6.700,4.710)--(6.687,4.703)--(6.680,4.697)--(6.673,4.692)--(6.667,4.690)--(6.667,4.686)%
  --(6.660,4.682)--(6.653,4.679)--(6.653,4.678)--(6.647,4.676)--(6.640,4.673)--(6.634,4.668)%
  --(6.634,4.662)--(6.627,4.658)--(6.614,4.656)--(6.607,4.652)--(6.600,4.644)--(6.587,4.631)%
  --(6.580,4.624)--(6.567,4.625)--(6.560,4.621)--(6.560,4.617)--(6.554,4.612)--(6.547,4.605)%
  --(6.541,4.603)--(6.541,4.601)--(6.541,4.600)--(6.534,4.596)--(6.534,4.593)--(6.527,4.591)%
  --(6.527,4.587)--(6.521,4.586)--(6.514,4.582)--(6.514,4.579)--(6.514,4.576)--(6.507,4.573)%
  --(6.507,4.572)--(6.501,4.572)--(6.494,4.568)--(6.487,4.565)--(6.481,4.561)--(6.467,4.555)%
  --(6.454,4.547)--(6.441,4.540)--(6.441,4.534)--(6.428,4.526)--(6.428,4.523)--(6.428,4.525)%
  --(6.428,4.524)--(6.428,4.522)--(6.421,4.519)--(6.408,4.510)--(6.401,4.503)--(6.388,4.500)%
  --(6.381,4.494)--(6.374,4.489)--(6.368,4.485)--(6.361,4.481)--(6.355,4.478)--(6.341,4.470)%
  --(6.335,4.465)--(6.321,4.458)--(6.315,4.453)--(6.315,4.447)--(6.301,4.443)--(6.295,4.440)%
  --(6.295,4.438)--(6.288,4.434)--(6.275,4.429)--(6.275,4.424)--(6.268,4.421)--(6.268,4.419)%
  --(6.262,4.416)--(6.268,4.415)--(6.268,4.412)--(6.268,4.415)--(6.262,4.415)--(6.268,4.415)%
  --(6.268,4.410)--(6.268,4.411)--(6.262,4.410)--(6.255,4.405)--(6.248,4.403)--(6.242,4.396)%
  --(6.235,4.396)--(6.228,4.393)--(6.222,4.388)--(6.215,4.381)--(6.208,4.377)--(6.202,4.374)%
  --(6.195,4.370)--(6.188,4.367)--(6.188,4.364)--(6.175,4.356)--(6.175,4.354)--(6.169,4.350)%
  --(6.162,4.348)--(6.155,4.341)--(6.149,4.336)--(6.142,4.331)--(6.129,4.326)--(6.129,4.325)%
  --(6.122,4.320)--(6.115,4.316)--(6.109,4.311)--(6.109,4.312)--(6.102,4.306)--(6.096,4.299)%
  --(6.089,4.294)--(6.082,4.289)--(6.076,4.285)--(6.069,4.282)--(6.062,4.279)--(6.056,4.275)%
  --(6.049,4.270)--(6.049,4.266)--(6.042,4.262)--(6.036,4.260)--(6.036,4.257)--(6.029,4.251)%
  --(6.022,4.247)--(6.016,4.243)--(6.009,4.242)--(6.009,4.238)--(6.003,4.235)--(5.996,4.229)%
  --(5.996,4.227)--(5.983,4.224)--(5.976,4.217)--(5.963,4.210)--(5.956,4.207)--(5.956,4.206)%
  --(5.949,4.202)--(5.943,4.196)--(5.936,4.191)--(5.929,4.186)--(5.923,4.181)--(5.916,4.174)%
  --(5.910,4.172)--(5.910,4.165)--(5.896,4.162)--(5.890,4.159)--(5.890,4.155)--(5.883,4.149)%
  --(5.876,4.145)--(5.870,4.139)--(5.856,4.133)--(5.850,4.128)--(5.850,4.124)--(5.836,4.118)%
  --(5.836,4.117)--(5.836,4.114)--(5.836,4.111)--(5.830,4.109)--(5.830,4.108)--(5.817,4.104)%
  --(5.810,4.102)--(5.797,4.096)--(5.797,4.093)--(5.797,4.090)--(5.797,4.086)--(5.790,4.084)%
  --(5.790,4.080)--(5.777,4.076)--(5.777,4.071)--(5.770,4.067)--(5.770,4.068)--(5.757,4.064)%
  --(5.757,4.060)--(5.750,4.054)--(5.743,4.052)--(5.737,4.046)--(5.737,4.048)--(5.737,4.045)%
  --(5.730,4.043)--(5.730,4.039)--(5.724,4.036)--(5.717,4.034)--(5.704,4.030)--(5.704,4.026)%
  --(5.697,4.021)--(5.690,4.016)--(5.684,4.011)--(5.670,4.004)--(5.664,4.001)--(5.657,3.997)%
  --(5.651,3.992)--(5.644,3.989)--(5.644,3.987)--(5.637,3.983)--(5.631,3.981)--(5.631,3.976)%
  --(5.624,3.971)--(5.617,3.970)--(5.611,3.966)--(5.604,3.960)--(5.591,3.950)--(5.577,3.945)%
  --(5.571,3.941)--(5.571,3.937)--(5.558,3.928)--(5.551,3.925)--(5.544,3.922)--(5.544,3.917)%
  --(5.538,3.914)--(5.531,3.909)--(5.524,3.907)--(5.524,3.904)--(5.518,3.902)--(5.511,3.897)%
  --(5.511,3.893)--(5.504,3.884)--(5.498,3.882)--(5.491,3.881)--(5.491,3.878)--(5.484,3.876)%
  --(5.484,3.872)--(5.478,3.868)--(5.478,3.869)--(5.471,3.868)--(5.465,3.864)--(5.458,3.859)%
  --(5.451,3.857)--(5.445,3.856)--(5.438,3.848)--(5.425,3.838)--(5.425,3.833)--(5.418,3.832)%
  --(5.411,3.830)--(5.411,3.829)--(5.405,3.827)--(5.405,3.823)--(5.398,3.820)--(5.391,3.815)%
  --(5.391,3.812)--(5.385,3.810)--(5.385,3.803)--(5.378,3.806)--(5.372,3.802)--(5.365,3.800)%
  --(5.365,3.797)--(5.358,3.790)--(5.345,3.784)--(5.338,3.778)--(5.325,3.772)--(5.325,3.766)%
  --(5.318,3.763)--(5.318,3.761)--(5.312,3.756)--(5.312,3.753)--(5.305,3.751)--(5.298,3.748)%
  --(5.292,3.742)--(5.285,3.737)--(5.272,3.732)--(5.272,3.728)--(5.259,3.725)--(5.259,3.720)%
  --(5.252,3.715)--(5.252,3.714)--(5.245,3.712)--(5.245,3.709)--(5.239,3.705)--(5.232,3.702)%
  --(5.225,3.699)--(5.219,3.694)--(5.212,3.690)--(5.199,3.681)--(5.192,3.677)--(5.186,3.675)%
  --(5.186,3.672)--(5.179,3.669)--(5.166,3.665)--(5.159,3.657)--(5.152,3.654)--(5.146,3.650)%
  --(5.146,3.645)--(5.139,3.640)--(5.132,3.636)--(5.132,3.635)--(5.126,3.632)--(5.119,3.630)%
  --(5.119,3.626)--(5.113,3.622)--(5.113,3.619)--(5.106,3.617)--(5.099,3.612)--(5.093,3.610)%
  --(5.086,3.603)--(5.079,3.596)--(5.066,3.589)--(5.059,3.585)--(5.059,3.581)--(5.053,3.574)%
  --(5.046,3.568)--(5.039,3.563)--(5.026,3.560)--(5.020,3.556)--(5.020,3.553)--(5.013,3.551)%
  --(5.013,3.547)--(5.006,3.543)--(5.006,3.541)--(5.000,3.538)--(4.993,3.534)--(4.986,3.528)%
  --(4.980,3.528)--(4.980,3.524)--(4.973,3.520)--(4.966,3.515)--(4.960,3.512)--(4.953,3.507)%
  --(4.946,3.502)--(4.940,3.500)--(4.940,3.496)--(4.927,3.493)--(4.927,3.489)--(4.913,3.487)%
  --(4.913,3.483)--(4.907,3.478)--(4.900,3.476)--(4.893,3.472)--(4.893,3.468)--(4.887,3.464)%
  --(4.887,3.461)--(4.880,3.461)--(4.880,3.457)--(4.873,3.453)--(4.867,3.450)--(4.867,3.448)%
  --(4.860,3.445)--(4.853,3.442)--(4.853,3.440)--(4.847,3.435)--(4.840,3.431)--(4.840,3.428)%
  --(4.834,3.425)--(4.827,3.422)--(4.827,3.418)--(4.820,3.417)--(4.814,3.412)--(4.807,3.405)%
  --(4.800,3.401)--(4.794,3.399)--(4.794,3.395)--(4.787,3.392)--(4.780,3.388)--(4.774,3.383)%
  --(4.774,3.380)--(4.767,3.376)--(4.760,3.375)--(4.754,3.371)--(4.754,3.368)--(4.754,3.367)%
  --(4.747,3.363)--(4.747,3.360)--(4.741,3.358)--(4.734,3.357)--(4.734,3.352)--(4.727,3.347)%
  --(4.721,3.344)--(4.721,3.343)--(4.714,3.343)--(4.707,3.340)--(4.701,3.335)--(4.701,3.331)%
  --(4.687,3.328)--(4.681,3.324)--(4.681,3.322)--(4.681,3.321)--(4.674,3.317)--(4.668,3.314)%
  --(4.654,3.307)--(4.648,3.301)--(4.641,3.296)--(4.634,3.292)--(4.621,3.284)--(4.614,3.279)%
  --(4.608,3.275)--(4.601,3.270)--(4.594,3.269)--(4.594,3.266)--(4.588,3.263)--(4.588,3.260)%
  --(4.581,3.257)--(4.575,3.253)--(4.575,3.250)--(4.568,3.245)--(4.561,3.238)--(4.555,3.233)%
  --(4.548,3.227)--(4.541,3.225)--(4.535,3.220)--(4.528,3.214)--(4.521,3.208)--(4.515,3.204)%
  --(4.508,3.199)--(4.501,3.195)--(4.501,3.193)--(4.495,3.190)--(4.495,3.188)--(4.488,3.186)%
  --(4.482,3.185)--(4.482,3.182)--(4.475,3.176)--(4.462,3.169)--(4.462,3.166)--(4.455,3.162)%
  --(4.448,3.157)--(4.442,3.153)--(4.435,3.149)--(4.428,3.146)--(4.422,3.143)--(4.422,3.142)%
  --(4.408,3.139)--(4.402,3.135)--(4.395,3.128)--(4.395,3.124)--(4.389,3.120)--(4.389,3.119)%
  --(4.382,3.116)--(4.375,3.112)--(4.369,3.109)--(4.362,3.102)--(4.355,3.097)--(4.349,3.096)%
  --(4.342,3.090)--(4.335,3.083)--(4.329,3.078)--(4.322,3.075)--(4.322,3.072)--(4.315,3.066)%
  --(4.302,3.060)--(4.296,3.055)--(4.296,3.051)--(4.289,3.051)--(4.282,3.047)--(4.269,3.041)%
  --(4.269,3.036)--(4.262,3.032)--(4.256,3.026)--(4.249,3.021)--(4.249,3.018)--(4.242,3.015)%
  --(4.236,3.013)--(4.229,3.006)--(4.222,3.004)--(4.222,3.000)--(4.216,2.999)--(4.209,2.996)%
  --(4.203,2.993)--(4.196,2.987)--(4.189,2.984)--(4.183,2.980)--(4.183,2.977)--(4.176,2.977)%
  --(4.176,2.974)--(4.176,2.970)--(4.169,2.967)--(4.163,2.966)--(4.156,2.961)--(4.149,2.957)%
  --(4.143,2.953)--(4.130,2.948)--(4.123,2.942)--(4.116,2.937)--(4.110,2.932)--(4.103,2.926)%
  --(4.096,2.921)--(4.090,2.916)--(4.083,2.911)--(4.076,2.906)--(4.070,2.900)--(4.063,2.898)%
  --(4.063,2.896)--(4.056,2.891)--(4.050,2.886)--(4.043,2.882)--(4.037,2.878)--(4.030,2.873)%
  --(4.023,2.867)--(4.017,2.862)--(4.003,2.854)--(4.003,2.850)--(3.997,2.849)--(3.990,2.847)%
  --(3.990,2.842)--(3.983,2.838)--(3.970,2.831)--(3.963,2.824)--(3.950,2.817)--(3.944,2.811)%
  --(3.930,2.807)--(3.924,2.801)--(3.924,2.800)--(3.917,2.800)--(3.917,2.796)--(3.910,2.790)%
  --(3.910,2.787)--(3.897,2.785)--(3.890,2.781)--(3.884,2.776)--(3.877,2.769)--(3.870,2.763)%
  --(3.864,2.758)--(3.857,2.755)--(3.851,2.750)--(3.851,2.747)--(3.837,2.744)--(3.837,2.738)%
  --(3.831,2.734)--(3.831,2.729)--(3.824,2.726)--(3.817,2.722)--(3.811,2.720)--(3.804,2.717)%
  --(3.797,2.711)--(3.797,2.708)--(3.797,2.705)--(3.791,2.699)--(3.784,2.695)--(3.771,2.690)%
  --(3.764,2.686)--(3.758,2.684)--(3.751,2.678)--(3.744,2.673)--(3.744,2.667)--(3.738,2.663)%
  --(3.724,2.659)--(3.718,2.652)--(3.718,2.648)--(3.711,2.646)--(3.704,2.643)--(3.698,2.640)%
  --(3.698,2.636)--(3.691,2.633)--(3.678,2.629)--(3.671,2.622)--(3.658,2.614)--(3.658,2.610)%
  --(3.651,2.611)--(3.651,2.608)--(3.645,2.605)--(3.638,2.603)--(3.631,2.598)--(3.618,2.591)%
  --(3.611,2.583)--(3.605,2.578)--(3.598,2.573)--(3.592,2.569)--(3.585,2.565)--(3.578,2.562)%
  --(3.572,2.558)--(3.565,2.553)--(3.558,2.544)--(3.552,2.539)--(3.545,2.535)--(3.532,2.527)%
  --(3.525,2.521)--(3.518,2.516)--(3.512,2.513)--(3.505,2.509)--(3.505,2.507)--(3.505,2.503)%
  --(3.499,2.499)--(3.499,2.497)--(3.492,2.495)--(3.485,2.493)--(3.479,2.488)--(3.472,2.483)%
  --(3.465,2.480)--(3.459,2.475)--(3.452,2.469)--(3.445,2.466)--(3.439,2.461)--(3.432,2.454)%
  --(3.425,2.448)--(3.419,2.443)--(3.406,2.439)--(3.399,2.435)--(3.399,2.434)--(3.392,2.431)%
  --(3.386,2.428)--(3.379,2.424)--(3.372,2.420)--(3.372,2.417)--(3.366,2.415)--(3.366,2.414)%
  --(3.359,2.409)--(3.352,2.405)--(3.346,2.402)--(3.346,2.400)--(3.346,2.398)--(3.346,2.397)%
  --(3.339,2.394)--(3.332,2.389)--(3.319,2.383)--(3.313,2.377)--(3.299,2.368)--(3.293,2.361)%
  --(3.279,2.355)--(3.273,2.348)--(3.259,2.343)--(3.259,2.340)--(3.259,2.336)--(3.253,2.332)%
  --(3.246,2.328)--(3.240,2.323)--(3.233,2.318)--(3.220,2.313)--(3.213,2.306)--(3.206,2.303)%
  --(3.200,2.299)--(3.193,2.293)--(3.193,2.289)--(3.186,2.285)--(3.186,2.283)--(3.180,2.281)%
  --(3.173,2.276)--(3.166,2.269)--(3.160,2.266)--(3.160,2.263)--(3.153,2.262)--(3.147,2.261)%
  --(3.140,2.258)--(3.133,2.255)--(3.127,2.252)--(3.120,2.248)--(3.113,2.244)--(3.107,2.239)%
  --(3.100,2.234)--(3.093,2.228)--(3.087,2.224)--(3.087,2.220)--(3.080,2.217)--(3.073,2.215)%
  --(3.067,2.209)--(3.054,2.201)--(3.047,2.197)--(3.047,2.193)--(3.034,2.185)--(3.020,2.178)%
  --(3.014,2.174)--(3.014,2.170)--(3.014,2.168)--(3.007,2.163)--(2.994,2.155)--(2.987,2.149)%
  --(2.974,2.144)--(2.974,2.138)--(2.961,2.135)--(2.954,2.131)--(2.947,2.127)--(2.941,2.122)%
  --(2.934,2.116)--(2.927,2.112)--(2.921,2.106)--(2.914,2.101)--(2.901,2.095)--(2.887,2.087)%
  --(2.881,2.085)--(2.874,2.081)--(2.874,2.077)--(2.868,2.074)--(2.861,2.070)--(2.861,2.067)%
  --(2.854,2.063)--(2.848,2.060)--(2.848,2.058)--(2.841,2.055)--(2.841,2.051)--(2.834,2.047)%
  --(2.828,2.044)--(2.828,2.040)--(2.821,2.039)--(2.814,2.033)--(2.808,2.027)--(2.801,2.021)%
  --(2.794,2.017)--(2.788,2.011)--(2.775,2.004)--(2.768,1.998)--(2.761,1.993)--(2.755,1.989)%
  --(2.755,1.986)--(2.748,1.980)--(2.741,1.975)--(2.735,1.974)--(2.735,1.973)--(2.735,1.971)%
  --(2.728,1.968)--(2.721,1.965)--(2.721,1.964)--(2.715,1.960)--(2.715,1.959)--(2.715,1.957)%
  --(2.708,1.952)--(2.708,1.951)--(2.702,1.947)--(2.695,1.942)--(2.688,1.937)--(2.682,1.933)%
  --(2.675,1.932)--(2.675,1.928)--(2.668,1.926)--(2.662,1.921)--(2.648,1.914)--(2.648,1.911)%
  --(2.648,1.910)--(2.642,1.908)--(2.628,1.903)--(2.622,1.896)--(2.615,1.890)--(2.602,1.886)%
  --(2.595,1.881)--(2.589,1.877)--(2.582,1.872)--(2.569,1.865)--(2.562,1.859)--(2.555,1.853)%
  --(2.549,1.845)--(2.549,1.842)--(2.542,1.840)--(2.542,1.838)--(2.542,1.835)--(2.535,1.832)%
  --(2.535,1.831)--(2.535,1.830)--(2.529,1.829)--(2.522,1.826)--(2.516,1.816)--(2.502,1.808)%
  --(2.489,1.802)--(2.482,1.795)--(2.469,1.788)--(2.462,1.783)--(2.462,1.780)--(2.456,1.779)%
  --(2.456,1.777)--(2.449,1.775)--(2.449,1.772)--(2.442,1.769)--(2.442,1.764)--(2.436,1.760)%
  --(2.429,1.756)--(2.423,1.751)--(2.416,1.747)--(2.409,1.743)--(2.409,1.738)--(2.403,1.734)%
  --(2.396,1.733)--(2.389,1.729)--(2.383,1.727)--(2.376,1.724)--(2.369,1.721)--(2.356,1.718)%
  --(2.349,1.715)--(2.349,1.714)--(2.343,1.712)--(2.336,1.710)--(2.330,1.710)--(2.330,1.708)%
  --(2.323,1.707)--(2.316,1.705);
\gpcolor{color=gp lt color 1}
\gpsetlinetype{gp lt plot 1}
\draw[gp path] (10.479,7.460)--(10.479,7.456)--(10.473,7.455)--(10.466,7.453)--(10.459,7.451)%
  --(10.453,7.457)--(10.453,7.456)--(10.453,7.455)--(10.433,7.454)--(10.433,7.453)--(10.426,7.445)%
  --(10.426,7.444)--(10.419,7.441)--(10.419,7.440)--(10.413,7.441)--(10.413,7.443)--(10.406,7.442)%
  --(10.406,7.440)--(10.406,7.438)--(10.399,7.432)--(10.399,7.428)--(10.393,7.425)--(10.393,7.424)%
  --(10.386,7.429)--(10.386,7.428)--(10.386,7.425)--(10.373,7.422)--(10.373,7.418)--(10.373,7.407)%
  --(10.373,7.404)--(10.366,7.407)--(10.360,7.405)--(10.360,7.404)--(10.360,7.402)--(10.360,7.400)%
  --(10.353,7.398)--(10.353,7.396)--(10.346,7.391)--(10.353,7.385)--(10.346,7.383)--(10.340,7.383)%
  --(10.346,7.381)--(10.333,7.385)--(10.326,7.382)--(10.326,7.379)--(10.326,7.376)--(10.326,7.373)%
  --(10.320,7.361)--(10.313,7.359)--(10.306,7.357)--(10.300,7.358)--(10.300,7.353)--(10.306,7.358)%
  --(10.293,7.354)--(10.293,7.348)--(10.273,7.344)--(10.273,7.341)--(10.267,7.338)--(10.260,7.335)%
  --(10.260,7.326)--(10.260,7.329)--(10.253,7.326)--(10.247,7.321)--(10.233,7.315)--(10.227,7.311)%
  --(10.227,7.300)--(10.220,7.297)--(10.213,7.292)--(10.213,7.286)--(10.207,7.290)--(10.207,7.288)%
  --(10.200,7.287)--(10.200,7.286)--(10.187,7.279)--(10.187,7.277)--(10.187,7.281)--(10.180,7.281)%
  --(10.174,7.279)--(10.174,7.278)--(10.174,7.277)--(10.167,7.276)--(10.160,7.274)--(10.167,7.266)%
  --(10.160,7.263)--(10.160,7.259)--(10.154,7.258)--(10.154,7.260)--(10.154,7.261)--(10.154,7.259)%
  --(10.147,7.257)--(10.134,7.253)--(10.140,7.248)--(10.140,7.245)--(10.134,7.244)--(10.134,7.242)%
  --(10.127,7.240)--(10.120,7.244)--(10.114,7.237)--(10.114,7.233)--(10.114,7.230)--(10.107,7.228)%
  --(10.107,7.219)--(10.101,7.216)--(10.101,7.215)--(10.101,7.214)--(10.101,7.217)--(10.094,7.217)%
  --(10.101,7.216)--(10.087,7.214)--(10.087,7.211)--(10.087,7.204)--(10.081,7.193)--(10.081,7.191)%
  --(10.074,7.189)--(10.074,7.188)--(10.067,7.193)--(10.067,7.190)--(10.061,7.189)--(10.061,7.188)%
  --(10.061,7.186)--(10.047,7.174)--(10.034,7.168)--(10.027,7.162)--(10.021,7.160)--(10.021,7.159)%
  --(10.021,7.157)--(10.014,7.155)--(10.021,7.154)--(10.014,7.152)--(10.014,7.149)--(10.008,7.142)%
  --(10.008,7.139)--(9.994,7.136)--(9.994,7.134)--(9.988,7.139)--(9.988,7.136)--(9.981,7.132)%
  --(9.968,7.128)--(9.968,7.126)--(9.968,7.124)--(9.968,7.123)--(9.961,7.113)--(9.948,7.108)%
  --(9.948,7.109)--(9.941,7.110)--(9.941,7.109)--(9.935,7.108)--(9.928,7.105)--(9.928,7.098)%
  --(9.928,7.096)--(9.928,7.094)--(9.921,7.099)--(9.915,7.096)--(9.921,7.093)--(9.915,7.090)%
  --(9.901,7.086)--(9.901,7.083)--(9.895,7.083)--(9.895,7.082)--(9.888,7.079)--(9.888,7.078)%
  --(9.888,7.074)--(9.881,7.071)--(9.875,7.068)--(9.875,7.065)--(9.875,7.057)--(9.868,7.051)%
  --(9.868,7.048)--(9.861,7.048)--(9.861,7.049)--(9.855,7.047)--(9.855,7.045)--(9.855,7.044)%
  --(9.848,7.043)--(9.848,7.035)--(9.848,7.034)--(9.848,7.032)--(9.842,7.035)--(9.848,7.035)%
  --(9.842,7.032)--(9.842,7.031)--(9.835,7.029)--(9.835,7.023)--(9.835,7.019)--(9.828,7.018)%
  --(9.828,7.016)--(9.822,7.021)--(9.822,7.018)--(9.815,7.016)--(9.815,7.014)--(9.808,7.012)%
  --(9.808,7.008)--(9.808,7.004)--(9.802,7.004)--(9.802,7.001)--(9.795,6.998)--(9.795,6.994)%
  --(9.788,6.992)--(9.782,6.988)--(9.775,6.983)--(9.768,6.981)--(9.768,6.975)--(9.768,6.978)%
  --(9.755,6.977)--(9.755,6.974)--(9.762,6.973)--(9.755,6.971)--(9.749,6.970)--(9.755,6.969)%
  --(9.755,6.966)--(9.742,6.962)--(9.749,6.953)--(9.749,6.961)--(9.742,6.958)--(9.742,6.956)%
  --(9.735,6.956)--(9.735,6.953)--(9.729,6.952)--(9.729,6.951)--(9.729,6.948)--(9.722,6.945)%
  --(9.715,6.943)--(9.715,6.944)--(9.715,6.943)--(9.709,6.943)--(9.715,6.943)--(9.709,6.941)%
  --(9.702,6.940)--(9.702,6.938)--(9.695,6.936)--(9.695,6.935)--(9.702,6.934)--(9.695,6.927)%
  --(9.695,6.933)--(9.695,6.931)--(9.689,6.930)--(9.695,6.929)--(9.689,6.927)--(9.689,6.925)%
  --(9.689,6.929)--(9.689,6.928)--(9.682,6.926)--(9.682,6.924)--(9.675,6.923)--(9.689,6.923)%
  --(9.669,6.921)--(9.675,6.920)--(9.669,6.911)--(9.675,6.919)--(9.669,6.919)--(9.662,6.918)%
  --(9.662,6.917)--(9.662,6.916)--(9.662,6.915)--(9.656,6.914)--(9.649,6.904)--(9.649,6.910)%
  --(9.649,6.909)--(9.642,6.907)--(9.649,6.904)--(9.642,6.902)--(9.642,6.901)--(9.629,6.900)%
  --(9.629,6.898)--(9.622,6.890)--(9.622,6.896)--(9.616,6.896)--(9.616,6.893)--(9.616,6.892)%
  --(9.616,6.889)--(9.616,6.886)--(9.609,6.882)--(9.609,6.878)--(9.602,6.873)--(9.602,6.869)%
  --(9.602,6.867)--(9.602,6.859)--(9.596,6.858)--(9.596,6.863)--(9.589,6.860)--(9.589,6.858)%
  --(9.596,6.856)--(9.576,6.848)--(9.576,6.845)--(9.576,6.843)--(9.569,6.840)--(9.569,6.839)%
  --(9.569,6.840)--(9.563,6.837)--(9.563,6.835)--(9.556,6.831)--(9.556,6.825)--(9.556,6.818)%
  --(9.549,6.817)--(9.549,6.816)--(9.536,6.815)--(9.543,6.818)--(9.543,6.817)--(9.529,6.815)%
  --(9.529,6.812)--(9.523,6.811)--(9.523,6.810)--(9.523,6.808)--(9.516,6.807)--(9.516,6.803)%
  --(9.516,6.801)--(9.509,6.799)--(9.516,6.798)--(9.509,6.798)--(9.509,6.796)--(9.503,6.788)%
  --(9.509,6.787)--(9.503,6.786)--(9.496,6.785)--(9.503,6.789)--(9.496,6.788)--(9.503,6.787)%
  --(9.503,6.785)--(9.490,6.784)--(9.490,6.777)--(9.490,6.775)--(9.483,6.774)--(9.483,6.771)%
  --(9.476,6.770)--(9.476,6.774)--(9.476,6.773)--(9.470,6.771)--(9.470,6.770)--(9.463,6.769)%
  --(9.456,6.761)--(9.463,6.760)--(9.456,6.758)--(9.450,6.757)--(9.443,6.760)--(9.443,6.761)%
  --(9.450,6.761)--(9.443,6.760)--(9.443,6.758)--(9.436,6.752)--(9.436,6.748)--(9.436,6.747)%
  --(9.430,6.746)--(9.436,6.747)--(9.430,6.750)--(9.430,6.748)--(9.423,6.748)--(9.423,6.747)%
  --(9.416,6.744)--(9.403,6.740)--(9.397,6.738)--(9.397,6.736)--(9.397,6.737)--(9.397,6.735)%
  --(9.397,6.734)--(9.397,6.732)--(9.397,6.730)--(9.390,6.730)--(9.390,6.726)--(9.390,6.725)%
  --(9.397,6.728)--(9.390,6.725)--(9.383,6.725)--(9.383,6.722)--(9.390,6.715)--(9.383,6.713)%
  --(9.377,6.710)--(9.370,6.709)--(9.377,6.712)--(9.363,6.710)--(9.363,6.706)--(9.357,6.704)%
  --(9.357,6.697)--(9.357,6.694)--(9.350,6.693)--(9.357,6.690)--(9.357,6.687)--(9.350,6.688)%
  --(9.343,6.687)--(9.343,6.685)--(9.343,6.682)--(9.343,6.679)--(9.337,6.672)--(9.330,6.669)%
  --(9.330,6.667)--(9.323,6.664)--(9.317,6.663)--(9.317,6.664)--(9.317,6.663)--(9.310,6.661)%
  --(9.310,6.660)--(9.310,6.657)--(9.304,6.656)--(9.304,6.655)--(9.304,6.652)--(9.304,6.650)%
  --(9.297,6.647)--(9.297,6.645)--(9.290,6.644)--(9.284,6.642)--(9.277,6.640)--(9.277,6.632)%
  --(9.277,6.630)--(9.270,6.627)--(9.270,6.628)--(9.264,6.626)--(9.264,6.623)--(9.257,6.622)%
  --(9.264,6.620)--(9.250,6.614)--(9.250,6.611)--(9.244,6.609)--(9.237,6.607)--(9.237,6.605)%
  --(9.237,6.607)--(9.230,6.605)--(9.237,6.604)--(9.230,6.603)--(9.230,6.601)--(9.230,6.595)%
  --(9.224,6.594)--(9.230,6.593)--(9.217,6.595)--(9.217,6.596)--(9.217,6.593)--(9.204,6.589)%
  --(9.204,6.585)--(9.197,6.582)--(9.197,6.574)--(9.191,6.572)--(9.191,6.569)--(9.184,6.567)%
  --(9.177,6.572)--(9.177,6.571)--(9.177,6.568)--(9.171,6.567)--(9.164,6.565)--(9.157,6.556)%
  --(9.157,6.553)--(9.157,6.554)--(9.157,6.557)--(9.151,6.556)--(9.151,6.553)--(9.144,6.551)%
  --(9.144,6.550)--(9.144,6.549)--(9.137,6.548)--(9.137,6.546)--(9.137,6.539)--(9.137,6.538)%
  --(9.131,6.537)--(9.137,6.541)--(9.124,6.539)--(9.131,6.541)--(9.124,6.539)--(9.124,6.538)%
  --(9.118,6.536)--(9.118,6.531)--(9.118,6.528)--(9.111,6.523)--(9.111,6.525)--(9.104,6.525)%
  --(9.104,6.522)--(9.111,6.520)--(9.104,6.514)--(9.104,6.509)--(9.098,6.504)--(9.098,6.503)%
  --(9.091,6.502)--(9.091,6.505)--(9.091,6.503)--(9.078,6.500)--(9.071,6.495)--(9.071,6.490)%
  --(9.071,6.483)--(9.064,6.483)--(9.071,6.483)--(9.064,6.479)--(9.058,6.478)--(9.051,6.476)%
  --(9.038,6.473)--(9.038,6.468)--(9.031,6.465)--(9.031,6.460)--(9.025,6.455)--(9.025,6.454)%
  --(9.025,6.449)--(9.018,6.453)--(9.018,6.452)--(9.011,6.451)--(9.018,6.449)--(9.011,6.447)%
  --(8.998,6.442)--(8.991,6.437)--(8.991,6.434)--(8.991,6.432)--(8.985,6.425)--(8.985,6.427)%
  --(8.978,6.425)--(8.978,6.421)--(8.978,6.418)--(8.971,6.417)--(8.965,6.414)--(8.958,6.411)%
  --(8.952,6.402)--(8.952,6.400)--(8.952,6.402)--(8.945,6.403)--(8.945,6.400)--(8.938,6.398)%
  --(8.938,6.395)--(8.932,6.394)--(8.925,6.390)--(8.918,6.384)--(8.918,6.381)--(8.918,6.378)%
  --(8.912,6.376)--(8.905,6.376)--(8.905,6.373)--(8.898,6.373)--(8.898,6.372)--(8.898,6.365)%
  --(8.898,6.364)--(8.898,6.369)--(8.892,6.371)--(8.892,6.370)--(8.885,6.367)--(8.885,6.364)%
  --(8.878,6.358)--(8.878,6.356)--(8.878,6.351)--(8.872,6.352)--(8.865,6.351)--(8.865,6.349)%
  --(8.865,6.348)--(8.865,6.344)--(8.859,6.341)--(8.852,6.340)--(8.845,6.335)--(8.845,6.332)%
  --(8.839,6.329)--(8.839,6.326)--(8.832,6.326)--(8.832,6.322)--(8.825,6.315)--(8.825,6.319)%
  --(8.819,6.318)--(8.819,6.316)--(8.812,6.312)--(8.805,6.308)--(8.805,6.304)--(8.799,6.300)%
  --(8.792,6.294)--(8.785,6.292)--(8.779,6.288)--(8.772,6.283)--(8.766,6.278)--(8.759,6.272)%
  --(8.746,6.266)--(8.739,6.253)--(8.746,6.259)--(8.732,6.257)--(8.732,6.256)--(8.726,6.253)%
  --(8.726,6.251)--(8.726,6.248)--(8.719,6.247)--(8.719,6.243)--(8.712,6.237)--(8.712,6.239)%
  --(8.706,6.234)--(8.706,6.233)--(8.699,6.229)--(8.692,6.224)--(8.686,6.223)--(8.692,6.223)%
  --(8.686,6.215)--(8.686,6.224)--(8.686,6.223)--(8.686,6.221)--(8.679,6.219)--(8.679,6.218)%
  --(8.686,6.216)--(8.679,6.215)--(8.673,6.214)--(8.666,6.212)--(8.673,6.212)--(8.666,6.212)%
  --(8.673,6.212)--(8.673,6.206)--(8.673,6.212)--(8.673,6.211)--(8.666,6.211)--(8.659,6.207)%
  --(8.646,6.201)--(8.639,6.196)--(8.626,6.189)--(8.619,6.184)--(8.613,6.178)--(8.613,6.180)%
  --(8.606,6.176)--(8.599,6.170)--(8.586,6.163)--(8.580,6.157)--(8.586,6.154)--(8.580,6.150)%
  --(8.573,6.145)--(8.566,6.140)--(8.560,6.135)--(8.553,6.129)--(8.546,6.129)--(8.540,6.124)%
  --(8.540,6.120)--(8.526,6.113)--(8.520,6.104)--(8.513,6.099)--(8.507,6.095)--(8.500,6.091)%
  --(8.493,6.085)--(8.487,6.084)--(8.473,6.080)--(8.473,6.077)--(8.473,6.075)--(8.467,6.073)%
  --(8.460,6.063)--(8.453,6.056)--(8.440,6.052)--(8.433,6.048)--(8.427,6.044)--(8.427,6.043)%
  --(8.420,6.041)--(8.414,6.037)--(8.407,6.034)--(8.400,6.031)--(8.400,6.029)--(8.394,6.028)%
  --(8.394,6.026)--(8.394,6.020)--(8.394,6.021)--(8.387,6.020)--(8.374,6.018)--(8.367,6.012)%
  --(8.367,6.004)--(8.360,5.995)--(8.354,5.990)--(8.354,5.985)--(8.340,5.982)--(8.334,5.979)%
  --(8.327,5.978)--(8.327,5.976)--(8.327,5.973)--(8.321,5.972)--(8.321,5.969)--(8.321,5.968)%
  --(8.321,5.963)--(8.307,5.958)--(8.301,5.953)--(8.301,5.952)--(8.294,5.950)--(8.287,5.946)%
  --(8.287,5.943)--(8.287,5.941)--(8.274,5.931)--(8.281,5.928)--(8.267,5.927)--(8.267,5.924)%
  --(8.267,5.925)--(8.261,5.926)--(8.261,5.923)--(8.261,5.922)--(8.261,5.921)--(8.261,5.919)%
  --(8.254,5.913)--(8.247,5.910)--(8.241,5.908)--(8.234,5.908)--(8.228,5.906)--(8.228,5.902)%
  --(8.221,5.899)--(8.221,5.897)--(8.214,5.894)--(8.214,5.892)--(8.208,5.890)--(8.201,5.887)%
  --(8.194,5.879)--(8.194,5.876)--(8.188,5.873)--(8.188,5.871)--(8.188,5.868)--(8.181,5.861)%
  --(8.168,5.857)--(8.161,5.854)--(8.154,5.852)--(8.154,5.855)--(8.154,5.853)--(8.148,5.850)%
  --(8.148,5.847)--(8.135,5.844)--(8.135,5.838)--(8.135,5.835)--(8.128,5.835)--(8.128,5.836)%
  --(8.128,5.838)--(8.128,5.835)--(8.121,5.834)--(8.121,5.832)--(8.115,5.829)--(8.115,5.827)%
  --(8.115,5.826)--(8.115,5.828)--(8.108,5.824)--(8.115,5.824)--(8.115,5.821)--(8.101,5.820)%
  --(8.101,5.819)--(8.101,5.817)--(8.101,5.811)--(8.101,5.809)--(8.095,5.806)--(8.088,5.803)%
  --(8.088,5.802)--(8.081,5.803)--(8.081,5.801)--(8.075,5.798)--(8.068,5.796)--(8.068,5.794)%
  --(8.062,5.792)--(8.062,5.791)--(8.062,5.790)--(8.055,5.787)--(8.055,5.785)--(8.048,5.781)%
  --(8.048,5.778)--(8.048,5.777)--(8.048,5.773)--(8.042,5.770)--(8.042,5.769)--(8.035,5.766)%
  --(8.035,5.764)--(8.035,5.767)--(8.035,5.765)--(8.028,5.764)--(8.028,5.762)--(8.022,5.760)%
  --(8.015,5.751)--(8.015,5.750)--(8.008,5.748)--(8.008,5.746)--(8.002,5.746)--(8.002,5.747)%
  --(7.995,5.744)--(7.995,5.742)--(7.995,5.741)--(7.988,5.740)--(7.995,5.739)--(7.988,5.737)%
  --(7.982,5.734)--(7.982,5.732)--(7.975,5.732)--(7.969,5.729)--(7.969,5.726)--(7.962,5.722)%
  --(7.955,5.719)--(7.942,5.710)--(7.935,5.705)--(7.929,5.699)--(7.909,5.689)--(7.902,5.682)%
  --(7.895,5.683)--(7.895,5.681)--(7.895,5.680)--(7.882,5.677)--(7.882,5.675)--(7.876,5.673)%
  --(7.876,5.672)--(7.876,5.669)--(7.869,5.665)--(7.869,5.664)--(7.862,5.658)--(7.862,5.655)%
  --(7.849,5.653)--(7.849,5.649)--(7.842,5.647)--(7.842,5.639)--(7.836,5.636)--(7.829,5.637)%
  --(7.816,5.634)--(7.816,5.629)--(7.816,5.625)--(7.809,5.622)--(7.809,5.616)--(7.802,5.612)%
  --(7.796,5.611)--(7.796,5.609)--(7.789,5.609)--(7.789,5.607)--(7.783,5.602)--(7.776,5.598)%
  --(7.769,5.595)--(7.769,5.591)--(7.756,5.586)--(7.749,5.584)--(7.749,5.578)--(7.743,5.576)%
  --(7.736,5.573)--(7.736,5.570)--(7.729,5.567)--(7.723,5.566)--(7.723,5.560)--(7.716,5.557)%
  --(7.716,5.556)--(7.716,5.554)--(7.716,5.556)--(7.709,5.556)--(7.709,5.554)--(7.709,5.549)%
  --(7.703,5.545)--(7.696,5.542)--(7.690,5.537)--(7.683,5.536)--(7.676,5.532)--(7.670,5.526)%
  --(7.656,5.521)--(7.656,5.517)--(7.656,5.515)--(7.650,5.511)--(7.643,5.509)--(7.636,5.506)%
  --(7.630,5.501)--(7.630,5.498)--(7.630,5.499)--(7.623,5.497)--(7.616,5.495)--(7.610,5.491)%
  --(7.610,5.488)--(7.603,5.483)--(7.603,5.482)--(7.603,5.478)--(7.603,5.475)--(7.597,5.477)%
  --(7.597,5.474)--(7.590,5.472)--(7.583,5.469)--(7.583,5.467)--(7.577,5.463)--(7.577,5.462)%
  --(7.577,5.461)--(7.577,5.459)--(7.577,5.460)--(7.570,5.460)--(7.570,5.458)--(7.570,5.456)%
  --(7.570,5.454)--(7.563,5.453)--(7.563,5.450)--(7.557,5.449)--(7.557,5.447)--(7.557,5.442)%
  --(7.550,5.441)--(7.543,5.440)--(7.543,5.439)--(7.543,5.437)--(7.537,5.435)--(7.537,5.431)%
  --(7.530,5.430)--(7.530,5.426)--(7.517,5.423)--(7.517,5.419)--(7.510,5.417)--(7.510,5.416)%
  --(7.510,5.413)--(7.504,5.410)--(7.497,5.407)--(7.497,5.403)--(7.484,5.401)--(7.484,5.393)%
  --(7.484,5.392)--(7.477,5.393)--(7.477,5.389)--(7.464,5.388)--(7.464,5.387)--(7.464,5.385)%
  --(7.464,5.382)--(7.457,5.379)--(7.457,5.376)--(7.450,5.374)--(7.450,5.372)--(7.450,5.374)%
  --(7.450,5.373)--(7.450,5.372)--(7.444,5.370)--(7.437,5.368)--(7.431,5.360)--(7.431,5.357)%
  --(7.424,5.355)--(7.424,5.352)--(7.424,5.355)--(7.411,5.350)--(7.411,5.347)--(7.404,5.342)%
  --(7.397,5.341)--(7.391,5.338)--(7.391,5.335)--(7.384,5.333)--(7.377,5.331)--(7.377,5.324)%
  --(7.371,5.325)--(7.371,5.323)--(7.371,5.317)--(7.364,5.314)--(7.357,5.313)--(7.357,5.304)%
  --(7.357,5.309)--(7.351,5.309)--(7.351,5.308)--(7.344,5.306)--(7.344,5.305)--(7.338,5.300)%
  --(7.331,5.296)--(7.331,5.292)--(7.324,5.290)--(7.324,5.287)--(7.318,5.283)--(7.318,5.285)%
  --(7.311,5.284)--(7.311,5.282)--(7.311,5.278)--(7.304,5.276)--(7.304,5.273)--(7.298,5.271)%
  --(7.291,5.263)--(7.291,5.265)--(7.291,5.263)--(7.284,5.259)--(7.278,5.256)--(7.271,5.252)%
  --(7.264,5.250)--(7.264,5.248)--(7.258,5.246)--(7.258,5.240)--(7.251,5.240)--(7.251,5.239)%
  --(7.251,5.236)--(7.245,5.233)--(7.238,5.231)--(7.238,5.230)--(7.231,5.227)--(7.231,5.226)%
  --(7.225,5.224)--(7.225,5.222)--(7.225,5.221)--(7.218,5.217)--(7.211,5.214)--(7.205,5.209)%
  --(7.205,5.207)--(7.205,5.204)--(7.198,5.201)--(7.198,5.198)--(7.191,5.195)--(7.185,5.194)%
  --(7.178,5.190)--(7.178,5.187)--(7.171,5.183)--(7.165,5.179)--(7.158,5.175)--(7.152,5.172)%
  --(7.145,5.168)--(7.145,5.166)--(7.145,5.165)--(7.152,5.165)--(7.145,5.158)--(7.145,5.163)%
  --(7.145,5.162)--(7.138,5.159)--(7.132,5.154)--(7.118,5.149)--(7.112,5.144)--(7.112,5.137)%
  --(7.105,5.138)--(7.098,5.137)--(7.098,5.134)--(7.098,5.132)--(7.092,5.126)--(7.092,5.123)%
  --(7.085,5.120)--(7.085,5.117)--(7.079,5.115)--(7.079,5.110)--(7.072,5.113)--(7.072,5.111)%
  --(7.065,5.110)--(7.065,5.107)--(7.065,5.106)--(7.059,5.105)--(7.059,5.104)--(7.052,5.102)%
  --(7.052,5.094)--(7.045,5.089)--(7.039,5.087)--(7.039,5.085)--(7.025,5.082)--(7.019,5.076)%
  --(7.012,5.072)--(7.012,5.071)--(7.019,5.071)--(7.019,5.067)--(7.012,5.065)--(7.005,5.065)%
  --(6.999,5.064)--(6.999,5.062)--(6.999,5.059)--(6.992,5.056)--(6.992,5.054)--(6.992,5.053)%
  --(6.986,5.050)--(6.979,5.045)--(6.979,5.044)--(6.979,5.042)--(6.972,5.041)--(6.966,5.039)%
  --(6.966,5.036)--(6.966,5.034)--(6.959,5.034)--(6.952,5.028)--(6.952,5.025)--(6.946,5.020)%
  --(6.946,5.018)--(6.939,5.018)--(6.939,5.019)--(6.939,5.018)--(6.932,5.013)--(6.926,5.011)%
  --(6.919,5.008)--(6.919,5.002)--(6.912,4.999)--(6.912,4.994)--(6.906,4.991)--(6.899,4.991)%
  --(6.893,4.989)--(6.893,4.985)--(6.886,4.983)--(6.886,4.980)--(6.879,4.976)--(6.879,4.972)%
  --(6.873,4.968)--(6.866,4.966)--(6.866,4.964)--(6.859,4.964)--(6.853,4.962)--(6.853,4.958)%
  --(6.846,4.958)--(6.846,4.955)--(6.846,4.953)--(6.839,4.952)--(6.839,4.948)--(6.833,4.944)%
  --(6.826,4.942)--(6.826,4.939)--(6.826,4.937)--(6.819,4.934)--(6.819,4.933)--(6.819,4.932)%
  --(6.813,4.928)--(6.813,4.924)--(6.813,4.925)--(6.806,4.924)--(6.806,4.922)--(6.800,4.919)%
  --(6.786,4.914)--(6.780,4.905)--(6.773,4.901)--(6.773,4.898)--(6.773,4.897)--(6.773,4.899)%
  --(6.760,4.893)--(6.760,4.892)--(6.760,4.891)--(6.753,4.886)--(6.746,4.882)--(6.733,4.873)%
  --(6.726,4.864)--(6.726,4.862)--(6.720,4.861)--(6.713,4.857)--(6.707,4.852)--(6.700,4.849)%
  --(6.693,4.844)--(6.680,4.837)--(6.673,4.831)--(6.673,4.829)--(6.667,4.824)--(6.667,4.826)%
  --(6.660,4.824)--(6.660,4.821)--(6.660,4.818)--(6.647,4.813)--(6.647,4.809)--(6.647,4.811)%
  --(6.647,4.809)--(6.640,4.809)--(6.634,4.804)--(6.627,4.798)--(6.614,4.795)--(6.614,4.793)%
  --(6.607,4.788)--(6.607,4.785)--(6.600,4.778)--(6.594,4.776)--(6.594,4.777)--(6.594,4.776)%
  --(6.587,4.773)--(6.587,4.770)--(6.580,4.768)--(6.580,4.763)--(6.574,4.759)--(6.574,4.756)%
  --(6.567,4.754)--(6.560,4.754)--(6.560,4.753)--(6.560,4.749)--(6.547,4.744)--(6.547,4.739)%
  --(6.541,4.733)--(6.534,4.730)--(6.534,4.728)--(6.521,4.725)--(6.521,4.723)--(6.514,4.720)%
  --(6.507,4.716)--(6.507,4.712)--(6.494,4.708)--(6.494,4.703)--(6.487,4.698)--(6.494,4.698)%
  --(6.487,4.695)--(6.481,4.693)--(6.474,4.693)--(6.467,4.685)--(6.461,4.679)--(6.454,4.674)%
  --(6.448,4.670)--(6.448,4.665)--(6.441,4.663)--(6.434,4.662)--(6.428,4.662)--(6.428,4.660)%
  --(6.421,4.656)--(6.421,4.652)--(6.414,4.648)--(6.408,4.644)--(6.408,4.640)--(6.401,4.638)%
  --(6.401,4.636)--(6.388,4.634)--(6.381,4.633)--(6.381,4.628)--(6.374,4.624)--(6.374,4.619)%
  --(6.368,4.617)--(6.368,4.614)--(6.361,4.608)--(6.361,4.606)--(6.355,4.603)--(6.348,4.600)%
  --(6.341,4.600)--(6.341,4.599)--(6.341,4.597)--(6.341,4.595)--(6.335,4.591)--(6.335,4.586)%
  --(6.328,4.581)--(6.321,4.579)--(6.328,4.579)--(6.328,4.578)--(6.321,4.581)--(6.321,4.580)%
  --(6.321,4.576)--(6.321,4.575)--(6.321,4.578)--(6.315,4.577)--(6.308,4.573)--(6.295,4.565)%
  --(6.268,4.552)--(6.255,4.543)--(6.255,4.539)--(6.248,4.536)--(6.248,4.535)--(6.242,4.531)%
  --(6.235,4.527)--(6.228,4.522)--(6.222,4.516)--(6.222,4.513)--(6.215,4.510)--(6.208,4.505)%
  --(6.208,4.503)--(6.202,4.502)--(6.195,4.500)--(6.195,4.497)--(6.195,4.495)--(6.188,4.494)%
  --(6.188,4.492)--(6.182,4.489)--(6.175,4.487)--(6.175,4.483)--(6.169,4.484)--(6.169,4.480)%
  --(6.162,4.476)--(6.155,4.470)--(6.142,4.463)--(6.129,4.454)--(6.122,4.445)--(6.109,4.440)%
  --(6.102,4.433)--(6.109,4.433)--(6.109,4.434)--(6.109,4.433)--(6.109,4.432)--(6.109,4.429)%
  --(6.109,4.431)--(6.109,4.433)--(6.109,4.432)--(6.102,4.429)--(6.102,4.426)--(6.096,4.424)%
  --(6.089,4.419)--(6.089,4.421)--(6.082,4.419)--(6.076,4.416)--(6.062,4.409)--(6.056,4.403)%
  --(6.056,4.401)--(6.056,4.399)--(6.049,4.397)--(6.049,4.393)--(6.036,4.389)--(6.029,4.383)%
  --(6.029,4.385)--(6.022,4.380)--(6.022,4.377)--(6.016,4.375)--(6.009,4.373)--(6.009,4.369)%
  --(6.009,4.364)--(6.003,4.365)--(6.003,4.364)--(6.003,4.363)--(5.996,4.362)--(6.003,4.361)%
  --(5.996,4.359)--(5.989,4.357)--(5.989,4.355)--(5.989,4.354)--(5.983,4.351)--(5.983,4.350)%
  --(5.983,4.349)--(5.976,4.348)--(5.969,4.341)--(5.969,4.340)--(5.969,4.339)--(5.963,4.336)%
  --(5.956,4.334)--(5.949,4.328)--(5.949,4.327)--(5.949,4.326)--(5.943,4.323)--(5.936,4.319)%
  --(5.929,4.313)--(5.929,4.310)--(5.923,4.308)--(5.910,4.301)--(5.896,4.290)--(5.890,4.281)%
  --(5.890,4.283)--(5.883,4.280)--(5.883,4.278)--(5.883,4.277)--(5.876,4.274)--(5.870,4.272)%
  --(5.870,4.270)--(5.863,4.265)--(5.863,4.262)--(5.856,4.260)--(5.856,4.258)--(5.850,4.256)%
  --(5.850,4.253)--(5.850,4.251)--(5.843,4.249)--(5.843,4.247)--(5.836,4.240)--(5.836,4.242)%
  --(5.830,4.240)--(5.830,4.239)--(5.830,4.237)--(5.823,4.234)--(5.817,4.230)--(5.810,4.228)%
  --(5.810,4.226)--(5.810,4.224)--(5.803,4.221)--(5.797,4.218)--(5.797,4.215)--(5.790,4.215)%
  --(5.790,4.213)--(5.783,4.210)--(5.783,4.209)--(5.777,4.207)--(5.777,4.204)--(5.777,4.198)%
  --(5.770,4.199)--(5.763,4.194)--(5.763,4.192)--(5.757,4.190)--(5.757,4.187)--(5.750,4.184)%
  --(5.743,4.182)--(5.743,4.180)--(5.743,4.176)--(5.737,4.174)--(5.730,4.173)--(5.724,4.171)%
  --(5.724,4.167)--(5.724,4.164)--(5.717,4.160)--(5.717,4.159)--(5.717,4.158)--(5.710,4.152)%
  --(5.704,4.147)--(5.704,4.144)--(5.697,4.140)--(5.684,4.136)--(5.684,4.133)--(5.677,4.129)%
  --(5.664,4.123)--(5.657,4.117)--(5.651,4.112)--(5.657,4.108)--(5.651,4.107)--(5.651,4.104)%
  --(5.644,4.099)--(5.637,4.098)--(5.631,4.096)--(5.631,4.095)--(5.631,4.094)--(5.624,4.093)%
  --(5.624,4.090)--(5.624,4.089)--(5.617,4.087)--(5.617,4.085)--(5.617,4.084)--(5.611,4.083)%
  --(5.611,4.081)--(5.611,4.080)--(5.611,4.078)--(5.611,4.077)--(5.604,4.072)--(5.597,4.068)%
  --(5.597,4.065)--(5.591,4.062)--(5.591,4.063)--(5.584,4.060)--(5.577,4.057)--(5.577,4.056)%
  --(5.577,4.055)--(5.577,4.052)--(5.571,4.049)--(5.564,4.048)--(5.564,4.046)--(5.564,4.045)%
  --(5.558,4.044)--(5.551,4.039)--(5.551,4.035)--(5.544,4.034)--(5.544,4.030)--(5.544,4.029)%
  --(5.538,4.027)--(5.538,4.026)--(5.538,4.027)--(5.531,4.025)--(5.531,4.024)--(5.524,4.022)%
  --(5.524,4.020)--(5.518,4.017)--(5.511,4.013)--(5.511,4.011)--(5.511,4.010)--(5.511,4.011)%
  --(5.511,4.009)--(5.504,4.008)--(5.504,4.006)--(5.498,4.004)--(5.491,3.998)--(5.491,3.996)%
  --(5.491,3.995)--(5.491,3.993)--(5.491,3.995)--(5.484,3.994)--(5.484,3.992)--(5.478,3.990)%
  --(5.478,3.988)--(5.471,3.986)--(5.471,3.981)--(5.465,3.978)--(5.458,3.976)--(5.451,3.974)%
  --(5.451,3.971)--(5.445,3.966)--(5.438,3.961)--(5.438,3.958)--(5.431,3.954)--(5.425,3.953)%
  --(5.425,3.951)--(5.418,3.948)--(5.411,3.945)--(5.411,3.941)--(5.405,3.939)--(5.405,3.934)%
  --(5.398,3.929)--(5.391,3.927)--(5.385,3.925)--(5.385,3.923)--(5.385,3.922)--(5.378,3.920)%
  --(5.378,3.917)--(5.372,3.914)--(5.372,3.910)--(5.365,3.907)--(5.358,3.904)--(5.358,3.902)%
  --(5.358,3.901)--(5.358,3.900)--(5.352,3.900)--(5.352,3.898)--(5.345,3.894)--(5.345,3.892)%
  --(5.345,3.890)--(5.338,3.888)--(5.338,3.889)--(5.332,3.886)--(5.325,3.883)--(5.325,3.879)%
  --(5.312,3.874)--(5.312,3.868)--(5.305,3.867)--(5.305,3.863)--(5.292,3.857)--(5.285,3.856)%
  --(5.285,3.853)--(5.279,3.851)--(5.272,3.846)--(5.265,3.843)--(5.259,3.839)--(5.259,3.838)%
  --(5.252,3.837)--(5.252,3.832)--(5.245,3.826)--(5.245,3.821)--(5.245,3.820)--(5.239,3.820)%
  --(5.239,3.816)--(5.232,3.815)--(5.225,3.815)--(5.225,3.812)--(5.219,3.809)--(5.212,3.803)%
  --(5.212,3.800)--(5.205,3.798)--(5.199,3.791)--(5.192,3.788)--(5.192,3.784)--(5.186,3.783)%
  --(5.179,3.783)--(5.179,3.780)--(5.172,3.775)--(5.166,3.773)--(5.159,3.765)--(5.152,3.760)%
  --(5.152,3.756)--(5.146,3.754)--(5.139,3.749)--(5.132,3.748)--(5.132,3.743)--(5.126,3.740)%
  --(5.126,3.738)--(5.119,3.737)--(5.113,3.734)--(5.113,3.732)--(5.113,3.731)--(5.113,3.729)%
  --(5.106,3.729)--(5.106,3.726)--(5.106,3.724)--(5.099,3.723)--(5.099,3.720)--(5.093,3.716)%
  --(5.093,3.714)--(5.093,3.713)--(5.086,3.710)--(5.086,3.711)--(5.086,3.709)--(5.086,3.708)%
  --(5.079,3.707)--(5.079,3.705)--(5.073,3.702)--(5.073,3.699)--(5.066,3.696)--(5.066,3.694)%
  --(5.059,3.692)--(5.059,3.690)--(5.053,3.688)--(5.053,3.686)--(5.046,3.683)--(5.039,3.678)%
  --(5.033,3.677)--(5.026,3.675)--(5.026,3.672)--(5.020,3.669)--(5.013,3.663)--(5.006,3.659)%
  --(5.006,3.655)--(5.000,3.653)--(4.993,3.649)--(4.986,3.643)--(4.986,3.637)--(4.980,3.634)%
  --(4.966,3.629)--(4.960,3.625)--(4.953,3.621)--(4.946,3.618)--(4.940,3.617)--(4.940,3.614)%
  --(4.933,3.610)--(4.933,3.608)--(4.927,3.605)--(4.927,3.604)--(4.927,3.603)--(4.920,3.601)%
  --(4.920,3.598)--(4.913,3.594)--(4.913,3.588)--(4.907,3.583)--(4.907,3.580)--(4.900,3.576)%
  --(4.900,3.575)--(4.893,3.575)--(4.887,3.572)--(4.887,3.569)--(4.880,3.569)--(4.880,3.567)%
  --(4.873,3.558)--(4.867,3.555)--(4.860,3.552)--(4.853,3.546)--(4.853,3.543)--(4.847,3.541)%
  --(4.840,3.537)--(4.840,3.534)--(4.834,3.531)--(4.834,3.528)--(4.827,3.525)--(4.827,3.523)%
  --(4.820,3.521)--(4.820,3.519)--(4.814,3.517)--(4.814,3.514)--(4.807,3.510)--(4.800,3.509)%
  --(4.800,3.504)--(4.794,3.499)--(4.787,3.497)--(4.780,3.495)--(4.780,3.493)--(4.774,3.491)%
  --(4.774,3.488)--(4.767,3.484)--(4.767,3.482)--(4.760,3.481)--(4.760,3.479)--(4.760,3.477)%
  --(4.754,3.477)--(4.754,3.476)--(4.747,3.474)--(4.747,3.473)--(4.741,3.471)--(4.741,3.466)%
  --(4.734,3.463)--(4.727,3.459)--(4.721,3.456)--(4.714,3.453)--(4.714,3.451)--(4.714,3.450)%
  --(4.714,3.449)--(4.707,3.445)--(4.707,3.442)--(4.707,3.440)--(4.701,3.439)--(4.694,3.434)%
  --(4.687,3.431)--(4.687,3.430)--(4.681,3.427)--(4.681,3.425)--(4.674,3.421)--(4.674,3.412)%
  --(4.661,3.409)--(4.661,3.405)--(4.654,3.405)--(4.648,3.403)--(4.641,3.398)--(4.641,3.394)%
  --(4.634,3.390)--(4.634,3.388)--(4.628,3.383)--(4.621,3.381)--(4.614,3.376)--(4.608,3.371)%
  --(4.601,3.370)--(4.601,3.366)--(4.594,3.363)--(4.588,3.359)--(4.581,3.354)--(4.581,3.350)%
  --(4.575,3.347)--(4.568,3.346)--(4.568,3.344)--(4.568,3.343)--(4.568,3.341)--(4.561,3.338)%
  --(4.555,3.335)--(4.548,3.333)--(4.548,3.331)--(4.548,3.330)--(4.541,3.329)--(4.535,3.323)%
  --(4.528,3.317)--(4.521,3.315)--(4.521,3.314)--(4.508,3.311)--(4.508,3.307)--(4.501,3.303)%
  --(4.501,3.301)--(4.495,3.298)--(4.495,3.292)--(4.488,3.292)--(4.482,3.289)--(4.482,3.286)%
  --(4.475,3.282)--(4.468,3.279)--(4.468,3.278)--(4.468,3.274)--(4.462,3.273)--(4.455,3.268)%
  --(4.448,3.264)--(4.442,3.262)--(4.442,3.259)--(4.442,3.258)--(4.435,3.254)--(4.435,3.253)%
  --(4.428,3.249)--(4.422,3.244)--(4.415,3.240)--(4.402,3.232)--(4.395,3.227)--(4.395,3.223)%
  --(4.389,3.218)--(4.382,3.214)--(4.382,3.215)--(4.375,3.213)--(4.369,3.208)--(4.369,3.205)%
  --(4.362,3.200)--(4.362,3.196)--(4.362,3.194)--(4.355,3.192)--(4.355,3.190)--(4.349,3.185)%
  --(4.349,3.181)--(4.342,3.181)--(4.342,3.179)--(4.335,3.178)--(4.335,3.177)--(4.335,3.176)%
  --(4.329,3.175)--(4.322,3.173)--(4.322,3.171)--(4.315,3.166)--(4.309,3.161)--(4.302,3.154)%
  --(4.296,3.149)--(4.289,3.150)--(4.296,3.150)--(4.289,3.148)--(4.289,3.146)--(4.282,3.143)%
  --(4.276,3.139)--(4.269,3.135)--(4.262,3.131)--(4.256,3.129)--(4.256,3.126)--(4.249,3.122)%
  --(4.242,3.117)--(4.236,3.113)--(4.229,3.109)--(4.229,3.107)--(4.222,3.103)--(4.216,3.100)%
  --(4.209,3.095)--(4.203,3.092)--(4.196,3.089)--(4.189,3.087)--(4.183,3.081)--(4.176,3.075)%
  --(4.176,3.068)--(4.176,3.065)--(4.169,3.061)--(4.163,3.056)--(4.156,3.053)--(4.149,3.051)%
  --(4.149,3.048)--(4.143,3.046)--(4.136,3.044)--(4.136,3.042)--(4.136,3.039)--(4.130,3.037)%
  --(4.130,3.035)--(4.123,3.032)--(4.123,3.028)--(4.116,3.027)--(4.116,3.024)--(4.116,3.023)%
  --(4.110,3.019)--(4.103,3.014)--(4.103,3.010)--(4.096,3.007)--(4.090,3.005)--(4.090,3.004)%
  --(4.083,2.999)--(4.076,2.995)--(4.076,2.994)--(4.076,2.993)--(4.070,2.991)--(4.070,2.990)%
  --(4.063,2.988)--(4.063,2.986)--(4.056,2.983)--(4.050,2.978)--(4.043,2.974)--(4.037,2.972)%
  --(4.037,2.967)--(4.023,2.963)--(4.017,2.958)--(4.017,2.955)--(4.010,2.953)--(4.003,2.950)%
  --(4.003,2.948)--(3.997,2.944)--(3.997,2.942)--(3.990,2.940)--(3.983,2.937)--(3.983,2.935)%
  --(3.977,2.929)--(3.963,2.925)--(3.963,2.920)--(3.957,2.916)--(3.950,2.910)--(3.944,2.905)%
  --(3.937,2.901)--(3.937,2.898)--(3.937,2.895)--(3.930,2.893)--(3.930,2.890)--(3.924,2.888)%
  --(3.917,2.885)--(3.917,2.884)--(3.910,2.879)--(3.897,2.872)--(3.897,2.870)--(3.890,2.867)%
  --(3.890,2.864)--(3.884,2.861)--(3.884,2.860)--(3.884,2.858)--(3.877,2.856)--(3.877,2.854)%
  --(3.877,2.851)--(3.870,2.849)--(3.870,2.847)--(3.870,2.845)--(3.864,2.843)--(3.857,2.842)%
  --(3.851,2.838)--(3.844,2.832)--(3.831,2.826)--(3.824,2.819)--(3.824,2.817)--(3.824,2.815)%
  --(3.824,2.814)--(3.817,2.811)--(3.811,2.807)--(3.804,2.804)--(3.797,2.800)--(3.791,2.799)%
  --(3.791,2.796)--(3.784,2.792)--(3.777,2.788)--(3.777,2.784)--(3.777,2.783)--(3.777,2.782)%
  --(3.771,2.782)--(3.764,2.780)--(3.758,2.774)--(3.758,2.773)--(3.758,2.772)--(3.758,2.770)%
  --(3.744,2.763)--(3.744,2.762)--(3.738,2.758)--(3.731,2.755)--(3.724,2.751)--(3.718,2.746)%
  --(3.711,2.740)--(3.704,2.737)--(3.704,2.734)--(3.704,2.730)--(3.698,2.728)--(3.691,2.724)%
  --(3.685,2.717)--(3.671,2.712)--(3.665,2.705)--(3.651,2.693)--(3.645,2.687)--(3.631,2.681)%
  --(3.625,2.678)--(3.625,2.675)--(3.618,2.672)--(3.611,2.668)--(3.611,2.665)--(3.611,2.662)%
  --(3.605,2.660)--(3.598,2.656)--(3.598,2.653)--(3.598,2.652)--(3.592,2.650)--(3.592,2.648)%
  --(3.592,2.645)--(3.585,2.643)--(3.585,2.641)--(3.578,2.638)--(3.572,2.634)--(3.572,2.633)%
  --(3.565,2.629)--(3.558,2.625)--(3.552,2.623)--(3.545,2.619)--(3.545,2.616)--(3.538,2.613)%
  --(3.532,2.610)--(3.532,2.607)--(3.525,2.603)--(3.518,2.598)--(3.512,2.596)--(3.505,2.595)%
  --(3.505,2.592)--(3.499,2.587)--(3.492,2.586)--(3.492,2.584)--(3.485,2.581)--(3.485,2.578)%
  --(3.479,2.576)--(3.479,2.573)--(3.472,2.570)--(3.465,2.564)--(3.459,2.560)--(3.459,2.557)%
  --(3.452,2.553)--(3.445,2.551)--(3.445,2.549)--(3.439,2.545)--(3.439,2.541)--(3.432,2.537)%
  --(3.432,2.536)--(3.425,2.534)--(3.425,2.531)--(3.419,2.527)--(3.412,2.522)--(3.406,2.518)%
  --(3.399,2.513)--(3.392,2.510)--(3.392,2.508)--(3.392,2.507)--(3.386,2.503)--(3.379,2.499)%
  --(3.379,2.496)--(3.379,2.494)--(3.372,2.494)--(3.372,2.492)--(3.372,2.490)--(3.366,2.489)%
  --(3.366,2.488)--(3.366,2.486)--(3.359,2.485)--(3.359,2.483)--(3.359,2.482)--(3.359,2.480)%
  --(3.352,2.478)--(3.346,2.475)--(3.346,2.473)--(3.339,2.471)--(3.339,2.470)--(3.332,2.468)%
  --(3.332,2.467)--(3.332,2.466)--(3.326,2.461)--(3.319,2.456)--(3.313,2.452)--(3.306,2.447)%
  --(3.299,2.438)--(3.299,2.437)--(3.293,2.437)--(3.293,2.435)--(3.286,2.431)--(3.279,2.429)%
  --(3.273,2.425)--(3.266,2.420)--(3.266,2.418)--(3.259,2.417)--(3.259,2.415)--(3.253,2.414)%
  --(3.253,2.413)--(3.253,2.411)--(3.253,2.410)--(3.246,2.410)--(3.246,2.406)--(3.240,2.402)%
  --(3.240,2.401)--(3.233,2.397)--(3.233,2.396)--(3.226,2.394)--(3.226,2.393)--(3.220,2.391)%
  --(3.220,2.388)--(3.220,2.386)--(3.213,2.384)--(3.213,2.383)--(3.213,2.380)--(3.206,2.378)%
  --(3.200,2.374)--(3.193,2.367)--(3.186,2.362)--(3.180,2.358)--(3.173,2.353)--(3.173,2.350)%
  --(3.166,2.347)--(3.166,2.346)--(3.153,2.339)--(3.147,2.334)--(3.140,2.330)--(3.140,2.329)%
  --(3.133,2.326)--(3.127,2.320)--(3.113,2.313)--(3.107,2.306)--(3.093,2.300)--(3.093,2.299)%
  --(3.093,2.298)--(3.093,2.297)--(3.093,2.295)--(3.087,2.290)--(3.080,2.286)--(3.080,2.285)%
  --(3.080,2.284)--(3.080,2.282)--(3.073,2.281)--(3.067,2.277)--(3.060,2.272)--(3.060,2.271)%
  --(3.060,2.269)--(3.054,2.264)--(3.047,2.262)--(3.047,2.261)--(3.047,2.258)--(3.040,2.256)%
  --(3.034,2.253)--(3.027,2.253)--(3.027,2.252)--(3.027,2.251)--(3.020,2.248)--(3.014,2.247)%
  --(3.014,2.244)--(3.014,2.242)--(3.007,2.239)--(3.000,2.235)--(2.994,2.230)--(2.987,2.225)%
  --(2.987,2.223)--(2.987,2.221)--(2.980,2.220)--(2.980,2.216)--(2.974,2.212)--(2.967,2.210)%
  --(2.961,2.205)--(2.954,2.202)--(2.954,2.198)--(2.947,2.196)--(2.947,2.193)--(2.941,2.192)%
  --(2.941,2.190)--(2.934,2.184)--(2.934,2.182)--(2.934,2.180)--(2.927,2.177)--(2.921,2.174)%
  --(2.914,2.170)--(2.907,2.166)--(2.907,2.160)--(2.901,2.157)--(2.894,2.154)--(2.887,2.149)%
  --(2.887,2.147)--(2.881,2.144)--(2.874,2.141)--(2.874,2.138)--(2.868,2.136)--(2.868,2.132)%
  --(2.861,2.132)--(2.861,2.131)--(2.861,2.128)--(2.854,2.127)--(2.848,2.124)--(2.841,2.119)%
  --(2.841,2.114)--(2.828,2.110)--(2.821,2.102)--(2.814,2.097)--(2.814,2.095)--(2.808,2.092)%
  --(2.801,2.090)--(2.794,2.086)--(2.788,2.084)--(2.788,2.081)--(2.781,2.079)--(2.775,2.076)%
  --(2.775,2.073)--(2.775,2.072)--(2.768,2.069)--(2.768,2.066)--(2.761,2.063)--(2.761,2.062)%
  --(2.761,2.060)--(2.755,2.058)--(2.755,2.057)--(2.748,2.055)--(2.741,2.051)--(2.741,2.047)%
  --(2.735,2.043)--(2.728,2.039)--(2.721,2.036)--(2.721,2.033)--(2.715,2.028)--(2.702,2.021)%
  --(2.702,2.018)--(2.695,2.012)--(2.688,2.006)--(2.682,2.002)--(2.675,1.998)--(2.675,1.995)%
  --(2.668,1.993)--(2.668,1.990)--(2.668,1.989)--(2.662,1.988)--(2.662,1.987)--(2.662,1.985)%
  --(2.662,1.984)--(2.655,1.980)--(2.648,1.979)--(2.648,1.975)--(2.642,1.971)--(2.642,1.970)%
  --(2.635,1.968)--(2.635,1.967)--(2.635,1.966)--(2.628,1.964)--(2.628,1.960)--(2.628,1.959)%
  --(2.622,1.956)--(2.615,1.951)--(2.615,1.948)--(2.609,1.947)--(2.602,1.943)--(2.595,1.940)%
  --(2.595,1.935)--(2.589,1.933)--(2.582,1.930)--(2.582,1.928)--(2.575,1.926)--(2.575,1.924)%
  --(2.569,1.921)--(2.562,1.918)--(2.562,1.914)--(2.555,1.911)--(2.555,1.910)--(2.549,1.907)%
  --(2.549,1.905)--(2.542,1.904)--(2.542,1.901)--(2.535,1.899)--(2.535,1.894)--(2.529,1.891)%
  --(2.522,1.889)--(2.522,1.887)--(2.516,1.886)--(2.516,1.885)--(2.516,1.883)--(2.509,1.882)%
  --(2.509,1.881)--(2.502,1.877)--(2.496,1.873)--(2.489,1.871)--(2.489,1.867)--(2.482,1.862)%
  --(2.476,1.859)--(2.469,1.854)--(2.462,1.849)--(2.462,1.843)--(2.456,1.840)--(2.449,1.832)%
  --(2.442,1.829)--(2.436,1.827)--(2.429,1.824)--(2.416,1.815)--(2.416,1.811)--(2.409,1.808)%
  --(2.409,1.805)--(2.409,1.803)--(2.403,1.800)--(2.403,1.798)--(2.396,1.796)--(2.396,1.794)%
  --(2.389,1.791)--(2.389,1.789)--(2.383,1.785)--(2.376,1.781)--(2.369,1.778)--(2.363,1.775)%
  --(2.363,1.772)--(2.356,1.767)--(2.349,1.762)--(2.349,1.761)--(2.349,1.760)--(2.343,1.758)%
  --(2.343,1.756)--(2.336,1.751)--(2.330,1.746)--(2.323,1.742)--(2.316,1.737);
\gpcolor{color=gp lt color 2}
\gpsetlinetype{gp lt plot 2}
\draw[gp path] (10.333,7.476)--(10.333,7.474)--(10.326,7.473)--(10.320,7.472)--(10.320,7.464)%
  --(10.313,7.462)--(10.313,7.461)--(10.306,7.461)--(10.300,7.464)--(10.300,7.465)--(10.300,7.464)%
  --(10.287,7.462)--(10.280,7.462)--(10.280,7.457)--(10.280,7.453)--(10.273,7.451)--(10.273,7.450)%
  --(10.267,7.450)--(10.260,7.455)--(10.253,7.454)--(10.247,7.453)--(10.233,7.448)--(10.227,7.439)%
  --(10.213,7.434)--(10.213,7.433)--(10.213,7.431)--(10.213,7.434)--(10.207,7.434)--(10.213,7.434)%
  --(10.207,7.433)--(10.200,7.430)--(10.200,7.424)--(10.194,7.421)--(10.194,7.422)--(10.187,7.417)%
  --(10.187,7.413)--(10.180,7.410)--(10.174,7.406)--(10.180,7.404)--(10.174,7.403)--(10.167,7.402)%
  --(10.174,7.394)--(10.174,7.393)--(10.167,7.391)--(10.167,7.389)--(10.160,7.390)--(10.167,7.393)%
  --(10.160,7.393)--(10.167,7.392)--(10.160,7.391)--(10.160,7.390)--(10.160,7.389)--(10.160,7.382)%
  --(10.160,7.380)--(10.154,7.386)--(10.160,7.386)--(10.154,7.386)--(10.154,7.384)--(10.154,7.383)%
  --(10.154,7.377)--(10.154,7.374)--(10.140,7.372)--(10.147,7.370)--(10.140,7.371)--(10.147,7.373)%
  --(10.140,7.371)--(10.140,7.369)--(10.134,7.368)--(10.134,7.366)--(10.134,7.359)--(10.134,7.358)%
  --(10.127,7.358)--(10.127,7.356)--(10.120,7.360)--(10.127,7.361)--(10.120,7.360)--(10.127,7.359)%
  --(10.120,7.358)--(10.114,7.352)--(10.114,7.346)--(10.114,7.344)--(10.107,7.347)--(10.107,7.345)%
  --(10.101,7.344)--(10.087,7.342)--(10.094,7.341)--(10.087,7.339)--(10.081,7.330)--(10.081,7.328)%
  --(10.081,7.326)--(10.081,7.333)--(10.081,7.330)--(10.074,7.327)--(10.074,7.323)--(10.061,7.320)%
  --(10.067,7.314)--(10.061,7.314)--(10.061,7.315)--(10.061,7.313)--(10.047,7.309)--(10.041,7.303)%
  --(10.041,7.298)--(10.034,7.293)--(10.034,7.291)--(10.034,7.290)--(10.027,7.282)--(10.021,7.288)%
  --(10.021,7.287)--(10.021,7.286)--(10.014,7.285)--(10.014,7.284)--(10.014,7.282)--(10.008,7.281)%
  --(10.008,7.279)--(10.008,7.278)--(10.001,7.280)--(10.008,7.279)--(9.994,7.278)--(9.988,7.277)%
  --(9.988,7.274)--(9.988,7.272)--(9.981,7.268)--(9.974,7.256)--(9.974,7.263)--(9.968,7.258)%
  --(9.968,7.254)--(9.961,7.252)--(9.954,7.249)--(9.948,7.246)--(9.941,7.243)--(9.948,7.239)%
  --(9.941,7.239)--(9.935,7.236)--(9.935,7.234)--(9.935,7.232)--(9.935,7.229)--(9.928,7.228)%
  --(9.928,7.225)--(9.921,7.225)--(9.928,7.214)--(9.921,7.219)--(9.921,7.216)--(9.921,7.215)%
  --(9.921,7.211)--(9.908,7.202)--(9.901,7.198)--(9.901,7.195)--(9.895,7.193)--(9.888,7.187)%
  --(9.881,7.187)--(9.881,7.185)--(9.875,7.182)--(9.868,7.179)--(9.875,7.179)--(9.861,7.176)%
  --(9.868,7.175)--(9.861,7.173)--(9.855,7.165)--(9.848,7.154)--(9.848,7.160)--(9.835,7.156)%
  --(9.835,7.152)--(9.835,7.148)--(9.822,7.144)--(9.822,7.139)--(9.822,7.136)--(9.802,7.133)%
  --(9.802,7.124)--(9.788,7.122)--(9.782,7.117)--(9.782,7.115)--(9.782,7.110)--(9.775,7.109)%
  --(9.768,7.105)--(9.762,7.105)--(9.762,7.104)--(9.762,7.095)--(9.755,7.094)--(9.749,7.091)%
  --(9.742,7.088)--(9.742,7.089)--(9.735,7.089)--(9.735,7.087)--(9.729,7.085)--(9.735,7.083)%
  --(9.729,7.082)--(9.729,7.081)--(9.722,7.080)--(9.722,7.071)--(9.722,7.069)--(9.715,7.074)%
  --(9.715,7.073)--(9.715,7.070)--(9.709,7.068)--(9.709,7.065)--(9.695,7.056)--(9.702,7.052)%
  --(9.695,7.049)--(9.695,7.045)--(9.689,7.043)--(9.689,7.045)--(9.682,7.044)--(9.675,7.041)%
  --(9.675,7.040)--(9.682,7.037)--(9.675,7.029)--(9.669,7.026)--(9.662,7.024)--(9.662,7.021)%
  --(9.662,7.025)--(9.656,7.025)--(9.656,7.023)--(9.649,7.021)--(9.649,7.015)--(9.649,7.019)%
  --(9.649,7.016)--(9.649,7.013)--(9.642,7.011)--(9.636,7.009)--(9.636,7.007)--(9.629,6.998)%
  --(9.622,6.995)--(9.629,6.994)--(9.622,6.992)--(9.616,6.995)--(9.609,6.993)--(9.609,6.991)%
  --(9.609,6.985)--(9.602,6.983)--(9.596,6.980)--(9.589,6.977)--(9.582,6.973)--(9.576,6.961)%
  --(9.576,6.959)--(9.569,6.959)--(9.556,6.953)--(9.543,6.944)--(9.529,6.934)--(9.523,6.927)%
  --(9.516,6.921)--(9.516,6.918)--(9.516,6.915)--(9.509,6.911)--(9.496,6.915)--(9.490,6.913)%
  --(9.490,6.911)--(9.483,6.907)--(9.476,6.902)--(9.476,6.896)--(9.470,6.892)--(9.463,6.890)%
  --(9.456,6.886)--(9.443,6.882)--(9.436,6.875)--(9.430,6.867)--(9.430,6.864)--(9.423,6.859)%
  --(9.410,6.853)--(9.410,6.845)--(9.416,6.845)--(9.410,6.844)--(9.410,6.847)--(9.403,6.846)%
  --(9.397,6.840)--(9.390,6.835)--(9.383,6.828)--(9.383,6.824)--(9.383,6.816)--(9.363,6.812)%
  --(9.363,6.810)--(9.357,6.808)--(9.350,6.805)--(9.343,6.802)--(9.343,6.794)--(9.337,6.790)%
  --(9.330,6.788)--(9.323,6.785)--(9.323,6.781)--(9.317,6.778)--(9.317,6.773)--(9.317,6.771)%
  --(9.317,6.775)--(9.310,6.774)--(9.304,6.772)--(9.304,6.771)--(9.304,6.767)--(9.290,6.758)%
  --(9.277,6.755)--(9.277,6.753)--(9.277,6.750)--(9.277,6.752)--(9.270,6.749)--(9.270,6.747)%
  --(9.264,6.746)--(9.264,6.745)--(9.264,6.741)--(9.257,6.734)--(9.250,6.733)--(9.244,6.731)%
  --(9.244,6.734)--(9.237,6.732)--(9.230,6.729)--(9.230,6.727)--(9.224,6.725)--(9.224,6.715)%
  --(9.217,6.711)--(9.211,6.708)--(9.204,6.704)--(9.197,6.703)--(9.191,6.700)--(9.184,6.693)%
  --(9.184,6.687)--(9.177,6.683)--(9.177,6.680)--(9.177,6.672)--(9.177,6.669)--(9.177,6.668)%
  --(9.171,6.664)--(9.157,6.665)--(9.157,6.664)--(9.151,6.661)--(9.151,6.658)--(9.144,6.656)%
  --(9.144,6.650)--(9.144,6.647)--(9.144,6.645)--(9.144,6.644)--(9.137,6.642)--(9.131,6.645)%
  --(9.131,6.642)--(9.124,6.640)--(9.111,6.636)--(9.111,6.634)--(9.111,6.626)--(9.111,6.624)%
  --(9.104,6.621)--(9.104,6.618)--(9.091,6.619)--(9.098,6.618)--(9.078,6.615)--(9.084,6.609)%
  --(9.071,6.604)--(9.071,6.597)--(9.064,6.593)--(9.064,6.589)--(9.058,6.585)--(9.045,6.581)%
  --(9.038,6.578)--(9.031,6.577)--(9.031,6.576)--(9.031,6.571)--(9.031,6.575)--(9.025,6.573)%
  --(9.031,6.571)--(9.025,6.572)--(9.025,6.568)--(9.018,6.567)--(9.018,6.563)--(9.011,6.561)%
  --(9.011,6.559)--(9.005,6.552)--(9.005,6.551)--(9.005,6.549)--(8.998,6.549)--(8.991,6.552)%
  --(8.985,6.550)--(8.985,6.547)--(8.978,6.542)--(8.965,6.536)--(8.952,6.529)--(8.952,6.524)%
  --(8.945,6.512)--(8.945,6.508)--(8.945,6.511)--(8.938,6.509)--(8.938,6.506)--(8.932,6.504)%
  --(8.932,6.502)--(8.925,6.495)--(8.932,6.493)--(8.918,6.493)--(8.925,6.491)--(8.918,6.490)%
  --(8.912,6.489)--(8.912,6.485)--(8.905,6.482)--(8.905,6.478)--(8.898,6.473)--(8.885,6.465)%
  --(8.885,6.462)--(8.878,6.460)--(8.878,6.459)--(8.878,6.462)--(8.878,6.463)--(8.872,6.461)%
  --(8.872,6.458)--(8.865,6.454)--(8.859,6.447)--(8.859,6.448)--(8.845,6.445)--(8.839,6.438)%
  --(8.839,6.433)--(8.825,6.429)--(8.825,6.421)--(8.812,6.416)--(8.805,6.413)--(8.799,6.411)%
  --(8.799,6.407)--(8.792,6.398)--(8.792,6.395)--(8.792,6.394)--(8.792,6.396)--(8.785,6.395)%
  --(8.785,6.394)--(8.772,6.391)--(8.766,6.388)--(8.759,6.380)--(8.752,6.374)--(8.752,6.371)%
  --(8.746,6.367)--(8.739,6.366)--(8.732,6.370)--(8.726,6.367)--(8.726,6.364)--(8.719,6.362)%
  --(8.719,6.358)--(8.706,6.346)--(8.699,6.338)--(8.692,6.331)--(8.686,6.328)--(8.679,6.325)%
  --(8.679,6.321)--(8.673,6.319)--(8.673,6.318)--(8.673,6.316)--(8.673,6.310)--(8.666,6.311)%
  --(8.659,6.310)--(8.653,6.304)--(8.653,6.299)--(8.646,6.294)--(8.639,6.290)--(8.633,6.286)%
  --(8.626,6.280)--(8.619,6.278)--(8.613,6.270)--(8.613,6.273)--(8.606,6.269)--(8.599,6.262)%
  --(8.593,6.259)--(8.586,6.256)--(8.580,6.252)--(8.573,6.247)--(8.566,6.241)--(8.553,6.234)%
  --(8.546,6.229)--(8.540,6.213)--(8.533,6.217)--(8.526,6.216)--(8.526,6.215)--(8.526,6.214)%
  --(8.526,6.212)--(8.520,6.209)--(8.513,6.200)--(8.507,6.204)--(8.500,6.201)--(8.493,6.197)%
  --(8.493,6.195)--(8.487,6.191)--(8.473,6.182)--(8.467,6.177)--(8.460,6.174)--(8.453,6.166)%
  --(8.453,6.161)--(8.447,6.160)--(8.447,6.158)--(8.440,6.156)--(8.440,6.155)--(8.447,6.155)%
  --(8.440,6.154)--(8.440,6.152)--(8.433,6.149)--(8.433,6.147)--(8.433,6.140)--(8.433,6.144)%
  --(8.427,6.142)--(8.433,6.139)--(8.420,6.134)--(8.420,6.132)--(8.414,6.131)--(8.414,6.130)%
  --(8.407,6.128)--(8.407,6.130)--(8.407,6.127)--(8.400,6.124)--(8.400,6.121)--(8.394,6.119)%
  --(8.394,6.118)--(8.394,6.116)--(8.387,6.113)--(8.380,6.108)--(8.367,6.103)--(8.367,6.092)%
  --(8.360,6.094)--(8.354,6.088)--(8.340,6.082)--(8.340,6.079)--(8.334,6.075)--(8.327,6.073)%
  --(8.327,6.070)--(8.321,6.060)--(8.314,6.061)--(8.301,6.057)--(8.301,6.055)--(8.301,6.053)%
  --(8.294,6.052)--(8.294,6.047)--(8.287,6.046)--(8.287,6.045)--(8.287,6.041)--(8.281,6.036)%
  --(8.274,6.031)--(8.274,6.029)--(8.267,6.027)--(8.261,6.028)--(8.254,6.026)--(8.247,6.024)%
  --(8.254,6.023)--(8.247,6.021)--(8.241,6.013)--(8.234,6.006)--(8.234,6.002)--(8.228,5.998)%
  --(8.221,5.995)--(8.221,5.994)--(8.214,5.991)--(8.208,5.987)--(8.201,5.987)--(8.208,5.987)%
  --(8.208,5.982)--(8.201,5.985)--(8.201,5.982)--(8.201,5.981)--(8.201,5.980)--(8.194,5.978)%
  --(8.194,5.979)--(8.194,5.978)--(8.194,5.976)--(8.188,5.974)--(8.188,5.973)--(8.188,5.967)%
  --(8.188,5.965)--(8.181,5.964)--(8.181,5.965)--(8.174,5.963)--(8.168,5.959)--(8.168,5.956)%
  --(8.161,5.953)--(8.161,5.945)--(8.161,5.943)--(8.154,5.941)--(8.154,5.938)--(8.148,5.939)%
  --(8.141,5.936)--(8.141,5.935)--(8.141,5.931)--(8.128,5.924)--(8.128,5.918)--(8.121,5.916)%
  --(8.121,5.914)--(8.121,5.915)--(8.115,5.917)--(8.108,5.916)--(8.101,5.913)--(8.101,5.912)%
  --(8.095,5.909)--(8.095,5.899)--(8.088,5.894)--(8.075,5.890)--(8.075,5.888)--(8.068,5.889)%
  --(8.068,5.887)--(8.062,5.885)--(8.062,5.881)--(8.055,5.877)--(8.055,5.872)--(8.042,5.868)%
  --(8.035,5.866)--(8.028,5.863)--(8.035,5.862)--(8.028,5.861)--(8.028,5.859)--(8.022,5.857)%
  --(8.015,5.854)--(8.008,5.851)--(8.002,5.847)--(8.002,5.839)--(7.995,5.838)--(7.995,5.836)%
  --(7.988,5.833)--(7.988,5.830)--(7.982,5.828)--(7.982,5.825)--(7.975,5.823)--(7.975,5.816)%
  --(7.969,5.815)--(7.962,5.816)--(7.969,5.815)--(7.962,5.813)--(7.962,5.811)--(7.955,5.808)%
  --(7.955,5.802)--(7.955,5.800)--(7.949,5.798)--(7.949,5.797)--(7.935,5.795)--(7.935,5.796)%
  --(7.935,5.795)--(7.935,5.794)--(7.929,5.792)--(7.929,5.790)--(7.929,5.786)--(7.929,5.784)%
  --(7.929,5.783)--(7.922,5.783)--(7.922,5.784)--(7.915,5.785)--(7.922,5.783)--(7.915,5.781)%
  --(7.909,5.779)--(7.909,5.774)--(7.902,5.769)--(7.909,5.767)--(7.902,5.764)--(7.895,5.762)%
  --(7.895,5.764)--(7.889,5.760)--(7.882,5.757)--(7.876,5.756)--(7.876,5.752)--(7.876,5.751)%
  --(7.869,5.749)--(7.869,5.746)--(7.862,5.740)--(7.856,5.738)--(7.849,5.737)--(7.849,5.735)%
  --(7.842,5.731)--(7.836,5.728)--(7.836,5.725)--(7.829,5.718)--(7.816,5.712)--(7.816,5.708)%
  --(7.809,5.706)--(7.809,5.709)--(7.809,5.708)--(7.802,5.706)--(7.802,5.704)--(7.802,5.698)%
  --(7.789,5.690)--(7.783,5.686)--(7.783,5.684)--(7.776,5.683)--(7.776,5.684)--(7.769,5.682)%
  --(7.769,5.681)--(7.763,5.680)--(7.756,5.670)--(7.749,5.667)--(7.743,5.663)--(7.743,5.662)%
  --(7.736,5.661)--(7.736,5.660)--(7.743,5.659)--(7.736,5.659)--(7.736,5.655)--(7.736,5.653)%
  --(7.736,5.651)--(7.729,5.653)--(7.723,5.651)--(7.716,5.649)--(7.723,5.648)--(7.716,5.649)%
  --(7.716,5.648)--(7.716,5.647)--(7.716,5.644)--(7.709,5.643)--(7.709,5.642)--(7.703,5.634)%
  --(7.703,5.637)--(7.696,5.635)--(7.696,5.634)--(7.690,5.630)--(7.690,5.628)--(7.690,5.626)%
  --(7.690,5.625)--(7.690,5.623)--(7.683,5.617)--(7.683,5.618)--(7.683,5.621)--(7.683,5.620)%
  --(7.676,5.618)--(7.676,5.615)--(7.670,5.613)--(7.663,5.607)--(7.663,5.604)--(7.663,5.601)%
  --(7.656,5.599)--(7.656,5.601)--(7.656,5.602)--(7.656,5.601)--(7.656,5.599)--(7.650,5.597)%
  --(7.643,5.591)--(7.643,5.586)--(7.630,5.582)--(7.623,5.577)--(7.623,5.574)--(7.623,5.576)%
  --(7.616,5.575)--(7.616,5.574)--(7.616,5.572)--(7.610,5.566)--(7.610,5.570)--(7.610,5.569)%
  --(7.603,5.567)--(7.603,5.565)--(7.597,5.562)--(7.590,5.560)--(7.590,5.558)--(7.583,5.554)%
  --(7.577,5.548)--(7.570,5.543)--(7.570,5.540)--(7.563,5.537)--(7.563,5.532)--(7.563,5.535)%
  --(7.557,5.534)--(7.563,5.533)--(7.557,5.534)--(7.563,5.534)--(7.563,5.533)--(7.557,5.530)%
  --(7.557,5.524)--(7.543,5.523)--(7.543,5.524)--(7.543,5.521)--(7.537,5.518)--(7.537,5.516)%
  --(7.537,5.515)--(7.530,5.507)--(7.530,5.505)--(7.524,5.505)--(7.517,5.507)--(7.517,5.505)%
  --(7.510,5.503)--(7.504,5.498)--(7.497,5.493)--(7.490,5.486)--(7.484,5.480)--(7.484,5.478)%
  --(7.477,5.477)--(7.477,5.475)--(7.477,5.479)--(7.477,5.477)--(7.470,5.475)--(7.470,5.472)%
  --(7.464,5.471)--(7.457,5.463)--(7.457,5.467)--(7.457,5.465)--(7.457,5.463)--(7.450,5.463)%
  --(7.450,5.460)--(7.444,5.458)--(7.444,5.454)--(7.444,5.452)--(7.437,5.449)--(7.431,5.446)%
  --(7.431,5.442)--(7.424,5.435)--(7.424,5.439)--(7.424,5.437)--(7.417,5.435)--(7.411,5.431)%
  --(7.404,5.425)--(7.391,5.415)--(7.371,5.406)--(7.364,5.397)--(7.357,5.393)--(7.351,5.390)%
  --(7.344,5.386)--(7.344,5.388)--(7.344,5.386)--(7.338,5.379)--(7.338,5.377)--(7.324,5.372)%
  --(7.311,5.363)--(7.304,5.358)--(7.298,5.356)--(7.291,5.353)--(7.291,5.351)--(7.291,5.347)%
  --(7.291,5.344)--(7.284,5.334)--(7.271,5.335)--(7.264,5.332)--(7.258,5.330)--(7.258,5.326)%
  --(7.251,5.321)--(7.251,5.320)--(7.251,5.317)--(7.251,5.314)--(7.245,5.307)--(7.238,5.311)%
  --(7.231,5.309)--(7.225,5.306)--(7.218,5.303)--(7.211,5.299)--(7.211,5.295)--(7.211,5.292)%
  --(7.205,5.292)--(7.205,5.288)--(7.198,5.286)--(7.198,5.282)--(7.198,5.280)--(7.191,5.275)%
  --(7.185,5.271)--(7.185,5.272)--(7.178,5.270)--(7.178,5.268)--(7.178,5.265)--(7.171,5.261)%
  --(7.171,5.259)--(7.158,5.258)--(7.158,5.254)--(7.152,5.250)--(7.152,5.247)--(7.132,5.239)%
  --(7.132,5.233)--(7.125,5.224)--(7.118,5.227)--(7.125,5.227)--(7.118,5.226)--(7.112,5.224)%
  --(7.105,5.218)--(7.098,5.212)--(7.092,5.208)--(7.092,5.204)--(7.085,5.202)--(7.085,5.199)%
  --(7.079,5.194)--(7.072,5.194)--(7.072,5.189)--(7.065,5.185)--(7.059,5.181)--(7.052,5.178)%
  --(7.045,5.175)--(7.045,5.173)--(7.039,5.165)--(7.039,5.169)--(7.032,5.166)--(7.025,5.163)%
  --(7.025,5.161)--(7.019,5.157)--(7.019,5.155)--(7.012,5.152)--(6.999,5.148)--(6.999,5.144)%
  --(6.992,5.140)--(6.986,5.138)--(6.986,5.133)--(6.972,5.126)--(6.966,5.118)--(6.966,5.116)%
  --(6.959,5.114)--(6.952,5.111)--(6.952,5.110)--(6.946,5.103)--(6.939,5.101)--(6.939,5.096)%
  --(6.932,5.092)--(6.926,5.091)--(6.919,5.087)--(6.919,5.086)--(6.919,5.083)--(6.912,5.080)%
  --(6.912,5.077)--(6.899,5.067)--(6.893,5.064)--(6.893,5.059)--(6.879,5.054)--(6.873,5.052)%
  --(6.873,5.050)--(6.859,5.045)--(6.846,5.039)--(6.846,5.033)--(6.839,5.026)--(6.833,5.020)%
  --(6.826,5.015)--(6.813,5.008)--(6.800,5.001)--(6.793,4.996)--(6.786,4.992)--(6.786,4.990)%
  --(6.780,4.986)--(6.780,4.983)--(6.766,4.975)--(6.766,4.976)--(6.760,4.972)--(6.760,4.971)%
  --(6.760,4.970)--(6.753,4.967)--(6.753,4.966)--(6.746,4.962)--(6.740,4.959)--(6.733,4.954)%
  --(6.726,4.950)--(6.720,4.944)--(6.720,4.939)--(6.707,4.935)--(6.700,4.930)--(6.700,4.925)%
  --(6.693,4.921)--(6.680,4.918)--(6.680,4.913)--(6.673,4.910)--(6.673,4.905)--(6.667,4.898)%
  --(6.660,4.896)--(6.653,4.894)--(6.647,4.890)--(6.640,4.886)--(6.634,4.879)--(6.634,4.874)%
  --(6.627,4.869)--(6.620,4.864)--(6.614,4.861)--(6.607,4.858)--(6.600,4.855)--(6.594,4.850)%
  --(6.587,4.847)--(6.580,4.844)--(6.580,4.838)--(6.574,4.833)--(6.560,4.829)--(6.560,4.824)%
  --(6.547,4.818)--(6.541,4.814)--(6.534,4.809)--(6.527,4.804)--(6.514,4.800)--(6.507,4.790)%
  --(6.501,4.787)--(6.501,4.785)--(6.494,4.783)--(6.487,4.778)--(6.481,4.768)--(6.474,4.765)%
  --(6.461,4.758)--(6.454,4.753)--(6.448,4.746)--(6.434,4.739)--(6.428,4.734)--(6.421,4.731)%
  --(6.421,4.728)--(6.408,4.722)--(6.388,4.708)--(6.374,4.700)--(6.368,4.692)--(6.361,4.686)%
  --(6.341,4.671)--(6.335,4.669)--(6.328,4.663)--(6.321,4.658)--(6.321,4.656)--(6.315,4.655)%
  --(6.315,4.652)--(6.315,4.648)--(6.301,4.642)--(6.301,4.636)--(6.295,4.634)--(6.288,4.632)%
  --(6.288,4.630)--(6.275,4.628)--(6.275,4.627)--(6.275,4.626)--(6.275,4.627)--(6.275,4.625)%
  --(6.275,4.622)--(6.275,4.619)--(6.275,4.617)--(6.275,4.620)--(6.268,4.619)--(6.268,4.618)%
  --(6.268,4.617)--(6.262,4.614)--(6.248,4.606)--(6.235,4.597)--(6.235,4.591)--(6.228,4.586)%
  --(6.228,4.581)--(6.215,4.581)--(6.208,4.578)--(6.208,4.574)--(6.208,4.573)--(6.208,4.572)%
  --(6.208,4.571)--(6.208,4.572)--(6.202,4.568)--(6.195,4.565)--(6.188,4.563)--(6.188,4.559)%
  --(6.182,4.557)--(6.182,4.554)--(6.169,4.549)--(6.155,4.543)--(6.149,4.537)--(6.142,4.532)%
  --(6.135,4.525)--(6.129,4.521)--(6.122,4.517)--(6.115,4.510)--(6.109,4.505)--(6.096,4.499)%
  --(6.082,4.491)--(6.082,4.484)--(6.062,4.477)--(6.056,4.468)--(6.042,4.461)--(6.036,4.456)%
  --(6.022,4.450)--(6.016,4.447)--(6.009,4.443)--(6.009,4.440)--(6.009,4.436)--(6.009,4.432)%
  --(6.003,4.426)--(5.996,4.424)--(5.989,4.423)--(5.983,4.419)--(5.969,4.415)--(5.969,4.411)%
  --(5.969,4.408)--(5.963,4.405)--(5.963,4.400)--(5.963,4.396)--(5.956,4.394)--(5.949,4.391)%
  --(5.943,4.387)--(5.936,4.385)--(5.929,4.381)--(5.923,4.377)--(5.916,4.373)--(5.910,4.366)%
  --(5.903,4.361)--(5.896,4.359)--(5.890,4.355)--(5.890,4.353)--(5.883,4.350)--(5.883,4.347)%
  --(5.876,4.344)--(5.870,4.339)--(5.863,4.335)--(5.863,4.329)--(5.856,4.326)--(5.850,4.322)%
  --(5.843,4.317)--(5.836,4.313)--(5.830,4.312)--(5.830,4.308)--(5.823,4.305)--(5.817,4.300)%
  --(5.810,4.292)--(5.797,4.283)--(5.790,4.277)--(5.783,4.273)--(5.770,4.267)--(5.763,4.262)%
  --(5.750,4.255)--(5.743,4.248)--(5.737,4.242)--(5.724,4.237)--(5.717,4.229)--(5.717,4.226)%
  --(5.717,4.224)--(5.710,4.223)--(5.704,4.222)--(5.704,4.218)--(5.690,4.210)--(5.677,4.203)%
  --(5.664,4.193)--(5.651,4.185)--(5.644,4.179)--(5.631,4.172)--(5.617,4.164)--(5.611,4.156)%
  --(5.604,4.149)--(5.591,4.140)--(5.577,4.131)--(5.571,4.127)--(5.564,4.120)--(5.558,4.108)%
  --(5.538,4.099)--(5.518,4.090)--(5.504,4.080)--(5.498,4.076)--(5.491,4.068)--(5.491,4.066)%
  --(5.491,4.062)--(5.484,4.057)--(5.478,4.054)--(5.471,4.044)--(5.451,4.035)--(5.445,4.027)%
  --(5.431,4.021)--(5.418,4.016)--(5.418,4.013)--(5.418,4.011)--(5.411,4.009)--(5.411,4.006)%
  --(5.398,3.996)--(5.391,3.992)--(5.378,3.986)--(5.372,3.982)--(5.372,3.978)--(5.358,3.971)%
  --(5.345,3.962)--(5.332,3.953)--(5.325,3.947)--(5.312,3.935)--(5.305,3.932)--(5.305,3.928)%
  --(5.298,3.925)--(5.292,3.921)--(5.279,3.913)--(5.272,3.909)--(5.272,3.903)--(5.259,3.899)%
  --(5.259,3.893)--(5.252,3.892)--(5.245,3.890)--(5.245,3.886)--(5.239,3.882)--(5.232,3.877)%
  --(5.225,3.873)--(5.225,3.867)--(5.212,3.865)--(5.212,3.859)--(5.205,3.857)--(5.199,3.855)%
  --(5.192,3.851)--(5.186,3.846)--(5.179,3.842)--(5.179,3.841)--(5.172,3.840)--(5.166,3.837)%
  --(5.159,3.824)--(5.152,3.820)--(5.146,3.816)--(5.132,3.813)--(5.126,3.807)--(5.113,3.797)%
  --(5.099,3.789)--(5.086,3.783)--(5.079,3.777)--(5.073,3.771)--(5.073,3.767)--(5.066,3.764)%
  --(5.066,3.762)--(5.066,3.759)--(5.059,3.754)--(5.053,3.750)--(5.039,3.741)--(5.033,3.736)%
  --(5.020,3.728)--(5.013,3.723)--(5.006,3.717)--(5.000,3.711)--(4.993,3.707)--(4.993,3.704)%
  --(4.986,3.702)--(4.986,3.700)--(4.980,3.697)--(4.980,3.693)--(4.980,3.694)--(4.980,3.693)%
  --(4.973,3.691)--(4.966,3.689)--(4.966,3.686)--(4.960,3.682)--(4.953,3.681)--(4.946,3.677)%
  --(4.940,3.672)--(4.933,3.664)--(4.927,3.662)--(4.920,3.658)--(4.913,3.651)--(4.900,3.643)%
  --(4.893,3.639)--(4.893,3.636)--(4.887,3.633)--(4.880,3.627)--(4.873,3.621)--(4.867,3.616)%
  --(4.853,3.611)--(4.847,3.605)--(4.840,3.602)--(4.834,3.598)--(4.827,3.590)--(4.827,3.585)%
  --(4.827,3.584)--(4.820,3.582)--(4.814,3.573)--(4.800,3.566)--(4.780,3.554)--(4.774,3.546)%
  --(4.767,3.542)--(4.760,3.541)--(4.760,3.537)--(4.754,3.533)--(4.754,3.529)--(4.747,3.526)%
  --(4.741,3.520)--(4.734,3.515)--(4.727,3.512)--(4.727,3.510)--(4.721,3.509)--(4.714,3.505)%
  --(4.714,3.501)--(4.714,3.500)--(4.707,3.498)--(4.701,3.493)--(4.694,3.488)--(4.681,3.478)%
  --(4.674,3.477)--(4.681,3.478)--(4.674,3.477)--(4.674,3.476)--(4.668,3.472)--(4.661,3.465)%
  --(4.648,3.459)--(4.648,3.455)--(4.641,3.454)--(4.634,3.450)--(4.621,3.440)--(4.614,3.436)%
  --(4.608,3.433)--(4.608,3.430)--(4.594,3.423)--(4.594,3.418)--(4.588,3.412)--(4.588,3.411)%
  --(4.588,3.410)--(4.581,3.408)--(4.581,3.406)--(4.581,3.404)--(4.575,3.399)--(4.568,3.394)%
  --(4.561,3.390)--(4.555,3.386)--(4.548,3.381)--(4.535,3.376)--(4.535,3.370)--(4.528,3.366)%
  --(4.521,3.363)--(4.515,3.358)--(4.508,3.354)--(4.508,3.348)--(4.501,3.346)--(4.495,3.344)%
  --(4.488,3.341)--(4.482,3.338)--(4.482,3.334)--(4.468,3.330)--(4.462,3.322)--(4.455,3.316)%
  --(4.448,3.315)--(4.448,3.313)--(4.442,3.311)--(4.435,3.308)--(4.428,3.302)--(4.422,3.295)%
  --(4.415,3.291)--(4.408,3.286)--(4.402,3.281)--(4.395,3.277)--(4.389,3.272)--(4.382,3.270)%
  --(4.382,3.268)--(4.382,3.266)--(4.375,3.263)--(4.369,3.259)--(4.362,3.254)--(4.355,3.246)%
  --(4.349,3.242)--(4.342,3.238)--(4.342,3.235)--(4.335,3.231)--(4.329,3.226)--(4.322,3.221)%
  --(4.309,3.212)--(4.296,3.202)--(4.289,3.194)--(4.276,3.189)--(4.276,3.185)--(4.276,3.184)%
  --(4.269,3.181)--(4.256,3.176)--(4.249,3.173)--(4.242,3.168)--(4.236,3.160)--(4.236,3.157)%
  --(4.229,3.156)--(4.229,3.155)--(4.222,3.151)--(4.216,3.147)--(4.216,3.145)--(4.209,3.142)%
  --(4.203,3.137)--(4.196,3.134)--(4.196,3.132)--(4.189,3.130)--(4.183,3.127)--(4.176,3.118)%
  --(4.163,3.112)--(4.163,3.109)--(4.156,3.108)--(4.156,3.107)--(4.156,3.106)--(4.149,3.101)%
  --(4.143,3.096)--(4.136,3.089)--(4.130,3.088)--(4.123,3.085)--(4.116,3.080)--(4.116,3.077)%
  --(4.110,3.072)--(4.110,3.069)--(4.103,3.066)--(4.096,3.062)--(4.090,3.057)--(4.090,3.051)%
  --(4.090,3.052)--(4.083,3.050)--(4.083,3.047)--(4.070,3.045)--(4.063,3.039)--(4.056,3.029)%
  --(4.050,3.025)--(4.043,3.021)--(4.043,3.018)--(4.037,3.014)--(4.037,3.009)--(4.037,3.010)%
  --(4.030,3.009)--(4.023,3.007)--(4.023,3.004)--(4.017,3.001)--(4.017,2.999)--(4.017,2.997)%
  --(4.010,2.994)--(4.003,2.993)--(4.003,2.991)--(4.003,2.990)--(4.003,2.988)--(4.003,2.986)%
  --(3.997,2.986)--(3.997,2.985)--(3.990,2.984)--(3.990,2.981)--(3.983,2.980)--(3.983,2.977)%
  --(3.977,2.975)--(3.970,2.971)--(3.970,2.967)--(3.963,2.963)--(3.957,2.959)--(3.950,2.955)%
  --(3.944,2.953)--(3.937,2.949)--(3.930,2.944)--(3.917,2.937)--(3.910,2.931)--(3.910,2.928)%
  --(3.910,2.926)--(3.910,2.925)--(3.904,2.923)--(3.890,2.917)--(3.877,2.909)--(3.870,2.899)%
  --(3.864,2.893)--(3.857,2.890)--(3.857,2.888)--(3.851,2.884)--(3.851,2.883)--(3.851,2.881)%
  --(3.851,2.880)--(3.844,2.877)--(3.844,2.874)--(3.837,2.871)--(3.831,2.870)--(3.824,2.866)%
  --(3.824,2.862)--(3.817,2.859)--(3.817,2.856)--(3.811,2.852)--(3.804,2.847)--(3.804,2.845)%
  --(3.797,2.843)--(3.797,2.842)--(3.797,2.839)--(3.791,2.837)--(3.784,2.835)--(3.784,2.832)%
  --(3.777,2.828)--(3.771,2.825)--(3.771,2.823)--(3.771,2.820)--(3.771,2.817)--(3.764,2.815)%
  --(3.758,2.814)--(3.758,2.812)--(3.751,2.810)--(3.751,2.806)--(3.744,2.803)--(3.738,2.800)%
  --(3.731,2.799)--(3.731,2.796)--(3.724,2.791)--(3.718,2.787)--(3.711,2.782)--(3.704,2.778)%
  --(3.704,2.776)--(3.698,2.773)--(3.691,2.772)--(3.685,2.768)--(3.685,2.763)--(3.678,2.758)%
  --(3.678,2.756)--(3.671,2.753)--(3.665,2.749)--(3.658,2.744)--(3.651,2.740)--(3.651,2.738)%
  --(3.645,2.735)--(3.645,2.732)--(3.631,2.725)--(3.631,2.722)--(3.631,2.721)--(3.625,2.717)%
  --(3.625,2.716)--(3.625,2.715)--(3.618,2.712)--(3.611,2.708)--(3.611,2.706)--(3.611,2.703)%
  --(3.611,2.699)--(3.605,2.698)--(3.598,2.692)--(3.585,2.686)--(3.578,2.684)--(3.578,2.682)%
  --(3.572,2.679)--(3.565,2.675)--(3.558,2.670)--(3.552,2.662)--(3.545,2.657)--(3.538,2.654)%
  --(3.538,2.652)--(3.538,2.651)--(3.532,2.647)--(3.525,2.643)--(3.525,2.642)--(3.525,2.641)%
  --(3.525,2.640)--(3.525,2.638)--(3.518,2.637)--(3.512,2.633)--(3.505,2.629)--(3.499,2.625)%
  --(3.492,2.620)--(3.485,2.617)--(3.485,2.615)--(3.479,2.613)--(3.479,2.610)--(3.472,2.605)%
  --(3.465,2.604)--(3.465,2.601)--(3.459,2.600)--(3.452,2.596)--(3.439,2.589)--(3.432,2.582)%
  --(3.425,2.577)--(3.419,2.573)--(3.419,2.571)--(3.419,2.572)--(3.419,2.570)--(3.412,2.566)%
  --(3.412,2.563)--(3.406,2.560)--(3.399,2.557)--(3.399,2.554)--(3.392,2.553)--(3.392,2.551)%
  --(3.392,2.550)--(3.386,2.547)--(3.386,2.544)--(3.379,2.539)--(3.372,2.536)--(3.372,2.532)%
  --(3.366,2.530)--(3.366,2.527)--(3.366,2.526)--(3.359,2.522)--(3.359,2.519)--(3.352,2.517)%
  --(3.352,2.513)--(3.346,2.510)--(3.332,2.508)--(3.332,2.503)--(3.326,2.500)--(3.326,2.499)%
  --(3.326,2.497)--(3.326,2.494)--(3.319,2.492)--(3.319,2.490)--(3.313,2.488)--(3.306,2.485)%
  --(3.299,2.483)--(3.299,2.480)--(3.293,2.476)--(3.286,2.473)--(3.279,2.469)--(3.273,2.463)%
  --(3.266,2.459)--(3.266,2.456)--(3.259,2.451)--(3.253,2.448)--(3.253,2.446)--(3.246,2.445)%
  --(3.240,2.440)--(3.240,2.435)--(3.240,2.434)--(3.233,2.432)--(3.226,2.429)--(3.220,2.427)%
  --(3.220,2.424)--(3.213,2.422)--(3.213,2.420)--(3.206,2.419)--(3.206,2.415)--(3.206,2.414)%
  --(3.200,2.410)--(3.193,2.407)--(3.186,2.406)--(3.186,2.402)--(3.180,2.398)--(3.173,2.396)%
  --(3.173,2.394)--(3.173,2.392)--(3.166,2.391)--(3.166,2.387)--(3.160,2.383)--(3.153,2.378)%
  --(3.153,2.377)--(3.147,2.373)--(3.140,2.369)--(3.140,2.364)--(3.133,2.361)--(3.127,2.357)%
  --(3.120,2.353)--(3.113,2.349)--(3.113,2.347)--(3.113,2.346)--(3.107,2.343)--(3.107,2.341)%
  --(3.100,2.337)--(3.093,2.332)--(3.093,2.330)--(3.087,2.327)--(3.087,2.326)--(3.080,2.323)%
  --(3.073,2.321)--(3.073,2.318)--(3.067,2.313)--(3.060,2.310)--(3.060,2.309)--(3.054,2.304)%
  --(3.047,2.299)--(3.040,2.296)--(3.040,2.294)--(3.034,2.291)--(3.027,2.285)--(3.020,2.281)%
  --(3.014,2.276)--(3.014,2.274)--(3.007,2.270)--(3.000,2.263)--(2.994,2.258)--(2.987,2.257)%
  --(2.980,2.255)--(2.974,2.253)--(2.967,2.248)--(2.961,2.245)--(2.961,2.243)--(2.954,2.237)%
  --(2.947,2.234)--(2.947,2.233)--(2.947,2.232)--(2.947,2.230)--(2.941,2.227)--(2.934,2.220)%
  --(2.927,2.216)--(2.921,2.213)--(2.914,2.210)--(2.907,2.206)--(2.901,2.200)--(2.901,2.196)%
  --(2.894,2.192)--(2.887,2.188)--(2.887,2.184)--(2.881,2.180)--(2.881,2.179)--(2.881,2.177)%
  --(2.874,2.174)--(2.868,2.173)--(2.861,2.166)--(2.854,2.162)--(2.854,2.160)--(2.848,2.157)%
  --(2.848,2.153)--(2.834,2.147)--(2.828,2.142)--(2.828,2.139)--(2.821,2.137)--(2.821,2.134)%
  --(2.814,2.132)--(2.814,2.131)--(2.808,2.131)--(2.808,2.128)--(2.801,2.126)--(2.801,2.123)%
  --(2.794,2.119)--(2.788,2.114)--(2.788,2.112)--(2.781,2.109)--(2.775,2.104)--(2.768,2.100)%
  --(2.768,2.096)--(2.761,2.095)--(2.761,2.094)--(2.755,2.091)--(2.748,2.088)--(2.741,2.085)%
  --(2.741,2.082)--(2.735,2.078)--(2.728,2.074)--(2.721,2.070)--(2.715,2.063)--(2.708,2.059)%
  --(2.702,2.056)--(2.695,2.054)--(2.695,2.049)--(2.688,2.046)--(2.682,2.040)--(2.675,2.035)%
  --(2.668,2.030)--(2.662,2.027)--(2.655,2.022)--(2.648,2.017)--(2.635,2.007)--(2.635,2.003)%
  --(2.635,2.002)--(2.628,1.998)--(2.628,1.996)--(2.628,1.995)--(2.628,1.993)--(2.622,1.992)%
  --(2.622,1.989)--(2.615,1.986)--(2.609,1.981)--(2.602,1.979)--(2.595,1.973)--(2.589,1.968)%
  --(2.582,1.964)--(2.575,1.960)--(2.569,1.956)--(2.569,1.951)--(2.562,1.947)--(2.555,1.943)%
  --(2.549,1.937)--(2.542,1.933)--(2.535,1.930)--(2.535,1.928)--(2.529,1.927)--(2.529,1.925)%
  --(2.522,1.922)--(2.516,1.917)--(2.509,1.914)--(2.509,1.911)--(2.509,1.910)--(2.509,1.908)%
  --(2.502,1.905)--(2.496,1.904)--(2.496,1.902)--(2.496,1.900)--(2.489,1.896)--(2.482,1.891)%
  --(2.482,1.890)--(2.476,1.886)--(2.469,1.884)--(2.462,1.881)--(2.456,1.879)--(2.456,1.877)%
  --(2.456,1.876)--(2.449,1.874)--(2.449,1.871)--(2.442,1.866)--(2.436,1.863)--(2.436,1.861)%
  --(2.436,1.859)--(2.429,1.856)--(2.423,1.851)--(2.416,1.845)--(2.416,1.840)--(2.409,1.837)%
  --(2.403,1.831)--(2.396,1.828)--(2.389,1.826)--(2.389,1.824)--(2.383,1.821)--(2.376,1.815)%
  --(2.369,1.810)--(2.369,1.808)--(2.363,1.803)--(2.356,1.798)--(2.349,1.792)--(2.336,1.785)%
  --(2.330,1.780)--(2.330,1.778)--(2.323,1.775)--(2.316,1.775)--(2.316,1.774)--(2.316,1.770)%
  --(2.310,1.767)--(2.310,1.762)--(2.303,1.759)--(2.296,1.753)--(2.290,1.747)--(2.283,1.743)%
  --(2.276,1.737)--(2.263,1.732)--(2.257,1.729)--(2.257,1.727);
\gpcolor{color=gp lt color 3}
\gpsetlinetype{gp lt plot 3}
\draw[gp path] (9.948,7.388)--(9.941,7.386)--(9.935,7.381)--(9.935,7.377)--(9.928,7.372)%
  --(9.928,7.370)--(9.928,7.366)--(9.921,7.365)--(9.915,7.364)--(9.908,7.361)--(9.915,7.360)%
  --(9.915,7.358)--(9.901,7.355)--(9.901,7.351)--(9.895,7.347)--(9.895,7.345)--(9.895,7.337)%
  --(9.888,7.335)--(9.881,7.332)--(9.888,7.332)--(9.881,7.337)--(9.875,7.333)--(9.868,7.328)%
  --(9.868,7.323)--(9.848,7.318)--(9.848,7.307)--(9.842,7.301)--(9.842,7.296)--(9.835,7.295)%
  --(9.828,7.293)--(9.828,7.296)--(9.828,7.293)--(9.828,7.290)--(9.828,7.288)--(9.822,7.285)%
  --(9.822,7.278)--(9.808,7.277)--(9.808,7.276)--(9.802,7.274)--(9.795,7.277)--(9.788,7.276)%
  --(9.782,7.272)--(9.775,7.267)--(9.768,7.263)--(9.762,7.255)--(9.762,7.250)--(9.755,7.249)%
  --(9.749,7.244)--(9.749,7.241)--(9.749,7.244)--(9.742,7.239)--(9.742,7.235)--(9.729,7.231)%
  --(9.735,7.228)--(9.715,7.222)--(9.709,7.217)--(9.709,7.213)--(9.709,7.207)--(9.702,7.201)%
  --(9.695,7.198)--(9.695,7.195)--(9.689,7.193)--(9.682,7.191)--(9.689,7.186)--(9.682,7.182)%
  --(9.675,7.180)--(9.675,7.179)--(9.675,7.178)--(9.669,7.183)--(9.662,7.182)--(9.662,7.180)%
  --(9.662,7.178)--(9.656,7.173)--(9.656,7.167)--(9.649,7.156)--(9.642,7.151)--(9.636,7.147)%
  --(9.636,7.148)--(9.622,7.147)--(9.622,7.145)--(9.616,7.143)--(9.622,7.142)--(9.616,7.137)%
  --(9.609,7.133)--(9.609,7.130)--(9.602,7.128)--(9.602,7.132)--(9.596,7.129)--(9.582,7.124)%
  --(9.589,7.119)--(9.582,7.117)--(9.576,7.114)--(9.576,7.105)--(9.576,7.103)--(9.563,7.097)%
  --(9.556,7.094)--(9.549,7.095)--(9.543,7.093)--(9.543,7.091)--(9.536,7.089)--(9.529,7.085)%
  --(9.523,7.082)--(9.516,7.077)--(9.509,7.071)--(9.509,7.065)--(9.503,7.063)--(9.490,7.059)%
  --(9.490,7.056)--(9.483,7.049)--(9.483,7.044)--(9.476,7.035)--(9.476,7.031)--(9.470,7.029)%
  --(9.470,7.027)--(9.470,7.029)--(9.463,7.028)--(9.450,7.026)--(9.450,7.024)--(9.443,7.021)%
  --(9.443,7.014)--(9.436,7.006)--(9.436,7.003)--(9.430,7.000)--(9.430,6.997)--(9.423,6.999)%
  --(9.423,6.995)--(9.410,6.991)--(9.403,6.987)--(9.397,6.982)--(9.390,6.969)--(9.383,6.965)%
  --(9.370,6.962)--(9.363,6.957)--(9.357,6.953)--(9.350,6.947)--(9.350,6.943)--(9.343,6.940)%
  --(9.337,6.937)--(9.330,6.929)--(9.317,6.921)--(9.310,6.915)--(9.304,6.912)--(9.304,6.910)%
  --(9.304,6.915)--(9.297,6.912)--(9.290,6.909)--(9.284,6.904)--(9.277,6.900)--(9.270,6.898)%
  --(9.264,6.896)--(9.257,6.891)--(9.250,6.883)--(9.244,6.877)--(9.244,6.870)--(9.237,6.867)%
  --(9.237,6.864)--(9.230,6.862)--(9.230,6.857)--(9.224,6.849)--(9.224,6.845)--(9.217,6.843)%
  --(9.217,6.840)--(9.211,6.842)--(9.197,6.837)--(9.191,6.832)--(9.184,6.825)--(9.184,6.822)%
  --(9.184,6.813)--(9.171,6.812)--(9.184,6.817)--(9.177,6.818)--(9.177,6.816)--(9.164,6.813)%
  --(9.157,6.807)--(9.151,6.801)--(9.151,6.793)--(9.144,6.791)--(9.137,6.789)--(9.137,6.785)%
  --(9.118,6.780)--(9.118,6.775)--(9.104,6.768)--(9.104,6.766)--(9.104,6.764)--(9.098,6.763)%
  --(9.091,6.748)--(9.078,6.752)--(9.071,6.748)--(9.064,6.745)--(9.058,6.743)--(9.064,6.739)%
  --(9.058,6.736)--(9.051,6.733)--(9.038,6.728)--(9.031,6.720)--(9.018,6.713)--(9.011,6.706)%
  --(8.998,6.696)--(8.991,6.685)--(8.985,6.679)--(8.978,6.672)--(8.971,6.666)--(8.965,6.663)%
  --(8.958,6.658)--(8.952,6.647)--(8.945,6.647)--(8.932,6.642)--(8.918,6.636)--(8.918,6.626)%
  --(8.905,6.620)--(8.898,6.614)--(8.892,6.608)--(8.878,6.603)--(8.878,6.600)--(8.878,6.595)%
  --(8.878,6.599)--(8.872,6.597)--(8.872,6.595)--(8.872,6.590)--(8.865,6.587)--(8.859,6.583)%
  --(8.845,6.577)--(8.839,6.574)--(8.839,6.564)--(8.832,6.571)--(8.832,6.570)--(8.825,6.567)%
  --(8.819,6.563)--(8.812,6.558)--(8.805,6.556)--(8.805,6.552)--(8.792,6.547)--(8.785,6.542)%
  --(8.779,6.535)--(8.779,6.534)--(8.772,6.528)--(8.759,6.520)--(8.752,6.511)--(8.752,6.506)%
  --(8.746,6.502)--(8.732,6.497)--(8.732,6.492)--(8.726,6.479)--(8.719,6.483)--(8.712,6.476)%
  --(8.699,6.470)--(8.692,6.463)--(8.679,6.454)--(8.659,6.445)--(8.653,6.435)--(8.646,6.429)%
  --(8.639,6.421)--(8.619,6.410)--(8.613,6.411)--(8.606,6.405)--(8.599,6.400)--(8.586,6.396)%
  --(8.580,6.394)--(8.580,6.388)--(8.573,6.383)--(8.573,6.379)--(8.566,6.376)--(8.560,6.366)%
  --(8.546,6.370)--(8.540,6.365)--(8.540,6.364)--(8.546,6.363)--(8.540,6.359)--(8.533,6.353)%
  --(8.526,6.347)--(8.513,6.339)--(8.507,6.327)--(8.507,6.324)--(8.513,6.322)--(8.507,6.326)%
  --(8.507,6.325)--(8.500,6.324)--(8.500,6.322)--(8.500,6.321)--(8.493,6.314)--(8.493,6.311)%
  --(8.493,6.308)--(8.487,6.305)--(8.480,6.304)--(8.480,6.307)--(8.473,6.305)--(8.473,6.302)%
  --(8.473,6.301)--(8.467,6.299)--(8.467,6.289)--(8.460,6.290)--(8.460,6.292)--(8.447,6.287)%
  --(8.433,6.280)--(8.427,6.274)--(8.420,6.269)--(8.420,6.262)--(8.407,6.257)--(8.394,6.251)%
  --(8.394,6.246)--(8.380,6.239)--(8.374,6.232)--(8.367,6.226)--(8.354,6.221)--(8.347,6.215)%
  --(8.334,6.212)--(8.327,6.207)--(8.327,6.204)--(8.321,6.194)--(8.314,6.190)--(8.307,6.185)%
  --(8.307,6.184)--(8.301,6.187)--(8.301,6.188)--(8.301,6.186)--(8.301,6.182)--(8.294,6.176)%
  --(8.294,6.171)--(8.287,6.171)--(8.281,6.166)--(8.274,6.163)--(8.267,6.158)--(8.267,6.156)%
  --(8.267,6.154)--(8.261,6.150)--(8.254,6.146)--(8.254,6.134)--(8.247,6.131)--(8.241,6.128)%
  --(8.228,6.126)--(8.228,6.125)--(8.228,6.124)--(8.221,6.122)--(8.221,6.117)--(8.214,6.114)%
  --(8.208,6.111)--(8.201,6.107)--(8.194,6.103)--(8.181,6.096)--(8.174,6.089)--(8.168,6.084)%
  --(8.161,6.077)--(8.154,6.071)--(8.135,6.064)--(8.128,6.059)--(8.121,6.050)--(8.115,6.044)%
  --(8.101,6.036)--(8.095,6.029)--(8.088,6.028)--(8.081,6.026)--(8.075,6.023)--(8.068,6.020)%
  --(8.068,6.019)--(8.068,6.015)--(8.068,6.011)--(8.068,6.010)--(8.062,6.005)--(8.055,5.999)%
  --(8.055,6.000)--(8.048,5.997)--(8.042,5.992)--(8.035,5.989)--(8.035,5.986)--(8.028,5.978)%
  --(8.035,5.977)--(8.028,5.978)--(8.022,5.976)--(8.015,5.973)--(8.015,5.968)--(8.002,5.964)%
  --(8.002,5.958)--(7.995,5.953)--(7.988,5.947)--(7.988,5.942)--(7.982,5.941)--(7.982,5.938)%
  --(7.975,5.936)--(7.969,5.940)--(7.969,5.935)--(7.962,5.931)--(7.949,5.924)--(7.935,5.918)%
  --(7.935,5.917)--(7.935,5.914)--(7.929,5.913)--(7.929,5.909)--(7.922,5.906)--(7.922,5.900)%
  --(7.909,5.897)--(7.895,5.892)--(7.889,5.885)--(7.882,5.876)--(7.876,5.866)--(7.876,5.863)%
  --(7.869,5.862)--(7.862,5.859)--(7.856,5.860)--(7.849,5.857)--(7.842,5.850)--(7.836,5.845)%
  --(7.829,5.842)--(7.816,5.831)--(7.816,5.826)--(7.816,5.825)--(7.809,5.822)--(7.809,5.828)%
  --(7.802,5.820)--(7.796,5.816)--(7.789,5.811)--(7.783,5.806)--(7.776,5.802)--(7.776,5.795)%
  --(7.756,5.793)--(7.756,5.789)--(7.756,5.787)--(7.749,5.783)--(7.743,5.775)--(7.736,5.770)%
  --(7.729,5.767)--(7.723,5.762)--(7.716,5.751)--(7.709,5.752)--(7.709,5.751)--(7.696,5.750)%
  --(7.696,5.746)--(7.690,5.741)--(7.683,5.737)--(7.683,5.734)--(7.670,5.730)--(7.656,5.722)%
  --(7.650,5.712)--(7.630,5.708)--(7.630,5.703)--(7.623,5.691)--(7.603,5.686)--(7.597,5.683)%
  --(7.597,5.681)--(7.590,5.675)--(7.583,5.672)--(7.570,5.656)--(7.570,5.659)--(7.563,5.655)%
  --(7.557,5.649)--(7.543,5.643)--(7.537,5.638)--(7.524,5.628)--(7.517,5.621)--(7.510,5.619)%
  --(7.510,5.616)--(7.510,5.610)--(7.504,5.608)--(7.497,5.603)--(7.490,5.601)--(7.490,5.599)%
  --(7.477,5.594)--(7.470,5.588)--(7.464,5.585)--(7.464,5.580)--(7.450,5.566)--(7.437,5.567)%
  --(7.424,5.562)--(7.417,5.553)--(7.411,5.542)--(7.397,5.537)--(7.397,5.536)--(7.397,5.535)%
  --(7.391,5.532)--(7.384,5.526)--(7.377,5.516)--(7.371,5.513)--(7.357,5.510)--(7.357,5.508)%
  --(7.351,5.504)--(7.344,5.499)--(7.338,5.496)--(7.338,5.490)--(7.324,5.483)--(7.318,5.480)%
  --(7.318,5.477)--(7.311,5.470)--(7.298,5.465)--(7.291,5.462)--(7.284,5.455)--(7.278,5.449)%
  --(7.271,5.445)--(7.264,5.442)--(7.264,5.439)--(7.264,5.432)--(7.264,5.436)--(7.264,5.435)%
  --(7.258,5.434)--(7.258,5.431)--(7.251,5.429)--(7.245,5.426)--(7.245,5.423)--(7.238,5.419)%
  --(7.231,5.414)--(7.231,5.410)--(7.218,5.406)--(7.211,5.402)--(7.211,5.398)--(7.205,5.401)%
  --(7.205,5.398)--(7.198,5.393)--(7.191,5.388)--(7.185,5.385)--(7.178,5.376)--(7.171,5.370)%
  --(7.158,5.370)--(7.158,5.365)--(7.152,5.359)--(7.138,5.353)--(7.138,5.352)--(7.138,5.350)%
  --(7.138,5.348)--(7.132,5.344)--(7.125,5.337)--(7.125,5.333)--(7.112,5.329)--(7.105,5.326)%
  --(7.098,5.321)--(7.098,5.315)--(7.092,5.312)--(7.079,5.309)--(7.072,5.302)--(7.065,5.297)%
  --(7.065,5.296)--(7.065,5.294)--(7.065,5.301)--(7.065,5.297)--(7.065,5.296)--(7.052,5.289)%
  --(7.045,5.284)--(7.045,5.277)--(7.039,5.272)--(7.032,5.268)--(7.025,5.267)--(7.019,5.262)%
  --(7.019,5.259)--(7.012,5.255)--(6.999,5.251)--(6.999,5.246)--(6.979,5.234)--(6.966,5.224)%
  --(6.952,5.214)--(6.946,5.206)--(6.939,5.205)--(6.939,5.203)--(6.932,5.200)--(6.926,5.195)%
  --(6.919,5.188)--(6.906,5.178)--(6.899,5.175)--(6.893,5.174)--(6.886,5.168)--(6.879,5.162)%
  --(6.873,5.157)--(6.866,5.153)--(6.859,5.151)--(6.853,5.145)--(6.846,5.139)--(6.846,5.134)%
  --(6.846,5.132)--(6.839,5.127)--(6.839,5.126)--(6.833,5.129)--(6.833,5.125)--(6.826,5.123)%
  --(6.826,5.120)--(6.826,5.118)--(6.826,5.113)--(6.819,5.111)--(6.819,5.110)--(6.819,5.113)%
  --(6.813,5.111)--(6.806,5.110)--(6.806,5.108)--(6.806,5.106)--(6.800,5.105)--(6.800,5.100)%
  --(6.800,5.097)--(6.800,5.092)--(6.793,5.090)--(6.786,5.092)--(6.786,5.090)--(6.780,5.087)%
  --(6.780,5.084)--(6.773,5.083)--(6.773,5.078)--(6.773,5.077)--(6.773,5.076)--(6.773,5.075)%
  --(6.773,5.077)--(6.773,5.078)--(6.766,5.073)--(6.753,5.070)--(6.746,5.064)--(6.707,5.040)%
  --(6.693,5.031)--(6.693,5.027)--(6.680,5.022)--(6.680,5.019)--(6.673,5.013)--(6.667,5.009)%
  --(6.660,5.005)--(6.653,4.999)--(6.647,4.991)--(6.640,4.986)--(6.634,4.981)--(6.620,4.976)%
  --(6.607,4.970)--(6.607,4.966)--(6.600,4.960)--(6.600,4.957)--(6.594,4.954)--(6.594,4.951)%
  --(6.594,4.950)--(6.594,4.954)--(6.594,4.952)--(6.594,4.950)--(6.594,4.947)--(6.587,4.945)%
  --(6.594,4.945)--(6.587,4.948)--(6.587,4.947)--(6.587,4.944)--(6.587,4.945)--(6.587,4.943)%
  --(6.580,4.940)--(6.580,4.937)--(6.574,4.934)--(6.567,4.930)--(6.567,4.929)--(6.560,4.924)%
  --(6.560,4.923)--(6.554,4.921)--(6.554,4.920)--(6.547,4.919)--(6.541,4.916)--(6.541,4.914)%
  --(6.534,4.911)--(6.527,4.905)--(6.527,4.898)--(6.521,4.895)--(6.514,4.894)--(6.507,4.891)%
  --(6.507,4.888)--(6.501,4.886)--(6.501,4.883)--(6.494,4.878)--(6.487,4.874)--(6.487,4.869)%
  --(6.481,4.867)--(6.474,4.859)--(6.474,4.860)--(6.467,4.857)--(6.461,4.854)--(6.454,4.850)%
  --(6.448,4.846)--(6.434,4.839)--(6.408,4.816)--(6.381,4.804)--(6.368,4.795)--(6.368,4.793)%
  --(6.368,4.791)--(6.374,4.790)--(6.374,4.789)--(6.374,4.788)--(6.374,4.786)--(6.374,4.787)%
  --(6.374,4.788)--(6.374,4.783)--(6.374,4.788)--(6.374,4.787)--(6.368,4.785)--(6.368,4.783)%
  --(6.361,4.780)--(6.355,4.776)--(6.355,4.770)--(6.341,4.767)--(6.335,4.762)--(6.328,4.755)%
  --(6.321,4.751)--(6.315,4.747)--(6.315,4.744)--(6.315,4.743)--(6.315,4.741)--(6.308,4.739)%
  --(6.301,4.735)--(6.301,4.734)--(6.295,4.731)--(6.295,4.729)--(6.295,4.732)--(6.295,4.730)%
  --(6.288,4.727)--(6.281,4.725)--(6.281,4.724)--(6.281,4.723)--(6.275,4.718)--(6.281,4.717)%
  --(6.275,4.716)--(6.275,4.715)--(6.262,4.709)--(6.255,4.703)--(6.248,4.700)--(6.248,4.696)%
  --(6.242,4.690)--(6.242,4.688)--(6.235,4.683)--(6.222,4.676)--(6.215,4.670)--(6.208,4.668)%
  --(6.195,4.665)--(6.188,4.659)--(6.182,4.654)--(6.175,4.649)--(6.175,4.642)--(6.169,4.637)%
  --(6.162,4.633)--(6.155,4.631)--(6.149,4.630)--(6.142,4.626)--(6.142,4.620)--(6.135,4.616)%
  --(6.129,4.611)--(6.122,4.606)--(6.109,4.596)--(6.096,4.585)--(6.089,4.580)--(6.089,4.579)%
  --(6.089,4.581)--(6.089,4.580)--(6.082,4.575)--(6.076,4.572)--(6.069,4.568)--(6.062,4.564)%
  --(6.056,4.558)--(6.049,4.551)--(6.036,4.544)--(6.029,4.541)--(6.029,4.540)--(6.022,4.536)%
  --(6.022,4.534)--(6.009,4.529)--(6.003,4.519)--(5.996,4.510)--(5.989,4.504)--(5.983,4.502)%
  --(5.969,4.500)--(5.963,4.493)--(5.949,4.488)--(5.943,4.484)--(5.936,4.479)--(5.936,4.473)%
  --(5.936,4.471)--(5.936,4.470)--(5.929,4.468)--(5.929,4.466)--(5.923,4.463)--(5.916,4.460)%
  --(5.910,4.455)--(5.910,4.452)--(5.903,4.447)--(5.903,4.445)--(5.903,4.443)--(5.896,4.442)%
  --(5.890,4.438)--(5.890,4.437)--(5.890,4.434)--(5.883,4.431)--(5.883,4.429)--(5.883,4.424)%
  --(5.876,4.424)--(5.876,4.423)--(5.870,4.421)--(5.863,4.419)--(5.856,4.418)--(5.856,4.415)%
  --(5.850,4.413)--(5.850,4.410)--(5.843,4.406)--(5.836,4.402)--(5.830,4.396)--(5.830,4.391)%
  --(5.823,4.391)--(5.817,4.388)--(5.810,4.384)--(5.810,4.380)--(5.803,4.377)--(5.803,4.378)%
  --(5.803,4.373)--(5.797,4.372)--(5.790,4.363)--(5.783,4.362)--(5.777,4.360)--(5.770,4.356)%
  --(5.763,4.350)--(5.757,4.346)--(5.757,4.339)--(5.750,4.337)--(5.750,4.335)--(5.743,4.333)%
  --(5.743,4.334)--(5.737,4.332)--(5.737,4.328)--(5.730,4.326)--(5.724,4.322)--(5.717,4.313)%
  --(5.717,4.310)--(5.710,4.308)--(5.704,4.307)--(5.697,4.302)--(5.697,4.299)--(5.690,4.296)%
  --(5.684,4.289)--(5.677,4.282)--(5.670,4.277)--(5.664,4.274)--(5.657,4.270)--(5.651,4.264)%
  --(5.644,4.259)--(5.637,4.256)--(5.637,4.253)--(5.637,4.252)--(5.631,4.248)--(5.631,4.243)%
  --(5.624,4.244)--(5.617,4.240)--(5.611,4.238)--(5.611,4.235)--(5.611,4.231)--(5.597,4.228)%
  --(5.591,4.224)--(5.591,4.220)--(5.584,4.216)--(5.584,4.215)--(5.577,4.210)--(5.577,4.212)%
  --(5.577,4.210)--(5.577,4.209)--(5.571,4.206)--(5.564,4.204)--(5.564,4.201)--(5.558,4.194)%
  --(5.551,4.187)--(5.544,4.187)--(5.538,4.183)--(5.531,4.178)--(5.531,4.176)--(5.531,4.175)%
  --(5.524,4.171)--(5.511,4.163)--(5.504,4.158)--(5.504,4.155)--(5.504,4.151)--(5.504,4.153)%
  --(5.504,4.151)--(5.504,4.150)--(5.498,4.148)--(5.498,4.146)--(5.491,4.145)--(5.491,4.144)%
  --(5.491,4.141)--(5.484,4.135)--(5.478,4.133)--(5.471,4.131)--(5.471,4.129)--(5.458,4.127)%
  --(5.458,4.122)--(5.451,4.118)--(5.451,4.117)--(5.451,4.114)--(5.451,4.111)--(5.445,4.102)%
  --(5.438,4.099)--(5.425,4.098)--(5.425,4.095)--(5.418,4.090)--(5.418,4.089)--(5.418,4.087)%
  --(5.411,4.087)--(5.411,4.086)--(5.411,4.085)--(5.405,4.084)--(5.405,4.079)--(5.405,4.080)%
  --(5.398,4.077)--(5.398,4.076)--(5.398,4.073)--(5.398,4.071)--(5.391,4.071)--(5.391,4.069)%
  --(5.391,4.063)--(5.385,4.065)--(5.385,4.062)--(5.378,4.058)--(5.378,4.057)--(5.372,4.054)%
  --(5.372,4.053)--(5.365,4.049)--(5.358,4.047)--(5.358,4.044)--(5.358,4.041)--(5.358,4.040)%
  --(5.352,4.039)--(5.352,4.036)--(5.345,4.034)--(5.345,4.031)--(5.338,4.029)--(5.332,4.025)%
  --(5.332,4.023)--(5.325,4.017)--(5.318,4.016)--(5.318,4.013)--(5.318,4.011)--(5.312,4.011)%
  --(5.305,4.008)--(5.305,4.006)--(5.305,4.003)--(5.305,4.001)--(5.298,3.998)--(5.292,3.997)%
  --(5.292,3.993)--(5.285,3.993)--(5.279,3.990)--(5.272,3.986)--(5.265,3.982)--(5.259,3.978)%
  --(5.259,3.976)--(5.259,3.971)--(5.252,3.969)--(5.245,3.970)--(5.245,3.965)--(5.245,3.964)%
  --(5.239,3.962)--(5.239,3.959)--(5.232,3.954)--(5.225,3.951)--(5.225,3.949)--(5.225,3.946)%
  --(5.212,3.944)--(5.205,3.937)--(5.199,3.932)--(5.192,3.928)--(5.186,3.925)--(5.192,3.922)%
  --(5.186,3.921)--(5.186,3.922)--(5.179,3.919)--(5.179,3.918)--(5.179,3.914)--(5.172,3.912)%
  --(5.172,3.910)--(5.166,3.906)--(5.166,3.903)--(5.159,3.900)--(5.159,3.896)--(5.152,3.898)%
  --(5.159,3.898)--(5.159,3.899)--(5.159,3.898)--(5.152,3.897)--(5.152,3.896)--(5.152,3.894)%
  --(5.139,3.888)--(5.132,3.881)--(5.132,3.877)--(5.119,3.876)--(5.119,3.871)--(5.119,3.868)%
  --(5.113,3.866)--(5.106,3.863)--(5.106,3.862)--(5.106,3.857)--(5.093,3.853)--(5.086,3.852)%
  --(5.086,3.849)--(5.086,3.847)--(5.079,3.844)--(5.073,3.839)--(5.066,3.838)--(5.059,3.835)%
  --(5.059,3.826)--(5.046,3.821)--(5.039,3.816)--(5.026,3.810)--(5.013,3.802)--(5.006,3.797)%
  --(5.000,3.788)--(5.000,3.787)--(5.000,3.785)--(5.000,3.786)--(4.993,3.784)--(4.986,3.783)%
  --(4.986,3.780)--(4.986,3.777)--(4.980,3.772)--(4.973,3.766)--(4.966,3.760)--(4.953,3.753)%
  --(4.946,3.749)--(4.946,3.747)--(4.946,3.746)--(4.940,3.742)--(4.933,3.737)--(4.927,3.735)%
  --(4.927,3.734)--(4.927,3.728)--(4.920,3.726)--(4.920,3.723)--(4.913,3.722)--(4.913,3.720)%
  --(4.913,3.719)--(4.907,3.718)--(4.907,3.715)--(4.900,3.712)--(4.900,3.709)--(4.900,3.710)%
  --(4.900,3.709)--(4.893,3.707)--(4.893,3.705)--(4.887,3.702)--(4.887,3.699)--(4.887,3.696)%
  --(4.880,3.696)--(4.880,3.694)--(4.873,3.691)--(4.867,3.688)--(4.860,3.684)--(4.853,3.681)%
  --(4.853,3.679)--(4.847,3.678)--(4.847,3.674)--(4.834,3.670)--(4.827,3.665)--(4.827,3.662)%
  --(4.820,3.659)--(4.814,3.654)--(4.807,3.647)--(4.800,3.640)--(4.794,3.638)--(4.794,3.636)%
  --(4.794,3.633)--(4.794,3.634)--(4.787,3.632)--(4.787,3.631)--(4.780,3.626)--(4.774,3.622)%
  --(4.774,3.621)--(4.774,3.620)--(4.767,3.618)--(4.760,3.615)--(4.760,3.612)--(4.754,3.610)%
  --(4.754,3.607)--(4.747,3.605)--(4.741,3.602)--(4.734,3.598)--(4.734,3.594)--(4.727,3.589)%
  --(4.727,3.587)--(4.727,3.585)--(4.727,3.583)--(4.727,3.580)--(4.721,3.578)--(4.714,3.575)%
  --(4.714,3.570)--(4.701,3.566)--(4.694,3.558)--(4.694,3.556)--(4.687,3.556)--(4.687,3.555)%
  --(4.687,3.552)--(4.681,3.549)--(4.681,3.547)--(4.674,3.545)--(4.674,3.544)--(4.674,3.541)%
  --(4.668,3.536)--(4.661,3.533)--(4.654,3.530)--(4.654,3.528)--(4.654,3.525)--(4.648,3.524)%
  --(4.648,3.523)--(4.648,3.518)--(4.648,3.520)--(4.648,3.519)--(4.648,3.518)--(4.641,3.518)%
  --(4.641,3.515)--(4.634,3.513)--(4.634,3.511)--(4.634,3.510)--(4.628,3.507)--(4.628,3.502)%
  --(4.621,3.501)--(4.614,3.499)--(4.614,3.496)--(4.608,3.493)--(4.601,3.491)--(4.594,3.488)%
  --(4.588,3.481)--(4.581,3.479)--(4.581,3.477)--(4.581,3.476)--(4.575,3.473)--(4.575,3.468)%
  --(4.568,3.466)--(4.561,3.463)--(4.555,3.457)--(4.548,3.457)--(4.548,3.454)--(4.548,3.451)%
  --(4.541,3.450)--(4.535,3.445)--(4.528,3.441)--(4.528,3.436)--(4.521,3.435)--(4.521,3.433)%
  --(4.515,3.431)--(4.508,3.428)--(4.508,3.424)--(4.501,3.419)--(4.495,3.413)--(4.488,3.409)%
  --(4.482,3.403)--(4.475,3.400)--(4.468,3.393)--(4.462,3.389)--(4.455,3.387)--(4.448,3.384)%
  --(4.448,3.379)--(4.442,3.374)--(4.435,3.368)--(4.422,3.362)--(4.422,3.357)--(4.415,3.357)%
  --(4.408,3.351)--(4.402,3.347)--(4.402,3.343)--(4.395,3.338)--(4.389,3.334)--(4.382,3.331)%
  --(4.375,3.325)--(4.362,3.320)--(4.355,3.315)--(4.349,3.313)--(4.349,3.311)--(4.342,3.309)%
  --(4.342,3.306)--(4.335,3.305)--(4.335,3.300)--(4.322,3.294)--(4.315,3.289)--(4.315,3.284)%
  --(4.302,3.278)--(4.302,3.276)--(4.296,3.275)--(4.289,3.272)--(4.289,3.267)--(4.276,3.263)%
  --(4.269,3.259)--(4.269,3.256)--(4.269,3.254)--(4.262,3.251)--(4.256,3.246)--(4.249,3.244)%
  --(4.249,3.243)--(4.249,3.241)--(4.242,3.238)--(4.242,3.235)--(4.236,3.230)--(4.229,3.225)%
  --(4.222,3.219)--(4.216,3.214)--(4.209,3.213)--(4.209,3.210)--(4.209,3.206)--(4.203,3.201)%
  --(4.196,3.196)--(4.189,3.193)--(4.189,3.187)--(4.183,3.185)--(4.183,3.181)--(4.176,3.181)%
  --(4.176,3.179)--(4.169,3.177)--(4.169,3.176)--(4.169,3.175)--(4.169,3.174)--(4.163,3.171)%
  --(4.163,3.170)--(4.156,3.167)--(4.156,3.162)--(4.149,3.159)--(4.143,3.157)--(4.143,3.155)%
  --(4.136,3.152)--(4.130,3.148)--(4.130,3.145)--(4.123,3.140)--(4.123,3.138)--(4.116,3.134)%
  --(4.103,3.130)--(4.090,3.123)--(4.076,3.111)--(4.063,3.104)--(4.063,3.101)--(4.063,3.100)%
  --(4.056,3.097)--(4.056,3.098)--(4.056,3.095)--(4.050,3.092)--(4.043,3.091)--(4.043,3.088)%
  --(4.037,3.087)--(4.037,3.086)--(4.037,3.083)--(4.037,3.085)--(4.037,3.083)--(4.030,3.081)%
  --(4.030,3.077)--(4.023,3.073)--(4.023,3.069)--(4.017,3.065)--(4.010,3.063)--(4.003,3.061)%
  --(4.003,3.060)--(4.003,3.055)--(3.997,3.051)--(3.990,3.049)--(3.983,3.046)--(3.983,3.045)%
  --(3.977,3.043)--(3.977,3.041)--(3.970,3.038)--(3.970,3.031)--(3.963,3.028)--(3.963,3.026)%
  --(3.957,3.023)--(3.957,3.022)--(3.950,3.018)--(3.950,3.013)--(3.944,3.009)--(3.944,3.007)%
  --(3.937,3.004)--(3.930,3.000)--(3.924,2.995)--(3.917,2.991)--(3.910,2.988)--(3.904,2.984)%
  --(3.904,2.981)--(3.897,2.978)--(3.890,2.976)--(3.884,2.972)--(3.877,2.967)--(3.877,2.964)%
  --(3.870,2.958)--(3.857,2.952)--(3.851,2.948)--(3.837,2.944)--(3.837,2.939)--(3.831,2.935)%
  --(3.817,2.926)--(3.811,2.918)--(3.797,2.914)--(3.791,2.905)--(3.777,2.894)--(3.777,2.892)%
  --(3.771,2.891)--(3.771,2.888)--(3.764,2.885)--(3.764,2.882)--(3.764,2.878)--(3.758,2.874)%
  --(3.751,2.871)--(3.744,2.868)--(3.744,2.867)--(3.744,2.864)--(3.738,2.860)--(3.731,2.858)%
  --(3.724,2.853)--(3.724,2.847)--(3.718,2.846)--(3.718,2.843)--(3.718,2.842)--(3.711,2.839)%
  --(3.704,2.836)--(3.704,2.834)--(3.704,2.833)--(3.698,2.833)--(3.698,2.831)--(3.698,2.828)%
  --(3.698,2.823)--(3.685,2.820)--(3.685,2.815)--(3.678,2.814)--(3.678,2.812)--(3.678,2.810)%
  --(3.671,2.809)--(3.671,2.807)--(3.671,2.806)--(3.665,2.803)--(3.665,2.801)--(3.658,2.801)%
  --(3.658,2.800)--(3.651,2.798)--(3.651,2.795)--(3.638,2.788)--(3.631,2.781)--(3.625,2.776)%
  --(3.618,2.774)--(3.611,2.773)--(3.605,2.768)--(3.605,2.763)--(3.598,2.759)--(3.592,2.754)%
  --(3.585,2.750)--(3.578,2.749)--(3.578,2.746)--(3.578,2.744)--(3.578,2.742)--(3.572,2.739)%
  --(3.565,2.735)--(3.558,2.730)--(3.552,2.725)--(3.545,2.720)--(3.545,2.717)--(3.545,2.715)%
  --(3.538,2.714)--(3.538,2.712)--(3.532,2.706)--(3.518,2.696)--(3.512,2.691)--(3.512,2.689)%
  --(3.505,2.685)--(3.499,2.680)--(3.492,2.679)--(3.485,2.673)--(3.479,2.666)--(3.472,2.661)%
  --(3.465,2.656)--(3.459,2.652)--(3.452,2.645)--(3.439,2.638)--(3.425,2.632)--(3.425,2.629)%
  --(3.419,2.625)--(3.419,2.623)--(3.419,2.622)--(3.419,2.624)--(3.419,2.623)--(3.406,2.615)%
  --(3.399,2.609)--(3.399,2.607)--(3.399,2.605)--(3.386,2.600)--(3.379,2.598)--(3.372,2.596)%
  --(3.372,2.592)--(3.366,2.590)--(3.366,2.587)--(3.359,2.584)--(3.359,2.579)--(3.346,2.573)%
  --(3.339,2.568)--(3.339,2.564)--(3.332,2.564)--(3.326,2.560)--(3.326,2.557)--(3.319,2.554)%
  --(3.313,2.551)--(3.306,2.546)--(3.306,2.540)--(3.299,2.536)--(3.293,2.533)--(3.286,2.529)%
  --(3.286,2.524)--(3.279,2.519)--(3.273,2.515)--(3.266,2.512)--(3.266,2.509)--(3.259,2.506)%
  --(3.253,2.500)--(3.246,2.496)--(3.246,2.493)--(3.240,2.491)--(3.240,2.489)--(3.240,2.488)%
  --(3.233,2.487)--(3.233,2.485)--(3.226,2.480)--(3.220,2.476)--(3.213,2.473)--(3.213,2.470)%
  --(3.200,2.465)--(3.200,2.461)--(3.193,2.459)--(3.193,2.454)--(3.180,2.448)--(3.180,2.445)%
  --(3.180,2.444)--(3.180,2.443)--(3.173,2.439)--(3.173,2.435)--(3.166,2.435)--(3.166,2.434)%
  --(3.160,2.430)--(3.153,2.427)--(3.153,2.426)--(3.147,2.423)--(3.140,2.420)--(3.133,2.416)%
  --(3.127,2.411)--(3.120,2.405)--(3.113,2.401)--(3.107,2.398)--(3.107,2.396)--(3.100,2.393)%
  --(3.087,2.385)--(3.080,2.379)--(3.080,2.377)--(3.073,2.374)--(3.073,2.371)--(3.067,2.366)%
  --(3.060,2.362)--(3.054,2.358)--(3.047,2.353)--(3.047,2.350)--(3.040,2.346)--(3.034,2.341)%
  --(3.027,2.334)--(3.020,2.331)--(3.020,2.327)--(3.014,2.326)--(3.007,2.323)--(3.007,2.320)%
  --(3.007,2.318)--(3.000,2.314)--(2.994,2.309)--(2.987,2.306)--(2.987,2.301)--(2.980,2.298)%
  --(2.974,2.294)--(2.967,2.291)--(2.967,2.288)--(2.961,2.284)--(2.954,2.280)--(2.947,2.275)%
  --(2.941,2.271)--(2.941,2.269)--(2.934,2.266)--(2.934,2.262)--(2.927,2.261)--(2.927,2.260)%
  --(2.927,2.258)--(2.921,2.258)--(2.921,2.256)--(2.914,2.253)--(2.907,2.250)--(2.901,2.248)%
  --(2.901,2.245)--(2.894,2.243)--(2.894,2.241)--(2.887,2.237)--(2.881,2.232)--(2.868,2.224)%
  --(2.868,2.218)--(2.861,2.216)--(2.861,2.215)--(2.861,2.214)--(2.861,2.212)--(2.854,2.210)%
  --(2.848,2.206)--(2.841,2.201)--(2.841,2.198)--(2.834,2.196)--(2.834,2.193)--(2.828,2.187)%
  --(2.821,2.183)--(2.821,2.180)--(2.814,2.177)--(2.808,2.174)--(2.801,2.169)--(2.794,2.165)%
  --(2.794,2.162)--(2.781,2.156)--(2.775,2.149)--(2.768,2.146)--(2.768,2.141)--(2.761,2.141)%
  --(2.761,2.138)--(2.761,2.137)--(2.755,2.133)--(2.748,2.132)--(2.741,2.128)--(2.741,2.125)%
  --(2.735,2.121)--(2.721,2.113)--(2.715,2.109)--(2.715,2.106)--(2.715,2.104)--(2.715,2.103)%
  --(2.708,2.100)--(2.702,2.097)--(2.702,2.094)--(2.695,2.092)--(2.695,2.090)--(2.688,2.087)%
  --(2.682,2.084)--(2.675,2.080)--(2.668,2.076)--(2.662,2.073)--(2.662,2.071)--(2.655,2.067)%
  --(2.655,2.063)--(2.648,2.058)--(2.635,2.054)--(2.628,2.048)--(2.628,2.044)--(2.622,2.041)%
  --(2.615,2.037)--(2.609,2.032)--(2.602,2.028)--(2.602,2.024)--(2.595,2.021)--(2.595,2.019)%
  --(2.589,2.015)--(2.582,2.005)--(2.575,2.001)--(2.575,1.999)--(2.569,1.995)--(2.555,1.989)%
  --(2.549,1.986)--(2.549,1.984)--(2.549,1.979)--(2.542,1.975)--(2.535,1.971)--(2.522,1.966)%
  --(2.522,1.964)--(2.516,1.959)--(2.516,1.956)--(2.509,1.951)--(2.502,1.947)--(2.502,1.945)%
  --(2.496,1.942)--(2.496,1.940)--(2.489,1.936)--(2.482,1.933)--(2.482,1.930)--(2.482,1.929)%
  --(2.476,1.928)--(2.469,1.925)--(2.462,1.921)--(2.456,1.913)--(2.449,1.910)--(2.449,1.908)%
  --(2.442,1.905)--(2.442,1.903)--(2.436,1.897)--(2.429,1.892)--(2.429,1.891)--(2.423,1.888)%
  --(2.416,1.886)--(2.409,1.883)--(2.409,1.882)--(2.403,1.880)--(2.396,1.876)--(2.396,1.873)%
  --(2.389,1.871)--(2.383,1.866)--(2.383,1.863)--(2.376,1.860)--(2.376,1.859)--(2.369,1.854)%
  --(2.369,1.851)--(2.369,1.849)--(2.363,1.846)--(2.363,1.843)--(2.363,1.841)--(2.356,1.840)%
  --(2.349,1.834)--(2.343,1.829)--(2.343,1.828)--(2.343,1.827)--(2.336,1.825)--(2.330,1.819)%
  --(2.323,1.812)--(2.316,1.808)--(2.316,1.807)--(2.310,1.803)--(2.303,1.800)--(2.303,1.799)%
  --(2.303,1.798)--(2.296,1.794)--(2.290,1.790)--(2.290,1.789)--(2.290,1.786)--(2.283,1.783)%
  --(2.276,1.780)--(2.276,1.779)--(2.270,1.776)--(2.270,1.774)--(2.263,1.770)--(2.257,1.766)%
  --(2.257,1.762)--(2.250,1.760)--(2.243,1.758)--(2.243,1.753)--(2.237,1.748)--(2.230,1.745)%
  --(2.230,1.740)--(2.223,1.735)--(2.217,1.733)--(2.217,1.732);
\gpcolor{color=gp lt color 4}
\gpsetlinetype{gp lt plot 4}
\draw[gp path] (9.782,7.408)--(9.775,7.405)--(9.768,7.399)--(9.755,7.394)--(9.755,7.391)%
  --(9.755,7.389)--(9.749,7.388)--(9.749,7.377)--(9.742,7.376)--(9.735,7.371)--(9.735,7.372)%
  --(9.735,7.369)--(9.722,7.366)--(9.709,7.362)--(9.709,7.347)--(9.702,7.352)--(9.702,7.350)%
  --(9.695,7.347)--(9.689,7.344)--(9.689,7.339)--(9.682,7.337)--(9.682,7.335)--(9.675,7.333)%
  --(9.669,7.327)--(9.662,7.325)--(9.662,7.323)--(9.656,7.319)--(9.649,7.312)--(9.649,7.307)%
  --(9.642,7.300)--(9.629,7.293)--(9.622,7.290)--(9.616,7.287)--(9.616,7.277)--(9.609,7.282)%
  --(9.609,7.280)--(9.602,7.279)--(9.602,7.278)--(9.602,7.277)--(9.602,7.275)--(9.596,7.274)%
  --(9.596,7.266)--(9.582,7.261)--(9.576,7.260)--(9.576,7.256)--(9.576,7.253)--(9.576,7.251)%
  --(9.563,7.247)--(9.556,7.245)--(9.556,7.239)--(9.549,7.235)--(9.543,7.233)--(9.536,7.230)%
  --(9.529,7.227)--(9.529,7.217)--(9.529,7.220)--(9.523,7.214)--(9.516,7.209)--(9.509,7.202)%
  --(9.509,7.200)--(9.509,7.195)--(9.509,7.196)--(9.496,7.196)--(9.496,7.195)--(9.490,7.192)%
  --(9.483,7.189)--(9.476,7.185)--(9.470,7.179)--(9.450,7.167)--(9.443,7.156)--(9.443,7.146)%
  --(9.430,7.143)--(9.430,7.146)--(9.430,7.144)--(9.430,7.143)--(9.423,7.140)--(9.410,7.134)%
  --(9.403,7.132)--(9.397,7.126)--(9.397,7.119)--(9.390,7.109)--(9.383,7.109)--(9.377,7.109)%
  --(9.377,7.106)--(9.363,7.103)--(9.363,7.100)--(9.363,7.097)--(9.363,7.095)--(9.350,7.092)%
  --(9.350,7.084)--(9.350,7.081)--(9.343,7.083)--(9.343,7.080)--(9.337,7.074)--(9.323,7.068)%
  --(9.323,7.065)--(9.317,7.063)--(9.317,7.059)--(9.310,7.052)--(9.304,7.043)--(9.297,7.039)%
  --(9.297,7.037)--(9.297,7.038)--(9.290,7.040)--(9.284,7.035)--(9.277,7.031)--(9.277,7.030)%
  --(9.277,7.029)--(9.270,7.024)--(9.270,7.021)--(9.270,7.019)--(9.270,7.021)--(9.270,7.023)%
  --(9.264,7.022)--(9.264,7.020)--(9.264,7.017)--(9.264,7.016)--(9.250,7.010)--(9.250,7.005)%
  --(9.250,7.004)--(9.244,7.003)--(9.244,7.006)--(9.237,7.008)--(9.237,7.007)--(9.237,7.006)%
  --(9.237,7.002)--(9.237,6.995)--(9.230,6.994)--(9.224,6.993)--(9.230,6.995)--(9.224,6.994)%
  --(9.224,6.992)--(9.224,6.990)--(9.217,6.989)--(9.217,6.983)--(9.217,6.982)--(9.217,6.986)%
  --(9.211,6.984)--(9.204,6.982)--(9.191,6.971)--(9.184,6.963)--(9.177,6.961)--(9.177,6.960)%
  --(9.177,6.954)--(9.171,6.944)--(9.157,6.939)--(9.151,6.934)--(9.151,6.931)--(9.144,6.930)%
  --(9.137,6.926)--(9.131,6.922)--(9.118,6.919)--(9.118,6.918)--(9.118,6.911)--(9.111,6.908)%
  --(9.111,6.905)--(9.111,6.902)--(9.104,6.906)--(9.104,6.904)--(9.104,6.901)--(9.098,6.898)%
  --(9.091,6.896)--(9.084,6.891)--(9.084,6.888)--(9.078,6.884)--(9.071,6.881)--(9.064,6.878)%
  --(9.064,6.871)--(9.064,6.868)--(9.058,6.864)--(9.051,6.860)--(9.045,6.856)--(9.038,6.843)%
  --(9.031,6.850)--(9.031,6.847)--(9.025,6.845)--(9.025,6.842)--(9.018,6.837)--(9.018,6.836)%
  --(9.011,6.834)--(9.011,6.831)--(9.005,6.826)--(9.005,6.821)--(8.998,6.825)--(8.991,6.823)%
  --(8.998,6.822)--(8.991,6.820)--(8.952,6.793)--(8.932,6.777)--(8.918,6.771)--(8.912,6.758)%
  --(8.905,6.761)--(8.898,6.753)--(8.885,6.748)--(8.885,6.743)--(8.878,6.740)--(8.878,6.739)%
  --(8.878,6.737)--(8.872,6.734)--(8.872,6.729)--(8.872,6.726)--(8.872,6.728)--(8.872,6.726)%
  --(8.872,6.725)--(8.865,6.723)--(8.865,6.722)--(8.865,6.721)--(8.865,6.720)--(8.865,6.719)%
  --(8.859,6.706)--(8.799,6.679)--(8.799,6.677)--(8.799,6.675)--(8.799,6.674)--(8.805,6.673)%
  --(8.799,6.672)--(8.799,6.671)--(8.799,6.670)--(8.799,6.669)--(8.799,6.664)--(8.799,6.670)%
  --(8.799,6.669)--(8.792,6.669)--(8.792,6.668)--(8.792,6.666)--(8.779,6.659)--(8.766,6.652)%
  --(8.759,6.642)--(8.752,6.642)--(8.746,6.640)--(8.732,6.633)--(8.732,6.628)--(8.699,6.610)%
  --(8.666,6.583)--(8.666,6.582)--(8.666,6.579)--(8.673,6.578)--(8.673,6.577)--(8.666,6.577)%
  --(8.666,6.573)--(8.673,6.574)--(8.666,6.576)--(8.666,6.575)--(8.673,6.575)--(8.659,6.574)%
  --(8.666,6.569)--(8.666,6.567)--(8.666,6.571)--(8.666,6.569)--(8.659,6.567)--(8.653,6.563)%
  --(8.653,6.564)--(8.646,6.558)--(8.646,6.557)--(8.639,6.555)--(8.639,6.550)--(8.633,6.548)%
  --(8.619,6.535)--(8.606,6.525)--(8.599,6.519)--(8.586,6.515)--(8.580,6.509)--(8.573,6.504)%
  --(8.566,6.499)--(8.560,6.492)--(8.546,6.483)--(8.546,6.480)--(8.540,6.481)--(8.533,6.478)%
  --(8.526,6.474)--(8.526,6.468)--(8.507,6.462)--(8.500,6.454)--(8.493,6.448)--(8.487,6.436)%
  --(8.473,6.430)--(8.473,6.429)--(8.467,6.429)--(8.460,6.425)--(8.453,6.419)--(8.447,6.415)%
  --(8.440,6.413)--(8.440,6.409)--(8.440,6.406)--(8.433,6.404)--(8.433,6.400)--(8.420,6.389)%
  --(8.420,6.386)--(8.420,6.388)--(8.414,6.386)--(8.414,6.384)--(8.407,6.382)--(8.414,6.379)%
  --(8.394,6.365)--(8.340,6.335)--(8.340,6.333)--(8.340,6.337)--(8.347,6.336)--(8.347,6.335)%
  --(8.347,6.334)--(8.347,6.333)--(8.347,6.332)--(8.347,6.326)--(8.340,6.324)--(8.334,6.327)%
  --(8.334,6.325)--(8.327,6.322)--(8.321,6.319)--(8.314,6.316)--(8.307,6.305)--(8.307,6.304)%
  --(8.307,6.306)--(8.301,6.307)--(8.301,6.306)--(8.294,6.303)--(8.294,6.300)--(8.294,6.298)%
  --(8.294,6.296)--(8.287,6.291)--(8.281,6.286)--(8.274,6.281)--(8.267,6.280)--(8.261,6.274)%
  --(8.247,6.270)--(8.241,6.261)--(8.241,6.257)--(8.234,6.255)--(8.234,6.253)--(8.228,6.246)%
  --(8.228,6.245)--(8.228,6.243)--(8.228,6.242)--(8.221,6.244)--(8.221,6.239)--(8.214,6.236)%
  --(8.214,6.233)--(8.208,6.233)--(8.208,6.227)--(8.201,6.222)--(8.194,6.224)--(8.188,6.222)%
  --(8.181,6.219)--(8.181,6.215)--(8.174,6.214)--(8.174,6.208)--(8.161,6.203)--(8.161,6.199)%
  --(8.161,6.191)--(8.148,6.192)--(8.148,6.190)--(8.148,6.186)--(8.135,6.182)--(8.135,6.178)%
  --(8.128,6.177)--(8.135,6.172)--(8.121,6.165)--(8.108,6.157)--(8.101,6.154)--(8.101,6.153)%
  --(8.095,6.147)--(8.088,6.140)--(8.088,6.137)--(8.081,6.136)--(8.075,6.134)--(8.068,6.132)%
  --(8.068,6.131)--(8.062,6.123)--(8.062,6.128)--(8.062,6.126)--(8.055,6.124)--(8.048,6.121)%
  --(8.048,6.117)--(8.048,6.116)--(8.048,6.113)--(8.035,6.109)--(8.002,6.083)--(7.969,6.064)%
  --(7.969,6.063)--(7.969,6.054)--(7.969,6.060)--(7.969,6.059)--(7.969,6.058)--(7.969,6.057)%
  --(7.969,6.056)--(7.969,6.050)--(7.969,6.056)--(7.969,6.055)--(7.962,6.055)--(7.962,6.051)%
  --(7.962,6.050)--(7.962,6.047)--(7.955,6.045)--(7.955,6.040)--(7.935,6.034)--(7.935,6.031)%
  --(7.935,6.029)--(7.929,6.026)--(7.929,6.017)--(7.922,6.022)--(7.922,6.020)--(7.915,6.017)%
  --(7.909,6.014)--(7.909,6.010)--(7.902,6.007)--(7.902,6.003)--(7.895,6.001)--(7.895,5.996)%
  --(7.895,5.993)--(7.889,5.993)--(7.882,5.991)--(7.876,5.989)--(7.876,5.985)--(7.876,5.982)%
  --(7.869,5.981)--(7.869,5.978)--(7.862,5.974)--(7.856,5.963)--(7.849,5.965)--(7.842,5.964)%
  --(7.842,5.961)--(7.829,5.954)--(7.822,5.947)--(7.822,5.945)--(7.816,5.943)--(7.816,5.942)%
  --(7.809,5.938)--(7.809,5.932)--(7.796,5.929)--(7.796,5.925)--(7.789,5.920)--(7.783,5.922)%
  --(7.783,5.919)--(7.783,5.917)--(7.776,5.915)--(7.776,5.909)--(7.776,5.908)--(7.769,5.906)%
  --(7.763,5.907)--(7.763,5.903)--(7.756,5.901)--(7.756,5.899)--(7.756,5.898)--(7.749,5.895)%
  --(7.749,5.890)--(7.743,5.887)--(7.743,5.884)--(7.736,5.884)--(7.736,5.885)--(7.729,5.881)%
  --(7.729,5.879)--(7.729,5.876)--(7.723,5.874)--(7.723,5.867)--(7.723,5.865)--(7.716,5.862)%
  --(7.709,5.860)--(7.709,5.859)--(7.709,5.861)--(7.703,5.858)--(7.696,5.857)--(7.696,5.855)%
  --(7.696,5.852)--(7.690,5.843)--(7.683,5.839)--(7.683,5.835)--(7.676,5.832)--(7.670,5.833)%
  --(7.670,5.832)--(7.663,5.830)--(7.663,5.828)--(7.656,5.826)--(7.656,5.820)--(7.656,5.816)%
  --(7.650,5.814)--(7.650,5.811)--(7.643,5.815)--(7.643,5.812)--(7.636,5.810)--(7.636,5.808)%
  --(7.630,5.806)--(7.630,5.800)--(7.630,5.798)--(7.623,5.797)--(7.623,5.792)--(7.616,5.795)%
  --(7.616,5.794)--(7.610,5.792)--(7.603,5.789)--(7.597,5.787)--(7.597,5.783)--(7.590,5.775)%
  --(7.583,5.771)--(7.583,5.769)--(7.577,5.771)--(7.577,5.769)--(7.570,5.766)--(7.570,5.764)%
  --(7.570,5.763)--(7.570,5.759)--(7.570,5.757)--(7.563,5.756)--(7.563,5.759)--(7.543,5.752)%
  --(7.543,5.749)--(7.543,5.746)--(7.537,5.745)--(7.537,5.742)--(7.537,5.738)--(7.537,5.741)%
  --(7.537,5.737)--(7.530,5.736)--(7.530,5.734)--(7.530,5.732)--(7.517,5.727)--(7.517,5.722)%
  --(7.510,5.721)--(7.510,5.722)--(7.510,5.721)--(7.510,5.719)--(7.510,5.718)--(7.504,5.717)%
  --(7.504,5.715)--(7.504,5.713)--(7.504,5.705)--(7.497,5.712)--(7.497,5.710)--(7.490,5.709)%
  --(7.490,5.708)--(7.490,5.707)--(7.490,5.705)--(7.490,5.704)--(7.484,5.704)--(7.484,5.701)%
  --(7.484,5.694)--(7.484,5.696)--(7.477,5.695)--(7.477,5.694)--(7.470,5.690)--(7.464,5.688)%
  --(7.464,5.686)--(7.457,5.683)--(7.424,5.656)--(7.411,5.653)--(7.404,5.652)--(7.411,5.651)%
  --(7.417,5.650)--(7.411,5.649)--(7.411,5.648)--(7.411,5.641)--(7.411,5.648)--(7.411,5.647)%
  --(7.404,5.647)--(7.411,5.646)--(7.404,5.640)--(7.404,5.644)--(7.397,5.642)--(7.397,5.639)%
  --(7.391,5.638)--(7.391,5.635)--(7.384,5.633)--(7.377,5.628)--(7.377,5.623)--(7.371,5.619)%
  --(7.371,5.615)--(7.364,5.617)--(7.364,5.616)--(7.364,5.615)--(7.364,5.612)--(7.357,5.609)%
  --(7.351,5.606)--(7.344,5.604)--(7.344,5.597)--(7.338,5.602)--(7.338,5.600)--(7.338,5.594)%
  --(7.324,5.592)--(7.318,5.586)--(7.318,5.583)--(7.311,5.580)--(7.311,5.579)--(7.311,5.577)%
  --(7.311,5.570)--(7.311,5.571)--(7.298,5.570)--(7.251,5.537)--(7.251,5.535)--(7.245,5.534)%
  --(7.251,5.534)--(7.238,5.530)--(7.238,5.527)--(7.238,5.524)--(7.225,5.518)--(7.218,5.511)%
  --(7.218,5.506)--(7.211,5.509)--(7.211,5.506)--(7.205,5.505)--(7.205,5.504)--(7.205,5.502)%
  --(7.205,5.498)--(7.205,5.496)--(7.205,5.495)--(7.205,5.499)--(7.198,5.500)--(7.205,5.500)%
  --(7.205,5.499)--(7.198,5.499)--(7.198,5.494)--(7.198,5.493)--(7.178,5.482)--(7.138,5.455)%
  --(7.132,5.451)--(7.132,5.449)--(7.132,5.447)--(7.125,5.444)--(7.125,5.443)--(7.125,5.436)%
  --(7.125,5.435)--(7.125,5.434)--(7.125,5.433)--(7.125,5.436)--(7.118,5.436)--(7.125,5.435)%
  --(7.118,5.431)--(7.118,5.430)--(7.118,5.426)--(7.105,5.424)--(7.092,5.419)--(7.085,5.417)%
  --(7.079,5.412)--(7.072,5.407)--(7.059,5.400)--(7.019,5.371)--(6.999,5.361)--(6.999,5.354)%
  --(7.005,5.353)--(6.999,5.355)--(6.999,5.356)--(6.999,5.354)--(6.999,5.352)--(6.999,5.349)%
  --(6.972,5.334)--(6.932,5.306)--(6.932,5.310)--(6.932,5.308)--(6.932,5.307)--(6.932,5.306)%
  --(6.926,5.302)--(6.926,5.299)--(6.926,5.296)--(6.912,5.289)--(6.899,5.280)--(6.899,5.279)%
  --(6.899,5.276)--(6.893,5.273)--(6.886,5.269)--(6.886,5.266)--(6.886,5.258)--(6.873,5.254)%
  --(6.873,5.251)--(6.873,5.249)--(6.873,5.251)--(6.866,5.251)--(6.866,5.250)--(6.873,5.247)%
  --(6.866,5.246)--(6.873,5.246)--(6.873,5.248)--(6.873,5.249)--(6.866,5.250)--(6.873,5.251)%
  --(6.866,5.249)--(6.853,5.246)--(6.853,5.244)--(6.846,5.238)--(6.839,5.228)--(6.826,5.224)%
  --(6.813,5.217)--(6.806,5.211)--(6.800,5.208)--(6.800,5.205)--(6.800,5.203)--(6.793,5.197)%
  --(6.793,5.199)--(6.793,5.198)--(6.786,5.197)--(6.726,5.160)--(6.720,5.156)--(6.720,5.153)%
  --(6.713,5.151)--(6.713,5.149)--(6.713,5.148)--(6.713,5.144)--(6.713,5.146)--(6.713,5.144)%
  --(6.713,5.143)--(6.713,5.140)--(6.707,5.139)--(6.707,5.136)--(6.700,5.131)--(6.653,5.100)%
  --(6.634,5.082)--(6.627,5.081)--(6.627,5.079)--(6.627,5.078)--(6.634,5.078)--(6.620,5.070)%
  --(6.614,5.063)--(6.607,5.057)--(6.594,5.047)--(6.580,5.046)--(6.580,5.042)--(6.567,5.037)%
  --(6.560,5.032)--(6.554,5.023)--(6.534,5.014)--(6.514,5.001)--(6.494,4.989)--(6.494,4.984)%
  --(6.487,4.981)--(6.487,4.977)--(6.467,4.962)--(6.454,4.956)--(6.454,4.950)--(6.448,4.943)%
  --(6.441,4.938)--(6.428,4.931)--(6.421,4.922)--(6.408,4.910)--(6.401,4.908)--(6.394,4.901)%
  --(6.388,4.897)--(6.381,4.896)--(6.374,4.892)--(6.374,4.890)--(6.368,4.885)--(6.355,4.873)%
  --(6.341,4.861)--(6.335,4.855)--(6.328,4.854)--(6.328,4.851)--(6.321,4.850)--(6.308,4.843)%
  --(6.255,4.812)--(6.255,4.809)--(6.255,4.805)--(6.248,4.803)--(6.248,4.796)--(6.242,4.796)%
  --(6.235,4.790)--(6.228,4.787)--(6.222,4.784)--(6.222,4.782)--(6.222,4.781)--(6.222,4.780)%
  --(6.222,4.774)--(6.222,4.771)--(6.202,4.755)--(6.162,4.733)--(6.149,4.726)--(6.155,4.725)%
  --(6.155,4.722)--(6.142,4.711)--(6.135,4.707)--(6.122,4.701)--(6.109,4.693)--(6.102,4.680)%
  --(6.096,4.675)--(6.082,4.672)--(6.069,4.666)--(6.062,4.661)--(6.056,4.653)--(6.042,4.641)%
  --(6.016,4.629)--(6.003,4.613)--(5.983,4.602)--(5.976,4.593)--(5.963,4.581)--(5.943,4.570)%
  --(5.929,4.563)--(5.923,4.553)--(5.903,4.543)--(5.890,4.534)--(5.876,4.525)--(5.870,4.516)%
  --(5.863,4.507)--(5.863,4.505)--(5.863,4.507)--(5.856,4.504)--(5.850,4.502)--(5.850,4.500)%
  --(5.850,4.498)--(5.843,4.492)--(5.836,4.489)--(5.836,4.487)--(5.836,4.485)--(5.830,4.487)%
  --(5.797,4.470)--(5.777,4.452)--(5.777,4.449)--(5.770,4.445)--(5.770,4.442)--(5.770,4.440)%
  --(5.763,4.438)--(5.763,4.436)--(5.763,4.438)--(5.763,4.437)--(5.763,4.438)--(5.763,4.436)%
  --(5.763,4.435)--(5.763,4.432)--(5.763,4.434)--(5.763,4.432)--(5.757,4.426)--(5.750,4.422)%
  --(5.743,4.419)--(5.737,4.413)--(5.730,4.405)--(5.724,4.399)--(5.717,4.394)--(5.710,4.390)%
  --(5.704,4.389)--(5.697,4.386)--(5.690,4.382)--(5.684,4.377)--(5.684,4.374)--(5.677,4.370)%
  --(5.677,4.368)--(5.670,4.361)--(5.664,4.359)--(5.657,4.358)--(5.651,4.354)--(5.644,4.350)%
  --(5.637,4.343)--(5.631,4.339)--(5.624,4.334)--(5.624,4.331)--(5.617,4.327)--(5.611,4.321)%
  --(5.611,4.316)--(5.597,4.315)--(5.591,4.311)--(5.584,4.304)--(5.577,4.298)--(5.571,4.294)%
  --(5.564,4.285)--(5.564,4.281)--(5.558,4.276)--(5.551,4.274)--(5.544,4.272)--(5.538,4.268)%
  --(5.531,4.263)--(5.524,4.258)--(5.518,4.252)--(5.511,4.246)--(5.504,4.241)--(5.491,4.233)%
  --(5.484,4.228)--(5.478,4.221)--(5.478,4.220)--(5.471,4.217)--(5.465,4.213)--(5.458,4.209)%
  --(5.451,4.203)--(5.445,4.192)--(5.438,4.187)--(5.431,4.186)--(5.425,4.183)--(5.418,4.175)%
  --(5.411,4.173)--(5.405,4.168)--(5.391,4.161)--(5.391,4.155)--(5.385,4.148)--(5.378,4.142)%
  --(5.372,4.136)--(5.358,4.130)--(5.352,4.123)--(5.338,4.118)--(5.332,4.113)--(5.332,4.111)%
  --(5.325,4.106)--(5.318,4.099)--(5.312,4.093)--(5.305,4.094)--(5.298,4.088)--(5.292,4.081)%
  --(5.279,4.071)--(5.265,4.059)--(5.259,4.054)--(5.245,4.049)--(5.245,4.044)--(5.239,4.039)%
  --(5.232,4.034)--(5.225,4.032)--(5.219,4.027)--(5.212,4.021)--(5.199,4.016)--(5.199,4.013)%
  --(5.199,4.011)--(5.192,4.009)--(5.186,4.005)--(5.179,3.997)--(5.166,3.992)--(5.159,3.983)%
  --(5.152,3.980)--(5.146,3.975)--(5.139,3.970)--(5.132,3.965)--(5.126,3.961)--(5.119,3.953)%
  --(5.106,3.948)--(5.093,3.938)--(5.086,3.930)--(5.073,3.923)--(5.066,3.918)--(5.059,3.912)%
  --(5.053,3.908)--(5.053,3.903)--(5.046,3.896)--(5.039,3.894)--(5.026,3.889)--(5.026,3.884)%
  --(5.013,3.877)--(5.013,3.872)--(5.006,3.867)--(5.000,3.862)--(4.986,3.858)--(4.986,3.852)%
  --(4.980,3.850)--(4.973,3.846)--(4.973,3.844)--(4.966,3.844)--(4.966,3.843)--(4.960,3.840)%
  --(4.953,3.839)--(4.946,3.830)--(4.940,3.822)--(4.933,3.813)--(4.927,3.813)--(4.920,3.808)%
  --(4.907,3.801)--(4.900,3.795)--(4.893,3.789)--(4.880,3.784)--(4.873,3.779)--(4.867,3.775)%
  --(4.867,3.768)--(4.860,3.763)--(4.853,3.758)--(4.847,3.753)--(4.840,3.745)--(4.827,3.737)%
  --(4.814,3.731)--(4.807,3.724)--(4.800,3.719)--(4.800,3.716)--(4.787,3.706)--(4.787,3.701)%
  --(4.780,3.696)--(4.774,3.697)--(4.774,3.694)--(4.767,3.691)--(4.760,3.688)--(4.754,3.685)%
  --(4.747,3.682)--(4.747,3.680)--(4.741,3.677)--(4.734,3.673)--(4.721,3.664)--(4.714,3.658)%
  --(4.701,3.651)--(4.694,3.642)--(4.687,3.636)--(4.674,3.630)--(4.668,3.620)--(4.661,3.614)%
  --(4.648,3.607)--(4.634,3.600)--(4.628,3.591)--(4.621,3.587)--(4.614,3.579)--(4.608,3.574)%
  --(4.608,3.571)--(4.601,3.567)--(4.594,3.556)--(4.575,3.546)--(4.561,3.535)--(4.548,3.525)%
  --(4.541,3.520)--(4.535,3.516)--(4.535,3.515)--(4.535,3.512)--(4.528,3.510)--(4.521,3.505)%
  --(4.515,3.501)--(4.508,3.497)--(4.501,3.493)--(4.495,3.487)--(4.488,3.480)--(4.482,3.476)%
  --(4.468,3.471)--(4.462,3.465)--(4.455,3.462)--(4.455,3.456)--(4.448,3.451)--(4.442,3.446)%
  --(4.435,3.443)--(4.435,3.444)--(4.428,3.439)--(4.415,3.432)--(4.408,3.423)--(4.402,3.416)%
  --(4.389,3.410)--(4.375,3.400)--(4.369,3.392)--(4.369,3.390)--(4.362,3.389)--(4.362,3.382)%
  --(4.349,3.376)--(4.335,3.367)--(4.329,3.359)--(4.315,3.350)--(4.315,3.348)--(4.302,3.344)%
  --(4.296,3.336)--(4.289,3.331)--(4.282,3.327)--(4.276,3.321)--(4.269,3.317)--(4.256,3.311)%
  --(4.249,3.306)--(4.242,3.301)--(4.242,3.298)--(4.236,3.297)--(4.236,3.296)--(4.236,3.294)%
  --(4.229,3.292)--(4.229,3.288)--(4.216,3.282)--(4.209,3.279)--(4.209,3.277)--(4.203,3.271)%
  --(4.189,3.265)--(4.183,3.260)--(4.176,3.254)--(4.169,3.249)--(4.163,3.242)--(4.156,3.234)%
  --(4.143,3.227)--(4.143,3.223)--(4.143,3.222)--(4.143,3.219)--(4.136,3.216)--(4.130,3.213)%
  --(4.123,3.207)--(4.110,3.200)--(4.103,3.192)--(4.096,3.185)--(4.090,3.181)--(4.083,3.177)%
  --(4.083,3.175)--(4.083,3.173)--(4.076,3.171)--(4.070,3.166)--(4.070,3.163)--(4.063,3.162)%
  --(4.063,3.161)--(4.063,3.160)--(4.056,3.157)--(4.050,3.154)--(4.043,3.148)--(4.037,3.143)%
  --(4.023,3.134)--(4.023,3.130)--(4.017,3.130)--(4.010,3.129)--(4.003,3.119)--(3.997,3.114)%
  --(3.990,3.111)--(3.983,3.106)--(3.983,3.105)--(3.983,3.103)--(3.977,3.102)--(3.977,3.100)%
  --(3.970,3.097)--(3.963,3.092)--(3.957,3.089)--(3.957,3.086)--(3.950,3.081)--(3.950,3.077)%
  --(3.944,3.074)--(3.944,3.073)--(3.944,3.071)--(3.944,3.069)--(3.937,3.065)--(3.930,3.062)%
  --(3.924,3.056)--(3.910,3.051)--(3.910,3.047)--(3.910,3.046)--(3.904,3.043)--(3.897,3.040)%
  --(3.890,3.033)--(3.877,3.025)--(3.870,3.015)--(3.857,3.007)--(3.851,3.002)--(3.837,2.994)%
  --(3.824,2.990)--(3.817,2.986)--(3.811,2.979)--(3.791,2.966)--(3.777,2.955)--(3.764,2.950)%
  --(3.758,2.944)--(3.744,2.936)--(3.738,2.932)--(3.731,2.925)--(3.724,2.918)--(3.718,2.912)%
  --(3.718,2.909)--(3.704,2.901)--(3.698,2.893)--(3.698,2.891)--(3.698,2.889)--(3.691,2.884)%
  --(3.691,2.882)--(3.691,2.877)--(3.678,2.871)--(3.671,2.869)--(3.671,2.864)--(3.651,2.857)%
  --(3.645,2.851)--(3.645,2.847)--(3.638,2.843)--(3.638,2.842)--(3.631,2.838)--(3.625,2.835)%
  --(3.625,2.833)--(3.611,2.828)--(3.605,2.822)--(3.598,2.817)--(3.592,2.808)--(3.578,2.804)%
  --(3.572,2.800)--(3.565,2.795)--(3.558,2.790)--(3.558,2.784)--(3.552,2.782)--(3.552,2.781)%
  --(3.552,2.779)--(3.552,2.776)--(3.545,2.775)--(3.545,2.774)--(3.538,2.773)--(3.532,2.771)%
  --(3.525,2.766)--(3.518,2.758)--(3.505,2.749)--(3.492,2.741)--(3.485,2.736)--(3.485,2.732)%
  --(3.485,2.731)--(3.479,2.728)--(3.479,2.723)--(3.472,2.717)--(3.465,2.714)--(3.459,2.711)%
  --(3.459,2.708)--(3.452,2.705)--(3.452,2.698)--(3.445,2.696)--(3.439,2.692)--(3.439,2.690)%
  --(3.432,2.686)--(3.425,2.680)--(3.419,2.677)--(3.406,2.671)--(3.399,2.663)--(3.392,2.656)%
  --(3.386,2.650)--(3.379,2.646)--(3.372,2.641)--(3.366,2.636)--(3.352,2.631)--(3.346,2.624)%
  --(3.339,2.619)--(3.332,2.615)--(3.326,2.610)--(3.319,2.605)--(3.319,2.604)--(3.319,2.605)%
  --(3.319,2.603)--(3.313,2.601)--(3.313,2.600)--(3.306,2.599)--(3.306,2.596)--(3.299,2.595)%
  --(3.299,2.591)--(3.293,2.587)--(3.286,2.586)--(3.286,2.582)--(3.279,2.576)--(3.273,2.572)%
  --(3.266,2.567)--(3.259,2.562)--(3.253,2.558)--(3.253,2.554)--(3.246,2.550)--(3.240,2.548)%
  --(3.240,2.546)--(3.240,2.541)--(3.233,2.536)--(3.220,2.529)--(3.206,2.521)--(3.200,2.514)%
  --(3.200,2.510)--(3.193,2.508)--(3.193,2.506)--(3.180,2.500)--(3.180,2.496)--(3.173,2.492)%
  --(3.160,2.488)--(3.160,2.484)--(3.153,2.480)--(3.153,2.478)--(3.147,2.475)--(3.140,2.471)%
  --(3.133,2.469)--(3.133,2.464)--(3.127,2.462)--(3.127,2.461)--(3.127,2.459)--(3.120,2.455)%
  --(3.120,2.449)--(3.113,2.445)--(3.107,2.440)--(3.100,2.437)--(3.093,2.435)--(3.087,2.431)%
  --(3.087,2.430)--(3.087,2.429)--(3.080,2.429)--(3.080,2.426)--(3.067,2.420)--(3.060,2.416)%
  --(3.054,2.413)--(3.047,2.407)--(3.040,2.402)--(3.034,2.397)--(3.034,2.394)--(3.020,2.388)%
  --(3.014,2.380)--(3.007,2.374)--(3.000,2.370)--(2.994,2.364)--(2.987,2.359)--(2.987,2.357)%
  --(2.987,2.356)--(2.987,2.354)--(2.974,2.347)--(2.961,2.338)--(2.954,2.332)--(2.947,2.327)%
  --(2.947,2.326)--(2.941,2.323)--(2.941,2.319)--(2.934,2.315)--(2.927,2.309)--(2.921,2.304)%
  --(2.914,2.299)--(2.907,2.295)--(2.901,2.290)--(2.894,2.284)--(2.887,2.281)--(2.881,2.276)%
  --(2.881,2.273)--(2.874,2.269)--(2.868,2.263)--(2.861,2.261)--(2.854,2.256)--(2.848,2.253)%
  --(2.841,2.251)--(2.841,2.250)--(2.834,2.248)--(2.828,2.242)--(2.814,2.237)--(2.814,2.233)%
  --(2.808,2.230)--(2.801,2.223)--(2.794,2.216)--(2.788,2.211)--(2.781,2.207)--(2.775,2.202)%
  --(2.768,2.194)--(2.761,2.190)--(2.755,2.188)--(2.755,2.182)--(2.741,2.174)--(2.728,2.166)%
  --(2.721,2.160)--(2.721,2.155)--(2.715,2.152)--(2.708,2.146)--(2.702,2.144)--(2.702,2.141)%
  --(2.695,2.136)--(2.682,2.132)--(2.675,2.126)--(2.662,2.116)--(2.648,2.106)--(2.642,2.100)%
  --(2.635,2.097)--(2.635,2.095)--(2.635,2.094)--(2.635,2.093)--(2.628,2.090)--(2.622,2.085)%
  --(2.615,2.079)--(2.595,2.072)--(2.589,2.065)--(2.589,2.063)--(2.582,2.058)--(2.575,2.054)%
  --(2.569,2.049)--(2.562,2.046)--(2.562,2.043)--(2.555,2.039)--(2.555,2.038)--(2.555,2.036)%
  --(2.555,2.035)--(2.549,2.030)--(2.542,2.024)--(2.535,2.020)--(2.529,2.014)--(2.529,2.008)%
  --(2.522,2.005)--(2.509,1.998)--(2.496,1.989)--(2.496,1.986)--(2.489,1.983)--(2.476,1.975)%
  --(2.476,1.972)--(2.469,1.968)--(2.456,1.960)--(2.449,1.950)--(2.442,1.946)--(2.436,1.942)%
  --(2.436,1.940)--(2.429,1.937)--(2.429,1.933)--(2.423,1.931)--(2.416,1.928)--(2.416,1.927)%
  --(2.409,1.924)--(2.403,1.918)--(2.396,1.913)--(2.389,1.910)--(2.389,1.905)--(2.383,1.902)%
  --(2.376,1.897)--(2.369,1.892)--(2.369,1.888)--(2.363,1.886)--(2.356,1.886)--(2.349,1.883)%
  --(2.343,1.881)--(2.336,1.876)--(2.330,1.872)--(2.323,1.865)--(2.316,1.859)--(2.310,1.853)%
  --(2.303,1.844)--(2.296,1.837)--(2.290,1.831)--(2.290,1.830)--(2.283,1.827)--(2.276,1.825)%
  --(2.270,1.816)--(2.257,1.807)--(2.250,1.801)--(2.243,1.798)--(2.237,1.794)--(2.230,1.788)%
  --(2.223,1.782)--(2.217,1.780)--(2.217,1.779)--(2.217,1.777)--(2.210,1.775)--(2.203,1.769)%
  --(2.197,1.763)--(2.197,1.760)--(2.190,1.757)--(2.183,1.751)--(2.177,1.746)--(2.177,1.743)%
  --(2.170,1.740)--(2.170,1.737)--(2.164,1.733)--(2.150,1.729)--(2.150,1.728)--(2.150,1.726);
\gpcolor{color=gp lt color 5}
\gpsetlinetype{gp lt plot 5}
\draw[gp path] (9.563,7.428)--(9.563,7.425)--(9.563,7.420)--(9.556,7.419)--(9.556,7.416)%
  --(9.556,7.412)--(9.556,7.410)--(9.549,7.409)--(9.549,7.407)--(9.543,7.404)--(9.543,7.406)%
  --(9.536,7.403)--(9.536,7.401)--(9.529,7.398)--(9.523,7.396)--(9.529,7.393)--(9.523,7.390)%
  --(9.523,7.388)--(9.523,7.379)--(9.523,7.388)--(9.516,7.387)--(9.516,7.385)--(9.516,7.381)%
  --(9.509,7.376)--(9.509,7.372)--(9.509,7.370)--(9.503,7.369)--(9.503,7.367)--(9.496,7.362)%
  --(9.496,7.366)--(9.490,7.365)--(9.490,7.363)--(9.490,7.362)--(9.490,7.361)--(9.490,7.360)%
  --(9.483,7.358)--(9.483,7.354)--(9.476,7.349)--(9.476,7.347)--(9.476,7.342)--(9.476,7.346)%
  --(9.476,7.345)--(9.476,7.344)--(9.470,7.343)--(9.463,7.342)--(9.470,7.339)--(9.470,7.334)%
  --(9.463,7.333)--(9.456,7.331)--(9.450,7.339)--(9.450,7.333)--(9.450,7.332)--(9.450,7.331)%
  --(9.450,7.330)--(9.443,7.323)--(9.450,7.323)--(9.450,7.325)--(9.443,7.323)--(9.443,7.322)%
  --(9.436,7.321)--(9.436,7.316)--(9.430,7.310)--(9.430,7.308)--(9.430,7.306)--(9.423,7.308)%
  --(9.423,7.304)--(9.410,7.299)--(9.403,7.291)--(9.403,7.289)--(9.397,7.286)--(9.397,7.279)%
  --(9.390,7.277)--(9.397,7.277)--(9.390,7.276)--(9.390,7.277)--(9.383,7.280)--(9.383,7.279)%
  --(9.383,7.278)--(9.377,7.275)--(9.370,7.269)--(9.363,7.267)--(9.363,7.261)--(9.357,7.257)%
  --(9.357,7.261)--(9.350,7.259)--(9.350,7.257)--(9.350,7.255)--(9.343,7.253)--(9.343,7.242)%
  --(9.337,7.238)--(9.323,7.231)--(9.317,7.228)--(9.317,7.226)--(9.317,7.229)--(9.310,7.226)%
  --(9.310,7.223)--(9.304,7.221)--(9.310,7.220)--(9.304,7.212)--(9.304,7.208)--(9.297,7.204)%
  --(9.290,7.201)--(9.284,7.198)--(9.277,7.195)--(9.277,7.193)--(9.270,7.191)--(9.264,7.185)%
  --(9.264,7.179)--(9.257,7.175)--(9.237,7.161)--(9.230,7.154)--(9.224,7.157)--(9.224,7.153)%
  --(9.217,7.150)--(9.211,7.146)--(9.204,7.141)--(9.197,7.131)--(9.191,7.125)--(9.184,7.117)%
  --(9.164,7.108)--(9.164,7.106)--(9.157,7.106)--(9.157,7.105)--(9.157,7.102)--(9.157,7.098)%
  --(9.157,7.097)--(9.157,7.101)--(9.157,7.100)--(9.151,7.098)--(9.144,7.096)--(9.144,7.095)%
  --(9.144,7.094)--(9.144,7.086)--(9.144,7.092)--(9.137,7.091)--(9.131,7.087)--(9.131,7.083)%
  --(9.118,7.079)--(9.111,7.077)--(9.111,7.074)--(9.104,7.070)--(9.098,7.060)--(9.098,7.062)%
  --(9.098,7.060)--(9.098,7.057)--(9.091,7.052)--(9.091,7.049)--(9.084,7.047)--(9.084,7.045)%
  --(9.078,7.044)--(9.078,7.035)--(9.071,7.038)--(9.071,7.035)--(9.064,7.031)--(9.058,7.028)%
  --(9.064,7.027)--(9.058,7.026)--(9.058,7.025)--(9.058,7.023)--(9.051,7.017)--(9.051,7.021)%
  --(9.051,7.019)--(9.051,7.017)--(9.045,7.015)--(9.045,7.013)--(9.038,7.011)--(9.031,7.008)%
  --(9.031,7.004)--(9.031,7.001)--(9.025,6.989)--(9.025,6.996)--(9.018,6.994)--(9.018,6.990)%
  --(9.018,6.988)--(9.011,6.985)--(9.011,6.984)--(9.005,6.977)--(9.005,6.982)--(8.998,6.981)%
  --(8.991,6.980)--(8.998,6.979)--(8.991,6.976)--(8.991,6.975)--(8.991,6.971)--(8.991,6.970)%
  --(8.985,6.967)--(8.985,6.962)--(8.985,6.965)--(8.978,6.962)--(8.971,6.960)--(8.971,6.955)%
  --(8.965,6.952)--(8.958,6.951)--(8.952,6.946)--(8.952,6.943)--(8.952,6.932)--(8.945,6.938)%
  --(8.945,6.935)--(8.938,6.932)--(8.932,6.927)--(8.925,6.925)--(8.932,6.923)--(8.918,6.921)%
  --(8.918,6.919)--(8.912,6.917)--(8.905,6.914)--(8.905,6.907)--(8.912,6.912)--(8.905,6.911)%
  --(8.905,6.909)--(8.898,6.905)--(8.892,6.901)--(8.892,6.899)--(8.885,6.897)--(8.885,6.895)%
  --(8.872,6.891)--(8.872,6.886)--(8.872,6.878)--(8.872,6.875)--(8.872,6.878)--(8.865,6.875)%
  --(8.859,6.870)--(8.859,6.868)--(8.852,6.867)--(8.852,6.859)--(8.852,6.858)--(8.845,6.856)%
  --(8.845,6.854)--(8.845,6.853)--(8.839,6.854)--(8.832,6.851)--(8.832,6.849)--(8.825,6.845)%
  --(8.819,6.838)--(8.825,6.834)--(8.812,6.831)--(8.812,6.828)--(8.805,6.826)--(8.799,6.824)%
  --(8.799,6.822)--(8.792,6.821)--(8.799,6.820)--(8.792,6.812)--(8.785,6.808)--(8.772,6.803)%
  --(8.772,6.799)--(8.766,6.802)--(8.766,6.799)--(8.766,6.797)--(8.759,6.794)--(8.759,6.789)%
  --(8.752,6.787)--(8.752,6.778)--(8.746,6.775)--(8.739,6.771)--(8.732,6.774)--(8.732,6.773)%
  --(8.726,6.771)--(8.726,6.769)--(8.726,6.767)--(8.719,6.760)--(8.712,6.756)--(8.712,6.753)%
  --(8.712,6.750)--(8.706,6.748)--(8.699,6.751)--(8.692,6.748)--(8.686,6.744)--(8.686,6.742)%
  --(8.679,6.738)--(8.679,6.728)--(8.666,6.724)--(8.659,6.720)--(8.659,6.716)--(8.659,6.721)%
  --(8.653,6.718)--(8.653,6.716)--(8.646,6.713)--(8.646,6.710)--(8.646,6.703)--(8.639,6.698)%
  --(8.639,6.696)--(8.619,6.692)--(8.626,6.691)--(8.626,6.694)--(8.626,6.691)--(8.619,6.688)%
  --(8.619,6.687)--(8.613,6.686)--(8.619,6.683)--(8.613,6.678)--(8.613,6.677)--(8.613,6.676)%
  --(8.613,6.681)--(8.613,6.682)--(8.613,6.680)--(8.613,6.679)--(8.613,6.678)--(8.613,6.674)%
  --(8.613,6.670)--(8.606,6.669)--(8.606,6.671)--(8.606,6.673)--(8.606,6.672)--(8.606,6.671)%
  --(8.599,6.669)--(8.599,6.665)--(8.599,6.660)--(8.593,6.660)--(8.599,6.659)--(8.586,6.664)%
  --(8.586,6.661)--(8.586,6.659)--(8.580,6.657)--(8.573,6.653)--(8.573,6.650)--(8.573,6.641)%
  --(8.566,6.644)--(8.566,6.642)--(8.553,6.641)--(8.546,6.636)--(8.546,6.632)--(8.546,6.630)%
  --(8.540,6.627)--(8.540,6.623)--(8.533,6.614)--(8.526,6.619)--(8.526,6.618)--(8.526,6.615)%
  --(8.520,6.614)--(8.520,6.610)--(8.513,6.605)--(8.507,6.603)--(8.500,6.601)--(8.500,6.599)%
  --(8.500,6.596)--(8.500,6.592)--(8.500,6.595)--(8.493,6.593)--(8.493,6.592)--(8.493,6.590)%
  --(8.493,6.587)--(8.487,6.585)--(8.487,6.582)--(8.473,6.580)--(8.473,6.574)--(8.467,6.574)%
  --(8.467,6.573)--(8.467,6.571)--(8.460,6.570)--(8.460,6.567)--(8.460,6.566)--(8.453,6.565)%
  --(8.453,6.563)--(8.447,6.558)--(8.440,6.547)--(8.433,6.553)--(8.433,6.552)--(8.433,6.550)%
  --(8.427,6.548)--(8.427,6.547)--(8.420,6.544)--(8.427,6.538)--(8.427,6.542)--(8.420,6.543)%
  --(8.420,6.542)--(8.420,6.539)--(8.414,6.531)--(8.407,6.527)--(8.400,6.524)--(8.400,6.521)%
  --(8.394,6.520)--(8.394,6.512)--(8.394,6.516)--(8.394,6.514)--(8.394,6.512)--(8.394,6.510)%
  --(8.387,6.509)--(8.387,6.506)--(8.387,6.504)--(8.380,6.502)--(8.374,6.493)--(8.374,6.500)%
  --(8.367,6.497)--(8.367,6.495)--(8.367,6.493)--(8.367,6.491)--(8.367,6.490)--(8.367,6.489)%
  --(8.367,6.487)--(8.360,6.486)--(8.360,6.483)--(8.354,6.480)--(8.354,6.477)--(8.340,6.473)%
  --(8.347,6.476)--(8.347,6.475)--(8.340,6.474)--(8.334,6.473)--(8.334,6.470)--(8.334,6.468)%
  --(8.334,6.467)--(8.327,6.458)--(8.321,6.461)--(8.321,6.459)--(8.307,6.455)--(8.307,6.452)%
  --(8.301,6.447)--(8.294,6.434)--(8.287,6.430)--(8.281,6.426)--(8.281,6.423)--(8.281,6.425)%
  --(8.274,6.422)--(8.267,6.419)--(8.261,6.416)--(8.267,6.414)--(8.261,6.409)--(8.261,6.407)%
  --(8.254,6.403)--(8.247,6.399)--(8.247,6.398)--(8.241,6.400)--(8.234,6.397)--(8.228,6.391)%
  --(8.228,6.388)--(8.221,6.383)--(8.221,6.374)--(8.221,6.373)--(8.214,6.372)--(8.208,6.373)%
  --(8.208,6.374)--(8.201,6.373)--(8.201,6.372)--(8.194,6.370)--(8.194,6.368)--(8.194,6.367)%
  --(8.194,6.365)--(8.188,6.359)--(8.188,6.356)--(8.188,6.355)--(8.181,6.354)--(8.174,6.350)%
  --(8.168,6.341)--(8.168,6.337)--(8.161,6.335)--(8.161,6.332)--(8.161,6.334)--(8.154,6.331)%
  --(8.154,6.330)--(8.154,6.327)--(8.148,6.325)--(8.141,6.318)--(8.135,6.314)--(8.135,6.313)%
  --(8.135,6.312)--(8.128,6.314)--(8.128,6.316)--(8.128,6.314)--(8.128,6.310)--(8.121,6.307)%
  --(8.121,6.303)--(8.115,6.295)--(8.108,6.292)--(8.108,6.289)--(8.095,6.286)--(8.095,6.289)%
  --(8.095,6.287)--(8.088,6.283)--(8.081,6.279)--(8.088,6.273)--(8.081,6.270)--(8.081,6.268)%
  --(8.068,6.265)--(8.068,6.263)--(8.062,6.262)--(8.062,6.261)--(8.062,6.259)--(8.055,6.259)%
  --(8.055,6.256)--(8.048,6.252)--(8.035,6.248)--(8.035,6.243)--(8.035,6.240)--(8.028,6.235)%
  --(8.028,6.234)--(8.022,6.233)--(8.022,6.232)--(8.022,6.231)--(8.022,6.224)--(8.015,6.219)%
  --(8.008,6.217)--(8.008,6.216)--(8.002,6.215)--(8.002,6.216)--(7.995,6.212)--(7.988,6.207)%
  --(7.988,6.206)--(7.988,6.205)--(7.988,6.199)--(7.982,6.197)--(7.975,6.196)--(7.975,6.198)%
  --(7.969,6.197)--(7.969,6.194)--(7.962,6.191)--(7.962,6.189)--(7.962,6.180)--(7.962,6.186)%
  --(7.955,6.183)--(7.949,6.182)--(7.955,6.179)--(7.955,6.178)--(7.949,6.178)--(7.942,6.174)%
  --(7.942,6.175)--(7.935,6.172)--(7.935,6.168)--(7.935,6.164)--(7.929,6.159)--(7.922,6.157)%
  --(7.922,6.155)--(7.915,6.153)--(7.909,6.149)--(7.909,6.145)--(7.909,6.143)--(7.902,6.141)%
  --(7.902,6.140)--(7.902,6.139)--(7.902,6.137)--(7.895,6.137)--(7.895,6.136)--(7.895,6.134)%
  --(7.895,6.132)--(7.889,6.124)--(7.882,6.129)--(7.876,6.125)--(7.876,6.122)--(7.876,6.121)%
  --(7.869,6.120)--(7.869,6.117)--(7.869,6.107)--(7.869,6.113)--(7.862,6.112)--(7.862,6.111)%
  --(7.862,6.109)--(7.856,6.107)--(7.862,6.107)--(7.856,6.105)--(7.849,6.102)--(7.842,6.100)%
  --(7.842,6.095)--(7.842,6.098)--(7.842,6.096)--(7.836,6.092)--(7.829,6.086)--(7.829,6.083)%
  --(7.822,6.082)--(7.822,6.080)--(7.822,6.078)--(7.816,6.071)--(7.816,6.075)--(7.809,6.075)%
  --(7.809,6.074)--(7.809,6.071)--(7.802,6.068)--(7.802,6.066)--(7.796,6.064)--(7.796,6.063)%
  --(7.796,6.062)--(7.796,6.061)--(7.789,6.058)--(7.783,6.058)--(7.783,6.056)--(7.783,6.055)%
  --(7.783,6.054)--(7.783,6.052)--(7.776,6.050)--(7.776,6.047)--(7.769,6.045)--(7.769,6.038)%
  --(7.763,6.041)--(7.763,6.037)--(7.763,6.036)--(7.756,6.034)--(7.756,6.032)--(7.749,6.024)%
  --(7.756,6.030)--(7.749,6.029)--(7.749,6.028)--(7.749,6.025)--(7.743,6.023)--(7.743,6.020)%
  --(7.736,6.018)--(7.729,6.015)--(7.729,6.006)--(7.729,6.012)--(7.729,6.010)--(7.723,6.010)%
  --(7.729,6.009)--(7.723,6.008)--(7.723,6.006)--(7.723,6.001)--(7.723,5.999)--(7.716,5.998)%
  --(7.716,5.996)--(7.716,5.998)--(7.716,5.999)--(7.709,5.997)--(7.709,5.996)--(7.709,5.994)%
  --(7.709,5.992)--(7.703,5.990)--(7.696,5.987)--(7.696,5.985)--(7.690,5.983)--(7.690,5.981)%
  --(7.690,5.979)--(7.690,5.978)--(7.683,5.975)--(7.683,5.971)--(7.676,5.966)--(7.683,5.965)%
  --(7.676,5.964)--(7.676,5.963)--(7.676,5.967)--(7.670,5.966)--(7.670,5.964)--(7.663,5.963)%
  --(7.663,5.960)--(7.663,5.954)--(7.663,5.952)--(7.663,5.950)--(7.656,5.947)--(7.656,5.949)%
  --(7.656,5.951)--(7.656,5.950)--(7.656,5.949)--(7.650,5.946)--(7.650,5.944)--(7.650,5.942)%
  --(7.650,5.936)--(7.643,5.936)--(7.636,5.934)--(7.630,5.935)--(7.630,5.931)--(7.623,5.928)%
  --(7.623,5.927)--(7.623,5.925)--(7.623,5.918)--(7.616,5.915)--(7.610,5.912)--(7.603,5.909)%
  --(7.610,5.910)--(7.603,5.913)--(7.597,5.912)--(7.597,5.909)--(7.597,5.908)--(7.590,5.905)%
  --(7.590,5.904)--(7.583,5.901)--(7.583,5.900)--(7.583,5.899)--(7.583,5.900)--(7.577,5.897)%
  --(7.570,5.894)--(7.570,5.892)--(7.570,5.890)--(7.563,5.886)--(7.563,5.884)--(7.563,5.879)%
  --(7.563,5.876)--(7.557,5.873)--(7.550,5.872)--(7.543,5.871)--(7.543,5.868)--(7.543,5.866)%
  --(7.537,5.864)--(7.530,5.856)--(7.524,5.854)--(7.524,5.853)--(7.530,5.852)--(7.517,5.853)%
  --(7.517,5.854)--(7.517,5.853)--(7.517,5.852)--(7.517,5.849)--(7.517,5.845)--(7.517,5.842)%
  --(7.510,5.840)--(7.510,5.838)--(7.504,5.837)--(7.504,5.838)--(7.504,5.837)--(7.497,5.836)%
  --(7.504,5.835)--(7.497,5.834)--(7.497,5.829)--(7.497,5.827)--(7.497,5.826)--(7.490,5.829)%
  --(7.490,5.828)--(7.490,5.827)--(7.490,5.826)--(7.484,5.825)--(7.484,5.824)--(7.484,5.822)%
  --(7.484,5.821)--(7.477,5.820)--(7.484,5.819)--(7.477,5.818)--(7.477,5.811)--(7.470,5.811)%
  --(7.477,5.809)--(7.470,5.808)--(7.470,5.809)--(7.464,5.811)--(7.464,5.810)--(7.464,5.809)%
  --(7.464,5.808)--(7.464,5.805)--(7.457,5.802)--(7.457,5.801)--(7.457,5.804)--(7.457,5.803)%
  --(7.450,5.802)--(7.450,5.799)--(7.450,5.790)--(7.444,5.796)--(7.444,5.794)--(7.437,5.792)%
  --(7.437,5.791)--(7.437,5.789)--(7.437,5.788)--(7.431,5.787)--(7.431,5.786)--(7.431,5.785)%
  --(7.431,5.786)--(7.424,5.784)--(7.424,5.783)--(7.424,5.780)--(7.424,5.777)--(7.417,5.775)%
  --(7.417,5.774)--(7.417,5.772)--(7.411,5.765)--(7.411,5.770)--(7.411,5.768)--(7.404,5.767)%
  --(7.404,5.765)--(7.397,5.764)--(7.397,5.762)--(7.397,5.760)--(7.391,5.757)--(7.391,5.755)%
  --(7.391,5.754)--(7.391,5.748)--(7.391,5.752)--(7.384,5.751)--(7.377,5.749)--(7.384,5.747)%
  --(7.377,5.741)--(7.377,5.745)--(7.377,5.746)--(7.371,5.745)--(7.371,5.744)--(7.371,5.741)%
  --(7.371,5.739)--(7.364,5.737)--(7.364,5.735)--(7.364,5.729)--(7.364,5.734)--(7.364,5.733)%
  --(7.364,5.732)--(7.364,5.731)--(7.357,5.729)--(7.351,5.728)--(7.351,5.727)--(7.344,5.721)%
  --(7.344,5.722)--(7.338,5.722)--(7.344,5.719)--(7.338,5.717)--(7.324,5.713)--(7.324,5.711)%
  --(7.318,5.709)--(7.318,5.705)--(7.311,5.695)--(7.311,5.698)--(7.304,5.694)--(7.298,5.690)%
  --(7.298,5.689)--(7.298,5.688)--(7.298,5.686)--(7.291,5.686)--(7.291,5.684)--(7.291,5.681)%
  --(7.291,5.676)--(7.284,5.680)--(7.284,5.679)--(7.284,5.678)--(7.284,5.676)--(7.278,5.676)%
  --(7.278,5.670)--(7.278,5.671)--(7.278,5.675)--(7.278,5.674)--(7.278,5.673)--(7.271,5.669)%
  --(7.271,5.667)--(7.264,5.665)--(7.264,5.663)--(7.264,5.661)--(7.258,5.656)--(7.258,5.652)%
  --(7.251,5.650)--(7.258,5.649)--(7.251,5.654)--(7.251,5.653)--(7.258,5.653)--(7.251,5.652)%
  --(7.251,5.651)--(7.245,5.649)--(7.245,5.644)--(7.245,5.643)--(7.245,5.645)--(7.245,5.646)%
  --(7.238,5.644)--(7.238,5.643)--(7.238,5.641)--(7.231,5.638)--(7.225,5.635)--(7.218,5.631)%
  --(7.211,5.629)--(7.211,5.626)--(7.211,5.625)--(7.211,5.622)--(7.211,5.621)--(7.205,5.621)%
  --(7.205,5.619)--(7.205,5.614)--(7.205,5.615)--(7.205,5.618)--(7.205,5.617)--(7.205,5.616)%
  --(7.198,5.615)--(7.198,5.611)--(7.198,5.609)--(7.198,5.608)--(7.198,5.607)--(7.198,5.610)%
  --(7.198,5.611)--(7.191,5.610)--(7.191,5.608)--(7.185,5.604)--(7.185,5.601)--(7.185,5.600)%
  --(7.178,5.599)--(7.185,5.602)--(7.178,5.603)--(7.178,5.602)--(7.178,5.601)--(7.165,5.594)%
  --(7.125,5.569)--(7.125,5.566)--(7.125,5.561)--(7.118,5.560)--(7.125,5.561)--(7.118,5.562)%
  --(7.112,5.560)--(7.112,5.554)--(7.105,5.550)--(7.105,5.542)--(7.098,5.537)--(7.092,5.537)%
  --(7.079,5.530)--(7.072,5.522)--(7.052,5.515)--(7.045,5.509)--(7.039,5.504)--(7.025,5.498)%
  --(7.025,5.493)--(7.025,5.491)--(7.019,5.489)--(7.019,5.483)--(7.012,5.476)--(7.012,5.477)%
  --(7.012,5.480)--(7.012,5.479)--(7.012,5.477)--(7.005,5.475)--(7.005,5.474)--(6.999,5.468)%
  --(7.005,5.469)--(7.005,5.468)--(7.005,5.469)--(7.005,5.472)--(6.999,5.471)--(6.999,5.470)%
  --(6.999,5.469)--(6.999,5.465)--(6.999,5.463)--(6.999,5.462)--(6.992,5.462)--(6.992,5.463)%
  --(6.986,5.463)--(6.986,5.462)--(6.986,5.458)--(6.986,5.452)--(6.979,5.449)--(6.979,5.450)%
  --(6.979,5.451)--(6.972,5.449)--(6.972,5.447)--(6.972,5.444)--(6.966,5.443)--(6.966,5.441)%
  --(6.966,5.437)--(6.966,5.436)--(6.966,5.439)--(6.966,5.438)--(6.959,5.438)--(6.959,5.436)%
  --(6.959,5.435)--(6.952,5.435)--(6.952,5.434)--(6.952,5.429)--(6.952,5.428)--(6.952,5.432)%
  --(6.952,5.433)--(6.952,5.432)--(6.952,5.433)--(6.952,5.432)--(6.952,5.428)--(6.952,5.426)%
  --(6.946,5.426)--(6.952,5.425)--(6.946,5.429)--(6.946,5.426)--(6.939,5.425)--(6.926,5.420)%
  --(6.932,5.417)--(6.926,5.415)--(6.926,5.413)--(6.919,5.414)--(6.919,5.413)--(6.919,5.412)%
  --(6.919,5.409)--(6.912,5.406)--(6.906,5.403)--(6.899,5.401)--(6.899,5.399)--(6.899,5.392)%
  --(6.899,5.396)--(6.893,5.393)--(6.886,5.388)--(6.886,5.386)--(6.879,5.380)--(6.873,5.376)%
  --(6.873,5.374)--(6.873,5.372)--(6.873,5.369)--(6.866,5.365)--(6.859,5.366)--(6.853,5.361)%
  --(6.853,5.357)--(6.846,5.354)--(6.839,5.349)--(6.839,5.347)--(6.839,5.344)--(6.833,5.342)%
  --(6.826,5.341)--(6.826,5.336)--(6.826,5.338)--(6.819,5.336)--(6.819,5.333)--(6.813,5.330)%
  --(6.813,5.328)--(6.806,5.325)--(6.806,5.321)--(6.800,5.316)--(6.800,5.317)--(6.800,5.314)%
  --(6.793,5.312)--(6.786,5.311)--(6.786,5.309)--(6.786,5.306)--(6.780,5.304)--(6.780,5.303)%
  --(6.773,5.305)--(6.780,5.304)--(6.773,5.303)--(6.773,5.301)--(6.773,5.300)--(6.773,5.297)%
  --(6.766,5.293)--(6.760,5.285)--(6.753,5.287)--(6.753,5.283)--(6.740,5.276)--(6.726,5.268)%
  --(6.726,5.264)--(6.720,5.262)--(6.720,5.256)--(6.713,5.260)--(6.720,5.260)--(6.720,5.259)%
  --(6.713,5.256)--(6.700,5.250)--(6.693,5.247)--(6.693,5.244)--(6.687,5.240)--(6.680,5.236)%
  --(6.680,5.228)--(6.673,5.231)--(6.640,5.206)--(6.634,5.201)--(6.634,5.198)--(6.627,5.195)%
  --(6.620,5.188)--(6.607,5.179)--(6.594,5.171)--(6.587,5.159)--(6.580,5.152)--(6.580,5.149)%
  --(6.567,5.148)--(6.567,5.151)--(6.567,5.150)--(6.567,5.148)--(6.567,5.146)--(6.560,5.145)%
  --(6.560,5.139)--(6.560,5.138)--(6.560,5.140)--(6.560,5.139)--(6.560,5.138)--(6.554,5.136)%
  --(6.554,5.127)--(6.547,5.119)--(6.541,5.116)--(6.534,5.116)--(6.534,5.115)--(6.521,5.112)%
  --(6.521,5.110)--(6.514,5.106)--(6.507,5.100)--(6.507,5.095)--(6.501,5.089)--(6.494,5.085)%
  --(6.487,5.080)--(6.474,5.078)--(6.467,5.072)--(6.461,5.065)--(6.454,5.059)--(6.448,5.053)%
  --(6.441,5.050)--(6.434,5.046)--(6.434,5.045)--(6.428,5.041)--(6.421,5.032)--(6.414,5.028)%
  --(6.408,5.026)--(6.408,5.022)--(6.401,5.018)--(6.394,5.014)--(6.381,5.008)--(6.381,5.000)%
  --(6.374,4.994)--(6.368,4.990)--(6.361,4.988)--(6.361,4.987)--(6.355,4.984)--(6.355,4.980)%
  --(6.341,4.977)--(6.341,4.973)--(6.335,4.969)--(6.328,4.963)--(6.321,4.959)--(6.321,4.956)%
  --(6.321,4.953)--(6.315,4.954)--(6.308,4.951)--(6.308,4.948)--(6.301,4.943)--(6.301,4.940)%
  --(6.301,4.934)--(6.295,4.930)--(6.295,4.929)--(6.295,4.930)--(6.288,4.928)--(6.281,4.926)%
  --(6.281,4.925)--(6.281,4.923)--(6.275,4.921)--(6.268,4.917)--(6.268,4.911)--(6.255,4.908)%
  --(6.262,4.909)--(6.255,4.910)--(6.255,4.908)--(6.255,4.905)--(6.248,4.902)--(6.248,4.900)%
  --(6.248,4.897)--(6.242,4.897)--(6.242,4.895)--(6.235,4.893)--(6.228,4.887)--(6.228,4.885)%
  --(6.222,4.882)--(6.215,4.879)--(6.215,4.877)--(6.215,4.874)--(6.208,4.867)--(6.202,4.863)%
  --(6.202,4.860)--(6.195,4.858)--(6.195,4.857)--(6.188,4.856)--(6.182,4.848)--(6.175,4.845)%
  --(6.169,4.843)--(6.169,4.841)--(6.169,4.839)--(6.162,4.831)--(6.155,4.827)--(6.149,4.826)%
  --(6.149,4.823)--(6.142,4.820)--(6.142,4.818)--(6.135,4.814)--(6.129,4.807)--(6.129,4.809)%
  --(6.122,4.805)--(6.122,4.803)--(6.109,4.799)--(6.109,4.795)--(6.102,4.790)--(6.102,4.785)%
  --(6.096,4.784)--(6.089,4.778)--(6.082,4.774)--(6.082,4.770)--(6.076,4.768)--(6.069,4.767)%
  --(6.069,4.764)--(6.069,4.762)--(6.069,4.758)--(6.062,4.756)--(6.062,4.755)--(6.062,4.751)%
  --(6.056,4.752)--(6.056,4.750)--(6.056,4.749)--(6.049,4.747)--(6.049,4.745)--(6.049,4.744)%
  --(6.042,4.738)--(6.042,4.739)--(6.042,4.737)--(6.042,4.735)--(6.036,4.734)--(6.029,4.734)%
  --(6.029,4.731)--(6.016,4.728)--(6.009,4.723)--(6.009,4.719)--(6.003,4.712)--(5.989,4.702)%
  --(5.983,4.700)--(5.976,4.695)--(5.969,4.690)--(5.969,4.683)--(5.963,4.679)--(5.956,4.674)%
  --(5.949,4.666)--(5.943,4.668)--(5.943,4.666)--(5.936,4.665)--(5.929,4.663)--(5.929,4.660)%
  --(5.929,4.656)--(5.923,4.656)--(5.929,4.654)--(5.923,4.651)--(5.923,4.645)--(5.916,4.645)%
  --(5.910,4.641)--(5.910,4.639)--(5.910,4.638)--(5.903,4.636)--(5.896,4.635)--(5.896,4.633)%
  --(5.890,4.632)--(5.890,4.626)--(5.883,4.621)--(5.883,4.619)--(5.876,4.614)--(5.870,4.609)%
  --(5.870,4.606)--(5.856,4.605)--(5.856,4.600)--(5.850,4.595)--(5.843,4.589)--(5.836,4.582)%
  --(5.836,4.584)--(5.836,4.582)--(5.830,4.581)--(5.830,4.580)--(5.830,4.579)--(5.830,4.577)%
  --(5.830,4.575)--(5.823,4.572)--(5.817,4.567)--(5.817,4.563)--(5.810,4.561)--(5.810,4.559)%
  --(5.803,4.559)--(5.797,4.558)--(5.797,4.557)--(5.797,4.555)--(5.790,4.554)--(5.790,4.551)%
  --(5.783,4.545)--(5.777,4.540)--(5.777,4.536)--(5.770,4.536)--(5.763,4.533)--(5.757,4.529)%
  --(5.757,4.526)--(5.750,4.524)--(5.750,4.521)--(5.743,4.516)--(5.737,4.507)--(5.737,4.503)%
  --(5.724,4.500)--(5.724,4.498)--(5.717,4.498)--(5.710,4.491)--(5.704,4.489)--(5.697,4.485)%
  --(5.690,4.483)--(5.690,4.477)--(5.690,4.474)--(5.690,4.471)--(5.684,4.470)--(5.684,4.467)%
  --(5.684,4.468)--(5.670,4.465)--(5.670,4.463)--(5.670,4.462)--(5.664,4.456)--(5.664,4.449)%
  --(5.664,4.447)--(5.657,4.446)--(5.651,4.447)--(5.651,4.445)--(5.644,4.441)--(5.644,4.438)%
  --(5.637,4.433)--(5.631,4.426)--(5.624,4.421)--(5.617,4.419)--(5.617,4.417)--(5.611,4.417)%
  --(5.604,4.415)--(5.597,4.411)--(5.597,4.407)--(5.591,4.404)--(5.591,4.397)--(5.584,4.395)%
  --(5.577,4.391)--(5.577,4.390)--(5.571,4.387)--(5.564,4.383)--(5.558,4.380)--(5.558,4.377)%
  --(5.551,4.373)--(5.551,4.369)--(5.544,4.365)--(5.544,4.362)--(5.538,4.359)--(5.524,4.356)%
  --(5.518,4.350)--(5.504,4.342)--(5.504,4.338)--(5.498,4.335)--(5.491,4.333)--(5.491,4.331)%
  --(5.491,4.329)--(5.491,4.322)--(5.484,4.321)--(5.484,4.322)--(5.484,4.319)--(5.478,4.317)%
  --(5.478,4.315)--(5.478,4.312)--(5.471,4.307)--(5.465,4.302)--(5.458,4.299)--(5.451,4.296)%
  --(5.445,4.291)--(5.438,4.287)--(5.438,4.283)--(5.438,4.280)--(5.431,4.276)--(5.425,4.277)%
  --(5.425,4.275)--(5.418,4.273)--(5.418,4.270)--(5.418,4.266)--(5.411,4.265)--(5.411,4.264)%
  --(5.411,4.262)--(5.405,4.256)--(5.405,4.257)--(5.398,4.252)--(5.391,4.248)--(5.385,4.245)%
  --(5.385,4.243)--(5.385,4.240)--(5.378,4.238)--(5.372,4.236)--(5.372,4.232)--(5.372,4.228)%
  --(5.365,4.229)--(5.365,4.228)--(5.358,4.226)--(5.358,4.223)--(5.352,4.218)--(5.345,4.215)%
  --(5.345,4.211)--(5.338,4.209)--(5.332,4.205)--(5.332,4.201)--(5.325,4.199)--(5.325,4.196)%
  --(5.325,4.195)--(5.318,4.192)--(5.318,4.190)--(5.312,4.190)--(5.312,4.186)--(5.312,4.187)%
  --(5.305,4.185)--(5.305,4.183)--(5.305,4.182)--(5.305,4.179)--(5.298,4.177)--(5.292,4.174)%
  --(5.285,4.171)--(5.279,4.165)--(5.279,4.159)--(5.272,4.158)--(5.272,4.157)--(5.272,4.156)%
  --(5.272,4.155)--(5.272,4.151)--(5.265,4.149)--(5.259,4.146)--(5.259,4.145)--(5.259,4.142)%
  --(5.252,4.139)--(5.245,4.135)--(5.245,4.132)--(5.239,4.132)--(5.239,4.131)--(5.239,4.125)%
  --(5.232,4.124)--(5.225,4.121)--(5.225,4.117)--(5.212,4.112)--(5.212,4.107)--(5.205,4.102)%
  --(5.205,4.100)--(5.205,4.099)--(5.199,4.098)--(5.192,4.094)--(5.186,4.093)--(5.179,4.088)%
  --(5.172,4.084)--(5.166,4.078)--(5.166,4.075)--(5.166,4.073)--(5.166,4.071)--(5.159,4.067)%
  --(5.159,4.062)--(5.152,4.058)--(5.146,4.058)--(5.146,4.057)--(5.146,4.055)--(5.139,4.053)%
  --(5.139,4.050)--(5.132,4.047)--(5.126,4.042)--(5.119,4.036)--(5.119,4.034)--(5.113,4.031)%
  --(5.106,4.030)--(5.106,4.029)--(5.106,4.027)--(5.106,4.025)--(5.099,4.024)--(5.093,4.022)%
  --(5.093,4.020)--(5.086,4.018)--(5.086,4.012)--(5.079,4.009)--(5.073,4.008)--(5.066,4.003)%
  --(5.059,3.997)--(5.059,3.995)--(5.059,3.993)--(5.059,3.990)--(5.053,3.988)--(5.046,3.983)%
  --(5.039,3.980)--(5.033,3.978)--(5.033,3.976)--(5.026,3.974)--(5.026,3.971)--(5.020,3.969)%
  --(5.020,3.967)--(5.020,3.965)--(5.013,3.959)--(5.006,3.955)--(5.000,3.952)--(5.000,3.951)%
  --(4.993,3.948)--(4.993,3.946)--(4.993,3.941)--(4.993,3.940)--(4.986,3.938)--(4.986,3.935)%
  --(4.980,3.932)--(4.973,3.928)--(4.973,3.925)--(4.966,3.921)--(4.966,3.920)--(4.966,3.918)%
  --(4.960,3.919)--(4.960,3.918)--(4.953,3.917)--(4.953,3.914)--(4.953,3.910)--(4.953,3.909)%
  --(4.953,3.908)--(4.946,3.909)--(4.946,3.908)--(4.946,3.906)--(4.940,3.904)--(4.946,3.904)%
  --(4.940,3.903)--(4.940,3.900)--(4.940,3.899)--(4.940,3.896)--(4.933,3.895)--(4.933,3.897)%
  --(4.933,3.895)--(4.927,3.896)--(4.927,3.894)--(4.927,3.891)--(4.927,3.890)--(4.920,3.887)%
  --(4.920,3.885)--(4.913,3.883)--(4.913,3.882)--(4.913,3.881)--(4.913,3.878)--(4.913,3.876)%
  --(4.907,3.873)--(4.907,3.872)--(4.907,3.870)--(4.900,3.868)--(4.900,3.864)--(4.900,3.863)%
  --(4.900,3.862)--(4.900,3.866)--(4.900,3.867)--(4.900,3.866)--(4.893,3.866)--(4.893,3.865)%
  --(4.893,3.861)--(4.893,3.863)--(4.893,3.862)--(4.887,3.861)--(4.887,3.858)--(4.880,3.858)%
  --(4.880,3.856)--(4.880,3.854)--(4.873,3.853)--(4.867,3.847)--(4.860,3.844)--(4.853,3.843)%
  --(4.853,3.842)--(4.853,3.841)--(4.853,3.839)--(4.847,3.839)--(4.847,3.838)--(4.847,3.835)%
  --(4.847,3.832)--(4.840,3.828)--(4.840,3.823)--(4.840,3.825)--(4.840,3.823)--(4.834,3.821)%
  --(4.834,3.819)--(4.827,3.817)--(4.827,3.816)--(4.820,3.814)--(4.820,3.812)--(4.814,3.810)%
  --(4.814,3.808)--(4.807,3.805)--(4.800,3.799)--(4.800,3.797)--(4.794,3.793)--(4.787,3.792)%
  --(4.787,3.788)--(4.780,3.785)--(4.774,3.784)--(4.774,3.781)--(4.774,3.778)--(4.767,3.777)%
  --(4.767,3.775)--(4.760,3.770)--(4.760,3.768)--(4.754,3.765)--(4.754,3.761)--(4.754,3.762)%
  --(4.754,3.761)--(4.747,3.757)--(4.747,3.755)--(4.734,3.750)--(4.734,3.745)--(4.734,3.742)%
  --(4.727,3.742)--(4.727,3.738)--(4.727,3.737)--(4.721,3.736)--(4.721,3.735)--(4.714,3.733)%
  --(4.714,3.731)--(4.714,3.729)--(4.714,3.728)--(4.707,3.723)--(4.701,3.723)--(4.701,3.721)%
  --(4.701,3.716)--(4.701,3.718)--(4.701,3.719)--(4.701,3.716)--(4.694,3.713)--(4.694,3.711)%
  --(4.687,3.710)--(4.687,3.709)--(4.687,3.707)--(4.687,3.708)--(4.687,3.707)--(4.687,3.705)%
  --(4.681,3.702)--(4.681,3.699)--(4.674,3.697)--(4.674,3.695)--(4.674,3.694)--(4.674,3.692)%
  --(4.668,3.691)--(4.661,3.689)--(4.661,3.687)--(4.648,3.684)--(4.648,3.681)--(4.641,3.677)%
  --(4.641,3.676)--(4.641,3.677)--(4.634,3.674)--(4.634,3.670)--(4.628,3.668)--(4.621,3.664)%
  --(4.621,3.662)--(4.614,3.658)--(4.614,3.655)--(4.614,3.654)--(4.608,3.651)--(4.608,3.648)%
  --(4.601,3.647)--(4.601,3.644)--(4.594,3.643)--(4.594,3.641)--(4.594,3.640)--(4.588,3.640)%
  --(4.588,3.638)--(4.588,3.635)--(4.581,3.631)--(4.581,3.630)--(4.575,3.626)--(4.575,3.625)%
  --(4.568,3.622)--(4.568,3.619)--(4.568,3.617)--(4.561,3.616)--(4.561,3.614)--(4.555,3.612)%
  --(4.555,3.610)--(4.548,3.607)--(4.548,3.605)--(4.541,3.601)--(4.535,3.597)--(4.535,3.593)%
  --(4.528,3.588)--(4.528,3.586)--(4.521,3.584)--(4.521,3.581)--(4.521,3.580)--(4.521,3.579)%
  --(4.515,3.575)--(4.515,3.572)--(4.508,3.570)--(4.508,3.571)--(4.508,3.570)--(4.508,3.569)%
  --(4.508,3.567)--(4.501,3.566)--(4.501,3.565)--(4.501,3.561)--(4.495,3.560)--(4.495,3.556)%
  --(4.495,3.554)--(4.488,3.554)--(4.488,3.553)--(4.488,3.552)--(4.482,3.547)--(4.475,3.545)%
  --(4.475,3.542)--(4.468,3.537)--(4.468,3.538)--(4.462,3.534)--(4.462,3.533)--(4.462,3.531)%
  --(4.455,3.530)--(4.455,3.525)--(4.455,3.524)--(4.448,3.524)--(4.448,3.523)--(4.448,3.519)%
  --(4.448,3.518)--(4.442,3.518)--(4.442,3.515)--(4.435,3.510)--(4.428,3.510)--(4.428,3.507)%
  --(4.422,3.505)--(4.422,3.500)--(4.415,3.496)--(4.408,3.491)--(4.402,3.488)--(4.395,3.487)%
  --(4.395,3.485)--(4.395,3.483)--(4.395,3.482)--(4.389,3.478)--(4.382,3.474)--(4.382,3.477)%
  --(4.382,3.476)--(4.382,3.475)--(4.375,3.472)--(4.375,3.468)--(4.369,3.468)--(4.369,3.465)%
  --(4.362,3.463)--(4.362,3.461)--(4.362,3.459)--(4.355,3.457)--(4.355,3.455)--(4.355,3.452)%
  --(4.349,3.452)--(4.349,3.450)--(4.342,3.446)--(4.335,3.444)--(4.335,3.441)--(4.329,3.436)%
  --(4.329,3.437)--(4.322,3.436)--(4.322,3.434)--(4.315,3.433)--(4.315,3.431)--(4.315,3.429)%
  --(4.315,3.427)--(4.309,3.424)--(4.309,3.419)--(4.302,3.415)--(4.296,3.408)--(4.289,3.408)%
  --(4.289,3.406)--(4.282,3.403)--(4.276,3.400)--(4.276,3.399)--(4.276,3.397)--(4.269,3.394)%
  --(4.269,3.392)--(4.262,3.389)--(4.262,3.388)--(4.256,3.385)--(4.256,3.381)--(4.249,3.377)%
  --(4.249,3.375)--(4.242,3.371)--(4.242,3.367)--(4.242,3.368)--(4.242,3.366)--(4.236,3.365)%
  --(4.236,3.363)--(4.236,3.362)--(4.229,3.360)--(4.229,3.357)--(4.229,3.355)--(4.222,3.352)%
  --(4.222,3.349)--(4.222,3.348)--(4.222,3.347)--(4.216,3.346)--(4.216,3.342)--(4.209,3.343)%
  --(4.209,3.342)--(4.209,3.341)--(4.209,3.340)--(4.209,3.341)--(4.209,3.339)--(4.209,3.338)%
  --(4.209,3.336)--(4.203,3.335)--(4.203,3.334)--(4.203,3.333)--(4.196,3.329)--(4.196,3.332)%
  --(4.196,3.331)--(4.196,3.330)--(4.189,3.330)--(4.189,3.329)--(4.189,3.326)--(4.183,3.322)%
  --(4.176,3.320)--(4.176,3.319)--(4.169,3.316)--(4.169,3.314)--(4.163,3.311)--(4.156,3.306)%
  --(4.149,3.302)--(4.149,3.301)--(4.149,3.297)--(4.149,3.295)--(4.143,3.292)--(4.143,3.291)%
  --(4.136,3.290)--(4.136,3.289)--(4.130,3.287)--(4.130,3.283)--(4.130,3.282)--(4.123,3.281)%
  --(4.123,3.278)--(4.116,3.276)--(4.116,3.273)--(4.110,3.273)--(4.103,3.269)--(4.096,3.265)%
  --(4.096,3.261)--(4.090,3.257)--(4.083,3.255)--(4.083,3.254)--(4.083,3.253)--(4.076,3.251)%
  --(4.076,3.250)--(4.076,3.247)--(4.070,3.243)--(4.070,3.241)--(4.063,3.240)--(4.063,3.238)%
  --(4.063,3.237)--(4.056,3.236)--(4.056,3.233)--(4.056,3.232)--(4.050,3.231)--(4.050,3.227)%
  --(4.050,3.224)--(4.043,3.222)--(4.037,3.220)--(4.037,3.218)--(4.037,3.217)--(4.030,3.216)%
  --(4.030,3.214)--(4.030,3.210)--(4.030,3.208)--(4.023,3.205)--(4.023,3.204)--(4.023,3.203)%
  --(4.023,3.202)--(4.017,3.200)--(4.017,3.196)--(4.010,3.195)--(4.010,3.194)--(4.010,3.190)%
  --(4.003,3.189)--(4.003,3.186)--(4.003,3.183)--(4.003,3.181)--(3.997,3.178)--(3.990,3.176)%
  --(3.990,3.175)--(3.990,3.171)--(3.990,3.170)--(3.983,3.168)--(3.983,3.169)--(3.983,3.168)%
  --(3.977,3.167)--(3.977,3.166)--(3.977,3.163)--(3.977,3.160)--(3.970,3.157)--(3.970,3.156)%
  --(3.963,3.156)--(3.963,3.157)--(3.963,3.156)--(3.963,3.154)--(3.957,3.154)--(3.957,3.153)%
  --(3.957,3.152)--(3.957,3.149)--(3.950,3.147)--(3.950,3.145)--(3.944,3.143)--(3.944,3.139)%
  --(3.937,3.137)--(3.937,3.135)--(3.937,3.133)--(3.930,3.131)--(3.930,3.130)--(3.924,3.129)%
  --(3.924,3.126)--(3.917,3.125)--(3.910,3.120)--(3.910,3.116)--(3.904,3.113)--(3.897,3.111)%
  --(3.897,3.107)--(3.890,3.107)--(3.890,3.105)--(3.890,3.104)--(3.890,3.103)--(3.890,3.102)%
  --(3.884,3.100)--(3.884,3.097)--(3.877,3.094)--(3.877,3.093)--(3.870,3.092)--(3.870,3.091)%
  --(3.870,3.089)--(3.870,3.088)--(3.864,3.085)--(3.864,3.084)--(3.857,3.080)--(3.857,3.078)%
  --(3.851,3.074)--(3.851,3.071)--(3.844,3.069)--(3.844,3.066)--(3.837,3.064)--(3.831,3.062)%
  --(3.831,3.060)--(3.824,3.055)--(3.817,3.051)--(3.817,3.049)--(3.817,3.050)--(3.817,3.049)%
  --(3.811,3.047)--(3.811,3.045)--(3.804,3.042)--(3.797,3.038)--(3.797,3.039)--(3.797,3.037)%
  --(3.797,3.032)--(3.791,3.029)--(3.791,3.026)--(3.784,3.023)--(3.784,3.021)--(3.777,3.017)%
  --(3.777,3.013)--(3.771,3.010)--(3.771,3.008)--(3.764,3.005)--(3.764,3.004)--(3.764,3.002)%
  --(3.764,3.000)--(3.758,2.999)--(3.758,2.995)--(3.758,2.997)--(3.758,2.995)--(3.751,2.993)%
  --(3.751,2.992)--(3.751,2.991)--(3.751,2.988)--(3.744,2.988)--(3.744,2.986)--(3.738,2.985)%
  --(3.738,2.981)--(3.731,2.980)--(3.731,2.978)--(3.731,2.977)--(3.724,2.974)--(3.724,2.971)%
  --(3.718,2.968)--(3.718,2.967)--(3.711,2.961)--(3.704,2.958)--(3.704,2.957)--(3.704,2.956)%
  --(3.698,2.953)--(3.698,2.952)--(3.691,2.948)--(3.685,2.947)--(3.685,2.945)--(3.685,2.944)%
  --(3.678,2.943)--(3.678,2.942)--(3.671,2.939)--(3.671,2.935)--(3.665,2.932)--(3.658,2.930)%
  --(3.658,2.926)--(3.651,2.925)--(3.651,2.921)--(3.645,2.919)--(3.645,2.917)--(3.645,2.916)%
  --(3.638,2.913)--(3.631,2.909)--(3.631,2.907)--(3.631,2.906)--(3.625,2.902)--(3.618,2.899)%
  --(3.618,2.896)--(3.611,2.893)--(3.605,2.891)--(3.605,2.888)--(3.605,2.885)--(3.598,2.883)%
  --(3.598,2.881)--(3.598,2.879)--(3.592,2.877)--(3.592,2.875)--(3.592,2.874)--(3.585,2.873)%
  --(3.585,2.872)--(3.585,2.870)--(3.578,2.867)--(3.572,2.864)--(3.572,2.862)--(3.572,2.861)%
  --(3.572,2.860)--(3.565,2.858)--(3.565,2.856)--(3.565,2.853)--(3.565,2.852)--(3.565,2.851)%
  --(3.558,2.850)--(3.558,2.848)--(3.558,2.847)--(3.558,2.846)--(3.558,2.845)--(3.552,2.844)%
  --(3.552,2.842)--(3.552,2.840)--(3.545,2.839)--(3.545,2.838)--(3.545,2.837)--(3.545,2.833)%
  --(3.538,2.831)--(3.538,2.830)--(3.538,2.829)--(3.538,2.828)--(3.538,2.825)--(3.538,2.823)%
  --(3.532,2.823)--(3.532,2.822)--(3.532,2.821)--(3.532,2.822)--(3.532,2.820)--(3.525,2.820)%
  --(3.525,2.819)--(3.525,2.817)--(3.525,2.815)--(3.525,2.814)--(3.518,2.813)--(3.518,2.811)%
  --(3.518,2.810)--(3.518,2.809)--(3.512,2.807)--(3.512,2.805)--(3.512,2.804)--(3.505,2.801)%
  --(3.499,2.800)--(3.492,2.797)--(3.492,2.796)--(3.492,2.795)--(3.492,2.793)--(3.485,2.791)%
  --(3.485,2.786)--(3.479,2.784)--(3.479,2.783)--(3.479,2.782)--(3.479,2.781)--(3.472,2.777)%
  --(3.465,2.774)--(3.465,2.773)--(3.465,2.774)--(3.465,2.773)--(3.459,2.771)--(3.459,2.769)%
  --(3.452,2.766)--(3.452,2.763)--(3.445,2.762)--(3.445,2.761)--(3.445,2.758)--(3.439,2.756)%
  --(3.439,2.755)--(3.439,2.754)--(3.432,2.753)--(3.432,2.751)--(3.432,2.749)--(3.425,2.746)%
  --(3.425,2.747)--(3.425,2.746)--(3.425,2.744)--(3.425,2.743)--(3.419,2.740)--(3.419,2.739)%
  --(3.412,2.738)--(3.412,2.737)--(3.412,2.736)--(3.406,2.735)--(3.406,2.734)--(3.406,2.733)%
  --(3.406,2.731)--(3.406,2.730)--(3.406,2.729)--(3.406,2.727)--(3.399,2.725)--(3.399,2.726)%
  --(3.399,2.725)--(3.399,2.723)--(3.399,2.722)--(3.392,2.720)--(3.392,2.718)--(3.386,2.717)%
  --(3.379,2.714)--(3.372,2.709)--(3.372,2.708)--(3.372,2.706)--(3.366,2.701)--(3.359,2.697)%
  --(3.359,2.693)--(3.352,2.692)--(3.352,2.690)--(3.346,2.685)--(3.346,2.684)--(3.339,2.683)%
  --(3.339,2.680)--(3.332,2.679)--(3.332,2.676)--(3.326,2.675)--(3.326,2.673)--(3.326,2.670)%
  --(3.326,2.671)--(3.326,2.670)--(3.326,2.666)--(3.319,2.665)--(3.319,2.662)--(3.319,2.661)%
  --(3.319,2.659)--(3.313,2.657)--(3.313,2.655)--(3.313,2.654)--(3.313,2.652)--(3.306,2.651)%
  --(3.306,2.649)--(3.306,2.648)--(3.306,2.647)--(3.306,2.646)--(3.306,2.644)--(3.299,2.642)%
  --(3.299,2.643)--(3.299,2.640)--(3.299,2.638)--(3.293,2.638)--(3.293,2.636)--(3.286,2.633)%
  --(3.286,2.632)--(3.286,2.630)--(3.286,2.629)--(3.279,2.629)--(3.279,2.628)--(3.279,2.627)%
  --(3.279,2.626)--(3.279,2.625)--(3.273,2.624)--(3.279,2.624)--(3.273,2.623)--(3.273,2.624)%
  --(3.273,2.619)--(3.266,2.616)--(3.266,2.615)--(3.266,2.614)--(3.259,2.612)--(3.259,2.610)%
  --(3.259,2.609)--(3.253,2.609)--(3.253,2.610)--(3.253,2.609)--(3.246,2.607)--(3.246,2.604)%
  --(3.240,2.600)--(3.233,2.596)--(3.233,2.595)--(3.226,2.593)--(3.226,2.592)--(3.226,2.589)%
  --(3.220,2.586)--(3.220,2.583)--(3.213,2.582)--(3.213,2.581)--(3.213,2.579)--(3.206,2.577)%
  --(3.206,2.574)--(3.206,2.573)--(3.200,2.573)--(3.193,2.569)--(3.193,2.566)--(3.186,2.565)%
  --(3.180,2.563)--(3.173,2.559)--(3.173,2.556)--(3.166,2.551)--(3.166,2.548)--(3.160,2.545)%
  --(3.160,2.542)--(3.153,2.539)--(3.153,2.536)--(3.153,2.535)--(3.147,2.534)--(3.147,2.531)%
  --(3.140,2.527)--(3.133,2.523)--(3.133,2.522)--(3.127,2.521)--(3.127,2.520)--(3.127,2.518)%
  --(3.127,2.516)--(3.127,2.515)--(3.120,2.513)--(3.120,2.512)--(3.120,2.511)--(3.113,2.508)%
  --(3.113,2.507)--(3.113,2.505)--(3.113,2.503)--(3.107,2.502)--(3.107,2.499)--(3.107,2.496)%
  --(3.100,2.494)--(3.093,2.491)--(3.093,2.489)--(3.087,2.487)--(3.087,2.485)--(3.087,2.483)%
  --(3.080,2.480)--(3.080,2.478)--(3.073,2.477)--(3.073,2.473)--(3.067,2.471)--(3.060,2.467)%
  --(3.060,2.465)--(3.054,2.461)--(3.054,2.458)--(3.047,2.454)--(3.047,2.452)--(3.047,2.451)%
  --(3.047,2.448)--(3.040,2.448)--(3.047,2.448)--(3.040,2.446)--(3.040,2.445)--(3.040,2.443)%
  --(3.040,2.440)--(3.034,2.439)--(3.034,2.438)--(3.034,2.437)--(3.027,2.434)--(3.020,2.431)%
  --(3.014,2.429)--(3.007,2.424)--(3.007,2.422)--(3.007,2.421)--(3.000,2.420)--(3.000,2.418)%
  --(2.994,2.415)--(2.994,2.412)--(2.987,2.411)--(2.987,2.410)--(2.980,2.407)--(2.980,2.406)%
  --(2.980,2.404)--(2.974,2.402)--(2.974,2.401)--(2.974,2.399)--(2.967,2.397)--(2.967,2.396)%
  --(2.961,2.395)--(2.961,2.394)--(2.961,2.392)--(2.954,2.387)--(2.954,2.386)--(2.954,2.383)%
  --(2.947,2.380)--(2.941,2.377)--(2.941,2.374)--(2.934,2.370)--(2.927,2.364)--(2.921,2.362)%
  --(2.921,2.361)--(2.921,2.360)--(2.914,2.358)--(2.907,2.353)--(2.901,2.349)--(2.894,2.344)%
  --(2.887,2.339)--(2.881,2.335)--(2.881,2.334)--(2.881,2.332)--(2.881,2.329)--(2.874,2.327)%
  --(2.874,2.326)--(2.868,2.323)--(2.868,2.319)--(2.861,2.316)--(2.861,2.312)--(2.854,2.308)%
  --(2.854,2.305)--(2.854,2.304)--(2.848,2.299)--(2.841,2.295)--(2.834,2.293)--(2.834,2.289)%
  --(2.828,2.286)--(2.828,2.284)--(2.821,2.282)--(2.821,2.280)--(2.821,2.277)--(2.814,2.276)%
  --(2.808,2.271)--(2.801,2.262)--(2.788,2.258)--(2.781,2.254)--(2.781,2.253)--(2.775,2.250)%
  --(2.775,2.249)--(2.768,2.248)--(2.768,2.245)--(2.761,2.243)--(2.755,2.240)--(2.755,2.237)%
  --(2.748,2.234)--(2.748,2.231)--(2.741,2.230)--(2.741,2.228)--(2.741,2.226)--(2.741,2.223)%
  --(2.735,2.217)--(2.728,2.214)--(2.721,2.212)--(2.721,2.211)--(2.715,2.208)--(2.715,2.206)%
  --(2.708,2.202)--(2.702,2.197)--(2.702,2.196)--(2.702,2.193)--(2.695,2.190)--(2.688,2.185)%
  --(2.688,2.183)--(2.682,2.180)--(2.682,2.176)--(2.675,2.174)--(2.668,2.172)--(2.668,2.168)%
  --(2.662,2.165)--(2.662,2.163)--(2.655,2.161)--(2.655,2.160)--(2.655,2.156)--(2.648,2.155)%
  --(2.648,2.154)--(2.648,2.151)--(2.648,2.149)--(2.642,2.147)--(2.642,2.145)--(2.635,2.141)%
  --(2.635,2.139)--(2.635,2.137)--(2.628,2.136)--(2.628,2.133)--(2.622,2.132)--(2.622,2.128)%
  --(2.615,2.125)--(2.609,2.123)--(2.602,2.117)--(2.602,2.113)--(2.595,2.111)--(2.595,2.110)%
  --(2.589,2.103)--(2.582,2.101)--(2.582,2.100)--(2.582,2.098)--(2.575,2.094)--(2.562,2.089)%
  --(2.555,2.083)--(2.555,2.081)--(2.549,2.079)--(2.549,2.074)--(2.535,2.070)--(2.529,2.063)%
  --(2.529,2.061)--(2.529,2.059)--(2.522,2.058)--(2.522,2.057)--(2.522,2.055)--(2.516,2.055)%
  --(2.516,2.054)--(2.509,2.049)--(2.509,2.047)--(2.502,2.044)--(2.502,2.043)--(2.496,2.040)%
  --(2.489,2.036)--(2.489,2.032)--(2.482,2.029)--(2.476,2.024)--(2.469,2.021)--(2.469,2.017)%
  --(2.462,2.016)--(2.462,2.011)--(2.456,2.007)--(2.456,2.005)--(2.456,2.003)--(2.449,2.002)%
  --(2.449,2.000)--(2.442,1.998)--(2.442,1.997)--(2.442,1.995)--(2.442,1.993)--(2.436,1.992)%
  --(2.436,1.990)--(2.436,1.989)--(2.429,1.987)--(2.429,1.986)--(2.429,1.984)--(2.423,1.982)%
  --(2.423,1.979)--(2.416,1.976)--(2.416,1.975)--(2.416,1.971)--(2.409,1.970)--(2.409,1.968)%
  --(2.403,1.965)--(2.403,1.964)--(2.403,1.961)--(2.403,1.960)--(2.403,1.959)--(2.396,1.958)%
  --(2.396,1.955)--(2.396,1.952)--(2.396,1.951)--(2.389,1.948)--(2.389,1.947)--(2.389,1.946)%
  --(2.389,1.945)--(2.383,1.944)--(2.383,1.943)--(2.383,1.942)--(2.383,1.940)--(2.376,1.937)%
  --(2.376,1.934)--(2.369,1.933)--(2.369,1.932)--(2.369,1.931)--(2.369,1.930)--(2.363,1.928)%
  --(2.363,1.927)--(2.363,1.925)--(2.356,1.924)--(2.356,1.923)--(2.356,1.921)--(2.349,1.918)%
  --(2.343,1.914)--(2.343,1.910)--(2.336,1.910)--(2.336,1.909)--(2.336,1.906)--(2.330,1.905)%
  --(2.323,1.903)--(2.323,1.901)--(2.316,1.898)--(2.316,1.892)--(2.310,1.891)--(2.310,1.888)%
  --(2.303,1.886)--(2.296,1.883)--(2.290,1.881)--(2.290,1.878)--(2.283,1.876)--(2.283,1.875)%
  --(2.276,1.872)--(2.276,1.869)--(2.270,1.865)--(2.263,1.860)--(2.257,1.857)--(2.257,1.854)%
  --(2.257,1.850)--(2.250,1.847)--(2.250,1.845)--(2.250,1.843)--(2.250,1.841)--(2.243,1.840)%
  --(2.243,1.839)--(2.243,1.837)--(2.243,1.835)--(2.237,1.830)--(2.230,1.827)--(2.223,1.826)%
  --(2.223,1.824)--(2.223,1.821)--(2.217,1.820)--(2.217,1.817)--(2.210,1.811)--(2.197,1.802)%
  --(2.190,1.800)--(2.190,1.799)--(2.190,1.797)--(2.183,1.794)--(2.177,1.789)--(2.177,1.786)%
  --(2.170,1.783)--(2.164,1.780)--(2.164,1.776)--(2.157,1.773)--(2.150,1.767)--(2.137,1.759)%
  --(2.130,1.753)--(2.130,1.751)--(2.130,1.749)--(2.130,1.748);
\gpcolor{color=gp lt color 6}
\gpsetlinetype{gp lt plot 6}
\draw[gp path] (9.436,7.453)--(9.430,7.458)--(9.416,7.456)--(9.410,7.453)--(9.403,7.450)%
  --(9.397,7.448)--(9.397,7.439)--(9.390,7.437)--(9.390,7.436)--(9.383,7.437)--(9.383,7.442)%
  --(9.377,7.437)--(9.370,7.432)--(9.363,7.430)--(9.363,7.429)--(9.357,7.418)--(9.350,7.411)%
  --(9.350,7.409)--(9.350,7.405)--(9.343,7.404)--(9.343,7.407)--(9.350,7.406)--(9.343,7.404)%
  --(9.337,7.402)--(9.343,7.398)--(9.330,7.388)--(9.337,7.386)--(9.323,7.383)--(9.323,7.384)%
  --(9.323,7.389)--(9.317,7.389)--(9.323,7.388)--(9.317,7.386)--(9.317,7.383)--(9.317,7.382)%
  --(9.317,7.380)--(9.317,7.371)--(9.310,7.371)--(9.310,7.368)--(9.317,7.373)--(9.310,7.371)%
  --(9.310,7.370)--(9.304,7.369)--(9.310,7.368)--(9.310,7.361)--(9.304,7.361)--(9.304,7.360)%
  --(9.304,7.358)--(9.304,7.365)--(9.297,7.363)--(9.297,7.362)--(9.290,7.360)--(9.297,7.358)%
  --(9.284,7.349)--(9.284,7.343)--(9.277,7.349)--(9.277,7.348)--(9.270,7.346)--(9.270,7.342)%
  --(9.264,7.335)--(9.257,7.332)--(9.257,7.331)--(9.257,7.329)--(9.257,7.317)--(9.250,7.325)%
  --(9.244,7.323)--(9.244,7.321)--(9.237,7.320)--(9.244,7.318)--(9.237,7.318)--(9.237,7.316)%
  --(9.230,7.311)--(9.230,7.301)--(9.230,7.306)--(9.224,7.304)--(9.224,7.300)--(9.224,7.295)%
  --(9.217,7.292)--(9.211,7.291)--(9.204,7.289)--(9.204,7.288)--(9.204,7.287)--(9.197,7.285)%
  --(9.197,7.284)--(9.191,7.287)--(9.191,7.286)--(9.191,7.285)--(9.191,7.283)--(9.184,7.281)%
  --(9.177,7.272)--(9.177,7.279)--(9.177,7.277)--(9.171,7.275)--(9.164,7.270)--(9.157,7.265)%
  --(9.151,7.263)--(9.157,7.261)--(9.151,7.260)--(9.157,7.258)--(9.151,7.253)--(9.144,7.258)%
  --(9.151,7.257)--(9.151,7.255)--(9.151,7.253)--(9.137,7.253)--(9.144,7.250)--(9.137,7.249)%
  --(9.131,7.249)--(9.131,7.237)--(9.131,7.241)--(9.131,7.239)--(9.124,7.237)--(9.124,7.235)%
  --(9.118,7.232)--(9.118,7.230)--(9.111,7.226)--(9.111,7.225)--(9.111,7.220)--(9.111,7.215)%
  --(9.104,7.217)--(9.104,7.215)--(9.098,7.210)--(9.098,7.204)--(9.091,7.199)--(9.091,7.198)%
  --(9.084,7.195)--(9.084,7.193)--(9.078,7.192)--(9.084,7.190)--(9.078,7.187)--(9.078,7.184)%
  --(9.078,7.188)--(9.071,7.187)--(9.078,7.187)--(9.071,7.187)--(9.071,7.185)--(9.071,7.184)%
  --(9.064,7.182)--(9.064,7.181)--(9.064,7.173)--(9.064,7.175)--(9.064,7.179)--(9.064,7.178)%
  --(9.058,7.171)--(9.051,7.165)--(9.045,7.162)--(9.038,7.153)--(9.031,7.149)--(9.031,7.147)%
  --(9.031,7.154)--(9.025,7.149)--(9.018,7.142)--(9.018,7.141)--(9.018,7.138)--(9.018,7.132)%
  --(9.011,7.130)--(9.018,7.128)--(9.005,7.134)--(9.011,7.135)--(9.005,7.132)--(8.998,7.129)%
  --(8.991,7.124)--(8.985,7.119)--(8.978,7.107)--(8.965,7.103)--(8.958,7.097)--(8.958,7.095)%
  --(8.958,7.098)--(8.952,7.096)--(8.945,7.092)--(8.938,7.089)--(8.932,7.084)--(8.932,7.082)%
  --(8.918,7.079)--(8.918,7.075)--(8.912,7.070)--(8.912,7.060)--(8.905,7.062)--(8.905,7.060)%
  --(8.898,7.059)--(8.898,7.053)--(8.885,7.048)--(8.885,7.044)--(8.892,7.044)--(8.885,7.041)%
  --(8.878,7.037)--(8.878,7.027)--(8.872,7.030)--(8.865,7.026)--(8.872,7.024)--(8.865,7.021)%
  --(8.859,7.019)--(8.859,7.008)--(8.852,7.003)--(8.845,7.005)--(8.839,7.002)--(8.839,7.000)%
  --(8.845,6.999)--(8.839,6.999)--(8.845,6.999)--(8.839,6.995)--(8.832,6.987)--(8.825,6.983)%
  --(8.825,6.981)--(8.825,6.978)--(8.819,6.983)--(8.819,6.982)--(8.819,6.979)--(8.812,6.978)%
  --(8.805,6.975)--(8.805,6.966)--(8.799,6.962)--(8.799,6.960)--(8.792,6.960)--(8.792,6.958)%
  --(8.785,6.956)--(8.785,6.954)--(8.772,6.951)--(8.766,6.948)--(8.766,6.945)--(8.759,6.943)%
  --(8.759,6.941)--(8.759,6.940)--(8.752,6.934)--(8.752,6.937)--(8.752,6.936)--(8.752,6.934)%
  --(8.746,6.931)--(8.746,6.922)--(8.739,6.919)--(8.739,6.918)--(8.726,6.923)--(8.726,6.921)%
  --(8.726,6.919)--(8.719,6.918)--(8.719,6.916)--(8.712,6.914)--(8.712,6.911)--(8.706,6.904)%
  --(8.706,6.906)--(8.706,6.905)--(8.699,6.904)--(8.699,6.902)--(8.692,6.899)--(8.686,6.896)%
  --(8.686,6.893)--(8.679,6.888)--(8.679,6.884)--(8.679,6.882)--(8.673,6.876)--(8.666,6.871)%
  --(8.666,6.867)--(8.659,6.864)--(8.653,6.861)--(8.653,6.856)--(8.646,6.851)--(8.646,6.845)%
  --(8.646,6.849)--(8.646,6.848)--(8.646,6.846)--(8.639,6.843)--(8.639,6.839)--(8.633,6.836)%
  --(8.626,6.833)--(8.619,6.828)--(8.619,6.825)--(8.613,6.823)--(8.613,6.822)--(8.613,6.821)%
  --(8.606,6.812)--(8.606,6.818)--(8.593,6.814)--(8.580,6.801)--(8.573,6.788)--(8.566,6.783)%
  --(8.560,6.773)--(8.566,6.782)--(8.560,6.781)--(8.560,6.779)--(8.546,6.775)--(8.540,6.769)%
  --(8.533,6.766)--(8.533,6.757)--(8.526,6.761)--(8.520,6.757)--(8.520,6.753)--(8.507,6.749)%
  --(8.507,6.747)--(8.500,6.744)--(8.493,6.740)--(8.493,6.735)--(8.480,6.731)--(8.473,6.719)%
  --(8.473,6.721)--(8.467,6.719)--(8.467,6.716)--(8.467,6.714)--(8.467,6.710)--(8.460,6.706)%
  --(8.447,6.704)--(8.440,6.700)--(8.440,6.691)--(8.440,6.690)--(8.440,6.688)--(8.433,6.685)%
  --(8.433,6.683)--(8.433,6.680)--(8.433,6.678)--(8.427,6.677)--(8.427,6.674)--(8.420,6.671)%
  --(8.414,6.666)--(8.407,6.664)--(8.407,6.661)--(8.400,6.658)--(8.400,6.648)--(8.387,6.650)%
  --(8.380,6.645)--(8.367,6.638)--(8.354,6.626)--(8.354,6.623)--(8.354,6.622)--(8.354,6.615)%
  --(8.354,6.622)--(8.347,6.620)--(8.334,6.614)--(8.334,6.608)--(8.327,6.603)--(8.314,6.596)%
  --(8.307,6.587)--(8.301,6.580)--(8.301,6.576)--(8.294,6.581)--(8.294,6.576)--(8.287,6.573)%
  --(8.287,6.570)--(8.274,6.563)--(8.267,6.558)--(8.261,6.554)--(8.254,6.543)--(8.241,6.538)%
  --(8.234,6.533)--(8.228,6.530)--(8.228,6.525)--(8.221,6.521)--(8.221,6.520)--(8.214,6.516)%
  --(8.214,6.511)--(8.208,6.506)--(8.201,6.499)--(8.201,6.496)--(8.194,6.493)--(8.194,6.497)%
  --(8.188,6.493)--(8.188,6.488)--(8.181,6.484)--(8.174,6.481)--(8.174,6.479)--(8.168,6.477)%
  --(8.168,6.474)--(8.161,6.470)--(8.148,6.461)--(8.135,6.450)--(8.128,6.446)--(8.128,6.445)%
  --(8.121,6.445)--(8.128,6.441)--(8.128,6.438)--(8.121,6.437)--(8.115,6.435)--(8.115,6.438)%
  --(8.101,6.431)--(8.081,6.415)--(8.081,6.411)--(8.068,6.403)--(8.055,6.395)--(8.048,6.383)%
  --(8.035,6.376)--(8.035,6.371)--(8.028,6.376)--(8.028,6.374)--(8.022,6.371)--(8.022,6.366)%
  --(8.002,6.359)--(7.995,6.353)--(7.988,6.339)--(7.982,6.334)--(7.975,6.330)--(7.975,6.326)%
  --(7.969,6.322)--(7.962,6.319)--(7.955,6.313)--(7.955,6.307)--(7.942,6.302)--(7.935,6.295)%
  --(7.929,6.284)--(7.915,6.277)--(7.902,6.268)--(7.895,6.269)--(7.889,6.264)--(7.889,6.259)%
  --(7.876,6.252)--(7.869,6.247)--(7.856,6.241)--(7.849,6.234)--(7.829,6.223)--(7.822,6.216)%
  --(7.809,6.208)--(7.802,6.199)--(7.783,6.193)--(7.783,6.188)--(7.769,6.182)--(7.763,6.177)%
  --(7.749,6.166)--(7.749,6.158)--(7.743,6.147)--(7.736,6.138)--(7.729,6.136)--(7.723,6.132)%
  --(7.716,6.124)--(7.703,6.117)--(7.703,6.113)--(7.696,6.110)--(7.696,6.104)--(7.696,6.110)%
  --(7.696,6.109)--(7.690,6.109)--(7.690,6.107)--(7.683,6.103)--(7.630,6.061)--(7.610,6.058)%
  --(7.597,6.048)--(7.590,6.040)--(7.583,6.033)--(7.570,6.025)--(7.563,6.019)--(7.550,6.009)%
  --(7.543,6.004)--(7.543,6.000)--(7.537,5.992)--(7.530,5.986)--(7.524,5.980)--(7.517,5.974)%
  --(7.510,5.967)--(7.497,5.955)--(7.490,5.950)--(7.484,5.944)--(7.477,5.938)--(7.470,5.932)%
  --(7.470,5.931)--(7.477,5.931)--(7.470,5.925)--(7.470,5.931)--(7.457,5.928)--(7.450,5.921)%
  --(7.404,5.893)--(7.397,5.887)--(7.384,5.873)--(7.371,5.866)--(7.371,5.864)--(7.371,5.857)%
  --(7.364,5.857)--(7.351,5.849)--(7.338,5.842)--(7.331,5.836)--(7.324,5.833)--(7.324,5.829)%
  --(7.318,5.825)--(7.318,5.821)--(7.311,5.813)--(7.311,5.816)--(7.304,5.812)--(7.304,5.809)%
  --(7.298,5.806)--(7.298,5.803)--(7.291,5.796)--(7.284,5.791)--(7.278,5.787)--(7.271,5.778)%
  --(7.258,5.771)--(7.251,5.765)--(7.245,5.760)--(7.238,5.758)--(7.245,5.756)--(7.238,5.754)%
  --(7.231,5.751)--(7.231,5.747)--(7.231,5.746)--(7.225,5.743)--(7.218,5.739)--(7.205,5.730)%
  --(7.198,5.727)--(7.198,5.724)--(7.198,5.722)--(7.185,5.718)--(7.171,5.711)--(7.165,5.703)%
  --(7.152,5.695)--(7.145,5.692)--(7.145,5.683)--(7.138,5.686)--(7.132,5.683)--(7.125,5.681)%
  --(7.125,5.678)--(7.118,5.675)--(7.112,5.671)--(7.105,5.663)--(7.105,5.665)--(7.098,5.663)%
  --(7.098,5.662)--(7.098,5.657)--(7.092,5.652)--(7.079,5.648)--(7.079,5.642)--(7.072,5.639)%
  --(7.072,5.630)--(7.065,5.630)--(7.059,5.624)--(7.052,5.621)--(7.045,5.616)--(7.045,5.607)%
  --(7.032,5.600)--(7.019,5.591)--(7.012,5.586)--(7.012,5.584)--(6.999,5.572)--(6.992,5.565)%
  --(6.972,5.564)--(6.966,5.558)--(6.966,5.549)--(6.952,5.539)--(6.926,5.531)--(6.919,5.523)%
  --(6.906,5.517)--(6.899,5.511)--(6.886,5.502)--(6.879,5.496)--(6.873,5.491)--(6.866,5.489)%
  --(6.866,5.486)--(6.859,5.482)--(6.859,5.479)--(6.853,5.474)--(6.846,5.469)--(6.846,5.467)%
  --(6.839,5.463)--(6.839,5.460)--(6.839,5.454)--(6.839,5.452)--(6.826,5.450)--(6.819,5.444)%
  --(6.819,5.438)--(6.806,5.433)--(6.806,5.427)--(6.800,5.418)--(6.793,5.417)--(6.793,5.420)%
  --(6.793,5.418)--(6.793,5.416)--(6.760,5.401)--(6.726,5.377)--(6.726,5.374)--(6.720,5.369)%
  --(6.713,5.360)--(6.707,5.354)--(6.700,5.353)--(6.687,5.346)--(6.680,5.340)--(6.680,5.337)%
  --(6.673,5.333)--(6.660,5.329)--(6.647,5.319)--(6.647,5.311)--(6.647,5.307)--(6.634,5.310)%
  --(6.634,5.308)--(6.627,5.306)--(6.627,5.305)--(6.627,5.301)--(6.620,5.296)--(6.614,5.289)%
  --(6.607,5.281)--(6.594,5.270)--(6.587,5.265)--(6.580,5.259)--(6.574,5.254)--(6.567,5.249)%
  --(6.560,5.242)--(6.547,5.233)--(6.541,5.230)--(6.541,5.228)--(6.541,5.226)--(6.527,5.217)%
  --(6.514,5.202)--(6.514,5.201)--(6.514,5.203)--(6.507,5.198)--(6.501,5.194)--(6.494,5.189)%
  --(6.494,5.185)--(6.487,5.177)--(6.481,5.174)--(6.474,5.177)--(6.467,5.175)--(6.467,5.171)%
  --(6.461,5.167)--(6.454,5.166)--(6.448,5.163)--(6.448,5.161)--(6.448,5.152)--(6.441,5.152)%
  --(6.441,5.153)--(6.441,5.155)--(6.434,5.153)--(6.434,5.151)--(6.421,5.142)--(6.408,5.133)%
  --(6.408,5.130)--(6.401,5.121)--(6.394,5.116)--(6.388,5.115)--(6.388,5.112)--(6.381,5.108)%
  --(6.381,5.106)--(6.374,5.101)--(6.374,5.096)--(6.368,5.091)--(6.368,5.092)--(6.368,5.089)%
  --(6.368,5.091)--(6.361,5.091)--(6.361,5.085)--(6.355,5.081)--(6.348,5.078)--(6.348,5.073)%
  --(6.335,5.069)--(6.335,5.065)--(6.328,5.060)--(6.315,5.052)--(6.308,5.049)--(6.301,5.044)%
  --(6.295,5.040)--(6.288,5.034)--(6.275,5.026)--(6.222,4.990)--(6.208,4.984)--(6.208,4.980)%
  --(6.202,4.976)--(6.195,4.973)--(6.202,4.971)--(6.195,4.969)--(6.195,4.967)--(6.188,4.965)%
  --(6.188,4.962)--(6.188,4.954)--(6.182,4.956)--(6.175,4.950)--(6.149,4.927)--(6.142,4.923)%
  --(6.142,4.920)--(6.135,4.916)--(6.135,4.915)--(6.135,4.914)--(6.135,4.909)--(6.135,4.912)%
  --(6.135,4.913)--(6.135,4.912)--(6.135,4.911)--(6.129,4.910)--(6.129,4.907)--(6.129,4.904)%
  --(6.129,4.901)--(6.122,4.898)--(6.109,4.890)--(6.109,4.893)--(6.109,4.889)--(6.102,4.887)%
  --(6.096,4.883)--(6.089,4.879)--(6.082,4.871)--(6.069,4.861)--(6.056,4.854)--(6.049,4.847)%
  --(6.029,4.838)--(6.022,4.828)--(6.009,4.817)--(6.009,4.818)--(6.009,4.816)--(6.003,4.814)%
  --(6.003,4.811)--(5.996,4.809)--(5.996,4.806)--(5.989,4.799)--(5.983,4.801)--(5.983,4.799)%
  --(5.983,4.795)--(5.976,4.793)--(5.976,4.790)--(5.969,4.785)--(5.963,4.781)--(5.956,4.774)%
  --(5.949,4.768)--(5.949,4.764)--(5.949,4.766)--(5.949,4.765)--(5.943,4.760)--(5.936,4.755)%
  --(5.929,4.750)--(5.929,4.745)--(5.923,4.740)--(5.916,4.737)--(5.916,4.731)--(5.903,4.726)%
  --(5.896,4.725)--(5.890,4.718)--(5.876,4.709)--(5.870,4.704)--(5.870,4.702)--(5.863,4.699)%
  --(5.850,4.687)--(5.843,4.679)--(5.836,4.674)--(5.830,4.671)--(5.823,4.669)--(5.817,4.665)%
  --(5.810,4.659)--(5.803,4.654)--(5.797,4.649)--(5.790,4.643)--(5.783,4.638)--(5.777,4.635)%
  --(5.770,4.630)--(5.770,4.628)--(5.763,4.624)--(5.750,4.616)--(5.750,4.613)--(5.743,4.608)%
  --(5.737,4.603)--(5.737,4.596)--(5.730,4.593)--(5.724,4.589)--(5.717,4.582)--(5.704,4.576)%
  --(5.704,4.572)--(5.704,4.570)--(5.697,4.568)--(5.697,4.565)--(5.690,4.559)--(5.684,4.554)%
  --(5.670,4.549)--(5.657,4.544)--(5.651,4.540)--(5.644,4.535)--(5.637,4.529)--(5.631,4.525)%
  --(5.631,4.523)--(5.624,4.516)--(5.617,4.506)--(5.611,4.505)--(5.597,4.498)--(5.591,4.491)%
  --(5.577,4.488)--(5.577,4.483)--(5.564,4.477)--(5.564,4.472)--(5.551,4.465)--(5.544,4.457)%
  --(5.538,4.449)--(5.531,4.448)--(5.524,4.445)--(5.518,4.436)--(5.511,4.427)--(5.498,4.421)%
  --(5.491,4.417)--(5.491,4.414)--(5.484,4.406)--(5.471,4.396)--(5.465,4.389)--(5.458,4.388)%
  --(5.451,4.384)--(5.451,4.381)--(5.445,4.377)--(5.438,4.372)--(5.431,4.366)--(5.425,4.364)%
  --(5.418,4.357)--(5.411,4.351)--(5.398,4.345)--(5.385,4.339)--(5.385,4.334)--(5.378,4.328)%
  --(5.372,4.324)--(5.365,4.318)--(5.352,4.312)--(5.352,4.309)--(5.345,4.302)--(5.338,4.299)%
  --(5.338,4.298)--(5.338,4.295)--(5.325,4.289)--(5.325,4.285)--(5.325,4.281)--(5.312,4.277)%
  --(5.305,4.271)--(5.298,4.262)--(5.292,4.261)--(5.285,4.253)--(5.279,4.247)--(5.272,4.243)%
  --(5.265,4.238)--(5.259,4.234)--(5.259,4.229)--(5.245,4.224)--(5.239,4.216)--(5.239,4.211)%
  --(5.232,4.210)--(5.225,4.207)--(5.225,4.203)--(5.219,4.201)--(5.212,4.193)--(5.199,4.188)%
  --(5.186,4.180)--(5.179,4.173)--(5.172,4.168)--(5.166,4.162)--(5.159,4.152)--(5.146,4.145)%
  --(5.146,4.141)--(5.132,4.134)--(5.119,4.127)--(5.113,4.121)--(5.106,4.114)--(5.093,4.102)%
  --(5.086,4.098)--(5.086,4.097)--(5.079,4.094)--(5.073,4.089)--(5.066,4.084)--(5.059,4.076)%
  --(5.053,4.071)--(5.053,4.068)--(5.046,4.062)--(5.039,4.053)--(5.033,4.055)--(5.033,4.053)%
  --(5.033,4.051)--(5.026,4.045)--(5.013,4.038)--(5.013,4.036)--(5.013,4.033)--(5.006,4.030)%
  --(5.000,4.022)--(5.000,4.023)--(4.993,4.021)--(4.993,4.020)--(4.986,4.017)--(4.980,4.014)%
  --(4.973,4.008)--(4.960,3.999)--(4.953,3.994)--(4.946,3.990)--(4.940,3.983)--(4.933,3.981)%
  --(4.920,3.974)--(4.913,3.967)--(4.907,3.961)--(4.893,3.955)--(4.893,3.951)--(4.887,3.947)%
  --(4.880,3.937)--(4.867,3.927)--(4.860,3.924)--(4.853,3.922)--(4.840,3.912)--(4.834,3.904)%
  --(4.834,3.900)--(4.827,3.900)--(4.827,3.896)--(4.820,3.890)--(4.814,3.883)--(4.800,3.871)%
  --(4.794,3.867)--(4.787,3.865)--(4.787,3.862)--(4.780,3.859)--(4.780,3.858)--(4.780,3.856)%
  --(4.774,3.853)--(4.767,3.849)--(4.760,3.845)--(4.754,3.840)--(4.747,3.838)--(4.734,3.829)%
  --(4.727,3.818)--(4.714,3.813)--(4.701,3.806)--(4.701,3.800)--(4.694,3.796)--(4.681,3.790)%
  --(4.674,3.781)--(4.661,3.775)--(4.654,3.771)--(4.648,3.766)--(4.648,3.764)--(4.648,3.761)%
  --(4.641,3.756)--(4.634,3.753)--(4.634,3.751)--(4.628,3.746)--(4.621,3.738)--(4.621,3.737)%
  --(4.614,3.736)--(4.608,3.729)--(4.601,3.723)--(4.594,3.716)--(4.588,3.712)--(4.588,3.708)%
  --(4.581,3.702)--(4.575,3.696)--(4.575,3.694)--(4.568,3.691)--(4.561,3.683)--(4.555,3.681)%
  --(4.555,3.682)--(4.548,3.681)--(4.541,3.677)--(4.535,3.669)--(4.528,3.663)--(4.521,3.658)%
  --(4.521,3.654)--(4.508,3.653)--(4.501,3.646)--(4.495,3.641)--(4.495,3.639)--(4.482,3.634)%
  --(4.475,3.624)--(4.468,3.619)--(4.462,3.614)--(4.462,3.610)--(4.455,3.612)--(4.448,3.607)%
  --(4.442,3.603)--(4.435,3.598)--(4.428,3.590)--(4.415,3.580)--(4.408,3.575)--(4.408,3.573)%
  --(4.408,3.570)--(4.408,3.569)--(4.402,3.567)--(4.395,3.561)--(4.395,3.556)--(4.389,3.550)%
  --(4.375,3.544)--(4.375,3.540)--(4.369,3.534)--(4.362,3.525)--(4.355,3.520)--(4.349,3.516)%
  --(4.342,3.514)--(4.335,3.509)--(4.329,3.501)--(4.315,3.496)--(4.309,3.492)--(4.296,3.482)%
  --(4.282,3.473)--(4.276,3.468)--(4.262,3.463)--(4.256,3.455)--(4.249,3.449)--(4.236,3.441)%
  --(4.236,3.439)--(4.229,3.438)--(4.229,3.437)--(4.229,3.434)--(4.222,3.427)--(4.216,3.422)%
  --(4.209,3.421)--(4.209,3.419)--(4.209,3.418)--(4.209,3.415)--(4.203,3.411)--(4.196,3.407)%
  --(4.196,3.405)--(4.189,3.404)--(4.189,3.400)--(4.183,3.395)--(4.176,3.389)--(4.169,3.385)%
  --(4.163,3.379)--(4.156,3.373)--(4.143,3.365)--(4.136,3.358)--(4.130,3.351)--(4.130,3.344)%
  --(4.123,3.343)--(4.116,3.338)--(4.110,3.334)--(4.103,3.330)--(4.096,3.324)--(4.096,3.320)%
  --(4.090,3.318)--(4.083,3.314)--(4.076,3.308)--(4.070,3.307)--(4.070,3.306)--(4.063,3.304)%
  --(4.056,3.298)--(4.043,3.290)--(4.037,3.283)--(4.030,3.280)--(4.023,3.275)--(4.017,3.269)%
  --(4.003,3.265)--(3.997,3.256)--(3.983,3.248)--(3.970,3.240)--(3.963,3.231)--(3.957,3.226)%
  --(3.950,3.222)--(3.950,3.218)--(3.944,3.211)--(3.930,3.201)--(3.924,3.194)--(3.917,3.186)%
  --(3.904,3.180)--(3.904,3.176)--(3.897,3.172)--(3.890,3.169)--(3.890,3.167)--(3.890,3.164)%
  --(3.884,3.159)--(3.877,3.156)--(3.870,3.156)--(3.864,3.148)--(3.857,3.144)--(3.851,3.138)%
  --(3.844,3.132)--(3.837,3.130)--(3.831,3.130)--(3.831,3.125)--(3.824,3.115)--(3.817,3.110)%
  --(3.811,3.107)--(3.797,3.102)--(3.791,3.097)--(3.784,3.090)--(3.777,3.085)--(3.764,3.074)%
  --(3.758,3.066)--(3.744,3.056)--(3.731,3.047)--(3.724,3.042)--(3.724,3.040)--(3.718,3.037)%
  --(3.711,3.030)--(3.704,3.023)--(3.698,3.012)--(3.685,3.007)--(3.678,3.001)--(3.678,2.997)%
  --(3.671,2.993)--(3.658,2.986)--(3.651,2.983)--(3.645,2.979)--(3.638,2.972)--(3.625,2.964)%
  --(3.611,2.957)--(3.598,2.949)--(3.585,2.940)--(3.578,2.929)--(3.572,2.925)--(3.565,2.920)%
  --(3.558,2.914)--(3.552,2.905)--(3.532,2.893)--(3.525,2.884)--(3.512,2.875)--(3.505,2.869)%
  --(3.492,2.861)--(3.492,2.857)--(3.485,2.849)--(3.465,2.839)--(3.459,2.830)--(3.452,2.824)%
  --(3.445,2.820)--(3.445,2.817)--(3.432,2.809)--(3.425,2.803)--(3.419,2.799)--(3.412,2.795)%
  --(3.406,2.789)--(3.406,2.788)--(3.406,2.789)--(3.406,2.786)--(3.392,2.777)--(3.386,2.774)%
  --(3.379,2.772)--(3.372,2.765)--(3.359,2.755)--(3.352,2.749)--(3.339,2.741)--(3.332,2.739)%
  --(3.326,2.731)--(3.313,2.720)--(3.299,2.714)--(3.293,2.708)--(3.286,2.700)--(3.286,2.696)%
  --(3.286,2.693)--(3.279,2.690)--(3.273,2.689)--(3.273,2.685)--(3.273,2.684)--(3.266,2.681)%
  --(3.259,2.679)--(3.253,2.675)--(3.253,2.668)--(3.246,2.661)--(3.233,2.652)--(3.233,2.647)%
  --(3.233,2.644)--(3.226,2.642)--(3.226,2.639)--(3.220,2.637)--(3.220,2.633)--(3.206,2.627)%
  --(3.200,2.620)--(3.193,2.616)--(3.193,2.612)--(3.180,2.605)--(3.173,2.601)--(3.166,2.598)%
  --(3.153,2.595)--(3.147,2.587)--(3.140,2.583)--(3.133,2.578)--(3.133,2.577)--(3.120,2.566)%
  --(3.100,2.556)--(3.093,2.550)--(3.087,2.545)--(3.087,2.543)--(3.087,2.537)--(3.080,2.532)%
  --(3.073,2.530)--(3.073,2.528)--(3.067,2.524)--(3.060,2.516)--(3.054,2.511)--(3.047,2.510)%
  --(3.047,2.506)--(3.034,2.497)--(3.027,2.489)--(3.020,2.485)--(3.014,2.478)--(3.007,2.473)%
  --(3.000,2.468)--(2.994,2.466)--(2.994,2.463)--(2.987,2.458)--(2.980,2.449)--(2.974,2.444)%
  --(2.967,2.443)--(2.967,2.441)--(2.961,2.434)--(2.947,2.427)--(2.934,2.420)--(2.927,2.416)%
  --(2.914,2.410)--(2.907,2.403)--(2.901,2.397)--(2.894,2.394)--(2.887,2.391)--(2.881,2.382)%
  --(2.868,2.372)--(2.854,2.364)--(2.848,2.357)--(2.841,2.350)--(2.834,2.346)--(2.834,2.345)%
  --(2.828,2.340)--(2.821,2.332)--(2.821,2.329)--(2.814,2.327)--(2.808,2.321)--(2.794,2.310)%
  --(2.788,2.302)--(2.781,2.298)--(2.775,2.294)--(2.775,2.291)--(2.768,2.287)--(2.768,2.285)%
  --(2.761,2.279)--(2.755,2.278)--(2.748,2.274)--(2.735,2.262)--(2.721,2.254)--(2.715,2.250)%
  --(2.702,2.245)--(2.695,2.239)--(2.688,2.234)--(2.682,2.228)--(2.675,2.219)--(2.668,2.216)%
  --(2.668,2.215)--(2.662,2.214)--(2.662,2.211)--(2.648,2.203)--(2.642,2.196)--(2.635,2.192)%
  --(2.628,2.183)--(2.615,2.175)--(2.615,2.174)--(2.609,2.171)--(2.609,2.165)--(2.602,2.160)%
  --(2.595,2.155)--(2.595,2.151)--(2.589,2.147)--(2.582,2.142)--(2.575,2.137)--(2.569,2.133)%
  --(2.562,2.132)--(2.555,2.127)--(2.549,2.122)--(2.542,2.114)--(2.535,2.109)--(2.529,2.104)%
  --(2.529,2.101)--(2.522,2.098)--(2.509,2.091)--(2.502,2.083)--(2.489,2.076)--(2.482,2.072)%
  --(2.469,2.063)--(2.462,2.057)--(2.449,2.049)--(2.442,2.041)--(2.436,2.036)--(2.429,2.031)%
  --(2.423,2.024)--(2.416,2.020)--(2.416,2.018)--(2.409,2.016)--(2.409,2.013)--(2.409,2.012)%
  --(2.409,2.009)--(2.409,2.006)--(2.403,2.002)--(2.396,1.996)--(2.389,1.992)--(2.389,1.990)%
  --(2.389,1.989)--(2.383,1.984)--(2.369,1.977)--(2.356,1.970)--(2.356,1.965)--(2.349,1.964)%
  --(2.349,1.960)--(2.343,1.956)--(2.336,1.950)--(2.336,1.947)--(2.330,1.945)--(2.323,1.938)%
  --(2.310,1.931)--(2.296,1.923)--(2.283,1.913)--(2.276,1.905)--(2.270,1.904)--(2.270,1.903)%
  --(2.270,1.900)--(2.263,1.894)--(2.257,1.889)--(2.257,1.888)--(2.250,1.886)--(2.243,1.884)%
  --(2.243,1.883)--(2.237,1.881)--(2.230,1.874)--(2.223,1.870)--(2.223,1.869)--(2.217,1.865)%
  --(2.210,1.857)--(2.190,1.843)--(2.183,1.840)--(2.183,1.836)--(2.177,1.829)--(2.150,1.815)%
  --(2.150,1.811)--(2.150,1.810)--(2.144,1.806)--(2.137,1.801)--(2.137,1.797)--(2.130,1.793)%
  --(2.124,1.788)--(2.117,1.781)--(2.110,1.777)--(2.104,1.773)--(2.097,1.767);
\gpcolor{color=gp lt color border}
\gpsetlinetype{gp lt border}
\draw[gp path] (1.320,7.524)--(1.320,1.705);
\draw[gp path] (2.097,0.985)--(10.898,0.985);
%% coordinates of the plot area
\gpdefrectangularnode{gp plot 1}{\pgfpoint{1.320cm}{0.985cm}}{\pgfpoint{11.947cm}{7.825cm}}
\end{tikzpicture}
%% gnuplot variables

\caption{Le varie isoterme dell'esperimento (dati in andata)}
\label{img:isoa}
\end{grafico}

\begin{grafico}
  \centering
\begin{tikzpicture}[gnuplot]
%% generated with GNUPLOT 4.6p3 (Lua 5.1; terminal rev. 99, script rev. 100)
%% mar 27 mag 2014 22:34:39 CEST
\path (0.000,0.000) rectangle (12.500,8.750);
\gpcolor{color=gp lt color border}
\gpsetlinetype{gp lt border}
\gpsetlinewidth{1.00}
\draw[gp path] (1.320,1.745)--(1.500,1.745);
\node[gp node right] at (1.136,1.745) { 6};
\draw[gp path] (1.320,2.505)--(1.500,2.505);
\node[gp node right] at (1.136,2.505) { 8};
\draw[gp path] (1.320,3.265)--(1.500,3.265);
\node[gp node right] at (1.136,3.265) { 10};
\draw[gp path] (1.320,4.025)--(1.500,4.025);
\node[gp node right] at (1.136,4.025) { 12};
\draw[gp path] (1.320,4.785)--(1.500,4.785);
\node[gp node right] at (1.136,4.785) { 14};
\draw[gp path] (1.320,5.545)--(1.500,5.545);
\node[gp node right] at (1.136,5.545) { 16};
\draw[gp path] (1.320,6.305)--(1.500,6.305);
\node[gp node right] at (1.136,6.305) { 18};
\draw[gp path] (1.320,7.065)--(1.500,7.065);
\node[gp node right] at (1.136,7.065) { 20};
\draw[gp path] (2.648,0.985)--(2.648,1.165);
\node[gp node center] at (2.648,0.677) { 0.6};
\draw[gp path] (3.977,0.985)--(3.977,1.165);
\node[gp node center] at (3.977,0.677) { 0.8};
\draw[gp path] (5.305,0.985)--(5.305,1.165);
\node[gp node center] at (5.305,0.677) { 1};
\draw[gp path] (6.634,0.985)--(6.634,1.165);
\node[gp node center] at (6.634,0.677) { 1.2};
\draw[gp path] (7.962,0.985)--(7.962,1.165);
\node[gp node center] at (7.962,0.677) { 1.4};
\draw[gp path] (9.290,0.985)--(9.290,1.165);
\node[gp node center] at (9.290,0.677) { 1.6};
\draw[gp path] (10.619,0.985)--(10.619,1.165);
\node[gp node center] at (10.619,0.677) { 1.8};
\draw[gp path] (1.320,7.496)--(1.320,1.708);
\draw[gp path] (2.057,0.985)--(10.825,0.985);
\node[gp node center,rotate=-270] at (0.246,4.405) {Volume $[cm^3]$};
\node[gp node center] at (6.633,0.215) {Inverso della pressione $[\frac{cm^2}{Kgf}]$};
\node[gp node center] at (6.633,8.287) {Prima giornata};
\gpcolor{color=gp lt color 0}
\gpsetlinetype{gp lt plot 0}
\draw[gp path] (2.369,1.708)--(2.376,1.709)--(2.376,1.710)--(2.383,1.710)--(2.383,1.711)%
  --(2.389,1.713)--(2.396,1.715)--(2.403,1.717)--(2.409,1.720)--(2.416,1.723)--(2.416,1.724)%
  --(2.423,1.728)--(2.429,1.730)--(2.436,1.733)--(2.436,1.734)--(2.442,1.735)--(2.442,1.738)%
  --(2.449,1.743)--(2.449,1.745)--(2.456,1.747)--(2.462,1.751)--(2.469,1.756)--(2.469,1.759)%
  --(2.469,1.760)--(2.469,1.761)--(2.476,1.761)--(2.476,1.764)--(2.482,1.769)--(2.482,1.770)%
  --(2.489,1.775)--(2.489,1.776)--(2.489,1.777)--(2.496,1.777)--(2.496,1.779)--(2.502,1.780)%
  --(2.502,1.783)--(2.509,1.787)--(2.516,1.790)--(2.522,1.794)--(2.529,1.798)--(2.529,1.800)%
  --(2.535,1.802)--(2.535,1.803)--(2.542,1.806)--(2.542,1.808)--(2.542,1.810)--(2.549,1.812)%
  --(2.549,1.814)--(2.555,1.818)--(2.562,1.823)--(2.569,1.827)--(2.575,1.829)--(2.582,1.831)%
  --(2.582,1.834)--(2.589,1.839)--(2.595,1.843)--(2.602,1.846)--(2.609,1.851)--(2.615,1.858)%
  --(2.622,1.861)--(2.622,1.863)--(2.628,1.867)--(2.628,1.871)--(2.635,1.875)--(2.648,1.880)%
  --(2.648,1.883)--(2.655,1.886)--(2.662,1.887)--(2.668,1.891)--(2.675,1.895)--(2.682,1.900)%
  --(2.682,1.904)--(2.688,1.907)--(2.695,1.910)--(2.702,1.916)--(2.708,1.922)--(2.715,1.926)%
  --(2.715,1.927)--(2.721,1.928)--(2.721,1.931)--(2.721,1.932)--(2.728,1.933)--(2.728,1.935)%
  --(2.735,1.938)--(2.741,1.942)--(2.741,1.945)--(2.748,1.947)--(2.755,1.951)--(2.755,1.954)%
  --(2.761,1.958)--(2.761,1.960)--(2.761,1.961)--(2.768,1.964)--(2.768,1.965)--(2.768,1.968)%
  --(2.781,1.972)--(2.788,1.976)--(2.794,1.980)--(2.794,1.983)--(2.801,1.984)--(2.801,1.986)%
  --(2.808,1.990)--(2.821,1.998)--(2.834,2.005)--(2.841,2.010)--(2.841,2.012)--(2.841,2.016)%
  --(2.848,2.017)--(2.848,2.020)--(2.848,2.021)--(2.854,2.024)--(2.861,2.026)--(2.868,2.030)%
  --(2.868,2.033)--(2.874,2.036)--(2.874,2.039)--(2.881,2.040)--(2.881,2.042)--(2.887,2.046)%
  --(2.894,2.049)--(2.901,2.054)--(2.907,2.058)--(2.914,2.063)--(2.914,2.065)--(2.921,2.070)%
  --(2.927,2.075)--(2.934,2.077)--(2.934,2.079)--(2.934,2.080)--(2.934,2.081)--(2.934,2.082)%
  --(2.941,2.084)--(2.941,2.086)--(2.941,2.088)--(2.947,2.090)--(2.947,2.092)--(2.954,2.095)%
  --(2.961,2.097)--(2.961,2.100)--(2.961,2.102)--(2.967,2.104)--(2.974,2.109)--(2.980,2.114)%
  --(2.987,2.116)--(2.994,2.123)--(3.000,2.127)--(3.007,2.130)--(3.014,2.133)--(3.014,2.136)%
  --(3.020,2.137)--(3.020,2.138)--(3.020,2.141)--(3.027,2.142)--(3.034,2.146)--(3.034,2.149)%
  --(3.040,2.151)--(3.047,2.157)--(3.054,2.161)--(3.060,2.165)--(3.067,2.170)--(3.073,2.176)%
  --(3.073,2.178)--(3.080,2.179)--(3.087,2.182)--(3.087,2.185)--(3.093,2.188)--(3.100,2.194)%
  --(3.113,2.198)--(3.120,2.204)--(3.127,2.210)--(3.133,2.214)--(3.140,2.216)--(3.147,2.219)%
  --(3.153,2.224)--(3.160,2.229)--(3.160,2.233)--(3.166,2.238)--(3.173,2.243)--(3.180,2.246)%
  --(3.180,2.249)--(3.186,2.252)--(3.186,2.253)--(3.193,2.256)--(3.200,2.258)--(3.200,2.261)%
  --(3.206,2.263)--(3.213,2.266)--(3.213,2.268)--(3.220,2.271)--(3.220,2.274)--(3.220,2.276)%
  --(3.226,2.279)--(3.233,2.282)--(3.233,2.285)--(3.240,2.289)--(3.240,2.291)--(3.246,2.294)%
  --(3.246,2.297)--(3.253,2.299)--(3.253,2.303)--(3.259,2.304)--(3.259,2.305)--(3.266,2.307)%
  --(3.266,2.309)--(3.266,2.311)--(3.273,2.313)--(3.279,2.318)--(3.286,2.321)--(3.293,2.326)%
  --(3.293,2.327)--(3.299,2.331)--(3.306,2.335)--(3.313,2.340)--(3.313,2.342)--(3.319,2.344)%
  --(3.326,2.348)--(3.326,2.350)--(3.332,2.352)--(3.339,2.357)--(3.346,2.361)--(3.352,2.365)%
  --(3.359,2.369)--(3.359,2.372)--(3.359,2.374)--(3.366,2.377)--(3.372,2.378)--(3.372,2.381)%
  --(3.379,2.384)--(3.386,2.387)--(3.386,2.391)--(3.392,2.395)--(3.392,2.397)--(3.399,2.399)%
  --(3.399,2.401)--(3.406,2.402)--(3.406,2.406)--(3.412,2.408)--(3.419,2.412)--(3.419,2.415)%
  --(3.425,2.416)--(3.425,2.420)--(3.432,2.424)--(3.439,2.425)--(3.439,2.429)--(3.445,2.432)%
  --(3.452,2.434)--(3.459,2.438)--(3.459,2.439)--(3.459,2.440)--(3.459,2.441)--(3.459,2.443)%
  --(3.465,2.445)--(3.465,2.448)--(3.465,2.450)--(3.472,2.452)--(3.472,2.456)--(3.479,2.460)%
  --(3.485,2.462)--(3.485,2.463)--(3.485,2.466)--(3.492,2.469)--(3.492,2.472)--(3.499,2.475)%
  --(3.505,2.477)--(3.505,2.478)--(3.512,2.482)--(3.518,2.484)--(3.525,2.488)--(3.525,2.491)%
  --(3.532,2.494)--(3.538,2.496)--(3.545,2.499)--(3.552,2.503)--(3.558,2.510)--(3.565,2.513)%
  --(3.565,2.514)--(3.565,2.515)--(3.565,2.517)--(3.572,2.519)--(3.578,2.525)--(3.585,2.528)%
  --(3.598,2.534)--(3.605,2.540)--(3.625,2.548)--(3.631,2.557)--(3.651,2.565)--(3.665,2.575)%
  --(3.671,2.583)--(3.685,2.591)--(3.698,2.599)--(3.698,2.604)--(3.704,2.607)--(3.718,2.611)%
  --(3.718,2.617)--(3.724,2.623)--(3.731,2.627)--(3.738,2.632)--(3.744,2.636)--(3.751,2.639)%
  --(3.751,2.641)--(3.751,2.643)--(3.758,2.646)--(3.764,2.648)--(3.764,2.651)--(3.771,2.655)%
  --(3.771,2.656)--(3.777,2.658)--(3.777,2.661)--(3.784,2.664)--(3.791,2.667)--(3.791,2.670)%
  --(3.797,2.675)--(3.797,2.677)--(3.804,2.681)--(3.811,2.685)--(3.817,2.686)--(3.831,2.692)%
  --(3.844,2.699)--(3.851,2.706)--(3.864,2.716)--(3.870,2.721)--(3.877,2.722)--(3.884,2.726)%
  --(3.884,2.730)--(3.890,2.736)--(3.897,2.741)--(3.910,2.747)--(3.917,2.754)--(3.930,2.761)%
  --(3.937,2.768)--(3.944,2.771)--(3.937,2.772)--(3.944,2.774)--(3.950,2.777)--(3.950,2.779)%
  --(3.950,2.783)--(3.957,2.786)--(3.957,2.787)--(3.963,2.789)--(3.963,2.790)--(3.963,2.791)%
  --(3.963,2.794)--(3.970,2.797)--(3.970,2.802)--(3.970,2.805)--(3.977,2.806)--(3.977,2.807)%
  --(3.983,2.809)--(3.983,2.811)--(3.990,2.813)--(3.997,2.817)--(4.003,2.822)--(4.010,2.824)%
  --(4.010,2.826)--(4.010,2.828)--(4.017,2.830)--(4.023,2.833)--(4.023,2.836)--(4.030,2.839)%
  --(4.037,2.842)--(4.037,2.845)--(4.043,2.850)--(4.056,2.853)--(4.056,2.856)--(4.063,2.861)%
  --(4.063,2.863)--(4.070,2.865)--(4.076,2.869)--(4.090,2.874)--(4.096,2.880)--(4.103,2.888)%
  --(4.110,2.890)--(4.110,2.893)--(4.123,2.898)--(4.123,2.901)--(4.130,2.902)--(4.136,2.908)%
  --(4.143,2.914)--(4.149,2.918)--(4.156,2.923)--(4.156,2.925)--(4.149,2.926)--(4.156,2.926)%
  --(4.149,2.925)--(4.156,2.925)--(4.163,2.931)--(4.183,2.945)--(4.209,2.958)--(4.209,2.963)%
  --(4.209,2.964)--(4.209,2.965)--(4.222,2.972)--(4.242,2.988)--(4.256,2.997)--(4.262,2.998)%
  --(4.262,2.999)--(4.256,2.999)--(4.256,3.000)--(4.262,3.000)--(4.262,3.002)--(4.269,3.004)%
  --(4.269,3.006)--(4.276,3.009)--(4.276,3.010)--(4.276,3.013)--(4.282,3.015)--(4.289,3.018)%
  --(4.289,3.016)--(4.289,3.018)--(4.296,3.018)--(4.302,3.023)--(4.302,3.026)--(4.309,3.030)%
  --(4.309,3.033)--(4.315,3.036)--(4.322,3.039)--(4.329,3.043)--(4.335,3.049)--(4.342,3.052)%
  --(4.349,3.055)--(4.355,3.057)--(4.362,3.062)--(4.369,3.066)--(4.375,3.069)--(4.382,3.074)%
  --(4.389,3.080)--(4.395,3.087)--(4.402,3.093)--(4.402,3.098)--(4.408,3.100)--(4.422,3.104)%
  --(4.422,3.110)--(4.435,3.116)--(4.448,3.122)--(4.455,3.129)--(4.455,3.134)--(4.462,3.136)%
  --(4.468,3.139)--(4.468,3.140)--(4.475,3.142)--(4.482,3.148)--(4.482,3.149)--(4.482,3.153)%
  --(4.488,3.157)--(4.495,3.160)--(4.495,3.162)--(4.501,3.167)--(4.508,3.167)--(4.515,3.168)%
  --(4.515,3.171)--(4.515,3.173)--(4.521,3.177)--(4.521,3.181)--(4.528,3.184)--(4.528,3.185)%
  --(4.535,3.185)--(4.535,3.186)--(4.541,3.189)--(4.541,3.191)--(4.548,3.194)--(4.548,3.195)%
  --(4.555,3.201)--(4.561,3.205)--(4.575,3.210)--(4.575,3.214)--(4.581,3.218)--(4.588,3.221)%
  --(4.588,3.222)--(4.601,3.227)--(4.608,3.233)--(4.614,3.240)--(4.628,3.248)--(4.641,3.257)%
  --(4.648,3.259)--(4.648,3.265)--(4.661,3.269)--(4.661,3.274)--(4.668,3.276)--(4.674,3.282)%
  --(4.681,3.286)--(4.687,3.291)--(4.701,3.300)--(4.714,3.306)--(4.721,3.310)--(4.727,3.315)%
  --(4.727,3.319)--(4.727,3.320)--(4.734,3.322)--(4.734,3.324)--(4.734,3.325)--(4.741,3.327)%
  --(4.747,3.330)--(4.747,3.333)--(4.754,3.338)--(4.760,3.339)--(4.760,3.341)--(4.767,3.344)%
  --(4.767,3.347)--(4.774,3.348)--(4.774,3.351)--(4.774,3.353)--(4.780,3.355)--(4.787,3.357)%
  --(4.787,3.358)--(4.794,3.364)--(4.794,3.366)--(4.794,3.368)--(4.800,3.371)--(4.800,3.372)%
  --(4.800,3.374)--(4.807,3.376)--(4.814,3.379)--(4.820,3.382)--(4.820,3.386)--(4.827,3.390)%
  --(4.834,3.392)--(4.840,3.394)--(4.840,3.399)--(4.853,3.403)--(4.853,3.406)--(4.867,3.411)%
  --(4.873,3.416)--(4.880,3.417)--(4.887,3.423)--(4.893,3.427)--(4.900,3.433)--(4.900,3.438)%
  --(4.907,3.442)--(4.913,3.442)--(4.913,3.445)--(4.920,3.446)--(4.927,3.450)--(4.927,3.453)%
  --(4.927,3.456)--(4.933,3.461)--(4.940,3.463)--(4.940,3.467)--(4.946,3.469)--(4.946,3.468)%
  --(4.946,3.470)--(4.953,3.473)--(4.960,3.477)--(4.966,3.483)--(4.973,3.487)--(4.980,3.491)%
  --(4.986,3.496)--(4.986,3.498)--(4.986,3.497)--(4.993,3.501)--(4.993,3.504)--(5.000,3.505)%
  --(5.000,3.508)--(5.006,3.510)--(5.006,3.513)--(5.013,3.515)--(5.013,3.518)--(5.013,3.519)%
  --(5.020,3.522)--(5.026,3.524)--(5.026,3.525)--(5.033,3.525)--(5.033,3.530)--(5.039,3.533)%
  --(5.039,3.535)--(5.046,3.539)--(5.053,3.542)--(5.053,3.545)--(5.053,3.547)--(5.059,3.548)%
  --(5.059,3.550)--(5.059,3.553)--(5.066,3.556)--(5.066,3.557)--(5.073,3.560)--(5.073,3.562)%
  --(5.079,3.563)--(5.086,3.567)--(5.086,3.570)--(5.093,3.572)--(5.099,3.578)--(5.106,3.583)%
  --(5.119,3.586)--(5.132,3.594)--(5.139,3.601)--(5.146,3.609)--(5.152,3.615)--(5.166,3.620)%
  --(5.172,3.621)--(5.179,3.627)--(5.186,3.634)--(5.199,3.642)--(5.212,3.648)--(5.212,3.652)%
  --(5.212,3.654)--(5.219,3.654)--(5.225,3.663)--(5.232,3.668)--(5.239,3.672)--(5.239,3.676)%
  --(5.245,3.679)--(5.252,3.683)--(5.259,3.690)--(5.259,3.693)--(5.272,3.698)--(5.279,3.700)%
  --(5.279,3.702)--(5.279,3.705)--(5.285,3.702)--(5.285,3.707)--(5.285,3.708)--(5.292,3.710)%
  --(5.292,3.713)--(5.298,3.715)--(5.305,3.718)--(5.305,3.723)--(5.312,3.726)--(5.318,3.729)%
  --(5.325,3.731)--(5.325,3.737)--(5.332,3.740)--(5.338,3.744)--(5.345,3.748)--(5.345,3.750)%
  --(5.358,3.755)--(5.365,3.762)--(5.372,3.766)--(5.385,3.769)--(5.391,3.774)--(5.405,3.782)%
  --(5.405,3.786)--(5.411,3.789)--(5.418,3.792)--(5.418,3.793)--(5.418,3.799)--(5.425,3.802)%
  --(5.425,3.804)--(5.438,3.806)--(5.438,3.807)--(5.438,3.810)--(5.445,3.811)--(5.445,3.813)%
  --(5.451,3.817)--(5.458,3.822)--(5.471,3.826)--(5.471,3.829)--(5.478,3.835)--(5.478,3.838)%
  --(5.484,3.839)--(5.484,3.840)--(5.484,3.841)--(5.491,3.848)--(5.491,3.852)--(5.491,3.855)%
  --(5.504,3.857)--(5.504,3.858)--(5.511,3.861)--(5.518,3.865)--(5.524,3.867)--(5.524,3.871)%
  --(5.531,3.873)--(5.531,3.876)--(5.538,3.877)--(5.538,3.880)--(5.544,3.878)--(5.544,3.882)%
  --(5.544,3.884)--(5.551,3.887)--(5.558,3.891)--(5.558,3.895)--(5.564,3.897)--(5.571,3.902)%
  --(5.577,3.907)--(5.584,3.912)--(5.584,3.914)--(5.591,3.919)--(5.604,3.925)--(5.611,3.929)%
  --(5.617,3.933)--(5.631,3.938)--(5.644,3.943)--(5.651,3.946)--(5.664,3.957)--(5.670,3.966)%
  --(5.690,3.973)--(5.704,3.985)--(5.717,3.992)--(5.724,3.998)--(5.737,4.007)--(5.750,4.014)%
  --(5.757,4.021)--(5.757,4.025)--(5.757,4.027)--(5.763,4.029)--(5.763,4.030)--(5.770,4.033)%
  --(5.770,4.034)--(5.777,4.036)--(5.777,4.039)--(5.777,4.040)--(5.777,4.042)--(5.783,4.045)%
  --(5.790,4.045)--(5.790,4.048)--(5.790,4.050)--(5.797,4.053)--(5.803,4.055)--(5.803,4.058)%
  --(5.810,4.062)--(5.810,4.064)--(5.810,4.065)--(5.817,4.067)--(5.817,4.069)--(5.817,4.070)%
  --(5.817,4.071)--(5.823,4.072)--(5.830,4.072)--(5.830,4.077)--(5.836,4.080)--(5.850,4.086)%
  --(5.850,4.093)--(5.856,4.096)--(5.863,4.101)--(5.863,4.104)--(5.870,4.108)--(5.876,4.109)%
  --(5.883,4.110)--(5.883,4.112)--(5.890,4.113)--(5.890,4.114)--(5.890,4.118)--(5.896,4.119)%
  --(5.910,4.125)--(5.910,4.131)--(5.916,4.133)--(5.923,4.136)--(5.929,4.141)--(5.929,4.142)%
  --(5.929,4.143)--(5.929,4.147)--(5.936,4.149)--(5.943,4.153)--(5.949,4.156)--(5.949,4.158)%
  --(5.956,4.161)--(5.956,4.164)--(5.963,4.165)--(5.963,4.166)--(5.976,4.173)--(5.983,4.180)%
  --(5.983,4.183)--(5.983,4.187)--(5.989,4.190)--(5.996,4.192)--(6.003,4.194)--(6.003,4.198)%
  --(6.003,4.202)--(6.009,4.202)--(6.016,4.206)--(6.016,4.209)--(6.022,4.211)--(6.029,4.215)%
  --(6.029,4.219)--(6.036,4.219)--(6.036,4.220)--(6.042,4.223)--(6.042,4.225)--(6.049,4.232)%
  --(6.049,4.235)--(6.062,4.238)--(6.062,4.242)--(6.062,4.244)--(6.069,4.245)--(6.069,4.247)%
  --(6.076,4.247)--(6.082,4.251)--(6.082,4.253)--(6.089,4.259)--(6.089,4.261)--(6.096,4.263)%
  --(6.102,4.267)--(6.109,4.272)--(6.122,4.278)--(6.129,4.287)--(6.142,4.294)--(6.149,4.297)%
  --(6.155,4.299)--(6.162,4.304)--(6.169,4.310)--(6.175,4.315)--(6.175,4.317)--(6.182,4.321)%
  --(6.188,4.326)--(6.195,4.326)--(6.202,4.332)--(6.202,4.334)--(6.208,4.338)--(6.215,4.343)%
  --(6.215,4.349)--(6.228,4.354)--(6.235,4.356)--(6.235,4.361)--(6.242,4.362)--(6.248,4.365)%
  --(6.248,4.370)--(6.255,4.373)--(6.262,4.377)--(6.268,4.380)--(6.268,4.382)--(6.275,4.382)%
  --(6.281,4.387)--(6.281,4.391)--(6.288,4.392)--(6.288,4.396)--(6.295,4.397)--(6.288,4.399)%
  --(6.295,4.400)--(6.295,4.401)--(6.295,4.402)--(6.301,4.405)--(6.301,4.403)--(6.308,4.405)%
  --(6.315,4.409)--(6.315,4.415)--(6.315,4.417)--(6.321,4.418)--(6.321,4.419)--(6.328,4.419)%
  --(6.328,4.422)--(6.335,4.424)--(6.335,4.427)--(6.341,4.434)--(6.348,4.437)--(6.355,4.439)%
  --(6.361,4.441)--(6.361,4.444)--(6.361,4.446)--(6.368,4.448)--(6.374,4.451)--(6.381,4.453)%
  --(6.388,4.462)--(6.401,4.468)--(6.401,4.470)--(6.408,4.475)--(6.414,4.480)--(6.421,4.482)%
  --(6.428,4.484)--(6.428,4.486)--(6.428,4.489)--(6.434,4.489)--(6.441,4.494)--(6.448,4.499)%
  --(6.454,4.504)--(6.461,4.507)--(6.461,4.508)--(6.467,4.507)--(6.474,4.511)--(6.474,4.516)%
  --(6.487,4.521)--(6.487,4.524)--(6.487,4.526)--(6.494,4.528)--(6.494,4.530)--(6.494,4.534)%
  --(6.507,4.536)--(6.507,4.540)--(6.514,4.544)--(6.521,4.549)--(6.527,4.552)--(6.534,4.559)%
  --(6.541,4.563)--(6.541,4.567)--(6.547,4.568)--(6.554,4.571)--(6.560,4.571)--(6.560,4.573)%
  --(6.560,4.577)--(6.567,4.580)--(6.574,4.586)--(6.580,4.590)--(6.587,4.592)--(6.587,4.594)%
  --(6.587,4.598)--(6.587,4.599)--(6.594,4.600)--(6.594,4.602)--(6.600,4.603)--(6.600,4.602)%
  --(6.600,4.608)--(6.607,4.611)--(6.614,4.616)--(6.620,4.619)--(6.627,4.622)--(6.627,4.626)%
  --(6.634,4.629)--(6.647,4.636)--(6.660,4.646)--(6.673,4.653)--(6.680,4.656)--(6.693,4.660)%
  --(6.700,4.667)--(6.707,4.674)--(6.720,4.679)--(6.733,4.684)--(6.746,4.690)--(6.746,4.696)%
  --(6.760,4.704)--(6.766,4.711)--(6.773,4.715)--(6.773,4.720)--(6.780,4.723)--(6.780,4.725)%
  --(6.786,4.727)--(6.786,4.730)--(6.793,4.731)--(6.793,4.734)--(6.800,4.743)--(6.800,4.747)%
  --(6.806,4.749)--(6.813,4.751)--(6.813,4.753)--(6.819,4.752)--(6.819,4.753)--(6.826,4.755)%
  --(6.826,4.757)--(6.833,4.762)--(6.833,4.764)--(6.833,4.767)--(6.833,4.768)--(6.839,4.770)%
  --(6.839,4.771)--(6.846,4.775)--(6.846,4.777)--(6.853,4.779)--(6.853,4.777)--(6.859,4.787)%
  --(6.866,4.789)--(6.873,4.793)--(6.873,4.797)--(6.873,4.801)--(6.879,4.802)--(6.886,4.804)%
  --(6.899,4.810)--(6.912,4.817)--(6.919,4.820)--(6.926,4.827)--(6.932,4.831)--(6.939,4.833)%
  --(6.946,4.837)--(6.946,4.838)--(6.946,4.839)--(6.959,4.846)--(6.966,4.851)--(6.972,4.858)%
  --(6.979,4.863)--(6.986,4.867)--(6.992,4.872)--(6.999,4.877)--(7.005,4.881)--(7.019,4.885)%
  --(7.019,4.891)--(7.025,4.893)--(7.032,4.902)--(7.039,4.906)--(7.045,4.907)--(7.045,4.911)%
  --(7.052,4.915)--(7.059,4.918)--(7.059,4.919)--(7.065,4.921)--(7.072,4.928)--(7.072,4.931)%
  --(7.072,4.934)--(7.079,4.937)--(7.079,4.939)--(7.079,4.943)--(7.092,4.944)--(7.092,4.945)%
  --(7.098,4.947)--(7.098,4.948)--(7.098,4.946)--(7.098,4.952)--(7.105,4.954)--(7.112,4.959)%
  --(7.118,4.962)--(7.118,4.969)--(7.125,4.971)--(7.125,4.972)--(7.125,4.975)--(7.132,4.976)%
  --(7.132,4.972)--(7.138,4.980)--(7.145,4.983)--(7.152,4.987)--(7.158,4.991)--(7.158,4.993)%
  --(7.171,4.999)--(7.178,5.008)--(7.178,5.009)--(7.185,5.011)--(7.198,5.013)--(7.198,5.017)%
  --(7.205,5.020)--(7.205,5.022)--(7.211,5.027)--(7.218,5.029)--(7.225,5.031)--(7.225,5.030)%
  --(7.231,5.033)--(7.231,5.041)--(7.245,5.045)--(7.251,5.050)--(7.258,5.057)--(7.271,5.064)%
  --(7.278,5.067)--(7.284,5.069)--(7.291,5.073)--(7.298,5.076)--(7.291,5.077)--(7.298,5.082)%
  --(7.298,5.086)--(7.304,5.089)--(7.311,5.092)--(7.311,5.096)--(7.318,5.098)--(7.318,5.101)%
  --(7.324,5.103)--(7.324,5.104)--(7.331,5.112)--(7.338,5.114)--(7.338,5.116)--(7.344,5.118)%
  --(7.344,5.119)--(7.351,5.122)--(7.357,5.128)--(7.364,5.131)--(7.371,5.136)--(7.377,5.138)%
  --(7.377,5.141)--(7.391,5.143)--(7.391,5.150)--(7.397,5.151)--(7.404,5.154)--(7.404,5.157)%
  --(7.404,5.160)--(7.417,5.165)--(7.424,5.171)--(7.424,5.176)--(7.431,5.180)--(7.437,5.184)%
  --(7.444,5.184)--(7.450,5.186)--(7.457,5.189)--(7.464,5.192)--(7.470,5.196)--(7.484,5.204)%
  --(7.484,5.208)--(7.490,5.211)--(7.497,5.217)--(7.504,5.220)--(7.510,5.222)--(7.510,5.225)%
  --(7.517,5.228)--(7.524,5.232)--(7.530,5.241)--(7.537,5.247)--(7.543,5.251)--(7.550,5.255)%
  --(7.563,5.262)--(7.570,5.269)--(7.570,5.272)--(7.577,5.275)--(7.583,5.273)--(7.590,5.281)%
  --(7.590,5.284)--(7.597,5.288)--(7.597,5.292)--(7.603,5.296)--(7.603,5.299)--(7.616,5.302)%
  --(7.616,5.303)--(7.623,5.306)--(7.623,5.305)--(7.630,5.310)--(7.630,5.314)--(7.630,5.317)%
  --(7.643,5.323)--(7.650,5.328)--(7.656,5.327)--(7.663,5.330)--(7.670,5.332)--(7.676,5.335)%
  --(7.683,5.348)--(7.690,5.356)--(7.696,5.360)--(7.703,5.363)--(7.709,5.366)--(7.716,5.370)%
  --(7.723,5.374)--(7.736,5.374)--(7.736,5.376)--(7.743,5.382)--(7.749,5.388)--(7.749,5.391)%
  --(7.756,5.393)--(7.756,5.397)--(7.756,5.396)--(7.756,5.394)--(7.763,5.396)--(7.769,5.399)%
  --(7.776,5.404)--(7.783,5.413)--(7.789,5.417)--(7.796,5.422)--(7.802,5.427)--(7.809,5.431)%
  --(7.816,5.434)--(7.816,5.435)--(7.822,5.438)--(7.822,5.437)--(7.829,5.442)--(7.836,5.446)%
  --(7.836,5.450)--(7.842,5.455)--(7.849,5.459)--(7.849,5.461)--(7.856,5.463)--(7.862,5.466)%
  --(7.862,5.467)--(7.862,5.466)--(7.862,5.471)--(7.869,5.472)--(7.869,5.477)--(7.876,5.480)%
  --(7.882,5.484)--(7.889,5.483)--(7.895,5.486)--(7.902,5.491)--(7.902,5.494)--(7.909,5.500)%
  --(7.909,5.505)--(7.922,5.507)--(7.915,5.508)--(7.922,5.509)--(7.929,5.513)--(7.929,5.518)%
  --(7.929,5.517)--(7.935,5.519)--(7.949,5.524)--(7.955,5.531)--(7.955,5.534)--(7.962,5.537)%
  --(7.975,5.538)--(7.975,5.541)--(7.982,5.540)--(7.988,5.543)--(7.995,5.546)--(7.995,5.549)%
  --(8.002,5.556)--(8.002,5.558)--(8.002,5.560)--(8.008,5.561)--(8.022,5.562)--(8.022,5.566)%
  --(8.028,5.571)--(8.035,5.577)--(8.035,5.582)--(8.042,5.582)--(8.042,5.588)--(8.048,5.590)%
  --(8.055,5.593)--(8.062,5.597)--(8.068,5.601)--(8.075,5.604)--(8.088,5.608)--(8.088,5.611)%
  --(8.088,5.612)--(8.095,5.617)--(8.095,5.622)--(8.101,5.625)--(8.108,5.626)--(8.115,5.630)%
  --(8.121,5.636)--(8.128,5.639)--(8.135,5.643)--(8.141,5.648)--(8.141,5.657)--(8.154,5.662)%
  --(8.154,5.667)--(8.161,5.669)--(8.161,5.670)--(8.161,5.667)--(8.161,5.668)--(8.168,5.668)%
  --(8.168,5.675)--(8.168,5.676)--(8.174,5.679)--(8.174,5.681)--(8.181,5.684)--(8.194,5.684)%
  --(8.194,5.687)--(8.194,5.690)--(8.201,5.695)--(8.201,5.698)--(8.201,5.704)--(8.214,5.705)%
  --(8.214,5.708)--(8.221,5.710)--(8.221,5.712)--(8.228,5.714)--(8.234,5.716)--(8.234,5.718)%
  --(8.241,5.720)--(8.247,5.722)--(8.247,5.723)--(8.254,5.727)--(8.254,5.729)--(8.261,5.732)%
  --(8.261,5.734)--(8.261,5.737)--(8.274,5.740)--(8.281,5.741)--(8.281,5.736)--(8.287,5.744)%
  --(8.287,5.748)--(8.294,5.752)--(8.301,5.756)--(8.307,5.759)--(8.314,5.759)--(8.327,5.763)%
  --(8.321,5.765)--(8.321,5.767)--(8.327,5.775)--(8.334,5.778)--(8.340,5.782)--(8.347,5.783)%
  --(8.347,5.786)--(8.347,5.787)--(8.354,5.790)--(8.354,5.789)--(8.360,5.789)--(8.360,5.794)%
  --(8.367,5.803)--(8.367,5.806)--(8.374,5.809)--(8.374,5.813)--(8.380,5.814)--(8.387,5.813)%
  --(8.387,5.816)--(8.394,5.817)--(8.400,5.820)--(8.400,5.828)--(8.400,5.830)--(8.414,5.832)%
  --(8.420,5.836)--(8.420,5.841)--(8.427,5.845)--(8.433,5.847)--(8.433,5.851)--(8.433,5.852)%
  --(8.440,5.856)--(8.447,5.858)--(8.453,5.863)--(8.453,5.868)--(8.460,5.872)--(8.467,5.877)%
  --(8.467,5.879)--(8.473,5.880)--(8.473,5.884)--(8.487,5.881)--(8.493,5.889)--(8.500,5.890)%
  --(8.500,5.891)--(8.507,5.894)--(8.507,5.893)--(8.513,5.891)--(8.513,5.892)--(8.513,5.893)%
  --(8.533,5.898)--(8.533,5.908)--(8.546,5.917)--(8.553,5.922)--(8.566,5.927)--(8.573,5.936)%
  --(8.586,5.940)--(8.599,5.944)--(8.606,5.949)--(8.613,5.954)--(8.613,5.958)--(8.619,5.970)%
  --(8.633,5.974)--(8.639,5.979)--(8.646,5.985)--(8.653,5.987)--(8.646,5.987)--(8.646,5.993)%
  --(8.653,5.997)--(8.653,5.999)--(8.653,6.001)--(8.659,6.004)--(8.659,6.006)--(8.666,6.007)%
  --(8.673,6.011)--(8.679,6.015)--(8.679,6.018)--(8.686,6.022)--(8.692,6.024)--(8.699,6.029)%
  --(8.692,6.029)--(8.699,6.032)--(8.712,6.040)--(8.726,6.054)--(8.739,6.061)--(8.746,6.061)%
  --(8.746,6.059)--(8.746,6.063)--(8.746,6.066)--(8.752,6.069)--(8.759,6.071)--(8.759,6.074)%
  --(8.759,6.069)--(8.772,6.079)--(8.779,6.082)--(8.785,6.084)--(8.785,6.085)--(8.792,6.086)%
  --(8.792,6.089)--(8.799,6.091)--(8.805,6.095)--(8.812,6.099)--(8.825,6.104)--(8.825,6.108)%
  --(8.832,6.113)--(8.839,6.120)--(8.845,6.127)--(8.859,6.132)--(8.865,6.134)--(8.865,6.136)%
  --(8.865,6.139)--(8.872,6.134)--(8.878,6.144)--(8.878,6.147)--(8.885,6.150)--(8.885,6.154)%
  --(8.892,6.158)--(8.898,6.161)--(8.905,6.164)--(8.912,6.167)--(8.912,6.169)--(8.912,6.174)%
  --(8.918,6.177)--(8.918,6.180)--(8.925,6.178)--(8.938,6.191)--(8.945,6.196)--(8.945,6.200)%
  --(8.952,6.206)--(8.952,6.210)--(8.958,6.210)--(8.958,6.212)--(8.958,6.213)--(8.965,6.216)%
  --(8.965,6.215)--(8.978,6.223)--(8.978,6.228)--(8.985,6.232)--(8.991,6.235)--(8.991,6.231)%
  --(8.991,6.237)--(8.991,6.238)--(8.998,6.240)--(8.998,6.242)--(9.005,6.244)--(9.011,6.245)%
  --(9.018,6.247)--(9.025,6.250)--(9.031,6.250)--(9.038,6.251)--(9.038,6.255)--(9.051,6.260)%
  --(9.051,6.263)--(9.058,6.265)--(9.064,6.271)--(9.071,6.281)--(9.084,6.286)--(9.091,6.284)%
  --(9.098,6.286)--(9.098,6.292)--(9.104,6.297)--(9.118,6.304)--(9.118,6.311)--(9.124,6.314)%
  --(9.124,6.318)--(9.137,6.324)--(9.144,6.328)--(9.151,6.334)--(9.157,6.341)--(9.177,6.351)%
  --(9.177,6.350)--(9.184,6.357)--(9.184,6.360)--(9.184,6.362)--(9.191,6.367)--(9.191,6.373)%
  --(9.197,6.380)--(9.211,6.385)--(9.224,6.388)--(9.224,6.387)--(9.217,6.393)--(9.217,6.394)%
  --(9.224,6.397)--(9.230,6.399)--(9.230,6.401)--(9.237,6.400)--(9.244,6.403)--(9.257,6.406)%
  --(9.264,6.411)--(9.264,6.418)--(9.270,6.424)--(9.277,6.426)--(9.277,6.428)--(9.284,6.430)%
  --(9.284,6.432)--(9.284,6.433)--(9.297,6.429)--(9.304,6.432)--(9.304,6.437)--(9.310,6.444)%
  --(9.310,6.447)--(9.317,6.448)--(9.323,6.452)--(9.330,6.454)--(9.330,6.453)--(9.343,6.457)%
  --(9.350,6.459)--(9.350,6.465)--(9.350,6.473)--(9.357,6.478)--(9.357,6.479)--(9.363,6.483)%
  --(9.370,6.487)--(9.383,6.493)--(9.390,6.498)--(9.390,6.503)--(9.403,6.503)--(9.403,6.513)%
  --(9.416,6.519)--(9.430,6.527)--(9.430,6.535)--(9.436,6.538)--(9.443,6.539)--(9.443,6.543)%
  --(9.450,6.545)--(9.443,6.545)--(9.450,6.547)--(9.450,6.550)--(9.456,6.554)--(9.470,6.561)%
  --(9.476,6.569)--(9.483,6.571)--(9.483,6.572)--(9.490,6.576)--(9.490,6.578)--(9.496,6.585)%
  --(9.503,6.590)--(9.509,6.594)--(9.509,6.595)--(9.516,6.597)--(9.523,6.599)--(9.523,6.602)%
  --(9.523,6.600)--(9.529,6.599)--(9.536,6.606)--(9.543,6.612)--(9.556,6.616)--(9.563,6.621)%
  --(9.569,6.624)--(9.576,6.626)--(9.582,6.629)--(9.589,6.633)--(9.602,6.640)--(9.609,6.653)%
  --(9.622,6.658)--(9.629,6.664)--(9.629,6.669)--(9.642,6.672)--(9.642,6.674)--(9.649,6.678)%
  --(9.649,6.682)--(9.662,6.683)--(9.669,6.690)--(9.675,6.699)--(9.682,6.705)--(9.689,6.708)%
  --(9.689,6.709)--(9.689,6.710)--(9.689,6.711)--(9.689,6.713)--(9.695,6.712)--(9.702,6.723)%
  --(9.709,6.730)--(9.709,6.736)--(9.715,6.739)--(9.722,6.743)--(9.735,6.745)--(9.742,6.750)%
  --(9.749,6.756)--(9.755,6.762)--(9.762,6.772)--(9.762,6.773)--(9.768,6.775)--(9.762,6.776)%
  --(9.768,6.777)--(9.775,6.777)--(9.775,6.775)--(9.782,6.774)--(9.782,6.777)--(9.782,6.780)%
  --(9.782,6.783)--(9.788,6.781)--(9.782,6.778)--(9.788,6.780)--(9.802,6.785)--(9.808,6.791)%
  --(9.822,6.802)--(9.828,6.806)--(9.835,6.808)--(9.835,6.810)--(9.835,6.813)--(9.842,6.815)%
  --(9.842,6.817)--(9.848,6.821)--(9.855,6.823)--(9.868,6.828)--(9.868,6.835)--(9.875,6.838)%
  --(9.881,6.843)--(9.881,6.848)--(9.888,6.851)--(9.895,6.854)--(9.888,6.854)--(9.901,6.856)%
  --(9.908,6.855)--(9.915,6.865)--(9.915,6.871)--(9.921,6.875)--(9.935,6.880)--(9.935,6.884)%
  --(9.941,6.884)--(9.948,6.889)--(9.954,6.894)--(9.954,6.896)--(9.961,6.899)--(9.961,6.903)%
  --(9.961,6.905)--(9.961,6.907)--(9.968,6.907)--(9.968,6.912)--(9.974,6.917)--(9.974,6.922)%
  --(9.981,6.921)--(9.981,6.922)--(9.988,6.932)--(9.994,6.934)--(10.001,6.938)--(10.008,6.942)%
  --(10.014,6.947)--(10.027,6.947)--(10.027,6.948)--(10.027,6.949)--(10.034,6.951)--(10.034,6.954)%
  --(10.047,6.959)--(10.054,6.961)--(10.061,6.964)--(10.054,6.969)--(10.061,6.972)--(10.067,6.975)%
  --(10.067,6.976)--(10.067,6.977)--(10.074,6.975)--(10.074,6.984)--(10.081,6.986)--(10.087,6.988)%
  --(10.101,6.992)--(10.101,6.995)--(10.114,6.999)--(10.120,7.006)--(10.120,7.011)--(10.127,7.014)%
  --(10.134,7.015)--(10.127,7.019)--(10.134,7.022)--(10.140,7.025)--(10.147,7.027)--(10.154,7.027)%
  --(10.154,7.023)--(10.154,7.026)--(10.160,7.030)--(10.167,7.033)--(10.167,7.041)--(10.167,7.047)%
  --(10.167,7.050)--(10.174,7.055)--(10.180,7.058)--(10.180,7.054)--(10.187,7.057)--(10.200,7.063)%
  --(10.200,7.065)--(10.213,7.074)--(10.207,7.079)--(10.213,7.082)--(10.207,7.084)--(10.213,7.086)%
  --(10.220,7.085)--(10.220,7.086)--(10.220,7.088)--(10.233,7.096)--(10.233,7.099)--(10.240,7.102)%
  --(10.233,7.105)--(10.240,7.106)--(10.247,7.109)--(10.247,7.113)--(10.253,7.117)--(10.267,7.123)%
  --(10.273,7.127)--(10.273,7.125)--(10.280,7.136)--(10.293,7.137)--(10.293,7.140)--(10.300,7.142)%
  --(10.300,7.145)--(10.306,7.147)--(10.313,7.149)--(10.320,7.153)--(10.320,7.157)--(10.326,7.152)%
  --(10.326,7.164)--(10.333,7.167)--(10.353,7.171)--(10.353,7.173)--(10.360,7.175)--(10.366,7.176)%
  --(10.373,7.187)--(10.373,7.189)--(10.373,7.191)--(10.380,7.192)--(10.393,7.197)--(10.399,7.203)%
  --(10.399,7.204)--(10.393,7.197)--(10.406,7.212)--(10.413,7.217)--(10.413,7.220)--(10.413,7.222)%
  --(10.413,7.223)--(10.419,7.226)--(10.426,7.228)--(10.433,7.227)--(10.433,7.232)--(10.439,7.236)%
  --(10.439,7.238)--(10.453,7.242)--(10.453,7.248)--(10.453,7.254)--(10.459,7.258)--(10.466,7.263)%
  --(10.466,7.267)--(10.473,7.270)--(10.479,7.274)--(10.486,7.279)--(10.499,7.281)--(10.506,7.285)%
  --(10.499,7.290)--(10.506,7.291)--(10.512,7.293)--(10.512,7.296)--(10.512,7.299)--(10.519,7.301)%
  --(10.532,7.309)--(10.539,7.315)--(10.539,7.312)--(10.552,7.318)--(10.546,7.318)--(10.546,7.319)%
  --(10.546,7.318)--(10.546,7.320)--(10.552,7.312)--(10.565,7.322)--(10.559,7.323)--(10.572,7.325)%
  --(10.572,7.328)--(10.572,7.329)--(10.585,7.331)--(10.585,7.333)--(10.585,7.336)--(10.585,7.337)%
  --(10.599,7.344)--(10.592,7.347)--(10.605,7.350)--(10.605,7.353)--(10.605,7.355)--(10.612,7.358)%
  --(10.619,7.360)--(10.625,7.365)--(10.632,7.370)--(10.632,7.367)--(10.632,7.369)--(10.639,7.369)%
  --(10.639,7.371)--(10.639,7.374)--(10.632,7.380)--(10.639,7.381)--(10.639,7.382)--(10.645,7.384)%
  --(10.652,7.385)--(10.658,7.389)--(10.665,7.393)--(10.672,7.397)--(10.672,7.396)--(10.678,7.404)%
  --(10.685,7.407)--(10.685,7.410)--(10.685,7.411)--(10.692,7.416)--(10.698,7.420)--(10.698,7.424)%
  --(10.712,7.427)--(10.705,7.429)--(10.705,7.426)--(10.718,7.437)--(10.725,7.441)--(10.738,7.447)%
  --(10.738,7.453)--(10.745,7.459)--(10.751,7.461)--(10.758,7.466)--(10.765,7.466)--(10.771,7.469)%
  --(10.771,7.477)--(10.771,7.480)--(10.771,7.482)--(10.778,7.484)--(10.778,7.485)--(10.778,7.486)%
  --(10.785,7.487)--(10.785,7.483)--(10.791,7.483)--(10.798,7.486)--(10.805,7.494)--(10.811,7.496)%
  --(10.818,7.496)--(10.825,7.495)--(10.825,7.490);
\gpcolor{color=gp lt color 1}
\gpsetlinetype{gp lt plot 1}
\draw[gp path] (2.562,1.878)--(2.569,1.882)--(2.575,1.886)--(2.589,1.892)--(2.595,1.900)%
  --(2.602,1.905)--(2.609,1.910)--(2.622,1.917)--(2.622,1.920)--(2.622,1.923)--(2.628,1.924)%
  --(2.628,1.926)--(2.635,1.929)--(2.635,1.930)--(2.635,1.932)--(2.642,1.933)--(2.648,1.937)%
  --(2.655,1.942)--(2.655,1.945)--(2.662,1.947)--(2.668,1.951)--(2.675,1.956)--(2.675,1.960)%
  --(2.682,1.963)--(2.688,1.965)--(2.695,1.970)--(2.702,1.975)--(2.708,1.979)--(2.715,1.984)%
  --(2.721,1.987)--(2.728,1.990)--(2.735,1.995)--(2.741,2.002)--(2.755,2.008)--(2.761,2.016)%
  --(2.768,2.020)--(2.775,2.022)--(2.775,2.025)--(2.775,2.027)--(2.775,2.028)--(2.781,2.030)%
  --(2.781,2.035)--(2.788,2.036)--(2.788,2.038)--(2.788,2.040)--(2.794,2.043)--(2.801,2.046)%
  --(2.808,2.053)--(2.814,2.056)--(2.821,2.059)--(2.821,2.063)--(2.828,2.067)--(2.834,2.072)%
  --(2.841,2.075)--(2.848,2.078)--(2.848,2.081)--(2.854,2.085)--(2.854,2.089)--(2.861,2.091)%
  --(2.868,2.094)--(2.868,2.097)--(2.874,2.100)--(2.874,2.102)--(2.881,2.105)--(2.881,2.109)%
  --(2.887,2.111)--(2.894,2.114)--(2.894,2.118)--(2.901,2.123)--(2.901,2.124)--(2.907,2.126)%
  --(2.907,2.128)--(2.914,2.132)--(2.927,2.135)--(2.934,2.139)--(2.934,2.142)--(2.941,2.145)%
  --(2.941,2.146)--(2.947,2.150)--(2.954,2.155)--(2.961,2.162)--(2.967,2.166)--(2.974,2.171)%
  --(2.980,2.174)--(2.987,2.177)--(2.994,2.182)--(3.000,2.186)--(3.000,2.189)--(3.000,2.191)%
  --(3.007,2.193)--(3.007,2.194)--(3.014,2.197)--(3.014,2.200)--(3.020,2.204)--(3.027,2.206)%
  --(3.027,2.209)--(3.034,2.211)--(3.034,2.213)--(3.034,2.215)--(3.040,2.217)--(3.047,2.220)%
  --(3.047,2.225)--(3.054,2.230)--(3.060,2.235)--(3.067,2.240)--(3.080,2.245)--(3.087,2.250)%
  --(3.093,2.253)--(3.093,2.256)--(3.100,2.258)--(3.107,2.261)--(3.107,2.262)--(3.107,2.264)%
  --(3.113,2.266)--(3.113,2.269)--(3.120,2.273)--(3.120,2.277)--(3.127,2.281)--(3.127,2.283)%
  --(3.127,2.285)--(3.133,2.287)--(3.140,2.290)--(3.140,2.294)--(3.147,2.296)--(3.147,2.298)%
  --(3.147,2.299)--(3.160,2.303)--(3.166,2.308)--(3.173,2.313)--(3.180,2.316)--(3.180,2.318)%
  --(3.180,2.322)--(3.186,2.323)--(3.193,2.327)--(3.200,2.331)--(3.206,2.334)--(3.206,2.337)%
  --(3.213,2.342)--(3.220,2.346)--(3.226,2.349)--(3.226,2.350)--(3.233,2.351)--(3.233,2.355)%
  --(3.240,2.359)--(3.246,2.363)--(3.253,2.369)--(3.259,2.373)--(3.266,2.376)--(3.266,2.378)%
  --(3.273,2.381)--(3.273,2.384)--(3.279,2.389)--(3.286,2.393)--(3.293,2.397)--(3.299,2.401)%
  --(3.306,2.405)--(3.313,2.409)--(3.313,2.412)--(3.319,2.415)--(3.326,2.418)--(3.332,2.421)%
  --(3.332,2.424)--(3.339,2.429)--(3.346,2.432)--(3.352,2.435)--(3.359,2.440)--(3.359,2.445)%
  --(3.366,2.449)--(3.372,2.455)--(3.379,2.460)--(3.386,2.465)--(3.392,2.471)--(3.399,2.475)%
  --(3.406,2.477)--(3.399,2.478)--(3.406,2.478)--(3.406,2.477)--(3.406,2.478)--(3.406,2.480)%
  --(3.412,2.484)--(3.419,2.487)--(3.419,2.489)--(3.419,2.490)--(3.425,2.493)--(3.432,2.494)%
  --(3.432,2.495)--(3.432,2.496)--(3.432,2.499)--(3.439,2.501)--(3.439,2.505)--(3.445,2.508)%
  --(3.452,2.511)--(3.459,2.513)--(3.459,2.517)--(3.465,2.522)--(3.472,2.523)--(3.472,2.525)%
  --(3.479,2.527)--(3.479,2.530)--(3.485,2.532)--(3.485,2.536)--(3.492,2.537)--(3.492,2.540)%
  --(3.499,2.544)--(3.499,2.547)--(3.505,2.550)--(3.512,2.553)--(3.512,2.554)--(3.518,2.555)%
  --(3.518,2.559)--(3.518,2.561)--(3.525,2.564)--(3.525,2.565)--(3.532,2.569)--(3.538,2.573)%
  --(3.545,2.575)--(3.552,2.580)--(3.552,2.584)--(3.558,2.587)--(3.565,2.592)--(3.565,2.596)%
  --(3.572,2.598)--(3.572,2.600)--(3.578,2.601)--(3.585,2.606)--(3.592,2.610)--(3.592,2.613)%
  --(3.598,2.615)--(3.598,2.617)--(3.605,2.619)--(3.605,2.622)--(3.611,2.625)--(3.618,2.629)%
  --(3.625,2.633)--(3.631,2.637)--(3.631,2.639)--(3.638,2.643)--(3.645,2.646)--(3.645,2.648)%
  --(3.651,2.651)--(3.651,2.653)--(3.658,2.656)--(3.658,2.658)--(3.658,2.660)--(3.665,2.662)%
  --(3.671,2.667)--(3.678,2.672)--(3.685,2.677)--(3.691,2.680)--(3.698,2.684)--(3.704,2.686)%
  --(3.704,2.689)--(3.711,2.693)--(3.718,2.696)--(3.724,2.699)--(3.724,2.703)--(3.724,2.708)%
  --(3.731,2.711)--(3.738,2.713)--(3.744,2.715)--(3.744,2.716)--(3.744,2.717)--(3.751,2.719)%
  --(3.751,2.722)--(3.758,2.726)--(3.764,2.732)--(3.771,2.737)--(3.771,2.741)--(3.777,2.742)%
  --(3.777,2.744)--(3.784,2.748)--(3.784,2.749)--(3.791,2.751)--(3.797,2.754)--(3.797,2.758)%
  --(3.804,2.763)--(3.811,2.766)--(3.811,2.769)--(3.811,2.772)--(3.817,2.773)--(3.817,2.775)%
  --(3.824,2.778)--(3.831,2.780)--(3.831,2.783)--(3.837,2.787)--(3.837,2.792)--(3.837,2.796)%
  --(3.844,2.798)--(3.844,2.800)--(3.857,2.803)--(3.864,2.806)--(3.870,2.811)--(3.870,2.815)%
  --(3.877,2.817)--(3.884,2.820)--(3.884,2.823)--(3.890,2.828)--(3.897,2.832)--(3.897,2.833)%
  --(3.897,2.834)--(3.897,2.835)--(3.904,2.838)--(3.910,2.840)--(3.910,2.842)--(3.917,2.845)%
  --(3.924,2.847)--(3.924,2.849)--(3.930,2.851)--(3.930,2.853)--(3.930,2.855)--(3.930,2.856)%
  --(3.930,2.857)--(3.930,2.858)--(3.930,2.859)--(3.930,2.860)--(3.930,2.858)--(3.930,2.860)%
  --(3.937,2.861)--(3.937,2.863)--(3.997,2.892)--(4.017,2.904)--(4.017,2.907)--(4.023,2.913)%
  --(4.030,2.919)--(4.043,2.930)--(4.056,2.940)--(4.063,2.945)--(4.063,2.947)--(4.070,2.949)%
  --(4.083,2.956)--(4.090,2.963)--(4.103,2.973)--(4.116,2.980)--(4.123,2.986)--(4.136,2.993)%
  --(4.143,2.999)--(4.149,3.004)--(4.149,3.005)--(4.149,3.006)--(4.149,3.007)--(4.149,3.008)%
  --(4.149,3.006)--(4.156,3.008)--(4.163,3.012)--(4.163,3.018)--(4.169,3.023)--(4.176,3.026)%
  --(4.183,3.029)--(4.189,3.037)--(4.196,3.042)--(4.209,3.045)--(4.209,3.048)--(4.222,3.051)%
  --(4.222,3.055)--(4.229,3.060)--(4.236,3.064)--(4.242,3.069)--(4.242,3.072)--(4.242,3.071)%
  --(4.242,3.072)--(4.242,3.073)--(4.242,3.074)--(4.256,3.085)--(4.289,3.102)--(4.309,3.116)%
  --(4.315,3.117)--(4.309,3.119)--(4.309,3.120)--(4.315,3.123)--(4.322,3.130)--(4.335,3.135)%
  --(4.342,3.143)--(4.349,3.149)--(4.362,3.156)--(4.375,3.159)--(4.375,3.165)--(4.382,3.168)%
  --(4.382,3.171)--(4.382,3.174)--(4.389,3.175)--(4.389,3.178)--(4.395,3.180)--(4.395,3.181)%
  --(4.402,3.185)--(4.408,3.189)--(4.408,3.191)--(4.415,3.194)--(4.415,3.197)--(4.422,3.200)%
  --(4.422,3.204)--(4.428,3.208)--(4.435,3.211)--(4.442,3.213)--(4.448,3.216)--(4.455,3.222)%
  --(4.468,3.229)--(4.482,3.240)--(4.488,3.246)--(4.495,3.250)--(4.501,3.255)--(4.508,3.260)%
  --(4.515,3.265)--(4.528,3.271)--(4.535,3.278)--(4.541,3.284)--(4.548,3.287)--(4.548,3.292)%
  --(4.555,3.293)--(4.555,3.294)--(4.555,3.295)--(4.561,3.301)--(4.568,3.308)--(4.581,3.312)%
  --(4.581,3.315)--(4.588,3.318)--(4.588,3.320)--(4.594,3.322)--(4.594,3.327)--(4.601,3.330)%
  --(4.608,3.333)--(4.614,3.334)--(4.621,3.338)--(4.621,3.341)--(4.621,3.343)--(4.628,3.344)%
  --(4.634,3.347)--(4.634,3.351)--(4.648,3.357)--(4.648,3.362)--(4.654,3.366)--(4.668,3.371)%
  --(4.674,3.377)--(4.681,3.381)--(4.687,3.383)--(4.687,3.388)--(4.694,3.392)--(4.714,3.402)%
  --(4.727,3.408)--(4.734,3.414)--(4.747,3.422)--(4.754,3.428)--(4.760,3.432)--(4.767,3.436)%
  --(4.780,3.445)--(4.794,3.456)--(4.800,3.465)--(4.814,3.471)--(4.827,3.478)--(4.827,3.483)%
  --(4.834,3.484)--(4.840,3.487)--(4.840,3.490)--(4.847,3.493)--(4.853,3.498)--(4.853,3.502)%
  --(4.853,3.505)--(4.860,3.509)--(4.860,3.510)--(4.860,3.511)--(4.867,3.515)--(4.867,3.518)%
  --(4.873,3.521)--(4.880,3.524)--(4.887,3.528)--(4.887,3.530)--(4.893,3.533)--(4.893,3.536)%
  --(4.900,3.536)--(4.900,3.542)--(4.907,3.546)--(4.920,3.551)--(4.927,3.556)--(4.933,3.561)%
  --(4.940,3.566)--(4.946,3.570)--(4.953,3.575)--(4.960,3.580)--(4.966,3.584)--(4.973,3.589)%
  --(4.980,3.595)--(4.986,3.602)--(4.993,3.606)--(5.000,3.609)--(5.006,3.612)--(5.006,3.616)%
  --(5.013,3.619)--(5.026,3.621)--(5.033,3.633)--(5.046,3.640)--(5.059,3.648)--(5.066,3.654)%
  --(5.073,3.660)--(5.079,3.666)--(5.086,3.673)--(5.093,3.680)--(5.099,3.684)--(5.106,3.686)%
  --(5.113,3.688)--(5.119,3.692)--(5.119,3.695)--(5.126,3.698)--(5.132,3.704)--(5.146,3.709)%
  --(5.152,3.715)--(5.152,3.719)--(5.166,3.724)--(5.172,3.729)--(5.179,3.735)--(5.192,3.737)%
  --(5.199,3.751)--(5.212,3.756)--(5.219,3.761)--(5.219,3.762)--(5.225,3.765)--(5.225,3.767)%
  --(5.232,3.772)--(5.239,3.776)--(5.245,3.782)--(5.245,3.786)--(5.252,3.788)--(5.259,3.790)%
  --(5.265,3.798)--(5.279,3.806)--(5.285,3.813)--(5.298,3.818)--(5.312,3.828)--(5.318,3.839)%
  --(5.325,3.840)--(5.338,3.846)--(5.345,3.853)--(5.352,3.856)--(5.352,3.858)--(5.358,3.859)%
  --(5.358,3.860)--(5.365,3.864)--(5.372,3.870)--(5.385,3.878)--(5.391,3.883)--(5.391,3.887)%
  --(5.398,3.890)--(5.398,3.891)--(5.405,3.895)--(5.405,3.898)--(5.411,3.898)--(5.418,3.900)%
  --(5.418,3.904)--(5.425,3.908)--(5.431,3.915)--(5.438,3.921)--(5.445,3.923)--(5.451,3.926)%
  --(5.451,3.928)--(5.458,3.932)--(5.465,3.934)--(5.471,3.938)--(5.478,3.942)--(5.478,3.945)%
  --(5.484,3.951)--(5.491,3.956)--(5.498,3.961)--(5.504,3.965)--(5.511,3.970)--(5.518,3.974)%
  --(5.531,3.981)--(5.538,3.986)--(5.544,3.991)--(5.558,3.997)--(5.564,4.006)--(5.571,4.011)%
  --(5.577,4.016)--(5.584,4.020)--(5.591,4.023)--(5.591,4.027)--(5.597,4.030)--(5.597,4.032)%
  --(5.604,4.033)--(5.611,4.038)--(5.617,4.043)--(5.624,4.048)--(5.631,4.053)--(5.631,4.056)%
  --(5.637,4.057)--(5.644,4.060)--(5.651,4.064)--(5.657,4.070)--(5.670,4.082)--(5.690,4.094)%
  --(5.704,4.099)--(5.717,4.105)--(5.724,4.113)--(5.730,4.119)--(5.737,4.122)--(5.743,4.127)%
  --(5.750,4.131)--(5.750,4.134)--(5.763,4.142)--(5.770,4.149)--(5.790,4.161)--(5.797,4.172)%
  --(5.810,4.180)--(5.823,4.187)--(5.830,4.193)--(5.836,4.201)--(5.843,4.205)--(5.850,4.209)%
  --(5.856,4.212)--(5.870,4.220)--(5.876,4.227)--(5.883,4.233)--(5.890,4.238)--(5.896,4.241)%
  --(5.896,4.243)--(5.903,4.247)--(5.910,4.252)--(5.910,4.258)--(5.923,4.263)--(5.929,4.271)%
  --(5.936,4.275)--(5.943,4.277)--(5.949,4.280)--(5.956,4.285)--(5.963,4.286)--(5.969,4.295)%
  --(5.976,4.300)--(5.983,4.307)--(5.989,4.312)--(6.003,4.316)--(6.009,4.320)--(6.016,4.326)%
  --(6.022,4.331)--(6.029,4.339)--(6.049,4.351)--(6.062,4.360)--(6.076,4.369)--(6.089,4.377)%
  --(6.096,4.384)--(6.109,4.388)--(6.109,4.392)--(6.109,4.395)--(6.109,4.396)--(6.109,4.397)%
  --(6.115,4.403)--(6.122,4.407)--(6.122,4.409)--(6.129,4.410)--(6.129,4.415)--(6.129,4.414)%
  --(6.135,4.415)--(6.142,4.417)--(6.149,4.421)--(6.149,4.426)--(6.155,4.429)--(6.162,4.432)%
  --(6.169,4.437)--(6.169,4.442)--(6.175,4.445)--(6.188,4.447)--(6.195,4.452)--(6.202,4.457)%
  --(6.208,4.460)--(6.208,4.465)--(6.215,4.468)--(6.222,4.471)--(6.228,4.476)--(6.228,4.479)%
  --(6.235,4.483)--(6.242,4.487)--(6.248,4.490)--(6.255,4.495)--(6.268,4.502)--(6.275,4.507)%
  --(6.281,4.512)--(6.288,4.517)--(6.295,4.523)--(6.301,4.527)--(6.315,4.532)--(6.321,4.539)%
  --(6.328,4.544)--(6.335,4.549)--(6.341,4.554)--(6.348,4.559)--(6.355,4.565)--(6.361,4.569)%
  --(6.361,4.572)--(6.368,4.574)--(6.374,4.574)--(6.374,4.577)--(6.381,4.580)--(6.381,4.585)%
  --(6.388,4.588)--(6.388,4.590)--(6.388,4.592)--(6.394,4.595)--(6.401,4.597)--(6.408,4.600)%
  --(6.408,4.604)--(6.421,4.608)--(6.428,4.616)--(6.434,4.627)--(6.448,4.631)--(6.448,4.633)%
  --(6.448,4.635)--(6.461,4.637)--(6.467,4.636)--(6.474,4.642)--(6.481,4.650)--(6.494,4.657)%
  --(6.501,4.664)--(6.514,4.669)--(6.514,4.673)--(6.521,4.679)--(6.527,4.685)--(6.534,4.690)%
  --(6.541,4.693)--(6.547,4.700)--(6.554,4.706)--(6.567,4.715)--(6.574,4.723)--(6.580,4.729)%
  --(6.587,4.732)--(6.587,4.733)--(6.594,4.735)--(6.600,4.735)--(6.607,4.740)--(6.614,4.746)%
  --(6.620,4.753)--(6.627,4.761)--(6.634,4.767)--(6.647,4.771)--(6.647,4.777)--(6.653,4.781)%
  --(6.667,4.784)--(6.673,4.790)--(6.680,4.798)--(6.680,4.799)--(6.687,4.804)--(6.687,4.805)%
  --(6.687,4.806)--(6.700,4.813)--(6.713,4.819)--(6.720,4.826)--(6.733,4.837)--(6.746,4.844)%
  --(6.766,4.852)--(6.780,4.863)--(6.800,4.877)--(6.813,4.886)--(6.819,4.892)--(6.826,4.899)%
  --(6.839,4.906)--(6.846,4.911)--(6.853,4.921)--(6.859,4.925)--(6.873,4.932)--(6.879,4.939)%
  --(6.886,4.947)--(6.893,4.951)--(6.893,4.953)--(6.899,4.956)--(6.899,4.961)--(6.912,4.966)%
  --(6.926,4.973)--(6.932,4.981)--(6.939,4.985)--(6.946,4.988)--(6.952,4.989)--(6.959,4.991)%
  --(6.966,4.997)--(6.979,5.005)--(6.992,5.013)--(7.012,5.027)--(7.025,5.040)--(7.039,5.048)%
  --(7.045,5.052)--(7.052,5.057)--(7.052,5.059)--(7.065,5.063)--(7.072,5.067)--(7.072,5.073)%
  --(7.079,5.077)--(7.085,5.087)--(7.098,5.092)--(7.105,5.099)--(7.118,5.108)--(7.125,5.114)%
  --(7.132,5.114)--(7.145,5.124)--(7.152,5.130)--(7.152,5.133)--(7.152,5.138)--(7.158,5.143)%
  --(7.165,5.148)--(7.171,5.151)--(7.178,5.154)--(7.178,5.155)--(7.185,5.156)--(7.191,5.158)%
  --(7.198,5.160)--(7.205,5.168)--(7.205,5.171)--(7.211,5.175)--(7.225,5.179)--(7.231,5.180)%
  --(7.231,5.184)--(7.238,5.190)--(7.245,5.195)--(7.251,5.200)--(7.258,5.205)--(7.264,5.211)%
  --(7.278,5.218)--(7.284,5.225)--(7.291,5.228)--(7.298,5.233)--(7.304,5.236)--(7.311,5.243)%
  --(7.311,5.249)--(7.318,5.252)--(7.324,5.256)--(7.324,5.259)--(7.331,5.262)--(7.338,5.264)%
  --(7.338,5.269)--(7.338,5.268)--(7.338,5.271)--(7.344,5.273)--(7.344,5.275)--(7.351,5.275)%
  --(7.357,5.282)--(7.357,5.286)--(7.371,5.290)--(7.371,5.293)--(7.377,5.298)--(7.384,5.299)%
  --(7.397,5.308)--(7.404,5.311)--(7.404,5.314)--(7.411,5.318)--(7.417,5.327)--(7.424,5.331)%
  --(7.431,5.334)--(7.437,5.337)--(7.437,5.334)--(7.444,5.341)--(7.450,5.343)--(7.450,5.347)%
  --(7.457,5.352)--(7.470,5.357)--(7.477,5.362)--(7.484,5.369)--(7.497,5.373)--(7.510,5.379)%
  --(7.510,5.385)--(7.524,5.391)--(7.530,5.398)--(7.537,5.403)--(7.550,5.410)--(7.563,5.416)%
  --(7.570,5.425)--(7.577,5.433)--(7.590,5.436)--(7.597,5.437)--(7.597,5.445)--(7.603,5.450)%
  --(7.603,5.453)--(7.610,5.459)--(7.623,5.464)--(7.630,5.469)--(7.630,5.472)--(7.636,5.476)%
  --(7.643,5.479)--(7.650,5.482)--(7.656,5.490)--(7.656,5.493)--(7.663,5.493)--(7.663,5.500)%
  --(7.663,5.502)--(7.670,5.505)--(7.676,5.507)--(7.676,5.509)--(7.683,5.507)--(7.690,5.511)%
  --(7.690,5.513)--(7.690,5.515)--(7.696,5.518)--(7.703,5.521)--(7.709,5.524)--(7.716,5.529)%
  --(7.723,5.532)--(7.729,5.534)--(7.736,5.531)--(7.743,5.543)--(7.749,5.549)--(7.756,5.559)%
  --(7.763,5.562)--(7.776,5.566)--(7.783,5.570)--(7.789,5.576)--(7.802,5.583)--(7.809,5.589)%
  --(7.816,5.596)--(7.816,5.598)--(7.822,5.601)--(7.829,5.603)--(7.829,5.608)--(7.836,5.612)%
  --(7.842,5.616)--(7.842,5.621)--(7.842,5.622)--(7.849,5.618)--(7.849,5.620)--(7.856,5.623)%
  --(7.862,5.625)--(7.862,5.632)--(7.862,5.634)--(7.869,5.638)--(7.869,5.639)--(7.869,5.641)%
  --(7.876,5.642)--(7.876,5.640)--(7.876,5.642)--(7.876,5.643)--(7.876,5.644)--(7.882,5.651)%
  --(7.882,5.653)--(7.895,5.656)--(7.895,5.657)--(7.895,5.662)--(7.902,5.662)--(7.902,5.669)%
  --(7.909,5.672)--(7.909,5.675)--(7.915,5.676)--(7.915,5.678)--(7.922,5.679)--(7.929,5.681)%
  --(7.929,5.682)--(7.929,5.681)--(7.935,5.683)--(7.942,5.686)--(7.942,5.687)--(7.955,5.692)%
  --(7.955,5.695)--(7.962,5.698)--(7.962,5.700)--(7.969,5.702)--(7.969,5.703)--(7.969,5.707)%
  --(7.969,5.708)--(7.975,5.711)--(7.982,5.713)--(7.988,5.716)--(7.995,5.719)--(8.002,5.723)%
  --(8.015,5.729)--(8.022,5.736)--(8.035,5.738)--(8.048,5.744)--(8.055,5.752)--(8.062,5.757)%
  --(8.068,5.765)--(8.075,5.772)--(8.081,5.778)--(8.088,5.783)--(8.095,5.788)--(8.101,5.789)%
  --(8.108,5.790)--(8.108,5.793)--(8.108,5.797)--(8.121,5.798)--(8.121,5.806)--(8.128,5.808)%
  --(8.128,5.812)--(8.135,5.816)--(8.141,5.821)--(8.154,5.830)--(8.161,5.839)--(8.168,5.845)%
  --(8.174,5.849)--(8.188,5.854)--(8.201,5.858)--(8.201,5.862)--(8.214,5.865)--(8.214,5.863)%
  --(8.221,5.864)--(8.221,5.867)--(8.228,5.873)--(8.234,5.881)--(8.254,5.893)--(8.261,5.898)%
  --(8.267,5.907)--(8.287,5.912)--(8.287,5.917)--(8.301,5.924)--(8.321,5.936)--(8.327,5.945)%
  --(8.334,5.952)--(8.340,5.958)--(8.340,5.960)--(8.347,5.965)--(8.354,5.968)--(8.354,5.971)%
  --(8.360,5.973)--(8.367,5.974)--(8.374,5.981)--(8.380,5.985)--(8.380,5.987)--(8.387,5.998)%
  --(8.394,6.003)--(8.407,6.010)--(8.407,6.017)--(8.414,6.021)--(8.420,6.021)--(8.433,6.025)%
  --(8.440,6.027)--(8.447,6.031)--(8.453,6.036)--(8.460,6.045)--(8.467,6.048)--(8.467,6.052)%
  --(8.480,6.056)--(8.487,6.059)--(8.500,6.060)--(8.507,6.068)--(8.520,6.075)--(8.533,6.082)%
  --(8.546,6.099)--(8.560,6.106)--(8.573,6.112)--(8.573,6.118)--(8.573,6.121)--(8.580,6.124)%
  --(8.586,6.128)--(8.599,6.132)--(8.606,6.140)--(8.613,6.145)--(8.613,6.153)--(8.619,6.156)%
  --(8.619,6.161)--(8.633,6.165)--(8.633,6.169)--(8.633,6.166)--(8.639,6.178)--(8.646,6.180)%
  --(8.653,6.186)--(8.659,6.194)--(8.679,6.201)--(8.679,6.206)--(8.686,6.209)--(8.692,6.212)%
  --(8.699,6.212)--(8.706,6.213)--(8.719,6.218)--(8.726,6.224)--(8.739,6.228)--(8.746,6.239)%
  --(8.752,6.247)--(8.759,6.252)--(8.772,6.258)--(8.785,6.269)--(8.792,6.274)--(8.805,6.275)%
  --(8.805,6.280)--(8.812,6.285)--(8.825,6.291)--(8.832,6.302)--(8.839,6.307)--(8.839,6.315)%
  --(8.852,6.321)--(8.859,6.322)--(8.859,6.326)--(8.865,6.330)--(8.872,6.335)--(8.872,6.337)%
  --(8.878,6.343)--(8.878,6.349)--(8.892,6.354)--(8.905,6.362)--(8.905,6.367)--(8.918,6.367)%
  --(8.925,6.370)--(8.932,6.374)--(8.945,6.379)--(8.952,6.387)--(8.958,6.396)--(8.971,6.401)%
  --(8.985,6.408)--(8.985,6.411)--(8.991,6.414)--(8.991,6.417)--(8.998,6.423)--(9.011,6.430)%
  --(9.025,6.440)--(9.038,6.451)--(9.058,6.459)--(9.071,6.471)--(9.084,6.482)--(9.098,6.491)%
  --(9.104,6.492)--(9.098,6.494)--(9.104,6.497)--(9.104,6.499)--(9.104,6.501)--(9.111,6.508)%
  --(9.111,6.510)--(9.111,6.511)--(9.118,6.508)--(9.118,6.510)--(9.124,6.513)--(9.131,6.516)%
  --(9.124,6.523)--(9.137,6.528)--(9.137,6.531)--(9.144,6.536)--(9.151,6.542)--(9.151,6.546)%
  --(9.157,6.549)--(9.157,6.552)--(9.171,6.556)--(9.177,6.552)--(9.184,6.559)--(9.184,6.563)%
  --(9.191,6.566)--(9.197,6.568)--(9.197,6.572)--(9.211,6.570)--(9.217,6.575)--(9.224,6.579)%
  --(9.230,6.585)--(9.230,6.592)--(9.237,6.596)--(9.237,6.601)--(9.250,6.603)--(9.264,6.608)%
  --(9.270,6.616)--(9.270,6.623)--(9.290,6.632)--(9.297,6.637)--(9.297,6.641)--(9.304,6.645)%
  --(9.310,6.646)--(9.317,6.650)--(9.317,6.652)--(9.323,6.656)--(9.323,6.655)--(9.337,6.659)%
  --(9.337,6.663)--(9.350,6.668)--(9.343,6.676)--(9.350,6.682)--(9.350,6.686)--(9.363,6.690)%
  --(9.363,6.695)--(9.370,6.697)--(9.377,6.696)--(9.383,6.700)--(9.383,6.706)--(9.390,6.710)%
  --(9.397,6.720)--(9.397,6.724)--(9.403,6.727)--(9.410,6.729)--(9.416,6.733)--(9.416,6.731)%
  --(9.430,6.739)--(9.430,6.740)--(9.436,6.743)--(9.443,6.745)--(9.450,6.750)--(9.456,6.755)%
  --(9.463,6.761)--(9.476,6.764)--(9.476,6.769)--(9.483,6.765)--(9.483,6.769)--(9.496,6.778)%
  --(9.509,6.783)--(9.516,6.791)--(9.523,6.800)--(9.529,6.810)--(9.536,6.816)--(9.549,6.817)%
  --(9.549,6.821)--(9.563,6.824)--(9.563,6.831)--(9.569,6.836)--(9.569,6.841)--(9.576,6.845)%
  --(9.576,6.848)--(9.589,6.850)--(9.582,6.854)--(9.602,6.857)--(9.609,6.861)--(9.616,6.868)%
  --(9.616,6.872)--(9.622,6.878)--(9.622,6.884)--(9.629,6.889)--(9.642,6.895)--(9.649,6.900)%
  --(9.662,6.903)--(9.656,6.897)--(9.656,6.908)--(9.669,6.912)--(9.662,6.915)--(9.669,6.917)%
  --(9.682,6.917)--(9.689,6.919)--(9.702,6.928)--(9.709,6.936)--(9.722,6.938)--(9.729,6.941)%
  --(9.729,6.946)--(9.735,6.949)--(9.735,6.954)--(9.742,6.959)--(9.755,6.968)--(9.775,6.979)%
  --(9.782,6.984)--(9.782,6.988)--(9.788,6.979)--(9.788,6.991)--(9.795,6.994)--(9.795,6.997)%
  --(9.795,7.001)--(9.802,7.005)--(9.808,7.009)--(9.822,7.016)--(9.828,7.021)--(9.835,7.021)%
  --(9.835,7.024)--(9.835,7.030)--(9.835,7.032)--(9.842,7.036)--(9.842,7.037)--(9.842,7.040)%
  --(9.848,7.044)--(9.861,7.048)--(9.861,7.044)--(9.861,7.056)--(9.875,7.061)--(9.881,7.068)%
  --(9.888,7.073)--(9.881,7.074)--(9.888,7.077)--(9.895,7.079)--(9.895,7.082)--(9.908,7.085)%
  --(9.915,7.089)--(9.921,7.086)--(9.928,7.095)--(9.928,7.097)--(9.935,7.102)--(9.935,7.106)%
  --(9.941,7.110)--(9.948,7.113)--(9.961,7.115)--(9.961,7.111)--(9.974,7.126)--(9.974,7.130)%
  --(9.981,7.136)--(9.994,7.142)--(10.008,7.147)--(10.014,7.152)--(10.021,7.157)--(10.027,7.161)%
  --(10.027,7.163)--(10.027,7.170)--(10.041,7.179)--(10.047,7.184)--(10.061,7.189)--(10.067,7.192)%
  --(10.074,7.197)--(10.074,7.203)--(10.081,7.212)--(10.087,7.216)--(10.101,7.215)--(10.101,7.220)%
  --(10.107,7.228)--(10.120,7.236)--(10.120,7.242)--(10.120,7.246)--(10.127,7.249)--(10.127,7.253)%
  --(10.134,7.254)--(10.134,7.257)--(10.134,7.253)--(10.147,7.255)--(10.154,7.259)--(10.160,7.264)%
  --(10.167,7.274)--(10.167,7.279)--(10.174,7.280)--(10.187,7.281)--(10.194,7.281)--(10.207,7.284)%
  --(10.207,7.287)--(10.213,7.297)--(10.227,7.307)--(10.233,7.318)--(10.253,7.327)--(10.260,7.336)%
  --(10.273,7.344)--(10.280,7.347)--(10.287,7.350)--(10.300,7.363)--(10.313,7.368)--(10.320,7.381)%
  --(10.326,7.388)--(10.333,7.391)--(10.326,7.393);
\gpcolor{color=gp lt color 2}
\gpsetlinetype{gp lt plot 2}
\draw[gp path] (2.569,1.931)--(2.575,1.934)--(2.582,1.939)--(2.589,1.943)--(2.595,1.948)%
  --(2.602,1.955)--(2.615,1.963)--(2.622,1.968)--(2.628,1.970)--(2.635,1.975)--(2.642,1.981)%
  --(2.648,1.985)--(2.648,1.989)--(2.655,1.991)--(2.655,1.993)--(2.662,1.997)--(2.668,2.001)%
  --(2.675,2.005)--(2.682,2.008)--(2.682,2.014)--(2.688,2.018)--(2.695,2.021)--(2.702,2.024)%
  --(2.702,2.028)--(2.708,2.030)--(2.715,2.035)--(2.721,2.037)--(2.721,2.041)--(2.728,2.046)%
  --(2.735,2.049)--(2.741,2.055)--(2.748,2.058)--(2.755,2.065)--(2.761,2.072)--(2.761,2.073)%
  --(2.761,2.074)--(2.768,2.077)--(2.775,2.083)--(2.781,2.085)--(2.775,2.086)--(2.781,2.086)%
  --(2.781,2.088)--(2.788,2.091)--(2.794,2.097)--(2.808,2.104)--(2.814,2.109)--(2.814,2.112)%
  --(2.821,2.114)--(2.828,2.121)--(2.834,2.123)--(2.834,2.127)--(2.841,2.130)--(2.848,2.132)%
  --(2.854,2.137)--(2.861,2.139)--(2.868,2.145)--(2.868,2.146)--(2.868,2.147)--(2.881,2.153)%
  --(2.887,2.159)--(2.887,2.163)--(2.894,2.165)--(2.894,2.168)--(2.901,2.171)--(2.914,2.176)%
  --(2.921,2.183)--(2.927,2.188)--(2.934,2.193)--(2.934,2.195)--(2.941,2.198)--(2.947,2.203)%
  --(2.947,2.206)--(2.954,2.209)--(2.954,2.211)--(2.961,2.214)--(2.967,2.216)--(2.974,2.220)%
  --(2.980,2.226)--(2.987,2.233)--(2.994,2.237)--(3.000,2.241)--(3.000,2.244)--(3.007,2.249)%
  --(3.020,2.254)--(3.027,2.258)--(3.027,2.259)--(3.027,2.260)--(3.027,2.259)--(3.027,2.262)%
  --(3.034,2.265)--(3.040,2.271)--(3.047,2.275)--(3.047,2.276)--(3.047,2.280)--(3.054,2.285)%
  --(3.060,2.288)--(3.067,2.291)--(3.067,2.294)--(3.073,2.296)--(3.073,2.299)--(3.073,2.300)%
  --(3.080,2.304)--(3.093,2.309)--(3.100,2.315)--(3.107,2.322)--(3.120,2.326)--(3.120,2.329)%
  --(3.127,2.331)--(3.133,2.336)--(3.133,2.337)--(3.133,2.338)--(3.133,2.340)--(3.140,2.345)%
  --(3.153,2.349)--(3.160,2.353)--(3.166,2.358)--(3.166,2.359)--(3.166,2.360)--(3.166,2.363)%
  --(3.173,2.366)--(3.173,2.369)--(3.180,2.372)--(3.186,2.375)--(3.193,2.378)--(3.193,2.381)%
  --(3.193,2.383)--(3.200,2.387)--(3.206,2.391)--(3.206,2.393)--(3.213,2.395)--(3.213,2.397)%
  --(3.220,2.400)--(3.226,2.403)--(3.233,2.407)--(3.246,2.416)--(3.253,2.421)--(3.253,2.423)%
  --(3.253,2.424)--(3.259,2.428)--(3.266,2.430)--(3.273,2.434)--(3.273,2.436)--(3.273,2.439)%
  --(3.279,2.443)--(3.279,2.445)--(3.286,2.448)--(3.286,2.453)--(3.293,2.457)--(3.293,2.459)%
  --(3.293,2.461)--(3.299,2.462)--(3.299,2.465)--(3.299,2.466)--(3.306,2.466)--(3.306,2.469)%
  --(3.306,2.471)--(3.313,2.473)--(3.313,2.475)--(3.319,2.476)--(3.319,2.478)--(3.326,2.480)%
  --(3.332,2.486)--(3.339,2.489)--(3.346,2.491)--(3.346,2.493)--(3.352,2.495)--(3.359,2.499)%
  --(3.359,2.502)--(3.366,2.507)--(3.366,2.508)--(3.372,2.508)--(3.379,2.513)--(3.379,2.517)%
  --(3.386,2.520)--(3.386,2.522)--(3.386,2.523)--(3.392,2.527)--(3.406,2.532)--(3.419,2.540)%
  --(3.419,2.545)--(3.425,2.549)--(3.432,2.553)--(3.439,2.558)--(3.445,2.563)--(3.459,2.570)%
  --(3.465,2.575)--(3.472,2.578)--(3.465,2.578)--(3.465,2.579)--(3.465,2.581)--(3.465,2.580)%
  --(3.465,2.581)--(3.465,2.580)--(3.465,2.579)--(3.465,2.581)--(3.465,2.582)--(3.465,2.583)%
  --(3.472,2.587)--(3.472,2.590)--(3.485,2.598)--(3.499,2.603)--(3.512,2.611)--(3.518,2.617)%
  --(3.525,2.624)--(3.532,2.629)--(3.538,2.632)--(3.552,2.637)--(3.558,2.640)--(3.565,2.643)%
  --(3.565,2.646)--(3.572,2.650)--(3.572,2.652)--(3.578,2.656)--(3.578,2.660)--(3.585,2.664)%
  --(3.598,2.672)--(3.611,2.681)--(3.631,2.688)--(3.638,2.693)--(3.638,2.695)--(3.638,2.696)%
  --(3.638,2.698)--(3.638,2.700)--(3.651,2.710)--(3.658,2.714)--(3.671,2.719)--(3.678,2.725)%
  --(3.685,2.731)--(3.691,2.736)--(3.691,2.737)--(3.698,2.739)--(3.698,2.742)--(3.698,2.745)%
  --(3.704,2.750)--(3.718,2.754)--(3.724,2.760)--(3.731,2.765)--(3.744,2.774)--(3.758,2.783)%
  --(3.764,2.790)--(3.764,2.794)--(3.764,2.795)--(3.771,2.798)--(3.771,2.800)--(3.777,2.801)%
  --(3.777,2.803)--(3.791,2.809)--(3.797,2.814)--(3.811,2.820)--(3.811,2.826)--(3.811,2.828)%
  --(3.811,2.830)--(3.824,2.835)--(3.831,2.840)--(3.831,2.842)--(3.837,2.842)--(3.837,2.846)%
  --(3.851,2.857)--(3.864,2.865)--(3.864,2.868)--(3.870,2.869)--(3.870,2.870)--(3.877,2.873)%
  --(3.884,2.877)--(3.890,2.883)--(3.904,2.888)--(3.910,2.893)--(3.917,2.898)--(3.917,2.902)%
  --(3.930,2.908)--(3.937,2.913)--(3.944,2.918)--(3.944,2.920)--(3.944,2.921)--(3.950,2.923)%
  --(3.950,2.927)--(3.957,2.930)--(3.963,2.936)--(3.963,2.938)--(3.970,2.940)--(3.970,2.944)%
  --(3.977,2.945)--(3.983,2.948)--(3.990,2.953)--(3.997,2.958)--(4.003,2.961)--(4.003,2.964)%
  --(4.010,2.968)--(4.017,2.974)--(4.017,2.977)--(4.023,2.980)--(4.030,2.984)--(4.037,2.986)%
  --(4.037,2.990)--(4.037,2.991)--(4.037,2.992)--(4.037,2.993)--(4.037,2.994)--(4.037,2.993)%
  --(4.043,2.996)--(4.083,3.020)--(4.096,3.028)--(4.096,3.032)--(4.103,3.039)--(4.110,3.043)%
  --(4.116,3.045)--(4.116,3.046)--(4.123,3.049)--(4.123,3.050)--(4.130,3.051)--(4.130,3.055)%
  --(4.136,3.058)--(4.143,3.063)--(4.149,3.068)--(4.156,3.072)--(4.163,3.076)--(4.163,3.081)%
  --(4.169,3.088)--(4.176,3.091)--(4.183,3.092)--(4.189,3.095)--(4.196,3.102)--(4.203,3.106)%
  --(4.209,3.110)--(4.209,3.114)--(4.222,3.120)--(4.236,3.129)--(4.236,3.130)--(4.236,3.133)%
  --(4.236,3.134)--(4.236,3.136)--(4.236,3.137)--(4.242,3.137)--(4.249,3.146)--(4.262,3.155)%
  --(4.269,3.158)--(4.276,3.160)--(4.276,3.161)--(4.276,3.162)--(4.276,3.164)--(4.282,3.168)%
  --(4.282,3.171)--(4.289,3.171)--(4.296,3.174)--(4.296,3.178)--(4.302,3.181)--(4.302,3.183)%
  --(4.302,3.184)--(4.309,3.186)--(4.309,3.189)--(4.315,3.190)--(4.329,3.196)--(4.342,3.208)%
  --(4.355,3.216)--(4.355,3.219)--(4.362,3.222)--(4.369,3.225)--(4.375,3.227)--(4.375,3.231)%
  --(4.389,3.237)--(4.389,3.241)--(4.395,3.248)--(4.408,3.254)--(4.415,3.259)--(4.422,3.265)%
  --(4.435,3.270)--(4.435,3.274)--(4.448,3.280)--(4.455,3.286)--(4.462,3.292)--(4.468,3.298)%
  --(4.482,3.307)--(4.488,3.311)--(4.495,3.315)--(4.495,3.318)--(4.501,3.320)--(4.501,3.321)%
  --(4.508,3.326)--(4.508,3.328)--(4.508,3.331)--(4.508,3.333)--(4.515,3.334)--(4.515,3.335)%
  --(4.521,3.337)--(4.521,3.338)--(4.521,3.341)--(4.528,3.341)--(4.528,3.343)--(4.535,3.347)%
  --(4.535,3.349)--(4.541,3.351)--(4.541,3.354)--(4.548,3.357)--(4.555,3.361)--(4.555,3.365)%
  --(4.561,3.366)--(4.561,3.368)--(4.568,3.373)--(4.575,3.377)--(4.575,3.380)--(4.581,3.383)%
  --(4.588,3.385)--(4.594,3.390)--(4.608,3.395)--(4.608,3.401)--(4.614,3.403)--(4.621,3.404)%
  --(4.621,3.407)--(4.621,3.408)--(4.628,3.410)--(4.628,3.412)--(4.634,3.417)--(4.641,3.423)%
  --(4.648,3.428)--(4.661,3.431)--(4.661,3.436)--(4.674,3.441)--(4.681,3.450)--(4.687,3.457)%
  --(4.701,3.465)--(4.707,3.468)--(4.707,3.469)--(4.707,3.468)--(4.707,3.470)--(4.714,3.472)%
  --(4.714,3.476)--(4.721,3.482)--(4.727,3.487)--(4.734,3.491)--(4.741,3.495)--(4.747,3.500)%
  --(4.754,3.503)--(4.754,3.506)--(4.760,3.510)--(4.760,3.508)--(4.760,3.510)--(4.760,3.514)%
  --(4.767,3.515)--(4.767,3.518)--(4.774,3.519)--(4.780,3.523)--(4.780,3.526)--(4.787,3.528)%
  --(4.787,3.531)--(4.794,3.535)--(4.794,3.537)--(4.794,3.538)--(4.794,3.539)--(4.794,3.541)%
  --(4.794,3.542)--(4.800,3.542)--(4.800,3.544)--(4.800,3.545)--(4.800,3.547)--(4.807,3.547)%
  --(4.807,3.549)--(4.814,3.552)--(4.820,3.558)--(4.827,3.564)--(4.840,3.569)--(4.853,3.575)%
  --(4.860,3.580)--(4.867,3.583)--(4.873,3.588)--(4.873,3.593)--(4.880,3.598)--(4.887,3.602)%
  --(4.893,3.607)--(4.900,3.612)--(4.913,3.617)--(4.920,3.625)--(4.927,3.631)--(4.940,3.634)%
  --(4.940,3.640)--(4.946,3.645)--(4.953,3.647)--(4.953,3.648)--(4.953,3.651)--(4.966,3.658)%
  --(4.973,3.665)--(4.973,3.668)--(4.980,3.672)--(4.986,3.678)--(4.993,3.681)--(5.000,3.682)%
  --(5.000,3.686)--(5.006,3.690)--(5.013,3.693)--(5.013,3.696)--(5.020,3.699)--(5.026,3.703)%
  --(5.033,3.705)--(5.033,3.709)--(5.039,3.709)--(5.039,3.711)--(5.046,3.716)--(5.046,3.718)%
  --(5.046,3.719)--(5.046,3.720)--(5.046,3.719)--(5.053,3.723)--(5.053,3.726)--(5.053,3.728)%
  --(5.059,3.731)--(5.059,3.732)--(5.059,3.735)--(5.066,3.734)--(5.073,3.736)--(5.073,3.737)%
  --(5.073,3.740)--(5.079,3.747)--(5.086,3.751)--(5.093,3.753)--(5.099,3.755)--(5.106,3.759)%
  --(5.113,3.762)--(5.113,3.765)--(5.119,3.770)--(5.126,3.774)--(5.132,3.776)--(5.132,3.783)%
  --(5.139,3.786)--(5.146,3.788)--(5.146,3.790)--(5.152,3.794)--(5.159,3.800)--(5.166,3.806)%
  --(5.172,3.806)--(5.172,3.811)--(5.179,3.813)--(5.179,3.815)--(5.186,3.816)--(5.192,3.820)%
  --(5.192,3.822)--(5.199,3.826)--(5.199,3.827)--(5.199,3.833)--(5.205,3.840)--(5.212,3.842)%
  --(5.225,3.845)--(5.225,3.849)--(5.232,3.851)--(5.232,3.852)--(5.232,3.855)--(5.239,3.857)%
  --(5.239,3.858)--(5.245,3.861)--(5.245,3.863)--(5.245,3.865)--(5.252,3.867)--(5.252,3.863)%
  --(5.252,3.867)--(5.252,3.868)--(5.259,3.869)--(5.259,3.870)--(5.259,3.872)--(5.259,3.874)%
  --(5.259,3.876)--(5.259,3.875)--(5.259,3.877)--(5.265,3.880)--(5.265,3.881)--(5.272,3.882)%
  --(5.272,3.884)--(5.272,3.886)--(5.279,3.889)--(5.279,3.890)--(5.285,3.891)--(5.285,3.892)%
  --(5.285,3.893)--(5.292,3.899)--(5.298,3.902)--(5.305,3.907)--(5.312,3.913)--(5.325,3.917)%
  --(5.325,3.920)--(5.332,3.923)--(5.338,3.927)--(5.358,3.933)--(5.365,3.944)--(5.372,3.949)%
  --(5.372,3.951)--(5.378,3.952)--(5.378,3.954)--(5.378,3.951)--(5.378,3.956)--(5.378,3.957)%
  --(5.378,3.959)--(5.385,3.960)--(5.385,3.962)--(5.391,3.965)--(5.398,3.970)--(5.405,3.973)%
  --(5.405,3.972)--(5.405,3.973)--(5.411,3.979)--(5.411,3.982)--(5.418,3.984)--(5.418,3.987)%
  --(5.431,3.993)--(5.431,3.995)--(5.438,3.997)--(5.445,4.000)--(5.451,4.008)--(5.458,4.012)%
  --(5.458,4.015)--(5.465,4.017)--(5.471,4.020)--(5.471,4.021)--(5.478,4.024)--(5.478,4.025)%
  --(5.478,4.029)--(5.484,4.030)--(5.484,4.032)--(5.491,4.035)--(5.498,4.038)--(5.504,4.043)%
  --(5.504,4.047)--(5.511,4.051)--(5.518,4.054)--(5.524,4.055)--(5.524,4.056)--(5.531,4.061)%
  --(5.538,4.063)--(5.538,4.066)--(5.538,4.071)--(5.538,4.072)--(5.538,4.073)--(5.538,4.071)%
  --(5.538,4.073)--(5.544,4.075)--(5.544,4.076)--(5.544,4.080)--(5.551,4.085)--(5.558,4.087)%
  --(5.558,4.089)--(5.564,4.091)--(5.564,4.094)--(5.571,4.094)--(5.571,4.095)--(5.577,4.095)%
  --(5.577,4.099)--(5.591,4.100)--(5.591,4.104)--(5.591,4.108)--(5.597,4.112)--(5.604,4.113)%
  --(5.611,4.117)--(5.611,4.122)--(5.611,4.124)--(5.611,4.123)--(5.611,4.124)--(5.611,4.125)%
  --(5.611,4.126)--(5.617,4.129)--(5.624,4.131)--(5.624,4.133)--(5.631,4.133)--(5.637,4.138)%
  --(5.651,4.147)--(5.657,4.153)--(5.664,4.157)--(5.664,4.160)--(5.664,4.162)--(5.664,4.161)%
  --(5.670,4.163)--(5.677,4.167)--(5.677,4.171)--(5.684,4.175)--(5.684,4.180)--(5.690,4.183)%
  --(5.690,4.188)--(5.697,4.188)--(5.697,4.189)--(5.704,4.191)--(5.704,4.193)--(5.704,4.196)%
  --(5.710,4.196)--(5.717,4.201)--(5.717,4.203)--(5.717,4.205)--(5.724,4.206)--(5.724,4.208)%
  --(5.724,4.206)--(5.724,4.207)--(5.724,4.209)--(5.724,4.210)--(5.730,4.211)--(5.730,4.215)%
  --(5.730,4.216)--(5.730,4.217)--(5.737,4.218)--(5.737,4.220)--(5.743,4.223)--(5.743,4.225)%
  --(5.743,4.228)--(5.750,4.229)--(5.757,4.230)--(5.757,4.233)--(5.763,4.236)--(5.763,4.238)%
  --(5.770,4.239)--(5.770,4.240)--(5.777,4.241)--(5.777,4.243)--(5.783,4.246)--(5.783,4.252)%
  --(5.783,4.253)--(5.790,4.255)--(5.790,4.256)--(5.790,4.257)--(5.790,4.258)--(5.797,4.261)%
  --(5.797,4.263)--(5.803,4.269)--(5.817,4.275)--(5.830,4.278)--(5.830,4.281)--(5.836,4.285)%
  --(5.836,4.288)--(5.836,4.289)--(5.836,4.290)--(5.843,4.290)--(5.850,4.293)--(5.850,4.296)%
  --(5.850,4.298)--(5.850,4.302)--(5.863,4.306)--(5.863,4.311)--(5.870,4.312)--(5.876,4.315)%
  --(5.890,4.320)--(5.890,4.327)--(5.903,4.334)--(5.910,4.340)--(5.916,4.345)--(5.923,4.350)%
  --(5.929,4.354)--(5.929,4.356)--(5.929,4.357)--(5.929,4.358)--(5.929,4.359)--(5.929,4.360)%
  --(5.936,4.362)--(5.943,4.364)--(5.943,4.368)--(5.949,4.373)--(5.949,4.376)--(5.956,4.377)%
  --(5.963,4.377)--(5.969,4.381)--(5.969,4.385)--(5.983,4.391)--(5.983,4.396)--(5.989,4.397)%
  --(5.989,4.399)--(5.996,4.400)--(6.003,4.402)--(5.996,4.405)--(6.009,4.412)--(6.016,4.415)%
  --(6.009,4.419)--(6.016,4.419)--(6.022,4.421)--(6.029,4.424)--(6.036,4.427)--(6.042,4.430)%
  --(6.042,4.434)--(6.049,4.441)--(6.056,4.447)--(6.062,4.450)--(6.062,4.452)--(6.069,4.452)%
  --(6.076,4.456)--(6.076,4.458)--(6.082,4.464)--(6.089,4.467)--(6.089,4.470)--(6.096,4.470)%
  --(6.096,4.473)--(6.102,4.475)--(6.102,4.476)--(6.102,4.478)--(6.102,4.479)--(6.109,4.481)%
  --(6.109,4.484)--(6.109,4.485)--(6.109,4.486)--(6.109,4.485)--(6.109,4.486)--(6.115,4.488)%
  --(6.122,4.489)--(6.129,4.492)--(6.135,4.500)--(6.142,4.505)--(6.149,4.509)--(6.162,4.519)%
  --(6.169,4.524)--(6.162,4.522)--(6.169,4.521)--(6.169,4.524)--(6.175,4.527)--(6.188,4.538)%
  --(6.195,4.544)--(6.208,4.551)--(6.208,4.556)--(6.215,4.558)--(6.222,4.560)--(6.222,4.564)%
  --(6.228,4.567)--(6.228,4.568)--(6.228,4.571)--(6.235,4.572)--(6.235,4.574)--(6.235,4.576)%
  --(6.235,4.573)--(6.248,4.580)--(6.248,4.582)--(6.255,4.586)--(6.262,4.591)--(6.262,4.593)%
  --(6.268,4.595)--(6.268,4.599)--(6.275,4.603)--(6.275,4.605)--(6.275,4.606)--(6.295,4.610)%
  --(6.301,4.618)--(6.308,4.627)--(6.315,4.631)--(6.321,4.633)--(6.321,4.634)--(6.321,4.635)%
  --(6.321,4.636)--(6.321,4.637)--(6.328,4.632)--(6.321,4.633)--(6.321,4.635)--(6.328,4.635)%
  --(6.328,4.638)--(6.335,4.642)--(6.341,4.647)--(6.355,4.648)--(6.361,4.658)--(6.368,4.665)%
  --(6.374,4.667)--(6.381,4.671)--(6.388,4.676)--(6.394,4.682)--(6.401,4.688)--(6.401,4.691)%
  --(6.408,4.691)--(6.414,4.695)--(6.421,4.703)--(6.428,4.707)--(6.428,4.711)--(6.434,4.713)%
  --(6.441,4.719)--(6.448,4.723)--(6.454,4.728)--(6.461,4.732)--(6.461,4.729)--(6.467,4.736)%
  --(6.467,4.739)--(6.467,4.740)--(6.467,4.742)--(6.474,4.743)--(6.474,4.744)--(6.474,4.746)%
  --(6.474,4.748)--(6.487,4.754)--(6.487,4.757)--(6.494,4.759)--(6.494,4.764)--(6.501,4.767)%
  --(6.507,4.768)--(6.507,4.769)--(6.507,4.770)--(6.507,4.772)--(6.514,4.773)--(6.521,4.781)%
  --(6.521,4.782)--(6.527,4.785)--(6.534,4.790)--(6.541,4.796)--(6.547,4.799)--(6.547,4.802)%
  --(6.554,4.804)--(6.560,4.803)--(6.567,4.807)--(6.567,4.812)--(6.580,4.816)--(6.580,4.818)%
  --(6.580,4.819)--(6.587,4.822)--(6.587,4.825)--(6.594,4.827)--(6.594,4.828)--(6.594,4.829)%
  --(6.594,4.826)--(6.594,4.832)--(6.594,4.833)--(6.600,4.835)--(6.607,4.837)--(6.607,4.841)%
  --(6.614,4.844)--(6.620,4.847)--(6.627,4.850)--(6.627,4.852)--(6.627,4.855)--(6.634,4.856)%
  --(6.640,4.863)--(6.653,4.870)--(6.667,4.879)--(6.667,4.883)--(6.673,4.887)--(6.673,4.886)%
  --(6.680,4.893)--(6.687,4.898)--(6.700,4.908)--(6.700,4.912)--(6.707,4.915)--(6.707,4.917)%
  --(6.713,4.920)--(6.720,4.922)--(6.720,4.923)--(6.720,4.921)--(6.720,4.923)--(6.720,4.924)%
  --(6.726,4.926)--(6.733,4.932)--(6.733,4.937)--(6.733,4.939)--(6.740,4.939)--(6.746,4.943)%
  --(6.746,4.945)--(6.753,4.945)--(6.753,4.947)--(6.753,4.949)--(6.753,4.950)--(6.753,4.955)%
  --(6.760,4.958)--(6.766,4.962)--(6.773,4.966)--(6.773,4.967)--(6.773,4.964)--(6.780,4.966)%
  --(6.780,4.970)--(6.793,4.979)--(6.800,4.985)--(6.813,4.989)--(6.819,4.994)--(6.833,5.001)%
  --(6.839,5.005)--(6.839,5.008)--(6.839,5.009)--(6.846,5.013)--(6.846,5.011)--(6.853,5.017)%
  --(6.853,5.018)--(6.859,5.022)--(6.866,5.023)--(6.866,5.027)--(6.873,5.030)--(6.879,5.038)%
  --(6.879,5.041)--(6.886,5.043)--(6.886,5.045)--(6.893,5.048)--(6.906,5.054)--(6.919,5.062)%
  --(6.926,5.070)--(6.932,5.075)--(6.939,5.078)--(6.939,5.076)--(6.939,5.079)--(6.946,5.088)%
  --(6.959,5.093)--(6.972,5.101)--(6.972,5.107)--(6.972,5.109)--(6.979,5.106)--(6.979,5.108)%
  --(6.979,5.109)--(6.979,5.110)--(6.986,5.115)--(6.992,5.118)--(6.999,5.122)--(6.999,5.125)%
  --(6.999,5.129)--(7.005,5.127)--(6.999,5.128)--(7.005,5.130)--(7.012,5.139)--(7.025,5.148)%
  --(7.032,5.153)--(7.039,5.157)--(7.045,5.160)--(7.052,5.161)--(7.065,5.170)--(7.072,5.171)%
  --(7.072,5.175)--(7.072,5.176)--(7.079,5.178)--(7.092,5.182)--(7.092,5.189)--(7.098,5.187)%
  --(7.098,5.193)--(7.098,5.195)--(7.105,5.198)--(7.112,5.203)--(7.118,5.207)--(7.125,5.212)%
  --(7.132,5.219)--(7.145,5.226)--(7.152,5.232)--(7.171,5.243)--(7.178,5.251)--(7.185,5.254)%
  --(7.185,5.255)--(7.191,5.262)--(7.198,5.266)--(7.205,5.269)--(7.198,5.271)--(7.205,5.273)%
  --(7.211,5.274)--(7.211,5.277)--(7.225,5.282)--(7.225,5.288)--(7.231,5.295)--(7.231,5.296)%
  --(7.231,5.297)--(7.238,5.300)--(7.238,5.303)--(7.251,5.301)--(7.251,5.304)--(7.258,5.305)%
  --(7.258,5.308)--(7.264,5.314)--(7.264,5.315)--(7.264,5.318)--(7.264,5.323)--(7.271,5.328)%
  --(7.284,5.328)--(7.291,5.333)--(7.291,5.335)--(7.298,5.337)--(7.304,5.341)--(7.311,5.348)%
  --(7.318,5.352)--(7.324,5.356)--(7.331,5.360)--(7.338,5.363)--(7.344,5.362)--(7.344,5.366)%
  --(7.351,5.372)--(7.357,5.375)--(7.364,5.379)--(7.364,5.382)--(7.371,5.386)--(7.377,5.389)%
  --(7.384,5.393)--(7.391,5.398)--(7.397,5.402)--(7.397,5.404)--(7.404,5.407)--(7.411,5.410)%
  --(7.417,5.418)--(7.424,5.426)--(7.437,5.434)--(7.444,5.439)--(7.450,5.444)--(7.457,5.446)%
  --(7.457,5.448)--(7.457,5.450)--(7.464,5.458)--(7.470,5.463)--(7.464,5.463)--(7.470,5.464)%
  --(7.477,5.467)--(7.484,5.466)--(7.484,5.467)--(7.484,5.470)--(7.490,5.474)--(7.497,5.481)%
  --(7.504,5.484)--(7.510,5.489)--(7.510,5.493)--(7.510,5.496)--(7.517,5.496)--(7.530,5.499)%
  --(7.530,5.507)--(7.537,5.510)--(7.543,5.514)--(7.557,5.519)--(7.563,5.523)--(7.570,5.528)%
  --(7.577,5.532)--(7.590,5.529)--(7.583,5.535)--(7.590,5.538)--(7.590,5.539)--(7.597,5.542)%
  --(7.597,5.543)--(7.597,5.546)--(7.603,5.555)--(7.616,5.560)--(7.623,5.565)--(7.630,5.569)%
  --(7.636,5.573)--(7.643,5.569)--(7.650,5.580)--(7.656,5.585)--(7.663,5.593)--(7.676,5.599)%
  --(7.683,5.603)--(7.683,5.609)--(7.690,5.608)--(7.696,5.618)--(7.703,5.621)--(7.696,5.621)%
  --(7.703,5.624)--(7.709,5.625)--(7.709,5.629)--(7.716,5.634)--(7.716,5.639)--(7.723,5.640)%
  --(7.723,5.641)--(7.723,5.645)--(7.729,5.647)--(7.729,5.648)--(7.736,5.649)--(7.743,5.653)%
  --(7.743,5.657)--(7.756,5.666)--(7.763,5.673)--(7.776,5.671)--(7.783,5.681)--(7.783,5.683)%
  --(7.796,5.686)--(7.809,5.691)--(7.816,5.696)--(7.829,5.702)--(7.829,5.707)--(7.836,5.709)%
  --(7.836,5.710)--(7.849,5.713)--(7.856,5.721)--(7.862,5.725)--(7.862,5.728)--(7.869,5.733)%
  --(7.876,5.739)--(7.889,5.747)--(7.895,5.752)--(7.909,5.758)--(7.915,5.762)--(7.915,5.768)%
  --(7.929,5.773)--(7.935,5.776)--(7.942,5.784)--(7.949,5.788)--(7.955,5.797)--(7.962,5.798)%
  --(7.955,5.800)--(7.962,5.801)--(7.962,5.795)--(7.969,5.803)--(7.969,5.806)--(7.975,5.809)%
  --(7.982,5.815)--(7.988,5.819)--(7.995,5.824)--(7.995,5.825)--(8.002,5.830)--(8.008,5.832)%
  --(8.008,5.838)--(8.015,5.843)--(8.022,5.849)--(8.035,5.852)--(8.042,5.856)--(8.055,5.859)%
  --(8.062,5.863)--(8.062,5.866)--(8.075,5.870)--(8.081,5.870)--(8.081,5.879)--(8.088,5.885)%
  --(8.095,5.889)--(8.095,5.892)--(8.101,5.894)--(8.108,5.898)--(8.115,5.903)--(8.121,5.908)%
  --(8.128,5.910)--(8.135,5.914)--(8.148,5.919)--(8.154,5.926)--(8.154,5.933)--(8.161,5.938)%
  --(8.168,5.942)--(8.174,5.946)--(8.181,5.950)--(8.188,5.955)--(8.188,5.954)--(8.194,5.960)%
  --(8.201,5.965)--(8.201,5.969)--(8.208,5.976)--(8.214,5.980)--(8.221,5.982)--(8.221,5.987)%
  --(8.228,5.992)--(8.234,5.992)--(8.234,5.995)--(8.234,5.997)--(8.241,6.004)--(8.254,6.012)%
  --(8.261,6.020)--(8.261,6.025)--(8.267,6.024)--(8.267,6.026)--(8.267,6.028)--(8.267,6.029)%
  --(8.287,6.031)--(8.294,6.035)--(8.307,6.044)--(8.321,6.052)--(8.327,6.056)--(8.327,6.057)%
  --(8.334,6.056)--(8.334,6.057)--(8.340,6.061)--(8.347,6.066)--(8.354,6.069)--(8.360,6.078)%
  --(8.367,6.084)--(8.374,6.090)--(8.387,6.100)--(8.407,6.106)--(8.407,6.109)--(8.420,6.115)%
  --(8.427,6.123)--(8.433,6.135)--(8.440,6.140)--(8.453,6.145)--(8.453,6.149)--(8.447,6.149)%
  --(8.453,6.151)--(8.453,6.149)--(8.460,6.151)--(8.460,6.155)--(8.467,6.156)--(8.467,6.159)%
  --(8.467,6.165)--(8.467,6.169)--(8.480,6.175)--(8.487,6.181)--(8.500,6.187)--(8.500,6.186)%
  --(8.500,6.189)--(8.507,6.193)--(8.520,6.196)--(8.520,6.206)--(8.533,6.209)--(8.533,6.210)%
  --(8.540,6.214)--(8.546,6.216)--(8.553,6.219)--(8.560,6.222)--(8.560,6.224)--(8.566,6.224)%
  --(8.566,6.226)--(8.573,6.232)--(8.573,6.233)--(8.580,6.236)--(8.580,6.240)--(8.586,6.241)%
  --(8.593,6.241)--(8.599,6.246)--(8.606,6.250)--(8.613,6.254)--(8.619,6.266)--(8.626,6.270)%
  --(8.639,6.276)--(8.639,6.280)--(8.639,6.283)--(8.646,6.282)--(8.653,6.286)--(8.659,6.290)%
  --(8.666,6.296)--(8.666,6.301)--(8.679,6.310)--(8.686,6.318)--(8.692,6.322)--(8.699,6.329)%
  --(8.706,6.336)--(8.706,6.334)--(8.706,6.335)--(8.712,6.335)--(8.719,6.337)--(8.719,6.346)%
  --(8.726,6.351)--(8.732,6.356)--(8.739,6.362)--(8.746,6.367)--(8.752,6.364)--(8.759,6.365)%
  --(8.759,6.368)--(8.766,6.372)--(8.779,6.375)--(8.785,6.381)--(8.785,6.386)--(8.792,6.394)%
  --(8.799,6.398)--(8.805,6.399)--(8.805,6.398)--(8.805,6.400)--(8.805,6.402)--(8.825,6.405)%
  --(8.825,6.412)--(8.825,6.414)--(8.832,6.417)--(8.832,6.418)--(8.839,6.423)--(8.839,6.424)%
  --(8.845,6.429)--(8.859,6.437)--(8.859,6.444)--(8.872,6.448)--(8.872,6.451)--(8.872,6.454)%
  --(8.878,6.455)--(8.878,6.460)--(8.898,6.462)--(8.898,6.463)--(8.905,6.467)--(8.905,6.470)%
  --(8.912,6.473)--(8.918,6.484)--(8.918,6.486)--(8.925,6.491)--(8.932,6.496)--(8.938,6.501)%
  --(8.938,6.500)--(8.945,6.499)--(8.945,6.502)--(8.945,6.505)--(8.952,6.511)--(8.952,6.516)%
  --(8.958,6.520)--(8.958,6.522)--(8.958,6.524)--(8.971,6.524)--(8.978,6.527)--(8.985,6.534)%
  --(8.985,6.539)--(8.985,6.543)--(8.998,6.551)--(8.998,6.554)--(9.011,6.559)--(9.025,6.568)%
  --(9.031,6.571)--(9.031,6.566)--(9.031,6.567)--(9.038,6.573)--(9.038,6.578)--(9.051,6.581)%
  --(9.051,6.584)--(9.058,6.587)--(9.058,6.588)--(9.064,6.588)--(9.064,6.597)--(9.071,6.599)%
  --(9.071,6.601)--(9.078,6.603)--(9.084,6.606)--(9.091,6.612)--(9.098,6.615)--(9.098,6.614)%
  --(9.104,6.619)--(9.111,6.622)--(9.118,6.635)--(9.131,6.639)--(9.131,6.643)--(9.137,6.645)%
  --(9.137,6.648)--(9.144,6.650)--(9.151,6.654)--(9.151,6.656)--(9.157,6.660)--(9.157,6.656)%
  --(9.164,6.666)--(9.164,6.668)--(9.171,6.673)--(9.171,6.677)--(9.177,6.680)--(9.184,6.683)%
  --(9.191,6.680)--(9.184,6.689)--(9.184,6.692)--(9.191,6.694)--(9.191,6.697)--(9.191,6.698)%
  --(9.191,6.700)--(9.197,6.701)--(9.197,6.698)--(9.197,6.703)--(9.197,6.706)--(9.204,6.710)%
  --(9.211,6.714)--(9.217,6.715)--(9.217,6.718)--(9.217,6.720)--(9.230,6.716)--(9.230,6.729)%
  --(9.237,6.734)--(9.237,6.738)--(9.257,6.743)--(9.257,6.747)--(9.264,6.750)--(9.270,6.753)%
  --(9.277,6.757)--(9.284,6.760)--(9.290,6.763)--(9.297,6.765)--(9.297,6.767)--(9.304,6.763)%
  --(9.304,6.772)--(9.317,6.776)--(9.317,6.777)--(9.317,6.780)--(9.323,6.784)--(9.337,6.788)%
  --(9.343,6.796)--(9.350,6.794)--(9.357,6.807)--(9.357,6.812)--(9.363,6.815)--(9.377,6.818)%
  --(9.383,6.821)--(9.383,6.824)--(9.383,6.826)--(9.390,6.829)--(9.397,6.837)--(9.397,6.841)%
  --(9.403,6.844)--(9.403,6.847)--(9.410,6.849)--(9.410,6.850)--(9.416,6.853)--(9.416,6.856)%
  --(9.430,6.859)--(9.430,6.854)--(9.436,6.866)--(9.436,6.868)--(9.436,6.873)--(9.443,6.877)%
  --(9.450,6.880)--(9.443,6.882)--(9.450,6.885)--(9.456,6.889)--(9.463,6.894)--(9.463,6.897)%
  --(9.470,6.898)--(9.470,6.900)--(9.470,6.902)--(9.476,6.906)--(9.483,6.913)--(9.490,6.919)%
  --(9.516,6.925)--(9.529,6.927)--(9.536,6.931)--(9.543,6.934)--(9.536,6.936)--(9.543,6.938)%
  --(9.549,6.945)--(9.549,6.947)--(9.556,6.951)--(9.556,6.952)--(9.556,6.953)--(9.563,6.955)%
  --(9.563,6.957)--(9.569,6.956)--(9.576,6.959)--(9.576,6.967)--(9.582,6.976)--(9.596,6.982)%
  --(9.602,6.985)--(9.609,6.989)--(9.616,6.992)--(9.622,6.994)--(9.622,6.997)--(9.629,7.001)%
  --(9.636,7.003)--(9.636,7.013)--(9.642,7.017)--(9.642,7.019)--(9.642,7.021)--(9.649,7.021)%
  --(9.649,7.018)--(9.656,7.020)--(9.649,7.021)--(9.656,7.021)--(9.656,7.027)--(9.662,7.033)%
  --(9.662,7.037)--(9.669,7.040)--(9.669,7.042)--(9.675,7.044)--(9.682,7.048)--(9.682,7.049)%
  --(9.682,7.053)--(9.689,7.056)--(9.689,7.058)--(9.695,7.059)--(9.702,7.064)--(9.702,7.068)%
  --(9.715,7.071)--(9.709,7.068)--(9.709,7.073)--(9.715,7.077)--(9.729,7.083)--(9.735,7.091)%
  --(9.735,7.096)--(9.749,7.101)--(9.749,7.104)--(9.749,7.105)--(9.755,7.108)--(9.762,7.109)%
  --(9.762,7.105)--(9.775,7.108)--(9.782,7.119)--(9.782,7.123)--(9.788,7.125)--(9.788,7.128)%
  --(9.788,7.130)--(9.802,7.128)--(9.802,7.130)--(9.808,7.133)--(9.815,7.135)--(9.822,7.142)%
  --(9.815,7.146)--(9.822,7.147)--(9.828,7.150)--(9.835,7.152)--(9.835,7.151)--(9.835,7.150)%
  --(9.842,7.154)--(9.842,7.158)--(9.842,7.160)--(9.848,7.167)--(9.848,7.169)--(9.848,7.171)%
  --(9.848,7.175)--(9.848,7.177)--(9.861,7.174)--(9.861,7.177)--(9.861,7.183)--(9.875,7.185)%
  --(9.868,7.188)--(9.875,7.189)--(9.881,7.190)--(9.881,7.192)--(9.888,7.193)--(9.888,7.194)%
  --(9.888,7.190)--(9.888,7.193)--(9.895,7.196)--(9.901,7.201)--(9.901,7.212)--(9.915,7.217)%
  --(9.915,7.222)--(9.921,7.226)--(9.921,7.229)--(9.921,7.230)--(9.928,7.233)--(9.928,7.235)%
  --(9.935,7.230)--(9.935,7.234)--(9.941,7.244)--(9.941,7.247)--(9.948,7.249)--(9.948,7.252)%
  --(9.954,7.253)--(9.961,7.250)--(9.954,7.253)--(9.968,7.255)--(9.968,7.258)--(9.968,7.269)%
  --(9.981,7.275)--(9.981,7.277)--(9.988,7.281)--(9.994,7.282)--(10.001,7.279)--(9.994,7.277)%
  --(10.001,7.278)--(10.008,7.278)--(10.001,7.280)--(10.014,7.287)--(10.014,7.290)--(10.021,7.292)%
  --(10.021,7.295)--(10.027,7.301)--(10.034,7.299)--(10.041,7.304)--(10.041,7.308)--(10.054,7.311)%
  --(10.054,7.316)--(10.061,7.323)--(10.067,7.326)--(10.067,7.328)--(10.074,7.332)--(10.074,7.330)%
  --(10.074,7.332)--(10.081,7.336)--(10.087,7.339)--(10.094,7.349)--(10.101,7.350)--(10.101,7.353)%
  --(10.101,7.355)--(10.101,7.356)--(10.107,7.358)--(10.107,7.359)--(10.107,7.361)--(10.114,7.356)%
  --(10.114,7.362)--(10.114,7.366)--(10.120,7.367)--(10.120,7.369)--(10.120,7.370)--(10.120,7.368)%
  --(10.127,7.369)--(10.127,7.374)--(10.134,7.379)--(10.134,7.381)--(10.147,7.392)--(10.160,7.397)%
  --(10.154,7.399)--(10.160,7.401)--(10.154,7.402)--(10.160,7.401)--(10.167,7.399)--(10.174,7.403)%
  --(10.167,7.406)--(10.174,7.416)--(10.180,7.424)--(10.194,7.432)--(10.207,7.436)--(10.207,7.439)%
  --(10.200,7.434)--(10.207,7.431)--(10.207,7.441);
\gpcolor{color=gp lt color 3}
\gpsetlinetype{gp lt plot 3}
\draw[gp path] (2.283,1.761)--(2.290,1.767)--(2.296,1.774)--(2.303,1.777)--(2.310,1.781)%
  --(2.316,1.786)--(2.323,1.792)--(2.330,1.798)--(2.336,1.803)--(2.349,1.811)--(2.349,1.814)%
  --(2.349,1.816)--(2.356,1.819)--(2.363,1.824)--(2.369,1.827)--(2.376,1.831)--(2.389,1.839)%
  --(2.396,1.844)--(2.403,1.852)--(2.416,1.861)--(2.429,1.871)--(2.442,1.879)--(2.449,1.884)%
  --(2.462,1.890)--(2.469,1.895)--(2.476,1.904)--(2.489,1.912)--(2.502,1.923)--(2.516,1.929)%
  --(2.522,1.935)--(2.535,1.943)--(2.542,1.947)--(2.549,1.955)--(2.555,1.963)--(2.562,1.968)%
  --(2.569,1.973)--(2.582,1.980)--(2.589,1.986)--(2.589,1.989)--(2.595,1.992)--(2.609,2.001)%
  --(2.622,2.013)--(2.635,2.020)--(2.635,2.022)--(2.635,2.024)--(2.635,2.026)--(2.642,2.030)%
  --(2.648,2.033)--(2.648,2.035)--(2.648,2.038)--(2.648,2.040)--(2.655,2.042)--(2.662,2.045)%
  --(2.662,2.049)--(2.668,2.052)--(2.675,2.056)--(2.682,2.060)--(2.688,2.068)--(2.702,2.077)%
  --(2.715,2.085)--(2.721,2.090)--(2.735,2.097)--(2.741,2.101)--(2.741,2.107)--(2.748,2.110)%
  --(2.748,2.112)--(2.755,2.117)--(2.761,2.123)--(2.775,2.128)--(2.781,2.132)--(2.788,2.137)%
  --(2.794,2.141)--(2.794,2.142)--(2.794,2.145)--(2.801,2.148)--(2.814,2.155)--(2.828,2.165)%
  --(2.834,2.172)--(2.834,2.174)--(2.848,2.180)--(2.854,2.187)--(2.861,2.191)--(2.868,2.193)%
  --(2.874,2.200)--(2.887,2.210)--(2.894,2.214)--(2.901,2.216)--(2.901,2.219)--(2.901,2.222)%
  --(2.907,2.225)--(2.907,2.230)--(2.914,2.233)--(2.921,2.236)--(2.927,2.243)--(2.941,2.251)%
  --(2.961,2.259)--(2.974,2.268)--(2.987,2.280)--(2.994,2.287)--(3.000,2.294)--(3.007,2.299)%
  --(3.020,2.307)--(3.034,2.317)--(3.040,2.323)--(3.047,2.327)--(3.060,2.332)--(3.060,2.336)%
  --(3.067,2.340)--(3.067,2.342)--(3.067,2.345)--(3.073,2.346)--(3.073,2.347)--(3.073,2.350)%
  --(3.080,2.353)--(3.087,2.358)--(3.100,2.363)--(3.107,2.370)--(3.120,2.377)--(3.127,2.383)%
  --(3.133,2.390)--(3.140,2.395)--(3.147,2.399)--(3.153,2.404)--(3.160,2.408)--(3.166,2.415)%
  --(3.180,2.420)--(3.186,2.429)--(3.193,2.433)--(3.200,2.437)--(3.213,2.445)--(3.220,2.451)%
  --(3.220,2.453)--(3.220,2.454)--(3.220,2.457)--(3.220,2.459)--(3.226,2.462)--(3.233,2.467)%
  --(3.240,2.472)--(3.246,2.476)--(3.253,2.479)--(3.259,2.484)--(3.259,2.487)--(3.266,2.489)%
  --(3.266,2.490)--(3.266,2.492)--(3.266,2.493)--(3.273,2.495)--(3.279,2.499)--(3.286,2.504)%
  --(3.293,2.508)--(3.293,2.510)--(3.293,2.511)--(3.293,2.510)--(3.299,2.513)--(3.306,2.517)%
  --(3.306,2.521)--(3.313,2.526)--(3.319,2.527)--(3.319,2.529)--(3.319,2.531)--(3.326,2.533)%
  --(3.326,2.535)--(3.326,2.536)--(3.332,2.539)--(3.339,2.545)--(3.339,2.547)--(3.346,2.550)%
  --(3.346,2.552)--(3.352,2.554)--(3.359,2.556)--(3.359,2.559)--(3.366,2.562)--(3.372,2.565)%
  --(3.372,2.570)--(3.379,2.573)--(3.386,2.577)--(3.386,2.579)--(3.392,2.582)--(3.392,2.583)%
  --(3.399,2.587)--(3.406,2.593)--(3.412,2.598)--(3.419,2.601)--(3.432,2.607)--(3.439,2.614)%
  --(3.445,2.619)--(3.452,2.624)--(3.459,2.628)--(3.459,2.630)--(3.465,2.633)--(3.472,2.638)%
  --(3.479,2.643)--(3.485,2.649)--(3.492,2.654)--(3.499,2.657)--(3.499,2.661)--(3.505,2.665)%
  --(3.512,2.669)--(3.518,2.674)--(3.525,2.678)--(3.532,2.680)--(3.532,2.682)--(3.532,2.684)%
  --(3.538,2.686)--(3.538,2.689)--(3.545,2.691)--(3.552,2.693)--(3.558,2.701)--(3.565,2.710)%
  --(3.572,2.712)--(3.578,2.716)--(3.578,2.717)--(3.585,2.721)--(3.592,2.727)--(3.605,2.735)%
  --(3.611,2.741)--(3.625,2.749)--(3.631,2.753)--(3.631,2.756)--(3.638,2.762)--(3.645,2.768)%
  --(3.651,2.773)--(3.665,2.778)--(3.671,2.782)--(3.671,2.787)--(3.678,2.793)--(3.685,2.798)%
  --(3.691,2.801)--(3.698,2.805)--(3.704,2.810)--(3.711,2.814)--(3.718,2.819)--(3.724,2.826)%
  --(3.731,2.833)--(3.738,2.836)--(3.744,2.838)--(3.744,2.841)--(3.751,2.843)--(3.758,2.849)%
  --(3.764,2.855)--(3.764,2.859)--(3.771,2.861)--(3.777,2.865)--(3.777,2.866)--(3.777,2.868)%
  --(3.791,2.871)--(3.804,2.882)--(3.817,2.890)--(3.831,2.898)--(3.844,2.908)--(3.857,2.918)%
  --(3.870,2.928)--(3.877,2.937)--(3.877,2.939)--(3.884,2.939)--(3.890,2.945)--(3.897,2.949)%
  --(3.904,2.954)--(3.910,2.960)--(3.917,2.967)--(3.930,2.976)--(3.937,2.979)--(3.937,2.984)%
  --(3.937,2.986)--(3.944,2.989)--(3.950,2.992)--(3.957,2.995)--(3.957,2.999)--(3.963,3.001)%
  --(3.970,3.006)--(3.977,3.009)--(3.983,3.010)--(3.983,3.016)--(3.990,3.020)--(3.997,3.023)%
  --(3.997,3.026)--(4.003,3.029)--(4.003,3.032)--(4.010,3.039)--(4.017,3.042)--(4.023,3.044)%
  --(4.037,3.051)--(4.043,3.057)--(4.050,3.061)--(4.063,3.070)--(4.076,3.080)--(4.083,3.088)%
  --(4.090,3.092)--(4.103,3.097)--(4.110,3.103)--(4.116,3.108)--(4.123,3.110)--(4.116,3.111)%
  --(4.116,3.112)--(4.116,3.113)--(4.116,3.115)--(4.116,3.116)--(4.116,3.117)--(4.116,3.119)%
  --(4.123,3.119)--(4.123,3.124)--(4.130,3.129)--(4.163,3.147)--(4.169,3.153)--(4.169,3.156)%
  --(4.176,3.157)--(4.183,3.161)--(4.189,3.167)--(4.196,3.171)--(4.196,3.173)--(4.209,3.179)%
  --(4.209,3.181)--(4.209,3.182)--(4.209,3.183)--(4.209,3.181)--(4.209,3.184)--(4.209,3.185)%
  --(4.209,3.186)--(4.216,3.188)--(4.222,3.191)--(4.222,3.194)--(4.229,3.195)--(4.229,3.196)%
  --(4.236,3.201)--(4.242,3.206)--(4.249,3.213)--(4.262,3.224)--(4.282,3.234)--(4.296,3.244)%
  --(4.302,3.250)--(4.309,3.256)--(4.315,3.261)--(4.322,3.265)--(4.329,3.267)--(4.335,3.273)%
  --(4.349,3.281)--(4.355,3.286)--(4.362,3.290)--(4.369,3.295)--(4.375,3.302)--(4.382,3.307)%
  --(4.382,3.310)--(4.389,3.312)--(4.395,3.314)--(4.395,3.316)--(4.402,3.320)--(4.408,3.326)%
  --(4.408,3.329)--(4.408,3.331)--(4.415,3.333)--(4.422,3.334)--(4.422,3.336)--(4.422,3.338)%
  --(4.428,3.342)--(4.435,3.346)--(4.435,3.348)--(4.442,3.349)--(4.442,3.352)--(4.442,3.354)%
  --(4.442,3.356)--(4.442,3.357)--(4.448,3.358)--(4.448,3.360)--(4.448,3.362)--(4.455,3.364)%
  --(4.455,3.366)--(4.462,3.371)--(4.468,3.371)--(4.468,3.374)--(4.468,3.376)--(4.475,3.376)%
  --(4.475,3.380)--(4.482,3.384)--(4.488,3.389)--(4.495,3.394)--(4.508,3.399)--(4.515,3.404)%
  --(4.521,3.408)--(4.528,3.413)--(4.535,3.419)--(4.541,3.425)--(4.555,3.431)--(4.561,3.436)%
  --(4.568,3.442)--(4.575,3.449)--(4.581,3.453)--(4.588,3.458)--(4.594,3.461)--(4.601,3.466)%
  --(4.601,3.468)--(4.608,3.471)--(4.614,3.477)--(4.621,3.480)--(4.621,3.483)--(4.621,3.486)%
  --(4.628,3.488)--(4.634,3.489)--(4.641,3.494)--(4.648,3.499)--(4.648,3.505)--(4.654,3.509)%
  --(4.654,3.510)--(4.661,3.514)--(4.661,3.515)--(4.668,3.517)--(4.668,3.519)--(4.674,3.520)%
  --(4.674,3.524)--(4.674,3.526)--(4.681,3.528)--(4.681,3.529)--(4.681,3.531)--(4.687,3.533)%
  --(4.687,3.535)--(4.687,3.534)--(4.687,3.536)--(4.694,3.538)--(4.694,3.539)--(4.694,3.543)%
  --(4.701,3.545)--(4.707,3.548)--(4.714,3.552)--(4.721,3.556)--(4.727,3.560)--(4.734,3.566)%
  --(4.741,3.571)--(4.747,3.572)--(4.754,3.577)--(4.760,3.584)--(4.767,3.592)--(4.787,3.606)%
  --(4.807,3.615)--(4.814,3.621)--(4.814,3.623)--(4.814,3.625)--(4.814,3.627)--(4.820,3.631)%
  --(4.827,3.635)--(4.834,3.639)--(4.840,3.642)--(4.834,3.643)--(4.840,3.645)--(4.847,3.649)%
  --(4.853,3.652)--(4.853,3.653)--(4.853,3.656)--(4.853,3.658)--(4.853,3.659)--(4.860,3.665)%
  --(4.867,3.668)--(4.873,3.672)--(4.880,3.677)--(4.887,3.682)--(4.900,3.686)--(4.907,3.693)%
  --(4.920,3.701)--(4.927,3.710)--(4.940,3.717)--(4.953,3.726)--(4.966,3.735)--(4.986,3.745)%
  --(5.013,3.764)--(5.039,3.782)--(5.059,3.795)--(5.066,3.805)--(5.086,3.813)--(5.093,3.818)%
  --(5.099,3.829)--(5.106,3.838)--(5.099,3.839)--(5.106,3.841)--(5.106,3.839)--(5.106,3.843)%
  --(5.106,3.842)--(5.099,3.843)--(5.106,3.842)--(5.106,3.843)--(5.099,3.842)--(5.106,3.843)%
  --(5.113,3.848)--(5.113,3.850)--(5.119,3.853)--(5.119,3.855)--(5.126,3.857)--(5.126,3.858)%
  --(5.126,3.856)--(5.132,3.862)--(5.139,3.867)--(5.146,3.867)--(5.146,3.870)--(5.152,3.872)%
  --(5.152,3.875)--(5.152,3.876)--(5.159,3.878)--(5.159,3.881)--(5.166,3.880)--(5.166,3.886)%
  --(5.166,3.887)--(5.172,3.890)--(5.172,3.892)--(5.179,3.895)--(5.179,3.897)--(5.186,3.900)%
  --(5.192,3.906)--(5.199,3.909)--(5.205,3.916)--(5.212,3.921)--(5.212,3.922)--(5.219,3.923)%
  --(5.225,3.925)--(5.225,3.927)--(5.232,3.930)--(5.239,3.932)--(5.239,3.933)--(5.245,3.936)%
  --(5.245,3.943)--(5.252,3.947)--(5.259,3.947)--(5.259,3.952)--(5.259,3.953)--(5.259,3.955)%
  --(5.265,3.956)--(5.272,3.960)--(5.272,3.961)--(5.279,3.964)--(5.279,3.967)--(5.279,3.969)%
  --(5.279,3.970)--(5.285,3.972)--(5.285,3.974)--(5.292,3.976)--(5.292,3.978)--(5.292,3.979)%
  --(5.298,3.980)--(5.305,3.986)--(5.305,3.988)--(5.305,3.990)--(5.312,3.992)--(5.312,3.994)%
  --(5.318,3.997)--(5.325,4.001)--(5.332,4.007)--(5.338,4.009)--(5.338,4.014)--(5.345,4.016)%
  --(5.352,4.020)--(5.358,4.023)--(5.358,4.026)--(5.365,4.030)--(5.372,4.033)--(5.372,4.035)%
  --(5.378,4.038)--(5.385,4.041)--(5.385,4.043)--(5.385,4.046)--(5.385,4.050)--(5.391,4.052)%
  --(5.391,4.053)--(5.391,4.056)--(5.398,4.056)--(5.405,4.057)--(5.405,4.060)--(5.405,4.062)%
  --(5.411,4.064)--(5.411,4.068)--(5.411,4.069)--(5.411,4.071)--(5.418,4.072)--(5.418,4.074)%
  --(5.418,4.076)--(5.425,4.080)--(5.431,4.084)--(5.438,4.086)--(5.445,4.092)--(5.451,4.096)%
  --(5.458,4.098)--(5.458,4.099)--(5.465,4.099)--(5.471,4.099)--(5.478,4.105)--(5.478,4.109)%
  --(5.491,4.118)--(5.491,4.126)--(5.504,4.130)--(5.511,4.133)--(5.511,4.136)--(5.518,4.141)%
  --(5.524,4.141)--(5.524,4.142)--(5.524,4.145)--(5.531,4.146)--(5.538,4.153)--(5.544,4.157)%
  --(5.544,4.160)--(5.544,4.164)--(5.551,4.169)--(5.558,4.173)--(5.564,4.177)--(5.564,4.181)%
  --(5.577,4.187)--(5.577,4.190)--(5.584,4.193)--(5.591,4.198)--(5.591,4.201)--(5.597,4.204)%
  --(5.597,4.206)--(5.597,4.207)--(5.604,4.210)--(5.611,4.212)--(5.611,4.213)--(5.617,4.216)%
  --(5.617,4.220)--(5.624,4.224)--(5.624,4.225)--(5.624,4.227)--(5.631,4.229)--(5.631,4.232)%
  --(5.637,4.231)--(5.637,4.234)--(5.644,4.239)--(5.644,4.241)--(5.644,4.242)--(5.651,4.243)%
  --(5.651,4.245)--(5.651,4.243)--(5.651,4.244)--(5.651,4.246)--(5.657,4.250)--(5.657,4.253)%
  --(5.664,4.256)--(5.664,4.259)--(5.670,4.261)--(5.670,4.262)--(5.670,4.264)--(5.684,4.268)%
  --(5.690,4.272)--(5.697,4.275)--(5.704,4.277)--(5.704,4.283)--(5.704,4.285)--(5.710,4.287)%
  --(5.717,4.289)--(5.717,4.292)--(5.717,4.295)--(5.724,4.296)--(5.724,4.298)--(5.724,4.299)%
  --(5.724,4.302)--(5.730,4.304)--(5.737,4.308)--(5.737,4.311)--(5.750,4.314)--(5.757,4.318)%
  --(5.763,4.323)--(5.763,4.328)--(5.770,4.334)--(5.783,4.340)--(5.790,4.344)--(5.790,4.345)%
  --(5.790,4.350)--(5.797,4.353)--(5.803,4.356)--(5.810,4.359)--(5.810,4.362)--(5.817,4.363)%
  --(5.817,4.364)--(5.823,4.369)--(5.830,4.374)--(5.830,4.377)--(5.836,4.379)--(5.836,4.382)%
  --(5.843,4.383)--(5.850,4.385)--(5.850,4.388)--(5.850,4.393)--(5.850,4.397)--(5.856,4.397)%
  --(5.856,4.399)--(5.863,4.400)--(5.863,4.403)--(5.870,4.402)--(5.870,4.406)--(5.870,4.409)%
  --(5.870,4.410)--(5.870,4.412)--(5.876,4.415)--(5.883,4.416)--(5.883,4.418)--(5.883,4.419)%
  --(5.883,4.418)--(5.890,4.419)--(5.890,4.422)--(5.890,4.423)--(5.890,4.424)--(5.896,4.424)%
  --(5.896,4.427)--(5.903,4.429)--(5.903,4.430)--(5.903,4.432)--(5.910,4.433)--(5.910,4.435)%
  --(5.910,4.438)--(5.910,4.439)--(5.916,4.443)--(5.923,4.447)--(5.923,4.448)--(5.923,4.447)%
  --(5.929,4.451)--(5.929,4.454)--(5.936,4.456)--(5.936,4.459)--(5.943,4.461)--(5.949,4.462)%
  --(5.949,4.461)--(5.949,4.464)--(5.949,4.465)--(5.963,4.468)--(5.963,4.470)--(5.963,4.474)%
  --(5.969,4.476)--(5.969,4.478)--(5.976,4.481)--(5.983,4.484)--(5.983,4.482)--(5.989,4.489)%
  --(5.996,4.492)--(6.003,4.494)--(6.003,4.497)--(6.003,4.500)--(6.009,4.502)--(6.009,4.503)%
  --(6.016,4.502)--(6.022,4.509)--(6.022,4.514)--(6.029,4.519)--(6.029,4.522)--(6.029,4.524)%
  --(6.036,4.525)--(6.036,4.526)--(6.036,4.527)--(6.042,4.529)--(6.042,4.532)--(6.049,4.535)%
  --(6.056,4.539)--(6.062,4.543)--(6.069,4.549)--(6.076,4.554)--(6.082,4.558)--(6.089,4.559)%
  --(6.082,4.557)--(6.082,4.562)--(6.089,4.563)--(6.089,4.567)--(6.089,4.569)--(6.096,4.572)%
  --(6.102,4.573)--(6.109,4.577)--(6.109,4.578)--(6.109,4.579)--(6.115,4.580)--(6.122,4.586)%
  --(6.122,4.588)--(6.122,4.590)--(6.122,4.592)--(6.129,4.594)--(6.129,4.595)--(6.129,4.596)%
  --(6.129,4.599)--(6.129,4.600)--(6.142,4.602)--(6.142,4.606)--(6.149,4.611)--(6.162,4.617)%
  --(6.169,4.627)--(6.182,4.633)--(6.188,4.635)--(6.195,4.636)--(6.202,4.637)--(6.202,4.636)%
  --(6.208,4.643)--(6.215,4.651)--(6.222,4.657)--(6.228,4.662)--(6.228,4.664)--(6.235,4.666)%
  --(6.248,4.671)--(6.248,4.674)--(6.248,4.678)--(6.255,4.679)--(6.255,4.681)--(6.255,4.684)%
  --(6.262,4.688)--(6.262,4.691)--(6.268,4.695)--(6.275,4.701)--(6.275,4.702)--(6.281,4.704)%
  --(6.275,4.705)--(6.281,4.706)--(6.281,4.704)--(6.288,4.708)--(6.288,4.711)--(6.288,4.713)%
  --(6.295,4.715)--(6.295,4.716)--(6.301,4.718)--(6.301,4.721)--(6.301,4.726)--(6.308,4.729)%
  --(6.308,4.730)--(6.315,4.732)--(6.315,4.734)--(6.321,4.736)--(6.321,4.734)--(6.328,4.737)%
  --(6.328,4.739)--(6.328,4.742)--(6.335,4.746)--(6.328,4.747)--(6.335,4.748)--(6.335,4.749)%
  --(6.341,4.749)--(6.341,4.751)--(6.341,4.754)--(6.341,4.755)--(6.341,4.757)--(6.341,4.758)%
  --(6.348,4.758)--(6.348,4.759)--(6.348,4.762)--(6.355,4.762)--(6.355,4.766)--(6.361,4.767)%
  --(6.361,4.768)--(6.361,4.771)--(6.361,4.772)--(6.368,4.776)--(6.368,4.777)--(6.368,4.776)%
  --(6.374,4.776)--(6.374,4.778)--(6.381,4.783)--(6.381,4.785)--(6.381,4.786)--(6.381,4.787)%
  --(6.388,4.789)--(6.388,4.787)--(6.394,4.790)--(6.394,4.792)--(6.401,4.798)--(6.401,4.802)%
  --(6.408,4.804)--(6.414,4.805)--(6.414,4.808)--(6.414,4.807)--(6.414,4.806)--(6.421,4.806)%
  --(6.421,4.808)--(6.428,4.811)--(6.428,4.814)--(6.434,4.815)--(6.434,4.816)--(6.434,4.817)%
  --(6.434,4.815)--(6.441,4.814)--(6.441,4.820)--(6.441,4.821)--(6.441,4.823)--(6.448,4.825)%
  --(6.448,4.827)--(6.448,4.828)--(6.454,4.830)--(6.454,4.831)--(6.461,4.835)--(6.461,4.839)%
  --(6.467,4.841)--(6.467,4.844)--(6.474,4.847)--(6.474,4.848)--(6.474,4.850)--(6.481,4.851)%
  --(6.487,4.855)--(6.494,4.855)--(6.494,4.856)--(6.494,4.859)--(6.501,4.863)--(6.501,4.867)%
  --(6.507,4.870)--(6.507,4.874)--(6.514,4.876)--(6.514,4.880)--(6.514,4.882)--(6.527,4.883)%
  --(6.534,4.892)--(6.534,4.896)--(6.541,4.898)--(6.541,4.902)--(6.547,4.907)--(6.554,4.911)%
  --(6.554,4.912)--(6.560,4.915)--(6.560,4.917)--(6.567,4.922)--(6.574,4.924)--(6.580,4.928)%
  --(6.580,4.931)--(6.587,4.933)--(6.587,4.935)--(6.587,4.938)--(6.594,4.939)--(6.594,4.940)%
  --(6.600,4.943)--(6.600,4.948)--(6.600,4.951)--(6.607,4.953)--(6.607,4.954)--(6.614,4.956)%
  --(6.614,4.957)--(6.614,4.959)--(6.620,4.956)--(6.620,4.966)--(6.627,4.967)--(6.634,4.970)%
  --(6.634,4.973)--(6.640,4.977)--(6.647,4.981)--(6.653,4.985)--(6.660,4.989)--(6.673,4.990)%
  --(6.680,4.996)--(6.680,4.998)--(6.680,4.999)--(6.680,5.000)--(6.680,5.003)--(6.687,5.005)%
  --(6.693,5.008)--(6.693,5.007)--(6.700,5.012)--(6.700,5.015)--(6.700,5.017)--(6.700,5.018)%
  --(6.700,5.019)--(6.707,5.021)--(6.707,5.022)--(6.713,5.023)--(6.707,5.024)--(6.707,5.026)%
  --(6.713,5.027)--(6.720,5.029)--(6.720,5.030)--(6.720,5.027)--(6.720,5.031)--(6.720,5.032)%
  --(6.720,5.034)--(6.726,5.040)--(6.733,5.044)--(6.740,5.046)--(6.746,5.050)--(6.753,5.048)%
  --(6.753,5.056)--(6.760,5.059)--(6.766,5.063)--(6.773,5.067)--(6.773,5.070)--(6.780,5.075)%
  --(6.780,5.078)--(6.780,5.079)--(6.793,5.082)--(6.793,5.085)--(6.793,5.087)--(6.800,5.092)%
  --(6.800,5.094)--(6.806,5.096)--(6.813,5.101)--(6.813,5.103)--(6.813,5.102)--(6.813,5.106)%
  --(6.813,5.107)--(6.819,5.109)--(6.819,5.110)--(6.819,5.111)--(6.819,5.112)--(6.826,5.113)%
  --(6.819,5.111)--(6.826,5.110)--(6.826,5.111)--(6.833,5.116)--(6.833,5.117)--(6.833,5.119)%
  --(6.839,5.122)--(6.839,5.124)--(6.846,5.121)--(6.846,5.123)--(6.846,5.128)--(6.846,5.129)%
  --(6.846,5.130)--(6.846,5.131)--(6.853,5.132)--(6.846,5.134)--(6.853,5.138)--(6.859,5.137)%
  --(6.866,5.138)--(6.866,5.145)--(6.873,5.152)--(6.879,5.155)--(6.886,5.157)--(6.886,5.160)%
  --(6.893,5.162)--(6.893,5.164)--(6.893,5.165)--(6.893,5.166)--(6.899,5.166)--(6.893,5.162)%
  --(6.899,5.164)--(6.899,5.167)--(6.906,5.170)--(6.912,5.172)--(6.919,5.175)--(6.919,5.177)%
  --(6.926,5.177)--(6.926,5.175)--(6.926,5.177)--(6.932,5.183)--(6.932,5.184)--(6.939,5.187)%
  --(6.939,5.189)--(6.939,5.192)--(6.939,5.194)--(6.946,5.195)--(6.946,5.193)--(6.946,5.194)%
  --(6.952,5.195)--(6.952,5.198)--(6.959,5.201)--(6.959,5.206)--(6.966,5.208)--(6.966,5.211)%
  --(6.972,5.216)--(6.972,5.217)--(6.979,5.218)--(6.979,5.221)--(6.986,5.226)--(6.986,5.227)%
  --(6.992,5.230)--(6.992,5.231)--(6.992,5.233)--(6.999,5.236)--(6.999,5.234)--(6.999,5.236)%
  --(6.999,5.244)--(7.005,5.246)--(7.005,5.248)--(7.012,5.249)--(7.019,5.250)--(7.019,5.252)%
  --(7.019,5.249)--(7.019,5.258)--(7.025,5.257)--(7.019,5.259)--(7.025,5.259)--(7.025,5.260)%
  --(7.025,5.261)--(7.025,5.262)--(7.025,5.263)--(7.039,5.268)--(7.039,5.269)--(7.039,5.268)%
  --(7.039,5.269)--(7.045,5.273)--(7.045,5.274)--(7.045,5.276)--(7.052,5.274)--(7.052,5.276)%
  --(7.052,5.282)--(7.052,5.283)--(7.052,5.282)--(7.052,5.284)--(7.052,5.285)--(7.052,5.282)%
  --(7.059,5.282)--(7.059,5.283)--(7.059,5.285)--(7.065,5.289)--(7.065,5.291)--(7.065,5.293)%
  --(7.072,5.293)--(7.072,5.296)--(7.079,5.301)--(7.079,5.304)--(7.085,5.306)--(7.085,5.308)%
  --(7.092,5.309)--(7.092,5.310)--(7.092,5.311)--(7.098,5.311)--(7.098,5.312)--(7.098,5.314)%
  --(7.105,5.317)--(7.105,5.324)--(7.112,5.328)--(7.112,5.330)--(7.118,5.333)--(7.118,5.335)%
  --(7.132,5.336)--(7.138,5.342)--(7.145,5.343)--(7.145,5.344)--(7.145,5.347)--(7.152,5.348)%
  --(7.152,5.351)--(7.158,5.353)--(7.158,5.356)--(7.165,5.352)--(7.165,5.359)--(7.171,5.360)%
  --(7.171,5.361)--(7.171,5.362)--(7.178,5.364)--(7.178,5.363)--(7.178,5.369)--(7.185,5.370)%
  --(7.185,5.372)--(7.191,5.375)--(7.198,5.378)--(7.198,5.380)--(7.205,5.383)--(7.205,5.385)%
  --(7.205,5.387)--(7.211,5.381)--(7.211,5.388)--(7.211,5.389)--(7.211,5.391)--(7.211,5.392)%
  --(7.211,5.393)--(7.211,5.396)--(7.218,5.397)--(7.218,5.396)--(7.218,5.398)--(7.218,5.399)%
  --(7.218,5.400)--(7.225,5.402)--(7.231,5.405)--(7.238,5.410)--(7.238,5.412)--(7.238,5.414)%
  --(7.238,5.412)--(7.245,5.416)--(7.251,5.419)--(7.258,5.421)--(7.258,5.426)--(7.264,5.429)%
  --(7.264,5.430)--(7.264,5.433)--(7.264,5.435)--(7.271,5.434)--(7.278,5.438)--(7.278,5.443)%
  --(7.284,5.446)--(7.284,5.449)--(7.284,5.451)--(7.291,5.453)--(7.291,5.454)--(7.298,5.457)%
  --(7.298,5.459)--(7.298,5.455)--(7.298,5.463)--(7.304,5.464)--(7.311,5.467)--(7.311,5.470)%
  --(7.311,5.472)--(7.318,5.475)--(7.318,5.474)--(7.324,5.480)--(7.331,5.482)--(7.331,5.485)%
  --(7.331,5.486)--(7.338,5.487)--(7.338,5.490)--(7.338,5.491)--(7.344,5.495)--(7.344,5.497)%
  --(7.351,5.500)--(7.351,5.499)--(7.357,5.501)--(7.364,5.507)--(7.364,5.509)--(7.364,5.510)%
  --(7.371,5.511)--(7.371,5.513)--(7.377,5.513)--(7.377,5.512)--(7.384,5.514)--(7.391,5.514)%
  --(7.391,5.516)--(7.391,5.523)--(7.397,5.528)--(7.404,5.531)--(7.411,5.532)--(7.417,5.535)%
  --(7.417,5.532)--(7.424,5.537)--(7.431,5.541)--(7.431,5.545)--(7.444,5.555)--(7.450,5.560)%
  --(7.450,5.562)--(7.457,5.564)--(7.457,5.566)--(7.464,5.567)--(7.464,5.565)--(7.464,5.567)%
  --(7.464,5.572)--(7.470,5.574)--(7.477,5.579)--(7.484,5.583)--(7.484,5.585)--(7.484,5.588)%
  --(7.484,5.589)--(7.484,5.587)--(7.497,5.589)--(7.497,5.592)--(7.497,5.593)--(7.497,5.597)%
  --(7.510,5.602)--(7.510,5.605)--(7.510,5.609)--(7.510,5.610)--(7.517,5.613)--(7.517,5.615)%
  --(7.517,5.616)--(7.524,5.618)--(7.530,5.615)--(7.530,5.622)--(7.530,5.624)--(7.537,5.626)%
  --(7.537,5.629)--(7.537,5.631)--(7.543,5.631)--(7.543,5.633)--(7.543,5.634)--(7.543,5.637)%
  --(7.550,5.644)--(7.550,5.645)--(7.557,5.648)--(7.563,5.651)--(7.570,5.648)--(7.570,5.650)%
  --(7.570,5.652)--(7.570,5.653)--(7.577,5.658)--(7.577,5.662)--(7.577,5.664)--(7.583,5.667)%
  --(7.583,5.668)--(7.583,5.669)--(7.590,5.667)--(7.590,5.669)--(7.590,5.672)--(7.597,5.677)%
  --(7.597,5.679)--(7.603,5.681)--(7.603,5.684)--(7.610,5.686)--(7.616,5.688)--(7.623,5.685)%
  --(7.623,5.686)--(7.623,5.687)--(7.623,5.691)--(7.630,5.693)--(7.630,5.694)--(7.630,5.695)%
  --(7.636,5.697)--(7.643,5.699)--(7.650,5.702)--(7.650,5.704)--(7.656,5.703)--(7.656,5.705)%
  --(7.663,5.712)--(7.663,5.714)--(7.670,5.718)--(7.676,5.721)--(7.676,5.722)--(7.676,5.720)%
  --(7.683,5.723)--(7.690,5.727)--(7.690,5.734)--(7.696,5.736)--(7.696,5.737)--(7.696,5.740)%
  --(7.709,5.743)--(7.709,5.746)--(7.716,5.749)--(7.716,5.747)--(7.729,5.759)--(7.736,5.761)%
  --(7.729,5.763)--(7.736,5.765)--(7.743,5.769)--(7.743,5.771)--(7.743,5.776)--(7.749,5.775)%
  --(7.756,5.787)--(7.763,5.791)--(7.769,5.794)--(7.769,5.796)--(7.776,5.797)--(7.776,5.799)%
  --(7.783,5.802)--(7.783,5.804)--(7.783,5.805)--(7.783,5.802)--(7.789,5.810)--(7.789,5.814)%
  --(7.802,5.817)--(7.802,5.820)--(7.809,5.825)--(7.816,5.828)--(7.816,5.833)--(7.822,5.836)%
  --(7.829,5.838)--(7.822,5.839)--(7.822,5.841)--(7.836,5.848)--(7.836,5.851)--(7.842,5.852)%
  --(7.842,5.854)--(7.849,5.857)--(7.849,5.858)--(7.856,5.853)--(7.856,5.861)--(7.869,5.863)%
  --(7.876,5.865)--(7.876,5.867)--(7.876,5.868)--(7.882,5.870)--(7.882,5.871)--(7.895,5.874)%
  --(7.902,5.879)--(7.902,5.884)--(7.902,5.887)--(7.902,5.890)--(7.909,5.893)--(7.915,5.894)%
  --(7.922,5.895)--(7.915,5.895)--(7.915,5.896)--(7.915,5.890)--(7.922,5.898)--(7.922,5.899)%
  --(7.922,5.901)--(7.929,5.905)--(7.935,5.909)--(7.935,5.912)--(7.942,5.913)--(7.942,5.912)%
  --(7.955,5.914)--(7.955,5.920)--(7.955,5.922)--(7.962,5.927)--(7.969,5.930)--(7.969,5.935)%
  --(7.975,5.938)--(7.975,5.940)--(7.982,5.941)--(7.988,5.943)--(7.988,5.945)--(7.988,5.943)%
  --(7.995,5.951)--(8.002,5.957)--(8.008,5.962)--(8.008,5.968)--(8.015,5.971)--(8.015,5.973)%
  --(8.015,5.972)--(8.022,5.971)--(8.022,5.977)--(8.022,5.978)--(8.028,5.981)--(8.028,5.982)%
  --(8.028,5.985)--(8.035,5.987)--(8.035,5.988)--(8.035,5.991)--(8.035,5.987)--(8.035,5.988)%
  --(8.042,5.990)--(8.048,5.994)--(8.042,6.000)--(8.055,6.001)--(8.055,6.006)--(8.068,6.017)%
  --(8.075,6.023)--(8.081,6.024)--(8.088,6.028)--(8.088,6.031)--(8.095,6.033)--(8.095,6.034)%
  --(8.095,6.037)--(8.101,6.039)--(8.101,6.043)--(8.108,6.044)--(8.115,6.040)--(8.115,6.041)%
  --(8.121,6.042)--(8.115,6.042)--(8.121,6.049)--(8.121,6.052)--(8.128,6.054)--(8.135,6.056)%
  --(8.128,6.056)--(8.135,6.058)--(8.148,6.061)--(8.154,6.066)--(8.161,6.068)--(8.168,6.071)%
  --(8.174,6.079)--(8.174,6.082)--(8.181,6.084)--(8.188,6.088)--(8.188,6.094)--(8.194,6.092)%
  --(8.194,6.094)--(8.194,6.097)--(8.201,6.098)--(8.201,6.104)--(8.201,6.106)--(8.208,6.108)%
  --(8.208,6.110)--(8.214,6.112)--(8.221,6.113)--(8.221,6.112)--(8.221,6.121)--(8.228,6.123)%
  --(8.234,6.126)--(8.234,6.131)--(8.241,6.134)--(8.247,6.137)--(8.254,6.143)--(8.254,6.148)%
  --(8.261,6.147)--(8.261,6.149)--(8.267,6.151)--(8.274,6.157)--(8.281,6.167)--(8.287,6.170)%
  --(8.287,6.176)--(8.294,6.183)--(8.307,6.191)--(8.321,6.196)--(8.327,6.198)--(8.334,6.204)%
  --(8.340,6.206)--(8.340,6.213)--(8.347,6.214)--(8.340,6.214)--(8.347,6.215)--(8.354,6.216)%
  --(8.354,6.218)--(8.354,6.213)--(8.354,6.214)--(8.360,6.215)--(8.367,6.224)--(8.360,6.224)%
  --(8.367,6.227)--(8.374,6.230)--(8.380,6.227)--(8.380,6.229)--(8.387,6.232)--(8.394,6.239)%
  --(8.394,6.246)--(8.400,6.248)--(8.400,6.251)--(8.400,6.253)--(8.407,6.255)--(8.407,6.256)%
  --(8.420,6.254)--(8.427,6.267)--(8.427,6.269)--(8.427,6.270)--(8.433,6.273)--(8.440,6.276)%
  --(8.440,6.279)--(8.440,6.283)--(8.447,6.284)--(8.460,6.282)--(8.460,6.292)--(8.467,6.294)%
  --(8.467,6.297)--(8.467,6.299)--(8.467,6.301)--(8.473,6.302)--(8.467,6.305)--(8.480,6.307)%
  --(8.473,6.306)--(8.487,6.313)--(8.487,6.318)--(8.493,6.321)--(8.493,6.322)--(8.500,6.324)%
  --(8.500,6.330)--(8.513,6.338)--(8.507,6.340)--(8.520,6.342)--(8.520,6.343)--(8.513,6.344)%
  --(8.520,6.343)--(8.533,6.351)--(8.526,6.356)--(8.533,6.358)--(8.533,6.360)--(8.533,6.362)%
  --(8.540,6.363)--(8.546,6.364)--(8.546,6.367)--(8.546,6.372)--(8.553,6.373)--(8.553,6.375)%
  --(8.560,6.376)--(8.566,6.378)--(8.573,6.381)--(8.573,6.384)--(8.580,6.387)--(8.580,6.381)%
  --(8.580,6.391)--(8.586,6.393)--(8.586,6.394)--(8.593,6.395)--(8.599,6.397)--(8.606,6.400)%
  --(8.606,6.404)--(8.619,6.409)--(8.619,6.408)--(8.626,6.416)--(8.639,6.421)--(8.646,6.426)%
  --(8.653,6.431)--(8.653,6.435)--(8.659,6.438)--(8.666,6.441)--(8.666,6.444)--(8.673,6.446)%
  --(8.673,6.442)--(8.673,6.451)--(8.679,6.454)--(8.686,6.455)--(8.679,6.458)--(8.686,6.460)%
  --(8.686,6.463)--(8.699,6.466)--(8.699,6.470)--(8.706,6.474)--(8.712,6.479)--(8.712,6.481)%
  --(8.712,6.478)--(8.719,6.485)--(8.719,6.487)--(8.719,6.488)--(8.726,6.492)--(8.726,6.493)%
  --(8.726,6.488)--(8.726,6.497)--(8.726,6.499)--(8.726,6.501)--(8.732,6.503)--(8.732,6.504)%
  --(8.739,6.506)--(8.746,6.508)--(8.752,6.510)--(8.746,6.509)--(8.752,6.505)--(8.746,6.511)%
  --(8.752,6.514)--(8.752,6.516)--(8.759,6.517)--(8.752,6.519)--(8.759,6.520)--(8.759,6.517)%
  --(8.759,6.516)--(8.759,6.522)--(8.759,6.525)--(8.759,6.528)--(8.766,6.530)--(8.772,6.534)%
  --(8.772,6.540)--(8.779,6.544)--(8.785,6.547)--(8.785,6.542)--(8.785,6.543)--(8.792,6.548)%
  --(8.792,6.553)--(8.792,6.554)--(8.799,6.555)--(8.799,6.558)--(8.805,6.560)--(8.812,6.562)%
  --(8.812,6.563)--(8.812,6.565)--(8.819,6.566)--(8.819,6.565)--(8.819,6.562)--(8.825,6.569)%
  --(8.825,6.571)--(8.832,6.571)--(8.832,6.572)--(8.832,6.573)--(8.832,6.569)--(8.839,6.570)%
  --(8.839,6.572)--(8.845,6.579)--(8.845,6.584)--(8.845,6.585)--(8.859,6.588)--(8.859,6.591)%
  --(8.865,6.593)--(8.865,6.590)--(8.865,6.596)--(8.872,6.601)--(8.878,6.603)--(8.878,6.604)%
  --(8.878,6.609)--(8.892,6.612)--(8.885,6.613)--(8.892,6.615)--(8.885,6.609)--(8.892,6.611)%
  --(8.898,6.613)--(8.898,6.621)--(8.912,6.629)--(8.918,6.633)--(8.918,6.637)--(8.925,6.641)%
  --(8.938,6.643)--(8.938,6.640)--(8.938,6.641)--(8.938,6.647)--(8.945,6.653)--(8.945,6.655)%
  --(8.952,6.657)--(8.952,6.661)--(8.958,6.665)--(8.958,6.668)--(8.971,6.672)--(8.965,6.675)%
  --(8.971,6.679)--(8.978,6.681)--(8.985,6.689)--(8.991,6.691)--(8.985,6.691)--(8.985,6.693)%
  --(8.991,6.691)--(8.991,6.692)--(8.991,6.694)--(8.991,6.696)--(8.998,6.704)--(8.998,6.705)%
  --(8.998,6.706)--(8.998,6.704)--(8.998,6.701)--(8.998,6.704)--(9.005,6.706)--(9.005,6.709)%
  --(9.018,6.715)--(9.018,6.718)--(9.018,6.721)--(9.025,6.723)--(9.025,6.724)--(9.025,6.721)%
  --(9.031,6.731)--(9.031,6.733)--(9.031,6.735)--(9.038,6.738)--(9.051,6.741)--(9.051,6.743)%
  --(9.058,6.744)--(9.058,6.737)--(9.051,6.745)--(9.058,6.746)--(9.064,6.747)--(9.064,6.749)%
  --(9.064,6.751)--(9.071,6.754)--(9.078,6.757)--(9.078,6.759)--(9.084,6.762)--(9.078,6.762)%
  --(9.084,6.761)--(9.091,6.764)--(9.084,6.766)--(9.091,6.768)--(9.098,6.769)--(9.098,6.766)%
  --(9.098,6.771)--(9.098,6.775)--(9.104,6.775)--(9.104,6.776)--(9.104,6.777)--(9.104,6.778)%
  --(9.118,6.780)--(9.118,6.781)--(9.118,6.783)--(9.124,6.779)--(9.131,6.792)--(9.137,6.799)%
  --(9.144,6.804)--(9.151,6.808)--(9.151,6.810)--(9.151,6.812)--(9.151,6.813)--(9.157,6.816)%
  --(9.157,6.810)--(9.164,6.819)--(9.171,6.822)--(9.164,6.823)--(9.171,6.825)--(9.177,6.827)%
  --(9.177,6.831)--(9.184,6.836)--(9.191,6.838)--(9.191,6.840)--(9.191,6.837)--(9.191,6.847)%
  --(9.197,6.849)--(9.197,6.850)--(9.197,6.851)--(9.197,6.853)--(9.204,6.856)--(9.211,6.860)%
  --(9.204,6.861)--(9.217,6.864)--(9.224,6.865)--(9.217,6.867)--(9.224,6.869)--(9.217,6.870)%
  --(9.224,6.872)--(9.230,6.873)--(9.230,6.877)--(9.230,6.872)--(9.237,6.882)--(9.230,6.883)%
  --(9.237,6.887)--(9.237,6.891)--(9.244,6.894)--(9.257,6.900)--(9.270,6.908)--(9.277,6.914)%
  --(9.277,6.911)--(9.284,6.918)--(9.290,6.919)--(9.297,6.919)--(9.304,6.919)--(9.304,6.920)%
  --(9.297,6.920)--(9.304,6.920)--(9.304,6.915)--(9.304,6.923)--(9.304,6.926)--(9.310,6.929)%
  --(9.310,6.930)--(9.317,6.933)--(9.323,6.936)--(9.323,6.939)--(9.330,6.942)--(9.337,6.941)%
  --(9.337,6.939)--(9.343,6.942)--(9.343,6.945)--(9.343,6.953)--(9.350,6.954)--(9.350,6.956)%
  --(9.357,6.959)--(9.357,6.960)--(9.357,6.962)--(9.357,6.965)--(9.363,6.967)--(9.363,6.970)%
  --(9.370,6.975)--(9.377,6.978)--(9.383,6.979)--(9.383,6.981)--(9.383,6.983)--(9.390,6.986)%
  --(9.397,6.987)--(9.403,6.986)--(9.403,6.988)--(9.410,6.991)--(9.410,6.994)--(9.410,7.005)%
  --(9.423,7.008)--(9.423,7.010)--(9.430,7.014)--(9.430,7.019)--(9.430,7.014)--(9.436,7.017)%
  --(9.436,7.018)--(9.436,7.019)--(9.436,7.023)--(9.443,7.021)--(9.443,7.032)--(9.450,7.038)%
  --(9.443,7.040)--(9.450,7.041)--(9.463,7.038)--(9.463,7.041)--(9.470,7.046)--(9.470,7.050)%
  --(9.476,7.058)--(9.483,7.062)--(9.476,7.064)--(9.483,7.068)--(9.490,7.068)--(9.490,7.066)%
  --(9.490,7.067)--(9.490,7.069)--(9.496,7.072)--(9.496,7.075)--(9.509,7.082)--(9.503,7.085)%
  --(9.509,7.087)--(9.516,7.090)--(9.516,7.092)--(9.523,7.096)--(9.523,7.097)--(9.529,7.099)%
  --(9.529,7.100)--(9.536,7.104)--(9.543,7.105)--(9.549,7.107)--(9.549,7.109)--(9.556,7.111)%
  --(9.556,7.108)--(9.563,7.112)--(9.563,7.116)--(9.569,7.119)--(9.576,7.129)--(9.582,7.133)%
  --(9.582,7.135)--(9.589,7.137)--(9.596,7.139)--(9.602,7.142)--(9.602,7.144)--(9.609,7.148)%
  --(9.609,7.151)--(9.609,7.153)--(9.609,7.154)--(9.616,7.154)--(9.616,7.156)--(9.616,7.157)%
  --(9.622,7.154)--(9.622,7.155)--(9.629,7.159)--(9.629,7.162)--(9.629,7.170)--(9.642,7.177)%
  --(9.649,7.179)--(9.649,7.180)--(9.649,7.182)--(9.656,7.181)--(9.656,7.180)--(9.662,7.182)%
  --(9.662,7.183)--(9.669,7.193)--(9.675,7.194)--(9.669,7.197)--(9.675,7.198)--(9.675,7.199)%
  --(9.682,7.201)--(9.682,7.203)--(9.682,7.206)--(9.682,7.209)--(9.675,7.212)--(9.689,7.214)%
  --(9.682,7.215)--(9.689,7.217)--(9.689,7.220)--(9.695,7.222)--(9.702,7.220)--(9.702,7.222)%
  --(9.702,7.232)--(9.709,7.235)--(9.709,7.236)--(9.709,7.238)--(9.715,7.241)--(9.715,7.242)%
  --(9.715,7.244)--(9.715,7.238)--(9.722,7.238)--(9.715,7.245)--(9.715,7.247)--(9.722,7.250)%
  --(9.735,7.253)--(9.735,7.257)--(9.749,7.262)--(9.755,7.269)--(9.768,7.268)--(9.782,7.279)%
  --(9.775,7.281)--(9.782,7.281)--(9.782,7.282)--(9.788,7.281)--(9.795,7.282)--(9.795,7.285)%
  --(9.795,7.287)--(9.802,7.280)--(9.802,7.293)--(9.808,7.298)--(9.808,7.301)--(9.822,7.304)%
  --(9.815,7.306)--(9.822,7.310)--(9.828,7.314)--(9.835,7.315)--(9.835,7.317)--(9.835,7.319)%
  --(9.835,7.321)--(9.835,7.324)--(9.842,7.327)--(9.842,7.331)--(9.848,7.333)--(9.848,7.336)%
  --(9.855,7.337)--(9.861,7.340)--(9.875,7.336)--(9.881,7.349)--(9.881,7.351)--(9.881,7.354)%
  --(9.888,7.358)--(9.888,7.360)--(9.895,7.362)--(9.895,7.364)--(9.895,7.361)--(9.895,7.365)%
  --(9.895,7.368)--(9.888,7.369)--(9.895,7.371)--(9.895,7.374)--(9.915,7.379)--(9.915,7.383)%
  --(9.915,7.386)--(9.915,7.387)--(9.921,7.381)--(9.915,7.391)--(9.928,7.394)--(9.928,7.398)%
  --(9.935,7.402)--(9.935,7.404)--(9.935,7.406)--(9.941,7.407)--(9.935,7.408)--(9.941,7.405)%
  --(9.941,7.411)--(9.941,7.415)--(9.954,7.417)--(9.954,7.420)--(9.961,7.422)--(9.961,7.426)%
  --(9.961,7.432)--(9.968,7.434)--(9.961,7.435)--(9.974,7.438)--(9.974,7.440)--(9.981,7.443)%
  --(9.981,7.446)--(9.988,7.449)--(10.008,7.451)--(10.008,7.453)--(10.008,7.454)--(10.014,7.455)%
  --(10.021,7.451)--(10.014,7.447)--(10.021,7.446)--(10.027,7.453)--(10.027,7.457)--(10.034,7.458)%
  --(10.034,7.459)--(10.041,7.461)--(10.041,7.462)--(10.047,7.453)--(10.054,7.465);
\gpcolor{color=gp lt color 4}
\gpsetlinetype{gp lt plot 4}
\draw[gp path] (2.190,1.734)--(2.190,1.737)--(2.197,1.740)--(2.197,1.743)--(2.197,1.744)%
  --(2.203,1.747)--(2.203,1.750)--(2.210,1.755)--(2.217,1.758)--(2.217,1.761)--(2.223,1.762)%
  --(2.230,1.767)--(2.230,1.772)--(2.237,1.775)--(2.243,1.779)--(2.250,1.781)--(2.250,1.785)%
  --(2.257,1.787)--(2.257,1.789)--(2.257,1.790)--(2.263,1.794)--(2.270,1.798)--(2.276,1.802)%
  --(2.276,1.805)--(2.283,1.807)--(2.283,1.808)--(2.290,1.813)--(2.303,1.823)--(2.316,1.829)%
  --(2.323,1.835)--(2.336,1.843)--(2.343,1.851)--(2.349,1.856)--(2.349,1.859)--(2.363,1.866)%
  --(2.363,1.870)--(2.369,1.873)--(2.369,1.874)--(2.369,1.875)--(2.383,1.881)--(2.389,1.884)%
  --(2.389,1.886)--(2.396,1.889)--(2.396,1.891)--(2.403,1.894)--(2.409,1.901)--(2.416,1.906)%
  --(2.423,1.914)--(2.436,1.921)--(2.442,1.925)--(2.442,1.928)--(2.449,1.930)--(2.456,1.933)%
  --(2.456,1.937)--(2.469,1.944)--(2.476,1.948)--(2.476,1.951)--(2.476,1.952)--(2.482,1.956)%
  --(2.482,1.960)--(2.489,1.963)--(2.489,1.966)--(2.496,1.970)--(2.509,1.976)--(2.516,1.981)%
  --(2.516,1.983)--(2.516,1.986)--(2.522,1.987)--(2.522,1.989)--(2.522,1.991)--(2.529,1.994)%
  --(2.535,1.999)--(2.542,2.003)--(2.549,2.006)--(2.549,2.008)--(2.555,2.014)--(2.562,2.018)%
  --(2.569,2.024)--(2.575,2.028)--(2.575,2.030)--(2.582,2.033)--(2.589,2.038)--(2.589,2.041)%
  --(2.589,2.042)--(2.595,2.046)--(2.609,2.053)--(2.615,2.058)--(2.622,2.065)--(2.628,2.073)%
  --(2.635,2.077)--(2.642,2.083)--(2.655,2.089)--(2.662,2.092)--(2.668,2.098)--(2.675,2.101)%
  --(2.675,2.104)--(2.675,2.109)--(2.682,2.112)--(2.688,2.116)--(2.695,2.123)--(2.702,2.127)%
  --(2.702,2.130)--(2.708,2.132)--(2.708,2.133)--(2.715,2.136)--(2.721,2.138)--(2.721,2.142)%
  --(2.728,2.145)--(2.728,2.147)--(2.735,2.151)--(2.741,2.155)--(2.748,2.159)--(2.755,2.165)%
  --(2.761,2.168)--(2.768,2.174)--(2.775,2.179)--(2.781,2.182)--(2.781,2.184)--(2.788,2.188)%
  --(2.788,2.192)--(2.801,2.196)--(2.801,2.201)--(2.814,2.207)--(2.828,2.213)--(2.834,2.218)%
  --(2.841,2.226)--(2.854,2.236)--(2.861,2.242)--(2.868,2.247)--(2.874,2.252)--(2.881,2.256)%
  --(2.887,2.259)--(2.887,2.262)--(2.894,2.263)--(2.887,2.263)--(2.887,2.264)--(2.894,2.270)%
  --(2.901,2.275)--(2.907,2.278)--(2.907,2.280)--(2.907,2.281)--(2.907,2.284)--(2.914,2.286)%
  --(2.914,2.290)--(2.921,2.294)--(2.927,2.298)--(2.934,2.300)--(2.941,2.304)--(2.947,2.309)%
  --(2.954,2.316)--(2.961,2.320)--(2.967,2.326)--(2.980,2.331)--(2.980,2.332)--(2.980,2.336)%
  --(2.994,2.341)--(2.994,2.345)--(2.994,2.346)--(3.000,2.347)--(3.007,2.351)--(3.014,2.359)%
  --(3.020,2.363)--(3.027,2.368)--(3.034,2.373)--(3.040,2.377)--(3.040,2.378)--(3.047,2.382)%
  --(3.054,2.386)--(3.060,2.392)--(3.073,2.399)--(3.087,2.408)--(3.100,2.416)--(3.107,2.421)%
  --(3.107,2.424)--(3.113,2.429)--(3.120,2.435)--(3.127,2.439)--(3.133,2.443)--(3.133,2.446)%
  --(3.140,2.452)--(3.147,2.457)--(3.147,2.460)--(3.147,2.462)--(3.153,2.466)--(3.160,2.470)%
  --(3.166,2.475)--(3.173,2.477)--(3.180,2.481)--(3.180,2.485)--(3.186,2.488)--(3.186,2.489)%
  --(3.193,2.491)--(3.193,2.492)--(3.193,2.493)--(3.200,2.497)--(3.213,2.505)--(3.213,2.508)%
  --(3.220,2.511)--(3.226,2.515)--(3.233,2.521)--(3.240,2.526)--(3.246,2.529)--(3.253,2.533)%
  --(3.259,2.536)--(3.259,2.538)--(3.266,2.541)--(3.266,2.546)--(3.273,2.550)--(3.273,2.553)%
  --(3.279,2.554)--(3.286,2.558)--(3.286,2.561)--(3.293,2.564)--(3.299,2.571)--(3.306,2.575)%
  --(3.313,2.579)--(3.319,2.586)--(3.326,2.594)--(3.332,2.598)--(3.326,2.599)--(3.332,2.600)%
  --(3.332,2.601)--(3.332,2.603)--(3.339,2.606)--(3.346,2.610)--(3.352,2.617)--(3.366,2.624)%
  --(3.372,2.627)--(3.379,2.632)--(3.386,2.635)--(3.386,2.636)--(3.392,2.638)--(3.399,2.642)%
  --(3.406,2.646)--(3.406,2.650)--(3.412,2.653)--(3.419,2.657)--(3.425,2.663)--(3.432,2.670)%
  --(3.439,2.675)--(3.439,2.677)--(3.445,2.679)--(3.445,2.680)--(3.452,2.684)--(3.452,2.685)%
  --(3.459,2.688)--(3.459,2.691)--(3.472,2.695)--(3.472,2.698)--(3.479,2.705)--(3.485,2.710)%
  --(3.492,2.714)--(3.499,2.717)--(3.505,2.722)--(3.512,2.727)--(3.518,2.732)--(3.525,2.736)%
  --(3.525,2.740)--(3.538,2.746)--(3.545,2.751)--(3.545,2.754)--(3.545,2.756)--(3.545,2.757)%
  --(3.552,2.760)--(3.552,2.765)--(3.558,2.770)--(3.565,2.773)--(3.565,2.774)--(3.565,2.775)%
  --(3.572,2.778)--(3.578,2.782)--(3.578,2.783)--(3.578,2.784)--(3.578,2.787)--(3.585,2.793)%
  --(3.598,2.800)--(3.611,2.806)--(3.618,2.810)--(3.618,2.814)--(3.625,2.815)--(3.625,2.820)%
  --(3.631,2.824)--(3.645,2.831)--(3.651,2.838)--(3.665,2.842)--(3.671,2.847)--(3.678,2.855)%
  --(3.685,2.861)--(3.691,2.865)--(3.691,2.868)--(3.698,2.870)--(3.704,2.870)--(3.711,2.875)%
  --(3.718,2.882)--(3.724,2.888)--(3.738,2.896)--(3.758,2.908)--(3.771,2.919)--(3.777,2.927)%
  --(3.784,2.935)--(3.791,2.939)--(3.804,2.944)--(3.811,2.951)--(3.817,2.956)--(3.824,2.960)%
  --(3.824,2.965)--(3.831,2.972)--(3.837,2.977)--(3.851,2.982)--(3.857,2.986)--(3.857,2.989)%
  --(3.857,2.992)--(3.870,2.997)--(3.877,3.005)--(3.890,3.010)--(3.897,3.015)--(3.904,3.023)%
  --(3.910,3.030)--(3.917,3.033)--(3.924,3.039)--(3.924,3.042)--(3.930,3.045)--(3.950,3.054)%
  --(3.963,3.062)--(3.970,3.069)--(3.970,3.072)--(3.970,3.074)--(3.977,3.074)--(3.983,3.085)%
  --(4.003,3.092)--(4.010,3.097)--(4.010,3.100)--(4.017,3.104)--(4.023,3.110)--(4.037,3.117)%
  --(4.043,3.124)--(4.050,3.130)--(4.063,3.133)--(4.070,3.142)--(4.076,3.148)--(4.076,3.153)%
  --(4.076,3.152)--(4.076,3.154)--(4.076,3.157)--(4.083,3.159)--(4.090,3.167)--(4.096,3.171)%
  --(4.103,3.175)--(4.110,3.176)--(4.110,3.180)--(4.110,3.181)--(4.116,3.183)--(4.116,3.189)%
  --(4.123,3.190)--(4.123,3.192)--(4.123,3.194)--(4.136,3.198)--(4.136,3.200)--(4.136,3.204)%
  --(4.143,3.208)--(4.156,3.215)--(4.169,3.222)--(4.183,3.229)--(4.196,3.237)--(4.203,3.243)%
  --(4.209,3.248)--(4.216,3.253)--(4.222,3.260)--(4.229,3.264)--(4.242,3.273)--(4.249,3.278)%
  --(4.256,3.281)--(4.256,3.282)--(4.262,3.288)--(4.276,3.296)--(4.289,3.305)--(4.296,3.311)%
  --(4.296,3.313)--(4.302,3.316)--(4.302,3.318)--(4.309,3.323)--(4.315,3.330)--(4.322,3.334)%
  --(4.322,3.336)--(4.329,3.340)--(4.329,3.344)--(4.335,3.347)--(4.342,3.351)--(4.342,3.357)%
  --(4.349,3.361)--(4.349,3.363)--(4.355,3.367)--(4.362,3.370)--(4.362,3.371)--(4.362,3.374)%
  --(4.369,3.376)--(4.369,3.377)--(4.369,3.376)--(4.369,3.379)--(4.362,3.379)--(4.362,3.378)%
  --(4.362,3.377)--(4.362,3.376)--(4.362,3.375)--(4.369,3.376)--(4.369,3.379)--(4.375,3.381)%
  --(4.382,3.387)--(4.389,3.392)--(4.402,3.395)--(4.408,3.403)--(4.422,3.407)--(4.428,3.411)%
  --(4.435,3.416)--(4.442,3.423)--(4.448,3.428)--(4.455,3.431)--(4.462,3.436)--(4.462,3.438)%
  --(4.475,3.445)--(4.482,3.452)--(4.488,3.456)--(4.495,3.460)--(4.508,3.469)--(4.515,3.477)%
  --(4.521,3.482)--(4.528,3.485)--(4.528,3.487)--(4.528,3.488)--(4.535,3.491)--(4.541,3.496)%
  --(4.541,3.499)--(4.548,3.504)--(4.548,3.506)--(4.548,3.510)--(4.555,3.512)--(4.561,3.515)%
  --(4.568,3.518)--(4.568,3.520)--(4.568,3.522)--(4.568,3.524)--(4.575,3.528)--(4.581,3.533)%
  --(4.581,3.535)--(4.581,3.537)--(4.588,3.539)--(4.588,3.542)--(4.588,3.541)--(4.594,3.542)%
  --(4.601,3.548)--(4.601,3.551)--(4.608,3.554)--(4.614,3.556)--(4.614,3.559)--(4.621,3.563)%
  --(4.634,3.568)--(4.648,3.574)--(4.654,3.580)--(4.661,3.589)--(4.674,3.599)--(4.681,3.605)%
  --(4.687,3.608)--(4.687,3.611)--(4.687,3.612)--(4.687,3.613)--(4.694,3.617)--(4.694,3.620)%
  --(4.707,3.625)--(4.714,3.632)--(4.721,3.636)--(4.727,3.638)--(4.734,3.643)--(4.747,3.650)%
  --(4.754,3.658)--(4.760,3.664)--(4.767,3.671)--(4.767,3.675)--(4.774,3.678)--(4.780,3.683)%
  --(4.787,3.686)--(4.794,3.687)--(4.800,3.692)--(4.807,3.700)--(4.807,3.704)--(4.814,3.707)%
  --(4.820,3.711)--(4.820,3.714)--(4.820,3.715)--(4.827,3.717)--(4.827,3.718)--(4.827,3.719)%
  --(4.827,3.720)--(4.834,3.721)--(4.834,3.723)--(4.840,3.728)--(4.840,3.731)--(4.847,3.735)%
  --(4.853,3.737)--(4.860,3.742)--(4.867,3.746)--(4.867,3.750)--(4.873,3.750)--(4.880,3.751)%
  --(4.887,3.761)--(4.893,3.765)--(4.907,3.773)--(4.913,3.778)--(4.913,3.783)--(4.920,3.784)%
  --(4.927,3.790)--(4.940,3.797)--(4.946,3.805)--(4.960,3.811)--(4.966,3.814)--(4.973,3.818)%
  --(4.980,3.824)--(4.980,3.829)--(4.986,3.839)--(4.986,3.840)--(5.000,3.843)--(5.000,3.847)%
  --(5.013,3.852)--(5.013,3.857)--(5.026,3.862)--(5.026,3.863)--(5.033,3.867)--(5.039,3.871)%
  --(5.046,3.873)--(5.046,3.881)--(5.053,3.885)--(5.053,3.887)--(5.059,3.889)--(5.059,3.891)%
  --(5.059,3.894)--(5.066,3.896)--(5.066,3.897)--(5.066,3.899)--(5.073,3.903)--(5.086,3.908)%
  --(5.093,3.913)--(5.106,3.920)--(5.113,3.925)--(5.119,3.929)--(5.126,3.932)--(5.132,3.935)%
  --(5.132,3.937)--(5.132,3.939)--(5.146,3.949)--(5.159,3.957)--(5.166,3.964)--(5.166,3.969)%
  --(5.166,3.970)--(5.172,3.972)--(5.179,3.975)--(5.179,3.978)--(5.186,3.982)--(5.192,3.989)%
  --(5.205,3.996)--(5.219,4.006)--(5.232,4.014)--(5.239,4.020)--(5.245,4.022)--(5.245,4.025)%
  --(5.252,4.028)--(5.259,4.032)--(5.265,4.036)--(5.272,4.039)--(5.272,4.046)--(5.279,4.050)%
  --(5.285,4.053)--(5.292,4.057)--(5.292,4.058)--(5.292,4.057)--(5.292,4.060)--(5.292,4.063)%
  --(5.298,4.068)--(5.312,4.077)--(5.318,4.082)--(5.325,4.086)--(5.325,4.088)--(5.325,4.090)%
  --(5.325,4.091)--(5.332,4.094)--(5.338,4.096)--(5.345,4.098)--(5.352,4.098)--(5.352,4.102)%
  --(5.358,4.107)--(5.365,4.113)--(5.372,4.116)--(5.372,4.121)--(5.378,4.126)--(5.391,4.131)%
  --(5.398,4.134)--(5.405,4.142)--(5.411,4.148)--(5.425,4.155)--(5.431,4.163)--(5.438,4.168)%
  --(5.445,4.174)--(5.458,4.180)--(5.465,4.189)--(5.471,4.192)--(5.471,4.195)--(5.471,4.199)%
  --(5.478,4.202)--(5.484,4.206)--(5.491,4.210)--(5.491,4.215)--(5.498,4.220)--(5.504,4.220)%
  --(5.504,4.224)--(5.511,4.224)--(5.511,4.227)--(5.511,4.231)--(5.518,4.234)--(5.518,4.237)%
  --(5.524,4.238)--(5.531,4.242)--(5.531,4.240)--(5.531,4.242)--(5.538,4.245)--(5.538,4.252)%
  --(5.544,4.255)--(5.544,4.257)--(5.544,4.261)--(5.551,4.262)--(5.558,4.265)--(5.558,4.271)%
  --(5.571,4.275)--(5.577,4.279)--(5.584,4.282)--(5.591,4.285)--(5.597,4.291)--(5.604,4.295)%
  --(5.611,4.299)--(5.617,4.304)--(5.624,4.312)--(5.631,4.316)--(5.644,4.324)--(5.651,4.331)%
  --(5.657,4.335)--(5.664,4.343)--(5.670,4.349)--(5.677,4.354)--(5.690,4.361)--(5.697,4.365)%
  --(5.704,4.370)--(5.704,4.372)--(5.710,4.375)--(5.717,4.378)--(5.717,4.382)--(5.724,4.387)%
  --(5.730,4.391)--(5.737,4.397)--(5.737,4.400)--(5.743,4.405)--(5.750,4.406)--(5.750,4.409)%
  --(5.757,4.413)--(5.757,4.415)--(5.763,4.416)--(5.763,4.420)--(5.770,4.423)--(5.770,4.424)%
  --(5.770,4.425)--(5.770,4.426)--(5.777,4.424)--(5.777,4.426)--(5.777,4.432)--(5.783,4.437)%
  --(5.790,4.443)--(5.797,4.447)--(5.803,4.449)--(5.810,4.452)--(5.817,4.457)--(5.823,4.461)%
  --(5.830,4.464)--(5.836,4.465)--(5.836,4.468)--(5.836,4.470)--(5.843,4.473)--(5.843,4.475)%
  --(5.850,4.480)--(5.856,4.484)--(5.856,4.486)--(5.856,4.487)--(5.863,4.489)--(5.863,4.487)%
  --(5.870,4.489)--(5.876,4.497)--(5.883,4.504)--(5.890,4.508)--(5.896,4.512)--(5.903,4.518)%
  --(5.910,4.522)--(5.916,4.527)--(5.923,4.530)--(5.929,4.536)--(5.936,4.544)--(5.943,4.550)%
  --(5.949,4.554)--(5.949,4.557)--(5.956,4.562)--(5.963,4.565)--(5.963,4.563)--(5.969,4.566)%
  --(5.969,4.568)--(5.969,4.570)--(5.976,4.576)--(5.983,4.581)--(5.989,4.586)--(5.996,4.591)%
  --(6.003,4.596)--(6.009,4.598)--(6.009,4.599)--(6.009,4.603)--(6.029,4.613)--(6.036,4.622)%
  --(6.049,4.631)--(6.062,4.634)--(6.069,4.637)--(6.082,4.646)--(6.089,4.655)--(6.096,4.661)%
  --(6.102,4.665)--(6.109,4.667)--(6.115,4.671)--(6.115,4.675)--(6.122,4.679)--(6.129,4.685)%
  --(6.135,4.691)--(6.142,4.696)--(6.149,4.701)--(6.149,4.704)--(6.155,4.707)--(6.162,4.712)%
  --(6.169,4.720)--(6.169,4.724)--(6.182,4.729)--(6.182,4.731)--(6.182,4.732)--(6.188,4.734)%
  --(6.188,4.732)--(6.188,4.738)--(6.195,4.739)--(6.195,4.743)--(6.202,4.746)--(6.202,4.748)%
  --(6.208,4.751)--(6.208,4.753)--(6.215,4.757)--(6.215,4.758)--(6.222,4.760)--(6.228,4.766)%
  --(6.228,4.767)--(6.235,4.769)--(6.235,4.771)--(6.235,4.772)--(6.235,4.774)--(6.235,4.777)%
  --(6.242,4.780)--(6.242,4.782)--(6.248,4.785)--(6.255,4.789)--(6.255,4.795)--(6.268,4.799)%
  --(6.275,4.802)--(6.281,4.805)--(6.295,4.809)--(6.301,4.815)--(6.308,4.820)--(6.315,4.826)%
  --(6.321,4.832)--(6.328,4.836)--(6.335,4.841)--(6.341,4.846)--(6.348,4.850)--(6.355,4.856)%
  --(6.361,4.860)--(6.374,4.867)--(6.381,4.869)--(6.381,4.880)--(6.388,4.883)--(6.388,4.886)%
  --(6.394,4.891)--(6.401,4.895)--(6.408,4.898)--(6.414,4.905)--(6.421,4.911)--(6.428,4.914)%
  --(6.428,4.918)--(6.441,4.923)--(6.441,4.928)--(6.448,4.932)--(6.448,4.934)--(6.448,4.935)%
  --(6.454,4.939)--(6.461,4.944)--(6.467,4.948)--(6.474,4.949)--(6.474,4.957)--(6.487,4.961)%
  --(6.494,4.968)--(6.501,4.972)--(6.501,4.975)--(6.507,4.978)--(6.514,4.980)--(6.514,4.982)%
  --(6.514,4.984)--(6.521,4.987)--(6.527,4.990)--(6.534,4.993)--(6.541,4.997)--(6.541,5.001)%
  --(6.547,5.005)--(6.547,5.007)--(6.554,5.010)--(6.567,5.017)--(6.580,5.024)--(6.594,5.039)%
  --(6.600,5.044)--(6.607,5.046)--(6.607,5.048)--(6.614,5.050)--(6.614,5.052)--(6.614,5.054)%
  --(6.620,5.058)--(6.627,5.063)--(6.634,5.061)--(6.634,5.071)--(6.640,5.074)--(6.647,5.078)%
  --(6.647,5.081)--(6.653,5.086)--(6.660,5.091)--(6.660,5.092)--(6.660,5.094)--(6.667,5.094)%
  --(6.667,5.097)--(6.673,5.104)--(6.680,5.109)--(6.680,5.111)--(6.687,5.113)--(6.693,5.116)%
  --(6.700,5.119)--(6.700,5.122)--(6.700,5.124)--(6.707,5.124)--(6.720,5.131)--(6.726,5.140)%
  --(6.733,5.144)--(6.733,5.149)--(6.740,5.153)--(6.746,5.156)--(6.753,5.158)--(6.760,5.161)%
  --(6.766,5.165)--(6.773,5.167)--(6.780,5.170)--(6.780,5.178)--(6.793,5.182)--(6.800,5.189)%
  --(6.806,5.194)--(6.806,5.198)--(6.813,5.202)--(6.819,5.207)--(6.826,5.211)--(6.833,5.217)%
  --(6.833,5.220)--(6.839,5.223)--(6.846,5.227)--(6.846,5.228)--(6.846,5.230)--(6.846,5.233)%
  --(6.853,5.237)--(6.866,5.245)--(6.866,5.249)--(6.873,5.252)--(6.886,5.254)--(6.886,5.255)%
  --(6.893,5.263)--(6.893,5.269)--(6.899,5.273)--(6.906,5.277)--(6.906,5.279)--(6.912,5.275)%
  --(6.919,5.281)--(6.919,5.284)--(6.919,5.287)--(6.926,5.289)--(6.926,5.288)--(6.932,5.300)%
  --(6.939,5.305)--(6.946,5.309)--(6.952,5.311)--(6.966,5.315)--(6.972,5.324)--(6.972,5.327)%
  --(6.979,5.328)--(6.979,5.334)--(6.979,5.336)--(6.986,5.338)--(6.992,5.338)--(6.992,5.336)%
  --(7.005,5.339)--(7.005,5.344)--(7.019,5.350)--(7.025,5.358)--(7.032,5.364)--(7.039,5.370)%
  --(7.045,5.374)--(7.052,5.379)--(7.065,5.387)--(7.065,5.388)--(7.072,5.396)--(7.072,5.398)%
  --(7.079,5.401)--(7.079,5.405)--(7.092,5.409)--(7.098,5.413)--(7.098,5.417)--(7.105,5.420)%
  --(7.105,5.426)--(7.118,5.430)--(7.118,5.433)--(7.125,5.433)--(7.132,5.437)--(7.132,5.445)%
  --(7.145,5.451)--(7.145,5.455)--(7.152,5.460)--(7.158,5.463)--(7.158,5.462)--(7.158,5.464)%
  --(7.171,5.467)--(7.171,5.475)--(7.178,5.480)--(7.178,5.482)--(7.185,5.486)--(7.198,5.495)%
  --(7.205,5.502)--(7.205,5.504)--(7.211,5.502)--(7.218,5.505)--(7.225,5.507)--(7.231,5.509)%
  --(7.238,5.519)--(7.251,5.524)--(7.258,5.529)--(7.258,5.532)--(7.264,5.535)--(7.271,5.535)%
  --(7.278,5.541)--(7.291,5.558)--(7.298,5.561)--(7.298,5.564)--(7.304,5.565)--(7.304,5.567)%
  --(7.311,5.570)--(7.318,5.574)--(7.324,5.580)--(7.331,5.586)--(7.338,5.592)--(7.344,5.597)%
  --(7.351,5.601)--(7.357,5.605)--(7.364,5.611)--(7.364,5.616)--(7.371,5.617)--(7.371,5.613)%
  --(7.371,5.623)--(7.377,5.625)--(7.377,5.629)--(7.384,5.634)--(7.391,5.637)--(7.391,5.639)%
  --(7.391,5.641)--(7.391,5.644)--(7.397,5.642)--(7.404,5.647)--(7.404,5.650)--(7.404,5.653)%
  --(7.411,5.654)--(7.417,5.659)--(7.424,5.667)--(7.431,5.673)--(7.437,5.676)--(7.444,5.676)%
  --(7.450,5.684)--(7.457,5.688)--(7.464,5.692)--(7.470,5.695)--(7.477,5.697)--(7.477,5.700)%
  --(7.484,5.703)--(7.490,5.705)--(7.497,5.705)--(7.504,5.712)--(7.510,5.718)--(7.517,5.723)%
  --(7.524,5.727)--(7.537,5.731)--(7.537,5.735)--(7.543,5.739)--(7.550,5.744)--(7.550,5.746)%
  --(7.563,5.743)--(7.563,5.756)--(7.570,5.760)--(7.570,5.764)--(7.577,5.768)--(7.590,5.772)%
  --(7.590,5.776)--(7.590,5.780)--(7.597,5.784)--(7.597,5.786)--(7.597,5.787)--(7.603,5.788)%
  --(7.603,5.785)--(7.616,5.796)--(7.616,5.799)--(7.623,5.802)--(7.623,5.803)--(7.623,5.805)%
  --(7.630,5.808)--(7.636,5.807)--(7.636,5.818)--(7.643,5.822)--(7.650,5.824)--(7.650,5.825)%
  --(7.650,5.827)--(7.656,5.829)--(7.656,5.833)--(7.656,5.836)--(7.663,5.838)--(7.670,5.846)%
  --(7.676,5.850)--(7.676,5.853)--(7.683,5.856)--(7.690,5.858)--(7.690,5.860)--(7.696,5.862)%
  --(7.703,5.862)--(7.709,5.861)--(7.716,5.863)--(7.723,5.866)--(7.723,5.876)--(7.736,5.884)%
  --(7.743,5.890)--(7.749,5.892)--(7.749,5.894)--(7.756,5.897)--(7.763,5.896)--(7.763,5.898)%
  --(7.763,5.905)--(7.769,5.906)--(7.769,5.909)--(7.776,5.912)--(7.783,5.916)--(7.789,5.920)%
  --(7.789,5.926)--(7.802,5.932)--(7.809,5.938)--(7.816,5.940)--(7.816,5.939)--(7.822,5.941)%
  --(7.822,5.949)--(7.822,5.950)--(7.829,5.951)--(7.836,5.957)--(7.836,5.964)--(7.842,5.965)%
  --(7.849,5.968)--(7.849,5.970)--(7.849,5.973)--(7.856,5.979)--(7.862,5.982)--(7.862,5.985)%
  --(7.869,5.987)--(7.869,5.990)--(7.876,5.994)--(7.876,5.996)--(7.882,6.000)--(7.882,6.001)%
  --(7.889,6.004)--(7.889,6.006)--(7.895,6.011)--(7.895,6.015)--(7.895,6.019)--(7.902,6.017)%
  --(7.902,6.021)--(7.909,6.024)--(7.922,6.026)--(7.922,6.034)--(7.929,6.038)--(7.935,6.043)%
  --(7.942,6.045)--(7.955,6.047)--(7.955,6.048)--(7.955,6.047)--(7.962,6.050)--(7.969,6.058)%
  --(7.975,6.061)--(7.982,6.067)--(7.988,6.072)--(7.995,6.075)--(8.002,6.079)--(8.002,6.082)%
  --(8.008,6.079)--(8.008,6.080)--(8.008,6.083)--(8.015,6.090)--(8.022,6.098)--(8.022,6.102)%
  --(8.028,6.105)--(8.035,6.108)--(8.042,6.112)--(8.042,6.110)--(8.042,6.107)--(8.042,6.108)%
  --(8.048,6.114)--(8.055,6.120)--(8.055,6.124)--(8.062,6.129)--(8.062,6.131)--(8.062,6.134)%
  --(8.068,6.136)--(8.068,6.137)--(8.075,6.140)--(8.081,6.142)--(8.081,6.145)--(8.088,6.148)%
  --(8.081,6.149)--(8.088,6.150)--(8.088,6.151)--(8.088,6.154)--(8.095,6.153)--(8.095,6.156)%
  --(8.101,6.158)--(8.101,6.162)--(8.108,6.168)--(8.108,6.169)--(8.108,6.171)--(8.115,6.174)%
  --(8.121,6.177)--(8.128,6.181)--(8.128,6.186)--(8.128,6.192)--(8.135,6.196)--(8.141,6.200)%
  --(8.148,6.204)--(8.148,6.207)--(8.161,6.212)--(8.161,6.213)--(8.174,6.213)--(8.181,6.219)%
  --(8.188,6.221)--(8.194,6.224)--(8.194,6.227)--(8.194,6.231)--(8.201,6.234)--(8.214,6.239)%
  --(8.221,6.244)--(8.221,6.241)--(8.221,6.251)--(8.228,6.253)--(8.228,6.254)--(8.234,6.257)%
  --(8.241,6.260)--(8.247,6.266)--(8.254,6.270)--(8.261,6.273)--(8.267,6.276)--(8.267,6.281)%
  --(8.281,6.290)--(8.287,6.293)--(8.287,6.296)--(8.294,6.299)--(8.294,6.302)--(8.294,6.304)%
  --(8.301,6.307)--(8.301,6.312)--(8.307,6.311)--(8.314,6.320)--(8.314,6.321)--(8.314,6.322)%
  --(8.321,6.324)--(8.321,6.326)--(8.321,6.327)--(8.327,6.330)--(8.327,6.334)--(8.334,6.338)%
  --(8.334,6.333)--(8.334,6.344)--(8.340,6.348)--(8.340,6.349)--(8.354,6.352)--(8.354,6.353)%
  --(8.354,6.356)--(8.360,6.361)--(8.360,6.367)--(8.374,6.367)--(8.367,6.373)--(8.374,6.376)%
  --(8.380,6.377)--(8.387,6.380)--(8.394,6.383)--(8.394,6.387)--(8.400,6.392)--(8.407,6.395)%
  --(8.407,6.397)--(8.420,6.390)--(8.427,6.401)--(8.427,6.404)--(8.427,6.407)--(8.433,6.409)%
  --(8.433,6.411)--(8.440,6.414)--(8.447,6.416)--(8.460,6.418)--(8.453,6.415)--(8.460,6.425)%
  --(8.460,6.429)--(8.467,6.432)--(8.473,6.435)--(8.473,6.438)--(8.473,6.442)--(8.487,6.446)%
  --(8.493,6.452)--(8.500,6.455)--(8.507,6.464)--(8.507,6.466)--(8.513,6.469)--(8.513,6.473)%
  --(8.520,6.475)--(8.526,6.479)--(8.533,6.483)--(8.540,6.487)--(8.540,6.488)--(8.546,6.488)%
  --(8.546,6.492)--(8.553,6.500)--(8.553,6.505)--(8.553,6.508)--(8.560,6.509)--(8.566,6.514)%
  --(8.566,6.517)--(8.573,6.521)--(8.580,6.525)--(8.580,6.529)--(8.586,6.531)--(8.599,6.534)%
  --(8.606,6.546)--(8.613,6.554)--(8.613,6.558)--(8.619,6.561)--(8.626,6.565)--(8.633,6.565)%
  --(8.639,6.567)--(8.646,6.568)--(8.653,6.574)--(8.666,6.582)--(8.673,6.587)--(8.679,6.592)%
  --(8.679,6.595)--(8.686,6.599)--(8.692,6.596)--(8.692,6.599)--(8.706,6.606)--(8.712,6.613)%
  --(8.712,6.620)--(8.712,6.622)--(8.719,6.624)--(8.719,6.626)--(8.726,6.629)--(8.726,6.628)%
  --(8.726,6.629)--(8.732,6.638)--(8.746,6.644)--(8.752,6.647)--(8.752,6.652)--(8.766,6.657)%
  --(8.766,6.662)--(8.772,6.666)--(8.779,6.668)--(8.779,6.666)--(8.785,6.669)--(8.785,6.673)%
  --(8.792,6.681)--(8.792,6.689)--(8.799,6.692)--(8.799,6.694)--(8.799,6.697)--(8.799,6.699)%
  --(8.805,6.699)--(8.812,6.699)--(8.812,6.702)--(8.819,6.707)--(8.825,6.716)--(8.825,6.718)%
  --(8.832,6.721)--(8.832,6.724)--(8.832,6.725)--(8.832,6.727)--(8.839,6.722)--(8.839,6.724)%
  --(8.839,6.727)--(8.852,6.731)--(8.859,6.743)--(8.865,6.748)--(8.872,6.749)--(8.878,6.752)%
  --(8.878,6.753)--(8.885,6.758)--(8.905,6.759)--(8.918,6.767)--(8.918,6.771)--(8.925,6.775)%
  --(8.938,6.785)--(8.945,6.791)--(8.958,6.800)--(8.958,6.805)--(8.965,6.812)--(8.978,6.811)%
  --(8.985,6.821)--(8.998,6.831)--(9.005,6.842)--(9.018,6.851)--(9.025,6.857)--(9.025,6.862)%
  --(9.031,6.865)--(9.031,6.868)--(9.038,6.872)--(9.058,6.879)--(9.058,6.896)--(9.071,6.901)%
  --(9.071,6.907)--(9.071,6.910)--(9.084,6.911)--(9.084,6.916)--(9.098,6.919)--(9.098,6.920)%
  --(9.104,6.922)--(9.111,6.926)--(9.118,6.926)--(9.131,6.936)--(9.137,6.941)--(9.137,6.942)%
  --(9.137,6.943)--(9.137,6.946)--(9.144,6.949)--(9.151,6.949)--(9.151,6.955)--(9.157,6.961)%
  --(9.171,6.966)--(9.171,6.970)--(9.177,6.973)--(9.177,6.975)--(9.184,6.979)--(9.184,6.982)%
  --(9.197,6.986)--(9.197,6.982)--(9.211,6.997)--(9.217,7.004)--(9.224,7.010)--(9.230,7.015)%
  --(9.230,7.019)--(9.230,7.021)--(9.237,7.023)--(9.237,7.026)--(9.250,7.027)--(9.257,7.037)%
  --(9.264,7.045)--(9.270,7.051)--(9.277,7.057)--(9.277,7.060)--(9.284,7.065)--(9.297,7.072)%
  --(9.304,7.081)--(9.310,7.087)--(9.317,7.083)--(9.323,7.095)--(9.330,7.100)--(9.337,7.102)%
  --(9.343,7.105)--(9.343,7.106)--(9.350,7.109)--(9.357,7.112)--(9.357,7.115)--(9.363,7.122)%
  --(9.377,7.127)--(9.377,7.130)--(9.383,7.123)--(9.383,7.135)--(9.390,7.138)--(9.403,7.144)%
  --(9.403,7.150)--(9.410,7.152)--(9.416,7.154)--(9.410,7.146)--(9.410,7.156)--(9.416,7.157)%
  --(9.423,7.160)--(9.423,7.163)--(9.430,7.165)--(9.430,7.169)--(9.436,7.174)--(9.443,7.178)%
  --(9.443,7.177)--(9.450,7.181)--(9.443,7.187)--(9.450,7.189)--(9.456,7.190)--(9.463,7.192)%
  --(9.463,7.194)--(9.470,7.196)--(9.476,7.200)--(9.476,7.206)--(9.483,7.204)--(9.490,7.219)%
  --(9.490,7.223)--(9.490,7.227)--(9.496,7.230)--(9.503,7.234)--(9.509,7.236)--(9.509,7.240)%
  --(9.516,7.246)--(9.516,7.247)--(9.523,7.246)--(9.529,7.248)--(9.523,7.250)--(9.529,7.260)%
  --(9.536,7.264)--(9.549,7.273)--(9.563,7.279)--(9.569,7.279)--(9.569,7.281)--(9.576,7.281)%
  --(9.576,7.282)--(9.582,7.278)--(9.589,7.283)--(9.596,7.292)--(9.602,7.301)--(9.622,7.311)%
  --(9.636,7.320)--(9.629,7.325)--(9.642,7.325)--(9.649,7.328)--(9.656,7.334)--(9.656,7.346)%
  --(9.669,7.350)--(9.675,7.357)--(9.682,7.362)--(9.689,7.365)--(9.695,7.366)--(9.695,7.362)%
  --(9.702,7.365)--(9.702,7.371)--(9.709,7.382)--(9.709,7.390)--(9.715,7.395)--(9.722,7.400)%
  --(9.729,7.404)--(9.735,7.409)--(9.742,7.409)--(9.749,7.411)--(9.749,7.417)--(9.749,7.421)%
  --(9.755,7.430)--(9.755,7.432)--(9.762,7.435)--(9.775,7.440)--(9.782,7.446);
\gpcolor{color=gp lt color 5}
\gpsetlinetype{gp lt plot 5}
\draw[gp path] (2.084,1.713)--(2.090,1.714)--(2.097,1.716)--(2.104,1.720)--(2.117,1.724)%
  --(2.124,1.728)--(2.130,1.731)--(2.137,1.734)--(2.137,1.736)--(2.137,1.737)--(2.137,1.738)%
  --(2.137,1.743)--(2.144,1.746)--(2.150,1.749)--(2.150,1.752)--(2.157,1.757)--(2.164,1.760)%
  --(2.164,1.763)--(2.170,1.769)--(2.177,1.775)--(2.183,1.776)--(2.183,1.778)--(2.183,1.780)%
  --(2.190,1.783)--(2.197,1.786)--(2.197,1.789)--(2.203,1.793)--(2.203,1.795)--(2.210,1.798)%
  --(2.217,1.800)--(2.217,1.803)--(2.217,1.805)--(2.223,1.808)--(2.230,1.810)--(2.230,1.813)%
  --(2.237,1.820)--(2.250,1.828)--(2.263,1.832)--(2.263,1.836)--(2.270,1.841)--(2.276,1.846)%
  --(2.283,1.854)--(2.290,1.859)--(2.296,1.863)--(2.303,1.868)--(2.310,1.875)--(2.323,1.881)%
  --(2.330,1.886)--(2.336,1.890)--(2.349,1.900)--(2.363,1.909)--(2.369,1.918)--(2.383,1.924)%
  --(2.389,1.928)--(2.396,1.932)--(2.403,1.937)--(2.403,1.942)--(2.409,1.946)--(2.416,1.950)%
  --(2.423,1.957)--(2.436,1.965)--(2.442,1.970)--(2.442,1.974)--(2.449,1.977)--(2.462,1.985)%
  --(2.462,1.988)--(2.469,1.989)--(2.469,1.990)--(2.469,1.992)--(2.476,1.997)--(2.489,2.008)%
  --(2.496,2.016)--(2.502,2.017)--(2.502,2.021)--(2.516,2.027)--(2.522,2.034)--(2.535,2.042)%
  --(2.542,2.049)--(2.549,2.054)--(2.549,2.056)--(2.555,2.059)--(2.569,2.066)--(2.575,2.075)%
  --(2.589,2.082)--(2.595,2.090)--(2.602,2.095)--(2.609,2.100)--(2.615,2.104)--(2.622,2.110)%
  --(2.628,2.115)--(2.635,2.123)--(2.648,2.130)--(2.655,2.137)--(2.668,2.145)--(2.675,2.148)%
  --(2.675,2.150)--(2.682,2.155)--(2.682,2.159)--(2.688,2.161)--(2.688,2.163)--(2.688,2.165)%
  --(2.688,2.167)--(2.695,2.171)--(2.708,2.177)--(2.715,2.183)--(2.721,2.189)--(2.728,2.193)%
  --(2.735,2.198)--(2.741,2.202)--(2.741,2.207)--(2.748,2.211)--(2.761,2.216)--(2.768,2.224)%
  --(2.781,2.233)--(2.788,2.239)--(2.794,2.244)--(2.801,2.248)--(2.808,2.252)--(2.814,2.256)%
  --(2.821,2.260)--(2.828,2.266)--(2.834,2.274)--(2.841,2.280)--(2.854,2.286)--(2.861,2.295)%
  --(2.874,2.303)--(2.887,2.312)--(2.894,2.320)--(2.901,2.323)--(2.901,2.327)--(2.914,2.332)%
  --(2.921,2.336)--(2.921,2.338)--(2.921,2.340)--(2.927,2.345)--(2.941,2.355)--(2.954,2.363)%
  --(2.961,2.368)--(2.967,2.372)--(2.974,2.377)--(2.980,2.383)--(2.987,2.391)--(3.000,2.396)%
  --(3.007,2.401)--(3.007,2.405)--(3.014,2.409)--(3.020,2.415)--(3.034,2.422)--(3.040,2.429)%
  --(3.047,2.434)--(3.060,2.442)--(3.067,2.449)--(3.073,2.455)--(3.073,2.458)--(3.080,2.465)%
  --(3.087,2.471)--(3.093,2.475)--(3.100,2.477)--(3.100,2.478)--(3.107,2.480)--(3.107,2.485)%
  --(3.113,2.489)--(3.120,2.491)--(3.120,2.492)--(3.120,2.493)--(3.120,2.494)--(3.127,2.497)%
  --(3.133,2.502)--(3.140,2.508)--(3.140,2.510)--(3.147,2.511)--(3.147,2.512)--(3.147,2.513)%
  --(3.147,2.516)--(3.153,2.518)--(3.160,2.522)--(3.173,2.530)--(3.180,2.536)--(3.186,2.541)%
  --(3.193,2.546)--(3.200,2.550)--(3.206,2.556)--(3.213,2.561)--(3.220,2.565)--(3.226,2.572)%
  --(3.233,2.577)--(3.240,2.583)--(3.253,2.590)--(3.259,2.598)--(3.273,2.602)--(3.279,2.607)%
  --(3.286,2.613)--(3.286,2.617)--(3.286,2.619)--(3.293,2.622)--(3.293,2.625)--(3.299,2.629)%
  --(3.306,2.633)--(3.313,2.638)--(3.319,2.641)--(3.319,2.643)--(3.326,2.647)--(3.332,2.651)%
  --(3.339,2.656)--(3.339,2.660)--(3.346,2.662)--(3.346,2.667)--(3.352,2.670)--(3.352,2.672)%
  --(3.352,2.674)--(3.359,2.675)--(3.359,2.678)--(3.366,2.679)--(3.366,2.681)--(3.366,2.682)%
  --(3.372,2.684)--(3.372,2.685)--(3.372,2.686)--(3.379,2.689)--(3.379,2.690)--(3.379,2.692)%
  --(3.386,2.693)--(3.386,2.695)--(3.392,2.698)--(3.392,2.703)--(3.399,2.709)--(3.406,2.712)%
  --(3.412,2.714)--(3.419,2.717)--(3.425,2.724)--(3.432,2.730)--(3.439,2.734)--(3.439,2.738)%
  --(3.445,2.740)--(3.452,2.744)--(3.459,2.749)--(3.459,2.751)--(3.465,2.754)--(3.465,2.758)%
  --(3.472,2.763)--(3.479,2.770)--(3.492,2.774)--(3.499,2.779)--(3.505,2.784)--(3.512,2.790)%
  --(3.518,2.796)--(3.525,2.800)--(3.525,2.802)--(3.532,2.805)--(3.532,2.809)--(3.538,2.814)%
  --(3.545,2.817)--(3.552,2.822)--(3.558,2.828)--(3.565,2.833)--(3.572,2.837)--(3.578,2.841)%
  --(3.578,2.842)--(3.585,2.846)--(3.585,2.849)--(3.585,2.851)--(3.592,2.856)--(3.598,2.859)%
  --(3.598,2.861)--(3.598,2.865)--(3.605,2.867)--(3.605,2.870)--(3.618,2.873)--(3.625,2.881)%
  --(3.631,2.885)--(3.638,2.889)--(3.638,2.891)--(3.645,2.892)--(3.651,2.893)--(3.658,2.898)%
  --(3.665,2.906)--(3.671,2.913)--(3.678,2.920)--(3.691,2.926)--(3.698,2.936)--(3.711,2.943)%
  --(3.718,2.946)--(3.718,2.947)--(3.718,2.948)--(3.724,2.954)--(3.738,2.961)--(3.744,2.968)%
  --(3.751,2.976)--(3.758,2.980)--(3.764,2.985)--(3.771,2.989)--(3.777,2.993)--(3.791,2.999)%
  --(3.797,3.004)--(3.804,3.010)--(3.811,3.018)--(3.824,3.024)--(3.824,3.030)--(3.831,3.038)%
  --(3.837,3.042)--(3.844,3.043)--(3.844,3.045)--(3.851,3.048)--(3.857,3.051)--(3.864,3.058)%
  --(3.870,3.063)--(3.877,3.069)--(3.890,3.077)--(3.897,3.085)--(3.904,3.090)--(3.910,3.092)%
  --(3.904,3.092)--(3.904,3.096)--(3.910,3.097)--(3.910,3.098)--(3.910,3.101)--(3.910,3.102)%
  --(3.917,3.104)--(3.944,3.122)--(3.970,3.137)--(3.983,3.146)--(3.990,3.154)--(3.997,3.158)%
  --(3.997,3.159)--(3.990,3.159)--(3.997,3.160)--(3.997,3.162)--(3.997,3.163)--(4.003,3.168)%
  --(4.003,3.171)--(4.003,3.172)--(4.010,3.176)--(4.017,3.177)--(4.017,3.180)--(4.023,3.180)%
  --(4.023,3.182)--(4.023,3.185)--(4.030,3.186)--(4.030,3.187)--(4.030,3.190)--(4.030,3.191)%
  --(4.037,3.192)--(4.030,3.193)--(4.037,3.193)--(4.037,3.195)--(4.050,3.200)--(4.090,3.228)%
  --(4.123,3.251)--(4.130,3.259)--(4.136,3.263)--(4.143,3.268)--(4.149,3.273)--(4.156,3.278)%
  --(4.169,3.286)--(4.176,3.296)--(4.189,3.305)--(4.196,3.311)--(4.203,3.315)--(4.209,3.317)%
  --(4.209,3.319)--(4.209,3.320)--(4.209,3.318)--(4.209,3.321)--(4.209,3.323)--(4.209,3.324)%
  --(4.222,3.332)--(4.229,3.338)--(4.242,3.346)--(4.256,3.354)--(4.256,3.357)--(4.262,3.360)%
  --(4.269,3.365)--(4.276,3.370)--(4.289,3.377)--(4.289,3.382)--(4.296,3.387)--(4.309,3.393)%
  --(4.315,3.400)--(4.329,3.407)--(4.342,3.414)--(4.355,3.423)--(4.369,3.435)--(4.382,3.444)%
  --(4.402,3.458)--(4.408,3.468)--(4.422,3.474)--(4.428,3.481)--(4.435,3.487)--(4.442,3.493)%
  --(4.455,3.500)--(4.462,3.505)--(4.462,3.512)--(4.468,3.515)--(4.468,3.516)--(4.468,3.518)%
  --(4.468,3.519)--(4.468,3.518)--(4.468,3.520)--(4.468,3.523)--(4.468,3.525)--(4.475,3.527)%
  --(4.475,3.528)--(4.482,3.532)--(4.482,3.535)--(4.488,3.538)--(4.488,3.537)--(4.495,3.539)%
  --(4.495,3.540)--(4.495,3.544)--(4.495,3.546)--(4.501,3.547)--(4.508,3.551)--(4.515,3.554)%
  --(4.515,3.558)--(4.528,3.566)--(4.541,3.572)--(4.548,3.577)--(4.548,3.578)--(4.548,3.580)%
  --(4.555,3.585)--(4.561,3.589)--(4.561,3.593)--(4.568,3.598)--(4.568,3.601)--(4.575,3.602)%
  --(4.575,3.603)--(4.575,3.604)--(4.581,3.607)--(4.581,3.611)--(4.588,3.612)--(4.588,3.615)%
  --(4.594,3.619)--(4.601,3.622)--(4.608,3.628)--(4.614,3.631)--(4.621,3.637)--(4.621,3.640)%
  --(4.628,3.643)--(4.634,3.647)--(4.634,3.650)--(4.641,3.654)--(4.648,3.660)--(4.654,3.664)%
  --(4.661,3.668)--(4.668,3.673)--(4.674,3.677)--(4.674,3.685)--(4.681,3.686)--(4.687,3.688)%
  --(4.687,3.691)--(4.694,3.693)--(4.694,3.696)--(4.701,3.698)--(4.707,3.701)--(4.707,3.705)%
  --(4.714,3.711)--(4.714,3.713)--(4.721,3.715)--(4.721,3.718)--(4.721,3.719)--(4.727,3.721)%
  --(4.727,3.722)--(4.727,3.723)--(4.734,3.725)--(4.741,3.732)--(4.741,3.734)--(4.747,3.737)%
  --(4.754,3.742)--(4.767,3.748)--(4.767,3.750)--(4.774,3.753)--(4.780,3.757)--(4.787,3.761)%
  --(4.794,3.769)--(4.800,3.775)--(4.807,3.781)--(4.814,3.787)--(4.827,3.792)--(4.834,3.797)%
  --(4.840,3.803)--(4.847,3.810)--(4.847,3.813)--(4.853,3.815)--(4.860,3.817)--(4.867,3.821)%
  --(4.867,3.826)--(4.880,3.838)--(4.887,3.842)--(4.893,3.844)--(4.900,3.849)--(4.907,3.852)%
  --(4.907,3.855)--(4.913,3.859)--(4.913,3.862)--(4.920,3.866)--(4.927,3.870)--(4.940,3.882)%
  --(4.940,3.885)--(4.946,3.889)--(4.946,3.890)--(4.953,3.892)--(4.953,3.895)--(4.960,3.897)%
  --(4.966,3.901)--(4.973,3.906)--(4.986,3.914)--(4.993,3.921)--(5.000,3.925)--(5.013,3.929)%
  --(5.013,3.931)--(5.013,3.936)--(5.026,3.943)--(5.026,3.949)--(5.039,3.954)--(5.039,3.957)%
  --(5.046,3.961)--(5.046,3.962)--(5.053,3.969)--(5.059,3.974)--(5.066,3.978)--(5.073,3.982)%
  --(5.073,3.985)--(5.079,3.990)--(5.093,3.995)--(5.099,3.999)--(5.099,4.003)--(5.106,4.006)%
  --(5.106,4.012)--(5.113,4.014)--(5.119,4.016)--(5.119,4.019)--(5.119,4.020)--(5.126,4.023)%
  --(5.126,4.025)--(5.132,4.025)--(5.132,4.029)--(5.132,4.031)--(5.139,4.034)--(5.146,4.038)%
  --(5.146,4.039)--(5.152,4.044)--(5.152,4.048)--(5.159,4.051)--(5.166,4.052)--(5.166,4.053)%
  --(5.172,4.057)--(5.172,4.060)--(5.179,4.062)--(5.179,4.063)--(5.179,4.066)--(5.186,4.071)%
  --(5.192,4.077)--(5.205,4.084)--(5.212,4.085)--(5.225,4.096)--(5.232,4.100)--(5.239,4.104)%
  --(5.252,4.112)--(5.252,4.115)--(5.259,4.118)--(5.259,4.122)--(5.259,4.125)--(5.272,4.128)%
  --(5.279,4.134)--(5.285,4.140)--(5.292,4.144)--(5.292,4.147)--(5.298,4.150)--(5.305,4.155)%
  --(5.305,4.160)--(5.312,4.164)--(5.318,4.169)--(5.325,4.171)--(5.332,4.180)--(5.338,4.187)%
  --(5.345,4.190)--(5.345,4.192)--(5.352,4.194)--(5.352,4.197)--(5.358,4.201)--(5.358,4.205)%
  --(5.358,4.206)--(5.372,4.207)--(5.372,4.213)--(5.378,4.218)--(5.378,4.220)--(5.385,4.224)%
  --(5.385,4.227)--(5.391,4.230)--(5.398,4.234)--(5.405,4.238)--(5.411,4.241)--(5.418,4.245)%
  --(5.425,4.251)--(5.431,4.259)--(5.438,4.266)--(5.445,4.272)--(5.458,4.277)--(5.465,4.280)%
  --(5.471,4.283)--(5.484,4.290)--(5.491,4.297)--(5.498,4.305)--(5.504,4.312)--(5.511,4.314)%
  --(5.511,4.319)--(5.518,4.324)--(5.524,4.331)--(5.531,4.335)--(5.538,4.335)--(5.538,4.339)%
  --(5.544,4.345)--(5.558,4.354)--(5.564,4.359)--(5.564,4.363)--(5.577,4.369)--(5.584,4.377)%
  --(5.597,4.383)--(5.611,4.390)--(5.617,4.395)--(5.624,4.403)--(5.624,4.409)--(5.631,4.410)%
  --(5.631,4.415)--(5.631,4.416)--(5.631,4.418)--(5.637,4.419)--(5.637,4.420)--(5.637,4.419)%
  --(5.644,4.423)--(5.657,4.428)--(5.664,4.438)--(5.670,4.445)--(5.677,4.447)--(5.684,4.450)%
  --(5.690,4.454)--(5.697,4.460)--(5.704,4.462)--(5.710,4.465)--(5.724,4.472)--(5.730,4.479)%
  --(5.737,4.484)--(5.743,4.490)--(5.750,4.494)--(5.750,4.498)--(5.757,4.502)--(5.757,4.505)%
  --(5.763,4.508)--(5.770,4.513)--(5.777,4.521)--(5.790,4.526)--(5.797,4.533)--(5.810,4.543)%
  --(5.817,4.549)--(5.823,4.554)--(5.823,4.558)--(5.830,4.562)--(5.836,4.563)--(5.843,4.567)%
  --(5.850,4.571)--(5.850,4.575)--(5.856,4.581)--(5.850,4.581)--(5.856,4.582)--(5.856,4.584)%
  --(5.863,4.586)--(5.863,4.587)--(5.870,4.590)--(5.883,4.602)--(5.890,4.609)--(5.896,4.614)%
  --(5.903,4.619)--(5.910,4.623)--(5.916,4.628)--(5.916,4.632)--(5.923,4.633)--(5.929,4.633)%
  --(5.936,4.633)--(5.943,4.637)--(5.943,4.641)--(5.949,4.648)--(5.949,4.651)--(5.956,4.655)%
  --(5.956,4.658)--(5.963,4.663)--(5.969,4.663)--(5.976,4.665)--(5.983,4.672)--(5.983,4.676)%
  --(5.989,4.680)--(5.996,4.684)--(6.003,4.690)--(6.003,4.695)--(6.009,4.698)--(6.009,4.701)%
  --(6.016,4.701)--(6.022,4.710)--(6.029,4.717)--(6.029,4.721)--(6.036,4.725)--(6.042,4.728)%
  --(6.049,4.732)--(6.056,4.734)--(6.062,4.739)--(6.069,4.746)--(6.069,4.749)--(6.069,4.752)%
  --(6.069,4.754)--(6.076,4.756)--(6.082,4.758)--(6.082,4.761)--(6.089,4.765)--(6.096,4.763)%
  --(6.102,4.772)--(6.102,4.777)--(6.102,4.779)--(6.102,4.781)--(6.109,4.782)--(6.109,4.785)%
  --(6.115,4.790)--(6.129,4.798)--(6.129,4.800)--(6.142,4.806)--(6.149,4.810)--(6.155,4.814)%
  --(6.162,4.817)--(6.169,4.819)--(6.169,4.822)--(6.175,4.825)--(6.175,4.827)--(6.182,4.831)%
  --(6.188,4.841)--(6.195,4.845)--(6.208,4.849)--(6.215,4.855)--(6.228,4.860)--(6.228,4.867)%
  --(6.235,4.873)--(6.248,4.880)--(6.255,4.885)--(6.268,4.896)--(6.268,4.901)--(6.275,4.907)%
  --(6.281,4.912)--(6.288,4.915)--(6.288,4.917)--(6.288,4.920)--(6.295,4.923)--(6.295,4.924)%
  --(6.301,4.920)--(6.301,4.928)--(6.301,4.931)--(6.308,4.932)--(6.308,4.934)--(6.315,4.938)%
  --(6.315,4.939)--(6.321,4.943)--(6.321,4.948)--(6.328,4.946)--(6.335,4.951)--(6.335,4.954)%
  --(6.335,4.957)--(6.335,4.959)--(6.341,4.959)--(6.341,4.962)--(6.341,4.963)--(6.341,4.966)%
  --(6.348,4.963)--(6.355,4.967)--(6.355,4.974)--(6.361,4.981)--(6.374,4.986)--(6.381,4.990)%
  --(6.388,4.993)--(6.394,4.997)--(6.401,4.999)--(6.401,5.000)--(6.408,5.002)--(6.408,5.004)%
  --(6.414,5.007)--(6.421,5.014)--(6.428,5.018)--(6.434,5.023)--(6.434,5.029)--(6.441,5.034)%
  --(6.448,5.036)--(6.454,5.037)--(6.454,5.040)--(6.461,5.044)--(6.467,5.048)--(6.474,5.054)%
  --(6.474,5.061)--(6.487,5.066)--(6.494,5.072)--(6.501,5.078)--(6.507,5.080)--(6.514,5.086)%
  --(6.514,5.088)--(6.521,5.090)--(6.521,5.095)--(6.521,5.099)--(6.527,5.104)--(6.534,5.107)%
  --(6.534,5.110)--(6.541,5.109)--(6.547,5.110)--(6.547,5.114)--(6.554,5.120)--(6.560,5.127)%
  --(6.567,5.130)--(6.574,5.137)--(6.580,5.143)--(6.587,5.149)--(6.594,5.154)--(6.600,5.155)%
  --(6.607,5.159)--(6.614,5.161)--(6.614,5.165)--(6.627,5.170)--(6.627,5.174)--(6.634,5.178)%
  --(6.634,5.180)--(6.647,5.183)--(6.653,5.187)--(6.660,5.193)--(6.673,5.201)--(6.680,5.209)%
  --(6.693,5.217)--(6.700,5.226)--(6.700,5.228)--(6.700,5.232)--(6.707,5.235)--(6.713,5.241)%
  --(6.726,5.249)--(6.733,5.254)--(6.746,5.261)--(6.753,5.267)--(6.760,5.276)--(6.766,5.282)%
  --(6.773,5.286)--(6.773,5.290)--(6.780,5.295)--(6.786,5.296)--(6.786,5.297)--(6.786,5.300)%
  --(6.793,5.301)--(6.793,5.306)--(6.800,5.309)--(6.800,5.310)--(6.800,5.311)--(6.806,5.310)%
  --(6.813,5.314)--(6.819,5.326)--(6.826,5.329)--(6.833,5.338)--(6.839,5.341)--(6.839,5.343)%
  --(6.846,5.346)--(6.853,5.349)--(6.859,5.352)--(6.859,5.350)--(6.866,5.353)--(6.866,5.356)%
  --(6.873,5.359)--(6.879,5.365)--(6.886,5.371)--(6.886,5.372)--(6.893,5.375)--(6.893,5.379)%
  --(6.899,5.383)--(6.912,5.384)--(6.912,5.391)--(6.919,5.395)--(6.919,5.399)--(6.926,5.404)%
  --(6.939,5.410)--(6.939,5.412)--(6.939,5.414)--(6.946,5.417)--(6.952,5.417)--(6.966,5.432)%
  --(6.972,5.437)--(6.972,5.440)--(6.979,5.442)--(6.979,5.444)--(6.979,5.447)--(6.986,5.449)%
  --(6.992,5.452)--(6.992,5.453)--(6.992,5.458)--(6.999,5.464)--(6.999,5.466)--(7.005,5.468)%
  --(7.005,5.471)--(7.012,5.472)--(7.019,5.474)--(7.019,5.479)--(7.025,5.482)--(7.032,5.483)%
  --(7.045,5.497)--(7.059,5.507)--(7.065,5.511)--(7.072,5.515)--(7.079,5.519)--(7.092,5.523)%
  --(7.098,5.528)--(7.098,5.532)--(7.105,5.534)--(7.112,5.529)--(7.118,5.539)--(7.118,5.545)%
  --(7.125,5.548)--(7.132,5.555)--(7.132,5.560)--(7.138,5.564)--(7.152,5.569)--(7.158,5.574)%
  --(7.171,5.574)--(7.178,5.587)--(7.185,5.594)--(7.191,5.600)--(7.205,5.608)--(7.211,5.616)%
  --(7.218,5.621)--(7.211,5.621)--(7.218,5.622)--(7.218,5.623)--(7.225,5.624)--(7.225,5.630)%
  --(7.231,5.634)--(7.231,5.638)--(7.231,5.639)--(7.238,5.643)--(7.238,5.645)--(7.238,5.648)%
  --(7.245,5.649)--(7.251,5.647)--(7.258,5.657)--(7.264,5.661)--(7.264,5.667)--(7.271,5.673)%
  --(7.278,5.676)--(7.284,5.681)--(7.291,5.683)--(7.298,5.687)--(7.304,5.688)--(7.318,5.693)%
  --(7.324,5.699)--(7.324,5.703)--(7.331,5.705)--(7.338,5.708)--(7.338,5.711)--(7.344,5.713)%
  --(7.344,5.718)--(7.357,5.721)--(7.364,5.723)--(7.364,5.724)--(7.371,5.734)--(7.377,5.738)%
  --(7.384,5.742)--(7.384,5.744)--(7.397,5.748)--(7.397,5.752)--(7.397,5.756)--(7.404,5.758)%
  --(7.411,5.756)--(7.404,5.758)--(7.411,5.760)--(7.417,5.769)--(7.417,5.773)--(7.424,5.776)%
  --(7.424,5.778)--(7.424,5.780)--(7.431,5.783)--(7.431,5.787)--(7.437,5.792)--(7.444,5.793)%
  --(7.450,5.792)--(7.450,5.795)--(7.450,5.801)--(7.450,5.802)--(7.457,5.805)--(7.457,5.806)%
  --(7.464,5.806)--(7.464,5.807)--(7.470,5.812)--(7.477,5.817)--(7.484,5.825)--(7.490,5.830)%
  --(7.490,5.834)--(7.490,5.837)--(7.497,5.839)--(7.497,5.842)--(7.504,5.840)--(7.504,5.843)%
  --(7.510,5.844)--(7.510,5.846)--(7.517,5.850)--(7.517,5.856)--(7.524,5.857)--(7.530,5.858)%
  --(7.530,5.860)--(7.537,5.862)--(7.537,5.863)--(7.550,5.862)--(7.557,5.867)--(7.563,5.880)%
  --(7.570,5.886)--(7.577,5.890)--(7.583,5.893)--(7.590,5.896)--(7.590,5.898)--(7.590,5.895)%
  --(7.597,5.894)--(7.597,5.897)--(7.597,5.901)--(7.603,5.907)--(7.603,5.909)--(7.610,5.912)%
  --(7.610,5.914)--(7.610,5.917)--(7.616,5.919)--(7.623,5.923)--(7.630,5.927)--(7.630,5.925)%
  --(7.636,5.927)--(7.636,5.934)--(7.636,5.936)--(7.636,5.939)--(7.650,5.941)--(7.643,5.944)%
  --(7.650,5.941)--(7.650,5.942)--(7.656,5.945)--(7.656,5.948)--(7.663,5.952)--(7.663,5.959)%
  --(7.663,5.961)--(7.663,5.964)--(7.663,5.966)--(7.670,5.966)--(7.670,5.964)--(7.676,5.966)%
  --(7.683,5.968)--(7.683,5.977)--(7.683,5.978)--(7.683,5.981)--(7.690,5.982)--(7.690,5.984)%
  --(7.690,5.986)--(7.690,5.987)--(7.690,5.989)--(7.690,5.986)--(7.703,5.987)--(7.696,5.991)%
  --(7.709,5.997)--(7.709,6.001)--(7.709,6.003)--(7.716,6.006)--(7.716,6.010)--(7.723,6.009)%
  --(7.723,6.020)--(7.729,6.023)--(7.736,6.027)--(7.743,6.029)--(7.749,6.031)--(7.749,6.034)%
  --(7.756,6.040)--(7.769,6.044)--(7.776,6.041)--(7.783,6.053)--(7.783,6.056)--(7.789,6.058)%
  --(7.796,6.061)--(7.802,6.063)--(7.809,6.069)--(7.816,6.073)--(7.816,6.075)--(7.822,6.075)%
  --(7.829,6.079)--(7.829,6.083)--(7.829,6.085)--(7.836,6.087)--(7.836,6.091)--(7.836,6.094)%
  --(7.842,6.097)--(7.842,6.098)--(7.842,6.100)--(7.842,6.097)--(7.849,6.102)--(7.856,6.105)%
  --(7.856,6.107)--(7.862,6.109)--(7.862,6.111)--(7.869,6.113)--(7.869,6.115)--(7.869,6.118)%
  --(7.876,6.121)--(7.882,6.123)--(7.876,6.120)--(7.876,6.127)--(7.876,6.130)--(7.889,6.131)%
  --(7.889,6.134)--(7.895,6.136)--(7.889,6.136)--(7.895,6.131)--(7.895,6.139)--(7.902,6.140)%
  --(7.902,6.142)--(7.902,6.145)--(7.902,6.147)--(7.902,6.149)--(7.909,6.152)--(7.909,6.154)%
  --(7.909,6.155)--(7.915,6.155)--(7.915,6.159)--(7.915,6.163)--(7.922,6.164)--(7.922,6.165)%
  --(7.929,6.168)--(7.929,6.171)--(7.929,6.175)--(7.935,6.177)--(7.935,6.172)--(7.935,6.181)%
  --(7.935,6.182)--(7.942,6.186)--(7.942,6.188)--(7.949,6.189)--(7.955,6.191)--(7.955,6.193)%
  --(7.955,6.196)--(7.962,6.194)--(7.962,6.195)--(7.962,6.201)--(7.962,6.202)--(7.969,6.204)%
  --(7.969,6.205)--(7.975,6.208)--(7.975,6.210)--(7.988,6.211)--(7.988,6.214)--(7.995,6.215)%
  --(7.995,6.216)--(7.995,6.213)--(8.002,6.219)--(8.002,6.220)--(8.008,6.222)--(8.002,6.223)%
  --(8.015,6.219)--(8.008,6.219)--(8.015,6.223)--(8.022,6.224)--(8.022,6.231)--(8.022,6.235)%
  --(8.028,6.239)--(8.035,6.242)--(8.042,6.247)--(8.042,6.251)--(8.048,6.253)--(8.055,6.255)%
  --(8.055,6.251)--(8.055,6.252)--(8.062,6.261)--(8.062,6.264)--(8.068,6.265)--(8.068,6.268)%
  --(8.068,6.270)--(8.075,6.265)--(8.075,6.266)--(8.075,6.267)--(8.081,6.269)--(8.075,6.275)%
  --(8.081,6.275)--(8.081,6.278)--(8.081,6.279)--(8.088,6.282)--(8.095,6.282)--(8.095,6.283)%
  --(8.095,6.285)--(8.101,6.287)--(8.095,6.294)--(8.101,6.294)--(8.108,6.297)--(8.108,6.298)%
  --(8.108,6.302)--(8.115,6.305)--(8.121,6.307)--(8.121,6.309)--(8.121,6.311)--(8.121,6.314)%
  --(8.121,6.317)--(8.128,6.319)--(8.128,6.321)--(8.135,6.319)--(8.135,6.321)--(8.141,6.324)%
  --(8.148,6.328)--(8.148,6.336)--(8.148,6.339)--(8.154,6.340)--(8.154,6.342)--(8.154,6.344)%
  --(8.161,6.346)--(8.154,6.348)--(8.161,6.345)--(8.161,6.349)--(8.168,6.355)--(8.168,6.359)%
  --(8.168,6.363)--(8.181,6.367)--(8.188,6.365)--(8.188,6.368)--(8.194,6.371)--(8.201,6.376)%
  --(8.201,6.379)--(8.208,6.381)--(8.214,6.384)--(8.214,6.387)--(8.221,6.389)--(8.221,6.386)%
  --(8.221,6.388)--(8.228,6.398)--(8.228,6.394)--(8.228,6.395)--(8.228,6.397)--(8.228,6.398)%
  --(8.234,6.398)--(8.234,6.400)--(8.234,6.401)--(8.241,6.403)--(8.247,6.400)--(8.247,6.407)%
  --(8.254,6.409)--(8.254,6.412)--(8.254,6.414)--(8.254,6.415)--(8.261,6.413)--(8.267,6.414)%
  --(8.261,6.415)--(8.267,6.421)--(8.274,6.425)--(8.267,6.427)--(8.274,6.428)--(8.281,6.430)%
  --(8.274,6.431)--(8.281,6.428)--(8.287,6.430)--(8.287,6.435)--(8.287,6.439)--(8.287,6.440)%
  --(8.287,6.441)--(8.287,6.442)--(8.287,6.435)--(8.287,6.444)--(8.294,6.446)--(8.294,6.449)%
  --(8.307,6.455)--(8.321,6.461)--(8.321,6.467)--(8.327,6.469)--(8.327,6.470)--(8.334,6.473)%
  --(8.334,6.467)--(8.334,6.478)--(8.334,6.480)--(8.340,6.483)--(8.347,6.486)--(8.354,6.488)%
  --(8.354,6.492)--(8.354,6.495)--(8.354,6.498)--(8.360,6.500)--(8.367,6.502)--(8.360,6.503)%
  --(8.367,6.504)--(8.367,6.498)--(8.367,6.508)--(8.374,6.510)--(8.374,6.512)--(8.387,6.519)%
  --(8.394,6.525)--(8.394,6.524)--(8.400,6.536)--(8.400,6.539)--(8.407,6.542)--(8.407,6.544)%
  --(8.407,6.545)--(8.407,6.547)--(8.407,6.548)--(8.407,6.547)--(8.414,6.544)--(8.414,6.550)%
  --(8.414,6.552)--(8.420,6.554)--(8.427,6.558)--(8.433,6.559)--(8.433,6.562)--(8.433,6.557)%
  --(8.440,6.567)--(8.440,6.568)--(8.453,6.569)--(8.453,6.570)--(8.453,6.571)--(8.460,6.573)%
  --(8.460,6.575)--(8.467,6.575)--(8.473,6.579)--(8.473,6.585)--(8.480,6.589)--(8.480,6.594)%
  --(8.493,6.598)--(8.500,6.599)--(8.500,6.601)--(8.507,6.603)--(8.507,6.606)--(8.507,6.610)%
  --(8.513,6.613)--(8.513,6.616)--(8.520,6.619)--(8.520,6.622)--(8.526,6.625)--(8.540,6.631)%
  --(8.540,6.636)--(8.540,6.638)--(8.540,6.640)--(8.553,6.639)--(8.553,6.645)--(8.553,6.648)%
  --(8.560,6.650)--(8.566,6.653)--(8.573,6.656)--(8.573,6.660)--(8.580,6.663)--(8.580,6.665)%
  --(8.580,6.661)--(8.580,6.671)--(8.586,6.674)--(8.593,6.678)--(8.593,6.679)--(8.593,6.680)%
  --(8.593,6.683)--(8.599,6.685)--(8.599,6.687)--(8.606,6.683)--(8.606,6.685)--(8.606,6.687)%
  --(8.606,6.688)--(8.606,6.694)--(8.613,6.698)--(8.613,6.700)--(8.613,6.702)--(8.613,6.704)%
  --(8.626,6.705)--(8.626,6.709)--(8.633,6.713)--(8.646,6.725)--(8.646,6.727)--(8.646,6.729)%
  --(8.653,6.733)--(8.659,6.736)--(8.659,6.739)--(8.666,6.740)--(8.673,6.742)--(8.679,6.744)%
  --(8.679,6.748)--(8.686,6.751)--(8.686,6.755)--(8.699,6.761)--(8.706,6.765)--(8.712,6.767)%
  --(8.706,6.765)--(8.712,6.763)--(8.719,6.766)--(8.719,6.769)--(8.726,6.772)--(8.726,6.780)%
  --(8.726,6.782)--(8.739,6.785)--(8.739,6.788)--(8.746,6.792)--(8.746,6.789)--(8.752,6.793)%
  --(8.752,6.800)--(8.766,6.806)--(8.772,6.816)--(8.779,6.820)--(8.785,6.822)--(8.792,6.823)%
  --(8.792,6.828)--(8.799,6.830)--(8.799,6.833)--(8.805,6.839)--(8.805,6.843)--(8.805,6.850)%
  --(8.812,6.854)--(8.825,6.859)--(8.825,6.864)--(8.832,6.865)--(8.832,6.861)--(8.832,6.862)%
  --(8.832,6.865)--(8.845,6.870)--(8.845,6.880)--(8.852,6.891)--(8.859,6.897)--(8.865,6.901)%
  --(8.872,6.906)--(8.872,6.911)--(8.878,6.913)--(8.878,6.915)--(8.885,6.918)--(8.892,6.919)%
  --(8.898,6.919)--(8.905,6.920)--(8.905,6.921)--(8.912,6.916)--(8.918,6.919)--(8.912,6.923)%
  --(8.925,6.927)--(8.932,6.932)--(8.938,6.943)--(8.945,6.948)--(8.952,6.954)--(8.958,6.959)%
  --(8.965,6.961)--(8.978,6.962)--(8.978,6.968)--(8.985,6.971)--(8.985,6.974)--(8.991,6.980)%
  --(8.985,6.984)--(8.991,6.986)--(8.991,6.988)--(8.998,6.989)--(8.998,6.988)--(9.005,6.993)%
  --(9.018,6.999)--(9.025,7.004)--(9.025,7.016)--(9.031,7.021)--(9.031,7.024)--(9.045,7.027)%
  --(9.045,7.031)--(9.051,7.030)--(9.051,7.034)--(9.058,7.038)--(9.064,7.041)--(9.064,7.047)%
  --(9.071,7.058)--(9.084,7.068)--(9.091,7.073)--(9.098,7.078)--(9.098,7.082)--(9.098,7.083)%
  --(9.098,7.086)--(9.104,7.088)--(9.104,7.093)--(9.111,7.097)--(9.118,7.101)--(9.124,7.105)%
  --(9.131,7.107)--(9.137,7.109)--(9.137,7.105)--(9.144,7.106)--(9.151,7.111)--(9.157,7.117)%
  --(9.171,7.129)--(9.177,7.132)--(9.184,7.136)--(9.184,7.140)--(9.191,7.143)--(9.191,7.147)%
  --(9.197,7.151)--(9.204,7.153)--(9.204,7.146)--(9.211,7.158)--(9.217,7.163)--(9.217,7.167)%
  --(9.224,7.174)--(9.230,7.179)--(9.237,7.181)--(9.237,7.176)--(9.237,7.184)--(9.237,7.187)%
  --(9.237,7.189)--(9.244,7.190)--(9.244,7.191)--(9.257,7.193)--(9.257,7.194)--(9.257,7.196)%
  --(9.264,7.190)--(9.270,7.203)--(9.264,7.210)--(9.270,7.214)--(9.270,7.217)--(9.277,7.220)%
  --(9.277,7.225)--(9.290,7.229)--(9.290,7.232)--(9.297,7.235)--(9.297,7.238)--(9.310,7.244)%
  --(9.310,7.249)--(9.317,7.253)--(9.323,7.259)--(9.330,7.266)--(9.337,7.272)--(9.343,7.275)%
  --(9.350,7.277)--(9.350,7.274)--(9.357,7.284)--(9.363,7.287)--(9.377,7.290)--(9.383,7.293)%
  --(9.390,7.300)--(9.397,7.307)--(9.403,7.313)--(9.410,7.316)--(9.410,7.319)--(9.416,7.323)%
  --(9.423,7.326)--(9.430,7.329)--(9.436,7.338)--(9.443,7.344)--(9.443,7.346)--(9.443,7.350)%
  --(9.463,7.354)--(9.463,7.358)--(9.470,7.355)--(9.470,7.365)--(9.476,7.369)--(9.476,7.375)%
  --(9.483,7.383)--(9.496,7.388)--(9.503,7.392)--(9.503,7.395)--(9.503,7.397)--(9.509,7.399)%
  --(9.509,7.392)--(9.516,7.404)--(9.516,7.406)--(9.516,7.407)--(9.516,7.408)--(9.523,7.410)%
  --(9.523,7.411)--(9.523,7.413)--(9.523,7.418)--(9.529,7.422)--(9.536,7.432)--(9.543,7.434)%
  --(9.556,7.439)--(9.556,7.443)--(9.563,7.447)--(9.576,7.450)--(9.576,7.453);
\gpcolor{color=gp lt color 6}
\gpsetlinetype{gp lt plot 6}
\draw[gp path] (2.057,1.722)--(2.064,1.724)--(2.071,1.726)--(2.077,1.729)--(2.077,1.730)%
  --(2.077,1.732)--(2.084,1.734)--(2.090,1.738)--(2.090,1.742)--(2.090,1.744)--(2.097,1.747)%
  --(2.104,1.751)--(2.110,1.758)--(2.110,1.760)--(2.110,1.761)--(2.117,1.766)--(2.124,1.770)%
  --(2.124,1.774)--(2.124,1.775)--(2.130,1.777)--(2.137,1.780)--(2.150,1.789)--(2.164,1.800)%
  --(2.170,1.805)--(2.170,1.807)--(2.177,1.808)--(2.183,1.813)--(2.190,1.821)--(2.190,1.824)%
  --(2.197,1.826)--(2.197,1.829)--(2.210,1.831)--(2.217,1.835)--(2.217,1.840)--(2.223,1.843)%
  --(2.230,1.847)--(2.230,1.854)--(2.237,1.858)--(2.243,1.861)--(2.250,1.866)--(2.250,1.870)%
  --(2.257,1.874)--(2.270,1.881)--(2.283,1.886)--(2.290,1.893)--(2.290,1.898)--(2.296,1.903)%
  --(2.303,1.907)--(2.310,1.914)--(2.316,1.921)--(2.323,1.925)--(2.330,1.928)--(2.336,1.930)%
  --(2.336,1.933)--(2.343,1.937)--(2.349,1.943)--(2.356,1.947)--(2.356,1.949)--(2.363,1.955)%
  --(2.369,1.962)--(2.369,1.964)--(2.376,1.967)--(2.383,1.971)--(2.396,1.979)--(2.409,1.988)%
  --(2.416,1.994)--(2.423,2.000)--(2.429,2.004)--(2.429,2.005)--(2.436,2.010)--(2.442,2.017)%
  --(2.442,2.018)--(2.442,2.019)--(2.449,2.021)--(2.456,2.027)--(2.456,2.030)--(2.462,2.031)%
  --(2.469,2.037)--(2.482,2.048)--(2.489,2.049)--(2.489,2.051)--(2.489,2.054)--(2.489,2.055)%
  --(2.496,2.058)--(2.502,2.063)--(2.509,2.068)--(2.516,2.074)--(2.522,2.080)--(2.529,2.087)%
  --(2.542,2.093)--(2.549,2.098)--(2.549,2.100)--(2.549,2.102)--(2.549,2.104)--(2.555,2.106)%
  --(2.555,2.110)--(2.562,2.113)--(2.569,2.117)--(2.575,2.126)--(2.589,2.131)--(2.589,2.133)%
  --(2.595,2.139)--(2.609,2.145)--(2.615,2.150)--(2.622,2.155)--(2.628,2.160)--(2.628,2.165)%
  --(2.635,2.170)--(2.642,2.174)--(2.648,2.175)--(2.648,2.179)--(2.655,2.182)--(2.655,2.184)%
  --(2.655,2.186)--(2.662,2.188)--(2.662,2.190)--(2.662,2.193)--(2.668,2.195)--(2.675,2.197)%
  --(2.682,2.204)--(2.682,2.208)--(2.688,2.211)--(2.695,2.214)--(2.695,2.216)--(2.708,2.223)%
  --(2.715,2.232)--(2.721,2.237)--(2.728,2.240)--(2.728,2.244)--(2.735,2.248)--(2.741,2.252)%
  --(2.748,2.253)--(2.748,2.257)--(2.755,2.261)--(2.761,2.266)--(2.768,2.271)--(2.768,2.272)%
  --(2.775,2.275)--(2.775,2.279)--(2.781,2.283)--(2.788,2.291)--(2.808,2.300)--(2.814,2.309)%
  --(2.821,2.315)--(2.828,2.318)--(2.828,2.322)--(2.834,2.326)--(2.841,2.329)--(2.848,2.335)%
  --(2.854,2.339)--(2.861,2.345)--(2.868,2.348)--(2.868,2.350)--(2.874,2.354)--(2.881,2.358)%
  --(2.887,2.362)--(2.901,2.374)--(2.921,2.386)--(2.934,2.396)--(2.941,2.400)--(2.941,2.404)%
  --(2.947,2.406)--(2.947,2.409)--(2.947,2.411)--(2.954,2.413)--(2.954,2.415)--(2.954,2.418)%
  --(2.961,2.420)--(2.967,2.425)--(2.967,2.429)--(2.974,2.433)--(2.980,2.435)--(2.980,2.437)%
  --(2.980,2.439)--(2.980,2.440)--(2.987,2.445)--(2.994,2.451)--(3.007,2.458)--(3.007,2.462)%
  --(3.014,2.469)--(3.027,2.475)--(3.027,2.477)--(3.034,2.478)--(3.034,2.481)--(3.040,2.486)%
  --(3.054,2.492)--(3.060,2.494)--(3.067,2.501)--(3.073,2.508)--(3.080,2.510)--(3.080,2.511)%
  --(3.080,2.514)--(3.087,2.518)--(3.093,2.521)--(3.093,2.526)--(3.100,2.529)--(3.107,2.532)%
  --(3.113,2.536)--(3.120,2.542)--(3.127,2.550)--(3.140,2.555)--(3.147,2.560)--(3.153,2.565)%
  --(3.160,2.570)--(3.160,2.576)--(3.166,2.581)--(3.173,2.586)--(3.186,2.594)--(3.193,2.600)%
  --(3.206,2.605)--(3.206,2.610)--(3.213,2.615)--(3.213,2.617)--(3.220,2.622)--(3.226,2.626)%
  --(3.226,2.629)--(3.233,2.633)--(3.240,2.638)--(3.246,2.643)--(3.253,2.647)--(3.259,2.652)%
  --(3.266,2.656)--(3.273,2.660)--(3.279,2.668)--(3.286,2.675)--(3.293,2.679)--(3.299,2.681)%
  --(3.306,2.684)--(3.306,2.687)--(3.313,2.691)--(3.326,2.696)--(3.326,2.701)--(3.332,2.708)%
  --(3.339,2.712)--(3.339,2.714)--(3.346,2.718)--(3.359,2.723)--(3.366,2.732)--(3.379,2.742)%
  --(3.392,2.751)--(3.399,2.755)--(3.399,2.758)--(3.406,2.763)--(3.412,2.771)--(3.425,2.776)%
  --(3.432,2.783)--(3.439,2.793)--(3.445,2.800)--(3.452,2.801)--(3.452,2.804)--(3.459,2.806)%
  --(3.465,2.812)--(3.472,2.819)--(3.479,2.823)--(3.485,2.828)--(3.492,2.834)--(3.499,2.841)%
  --(3.512,2.845)--(3.518,2.852)--(3.525,2.857)--(3.532,2.861)--(3.538,2.866)--(3.538,2.870)%
  --(3.545,2.874)--(3.552,2.879)--(3.558,2.884)--(3.565,2.890)--(3.578,2.896)--(3.598,2.912)%
  --(3.618,2.924)--(3.625,2.936)--(3.631,2.939)--(3.638,2.943)--(3.645,2.948)--(3.651,2.953)%
  --(3.658,2.960)--(3.665,2.967)--(3.671,2.976)--(3.678,2.980)--(3.685,2.987)--(3.691,2.989)%
  --(3.704,2.996)--(3.704,3.000)--(3.711,3.004)--(3.711,3.006)--(3.724,3.010)--(3.738,3.018)%
  --(3.751,3.032)--(3.764,3.043)--(3.777,3.047)--(3.777,3.052)--(3.784,3.058)--(3.791,3.060)%
  --(3.791,3.064)--(3.811,3.075)--(3.837,3.092)--(3.844,3.102)--(3.857,3.108)--(3.864,3.115)%
  --(3.870,3.126)--(3.877,3.131)--(3.890,3.137)--(3.897,3.146)--(3.910,3.155)--(3.917,3.161)%
  --(3.917,3.167)--(3.924,3.169)--(3.924,3.170)--(3.924,3.171)--(3.924,3.173)--(3.930,3.175)%
  --(3.930,3.177)--(3.937,3.179)--(3.937,3.180)--(3.937,3.181)--(3.944,3.181)--(3.950,3.186)%
  --(3.950,3.189)--(3.957,3.192)--(3.957,3.196)--(3.963,3.200)--(3.970,3.207)--(3.977,3.211)%
  --(3.977,3.216)--(3.983,3.217)--(3.990,3.220)--(3.997,3.225)--(4.003,3.229)--(4.010,3.234)%
  --(4.017,3.245)--(4.030,3.250)--(4.037,3.256)--(4.043,3.262)--(4.056,3.270)--(4.063,3.276)%
  --(4.076,3.285)--(4.083,3.291)--(4.090,3.298)--(4.096,3.306)--(4.103,3.309)--(4.103,3.311)%
  --(4.110,3.313)--(4.110,3.315)--(4.116,3.316)--(4.116,3.319)--(4.116,3.322)--(4.123,3.327)%
  --(4.130,3.332)--(4.136,3.336)--(4.143,3.341)--(4.149,3.345)--(4.156,3.349)--(4.163,3.355)%
  --(4.169,3.357)--(4.169,3.363)--(4.176,3.366)--(4.176,3.368)--(4.176,3.371)--(4.183,3.374)%
  --(4.189,3.377)--(4.189,3.380)--(4.196,3.381)--(4.196,3.383)--(4.196,3.385)--(4.203,3.387)%
  --(4.203,3.386)--(4.209,3.392)--(4.216,3.395)--(4.222,3.401)--(4.229,3.406)--(4.236,3.410)%
  --(4.242,3.412)--(4.242,3.416)--(4.249,3.421)--(4.256,3.427)--(4.262,3.434)--(4.282,3.444)%
  --(4.302,3.459)--(4.315,3.471)--(4.322,3.476)--(4.329,3.479)--(4.335,3.486)--(4.349,3.492)%
  --(4.355,3.499)--(4.362,3.505)--(4.362,3.510)--(4.369,3.514)--(4.375,3.515)--(4.375,3.517)%
  --(4.375,3.518)--(4.375,3.517)--(4.382,3.523)--(4.389,3.532)--(4.402,3.539)--(4.408,3.545)%
  --(4.415,3.548)--(4.422,3.553)--(4.428,3.556)--(4.435,3.563)--(4.448,3.572)--(4.468,3.582)%
  --(4.482,3.597)--(4.488,3.604)--(4.495,3.612)--(4.508,3.617)--(4.508,3.621)--(4.515,3.624)%
  --(4.515,3.628)--(4.521,3.631)--(4.528,3.635)--(4.535,3.641)--(4.535,3.645)--(4.541,3.649)%
  --(4.548,3.654)--(4.548,3.659)--(4.555,3.665)--(4.561,3.669)--(4.568,3.673)--(4.575,3.677)%
  --(4.581,3.681)--(4.588,3.686)--(4.588,3.688)--(4.594,3.691)--(4.594,3.693)--(4.601,3.696)%
  --(4.601,3.698)--(4.601,3.697)--(4.608,3.701)--(4.614,3.708)--(4.621,3.714)--(4.621,3.717)%
  --(4.628,3.720)--(4.634,3.724)--(4.641,3.729)--(4.648,3.734)--(4.648,3.735)--(4.654,3.737)%
  --(4.661,3.740)--(4.668,3.746)--(4.668,3.751)--(4.674,3.753)--(4.674,3.756)--(4.687,3.761)%
  --(4.687,3.764)--(4.694,3.768)--(4.701,3.772)--(4.714,3.781)--(4.721,3.787)--(4.734,3.794)%
  --(4.741,3.803)--(4.747,3.810)--(4.747,3.812)--(4.754,3.816)--(4.760,3.818)--(4.774,3.828)%
  --(4.780,3.839)--(4.787,3.843)--(4.794,3.847)--(4.800,3.850)--(4.807,3.858)--(4.814,3.863)%
  --(4.827,3.868)--(4.827,3.875)--(4.834,3.877)--(4.840,3.880)--(4.847,3.884)--(4.853,3.888)%
  --(4.860,3.895)--(4.867,3.900)--(4.867,3.904)--(4.873,3.912)--(4.880,3.918)--(4.887,3.921)%
  --(4.887,3.922)--(4.893,3.921)--(4.893,3.926)--(4.907,3.930)--(4.920,3.939)--(4.927,3.948)%
  --(4.927,3.951)--(4.940,3.956)--(4.940,3.960)--(4.946,3.961)--(4.946,3.968)--(4.953,3.973)%
  --(4.960,3.976)--(4.960,3.977)--(4.966,3.978)--(4.966,3.982)--(4.966,3.984)--(4.973,3.988)%
  --(4.980,3.990)--(4.986,3.999)--(4.993,4.002)--(4.993,4.003)--(4.993,4.006)--(5.000,4.009)%
  --(5.000,4.012)--(5.013,4.017)--(5.020,4.020)--(5.026,4.027)--(5.039,4.030)--(5.039,4.039)%
  --(5.039,4.041)--(5.046,4.044)--(5.053,4.048)--(5.059,4.053)--(5.059,4.055)--(5.066,4.057)%
  --(5.073,4.061)--(5.073,4.065)--(5.073,4.064)--(5.073,4.065)--(5.073,4.067)--(5.073,4.069)%
  --(5.079,4.071)--(5.079,4.074)--(5.086,4.078)--(5.093,4.081)--(5.099,4.085)--(5.106,4.095)%
  --(5.113,4.098)--(5.119,4.099)--(5.126,4.101)--(5.132,4.106)--(5.139,4.112)--(5.146,4.117)%
  --(5.152,4.122)--(5.159,4.126)--(5.159,4.130)--(5.166,4.134)--(5.172,4.139)--(5.179,4.142)%
  --(5.179,4.146)--(5.192,4.152)--(5.199,4.159)--(5.212,4.167)--(5.212,4.174)--(5.225,4.179)%
  --(5.232,4.188)--(5.239,4.191)--(5.239,4.193)--(5.245,4.201)--(5.252,4.205)--(5.252,4.207)%
  --(5.259,4.210)--(5.259,4.214)--(5.272,4.218)--(5.272,4.222)--(5.279,4.228)--(5.285,4.231)%
  --(5.285,4.232)--(5.285,4.234)--(5.292,4.236)--(5.292,4.238)--(5.292,4.239)--(5.298,4.243)%
  --(5.305,4.242)--(5.305,4.247)--(5.305,4.252)--(5.318,4.258)--(5.318,4.261)--(5.318,4.262)%
  --(5.318,4.265)--(5.325,4.268)--(5.332,4.270)--(5.338,4.272)--(5.352,4.278)--(5.358,4.287)%
  --(5.372,4.294)--(5.372,4.298)--(5.372,4.299)--(5.372,4.300)--(5.378,4.304)--(5.378,4.307)%
  --(5.385,4.308)--(5.391,4.312)--(5.398,4.319)--(5.405,4.324)--(5.405,4.328)--(5.411,4.334)%
  --(5.418,4.337)--(5.425,4.343)--(5.431,4.348)--(5.438,4.349)--(5.438,4.352)--(5.445,4.356)%
  --(5.451,4.362)--(5.458,4.370)--(5.471,4.377)--(5.471,4.381)--(5.478,4.384)--(5.484,4.386)%
  --(5.484,4.387)--(5.484,4.390)--(5.491,4.396)--(5.498,4.397)--(5.504,4.404)--(5.504,4.409)%
  --(5.511,4.414)--(5.511,4.416)--(5.518,4.417)--(5.518,4.415)--(5.518,4.416)--(5.524,4.419)%
  --(5.531,4.424)--(5.531,4.427)--(5.544,4.433)--(5.551,4.440)--(5.564,4.449)--(5.571,4.454)%
  --(5.577,4.461)--(5.584,4.459)--(5.591,4.462)--(5.597,4.468)--(5.597,4.475)--(5.611,4.480)%
  --(5.611,4.486)--(5.617,4.489)--(5.624,4.493)--(5.637,4.500)--(5.644,4.503)--(5.651,4.508)%
  --(5.657,4.519)--(5.664,4.526)--(5.670,4.530)--(5.684,4.538)--(5.697,4.548)--(5.710,4.557)%
  --(5.710,4.561)--(5.717,4.563)--(5.717,4.559)--(5.710,4.561)--(5.717,4.569)--(5.724,4.573)%
  --(5.730,4.576)--(5.737,4.581)--(5.737,4.582)--(5.737,4.586)--(5.743,4.587)--(5.743,4.590)%
  --(5.757,4.596)--(5.757,4.597)--(5.757,4.598)--(5.763,4.605)--(5.770,4.608)--(5.777,4.614)%
  --(5.783,4.623)--(5.797,4.630)--(5.803,4.629)--(5.810,4.636)--(5.817,4.640)--(5.830,4.647)%
  --(5.830,4.652)--(5.836,4.658)--(5.850,4.665)--(5.850,4.668)--(5.850,4.669)--(5.863,4.674)%
  --(5.870,4.676)--(5.876,4.692)--(5.890,4.703)--(5.903,4.714)--(5.923,4.725)--(5.923,4.730)%
  --(5.929,4.732)--(5.929,4.734)--(5.936,4.736)--(5.943,4.738)--(5.949,4.750)--(5.956,4.755)%
  --(5.956,4.757)--(5.963,4.759)--(5.963,4.764)--(5.969,4.766)--(5.969,4.767)--(5.969,4.770)%
  --(5.983,4.775)--(5.983,4.779)--(5.989,4.782)--(5.989,4.785)--(5.989,4.787)--(5.996,4.790)%
  --(6.003,4.795)--(6.003,4.800)--(6.009,4.803)--(6.016,4.804)--(6.022,4.809)--(6.029,4.814)%
  --(6.036,4.818)--(6.042,4.820)--(6.049,4.829)--(6.062,4.839)--(6.076,4.845)--(6.082,4.851)%
  --(6.096,4.859)--(6.109,4.866)--(6.122,4.880)--(6.129,4.889)--(6.135,4.895)--(6.142,4.899)%
  --(6.149,4.908)--(6.155,4.914)--(6.155,4.915)--(6.162,4.920)--(6.169,4.924)--(6.175,4.929)%
  --(6.182,4.927)--(6.182,4.937)--(6.188,4.943)--(6.195,4.948)--(6.202,4.953)--(6.208,4.958)%
  --(6.222,4.966)--(6.228,4.972)--(6.235,4.975)--(6.235,4.976)--(6.242,4.980)--(6.248,4.985)%
  --(6.248,4.988)--(6.255,4.991)--(6.262,4.996)--(6.268,5.001)--(6.275,5.007)--(6.288,5.012)%
  --(6.288,5.011)--(6.301,5.022)--(6.301,5.025)--(6.301,5.027)--(6.301,5.030)--(6.315,5.034)%
  --(6.315,5.040)--(6.321,5.045)--(6.328,5.048)--(6.335,5.047)--(6.335,5.051)--(6.341,5.062)%
  --(6.361,5.072)--(6.368,5.081)--(6.381,5.089)--(6.381,5.092)--(6.388,5.094)--(6.388,5.096)%
  --(6.388,5.100)--(6.394,5.101)--(6.401,5.102)--(6.401,5.106)--(6.408,5.115)--(6.421,5.123)%
  --(6.421,5.125)--(6.428,5.125)--(6.428,5.128)--(6.428,5.133)--(6.448,5.145)--(6.454,5.153)%
  --(6.461,5.153)--(6.461,5.156)--(6.467,5.158)--(6.474,5.163)--(6.474,5.168)--(6.481,5.170)%
  --(6.487,5.173)--(6.494,5.178)--(6.507,5.181)--(6.521,5.189)--(6.527,5.195)--(6.534,5.204)%
  --(6.534,5.209)--(6.541,5.212)--(6.541,5.216)--(6.554,5.224)--(6.560,5.228)--(6.574,5.232)%
  --(6.580,5.242)--(6.594,5.253)--(6.600,5.261)--(6.607,5.268)--(6.614,5.273)--(6.614,5.276)%
  --(6.620,5.281)--(6.627,5.284)--(6.634,5.291)--(6.634,5.293)--(6.634,5.295)--(6.640,5.294)%
  --(6.647,5.299)--(6.660,5.309)--(6.667,5.313)--(6.673,5.319)--(6.687,5.335)--(6.707,5.346)%
  --(6.713,5.346)--(6.720,5.349)--(6.726,5.352)--(6.733,5.359)--(6.740,5.367)--(6.746,5.372)%
  --(6.753,5.379)--(6.766,5.387)--(6.780,5.395)--(6.786,5.402)--(6.793,5.407)--(6.800,5.408)%
  --(6.806,5.412)--(6.819,5.423)--(6.826,5.434)--(6.833,5.439)--(6.833,5.441)--(6.839,5.443)%
  --(6.839,5.449)--(6.846,5.449)--(6.853,5.454)--(6.853,5.460)--(6.866,5.468)--(6.866,5.472)%
  --(6.873,5.474)--(6.873,5.477)--(6.879,5.482)--(6.886,5.484)--(6.893,5.486)--(6.893,5.489)%
  --(6.906,5.497)--(6.912,5.504)--(6.919,5.512)--(6.926,5.514)--(6.926,5.517)--(6.926,5.518)%
  --(6.939,5.522)--(6.939,5.526)--(6.952,5.524)--(6.959,5.533)--(6.959,5.536)--(6.966,5.541)%
  --(6.979,5.548)--(6.986,5.559)--(6.992,5.563)--(6.992,5.565)--(6.999,5.568)--(7.005,5.567)%
  --(7.012,5.576)--(7.019,5.581)--(7.025,5.585)--(7.025,5.590)--(7.032,5.598)--(7.039,5.600)%
  --(7.045,5.603)--(7.052,5.610)--(7.059,5.618)--(7.072,5.623)--(7.072,5.626)--(7.079,5.631)%
  --(7.079,5.638)--(7.079,5.640)--(7.085,5.644)--(7.092,5.646)--(7.098,5.648)--(7.098,5.653)%
  --(7.112,5.652)--(7.118,5.665)--(7.125,5.675)--(7.132,5.681)--(7.145,5.686)--(7.158,5.690)%
  --(7.171,5.695)--(7.185,5.704)--(7.191,5.710)--(7.198,5.715)--(7.205,5.718)--(7.218,5.730)%
  --(7.225,5.736)--(7.238,5.742)--(7.238,5.748)--(7.251,5.757)--(7.258,5.764)--(7.264,5.771)%
  --(7.278,5.779)--(7.284,5.780)--(7.291,5.792)--(7.298,5.797)--(7.298,5.802)--(7.311,5.806)%
  --(7.311,5.810)--(7.318,5.814)--(7.318,5.815)--(7.324,5.820)--(7.324,5.824)--(7.338,5.822)%
  --(7.338,5.835)--(7.344,5.841)--(7.351,5.848)--(7.364,5.854)--(7.371,5.858)--(7.377,5.862)%
  --(7.391,5.865)--(7.397,5.870)--(7.404,5.877)--(7.417,5.886)--(7.424,5.891)--(7.431,5.888)%
  --(7.444,5.901)--(7.450,5.909)--(7.457,5.917)--(7.470,5.925)--(7.477,5.930)--(7.484,5.931)%
  --(7.477,5.928)--(7.484,5.941)--(7.490,5.944)--(7.490,5.947)--(7.497,5.951)--(7.504,5.955)%
  --(7.510,5.960)--(7.510,5.967)--(7.517,5.970)--(7.517,5.969)--(7.524,5.969)--(7.524,5.976)%
  --(7.530,5.981)--(7.530,5.984)--(7.537,5.986)--(7.537,5.988)--(7.537,5.992)--(7.543,5.992)%
  --(7.543,5.995)--(7.550,5.994)--(7.550,5.997)--(7.563,6.003)--(7.570,6.013)--(7.570,6.020)%
  --(7.577,6.024)--(7.577,6.028)--(7.583,6.030)--(7.597,6.032)--(7.597,6.034)--(7.603,6.036)%
  --(7.610,6.041)--(7.623,6.052)--(7.630,6.057)--(7.643,6.061)--(7.650,6.069)--(7.663,6.079)%
  --(7.676,6.087)--(7.683,6.096)--(7.690,6.102)--(7.690,6.105)--(7.696,6.108)--(7.709,6.107)%
  --(7.716,6.114)--(7.723,6.126)--(7.723,6.132)--(7.736,6.138)--(7.743,6.147)--(7.749,6.153)%
  --(7.756,6.152)--(7.756,6.155)--(7.763,6.163)--(7.776,6.176)--(7.783,6.185)--(7.796,6.193)%
  --(7.802,6.199)--(7.809,6.206)--(7.816,6.210)--(7.829,6.208)--(7.842,6.212)--(7.849,6.221)%
  --(7.856,6.230)--(7.869,6.235)--(7.876,6.245)--(7.882,6.254)--(7.902,6.262)--(7.909,6.270)%
  --(7.922,6.275)--(7.929,6.280)--(7.942,6.288)--(7.949,6.302)--(7.955,6.311)--(7.969,6.319)%
  --(7.975,6.325)--(7.988,6.333)--(7.988,6.339)--(7.995,6.343)--(7.995,6.346)--(8.008,6.354)%
  --(8.022,6.368)--(8.028,6.376)--(8.042,6.382)--(8.048,6.390)--(8.055,6.397)--(8.068,6.401)%
  --(8.081,6.406)--(8.095,6.414)--(8.108,6.421)--(8.121,6.429)--(8.128,6.436)--(8.135,6.440)%
  --(8.141,6.452)--(8.148,6.456)--(8.154,6.461)--(8.154,6.465)--(8.161,6.468)--(8.174,6.476)%
  --(8.181,6.487)--(8.188,6.497)--(8.194,6.503)--(8.201,6.506)--(8.208,6.511)--(8.214,6.512)%
  --(8.214,6.516)--(8.221,6.518)--(8.221,6.525)--(8.228,6.529)--(8.234,6.534)--(8.234,6.539)%
  --(8.234,6.541)--(8.241,6.544)--(8.234,6.546)--(8.247,6.548)--(8.254,6.550)--(8.254,6.554)%
  --(8.254,6.557)--(8.261,6.551)--(8.261,6.563)--(8.267,6.568)--(8.274,6.572)--(8.274,6.575)%
  --(8.287,6.576)--(8.287,6.577)--(8.294,6.573)--(8.301,6.584)--(8.301,6.589)--(8.314,6.595)%
  --(8.321,6.599)--(8.327,6.601)--(8.334,6.604)--(8.334,6.609)--(8.340,6.617)--(8.354,6.620)%
  --(8.354,6.623)--(8.360,6.628)--(8.360,6.634)--(8.367,6.631)--(8.374,6.642)--(8.380,6.645)%
  --(8.387,6.649)--(8.387,6.652)--(8.394,6.655)--(8.394,6.652)--(8.400,6.663)--(8.407,6.666)%
  --(8.407,6.669)--(8.420,6.674)--(8.420,6.677)--(8.420,6.680)--(8.427,6.683)--(8.427,6.688)%
  --(8.440,6.690)--(8.440,6.701)--(8.440,6.706)--(8.447,6.709)--(8.460,6.715)--(8.467,6.722)%
  --(8.473,6.732)--(8.480,6.739)--(8.480,6.740)--(8.487,6.742)--(8.493,6.743)--(8.500,6.748)%
  --(8.500,6.751)--(8.513,6.753)--(8.513,6.758)--(8.526,6.762)--(8.533,6.768)--(8.540,6.773)%
  --(8.546,6.778)--(8.560,6.776)--(8.566,6.791)--(8.573,6.799)--(8.580,6.807)--(8.586,6.813)%
  --(8.593,6.818)--(8.606,6.820)--(8.606,6.823)--(8.613,6.832)--(8.619,6.837)--(8.626,6.842)%
  --(8.633,6.847)--(8.633,6.850)--(8.639,6.850)--(8.646,6.859)--(8.646,6.864)--(8.653,6.867)%
  --(8.653,6.869)--(8.659,6.872)--(8.666,6.878)--(8.673,6.880)--(8.679,6.886)--(8.679,6.896)%
  --(8.679,6.898)--(8.686,6.900)--(8.692,6.904)--(8.692,6.910)--(8.699,6.913)--(8.706,6.914)%
  --(8.712,6.911)--(8.712,6.913)--(8.719,6.915)--(8.719,6.924)--(8.726,6.925)--(8.726,6.928)%
  --(8.739,6.933)--(8.746,6.938)--(8.752,6.943)--(8.759,6.948)--(8.772,6.955)--(8.785,6.961)%
  --(8.792,6.959)--(8.792,6.965)--(8.799,6.977)--(8.799,6.982)--(8.812,6.986)--(8.825,6.994)%
  --(8.825,6.998)--(8.832,6.996)--(8.832,7.001)--(8.839,7.012)--(8.845,7.018)--(8.845,7.022)%
  --(8.845,7.024)--(8.859,7.027)--(8.859,7.031)--(8.865,7.035)--(8.865,7.038)--(8.872,7.041)%
  --(8.865,7.044)--(8.872,7.045)--(8.878,7.048)--(8.872,7.050)--(8.885,7.055)--(8.885,7.060)%
  --(8.892,7.063)--(8.898,7.069)--(8.905,7.071)--(8.912,7.071)--(8.912,7.074)--(8.912,7.086)%
  --(8.918,7.090)--(8.925,7.092)--(8.932,7.095)--(8.932,7.097)--(8.938,7.101)--(8.952,7.099)%
  --(8.952,7.103)--(8.958,7.113)--(8.965,7.116)--(8.971,7.122)--(8.978,7.126)--(8.985,7.131)%
  --(8.985,7.134)--(8.991,7.137)--(8.998,7.140)--(9.011,7.144)--(9.011,7.146)--(9.018,7.146)%
  --(9.025,7.154)--(9.025,7.162)--(9.038,7.168)--(9.038,7.176)--(9.038,7.180)--(9.051,7.177)%
  --(9.051,7.179)--(9.058,7.187)--(9.064,7.193)--(9.071,7.197)--(9.078,7.202)--(9.078,7.212)%
  --(9.091,7.219)--(9.084,7.222)--(9.091,7.226)--(9.098,7.230)--(9.104,7.234)--(9.104,7.240)%
  --(9.118,7.249)--(9.131,7.256)--(9.137,7.261)--(9.144,7.268)--(9.151,7.273)--(9.151,7.276)%
  --(9.157,7.275)--(9.157,7.273)--(9.164,7.275)--(9.171,7.284)--(9.177,7.288)--(9.184,7.291)%
  --(9.197,7.297)--(9.197,7.305)--(9.204,7.310)--(9.217,7.315)--(9.217,7.313)--(9.224,7.316)%
  --(9.230,7.325)--(9.230,7.329)--(9.230,7.333)--(9.230,7.334)--(9.230,7.337)--(9.237,7.339)%
  --(9.250,7.347)--(9.270,7.358)--(9.270,7.361)--(9.270,7.363)--(9.270,7.355)--(9.270,7.365)%
  --(9.277,7.367)--(9.284,7.372)--(9.284,7.380)--(9.297,7.386)--(9.304,7.390)--(9.310,7.391)%
  --(9.317,7.404)--(9.317,7.407)--(9.317,7.410)--(9.330,7.415)--(9.343,7.423)--(9.343,7.429)%
  --(9.350,7.431)--(9.343,7.433)--(9.350,7.435)--(9.357,7.441)--(9.363,7.445)--(9.363,7.447);
\gpcolor{color=gp lt color border}
\gpsetlinetype{gp lt border}
\draw[gp path] (1.320,7.496)--(1.320,1.708);
\draw[gp path] (2.057,0.985)--(10.825,0.985);
%% coordinates of the plot area
\gpdefrectangularnode{gp plot 1}{\pgfpoint{1.320cm}{0.985cm}}{\pgfpoint{11.947cm}{7.825cm}}
\end{tikzpicture}
%% gnuplot variables

\caption{Le varie isoterme dell'esperimento (dati al ritorno)}
\label{img:isoa}
\end{grafico}

\begin{grafico}
  \centering
\begin{tikzpicture}[gnuplot]
%% generated with GNUPLOT 4.6p3 (Lua 5.1; terminal rev. 99, script rev. 100)
%% mar 27 mag 2014 22:34:39 CEST
\path (0.000,0.000) rectangle (12.500,8.750);
\gpcolor{color=gp lt color border}
\gpsetlinetype{gp lt border}
\gpsetlinewidth{1.00}
\draw[gp path] (1.688,1.669)--(1.868,1.669);
\node[gp node right] at (1.504,1.669) { 2.15};
\draw[gp path] (1.688,2.353)--(1.868,2.353);
\node[gp node right] at (1.504,2.353) { 2.2};
\draw[gp path] (1.688,3.037)--(1.868,3.037);
\node[gp node right] at (1.504,3.037) { 2.25};
\draw[gp path] (1.688,3.721)--(1.868,3.721);
\node[gp node right] at (1.504,3.721) { 2.3};
\draw[gp path] (1.688,4.405)--(1.868,4.405);
\node[gp node right] at (1.504,4.405) { 2.35};
\draw[gp path] (1.688,5.089)--(1.868,5.089);
\node[gp node right] at (1.504,5.089) { 2.4};
\draw[gp path] (1.688,5.773)--(1.868,5.773);
\node[gp node right] at (1.504,5.773) { 2.45};
\draw[gp path] (1.688,6.457)--(1.868,6.457);
\node[gp node right] at (1.504,6.457) { 2.5};
\draw[gp path] (1.688,7.141)--(1.868,7.141);
\node[gp node right] at (1.504,7.141) { 2.55};
\draw[gp path] (1.688,0.985)--(1.688,1.165);
\node[gp node center] at (1.688,0.677) { 0};
\draw[gp path] (2.970,0.985)--(2.970,1.165);
\node[gp node center] at (2.970,0.677) { 200};
\draw[gp path] (4.253,0.985)--(4.253,1.165);
\node[gp node center] at (4.253,0.677) { 400};
\draw[gp path] (5.535,0.985)--(5.535,1.165);
\node[gp node center] at (5.535,0.677) { 600};
\draw[gp path] (6.818,0.985)--(6.818,1.165);
\node[gp node center] at (6.818,0.677) { 800};
\draw[gp path] (8.100,0.985)--(8.100,1.165);
\node[gp node center] at (8.100,0.677) { 1000};
\draw[gp path] (9.382,0.985)--(9.382,1.165);
\node[gp node center] at (9.382,0.677) { 1200};
\draw[gp path] (10.665,0.985)--(10.665,1.165);
\node[gp node center] at (10.665,0.677) { 1400};
\draw[gp path] (1.688,7.182)--(1.688,1.204);
\draw[gp path] (1.688,0.985)--(11.498,0.985);
\node[gp node center,rotate=-270] at (0.246,4.405) {Temperatura $[\gradi C]$};
\node[gp node center] at (6.817,0.215) {Numero della misura};
\node[gp node center] at (6.817,8.287) {Prima Isoterma Andata};
\gpcolor{color=gp lt color 0}
\gpsetlinetype{gp lt plot 0}
\draw[gp path] (1.688,7.155)--(1.694,7.032)--(1.701,7.059)--(1.707,7.018)--(1.714,7.059)%
  --(1.720,7.086)--(1.726,6.840)--(1.733,6.881)--(1.739,6.867)--(1.746,6.991)--(1.752,6.895)%
  --(1.759,6.881)--(1.765,6.854)--(1.771,6.840)--(1.778,6.854)--(1.784,6.840)--(1.791,6.840)%
  --(1.797,6.963)--(1.803,6.799)--(1.810,6.826)--(1.816,6.840)--(1.823,6.799)--(1.829,6.772)%
  --(1.835,6.813)--(1.842,7.182)--(1.848,7.182)--(1.855,7.182)--(1.861,7.155)--(1.868,6.758)%
  --(1.874,6.676)--(1.880,6.703)--(1.887,6.498)--(1.893,6.813)--(1.900,6.512)--(1.906,6.594)%
  --(1.912,6.539)--(1.919,6.649)--(1.925,6.799)--(1.932,6.471)--(1.938,6.512)--(1.944,6.649)%
  --(1.951,6.443)--(1.957,6.621)--(1.964,6.867)--(1.970,6.826)--(1.977,6.867)--(1.983,6.813)%
  --(1.989,6.525)--(1.996,6.334)--(2.002,6.375)--(2.009,6.293)--(2.015,6.320)--(2.021,6.635)%
  --(2.028,6.772)--(2.034,6.744)--(2.041,6.717)--(2.047,6.539)--(2.053,6.252)--(2.060,6.170)%
  --(2.066,6.197)--(2.073,6.142)--(2.079,6.156)--(2.086,6.156)--(2.092,6.156)--(2.098,6.183)%
  --(2.105,6.607)--(2.111,6.088)--(2.118,6.142)--(2.124,6.156)--(2.130,6.156)--(2.137,6.142)%
  --(2.143,6.389)--(2.150,6.525)--(2.156,6.457)--(2.162,6.484)--(2.169,6.320)--(2.175,6.088)%
  --(2.182,6.006)--(2.188,5.978)--(2.195,5.759)--(2.201,5.869)--(2.207,5.841)--(2.214,5.923)%
  --(2.220,5.814)--(2.227,6.211)--(2.233,5.828)--(2.239,5.855)--(2.246,5.841)--(2.252,5.800)%
  --(2.259,5.787)--(2.265,6.115)--(2.271,6.156)--(2.278,6.224)--(2.284,6.156)--(2.291,6.019)%
  --(2.297,5.855)--(2.304,5.609)--(2.310,5.595)--(2.316,5.554)--(2.323,5.540)--(2.329,5.499)%
  --(2.336,5.623)--(2.342,6.060)--(2.348,6.019)--(2.355,5.513)--(2.361,5.568)--(2.368,5.609)%
  --(2.374,5.499)--(2.380,5.458)--(2.387,5.527)--(2.393,5.472)--(2.400,5.458)--(2.406,5.759)%
  --(2.413,5.595)--(2.419,5.458)--(2.425,5.458)--(2.432,5.458)--(2.438,5.458)--(2.445,5.664)%
  --(2.451,5.828)--(2.457,5.828)--(2.464,5.814)--(2.470,5.759)--(2.477,5.349)--(2.483,5.322)%
  --(2.489,5.349)--(2.496,5.198)--(2.502,5.349)--(2.509,5.281)--(2.515,5.171)--(2.522,5.226)%
  --(2.528,5.499)--(2.534,5.253)--(2.541,5.198)--(2.547,5.390)--(2.554,5.308)--(2.560,5.157)%
  --(2.566,5.431)--(2.573,5.540)--(2.579,5.513)--(2.586,5.527)--(2.592,5.349)--(2.598,5.048)%
  --(2.605,4.897)--(2.611,4.870)--(2.618,4.952)--(2.624,4.925)--(2.631,5.103)--(2.637,5.021)%
  --(2.643,5.281)--(2.650,5.390)--(2.656,4.884)--(2.663,5.157)--(2.669,4.884)--(2.675,5.034)%
  --(2.682,4.774)--(2.688,4.925)--(2.695,4.829)--(2.701,4.815)--(2.707,5.157)--(2.714,4.993)%
  --(2.720,4.802)--(2.727,4.774)--(2.733,4.829)--(2.740,4.774)--(2.746,5.007)--(2.752,5.185)%
  --(2.759,5.198)--(2.765,5.198)--(2.772,5.171)--(2.778,4.747)--(2.784,4.733)--(2.791,4.774)%
  --(2.797,4.583)--(2.804,4.665)--(2.810,4.692)--(2.816,4.720)--(2.823,4.774)--(2.829,4.747)%
  --(2.836,4.569)--(2.842,4.747)--(2.849,4.720)--(2.855,4.733)--(2.861,4.774)--(2.868,4.870)%
  --(2.874,4.966)--(2.881,4.966)--(2.887,4.925)--(2.893,4.925)--(2.900,4.624)--(2.906,4.569)%
  --(2.913,4.638)--(2.919,4.555)--(2.925,4.460)--(2.932,4.747)--(2.938,4.829)--(2.945,4.829)%
  --(2.951,4.829)--(2.958,4.542)--(2.964,4.460)--(2.970,4.255)--(2.977,4.241)--(2.983,4.487)%
  --(2.990,4.323)--(2.996,4.268)--(3.002,4.282)--(3.009,4.446)--(3.015,4.706)--(3.022,4.296)%
  --(3.028,4.364)--(3.034,4.227)--(3.041,4.282)--(3.047,4.296)--(3.054,4.610)--(3.060,4.610)%
  --(3.067,4.651)--(3.073,4.610)--(3.079,4.555)--(3.086,4.145)--(3.092,4.077)--(3.099,4.131)%
  --(3.105,4.036)--(3.111,4.145)--(3.118,4.077)--(3.124,4.104)--(3.131,4.049)--(3.137,4.501)%
  --(3.143,4.049)--(3.150,4.227)--(3.156,4.172)--(3.163,4.227)--(3.169,3.981)--(3.176,4.460)%
  --(3.182,4.419)--(3.188,4.419)--(3.195,4.460)--(3.201,4.227)--(3.208,4.022)--(3.214,3.871)%
  --(3.220,3.954)--(3.227,3.954)--(3.233,3.871)--(3.240,3.995)--(3.246,4.049)--(3.252,4.296)%
  --(3.259,4.255)--(3.265,3.926)--(3.272,3.858)--(3.278,3.871)--(3.285,3.885)--(3.291,3.954)%
  --(3.297,3.721)--(3.304,3.789)--(3.310,3.707)--(3.317,4.104)--(3.323,3.803)--(3.329,3.748)%
  --(3.336,3.926)--(3.342,3.830)--(3.349,3.680)--(3.355,4.077)--(3.361,4.145)--(3.368,4.145)%
  --(3.374,4.118)--(3.381,4.008)--(3.387,3.721)--(3.394,3.694)--(3.400,3.639)--(3.406,3.543)%
  --(3.413,3.516)--(3.419,3.488)--(3.426,3.434)--(3.432,3.475)--(3.438,3.885)--(3.445,3.516)%
  --(3.451,3.721)--(3.458,3.420)--(3.464,3.475)--(3.471,3.502)--(3.477,3.735)--(3.483,3.502)%
  --(3.490,3.393)--(3.496,3.393)--(3.503,3.461)--(3.509,3.393)--(3.515,3.379)--(3.522,3.406)%
  --(3.528,3.393)--(3.535,3.571)--(3.541,3.393)--(3.547,3.379)--(3.554,3.598)--(3.560,3.434)%
  --(3.567,3.338)--(3.573,3.393)--(3.580,3.365)--(3.586,3.324)--(3.592,3.379)--(3.599,3.365)%
  --(3.605,3.365)--(3.612,3.393)--(3.618,3.529)--(3.624,3.215)--(3.631,3.365)--(3.637,3.338)%
  --(3.644,3.434)--(3.650,3.379)--(3.656,3.529)--(3.663,3.283)--(3.669,3.379)--(3.676,3.215)%
  --(3.682,3.187)--(3.689,3.215)--(3.695,3.229)--(3.701,3.256)--(3.708,3.119)--(3.714,3.092)%
  --(3.721,3.092)--(3.727,3.092)--(3.733,3.133)--(3.740,3.174)--(3.746,3.160)--(3.753,3.160)%
  --(3.759,3.146)--(3.765,3.146)--(3.772,3.187)--(3.778,3.352)--(3.785,3.174)--(3.791,3.023)%
  --(3.798,3.160)--(3.804,2.996)--(3.810,3.051)--(3.817,3.324)--(3.823,3.064)--(3.830,3.051)%
  --(3.836,2.914)--(3.842,3.297)--(3.849,3.023)--(3.855,3.010)--(3.862,2.777)--(3.868,2.818)%
  --(3.874,3.119)--(3.881,3.051)--(3.887,2.969)--(3.894,2.982)--(3.900,3.023)--(3.907,3.010)%
  --(3.913,3.010)--(3.919,2.900)--(3.926,3.324)--(3.932,2.982)--(3.939,2.928)--(3.945,2.941)%
  --(3.951,2.941)--(3.958,2.900)--(3.964,3.146)--(3.971,3.338)--(3.977,3.447)--(3.983,2.928)%
  --(3.990,2.804)--(3.996,3.023)--(4.003,2.969)--(4.009,2.955)--(4.016,2.804)--(4.022,2.969)%
  --(4.028,3.051)--(4.035,2.845)--(4.041,2.845)--(4.048,3.215)--(4.054,2.996)--(4.060,2.832)%
  --(4.067,3.010)--(4.073,2.914)--(4.080,2.900)--(4.086,3.201)--(4.092,3.352)--(4.099,3.338)%
  --(4.105,3.324)--(4.112,3.229)--(4.118,2.969)--(4.125,2.832)--(4.131,2.736)--(4.137,2.777)%
  --(4.144,3.023)--(4.150,3.297)--(4.157,3.283)--(4.163,3.283)--(4.169,3.283)--(4.176,2.791)%
  --(4.182,2.804)--(4.189,2.736)--(4.195,2.736)--(4.201,2.955)--(4.208,2.695)--(4.214,2.709)%
  --(4.221,2.736)--(4.227,3.037)--(4.234,3.037)--(4.240,2.777)--(4.246,2.777)--(4.253,2.722)%
  --(4.259,2.763)--(4.266,2.914)--(4.272,3.229)--(4.278,3.201)--(4.285,3.229)--(4.291,3.229)%
  --(4.298,2.695)--(4.304,2.859)--(4.310,2.845)--(4.317,2.695)--(4.323,2.832)--(4.330,2.695)%
  --(4.336,2.668)--(4.343,2.887)--(4.349,2.900)--(4.355,2.832)--(4.362,2.709)--(4.368,2.695)%
  --(4.375,2.695)--(4.381,2.900)--(4.387,2.969)--(4.394,3.174)--(4.400,3.160)--(4.407,3.160)%
  --(4.413,3.105)--(4.419,2.668)--(4.426,2.722)--(4.432,2.517)--(4.439,2.695)--(4.445,2.695)%
  --(4.452,2.668)--(4.458,2.709)--(4.464,3.064)--(4.471,3.105)--(4.477,2.845)--(4.484,2.695)%
  --(4.490,2.791)--(4.496,2.640)--(4.503,2.695)--(4.509,2.791)--(4.516,2.695)--(4.522,2.695)%
  --(4.528,2.654)--(4.535,3.146)--(4.541,2.654)--(4.548,2.668)--(4.554,2.681)--(4.561,2.681)%
  --(4.567,2.804)--(4.573,3.105)--(4.580,3.119)--(4.586,3.105)--(4.593,3.105)--(4.599,2.845)%
  --(4.605,2.681)--(4.612,2.627)--(4.618,2.695)--(4.625,2.586)--(4.631,2.627)--(4.637,2.654)%
  --(4.644,2.627)--(4.650,2.558)--(4.657,2.996)--(4.663,2.695)--(4.670,2.736)--(4.676,2.695)%
  --(4.682,2.668)--(4.689,2.640)--(4.695,3.037)--(4.702,3.064)--(4.708,3.078)--(4.714,3.064)%
  --(4.721,2.845)--(4.727,2.804)--(4.734,2.503)--(4.740,2.517)--(4.746,2.545)--(4.753,2.545)%
  --(4.759,2.503)--(4.766,2.545)--(4.772,3.051)--(4.779,3.037)--(4.785,2.763)--(4.791,2.709)%
  --(4.798,2.654)--(4.804,2.312)--(4.811,2.586)--(4.817,2.572)--(4.823,2.517)--(4.830,2.517)%
  --(4.836,2.777)--(4.843,2.832)--(4.849,2.558)--(4.855,2.613)--(4.862,2.681)--(4.868,2.558)%
  --(4.875,2.640)--(4.881,3.023)--(4.888,3.010)--(4.894,3.010)--(4.900,2.955)--(4.907,2.476)%
  --(4.913,2.531)--(4.920,2.517)--(4.926,2.408)--(4.932,2.462)--(4.939,2.476)--(4.945,2.490)%
  --(4.952,2.490)--(4.958,2.668)--(4.964,2.654)--(4.971,2.531)--(4.977,2.545)--(4.984,2.490)%
  --(4.990,2.558)--(4.997,2.681)--(5.003,2.955)--(5.009,2.941)--(5.016,2.941)--(5.022,2.845)%
  --(5.029,2.627)--(5.035,2.572)--(5.041,2.517)--(5.048,2.517)--(5.054,2.558)--(5.061,2.900)%
  --(5.067,2.914)--(5.073,2.941)--(5.080,2.928)--(5.086,2.545)--(5.093,2.545)--(5.099,2.545)%
  --(5.106,2.435)--(5.112,2.572)--(5.118,2.640)--(5.125,2.353)--(5.131,2.558)--(5.138,2.640)%
  --(5.144,2.804)--(5.150,2.449)--(5.157,2.421)--(5.163,2.421)--(5.170,2.476)--(5.176,2.640)%
  --(5.182,2.859)--(5.189,2.859)--(5.195,2.887)--(5.202,2.804)--(5.208,2.339)--(5.215,2.380)%
  --(5.221,2.394)--(5.227,2.285)--(5.234,2.339)--(5.240,2.421)--(5.247,2.503)--(5.253,2.421)%
  --(5.259,2.367)--(5.266,2.668)--(5.272,2.599)--(5.279,2.435)--(5.285,2.599)--(5.291,2.380)%
  --(5.298,2.462)--(5.304,2.818)--(5.311,2.791)--(5.317,2.763)--(5.324,2.804)--(5.330,2.380)%
  --(5.336,2.408)--(5.343,2.586)--(5.349,2.367)--(5.356,2.367)--(5.362,2.722)--(5.368,2.845)%
  --(5.375,2.326)--(5.381,2.380)--(5.388,2.517)--(5.394,2.503)--(5.400,2.558)--(5.407,2.326)%
  --(5.413,2.244)--(5.420,2.285)--(5.426,2.367)--(5.433,2.230)--(5.439,2.490)--(5.445,2.668)%
  --(5.452,2.449)--(5.458,2.503)--(5.465,2.449)--(5.471,2.312)--(5.477,2.408)--(5.484,2.627)%
  --(5.490,2.476)--(5.497,2.257)--(5.503,2.052)--(5.509,2.216)--(5.516,2.394)--(5.522,2.298)%
  --(5.529,2.298)--(5.535,2.298)--(5.542,2.203)--(5.548,2.189)--(5.554,2.244)--(5.561,2.230)%
  --(5.567,2.613)--(5.574,2.189)--(5.580,2.285)--(5.586,2.285)--(5.593,2.298)--(5.599,2.476)%
  --(5.606,2.627)--(5.612,2.408)--(5.618,2.148)--(5.625,2.421)--(5.631,2.285)--(5.638,2.517)%
  --(5.644,2.353)--(5.651,2.244)--(5.657,2.175)--(5.663,2.271)--(5.670,2.216)--(5.676,2.257)%
  --(5.683,2.654)--(5.689,2.189)--(5.695,2.257)--(5.702,2.230)--(5.708,2.326)--(5.715,2.394)%
  --(5.721,2.462)--(5.727,2.298)--(5.734,2.462)--(5.740,2.244)--(5.747,2.490)--(5.753,2.271)%
  --(5.760,2.189)--(5.766,2.271)--(5.772,2.079)--(5.779,2.175)--(5.785,2.312)--(5.792,2.312)%
  --(5.798,2.175)--(5.804,2.093)--(5.811,2.107)--(5.817,2.380)--(5.824,2.175)--(5.830,2.148)%
  --(5.836,2.038)--(5.843,2.066)--(5.849,2.120)--(5.856,2.107)--(5.862,2.244)--(5.869,2.339)%
  --(5.875,2.230)--(5.881,2.203)--(5.888,2.462)--(5.894,2.175)--(5.901,2.257)--(5.907,2.435)%
  --(5.913,2.134)--(5.920,2.066)--(5.926,2.203)--(5.933,2.161)--(5.939,2.285)--(5.945,2.175)%
  --(5.952,2.148)--(5.958,2.134)--(5.965,2.271)--(5.971,2.326)--(5.978,2.038)--(5.984,2.093)%
  --(5.990,2.025)--(5.997,2.107)--(6.003,2.189)--(6.010,2.257)--(6.016,2.107)--(6.022,2.271)%
  --(6.029,2.353)--(6.035,2.257)--(6.042,2.120)--(6.048,2.394)--(6.054,2.380)--(6.061,2.093)%
  --(6.067,2.161)--(6.074,2.189)--(6.080,2.134)--(6.087,2.408)--(6.093,2.599)--(6.099,2.572)%
  --(6.106,2.586)--(6.112,2.599)--(6.119,2.175)--(6.125,2.107)--(6.131,2.175)--(6.138,2.079)%
  --(6.144,2.161)--(6.151,2.052)--(6.157,2.148)--(6.163,2.120)--(6.170,2.312)--(6.176,2.271)%
  --(6.183,2.079)--(6.189,2.093)--(6.196,2.107)--(6.202,2.312)--(6.208,2.285)--(6.215,2.599)%
  --(6.221,2.599)--(6.228,2.558)--(6.234,2.572)--(6.240,2.148)--(6.247,2.148)--(6.253,2.052)%
  --(6.260,2.052)--(6.266,2.079)--(6.272,2.599)--(6.279,2.627)--(6.285,2.599)--(6.292,2.572)%
  --(6.298,2.257)--(6.305,2.038)--(6.311,2.066)--(6.317,2.052)--(6.324,2.148)--(6.330,2.134)%
  --(6.337,2.148)--(6.343,2.052)--(6.349,2.285)--(6.356,2.558)--(6.362,2.107)--(6.369,2.066)%
  --(6.375,2.038)--(6.381,2.052)--(6.388,2.134)--(6.394,2.545)--(6.401,2.586)--(6.407,2.545)%
  --(6.414,2.545)--(6.420,2.367)--(6.426,2.148)--(6.433,2.093)--(6.439,2.052)--(6.446,1.970)%
  --(6.452,1.997)--(6.458,1.997)--(6.465,1.997)--(6.471,1.997)--(6.478,2.517)--(6.484,2.011)%
  --(6.490,1.997)--(6.497,2.079)--(6.503,2.107)--(6.510,1.997)--(6.516,2.353)--(6.523,2.572)%
  --(6.529,2.558)--(6.535,2.558)--(6.542,2.271)--(6.548,1.997)--(6.555,1.956)--(6.561,1.997)%
  --(6.567,2.079)--(6.574,2.175)--(6.580,2.203)--(6.587,2.257)--(6.593,2.531)--(6.599,2.572)%
  --(6.606,2.134)--(6.612,2.161)--(6.619,2.134)--(6.625,2.175)--(6.632,2.285)--(6.638,2.093)%
  --(6.644,2.107)--(6.651,2.244)--(6.657,2.394)--(6.664,2.326)--(6.670,2.038)--(6.676,2.148)%
  --(6.683,2.408)--(6.689,2.230)--(6.696,2.367)--(6.702,2.503)--(6.708,2.531)--(6.715,2.531)%
  --(6.721,2.503)--(6.728,2.312)--(6.734,2.161)--(6.741,2.244)--(6.747,1.997)--(6.753,2.038)%
  --(6.760,2.038)--(6.766,2.025)--(6.773,2.093)--(6.779,2.257)--(6.785,2.216)--(6.792,1.997)%
  --(6.798,2.148)--(6.805,2.120)--(6.811,2.011)--(6.818,2.353)--(6.824,2.449)--(6.830,2.503)%
  --(6.837,2.462)--(6.843,2.394)--(6.850,2.148)--(6.856,2.079)--(6.862,1.888)--(6.869,2.011)%
  --(6.875,2.011)--(6.882,2.066)--(6.888,2.025)--(6.894,1.997)--(6.901,2.066)--(6.907,2.011)%
  --(6.914,2.038)--(6.920,2.038)--(6.927,2.025)--(6.933,2.011)--(6.939,1.997)--(6.946,2.025)%
  --(6.952,2.038)--(6.959,2.244)--(6.965,2.353)--(6.971,2.066)--(6.978,2.079)--(6.984,2.107)%
  --(6.991,2.093)--(6.997,2.052)--(7.003,2.339)--(7.010,2.408)--(7.016,2.408)--(7.023,2.394)%
  --(7.029,2.148)--(7.036,2.339)--(7.042,2.380)--(7.048,2.189)--(7.055,2.203)--(7.061,2.367)%
  --(7.068,2.339)--(7.074,2.380)--(7.080,2.339)--(7.087,2.353)--(7.093,2.161)--(7.100,2.367)%
  --(7.106,2.339)--(7.112,2.339)--(7.119,2.353)--(7.125,2.339)--(7.132,2.298)--(7.138,2.326)%
  --(7.145,2.285)--(7.151,2.052)--(7.157,2.120)--(7.164,2.079)--(7.170,2.148)--(7.177,1.819)%
  --(7.183,1.902)--(7.189,1.806)--(7.196,1.902)--(7.202,2.298)--(7.209,2.298)--(7.215,1.833)%
  --(7.221,1.696)--(7.228,1.888)--(7.234,1.628)--(7.241,1.861)--(7.247,1.861)--(7.254,1.847)%
  --(7.260,1.833)--(7.266,2.107)--(7.273,1.984)--(7.279,1.792)--(7.286,1.997)--(7.292,1.833)%
  --(7.298,1.847)--(7.305,1.929)--(7.311,2.216)--(7.318,2.271)--(7.324,2.189)--(7.330,2.203)%
  --(7.337,1.861)--(7.343,1.792)--(7.350,1.792)--(7.356,1.628)--(7.363,1.696)--(7.369,1.655)%
  --(7.375,1.669)--(7.382,1.915)--(7.388,1.861)--(7.395,1.902)--(7.401,1.696)--(7.407,1.902)%
  --(7.414,1.943)--(7.420,1.806)--(7.427,1.929)--(7.433,2.175)--(7.439,2.161)--(7.446,2.203)%
  --(7.452,2.161)--(7.459,1.929)--(7.465,1.915)--(7.472,1.943)--(7.478,1.984)--(7.484,1.929)%
  --(7.491,2.107)--(7.497,2.175)--(7.504,2.175)--(7.510,1.847)--(7.516,1.696)--(7.523,1.751)%
  --(7.529,1.861)--(7.536,1.806)--(7.542,1.778)--(7.548,1.655)--(7.555,1.861)--(7.561,1.696)%
  --(7.568,1.943)--(7.574,1.997)--(7.581,1.765)--(7.587,1.737)--(7.593,1.724)--(7.600,1.943)%
  --(7.606,1.778)--(7.613,1.997)--(7.619,1.710)--(7.625,1.943)--(7.632,1.847)--(7.638,1.915)%
  --(7.645,1.956)--(7.651,1.929)--(7.657,1.874)--(7.664,1.833)--(7.670,1.861)--(7.677,1.861)%
  --(7.683,1.902)--(7.690,1.819)--(7.696,2.038)--(7.702,1.902)--(7.709,1.929)--(7.715,1.984)%
  --(7.722,1.888)--(7.728,1.874)--(7.734,1.874)--(7.741,1.806)--(7.747,1.929)--(7.754,1.943)%
  --(7.760,1.833)--(7.766,1.806)--(7.773,1.669)--(7.779,1.819)--(7.786,1.792)--(7.792,1.724)%
  --(7.799,1.806)--(7.805,2.093)--(7.811,1.847)--(7.818,1.464)--(7.824,1.505)--(7.831,1.819)%
  --(7.837,1.819)--(7.843,1.710)--(7.850,1.806)--(7.856,1.669)--(7.863,1.737)--(7.869,1.861)%
  --(7.875,2.107)--(7.882,1.874)--(7.888,1.751)--(7.895,1.833)--(7.901,1.833)--(7.908,1.915)%
  --(7.914,1.943)--(7.920,1.819)--(7.927,1.751)--(7.933,1.642)--(7.940,1.669)--(7.946,1.847)%
  --(7.952,1.765)--(7.959,1.778)--(7.965,1.683)--(7.972,1.847)--(7.978,1.669)--(7.984,1.778)%
  --(7.991,1.765)--(7.997,1.751)--(8.004,1.902)--(8.010,1.655)--(8.017,1.997)--(8.023,1.861)%
  --(8.029,1.888)--(8.036,2.093)--(8.042,1.450)--(8.049,1.792)--(8.055,1.984)--(8.061,1.778)%
  --(8.068,1.943)--(8.074,1.997)--(8.081,1.833)--(8.087,1.833)--(8.093,1.902)--(8.100,1.888)%
  --(8.106,1.765)--(8.113,1.819)--(8.119,1.929)--(8.126,1.833)--(8.132,1.847)--(8.138,1.970)%
  --(8.145,1.833)--(8.151,1.915)--(8.158,1.874)--(8.164,1.902)--(8.170,1.997)--(8.177,2.093)%
  --(8.183,2.093)--(8.190,1.874)--(8.196,1.915)--(8.202,1.956)--(8.209,1.956)--(8.215,2.093)%
  --(8.222,2.326)--(8.228,2.326)--(8.235,2.353)--(8.241,2.312)--(8.247,1.970)--(8.254,2.025)%
  --(8.260,1.956)--(8.267,1.806)--(8.273,1.915)--(8.279,1.997)--(8.286,1.888)--(8.292,1.943)%
  --(8.299,2.079)--(8.305,2.230)--(8.311,1.997)--(8.318,2.052)--(8.324,2.079)--(8.331,1.997)%
  --(8.337,2.244)--(8.344,2.380)--(8.350,2.421)--(8.356,2.339)--(8.363,2.339)--(8.369,2.079)%
  --(8.376,1.997)--(8.382,1.997)--(8.388,2.079)--(8.395,1.997)--(8.401,2.298)--(8.408,2.449)%
  --(8.414,2.408)--(8.420,2.462)--(8.427,2.093)--(8.433,2.025)--(8.440,2.120)--(8.446,1.997)%
  --(8.453,2.093)--(8.459,1.997)--(8.465,1.997)--(8.472,1.997)--(8.478,2.134)--(8.485,2.380)%
  --(8.491,1.806)--(8.497,1.997)--(8.504,1.997)--(8.510,1.997)--(8.517,1.997)--(8.523,2.216)%
  --(8.529,2.476)--(8.536,2.517)--(8.542,2.503)--(8.549,2.052)--(8.555,2.271)--(8.562,1.997)%
  --(8.568,1.943)--(8.574,1.997)--(8.581,1.997)--(8.587,2.038)--(8.594,1.997)--(8.600,2.175)%
  --(8.606,2.367)--(8.613,2.011)--(8.619,2.011)--(8.626,2.271)--(8.632,2.161)--(8.638,2.189)%
  --(8.645,2.545)--(8.651,2.531)--(8.658,2.558)--(8.664,2.531)--(8.671,2.339)--(8.677,2.066)%
  --(8.683,2.011)--(8.690,2.011)--(8.696,2.093)--(8.703,2.134)--(8.709,2.038)--(8.715,2.449)%
  --(8.722,2.531)--(8.728,2.257)--(8.735,2.093)--(8.741,2.271)--(8.747,2.298)--(8.754,2.148)%
  --(8.760,2.134)--(8.767,2.312)--(8.773,2.161)--(8.780,2.271)--(8.786,2.586)--(8.792,2.079)%
  --(8.799,2.066)--(8.805,2.285)--(8.812,2.148)--(8.818,2.093)--(8.824,2.421)--(8.831,2.640)%
  --(8.837,2.613)--(8.844,2.613)--(8.850,2.435)--(8.856,2.120)--(8.863,2.257)--(8.869,2.134)%
  --(8.876,2.025)--(8.882,2.120)--(8.889,2.066)--(8.895,2.079)--(8.901,2.011)--(8.908,2.531)%
  --(8.914,2.203)--(8.921,2.394)--(8.927,2.244)--(8.933,2.189)--(8.940,2.189)--(8.946,2.586)%
  --(8.953,2.627)--(8.959,2.572)--(8.965,2.681)--(8.972,2.586)--(8.978,2.353)--(8.985,2.394)%
  --(8.991,2.134)--(8.998,2.271)--(9.004,2.093)--(9.010,2.230)--(9.017,2.203)--(9.023,2.681)%
  --(9.030,2.709)--(9.036,2.435)--(9.042,2.244)--(9.049,2.285)--(9.055,2.189)--(9.062,1.997)%
  --(9.068,2.298)--(9.074,2.244)--(9.081,2.203)--(9.087,2.285)--(9.094,2.640)--(9.100,2.476)%
  --(9.107,2.312)--(9.113,2.271)--(9.119,2.271)--(9.126,2.490)--(9.132,2.709)--(9.139,2.736)%
  --(9.145,2.763)--(9.151,2.709)--(9.158,2.517)--(9.164,2.298)--(9.171,2.216)--(9.177,2.353)%
  --(9.183,2.134)--(9.190,2.216)--(9.196,2.216)--(9.203,2.203)--(9.209,2.271)--(9.216,2.777)%
  --(9.222,2.367)--(9.228,2.326)--(9.235,2.367)--(9.241,2.394)--(9.248,2.312)--(9.254,2.804)%
  --(9.260,2.777)--(9.267,2.818)--(9.273,2.818)--(9.280,2.599)--(9.286,2.394)--(9.292,2.367)%
  --(9.299,2.380)--(9.305,2.353)--(9.312,2.709)--(9.318,2.845)--(9.325,2.763)--(9.331,2.326)%
  --(9.337,2.421)--(9.344,2.421)--(9.350,2.394)--(9.357,2.394)--(9.363,2.271)--(9.369,2.586)%
  --(9.376,2.367)--(9.382,2.558)--(9.389,2.558)--(9.395,2.668)--(9.401,2.353)--(9.408,2.627)%
  --(9.414,2.462)--(9.421,2.503)--(9.427,2.449)--(9.434,2.709)--(9.440,2.326)--(9.446,2.353)%
  --(9.453,2.339)--(9.459,2.326)--(9.466,2.503)--(9.472,2.462)--(9.478,2.462)--(9.485,2.244)%
  --(9.491,2.339)--(9.498,2.435)--(9.504,2.367)--(9.510,2.599)--(9.517,2.750)--(9.523,2.408)%
  --(9.530,2.476)--(9.536,2.681)--(9.543,2.777)--(9.549,2.736)--(9.555,2.887)--(9.562,2.804)%
  --(9.568,2.873)--(9.575,2.476)--(9.581,2.449)--(9.587,2.531)--(9.594,2.709)--(9.600,2.203)%
  --(9.607,2.230)--(9.613,2.640)--(9.619,2.503)--(9.626,2.421)--(9.632,2.476)--(9.639,2.572)%
  --(9.645,1.642)--(9.652,1.737)--(9.658,1.204)--(9.664,1.655)--(9.671,1.737)--(9.677,2.107)%
  --(9.684,2.586)--(9.690,2.654)--(9.696,2.654)--(9.703,2.449)--(9.709,2.517)--(9.716,2.613)%
  --(9.722,2.545)--(9.728,2.572)--(9.735,2.709)--(9.741,2.818)--(9.748,2.531)--(9.754,2.517)%
  --(9.761,2.490)--(9.767,2.586)--(9.773,2.640)--(9.780,2.572)--(9.786,2.572)--(9.793,2.531)%
  --(9.799,2.503)--(9.805,2.449)--(9.812,2.476)--(9.818,2.517)--(9.825,2.736)--(9.831,2.586)%
  --(9.837,2.531)--(9.844,2.586)--(9.850,2.627)--(9.857,2.531)--(9.863,2.586)--(9.870,2.503)%
  --(9.876,2.613)--(9.882,2.545)--(9.889,2.503)--(9.895,2.654)--(9.902,2.421)--(9.908,2.586)%
  --(9.914,2.558)--(9.921,2.545)--(9.927,2.572)--(9.934,2.640)--(9.940,2.955)--(9.946,2.928)%
  --(9.953,2.941)--(9.959,3.037)--(9.966,2.955)--(9.972,3.023)--(9.979,3.023)--(9.985,3.037)%
  --(9.991,3.037)--(9.998,3.010)--(10.004,3.010)--(10.011,2.763)--(10.017,3.051)--(10.023,3.023)%
  --(10.030,3.092)--(10.036,2.996)--(10.043,2.722)--(10.049,3.064)--(10.055,3.023)--(10.062,3.023)%
  --(10.068,2.763)--(10.075,2.955)--(10.081,2.654)--(10.088,2.476)--(10.094,2.640)--(10.100,2.668)%
  --(10.107,2.668)--(10.113,2.627)--(10.120,2.627)--(10.126,2.873)--(10.132,2.654)--(10.139,2.640)%
  --(10.145,2.681)--(10.152,2.586)--(10.158,2.654)--(10.164,2.736)--(10.171,2.640)--(10.177,2.695)%
  --(10.184,2.640)--(10.190,2.449)--(10.197,2.462)--(10.203,2.545)--(10.209,2.654)--(10.216,2.668)%
  --(10.222,2.941)--(10.229,2.572)--(10.235,2.531)--(10.241,2.668)--(10.248,2.736)--(10.254,2.681)%
  --(10.261,2.804)--(10.267,2.640)--(10.274,2.654)--(10.280,2.681)--(10.286,2.627)--(10.293,2.681)%
  --(10.299,2.613)--(10.306,3.037)--(10.312,2.613)--(10.318,2.627)--(10.325,2.613)--(10.331,2.709)%
  --(10.338,2.627)--(10.344,3.037)--(10.350,3.051)--(10.357,3.051)--(10.363,3.064)--(10.370,2.722)%
  --(10.376,2.654)--(10.383,2.695)--(10.389,2.668)--(10.395,2.545)--(10.402,2.640)--(10.408,2.503)%
  --(10.415,2.572)--(10.421,2.503)--(10.427,2.832)--(10.434,2.695)--(10.440,2.654)--(10.447,2.695)%
  --(10.453,2.695)--(10.459,2.709)--(10.466,3.051)--(10.472,3.078)--(10.479,3.078)--(10.485,3.051)%
  --(10.492,2.777)--(10.498,2.654)--(10.504,2.558)--(10.511,2.681)--(10.517,2.763)--(10.524,2.681)%
  --(10.530,2.695)--(10.536,2.640)--(10.543,3.078)--(10.549,2.887)--(10.556,2.722)--(10.562,2.627)%
  --(10.568,2.654)--(10.575,2.599)--(10.581,2.654)--(10.588,2.654)--(10.594,2.586)--(10.601,2.586)%
  --(10.607,3.023)--(10.613,2.613)--(10.620,2.627)--(10.626,2.668)--(10.633,2.654)--(10.639,2.627)%
  --(10.645,2.982)--(10.652,3.078)--(10.658,3.078)--(10.665,3.078)--(10.671,2.941)--(10.677,2.627)%
  --(10.684,2.763)--(10.690,2.599)--(10.697,2.462)--(10.703,2.613)--(10.710,2.654)--(10.716,2.545)%
  --(10.722,2.668)--(10.729,3.037)--(10.735,2.681)--(10.742,2.695)--(10.748,2.695)--(10.754,2.709)%
  --(10.761,2.613)--(10.767,3.105)--(10.774,3.023)--(10.780,3.105)--(10.786,3.078)--(10.793,3.010)%
  --(10.799,2.736)--(10.806,2.613)--(10.812,2.531)--(10.819,2.586)--(10.825,2.640)--(10.831,2.613)%
  --(10.838,2.640)--(10.844,2.845)--(10.851,3.078)--(10.857,2.750)--(10.863,2.668)--(10.870,2.654)%
  --(10.876,2.572)--(10.883,2.695)--(10.889,2.695)--(10.895,2.668)--(10.902,2.681)--(10.908,2.900)%
  --(10.915,2.873)--(10.921,2.695)--(10.928,2.695)--(10.934,2.763)--(10.940,2.695)--(10.947,2.941)%
  --(10.953,3.133)--(10.960,3.092)--(10.966,3.119)--(10.972,3.133)--(10.979,2.791)--(10.985,2.750)%
  --(10.992,2.709)--(10.998,2.517)--(11.004,2.545)--(11.011,2.668)--(11.017,2.736)--(11.024,2.654)%
  --(11.030,2.887)--(11.037,2.818)--(11.043,2.777)--(11.049,2.695)--(11.056,2.695)--(11.062,2.695)%
  --(11.069,2.969)--(11.075,3.174)--(11.081,3.174)--(11.088,3.146)--(11.094,3.037)--(11.101,2.695)%
  --(11.107,2.640)--(11.113,2.627)--(11.120,2.695)--(11.126,2.668)--(11.133,2.681)--(11.139,2.750)%
  --(11.146,3.146)--(11.152,3.160)--(11.158,2.695)--(11.165,2.695)--(11.171,2.695)--(11.178,2.681)%
  --(11.184,2.709)--(11.190,2.695)--(11.197,2.695)--(11.203,2.709)--(11.210,3.010)--(11.216,2.914)%
  --(11.222,2.736)--(11.229,2.695)--(11.235,2.695)--(11.242,2.804)--(11.248,2.982)--(11.255,3.174)%
  --(11.261,3.201)--(11.267,3.160)--(11.274,3.146)--(11.280,2.763)--(11.287,2.777)--(11.293,2.709)%
  --(11.299,2.640)--(11.306,2.531)--(11.312,2.531)--(11.319,2.668)--(11.325,2.695)--(11.331,2.832)%
  --(11.338,2.982)--(11.344,2.695)--(11.351,2.804)--(11.357,2.640)--(11.364,2.668)--(11.370,2.873)%
  --(11.376,3.174)--(11.383,3.215)--(11.389,3.187)--(11.396,3.215)--(11.402,2.873)--(11.408,2.722)%
  --(11.415,2.736)--(11.421,2.695)--(11.428,2.695)--(11.434,2.695)--(11.440,2.695)--(11.447,3.229)%
  --(11.453,3.242)--(11.460,3.064)--(11.466,2.695)--(11.473,2.777)--(11.479,2.777)--(11.485,2.845)%
  --(11.492,2.887)--(11.498,2.955);
\gpcolor{color=gp lt color border}
\gpsetlinetype{gp lt border}
\draw[gp path] (1.688,7.182)--(1.688,1.204);
\draw[gp path] (1.688,0.985)--(11.498,0.985);
%% coordinates of the plot area
\gpdefrectangularnode{gp plot 1}{\pgfpoint{1.688cm}{0.985cm}}{\pgfpoint{11.947cm}{7.825cm}}
\end{tikzpicture}
%% gnuplot variables

\caption{Variazione della temperatura, esempio 1}
\label{img:isoa}
\end{grafico}

\begin{grafico}
  \centering
\begin{tikzpicture}[gnuplot]
%% generated with GNUPLOT 4.6p3 (Lua 5.1; terminal rev. 99, script rev. 100)
%% mar 27 mag 2014 22:34:40 CEST
\path (0.000,0.000) rectangle (12.500,8.750);
\gpcolor{color=gp lt color border}
\gpsetlinetype{gp lt border}
\gpsetlinewidth{1.00}
\draw[gp path] (1.872,2.353)--(2.052,2.353);
\node[gp node right] at (1.688,2.353) { 15.4};
\draw[gp path] (1.872,3.721)--(2.052,3.721);
\node[gp node right] at (1.688,3.721) { 15.45};
\draw[gp path] (1.872,5.089)--(2.052,5.089);
\node[gp node right] at (1.688,5.089) { 15.5};
\draw[gp path] (1.872,6.457)--(2.052,6.457);
\node[gp node right] at (1.688,6.457) { 15.55};
\draw[gp path] (1.872,0.985)--(1.872,1.165);
\node[gp node center] at (1.872,0.677) { 0};
\draw[gp path] (2.880,0.985)--(2.880,1.165);
\node[gp node center] at (2.880,0.677) { 200};
\draw[gp path] (3.887,0.985)--(3.887,1.165);
\node[gp node center] at (3.887,0.677) { 400};
\draw[gp path] (4.895,0.985)--(4.895,1.165);
\node[gp node center] at (4.895,0.677) { 600};
\draw[gp path] (5.902,0.985)--(5.902,1.165);
\node[gp node center] at (5.902,0.677) { 800};
\draw[gp path] (6.910,0.985)--(6.910,1.165);
\node[gp node center] at (6.910,0.677) { 1000};
\draw[gp path] (7.917,0.985)--(7.917,1.165);
\node[gp node center] at (7.917,0.677) { 1200};
\draw[gp path] (8.925,0.985)--(8.925,1.165);
\node[gp node center] at (8.925,0.677) { 1400};
\draw[gp path] (9.932,0.985)--(9.932,1.165);
\node[gp node center] at (9.932,0.677) { 1600};
\draw[gp path] (10.940,0.985)--(10.940,1.165);
\node[gp node center] at (10.940,0.677) { 1800};
\draw[gp path] (1.872,7.716)--(1.872,1.970);
\draw[gp path] (1.872,0.985)--(11.005,0.985);
\node[gp node center,rotate=-270] at (0.246,4.405) {Temperatura $[\gradi C]$};
\node[gp node center] at (6.909,0.215) {Numero della misura};
\node[gp node center] at (6.909,8.287) {Terza Isoterma andata};
\gpcolor{color=gp lt color 0}
\gpsetlinetype{gp lt plot 0}
\draw[gp path] (1.872,1.970)--(1.877,2.079)--(1.882,2.052)--(1.887,1.997)--(1.892,2.052)%
  --(1.897,2.052)--(1.902,2.052)--(1.907,2.052)--(1.912,2.134)--(1.917,2.134)--(1.922,2.134)%
  --(1.927,2.161)--(1.932,2.134)--(1.937,2.134)--(1.943,2.216)--(1.948,2.134)--(1.953,2.107)%
  --(1.958,2.134)--(1.963,2.216)--(1.968,2.189)--(1.973,2.189)--(1.978,2.161)--(1.983,2.189)%
  --(1.988,2.244)--(1.993,2.271)--(1.998,2.271)--(2.003,2.271)--(2.008,2.244)--(2.013,2.271)%
  --(2.018,2.298)--(2.023,2.271)--(2.028,2.298)--(2.033,2.298)--(2.038,2.271)--(2.043,2.298)%
  --(2.048,2.353)--(2.053,2.380)--(2.058,2.271)--(2.063,2.353)--(2.068,2.353)--(2.074,2.353)%
  --(2.079,2.408)--(2.084,2.380)--(2.089,2.408)--(2.094,2.380)--(2.099,2.408)--(2.104,2.462)%
  --(2.109,2.380)--(2.114,2.408)--(2.119,2.408)--(2.124,2.435)--(2.129,2.408)--(2.134,2.490)%
  --(2.139,2.435)--(2.144,2.408)--(2.149,2.462)--(2.154,2.462)--(2.159,2.490)--(2.164,2.408)%
  --(2.169,2.462)--(2.174,2.435)--(2.179,2.490)--(2.184,2.490)--(2.189,2.545)--(2.194,2.490)%
  --(2.199,2.517)--(2.204,2.545)--(2.210,2.517)--(2.215,2.517)--(2.220,2.490)--(2.225,2.545)%
  --(2.230,2.572)--(2.235,2.545)--(2.240,2.545)--(2.245,2.599)--(2.250,2.572)--(2.255,2.572)%
  --(2.260,2.599)--(2.265,2.599)--(2.270,2.545)--(2.275,2.599)--(2.280,2.627)--(2.285,2.599)%
  --(2.290,2.654)--(2.295,2.627)--(2.300,2.627)--(2.305,2.599)--(2.310,2.654)--(2.315,2.627)%
  --(2.320,2.681)--(2.325,2.599)--(2.330,2.627)--(2.335,2.572)--(2.340,2.681)--(2.346,2.627)%
  --(2.351,2.654)--(2.356,2.599)--(2.361,2.627)--(2.366,2.599)--(2.371,2.654)--(2.376,2.681)%
  --(2.381,2.709)--(2.386,2.681)--(2.391,2.681)--(2.396,2.627)--(2.401,2.627)--(2.406,2.627)%
  --(2.411,2.654)--(2.416,2.681)--(2.421,2.736)--(2.426,2.681)--(2.431,2.736)--(2.436,2.681)%
  --(2.441,2.736)--(2.446,2.572)--(2.451,2.627)--(2.456,2.654)--(2.461,2.763)--(2.466,2.681)%
  --(2.471,2.654)--(2.477,2.709)--(2.482,2.709)--(2.487,2.654)--(2.492,2.736)--(2.497,2.681)%
  --(2.502,2.763)--(2.507,2.709)--(2.512,2.736)--(2.517,2.736)--(2.522,2.736)--(2.527,2.791)%
  --(2.532,2.763)--(2.537,2.736)--(2.542,2.736)--(2.547,2.763)--(2.552,2.791)--(2.557,2.763)%
  --(2.562,2.736)--(2.567,2.763)--(2.572,2.791)--(2.577,2.763)--(2.582,2.845)--(2.587,2.791)%
  --(2.592,2.845)--(2.597,2.845)--(2.602,2.845)--(2.607,2.791)--(2.613,2.791)--(2.618,2.845)%
  --(2.623,2.818)--(2.628,2.818)--(2.633,2.818)--(2.638,2.845)--(2.643,2.791)--(2.648,2.873)%
  --(2.653,2.845)--(2.658,2.818)--(2.663,2.873)--(2.668,2.928)--(2.673,2.873)--(2.678,2.873)%
  --(2.683,2.791)--(2.688,2.928)--(2.693,2.791)--(2.698,2.791)--(2.703,2.900)--(2.708,2.873)%
  --(2.713,2.845)--(2.718,2.873)--(2.723,2.845)--(2.728,2.818)--(2.733,2.818)--(2.738,2.928)%
  --(2.743,2.818)--(2.749,2.900)--(2.754,2.818)--(2.759,2.900)--(2.764,2.955)--(2.769,2.928)%
  --(2.774,2.900)--(2.779,2.873)--(2.784,2.900)--(2.789,2.928)--(2.794,2.873)--(2.799,2.900)%
  --(2.804,2.900)--(2.809,2.873)--(2.814,2.900)--(2.819,2.873)--(2.824,2.982)--(2.829,2.873)%
  --(2.834,2.955)--(2.839,2.955)--(2.844,2.955)--(2.849,2.955)--(2.854,2.928)--(2.859,2.982)%
  --(2.864,2.955)--(2.869,2.900)--(2.874,2.900)--(2.880,3.037)--(2.885,2.982)--(2.890,3.010)%
  --(2.895,2.955)--(2.900,2.928)--(2.905,3.010)--(2.910,2.955)--(2.915,2.955)--(2.920,2.955)%
  --(2.925,3.064)--(2.930,3.010)--(2.935,2.955)--(2.940,2.982)--(2.945,3.010)--(2.950,2.982)%
  --(2.955,3.010)--(2.960,2.955)--(2.965,3.010)--(2.970,3.010)--(2.975,3.010)--(2.980,3.010)%
  --(2.985,2.982)--(2.990,3.037)--(2.995,2.982)--(3.000,3.064)--(3.005,3.064)--(3.010,3.010)%
  --(3.016,3.037)--(3.021,3.037)--(3.026,3.010)--(3.031,3.010)--(3.036,3.010)--(3.041,3.064)%
  --(3.046,2.982)--(3.051,3.010)--(3.056,3.064)--(3.061,3.037)--(3.066,3.119)--(3.071,3.064)%
  --(3.076,3.064)--(3.081,3.037)--(3.086,3.119)--(3.091,3.092)--(3.096,3.092)--(3.101,3.037)%
  --(3.106,3.092)--(3.111,3.119)--(3.116,3.010)--(3.121,3.146)--(3.126,3.174)--(3.131,3.146)%
  --(3.136,3.174)--(3.141,3.092)--(3.146,3.064)--(3.152,3.092)--(3.157,3.119)--(3.162,3.119)%
  --(3.167,3.119)--(3.172,3.174)--(3.177,3.174)--(3.182,3.174)--(3.187,3.092)--(3.192,3.119)%
  --(3.197,3.174)--(3.202,3.146)--(3.207,3.229)--(3.212,3.174)--(3.217,3.201)--(3.222,3.201)%
  --(3.227,3.146)--(3.232,3.174)--(3.237,3.174)--(3.242,3.119)--(3.247,3.146)--(3.252,3.229)%
  --(3.257,3.201)--(3.262,3.256)--(3.267,3.229)--(3.272,3.256)--(3.277,3.256)--(3.283,3.174)%
  --(3.288,3.283)--(3.293,3.256)--(3.298,3.283)--(3.303,3.229)--(3.308,3.283)--(3.313,3.338)%
  --(3.318,3.283)--(3.323,3.338)--(3.328,3.338)--(3.333,3.311)--(3.338,3.311)--(3.343,3.311)%
  --(3.348,3.283)--(3.353,3.311)--(3.358,3.311)--(3.363,3.338)--(3.368,3.365)--(3.373,3.365)%
  --(3.378,3.283)--(3.383,3.311)--(3.388,3.447)--(3.393,3.338)--(3.398,3.338)--(3.403,3.338)%
  --(3.408,3.393)--(3.413,3.365)--(3.419,3.420)--(3.424,3.393)--(3.429,3.393)--(3.434,3.338)%
  --(3.439,3.338)--(3.444,3.311)--(3.449,3.393)--(3.454,3.365)--(3.459,3.420)--(3.464,3.393)%
  --(3.469,3.365)--(3.474,3.420)--(3.479,3.447)--(3.484,3.447)--(3.489,3.447)--(3.494,3.420)%
  --(3.499,3.420)--(3.504,3.420)--(3.509,3.420)--(3.514,3.475)--(3.519,3.475)--(3.524,3.475)%
  --(3.529,3.502)--(3.534,3.557)--(3.539,3.557)--(3.544,3.557)--(3.549,3.557)--(3.555,3.502)%
  --(3.560,3.502)--(3.565,3.502)--(3.570,3.557)--(3.575,3.584)--(3.580,3.529)--(3.585,3.557)%
  --(3.590,3.584)--(3.595,3.584)--(3.600,3.557)--(3.605,3.639)--(3.610,3.584)--(3.615,3.584)%
  --(3.620,3.612)--(3.625,3.639)--(3.630,3.694)--(3.635,3.584)--(3.640,3.666)--(3.645,3.639)%
  --(3.650,3.666)--(3.655,3.666)--(3.660,3.721)--(3.665,3.694)--(3.670,3.694)--(3.675,3.694)%
  --(3.680,3.584)--(3.686,3.584)--(3.691,3.639)--(3.696,3.639)--(3.701,3.666)--(3.706,3.612)%
  --(3.711,3.666)--(3.716,3.666)--(3.721,3.666)--(3.726,3.694)--(3.731,3.666)--(3.736,3.666)%
  --(3.741,3.694)--(3.746,3.721)--(3.751,3.721)--(3.756,3.776)--(3.761,3.721)--(3.766,3.748)%
  --(3.771,3.721)--(3.776,3.748)--(3.781,3.776)--(3.786,3.776)--(3.791,3.748)--(3.796,3.748)%
  --(3.801,3.776)--(3.806,3.748)--(3.811,3.776)--(3.816,3.803)--(3.822,3.803)--(3.827,3.830)%
  --(3.832,3.803)--(3.837,3.830)--(3.842,3.858)--(3.847,3.803)--(3.852,3.858)--(3.857,3.803)%
  --(3.862,3.940)--(3.867,3.885)--(3.872,3.803)--(3.877,3.858)--(3.882,3.858)--(3.887,3.913)%
  --(3.892,3.830)--(3.897,3.913)--(3.902,3.885)--(3.907,3.967)--(3.912,3.913)--(3.917,3.940)%
  --(3.922,3.967)--(3.927,3.967)--(3.932,3.913)--(3.937,3.940)--(3.942,3.913)--(3.947,3.913)%
  --(3.952,3.885)--(3.958,4.022)--(3.963,4.022)--(3.968,3.995)--(3.973,3.967)--(3.978,3.940)%
  --(3.983,3.967)--(3.988,3.995)--(3.993,3.967)--(3.998,4.049)--(4.003,3.995)--(4.008,3.995)%
  --(4.013,4.022)--(4.018,4.022)--(4.023,4.022)--(4.028,4.104)--(4.033,3.995)--(4.038,3.967)%
  --(4.043,4.049)--(4.048,4.049)--(4.053,4.131)--(4.058,4.131)--(4.063,4.077)--(4.068,4.131)%
  --(4.073,4.077)--(4.078,4.077)--(4.083,4.131)--(4.089,4.131)--(4.094,4.077)--(4.099,4.159)%
  --(4.104,4.131)--(4.109,4.159)--(4.114,4.131)--(4.119,4.159)--(4.124,4.186)--(4.129,4.159)%
  --(4.134,4.159)--(4.139,4.159)--(4.144,4.104)--(4.149,4.159)--(4.154,4.186)--(4.159,4.213)%
  --(4.164,4.186)--(4.169,4.213)--(4.174,4.159)--(4.179,4.186)--(4.184,4.213)--(4.189,4.268)%
  --(4.194,4.186)--(4.199,4.241)--(4.204,4.268)--(4.209,4.241)--(4.214,4.268)--(4.219,4.241)%
  --(4.225,4.241)--(4.230,4.268)--(4.235,4.241)--(4.240,4.241)--(4.245,4.296)--(4.250,4.268)%
  --(4.255,4.241)--(4.260,4.268)--(4.265,4.296)--(4.270,4.268)--(4.275,4.350)--(4.280,4.268)%
  --(4.285,4.350)--(4.290,4.350)--(4.295,4.268)--(4.300,4.350)--(4.305,4.296)--(4.310,4.350)%
  --(4.315,4.378)--(4.320,4.378)--(4.325,4.405)--(4.330,4.323)--(4.335,4.378)--(4.340,4.378)%
  --(4.345,4.460)--(4.350,4.405)--(4.355,4.405)--(4.361,4.432)--(4.366,4.432)--(4.371,4.460)%
  --(4.376,4.432)--(4.381,4.432)--(4.386,4.350)--(4.391,4.405)--(4.396,4.460)--(4.401,4.432)%
  --(4.406,4.460)--(4.411,4.460)--(4.416,4.405)--(4.421,4.460)--(4.426,4.432)--(4.431,4.432)%
  --(4.436,4.487)--(4.441,4.405)--(4.446,4.460)--(4.451,4.514)--(4.456,4.487)--(4.461,4.514)%
  --(4.466,4.487)--(4.471,4.460)--(4.476,4.487)--(4.481,4.514)--(4.486,4.514)--(4.492,4.542)%
  --(4.497,4.542)--(4.502,4.514)--(4.507,4.542)--(4.512,4.487)--(4.517,4.542)--(4.522,4.514)%
  --(4.527,4.569)--(4.532,4.569)--(4.537,4.542)--(4.542,4.514)--(4.547,4.597)--(4.552,4.542)%
  --(4.557,4.624)--(4.562,4.624)--(4.567,4.569)--(4.572,4.569)--(4.577,4.624)--(4.582,4.679)%
  --(4.587,4.706)--(4.592,4.624)--(4.597,4.679)--(4.602,4.624)--(4.607,4.651)--(4.612,4.624)%
  --(4.617,4.597)--(4.622,4.651)--(4.628,4.651)--(4.633,4.706)--(4.638,4.733)--(4.643,4.679)%
  --(4.648,4.679)--(4.653,4.651)--(4.658,4.651)--(4.663,4.624)--(4.668,4.651)--(4.673,4.651)%
  --(4.678,4.706)--(4.683,4.706)--(4.688,4.651)--(4.693,4.706)--(4.698,4.706)--(4.703,4.706)%
  --(4.708,4.733)--(4.713,4.624)--(4.718,4.651)--(4.723,4.679)--(4.728,4.706)--(4.733,4.733)%
  --(4.738,4.761)--(4.743,4.761)--(4.748,4.761)--(4.753,4.788)--(4.758,4.761)--(4.764,4.733)%
  --(4.769,4.761)--(4.774,4.761)--(4.779,4.788)--(4.784,4.788)--(4.789,4.815)--(4.794,4.843)%
  --(4.799,4.815)--(4.804,4.815)--(4.809,4.761)--(4.814,4.788)--(4.819,4.788)--(4.824,4.870)%
  --(4.829,4.843)--(4.834,4.870)--(4.839,4.843)--(4.844,4.870)--(4.849,4.815)--(4.854,4.870)%
  --(4.859,4.843)--(4.864,4.952)--(4.869,4.870)--(4.874,4.925)--(4.879,4.925)--(4.884,4.897)%
  --(4.889,4.925)--(4.895,4.925)--(4.900,4.897)--(4.905,4.897)--(4.910,4.980)--(4.915,4.980)%
  --(4.920,5.034)--(4.925,5.007)--(4.930,4.980)--(4.935,4.980)--(4.940,4.980)--(4.945,4.980)%
  --(4.950,4.952)--(4.955,4.980)--(4.960,5.007)--(4.965,5.007)--(4.970,5.007)--(4.975,4.952)%
  --(4.980,4.980)--(4.985,4.980)--(4.990,5.007)--(4.995,5.007)--(5.000,5.007)--(5.005,5.062)%
  --(5.010,5.007)--(5.015,4.980)--(5.020,5.007)--(5.025,4.925)--(5.031,5.034)--(5.036,5.007)%
  --(5.041,5.062)--(5.046,5.007)--(5.051,5.062)--(5.056,5.007)--(5.061,5.034)--(5.066,5.116)%
  --(5.071,5.062)--(5.076,5.034)--(5.081,5.062)--(5.086,5.062)--(5.091,5.034)--(5.096,5.062)%
  --(5.101,5.089)--(5.106,5.034)--(5.111,5.089)--(5.116,5.034)--(5.121,5.007)--(5.126,5.144)%
  --(5.131,5.089)--(5.136,5.144)--(5.141,5.144)--(5.146,5.144)--(5.151,5.144)--(5.156,5.171)%
  --(5.161,5.116)--(5.167,5.144)--(5.172,5.171)--(5.177,5.198)--(5.182,5.144)--(5.187,5.144)%
  --(5.192,5.198)--(5.197,5.198)--(5.202,5.253)--(5.207,5.226)--(5.212,5.144)--(5.217,5.226)%
  --(5.222,5.171)--(5.227,5.198)--(5.232,5.171)--(5.237,5.226)--(5.242,5.198)--(5.247,5.253)%
  --(5.252,5.226)--(5.257,5.116)--(5.262,5.198)--(5.267,5.226)--(5.272,5.253)--(5.277,5.253)%
  --(5.282,5.253)--(5.287,5.253)--(5.292,5.253)--(5.298,5.253)--(5.303,5.226)--(5.308,5.226)%
  --(5.313,5.253)--(5.318,5.198)--(5.323,5.281)--(5.328,5.308)--(5.333,5.308)--(5.338,5.363)%
  --(5.343,5.308)--(5.348,5.308)--(5.353,5.308)--(5.358,5.253)--(5.363,5.226)--(5.368,5.281)%
  --(5.373,5.281)--(5.378,5.363)--(5.383,5.363)--(5.388,5.335)--(5.393,5.363)--(5.398,5.281)%
  --(5.403,5.335)--(5.408,5.308)--(5.413,5.472)--(5.418,5.390)--(5.423,5.363)--(5.428,5.390)%
  --(5.434,5.417)--(5.439,5.363)--(5.444,5.390)--(5.449,5.445)--(5.454,5.445)--(5.459,5.363)%
  --(5.464,5.335)--(5.469,5.417)--(5.474,5.417)--(5.479,5.417)--(5.484,5.445)--(5.489,5.417)%
  --(5.494,5.417)--(5.499,5.472)--(5.504,5.472)--(5.509,5.417)--(5.514,5.445)--(5.519,5.472)%
  --(5.524,5.417)--(5.529,5.472)--(5.534,5.472)--(5.539,5.472)--(5.544,5.417)--(5.549,5.445)%
  --(5.554,5.390)--(5.559,5.445)--(5.564,5.499)--(5.570,5.499)--(5.575,5.527)--(5.580,5.554)%
  --(5.585,5.581)--(5.590,5.499)--(5.595,5.445)--(5.600,5.472)--(5.605,5.472)--(5.610,5.472)%
  --(5.615,5.609)--(5.620,5.609)--(5.625,5.554)--(5.630,5.527)--(5.635,5.609)--(5.640,5.554)%
  --(5.645,5.554)--(5.650,5.581)--(5.655,5.581)--(5.660,5.581)--(5.665,5.609)--(5.670,5.581)%
  --(5.675,5.554)--(5.680,5.581)--(5.685,5.581)--(5.690,5.609)--(5.695,5.581)--(5.701,5.581)%
  --(5.706,5.581)--(5.711,5.609)--(5.716,5.581)--(5.721,5.581)--(5.726,5.609)--(5.731,5.609)%
  --(5.736,5.636)--(5.741,5.636)--(5.746,5.636)--(5.751,5.554)--(5.756,5.581)--(5.761,5.691)%
  --(5.766,5.636)--(5.771,5.664)--(5.776,5.664)--(5.781,5.664)--(5.786,5.609)--(5.791,5.581)%
  --(5.796,5.581)--(5.801,5.609)--(5.806,5.664)--(5.811,5.664)--(5.816,5.691)--(5.821,5.664)%
  --(5.826,5.718)--(5.831,5.636)--(5.837,5.609)--(5.842,5.664)--(5.847,5.691)--(5.852,5.609)%
  --(5.857,5.718)--(5.862,5.718)--(5.867,5.746)--(5.872,5.718)--(5.877,5.691)--(5.882,5.664)%
  --(5.887,5.746)--(5.892,5.800)--(5.897,5.746)--(5.902,5.691)--(5.907,5.773)--(5.912,5.773)%
  --(5.917,5.773)--(5.922,5.746)--(5.927,5.746)--(5.932,5.773)--(5.937,5.800)--(5.942,5.800)%
  --(5.947,5.800)--(5.952,5.800)--(5.957,5.800)--(5.962,5.855)--(5.967,5.800)--(5.973,5.746)%
  --(5.978,5.828)--(5.983,5.828)--(5.988,5.746)--(5.993,5.718)--(5.998,5.800)--(6.003,5.800)%
  --(6.008,5.800)--(6.013,5.800)--(6.018,5.773)--(6.023,5.773)--(6.028,5.800)--(6.033,5.828)%
  --(6.038,5.773)--(6.043,5.800)--(6.048,5.800)--(6.053,5.882)--(6.058,5.828)--(6.063,5.828)%
  --(6.068,5.855)--(6.073,5.828)--(6.078,5.855)--(6.083,5.910)--(6.088,5.882)--(6.093,5.882)%
  --(6.098,5.882)--(6.104,5.910)--(6.109,5.910)--(6.114,5.882)--(6.119,5.882)--(6.124,5.910)%
  --(6.129,5.882)--(6.134,5.882)--(6.139,5.937)--(6.144,5.910)--(6.149,5.937)--(6.154,5.937)%
  --(6.159,5.965)--(6.164,5.882)--(6.169,5.882)--(6.174,5.882)--(6.179,5.965)--(6.184,5.910)%
  --(6.189,5.965)--(6.194,6.019)--(6.199,5.992)--(6.204,5.965)--(6.209,5.992)--(6.214,5.965)%
  --(6.219,5.937)--(6.224,6.019)--(6.229,5.992)--(6.234,5.937)--(6.240,5.992)--(6.245,5.992)%
  --(6.250,5.965)--(6.255,5.965)--(6.260,5.992)--(6.265,5.882)--(6.270,5.992)--(6.275,5.965)%
  --(6.280,5.992)--(6.285,5.965)--(6.290,5.992)--(6.295,5.992)--(6.300,6.019)--(6.305,5.992)%
  --(6.310,6.019)--(6.315,5.965)--(6.320,5.992)--(6.325,6.047)--(6.330,5.992)--(6.335,6.019)%
  --(6.340,6.101)--(6.345,5.992)--(6.350,6.019)--(6.355,6.101)--(6.360,6.047)--(6.365,6.047)%
  --(6.370,6.074)--(6.376,6.019)--(6.381,6.074)--(6.386,6.074)--(6.391,6.047)--(6.396,6.047)%
  --(6.401,6.074)--(6.406,6.129)--(6.411,6.074)--(6.416,6.074)--(6.421,6.183)--(6.426,6.101)%
  --(6.431,6.101)--(6.436,6.156)--(6.441,6.074)--(6.446,6.129)--(6.451,6.101)--(6.456,6.074)%
  --(6.461,6.129)--(6.466,6.129)--(6.471,6.183)--(6.476,6.074)--(6.481,6.101)--(6.486,6.129)%
  --(6.491,6.074)--(6.496,6.047)--(6.501,6.129)--(6.507,6.101)--(6.512,6.101)--(6.517,6.074)%
  --(6.522,6.074)--(6.527,6.101)--(6.532,6.129)--(6.537,6.183)--(6.542,6.074)--(6.547,6.129)%
  --(6.552,6.101)--(6.557,6.156)--(6.562,6.156)--(6.567,6.129)--(6.572,6.129)--(6.577,6.211)%
  --(6.582,6.183)--(6.587,6.156)--(6.592,6.156)--(6.597,6.156)--(6.602,6.156)--(6.607,6.129)%
  --(6.612,6.211)--(6.617,6.211)--(6.622,6.183)--(6.627,6.183)--(6.632,6.183)--(6.637,6.211)%
  --(6.643,6.265)--(6.648,6.238)--(6.653,6.238)--(6.658,6.183)--(6.663,6.183)--(6.668,6.211)%
  --(6.673,6.265)--(6.678,6.183)--(6.683,6.211)--(6.688,6.183)--(6.693,6.238)--(6.698,6.265)%
  --(6.703,6.183)--(6.708,6.183)--(6.713,6.238)--(6.718,6.265)--(6.723,6.238)--(6.728,6.293)%
  --(6.733,6.211)--(6.738,6.238)--(6.743,6.183)--(6.748,6.156)--(6.753,6.211)--(6.758,6.211)%
  --(6.763,6.293)--(6.768,6.293)--(6.773,6.293)--(6.779,6.293)--(6.784,6.265)--(6.789,6.265)%
  --(6.794,6.293)--(6.799,6.238)--(6.804,6.293)--(6.809,6.348)--(6.814,6.293)--(6.819,6.320)%
  --(6.824,6.293)--(6.829,6.265)--(6.834,6.348)--(6.839,6.293)--(6.844,6.293)--(6.849,6.293)%
  --(6.854,6.293)--(6.859,6.320)--(6.864,6.265)--(6.869,6.238)--(6.874,6.293)--(6.879,6.265)%
  --(6.884,6.265)--(6.889,6.320)--(6.894,6.348)--(6.899,6.320)--(6.904,6.348)--(6.910,6.348)%
  --(6.915,6.375)--(6.920,6.375)--(6.925,6.293)--(6.930,6.402)--(6.935,6.348)--(6.940,6.348)%
  --(6.945,6.320)--(6.950,6.293)--(6.955,6.265)--(6.960,6.348)--(6.965,6.348)--(6.970,6.265)%
  --(6.975,6.320)--(6.980,6.293)--(6.985,6.320)--(6.990,6.293)--(6.995,6.348)--(7.000,6.320)%
  --(7.005,6.402)--(7.010,6.348)--(7.015,6.375)--(7.020,6.320)--(7.025,6.375)--(7.030,6.402)%
  --(7.035,6.375)--(7.040,6.320)--(7.046,6.457)--(7.051,6.320)--(7.056,6.320)--(7.061,6.402)%
  --(7.066,6.430)--(7.071,6.402)--(7.076,6.375)--(7.081,6.348)--(7.086,6.348)--(7.091,6.402)%
  --(7.096,6.430)--(7.101,6.375)--(7.106,6.375)--(7.111,6.375)--(7.116,6.402)--(7.121,6.348)%
  --(7.126,6.402)--(7.131,6.430)--(7.136,6.430)--(7.141,6.430)--(7.146,6.512)--(7.151,6.512)%
  --(7.156,6.457)--(7.161,6.430)--(7.166,6.512)--(7.171,6.457)--(7.176,6.375)--(7.182,6.512)%
  --(7.187,6.320)--(7.192,6.457)--(7.197,6.457)--(7.202,6.430)--(7.207,6.402)--(7.212,6.457)%
  --(7.217,6.402)--(7.222,6.402)--(7.227,6.484)--(7.232,6.457)--(7.237,6.402)--(7.242,6.430)%
  --(7.247,6.457)--(7.252,6.512)--(7.257,6.512)--(7.262,6.512)--(7.267,6.512)--(7.272,6.457)%
  --(7.277,6.430)--(7.282,6.457)--(7.287,6.512)--(7.292,6.512)--(7.297,6.512)--(7.302,6.512)%
  --(7.307,6.512)--(7.313,6.539)--(7.318,6.539)--(7.323,6.484)--(7.328,6.512)--(7.333,6.457)%
  --(7.338,6.457)--(7.343,6.484)--(7.348,6.539)--(7.353,6.566)--(7.358,6.539)--(7.363,6.484)%
  --(7.368,6.484)--(7.373,6.484)--(7.378,6.484)--(7.383,6.512)--(7.388,6.512)--(7.393,6.539)%
  --(7.398,6.484)--(7.403,6.566)--(7.408,6.512)--(7.413,6.457)--(7.418,6.457)--(7.423,6.484)%
  --(7.428,6.512)--(7.433,6.484)--(7.438,6.566)--(7.443,6.566)--(7.449,6.512)--(7.454,6.594)%
  --(7.459,6.539)--(7.464,6.512)--(7.469,6.539)--(7.474,6.484)--(7.479,6.512)--(7.484,6.484)%
  --(7.489,6.594)--(7.494,6.512)--(7.499,6.594)--(7.504,6.539)--(7.509,6.512)--(7.514,6.566)%
  --(7.519,6.512)--(7.524,6.539)--(7.529,6.621)--(7.534,6.594)--(7.539,6.594)--(7.544,6.594)%
  --(7.549,6.566)--(7.554,6.539)--(7.559,6.594)--(7.564,6.621)--(7.569,6.566)--(7.574,6.594)%
  --(7.579,6.621)--(7.585,6.566)--(7.590,6.649)--(7.595,6.594)--(7.600,6.594)--(7.605,6.539)%
  --(7.610,6.649)--(7.615,6.594)--(7.620,6.594)--(7.625,6.621)--(7.630,6.649)--(7.635,6.621)%
  --(7.640,6.594)--(7.645,6.621)--(7.650,6.649)--(7.655,6.539)--(7.660,6.566)--(7.665,6.621)%
  --(7.670,6.676)--(7.675,6.649)--(7.680,6.649)--(7.685,6.649)--(7.690,6.649)--(7.695,6.649)%
  --(7.700,6.649)--(7.705,6.676)--(7.710,6.676)--(7.716,6.649)--(7.721,6.676)--(7.726,6.621)%
  --(7.731,6.649)--(7.736,6.621)--(7.741,6.649)--(7.746,6.621)--(7.751,6.676)--(7.756,6.649)%
  --(7.761,6.703)--(7.766,6.649)--(7.771,6.676)--(7.776,6.649)--(7.781,6.676)--(7.786,6.621)%
  --(7.791,6.703)--(7.796,6.676)--(7.801,6.621)--(7.806,6.621)--(7.811,6.676)--(7.816,6.649)%
  --(7.821,6.594)--(7.826,6.621)--(7.831,6.676)--(7.836,6.703)--(7.841,6.594)--(7.846,6.703)%
  --(7.852,6.621)--(7.857,6.703)--(7.862,6.703)--(7.867,6.676)--(7.872,6.703)--(7.877,6.703)%
  --(7.882,6.649)--(7.887,6.649)--(7.892,6.731)--(7.897,6.676)--(7.902,6.703)--(7.907,6.731)%
  --(7.912,6.676)--(7.917,6.731)--(7.922,6.703)--(7.927,6.703)--(7.932,6.758)--(7.937,6.758)%
  --(7.942,6.731)--(7.947,6.676)--(7.952,6.703)--(7.957,6.703)--(7.962,6.703)--(7.967,6.758)%
  --(7.972,6.731)--(7.977,6.758)--(7.982,6.703)--(7.988,6.813)--(7.993,6.731)--(7.998,6.703)%
  --(8.003,6.676)--(8.008,6.758)--(8.013,6.703)--(8.018,6.703)--(8.023,6.758)--(8.028,6.731)%
  --(8.033,6.785)--(8.038,6.758)--(8.043,6.758)--(8.048,6.785)--(8.053,6.758)--(8.058,6.813)%
  --(8.063,6.731)--(8.068,6.758)--(8.073,6.758)--(8.078,6.813)--(8.083,6.758)--(8.088,6.813)%
  --(8.093,6.758)--(8.098,6.758)--(8.103,6.840)--(8.108,6.785)--(8.113,6.785)--(8.119,6.813)%
  --(8.124,6.813)--(8.129,6.785)--(8.134,6.703)--(8.139,6.731)--(8.144,6.785)--(8.149,6.731)%
  --(8.154,6.758)--(8.159,6.785)--(8.164,6.731)--(8.169,6.758)--(8.174,6.785)--(8.179,6.785)%
  --(8.184,6.785)--(8.189,6.785)--(8.194,6.813)--(8.199,6.785)--(8.204,6.785)--(8.209,6.758)%
  --(8.214,6.813)--(8.219,6.813)--(8.224,6.785)--(8.229,6.758)--(8.234,6.731)--(8.239,6.785)%
  --(8.244,6.785)--(8.249,6.840)--(8.255,6.840)--(8.260,6.840)--(8.265,6.813)--(8.270,6.840)%
  --(8.275,6.785)--(8.280,6.840)--(8.285,6.813)--(8.290,6.867)--(8.295,6.867)--(8.300,6.922)%
  --(8.305,6.867)--(8.310,6.867)--(8.315,6.813)--(8.320,6.867)--(8.325,6.813)--(8.330,6.731)%
  --(8.335,6.840)--(8.340,6.813)--(8.345,6.840)--(8.350,6.840)--(8.355,6.840)--(8.360,6.867)%
  --(8.365,6.867)--(8.370,6.840)--(8.375,6.813)--(8.380,6.867)--(8.385,6.813)--(8.391,6.840)%
  --(8.396,6.895)--(8.401,6.867)--(8.406,6.867)--(8.411,6.840)--(8.416,6.813)--(8.421,6.867)%
  --(8.426,6.785)--(8.431,6.867)--(8.436,6.813)--(8.441,6.840)--(8.446,6.867)--(8.451,6.867)%
  --(8.456,6.922)--(8.461,6.895)--(8.466,6.840)--(8.471,6.922)--(8.476,6.895)--(8.481,6.949)%
  --(8.486,6.867)--(8.491,6.922)--(8.496,6.840)--(8.501,6.867)--(8.506,6.895)--(8.511,6.895)%
  --(8.516,6.977)--(8.522,6.867)--(8.527,6.813)--(8.532,6.895)--(8.537,6.895)--(8.542,6.895)%
  --(8.547,6.895)--(8.552,6.895)--(8.557,6.949)--(8.562,6.867)--(8.567,6.922)--(8.572,6.840)%
  --(8.577,6.922)--(8.582,6.922)--(8.587,6.922)--(8.592,6.949)--(8.597,6.922)--(8.602,6.895)%
  --(8.607,6.949)--(8.612,6.895)--(8.617,6.922)--(8.622,6.949)--(8.627,6.895)--(8.632,7.004)%
  --(8.637,6.949)--(8.642,6.949)--(8.647,6.977)--(8.652,6.895)--(8.658,6.867)--(8.663,6.949)%
  --(8.668,6.977)--(8.673,6.949)--(8.678,6.977)--(8.683,6.949)--(8.688,6.977)--(8.693,6.977)%
  --(8.698,6.922)--(8.703,6.949)--(8.708,6.949)--(8.713,6.949)--(8.718,7.032)--(8.723,6.949)%
  --(8.728,6.977)--(8.733,6.895)--(8.738,6.949)--(8.743,6.949)--(8.748,6.949)--(8.753,6.949)%
  --(8.758,7.032)--(8.763,6.977)--(8.768,6.977)--(8.773,6.977)--(8.778,6.977)--(8.783,7.004)%
  --(8.788,7.004)--(8.794,7.004)--(8.799,6.977)--(8.804,7.004)--(8.809,6.977)--(8.814,6.977)%
  --(8.819,7.032)--(8.824,7.004)--(8.829,7.004)--(8.834,7.059)--(8.839,7.004)--(8.844,7.004)%
  --(8.849,7.004)--(8.854,7.086)--(8.859,6.977)--(8.864,6.977)--(8.869,7.059)--(8.874,6.977)%
  --(8.879,7.004)--(8.884,7.114)--(8.889,7.086)--(8.894,7.004)--(8.899,7.059)--(8.904,7.059)%
  --(8.909,7.032)--(8.914,7.032)--(8.919,7.004)--(8.925,7.141)--(8.930,7.004)--(8.935,7.004)%
  --(8.940,7.032)--(8.945,7.059)--(8.950,7.086)--(8.955,7.059)--(8.960,7.059)--(8.965,7.032)%
  --(8.970,6.977)--(8.975,7.032)--(8.980,7.004)--(8.985,7.086)--(8.990,7.004)--(8.995,7.059)%
  --(9.000,7.086)--(9.005,7.004)--(9.010,7.032)--(9.015,7.086)--(9.020,7.059)--(9.025,7.086)%
  --(9.030,7.032)--(9.035,7.059)--(9.040,7.032)--(9.045,7.032)--(9.050,7.141)--(9.055,7.114)%
  --(9.061,7.141)--(9.066,7.086)--(9.071,7.032)--(9.076,7.059)--(9.081,7.059)--(9.086,7.114)%
  --(9.091,7.086)--(9.096,7.114)--(9.101,7.086)--(9.106,7.086)--(9.111,7.141)--(9.116,7.059)%
  --(9.121,7.141)--(9.126,7.114)--(9.131,7.086)--(9.136,7.086)--(9.141,7.114)--(9.146,7.114)%
  --(9.151,7.168)--(9.156,7.114)--(9.161,7.196)--(9.166,7.086)--(9.171,7.086)--(9.176,7.086)%
  --(9.181,7.032)--(9.186,7.086)--(9.191,7.114)--(9.197,7.086)--(9.202,7.114)--(9.207,7.114)%
  --(9.212,7.114)--(9.217,7.114)--(9.222,7.141)--(9.227,7.141)--(9.232,7.086)--(9.237,7.114)%
  --(9.242,7.114)--(9.247,7.141)--(9.252,7.141)--(9.257,7.114)--(9.262,7.114)--(9.267,7.114)%
  --(9.272,7.114)--(9.277,7.196)--(9.282,7.168)--(9.287,7.141)--(9.292,7.196)--(9.297,7.141)%
  --(9.302,7.141)--(9.307,7.141)--(9.312,7.168)--(9.317,7.114)--(9.322,7.114)--(9.328,7.114)%
  --(9.333,7.114)--(9.338,7.223)--(9.343,7.168)--(9.348,7.196)--(9.353,7.168)--(9.358,7.223)%
  --(9.363,7.141)--(9.368,7.196)--(9.373,7.196)--(9.378,7.250)--(9.383,7.141)--(9.388,7.168)%
  --(9.393,7.223)--(9.398,7.223)--(9.403,7.196)--(9.408,7.196)--(9.413,7.196)--(9.418,7.250)%
  --(9.423,7.223)--(9.428,7.168)--(9.433,7.196)--(9.438,7.223)--(9.443,7.223)--(9.448,7.223)%
  --(9.453,7.168)--(9.458,7.196)--(9.464,7.196)--(9.469,7.141)--(9.474,7.168)--(9.479,7.141)%
  --(9.484,7.141)--(9.489,7.141)--(9.494,7.141)--(9.499,7.278)--(9.504,7.196)--(9.509,7.196)%
  --(9.514,7.168)--(9.519,7.223)--(9.524,7.196)--(9.529,7.223)--(9.534,7.196)--(9.539,7.278)%
  --(9.544,7.305)--(9.549,7.223)--(9.554,7.278)--(9.559,7.278)--(9.564,7.250)--(9.569,7.223)%
  --(9.574,7.305)--(9.579,7.223)--(9.584,7.278)--(9.589,7.250)--(9.594,7.250)--(9.600,7.223)%
  --(9.605,7.278)--(9.610,7.223)--(9.615,7.196)--(9.620,7.305)--(9.625,7.250)--(9.630,7.305)%
  --(9.635,7.305)--(9.640,7.305)--(9.645,7.278)--(9.650,7.333)--(9.655,7.278)--(9.660,7.305)%
  --(9.665,7.250)--(9.670,7.333)--(9.675,7.250)--(9.680,7.305)--(9.685,7.250)--(9.690,7.278)%
  --(9.695,7.223)--(9.700,7.278)--(9.705,7.250)--(9.710,7.333)--(9.715,7.305)--(9.720,7.333)%
  --(9.725,7.333)--(9.731,7.278)--(9.736,7.305)--(9.741,7.305)--(9.746,7.305)--(9.751,7.278)%
  --(9.756,7.250)--(9.761,7.305)--(9.766,7.250)--(9.771,7.278)--(9.776,7.278)--(9.781,7.278)%
  --(9.786,7.305)--(9.791,7.305)--(9.796,7.360)--(9.801,7.278)--(9.806,7.387)--(9.811,7.305)%
  --(9.816,7.333)--(9.821,7.360)--(9.826,7.387)--(9.831,7.360)--(9.836,7.387)--(9.841,7.333)%
  --(9.846,7.333)--(9.851,7.333)--(9.856,7.387)--(9.861,7.305)--(9.867,7.333)--(9.872,7.333)%
  --(9.877,7.360)--(9.882,7.305)--(9.887,7.360)--(9.892,7.360)--(9.897,7.360)--(9.902,7.360)%
  --(9.907,7.415)--(9.912,7.360)--(9.917,7.387)--(9.922,7.305)--(9.927,7.333)--(9.932,7.415)%
  --(9.937,7.305)--(9.942,7.333)--(9.947,7.387)--(9.952,7.333)--(9.957,7.360)--(9.962,7.278)%
  --(9.967,7.360)--(9.972,7.305)--(9.977,7.333)--(9.982,7.387)--(9.987,7.387)--(9.992,7.360)%
  --(9.997,7.360)--(10.003,7.415)--(10.008,7.333)--(10.013,7.333)--(10.018,7.333)--(10.023,7.360)%
  --(10.028,7.333)--(10.033,7.387)--(10.038,7.415)--(10.043,7.360)--(10.048,7.387)--(10.053,7.387)%
  --(10.058,7.333)--(10.063,7.415)--(10.068,7.387)--(10.073,7.387)--(10.078,7.415)--(10.083,7.360)%
  --(10.088,7.360)--(10.093,7.360)--(10.098,7.415)--(10.103,7.360)--(10.108,7.333)--(10.113,7.415)%
  --(10.118,7.415)--(10.123,7.387)--(10.128,7.442)--(10.134,7.442)--(10.139,7.442)--(10.144,7.442)%
  --(10.149,7.497)--(10.154,7.469)--(10.159,7.387)--(10.164,7.415)--(10.169,7.469)--(10.174,7.387)%
  --(10.179,7.415)--(10.184,7.442)--(10.189,7.333)--(10.194,7.415)--(10.199,7.360)--(10.204,7.387)%
  --(10.209,7.415)--(10.214,7.415)--(10.219,7.415)--(10.224,7.442)--(10.229,7.442)--(10.234,7.469)%
  --(10.239,7.524)--(10.244,7.469)--(10.249,7.442)--(10.254,7.415)--(10.259,7.442)--(10.264,7.497)%
  --(10.270,7.442)--(10.275,7.469)--(10.280,7.524)--(10.285,7.497)--(10.290,7.497)--(10.295,7.469)%
  --(10.300,7.387)--(10.305,7.442)--(10.310,7.442)--(10.315,7.442)--(10.320,7.469)--(10.325,7.551)%
  --(10.330,7.469)--(10.335,7.524)--(10.340,7.497)--(10.345,7.497)--(10.350,7.524)--(10.355,7.469)%
  --(10.360,7.497)--(10.365,7.497)--(10.370,7.524)--(10.375,7.524)--(10.380,7.497)--(10.385,7.497)%
  --(10.390,7.551)--(10.395,7.524)--(10.400,7.497)--(10.406,7.469)--(10.411,7.524)--(10.416,7.387)%
  --(10.421,7.524)--(10.426,7.497)--(10.431,7.497)--(10.436,7.497)--(10.441,7.442)--(10.446,7.524)%
  --(10.451,7.524)--(10.456,7.497)--(10.461,7.551)--(10.466,7.469)--(10.471,7.524)--(10.476,7.497)%
  --(10.481,7.497)--(10.486,7.497)--(10.491,7.524)--(10.496,7.469)--(10.501,7.469)--(10.506,7.497)%
  --(10.511,7.551)--(10.516,7.579)--(10.521,7.497)--(10.526,7.524)--(10.531,7.551)--(10.537,7.524)%
  --(10.542,7.579)--(10.547,7.497)--(10.552,7.497)--(10.557,7.551)--(10.562,7.551)--(10.567,7.551)%
  --(10.572,7.497)--(10.577,7.551)--(10.582,7.606)--(10.587,7.524)--(10.592,7.579)--(10.597,7.579)%
  --(10.602,7.579)--(10.607,7.524)--(10.612,7.579)--(10.617,7.551)--(10.622,7.524)--(10.627,7.551)%
  --(10.632,7.579)--(10.637,7.524)--(10.642,7.551)--(10.647,7.551)--(10.652,7.579)--(10.657,7.579)%
  --(10.662,7.551)--(10.667,7.551)--(10.673,7.606)--(10.678,7.524)--(10.683,7.606)--(10.688,7.606)%
  --(10.693,7.633)--(10.698,7.579)--(10.703,7.551)--(10.708,7.551)--(10.713,7.551)--(10.718,7.579)%
  --(10.723,7.633)--(10.728,7.579)--(10.733,7.633)--(10.738,7.606)--(10.743,7.579)--(10.748,7.579)%
  --(10.753,7.661)--(10.758,7.633)--(10.763,7.633)--(10.768,7.606)--(10.773,7.633)--(10.778,7.606)%
  --(10.783,7.633)--(10.788,7.606)--(10.793,7.606)--(10.798,7.633)--(10.803,7.661)--(10.809,7.606)%
  --(10.814,7.688)--(10.819,7.661)--(10.824,7.579)--(10.829,7.661)--(10.834,7.633)--(10.839,7.661)%
  --(10.844,7.606)--(10.849,7.661)--(10.854,7.688)--(10.859,7.716)--(10.864,7.633)--(10.869,7.551)%
  --(10.874,7.606)--(10.879,7.551)--(10.884,7.606)--(10.889,7.606)--(10.894,7.551)--(10.899,7.606)%
  --(10.904,7.606)--(10.909,7.606)--(10.914,7.633)--(10.919,7.579)--(10.924,7.633)--(10.929,7.633)%
  --(10.934,7.633)--(10.940,7.661)--(10.945,7.633)--(10.950,7.661)--(10.955,7.633)--(10.960,7.579)%
  --(10.965,7.688)--(10.970,7.661)--(10.975,7.633)--(10.980,7.606)--(10.985,7.716)--(10.990,7.661)%
  --(10.995,7.688)--(11.000,7.661)--(11.005,7.688);
\gpcolor{color=gp lt color border}
\gpsetlinetype{gp lt border}
\draw[gp path] (1.872,7.716)--(1.872,1.970);
\draw[gp path] (1.872,0.985)--(11.005,0.985);
%% coordinates of the plot area
\gpdefrectangularnode{gp plot 1}{\pgfpoint{1.872cm}{0.985cm}}{\pgfpoint{11.947cm}{7.825cm}}
\end{tikzpicture}
%% gnuplot variables

\caption{Variazione della temperatura, esempio 2}
\label{img:isoa}
\end{grafico}

\begin{grafico}
  \centering
\begin{tikzpicture}[gnuplot]
%% generated with GNUPLOT 4.6p3 (Lua 5.1; terminal rev. 99, script rev. 100)
%% mar 27 mag 2014 22:34:40 CEST
\path (0.000,0.000) rectangle (12.500,8.750);
\gpcolor{color=gp lt color border}
\gpsetlinetype{gp lt border}
\gpsetlinewidth{1.00}
\draw[gp path] (2.056,1.840)--(2.236,1.840);
\node[gp node right] at (1.872,1.840) { 20.365};
\draw[gp path] (2.056,2.695)--(2.236,2.695);
\node[gp node right] at (1.872,2.695) { 20.37};
\draw[gp path] (2.056,3.550)--(2.236,3.550);
\node[gp node right] at (1.872,3.550) { 20.375};
\draw[gp path] (2.056,4.405)--(2.236,4.405);
\node[gp node right] at (1.872,4.405) { 20.38};
\draw[gp path] (2.056,5.260)--(2.236,5.260);
\node[gp node right] at (1.872,5.260) { 20.385};
\draw[gp path] (2.056,6.115)--(2.236,6.115);
\node[gp node right] at (1.872,6.115) { 20.39};
\draw[gp path] (2.056,6.970)--(2.236,6.970);
\node[gp node right] at (1.872,6.970) { 20.395};
\draw[gp path] (2.056,7.825)--(2.236,7.825);
\node[gp node right] at (1.872,7.825) { 20.4};
\draw[gp path] (2.056,0.985)--(2.056,1.165);
\node[gp node center] at (2.056,0.677) { 0};
\draw[gp path] (4.034,0.985)--(4.034,1.165);
\node[gp node center] at (4.034,0.677) { 500};
\draw[gp path] (6.012,0.985)--(6.012,1.165);
\node[gp node center] at (6.012,0.677) { 1000};
\draw[gp path] (7.991,0.985)--(7.991,1.165);
\node[gp node center] at (7.991,0.677) { 1500};
\draw[gp path] (9.969,0.985)--(9.969,1.165);
\node[gp node center] at (9.969,0.677) { 2000};
\draw[gp path] (2.056,7.825)--(2.056,1.327);
\draw[gp path] (2.056,0.985)--(10.325,0.985);
\node[gp node center,rotate=-270] at (0.246,4.405) {Temperatura $[\gradi C]$};
\node[gp node center] at (7.001,0.215) {Numero della misura};
\node[gp node center] at (7.001,8.287) {Quarta Isoterma Ritorno};
\gpcolor{color=gp lt color 0}
\gpsetlinetype{gp lt plot 0}
\draw[gp path] (2.056,6.628)--(2.060,6.628)--(2.064,6.286)--(2.068,7.483)--(2.072,7.654)%
  --(2.076,6.799)--(2.080,6.799)--(2.084,6.457)--(2.088,6.628)--(2.092,6.286)--(2.096,6.457)%
  --(2.100,6.628)--(2.103,6.628)--(2.107,7.483)--(2.111,7.312)--(2.115,7.312)--(2.119,7.483)%
  --(2.123,6.628)--(2.127,6.628)--(2.131,7.654)--(2.135,6.628)--(2.139,6.799)--(2.143,6.799)%
  --(2.147,6.628)--(2.151,7.312)--(2.155,7.825)--(2.159,7.312)--(2.163,6.799)--(2.167,6.799)%
  --(2.171,6.970)--(2.175,6.628)--(2.179,6.628)--(2.183,7.483)--(2.187,6.799)--(2.191,6.457)%
  --(2.194,6.286)--(2.198,6.286)--(2.202,6.457)--(2.206,6.457)--(2.210,6.628)--(2.214,6.970)%
  --(2.218,6.799)--(2.222,7.312)--(2.226,6.457)--(2.230,6.457)--(2.234,6.970)--(2.238,6.457)%
  --(2.242,6.799)--(2.246,6.628)--(2.250,6.628)--(2.254,6.628)--(2.258,6.970)--(2.262,6.970)%
  --(2.266,6.970)--(2.270,6.457)--(2.274,6.799)--(2.278,6.628)--(2.282,6.628)--(2.285,6.628)%
  --(2.289,6.799)--(2.293,6.970)--(2.297,6.799)--(2.301,7.483)--(2.305,7.141)--(2.309,7.654)%
  --(2.313,7.825)--(2.317,6.799)--(2.321,6.970)--(2.325,7.312)--(2.329,7.141)--(2.333,7.312)%
  --(2.337,7.141)--(2.341,7.141)--(2.345,7.654)--(2.349,6.970)--(2.353,7.654)--(2.357,7.483)%
  --(2.361,6.799)--(2.365,7.654)--(2.369,6.970)--(2.373,7.312)--(2.376,6.799)--(2.380,6.970)%
  --(2.384,6.286)--(2.388,6.628)--(2.392,6.628)--(2.396,6.457)--(2.400,6.970)--(2.404,6.457)%
  --(2.408,7.141)--(2.412,6.628)--(2.416,6.799)--(2.420,6.286)--(2.424,6.286)--(2.428,6.457)%
  --(2.432,6.457)--(2.436,6.628)--(2.440,6.628)--(2.444,6.457)--(2.448,6.457)--(2.452,6.799)%
  --(2.456,6.628)--(2.460,6.628)--(2.464,6.628)--(2.467,6.115)--(2.471,6.628)--(2.475,6.799)%
  --(2.479,6.970)--(2.483,7.654)--(2.487,6.628)--(2.491,6.628)--(2.495,6.457)--(2.499,6.628)%
  --(2.503,6.457)--(2.507,6.628)--(2.511,6.628)--(2.515,6.970)--(2.519,6.970)--(2.523,6.286)%
  --(2.527,6.115)--(2.531,6.457)--(2.535,6.115)--(2.539,6.457)--(2.543,6.628)--(2.547,6.457)%
  --(2.551,6.628)--(2.555,6.286)--(2.558,6.970)--(2.562,6.115)--(2.566,6.457)--(2.570,6.457)%
  --(2.574,6.628)--(2.578,6.799)--(2.582,6.286)--(2.586,6.628)--(2.590,6.457)--(2.594,6.799)%
  --(2.598,6.457)--(2.602,6.457)--(2.606,6.457)--(2.610,6.628)--(2.614,6.970)--(2.618,6.628)%
  --(2.622,6.799)--(2.626,6.628)--(2.630,6.457)--(2.634,6.799)--(2.638,6.970)--(2.642,6.457)%
  --(2.646,6.628)--(2.649,6.628)--(2.653,6.457)--(2.657,6.970)--(2.661,6.799)--(2.665,6.457)%
  --(2.669,6.628)--(2.673,7.312)--(2.677,7.141)--(2.681,7.312)--(2.685,6.970)--(2.689,6.457)%
  --(2.693,6.628)--(2.697,6.799)--(2.701,6.286)--(2.705,6.457)--(2.709,6.628)--(2.713,6.799)%
  --(2.717,6.628)--(2.721,6.970)--(2.725,6.799)--(2.729,6.457)--(2.733,6.799)--(2.737,6.799)%
  --(2.740,6.799)--(2.744,6.457)--(2.748,6.970)--(2.752,6.799)--(2.756,6.799)--(2.760,6.286)%
  --(2.764,6.457)--(2.768,6.286)--(2.772,6.628)--(2.776,6.970)--(2.780,6.457)--(2.784,6.970)%
  --(2.788,7.312)--(2.792,7.141)--(2.796,6.970)--(2.800,6.628)--(2.804,6.628)--(2.808,6.115)%
  --(2.812,6.457)--(2.816,6.286)--(2.820,6.970)--(2.824,7.312)--(2.827,6.628)--(2.831,6.970)%
  --(2.835,6.115)--(2.839,6.286)--(2.843,6.457)--(2.847,6.628)--(2.851,6.628)--(2.855,6.457)%
  --(2.859,6.628)--(2.863,6.970)--(2.867,7.312)--(2.871,7.141)--(2.875,6.628)--(2.879,6.457)%
  --(2.883,5.944)--(2.887,6.457)--(2.891,6.628)--(2.895,7.312)--(2.899,6.799)--(2.903,6.970)%
  --(2.907,6.628)--(2.911,6.457)--(2.915,6.286)--(2.918,6.286)--(2.922,6.457)--(2.926,6.457)%
  --(2.930,6.115)--(2.934,7.312)--(2.938,6.970)--(2.942,6.970)--(2.946,7.141)--(2.950,6.799)%
  --(2.954,6.457)--(2.958,6.457)--(2.962,5.944)--(2.966,6.286)--(2.970,6.799)--(2.974,6.799)%
  --(2.978,7.141)--(2.982,6.799)--(2.986,6.457)--(2.990,6.457)--(2.994,6.457)--(2.998,6.457)%
  --(3.002,6.628)--(3.006,6.628)--(3.009,6.970)--(3.013,6.970)--(3.017,6.799)--(3.021,6.286)%
  --(3.025,6.628)--(3.029,6.115)--(3.033,6.457)--(3.037,6.115)--(3.041,6.286)--(3.045,6.628)%
  --(3.049,7.141)--(3.053,7.312)--(3.057,6.799)--(3.061,6.970)--(3.065,6.115)--(3.069,6.457)%
  --(3.073,6.457)--(3.077,6.286)--(3.081,6.628)--(3.085,6.115)--(3.089,6.286)--(3.093,7.141)%
  --(3.097,6.628)--(3.100,6.286)--(3.104,6.286)--(3.108,5.944)--(3.112,6.286)--(3.116,5.944)%
  --(3.120,6.970)--(3.124,6.115)--(3.128,6.115)--(3.132,6.115)--(3.136,6.115)--(3.140,6.115)%
  --(3.144,6.115)--(3.148,6.286)--(3.152,6.286)--(3.156,6.286)--(3.160,6.457)--(3.164,6.286)%
  --(3.168,6.286)--(3.172,6.286)--(3.176,6.115)--(3.180,6.457)--(3.184,6.286)--(3.188,6.286)%
  --(3.191,6.457)--(3.195,6.970)--(3.199,6.286)--(3.203,6.457)--(3.207,6.286)--(3.211,6.457)%
  --(3.215,6.115)--(3.219,6.115)--(3.223,6.115)--(3.227,6.457)--(3.231,6.457)--(3.235,6.115)%
  --(3.239,7.141)--(3.243,6.286)--(3.247,6.286)--(3.251,5.944)--(3.255,6.628)--(3.259,6.628)%
  --(3.263,6.457)--(3.267,6.115)--(3.271,6.799)--(3.275,6.115)--(3.279,6.286)--(3.282,6.115)%
  --(3.286,6.457)--(3.290,6.286)--(3.294,6.628)--(3.298,6.115)--(3.302,6.286)--(3.306,6.286)%
  --(3.310,6.799)--(3.314,6.457)--(3.318,6.286)--(3.322,6.286)--(3.326,6.970)--(3.330,6.457)%
  --(3.334,6.457)--(3.338,6.457)--(3.342,6.286)--(3.346,6.799)--(3.350,6.457)--(3.354,6.115)%
  --(3.358,6.286)--(3.362,6.457)--(3.366,5.944)--(3.370,6.628)--(3.373,6.286)--(3.377,6.115)%
  --(3.381,6.457)--(3.385,6.628)--(3.389,6.286)--(3.393,6.457)--(3.397,6.628)--(3.401,6.115)%
  --(3.405,6.286)--(3.409,6.115)--(3.413,6.286)--(3.417,6.115)--(3.421,6.115)--(3.425,6.628)%
  --(3.429,6.970)--(3.433,6.115)--(3.437,6.286)--(3.441,6.628)--(3.445,6.286)--(3.449,5.944)%
  --(3.453,6.115)--(3.457,6.115)--(3.461,6.286)--(3.464,6.286)--(3.468,6.799)--(3.472,7.141)%
  --(3.476,6.286)--(3.480,5.944)--(3.484,6.115)--(3.488,6.628)--(3.492,6.286)--(3.496,7.141)%
  --(3.500,6.286)--(3.504,6.286)--(3.508,6.799)--(3.512,6.799)--(3.516,6.115)--(3.520,6.286)%
  --(3.524,6.286)--(3.528,6.286)--(3.532,5.944)--(3.536,6.286)--(3.540,6.286)--(3.544,6.115)%
  --(3.548,5.944)--(3.552,5.944)--(3.555,6.457)--(3.559,6.628)--(3.563,6.457)--(3.567,6.286)%
  --(3.571,6.628)--(3.575,6.799)--(3.579,6.628)--(3.583,6.628)--(3.587,6.628)--(3.591,6.286)%
  --(3.595,6.115)--(3.599,6.457)--(3.603,5.944)--(3.607,6.286)--(3.611,6.286)--(3.615,6.286)%
  --(3.619,6.115)--(3.623,6.628)--(3.627,6.457)--(3.631,6.115)--(3.635,6.115)--(3.639,6.286)%
  --(3.643,6.286)--(3.646,6.286)--(3.650,6.115)--(3.654,6.286)--(3.658,6.628)--(3.662,6.628)%
  --(3.666,6.628)--(3.670,6.457)--(3.674,6.457)--(3.678,6.115)--(3.682,6.286)--(3.686,6.799)%
  --(3.690,6.970)--(3.694,6.628)--(3.698,6.457)--(3.702,6.457)--(3.706,6.115)--(3.710,6.457)%
  --(3.714,6.286)--(3.718,6.115)--(3.722,6.286)--(3.726,6.115)--(3.730,5.944)--(3.734,6.115)%
  --(3.737,6.286)--(3.741,5.773)--(3.745,6.115)--(3.749,6.286)--(3.753,6.286)--(3.757,6.115)%
  --(3.761,6.628)--(3.765,6.628)--(3.769,6.628)--(3.773,7.141)--(3.777,6.286)--(3.781,6.286)%
  --(3.785,5.944)--(3.789,6.457)--(3.793,6.286)--(3.797,6.115)--(3.801,6.286)--(3.805,6.286)%
  --(3.809,6.457)--(3.813,6.457)--(3.817,5.944)--(3.821,5.944)--(3.825,5.944)--(3.828,5.944)%
  --(3.832,6.115)--(3.836,6.628)--(3.840,6.799)--(3.844,6.799)--(3.848,6.799)--(3.852,6.457)%
  --(3.856,6.115)--(3.860,6.115)--(3.864,6.457)--(3.868,6.115)--(3.872,6.286)--(3.876,7.141)%
  --(3.880,6.457)--(3.884,6.799)--(3.888,6.628)--(3.892,6.628)--(3.896,6.115)--(3.900,6.115)%
  --(3.904,5.944)--(3.908,6.115)--(3.912,6.457)--(3.916,6.457)--(3.919,7.141)--(3.923,7.141)%
  --(3.927,6.457)--(3.931,6.286)--(3.935,6.457)--(3.939,6.286)--(3.943,6.115)--(3.947,6.286)%
  --(3.951,6.286)--(3.955,6.457)--(3.959,6.286)--(3.963,6.286)--(3.967,5.944)--(3.971,6.115)%
  --(3.975,5.944)--(3.979,5.773)--(3.983,6.286)--(3.987,5.944)--(3.991,6.115)--(3.995,5.773)%
  --(3.999,6.628)--(4.003,5.944)--(4.007,6.115)--(4.010,6.457)--(4.014,6.286)--(4.018,5.773)%
  --(4.022,5.944)--(4.026,5.944)--(4.030,5.944)--(4.034,6.628)--(4.038,6.286)--(4.042,6.115)%
  --(4.046,5.773)--(4.050,5.944)--(4.054,6.286)--(4.058,6.115)--(4.062,6.286)--(4.066,6.115)%
  --(4.070,5.773)--(4.074,5.773)--(4.078,5.944)--(4.082,6.115)--(4.086,6.115)--(4.090,6.115)%
  --(4.094,5.944)--(4.098,5.773)--(4.101,6.628)--(4.105,6.115)--(4.109,5.944)--(4.113,6.115)%
  --(4.117,6.115)--(4.121,5.944)--(4.125,5.944)--(4.129,5.773)--(4.133,5.944)--(4.137,6.286)%
  --(4.141,5.773)--(4.145,6.115)--(4.149,5.944)--(4.153,6.286)--(4.157,6.115)--(4.161,6.286)%
  --(4.165,6.457)--(4.169,6.286)--(4.173,6.115)--(4.177,6.115)--(4.181,6.286)--(4.185,5.944)%
  --(4.188,6.628)--(4.192,5.944)--(4.196,6.457)--(4.200,5.944)--(4.204,5.773)--(4.208,6.115)%
  --(4.212,6.457)--(4.216,6.115)--(4.220,5.944)--(4.224,5.944)--(4.228,6.286)--(4.232,5.773)%
  --(4.236,6.115)--(4.240,5.944)--(4.244,6.286)--(4.248,5.944)--(4.252,6.628)--(4.256,5.944)%
  --(4.260,6.457)--(4.264,5.773)--(4.268,6.115)--(4.272,6.286)--(4.276,5.944)--(4.279,6.115)%
  --(4.283,6.115)--(4.287,6.115)--(4.291,6.628)--(4.295,6.115)--(4.299,5.773)--(4.303,5.944)%
  --(4.307,5.944)--(4.311,5.944)--(4.315,6.115)--(4.319,5.773)--(4.323,6.286)--(4.327,5.944)%
  --(4.331,6.799)--(4.335,6.286)--(4.339,6.628)--(4.343,6.457)--(4.347,6.115)--(4.351,5.773)%
  --(4.355,5.602)--(4.359,6.286)--(4.363,5.944)--(4.367,6.286)--(4.370,5.602)--(4.374,6.115)%
  --(4.378,5.944)--(4.382,6.457)--(4.386,6.115)--(4.390,6.286)--(4.394,5.944)--(4.398,6.115)%
  --(4.402,6.115)--(4.406,6.457)--(4.410,6.457)--(4.414,7.141)--(4.418,6.628)--(4.422,6.115)%
  --(4.426,6.115)--(4.430,6.115)--(4.434,5.773)--(4.438,5.773)--(4.442,6.286)--(4.446,6.799)%
  --(4.450,6.457)--(4.454,6.115)--(4.458,6.115)--(4.461,5.773)--(4.465,5.773)--(4.469,5.602)%
  --(4.473,5.944)--(4.477,5.773)--(4.481,6.115)--(4.485,5.944)--(4.489,6.628)--(4.493,6.457)%
  --(4.497,6.628)--(4.501,5.602)--(4.505,5.944)--(4.509,5.944)--(4.513,6.628)--(4.517,6.286)%
  --(4.521,5.773)--(4.525,5.773)--(4.529,5.944)--(4.533,5.773)--(4.537,5.944)--(4.541,6.286)%
  --(4.545,5.944)--(4.549,5.944)--(4.552,6.115)--(4.556,5.944)--(4.560,6.115)--(4.564,5.773)%
  --(4.568,6.115)--(4.572,6.457)--(4.576,6.115)--(4.580,6.286)--(4.584,6.115)--(4.588,5.944)%
  --(4.592,6.115)--(4.596,6.457)--(4.600,5.773)--(4.604,6.628)--(4.608,6.457)--(4.612,6.286)%
  --(4.616,6.286)--(4.620,6.115)--(4.624,5.773)--(4.628,5.944)--(4.632,6.799)--(4.636,6.457)%
  --(4.640,6.286)--(4.643,6.115)--(4.647,6.286)--(4.651,5.944)--(4.655,6.115)--(4.659,5.944)%
  --(4.663,6.286)--(4.667,6.286)--(4.671,6.286)--(4.675,6.286)--(4.679,6.628)--(4.683,6.457)%
  --(4.687,5.944)--(4.691,5.773)--(4.695,6.115)--(4.699,5.773)--(4.703,5.944)--(4.707,5.773)%
  --(4.711,5.773)--(4.715,5.944)--(4.719,5.773)--(4.723,6.115)--(4.727,5.944)--(4.731,6.115)%
  --(4.734,5.944)--(4.738,6.115)--(4.742,5.944)--(4.746,5.602)--(4.750,5.944)--(4.754,6.115)%
  --(4.758,6.115)--(4.762,5.944)--(4.766,6.115)--(4.770,5.944)--(4.774,5.944)--(4.778,5.773)%
  --(4.782,5.773)--(4.786,6.457)--(4.790,6.286)--(4.794,5.944)--(4.798,6.457)--(4.802,6.286)%
  --(4.806,6.115)--(4.810,5.773)--(4.814,6.115)--(4.818,6.628)--(4.822,6.457)--(4.825,6.457)%
  --(4.829,6.457)--(4.833,5.944)--(4.837,6.115)--(4.841,6.286)--(4.845,6.457)--(4.849,6.115)%
  --(4.853,5.773)--(4.857,6.457)--(4.861,6.286)--(4.865,5.944)--(4.869,5.773)--(4.873,6.286)%
  --(4.877,6.286)--(4.881,6.286)--(4.885,6.286)--(4.889,6.115)--(4.893,5.773)--(4.897,6.115)%
  --(4.901,6.115)--(4.905,5.602)--(4.909,5.773)--(4.913,5.944)--(4.916,5.944)--(4.920,5.944)%
  --(4.924,6.457)--(4.928,6.286)--(4.932,5.773)--(4.936,5.773)--(4.940,6.286)--(4.944,5.773)%
  --(4.948,6.115)--(4.952,5.944)--(4.956,5.773)--(4.960,5.944)--(4.964,5.773)--(4.968,5.944)%
  --(4.972,5.944)--(4.976,5.944)--(4.980,6.115)--(4.984,5.944)--(4.988,5.773)--(4.992,5.773)%
  --(4.996,6.115)--(5.000,6.286)--(5.004,5.773)--(5.007,5.773)--(5.011,5.944)--(5.015,5.773)%
  --(5.019,5.944)--(5.023,5.773)--(5.027,6.115)--(5.031,6.115)--(5.035,5.944)--(5.039,6.286)%
  --(5.043,5.944)--(5.047,6.457)--(5.051,5.602)--(5.055,5.602)--(5.059,6.286)--(5.063,5.773)%
  --(5.067,6.115)--(5.071,5.773)--(5.075,5.773)--(5.079,5.773)--(5.083,6.115)--(5.087,5.773)%
  --(5.091,5.944)--(5.095,5.944)--(5.098,6.115)--(5.102,5.773)--(5.106,5.944)--(5.110,5.773)%
  --(5.114,5.944)--(5.118,5.944)--(5.122,6.457)--(5.126,5.773)--(5.130,6.115)--(5.134,5.773)%
  --(5.138,5.431)--(5.142,5.944)--(5.146,6.115)--(5.150,5.602)--(5.154,6.115)--(5.158,6.115)%
  --(5.162,5.944)--(5.166,5.773)--(5.170,5.602)--(5.174,5.944)--(5.178,5.773)--(5.182,5.602)%
  --(5.186,6.115)--(5.189,5.944)--(5.193,6.115)--(5.197,6.628)--(5.201,6.115)--(5.205,6.286)%
  --(5.209,5.944)--(5.213,5.602)--(5.217,5.773)--(5.221,5.944)--(5.225,5.944)--(5.229,5.944)%
  --(5.233,5.944)--(5.237,5.944)--(5.241,6.115)--(5.245,5.944)--(5.249,5.431)--(5.253,6.115)%
  --(5.257,5.944)--(5.261,5.944)--(5.265,5.944)--(5.269,5.773)--(5.273,6.286)--(5.277,6.286)%
  --(5.280,5.944)--(5.284,5.944)--(5.288,6.115)--(5.292,6.115)--(5.296,6.115)--(5.300,6.457)%
  --(5.304,6.115)--(5.308,6.286)--(5.312,6.286)--(5.316,5.944)--(5.320,6.457)--(5.324,5.944)%
  --(5.328,5.944)--(5.332,5.944)--(5.336,5.602)--(5.340,5.944)--(5.344,5.944)--(5.348,5.944)%
  --(5.352,6.286)--(5.356,6.286)--(5.360,6.115)--(5.364,6.286)--(5.368,5.944)--(5.371,5.944)%
  --(5.375,6.286)--(5.379,6.115)--(5.383,6.286)--(5.387,6.628)--(5.391,6.799)--(5.395,6.799)%
  --(5.399,6.286)--(5.403,6.115)--(5.407,6.115)--(5.411,5.944)--(5.415,6.457)--(5.419,5.944)%
  --(5.423,7.141)--(5.427,6.970)--(5.431,6.799)--(5.435,6.799)--(5.439,6.286)--(5.443,6.115)%
  --(5.447,6.457)--(5.451,6.286)--(5.455,6.115)--(5.459,6.628)--(5.462,6.628)--(5.466,6.799)%
  --(5.470,6.286)--(5.474,6.286)--(5.478,6.628)--(5.482,6.457)--(5.486,6.286)--(5.490,6.286)%
  --(5.494,6.628)--(5.498,6.970)--(5.502,6.115)--(5.506,6.286)--(5.510,6.457)--(5.514,6.457)%
  --(5.518,6.457)--(5.522,6.286)--(5.526,6.628)--(5.530,6.628)--(5.534,6.286)--(5.538,6.970)%
  --(5.542,6.970)--(5.546,6.628)--(5.550,6.115)--(5.553,6.115)--(5.557,6.628)--(5.561,6.457)%
  --(5.565,6.457)--(5.569,6.286)--(5.573,6.628)--(5.577,6.970)--(5.581,6.970)--(5.585,6.457)%
  --(5.589,6.115)--(5.593,6.115)--(5.597,6.286)--(5.601,6.628)--(5.605,6.628)--(5.609,6.457)%
  --(5.613,6.970)--(5.617,7.141)--(5.621,7.141)--(5.625,6.457)--(5.629,6.457)--(5.633,6.457)%
  --(5.637,6.457)--(5.640,6.799)--(5.644,6.799)--(5.648,7.141)--(5.652,7.312)--(5.656,6.457)%
  --(5.660,6.115)--(5.664,6.799)--(5.668,6.457)--(5.672,6.628)--(5.676,6.286)--(5.680,6.457)%
  --(5.684,6.628)--(5.688,6.628)--(5.692,6.286)--(5.696,6.457)--(5.700,6.457)--(5.704,5.944)%
  --(5.708,6.457)--(5.712,6.457)--(5.716,6.286)--(5.720,6.457)--(5.724,6.799)--(5.728,6.799)%
  --(5.731,6.799)--(5.735,6.457)--(5.739,6.286)--(5.743,6.286)--(5.747,6.286)--(5.751,6.286)%
  --(5.755,6.286)--(5.759,6.970)--(5.763,6.799)--(5.767,6.457)--(5.771,6.115)--(5.775,5.773)%
  --(5.779,6.286)--(5.783,6.286)--(5.787,5.944)--(5.791,6.286)--(5.795,6.457)--(5.799,6.628)%
  --(5.803,6.115)--(5.807,5.773)--(5.811,5.773)--(5.815,5.773)--(5.819,6.115)--(5.822,5.773)%
  --(5.826,6.115)--(5.830,5.944)--(5.834,6.115)--(5.838,6.286)--(5.842,5.773)--(5.846,5.773)%
  --(5.850,5.944)--(5.854,5.773)--(5.858,6.115)--(5.862,5.944)--(5.866,5.773)--(5.870,6.115)%
  --(5.874,6.457)--(5.878,5.944)--(5.882,5.944)--(5.886,6.115)--(5.890,5.602)--(5.894,5.944)%
  --(5.898,5.944)--(5.902,5.773)--(5.906,6.115)--(5.910,5.773)--(5.913,6.286)--(5.917,5.773)%
  --(5.921,5.773)--(5.925,6.115)--(5.929,5.773)--(5.933,5.944)--(5.937,6.115)--(5.941,5.773)%
  --(5.945,5.431)--(5.949,6.115)--(5.953,5.431)--(5.957,5.260)--(5.961,5.602)--(5.965,5.602)%
  --(5.969,5.602)--(5.973,5.773)--(5.977,5.773)--(5.981,5.602)--(5.985,5.260)--(5.989,5.944)%
  --(5.993,5.260)--(5.997,5.773)--(6.001,5.431)--(6.004,5.602)--(6.008,5.260)--(6.012,5.431)%
  --(6.016,5.431)--(6.020,5.431)--(6.024,5.773)--(6.028,5.773)--(6.032,5.431)--(6.036,5.260)%
  --(6.040,5.089)--(6.044,5.431)--(6.048,5.431)--(6.052,5.773)--(6.056,5.602)--(6.060,5.431)%
  --(6.064,6.457)--(6.068,5.602)--(6.072,5.431)--(6.076,4.918)--(6.080,5.260)--(6.084,5.089)%
  --(6.088,5.089)--(6.092,5.431)--(6.095,5.089)--(6.099,5.089)--(6.103,4.918)--(6.107,5.431)%
  --(6.111,5.602)--(6.115,4.918)--(6.119,5.260)--(6.123,5.089)--(6.127,5.089)--(6.131,4.918)%
  --(6.135,4.918)--(6.139,5.602)--(6.143,5.089)--(6.147,5.089)--(6.151,5.431)--(6.155,5.089)%
  --(6.159,5.089)--(6.163,4.918)--(6.167,5.260)--(6.171,5.260)--(6.175,5.260)--(6.179,5.089)%
  --(6.183,5.260)--(6.186,5.602)--(6.190,5.089)--(6.194,4.747)--(6.198,4.918)--(6.202,5.260)%
  --(6.206,4.918)--(6.210,5.431)--(6.214,5.431)--(6.218,4.918)--(6.222,4.918)--(6.226,4.918)%
  --(6.230,4.747)--(6.234,4.918)--(6.238,5.089)--(6.242,4.918)--(6.246,4.918)--(6.250,4.918)%
  --(6.254,4.576)--(6.258,4.918)--(6.262,4.576)--(6.266,5.089)--(6.270,4.747)--(6.274,5.089)%
  --(6.277,4.918)--(6.281,4.234)--(6.285,4.405)--(6.289,4.405)--(6.293,4.576)--(6.297,4.747)%
  --(6.301,4.405)--(6.305,4.576)--(6.309,4.405)--(6.313,4.405)--(6.317,4.405)--(6.321,4.576)%
  --(6.325,4.405)--(6.329,4.576)--(6.333,4.405)--(6.337,4.747)--(6.341,4.405)--(6.345,4.405)%
  --(6.349,4.576)--(6.353,4.405)--(6.357,4.063)--(6.361,4.234)--(6.365,4.576)--(6.368,4.747)%
  --(6.372,4.576)--(6.376,4.234)--(6.380,4.576)--(6.384,4.063)--(6.388,4.576)--(6.392,4.063)%
  --(6.396,3.892)--(6.400,4.405)--(6.404,4.405)--(6.408,4.405)--(6.412,3.892)--(6.416,3.721)%
  --(6.420,3.379)--(6.424,3.892)--(6.428,3.892)--(6.432,3.550)--(6.436,3.550)--(6.440,4.234)%
  --(6.444,4.405)--(6.448,4.234)--(6.452,3.721)--(6.456,4.063)--(6.459,3.892)--(6.463,3.550)%
  --(6.467,3.892)--(6.471,4.063)--(6.475,3.892)--(6.479,4.063)--(6.483,4.063)--(6.487,3.892)%
  --(6.491,3.550)--(6.495,3.721)--(6.499,4.234)--(6.503,3.721)--(6.507,3.892)--(6.511,4.063)%
  --(6.515,4.234)--(6.519,3.892)--(6.523,3.892)--(6.527,4.063)--(6.531,4.234)--(6.535,3.721)%
  --(6.539,4.063)--(6.543,3.550)--(6.547,3.550)--(6.550,3.721)--(6.554,4.063)--(6.558,4.063)%
  --(6.562,4.234)--(6.566,3.721)--(6.570,3.550)--(6.574,4.063)--(6.578,3.379)--(6.582,3.379)%
  --(6.586,3.379)--(6.590,4.063)--(6.594,3.892)--(6.598,3.550)--(6.602,3.892)--(6.606,3.550)%
  --(6.610,3.379)--(6.614,3.379)--(6.618,3.721)--(6.622,3.721)--(6.626,3.379)--(6.630,3.892)%
  --(6.634,4.063)--(6.638,3.892)--(6.641,3.550)--(6.645,3.208)--(6.649,3.379)--(6.653,3.379)%
  --(6.657,3.379)--(6.661,3.721)--(6.665,3.379)--(6.669,3.208)--(6.673,3.379)--(6.677,3.208)%
  --(6.681,3.208)--(6.685,3.208)--(6.689,3.550)--(6.693,3.037)--(6.697,3.379)--(6.701,3.550)%
  --(6.705,3.208)--(6.709,3.379)--(6.713,3.379)--(6.717,4.063)--(6.721,3.379)--(6.725,3.208)%
  --(6.729,3.379)--(6.732,3.550)--(6.736,3.550)--(6.740,3.379)--(6.744,3.550)--(6.748,3.037)%
  --(6.752,3.721)--(6.756,3.208)--(6.760,3.037)--(6.764,3.208)--(6.768,3.208)--(6.772,3.208)%
  --(6.776,3.550)--(6.780,3.379)--(6.784,3.208)--(6.788,2.866)--(6.792,2.866)--(6.796,3.379)%
  --(6.800,2.695)--(6.804,3.208)--(6.808,3.037)--(6.812,3.037)--(6.816,3.721)--(6.820,3.037)%
  --(6.823,3.208)--(6.827,3.037)--(6.831,3.208)--(6.835,3.208)--(6.839,3.379)--(6.843,3.208)%
  --(6.847,3.379)--(6.851,3.208)--(6.855,3.379)--(6.859,3.037)--(6.863,3.550)--(6.867,3.037)%
  --(6.871,3.379)--(6.875,3.550)--(6.879,3.379)--(6.883,3.208)--(6.887,3.550)--(6.891,3.721)%
  --(6.895,3.037)--(6.899,3.208)--(6.903,3.037)--(6.907,3.379)--(6.911,3.208)--(6.914,3.208)%
  --(6.918,3.208)--(6.922,3.208)--(6.926,3.208)--(6.930,3.550)--(6.934,3.208)--(6.938,3.037)%
  --(6.942,3.208)--(6.946,3.037)--(6.950,3.208)--(6.954,3.037)--(6.958,3.037)--(6.962,3.379)%
  --(6.966,3.379)--(6.970,3.208)--(6.974,3.379)--(6.978,3.208)--(6.982,2.866)--(6.986,3.208)%
  --(6.990,3.208)--(6.994,2.866)--(6.998,3.037)--(7.002,3.721)--(7.005,3.379)--(7.009,3.037)%
  --(7.013,3.379)--(7.017,2.866)--(7.021,3.208)--(7.025,2.866)--(7.029,2.866)--(7.033,3.208)%
  --(7.037,3.037)--(7.041,2.866)--(7.045,3.037)--(7.049,2.866)--(7.053,3.208)--(7.057,2.866)%
  --(7.061,2.695)--(7.065,2.695)--(7.069,2.695)--(7.073,2.866)--(7.077,2.866)--(7.081,3.037)%
  --(7.085,3.208)--(7.089,3.208)--(7.092,3.037)--(7.096,2.866)--(7.100,2.695)--(7.104,3.037)%
  --(7.108,2.695)--(7.112,3.208)--(7.116,3.379)--(7.120,3.037)--(7.124,2.866)--(7.128,2.866)%
  --(7.132,3.037)--(7.136,3.037)--(7.140,2.866)--(7.144,2.866)--(7.148,2.695)--(7.152,2.695)%
  --(7.156,3.208)--(7.160,3.037)--(7.164,3.037)--(7.168,2.866)--(7.172,3.037)--(7.176,2.866)%
  --(7.180,2.866)--(7.183,2.866)--(7.187,2.695)--(7.191,3.037)--(7.195,3.379)--(7.199,2.866)%
  --(7.203,2.866)--(7.207,3.208)--(7.211,2.695)--(7.215,2.866)--(7.219,2.695)--(7.223,2.695)%
  --(7.227,2.866)--(7.231,2.866)--(7.235,2.695)--(7.239,2.866)--(7.243,2.524)--(7.247,2.866)%
  --(7.251,2.866)--(7.255,3.037)--(7.259,2.866)--(7.263,2.866)--(7.267,2.866)--(7.271,3.037)%
  --(7.274,2.866)--(7.278,3.208)--(7.282,3.208)--(7.286,2.353)--(7.290,2.524)--(7.294,2.524)%
  --(7.298,2.524)--(7.302,2.695)--(7.306,2.695)--(7.310,2.866)--(7.314,2.524)--(7.318,2.695)%
  --(7.322,2.695)--(7.326,2.524)--(7.330,2.695)--(7.334,2.866)--(7.338,2.695)--(7.342,2.866)%
  --(7.346,2.866)--(7.350,3.037)--(7.354,2.866)--(7.358,2.695)--(7.362,2.353)--(7.365,2.524)%
  --(7.369,2.353)--(7.373,2.695)--(7.377,2.695)--(7.381,2.695)--(7.385,2.353)--(7.389,3.037)%
  --(7.393,2.524)--(7.397,2.524)--(7.401,2.182)--(7.405,2.353)--(7.409,2.524)--(7.413,2.524)%
  --(7.417,2.695)--(7.421,2.866)--(7.425,2.866)--(7.429,2.695)--(7.433,2.866)--(7.437,2.695)%
  --(7.441,2.695)--(7.445,2.353)--(7.449,2.695)--(7.453,2.695)--(7.456,2.866)--(7.460,2.353)%
  --(7.464,2.695)--(7.468,2.695)--(7.472,2.524)--(7.476,2.866)--(7.480,2.866)--(7.484,2.524)%
  --(7.488,2.695)--(7.492,2.695)--(7.496,2.524)--(7.500,2.695)--(7.504,2.524)--(7.508,2.866)%
  --(7.512,2.182)--(7.516,2.695)--(7.520,2.353)--(7.524,2.695)--(7.528,2.524)--(7.532,2.695)%
  --(7.536,2.695)--(7.540,2.182)--(7.544,2.353)--(7.547,2.866)--(7.551,2.695)--(7.555,2.524)%
  --(7.559,2.353)--(7.563,2.182)--(7.567,2.524)--(7.571,2.695)--(7.575,2.524)--(7.579,2.353)%
  --(7.583,2.353)--(7.587,2.353)--(7.591,2.524)--(7.595,2.011)--(7.599,2.353)--(7.603,2.524)%
  --(7.607,2.353)--(7.611,2.182)--(7.615,2.353)--(7.619,2.695)--(7.623,2.524)--(7.627,2.524)%
  --(7.631,2.695)--(7.635,2.524)--(7.638,2.524)--(7.642,2.695)--(7.646,2.695)--(7.650,2.524)%
  --(7.654,2.353)--(7.658,2.353)--(7.662,2.524)--(7.666,2.524)--(7.670,2.182)--(7.674,2.353)%
  --(7.678,2.182)--(7.682,2.182)--(7.686,2.524)--(7.690,2.182)--(7.694,2.695)--(7.698,2.524)%
  --(7.702,2.353)--(7.706,2.353)--(7.710,2.182)--(7.714,2.182)--(7.718,2.524)--(7.722,2.353)%
  --(7.726,2.182)--(7.729,2.182)--(7.733,2.353)--(7.737,2.353)--(7.741,2.353)--(7.745,1.840)%
  --(7.749,2.182)--(7.753,2.182)--(7.757,2.353)--(7.761,2.695)--(7.765,2.182)--(7.769,2.182)%
  --(7.773,2.524)--(7.777,2.182)--(7.781,2.182)--(7.785,2.011)--(7.789,2.695)--(7.793,2.353)%
  --(7.797,1.840)--(7.801,2.695)--(7.805,2.353)--(7.809,2.353)--(7.813,2.182)--(7.817,2.182)%
  --(7.820,2.182)--(7.824,1.840)--(7.828,2.011)--(7.832,2.182)--(7.836,2.524)--(7.840,2.011)%
  --(7.844,2.182)--(7.848,2.353)--(7.852,2.182)--(7.856,2.011)--(7.860,2.011)--(7.864,2.182)%
  --(7.868,2.011)--(7.872,1.840)--(7.876,2.695)--(7.880,2.353)--(7.884,2.524)--(7.888,2.182)%
  --(7.892,2.353)--(7.896,2.353)--(7.900,1.840)--(7.904,2.353)--(7.908,2.353)--(7.911,2.524)%
  --(7.915,2.353)--(7.919,2.353)--(7.923,2.182)--(7.927,2.353)--(7.931,2.182)--(7.935,2.182)%
  --(7.939,2.353)--(7.943,2.524)--(7.947,2.353)--(7.951,2.353)--(7.955,2.353)--(7.959,2.353)%
  --(7.963,2.182)--(7.967,2.353)--(7.971,2.353)--(7.975,2.011)--(7.979,2.182)--(7.983,1.840)%
  --(7.987,2.353)--(7.991,2.353)--(7.995,2.182)--(7.999,2.182)--(8.002,2.353)--(8.006,2.524)%
  --(8.010,2.182)--(8.014,2.353)--(8.018,2.011)--(8.022,2.182)--(8.026,2.695)--(8.030,2.353)%
  --(8.034,2.182)--(8.038,2.182)--(8.042,2.011)--(8.046,2.011)--(8.050,2.011)--(8.054,2.182)%
  --(8.058,2.182)--(8.062,2.011)--(8.066,2.353)--(8.070,1.840)--(8.074,2.011)--(8.078,2.011)%
  --(8.082,2.182)--(8.086,2.182)--(8.090,1.840)--(8.093,2.353)--(8.097,2.011)--(8.101,2.524)%
  --(8.105,2.182)--(8.109,2.353)--(8.113,2.353)--(8.117,2.011)--(8.121,2.353)--(8.125,2.353)%
  --(8.129,2.182)--(8.133,2.182)--(8.137,2.182)--(8.141,2.353)--(8.145,2.182)--(8.149,2.011)%
  --(8.153,2.011)--(8.157,2.011)--(8.161,2.011)--(8.165,2.011)--(8.169,2.011)--(8.173,2.011)%
  --(8.177,2.353)--(8.181,2.011)--(8.184,1.840)--(8.188,2.182)--(8.192,1.840)--(8.196,1.840)%
  --(8.200,1.840)--(8.204,2.011)--(8.208,2.011)--(8.212,2.011)--(8.216,2.182)--(8.220,1.840)%
  --(8.224,1.669)--(8.228,2.182)--(8.232,2.011)--(8.236,2.011)--(8.240,2.011)--(8.244,1.840)%
  --(8.248,1.840)--(8.252,2.182)--(8.256,2.011)--(8.260,2.011)--(8.264,2.011)--(8.268,1.669)%
  --(8.272,2.011)--(8.275,2.011)--(8.279,2.011)--(8.283,1.840)--(8.287,2.182)--(8.291,2.182)%
  --(8.295,2.011)--(8.299,1.840)--(8.303,2.524)--(8.307,2.182)--(8.311,1.840)--(8.315,1.840)%
  --(8.319,2.011)--(8.323,2.182)--(8.327,2.182)--(8.331,2.182)--(8.335,2.182)--(8.339,2.182)%
  --(8.343,2.182)--(8.347,1.840)--(8.351,2.011)--(8.355,2.011)--(8.359,2.182)--(8.363,1.840)%
  --(8.366,1.840)--(8.370,2.182)--(8.374,2.011)--(8.378,1.840)--(8.382,1.669)--(8.386,1.840)%
  --(8.390,1.669)--(8.394,2.011)--(8.398,2.353)--(8.402,2.182)--(8.406,1.840)--(8.410,2.353)%
  --(8.414,1.840)--(8.418,2.011)--(8.422,2.011)--(8.426,2.011)--(8.430,2.182)--(8.434,2.353)%
  --(8.438,2.353)--(8.442,1.840)--(8.446,2.182)--(8.450,1.669)--(8.453,1.669)--(8.457,1.840)%
  --(8.461,1.840)--(8.465,2.011)--(8.469,1.840)--(8.473,2.011)--(8.477,2.011)--(8.481,2.011)%
  --(8.485,1.840)--(8.489,1.840)--(8.493,1.840)--(8.497,2.011)--(8.501,1.840)--(8.505,2.011)%
  --(8.509,1.840)--(8.513,2.011)--(8.517,2.182)--(8.521,1.840)--(8.525,1.840)--(8.529,2.011)%
  --(8.533,2.182)--(8.537,1.840)--(8.541,1.840)--(8.544,2.011)--(8.548,2.182)--(8.552,2.182)%
  --(8.556,2.182)--(8.560,2.182)--(8.564,1.840)--(8.568,1.840)--(8.572,1.840)--(8.576,2.353)%
  --(8.580,1.840)--(8.584,2.011)--(8.588,2.182)--(8.592,1.669)--(8.596,2.011)--(8.600,2.011)%
  --(8.604,2.011)--(8.608,1.840)--(8.612,2.011)--(8.616,1.669)--(8.620,2.524)--(8.624,2.182)%
  --(8.628,1.840)--(8.632,1.840)--(8.635,2.353)--(8.639,2.182)--(8.643,1.498)--(8.647,1.840)%
  --(8.651,1.840)--(8.655,2.011)--(8.659,2.011)--(8.663,1.669)--(8.667,2.182)--(8.671,2.011)%
  --(8.675,1.840)--(8.679,2.011)--(8.683,2.011)--(8.687,2.182)--(8.691,2.011)--(8.695,1.669)%
  --(8.699,1.840)--(8.703,2.182)--(8.707,1.840)--(8.711,2.011)--(8.715,2.182)--(8.719,1.840)%
  --(8.723,2.182)--(8.726,1.840)--(8.730,2.182)--(8.734,1.669)--(8.738,2.011)--(8.742,2.011)%
  --(8.746,1.669)--(8.750,1.669)--(8.754,2.182)--(8.758,1.840)--(8.762,1.669)--(8.766,1.840)%
  --(8.770,1.669)--(8.774,1.840)--(8.778,1.840)--(8.782,2.011)--(8.786,1.840)--(8.790,1.840)%
  --(8.794,2.011)--(8.798,1.498)--(8.802,1.840)--(8.806,1.840)--(8.810,2.011)--(8.814,1.840)%
  --(8.817,2.353)--(8.821,2.011)--(8.825,2.011)--(8.829,2.182)--(8.833,2.182)--(8.837,2.182)%
  --(8.841,1.840)--(8.845,2.011)--(8.849,2.353)--(8.853,2.524)--(8.857,2.011)--(8.861,2.524)%
  --(8.865,2.353)--(8.869,2.182)--(8.873,1.669)--(8.877,2.182)--(8.881,2.011)--(8.885,2.011)%
  --(8.889,2.182)--(8.893,2.353)--(8.897,2.011)--(8.901,2.182)--(8.905,2.011)--(8.908,2.011)%
  --(8.912,2.353)--(8.916,1.840)--(8.920,1.669)--(8.924,1.840)--(8.928,2.011)--(8.932,2.182)%
  --(8.936,2.011)--(8.940,2.182)--(8.944,2.182)--(8.948,1.840)--(8.952,1.840)--(8.956,2.011)%
  --(8.960,1.669)--(8.964,1.840)--(8.968,2.182)--(8.972,2.182)--(8.976,1.669)--(8.980,2.011)%
  --(8.984,2.182)--(8.988,1.840)--(8.992,1.840)--(8.996,1.840)--(8.999,2.011)--(9.003,2.011)%
  --(9.007,1.840)--(9.011,2.182)--(9.015,2.011)--(9.019,2.011)--(9.023,2.182)--(9.027,1.669)%
  --(9.031,1.840)--(9.035,1.840)--(9.039,1.840)--(9.043,2.011)--(9.047,2.524)--(9.051,2.011)%
  --(9.055,2.011)--(9.059,2.011)--(9.063,1.840)--(9.067,2.011)--(9.071,1.669)--(9.075,2.182)%
  --(9.079,2.182)--(9.083,2.182)--(9.087,2.182)--(9.090,2.011)--(9.094,2.011)--(9.098,2.011)%
  --(9.102,2.011)--(9.106,1.840)--(9.110,2.011)--(9.114,2.182)--(9.118,1.840)--(9.122,1.840)%
  --(9.126,2.182)--(9.130,2.182)--(9.134,2.353)--(9.138,2.182)--(9.142,2.353)--(9.146,1.840)%
  --(9.150,1.840)--(9.154,2.182)--(9.158,2.524)--(9.162,2.353)--(9.166,2.011)--(9.170,2.011)%
  --(9.174,1.840)--(9.178,2.011)--(9.181,1.669)--(9.185,2.182)--(9.189,2.353)--(9.193,2.182)%
  --(9.197,1.840)--(9.201,2.182)--(9.205,2.353)--(9.209,2.011)--(9.213,1.840)--(9.217,1.327)%
  --(9.221,2.353)--(9.225,2.182)--(9.229,2.182)--(9.233,2.353)--(9.237,2.011)--(9.241,1.840)%
  --(9.245,2.011)--(9.249,1.840)--(9.253,1.669)--(9.257,2.011)--(9.261,1.840)--(9.265,2.011)%
  --(9.269,2.182)--(9.272,1.840)--(9.276,2.011)--(9.280,1.840)--(9.284,2.011)--(9.288,1.498)%
  --(9.292,1.840)--(9.296,1.840)--(9.300,1.840)--(9.304,1.840)--(9.308,1.840)--(9.312,1.840)%
  --(9.316,2.182)--(9.320,2.011)--(9.324,2.011)--(9.328,1.498)--(9.332,1.498)--(9.336,2.011)%
  --(9.340,2.011)--(9.344,2.182)--(9.348,1.669)--(9.352,2.011)--(9.356,1.669)--(9.360,2.182)%
  --(9.363,2.011)--(9.367,2.182)--(9.371,1.840)--(9.375,2.182)--(9.379,2.695)--(9.383,2.011)%
  --(9.387,2.011)--(9.391,2.353)--(9.395,2.011)--(9.399,2.182)--(9.403,2.182)--(9.407,2.182)%
  --(9.411,1.498)--(9.415,1.840)--(9.419,2.524)--(9.423,2.011)--(9.427,1.840)--(9.431,1.840)%
  --(9.435,1.840)--(9.439,1.840)--(9.443,2.011)--(9.447,2.182)--(9.451,1.840)--(9.454,2.011)%
  --(9.458,2.182)--(9.462,2.011)--(9.466,2.011)--(9.470,2.353)--(9.474,2.011)--(9.478,2.011)%
  --(9.482,2.011)--(9.486,1.840)--(9.490,2.353)--(9.494,2.182)--(9.498,2.011)--(9.502,2.011)%
  --(9.506,2.182)--(9.510,2.011)--(9.514,2.011)--(9.518,1.840)--(9.522,1.840)--(9.526,2.011)%
  --(9.530,1.669)--(9.534,2.182)--(9.538,2.182)--(9.542,1.840)--(9.545,1.840)--(9.549,1.327)%
  --(9.553,1.840)--(9.557,1.669)--(9.561,1.669)--(9.565,2.011)--(9.569,2.182)--(9.573,2.011)%
  --(9.577,1.840)--(9.581,1.840)--(9.585,1.840)--(9.589,2.695)--(9.593,2.524)--(9.597,1.840)%
  --(9.601,2.353)--(9.605,1.840)--(9.609,2.353)--(9.613,2.353)--(9.617,2.011)--(9.621,2.182)%
  --(9.625,1.669)--(9.629,2.011)--(9.633,1.840)--(9.636,1.498)--(9.640,1.669)--(9.644,2.011)%
  --(9.648,2.011)--(9.652,2.182)--(9.656,1.840)--(9.660,2.011)--(9.664,2.182)--(9.668,1.840)%
  --(9.672,1.840)--(9.676,2.011)--(9.680,1.840)--(9.684,2.011)--(9.688,1.840)--(9.692,2.011)%
  --(9.696,2.011)--(9.700,2.011)--(9.704,1.840)--(9.708,2.182)--(9.712,2.182)--(9.716,1.669)%
  --(9.720,1.840)--(9.724,2.011)--(9.727,1.840)--(9.731,2.182)--(9.735,1.669)--(9.739,2.182)%
  --(9.743,2.182)--(9.747,2.011)--(9.751,2.011)--(9.755,1.840)--(9.759,1.669)--(9.763,2.011)%
  --(9.767,2.182)--(9.771,2.011)--(9.775,2.011)--(9.779,1.669)--(9.783,1.498)--(9.787,1.840)%
  --(9.791,1.840)--(9.795,2.182)--(9.799,1.840)--(9.803,2.182)--(9.807,1.840)--(9.811,1.669)%
  --(9.815,1.669)--(9.818,1.840)--(9.822,1.840)--(9.826,1.669)--(9.830,1.840)--(9.834,1.840)%
  --(9.838,2.353)--(9.842,2.353)--(9.846,2.182)--(9.850,1.840)--(9.854,2.182)--(9.858,1.840)%
  --(9.862,1.840)--(9.866,2.353)--(9.870,2.182)--(9.874,2.524)--(9.878,2.011)--(9.882,2.182)%
  --(9.886,2.182)--(9.890,2.182)--(9.894,2.353)--(9.898,2.182)--(9.902,2.011)--(9.905,1.840)%
  --(9.909,2.182)--(9.913,2.353)--(9.917,2.182)--(9.921,2.011)--(9.925,2.182)--(9.929,2.182)%
  --(9.933,1.840)--(9.937,2.011)--(9.941,2.011)--(9.945,2.011)--(9.949,2.011)--(9.953,2.011)%
  --(9.957,2.011)--(9.961,1.669)--(9.965,2.011)--(9.969,2.182)--(9.973,2.011)--(9.977,2.182)%
  --(9.981,2.182)--(9.985,1.840)--(9.989,2.182)--(9.993,2.182)--(9.996,1.840)--(10.000,1.840)%
  --(10.004,2.011)--(10.008,1.498)--(10.012,1.840)--(10.016,1.840)--(10.020,2.524)--(10.024,1.327)%
  --(10.028,1.840)--(10.032,1.840)--(10.036,1.840)--(10.040,1.840)--(10.044,1.327)--(10.048,1.840)%
  --(10.052,2.011)--(10.056,2.011)--(10.060,2.353)--(10.064,2.182)--(10.068,2.011)--(10.072,2.011)%
  --(10.076,2.182)--(10.080,2.011)--(10.084,2.524)--(10.087,2.353)--(10.091,2.182)--(10.095,2.182)%
  --(10.099,2.353)--(10.103,2.182)--(10.107,2.182)--(10.111,2.353)--(10.115,2.011)--(10.119,2.182)%
  --(10.123,1.840)--(10.127,2.011)--(10.131,2.182)--(10.135,2.353)--(10.139,1.669)--(10.143,1.840)%
  --(10.147,2.011)--(10.151,2.011)--(10.155,1.840)--(10.159,2.011)--(10.163,1.840)--(10.167,1.669)%
  --(10.171,2.695)--(10.175,2.011)--(10.178,2.182)--(10.182,2.011)--(10.186,1.840)--(10.190,2.182)%
  --(10.194,1.840)--(10.198,2.011)--(10.202,1.669)--(10.206,2.353)--(10.210,2.182)--(10.214,1.669)%
  --(10.218,2.182)--(10.222,2.182)--(10.226,2.011)--(10.230,1.840)--(10.234,2.011)--(10.238,2.011)%
  --(10.242,2.011)--(10.246,1.840)--(10.250,1.840)--(10.254,1.498)--(10.258,1.669)--(10.262,2.182)%
  --(10.266,2.011)--(10.269,1.840)--(10.273,1.840)--(10.277,1.840)--(10.281,1.840)--(10.285,2.182)%
  --(10.289,1.840)--(10.293,2.524)--(10.297,1.840)--(10.301,1.669)--(10.305,2.011)--(10.309,1.669)%
  --(10.313,2.011)--(10.317,2.011)--(10.321,2.353)--(10.325,1.669);
\gpcolor{color=gp lt color border}
\gpsetlinetype{gp lt border}
\draw[gp path] (2.056,7.825)--(2.056,1.327);
\draw[gp path] (2.056,0.985)--(10.325,0.985);
%% coordinates of the plot area
\gpdefrectangularnode{gp plot 1}{\pgfpoint{2.056cm}{0.985cm}}{\pgfpoint{11.947cm}{7.825cm}}
\end{tikzpicture}
%% gnuplot variables

\caption{Variazione della temperatura, esempio 3}
\label{img:isoa}
\end{grafico}
	
\clearpage
\section{Analisi dei dati}
	


\section{Conclusioni}
	

	
\section{Codice}
	

	
%\subsection{Esempio immagini}
%\begin{figure}[p]
% \centering
% \includegraphics[width=0.8\textwidth]{spazio1}
% \caption{Spazio!}
% \label{fig:spazio1}
%\end{figure}

%\end{multicols}

\end{document}
